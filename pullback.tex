\chapter{Homotopy pullbacks}

Suppose we are given a map $f:A\to B$, and type families $P$ over $A$, and $Q$ over $B$.
Then any fiberwise map
\begin{equation*}
g:\prd{x:A}P(x)\to Q(f(x))
\end{equation*}
gives rise to a commuting square
\begin{equation*}
\begin{tikzcd}[column sep=large]
\sm{x:A}P(x) \arrow[r,"{\total[f]{g}}"] \arrow[d,swap,"\proj 1"] & \sm{y:B}Q(y) \arrow[d,"\proj 1"] \\
A \arrow[r,swap,"f"] & B
\end{tikzcd}
\end{equation*}
where $\total[f]{g}$ is defined as $\lam{(x,p)}(f(x),g(x,y))$. 
We will show in \cref{thm:pb_fibequiv} that $g$ is a fiberwise equivalence if and only if this square is a \emph{pullback square}. This generalization of \cref{thm:fib_equiv} is one of the motivating theorems for pullbacks, and this will play an important role in the descent theorems.

\section{The universal property of homotopy pullbacks}

\begin{defn}
A \define{cospan}\index{cospan} consists of three types $A$, $X$, and $B$, and maps $f:A\to X$ and $g:B\to X$. We define the following further structures for any cospan $A \stackrel{f}{\rightarrow} X \stackrel{g}{\leftarrow} B$.
\begin{enumerate}
\item Given a type $C$, a \define{cone}\index{cone!on a cospan|textbf} on the cospan $A \stackrel{f}{\rightarrow} X \stackrel{g}{\leftarrow} B$ with \define{vertex} $C$\index{vertex!of a cone|textbf} consists of maps $p:C\to A$, $q:C\to B$ and a homotopy $H:f\circ p\htpy g\circ q$, rendering the square
\begin{equation*}
\begin{tikzcd}
C \arrow[r,"q"] \arrow[d,swap,"p"] & B \arrow[d,"g"] \\
A \arrow[r,swap,"f"] & X
\end{tikzcd}
\end{equation*}
commutative. We write $\mathsf{cone}(C)$\index{cone(C)@{$\mathsf{cone}(\blank)$}|textbf} for the type of cones with vertex $C$. 
\item For any cone $(p,q,H)$ with vertex $C$, we define a map\index{cone_map@{$\mathsf{cone\usc{}map}$}|textbf}
\begin{equation*}
\mathsf{cone\usc{}map}(p,q,H):(D\to C)\to\mathsf{cone}(D)
\end{equation*}
by $h\mapsto (p\circ h,q\circ h,H\circ h)$. 
\end{enumerate}
\end{defn}

\begin{defn}
We say that a commuting square
\begin{equation*}
\begin{tikzcd}
C \arrow[r,"q"] \arrow[d,swap,"p"] & B \arrow[d,"g"] \\
A \arrow[r,swap,"f"] & X
\end{tikzcd}
\end{equation*}
with $H:f\circ p\htpy g\circ q$ is a \define{pullback square}\index{pullback square|textbf} if it satisfies the \define{universal property} of pullbacks\index{universal property!of pullbacks}, which asserts that the map
\begin{equation*}
\mathsf{cone\usc{}map}(p,q,H):(D\to C)\to\mathsf{cone}(D)
\end{equation*}
is an equivalence for every type $D$. 
\end{defn}

\begin{rmk}
We spell out what the universal property of pullbacks means. By the characterization of equivalences as contractible maps of \cref{thm:contr_equiv}, we see that $C$ is a pullback of $f:A\to X$ and $g:B\to X$ if for every cone $(p',q',H')$ with vertex $D$, the fiber $\fib{\mathsf{cone\usc{}map}(p,q,H)}((p',q',H'))$ is contractible. By \cref{thm:eq_sigma} and function extensionality, this fiber is equivalent to the type of quadruples $(h,K,L,M)$ consisting of
\begin{align*}
h & : D\to C \\
K & : p' \htpy p\circ h \\
L & : q' \htpy q\circ h \\
M & : H' \htpy \ct{}{Hh}{}
\end{align*}
We often indicate this situation diagrammatically as follows:
\begin{equation*}
\begin{tikzcd}
D \arrow[drr,bend left=15] \arrow[dr,densely dotted] \arrow[ddr,bend right=15] \\
& C \arrow[r] \arrow[d] & B \arrow[d] \\
& A \arrow[r] & X
\end{tikzcd}
\end{equation*}
The universal property is now explained as follows: given a type $D$ with maps $p':D\to A$, $q':D\to B$, and a homotopy $H':f\circ p'\htpy g\circ q'$, there is a unique map $h:D\to C$ equipped with homotopies $K:p'\htpy p\circ h$, $L:q'\htpy q\circ h$, and a homotopy of homotopies $M:H'\htpy H\circ h$. This higher homotopy can be seen as a coherence, that fills
\end{rmk}

\begin{defn}
Let $f:A\to X$ and $B\to X$ be maps. Then we define
\begin{align*}
A\times_X B & \defeq \sm{x:A}{y:B}f(x)=g(y) \\
\pi_1 & \defeq \proj 1 & & : A\times_X B\to A \\
\pi_2 & \defeq \proj 1\circ\proj 2 & & : A\times_X B\to B\\
\pi_{htpy} & \defeq \proj 2\circ\proj 2 & & : f\circ \pi_1 \htpy g\circ\pi_2.
\end{align*}
The type $A\times_X B$ is called the \define{canonical pullback}\index{canonical pullback} of $f$ and $g$.
\end{defn}

Note that $A\times_X B$ depends on $f$ and $g$, although this dependency is not visible in the notation.

\begin{thm}
Given maps $f:A\to X$ and $g:B\to X$, the commuting square\index{canonical pullback|textit}
\begin{equation*}
\begin{tikzcd}
A\times_X B \arrow[r,"\pi_2"] \arrow[d,swap,"\pi_1"] & B \arrow[d,"g"] \\
A \arrow[r,swap,"f"] & X,
\end{tikzcd}
\end{equation*}
is a pullback square.
\end{thm}

\begin{proof}
Let $C$ be a type. Our goal is to show that the map
\begin{equation*}
\lam{f}(\pi_1 \circ f, \pi_2\circ f,\pi_3\circ f): (C\to A\times_X B)\to \mathsf{cone}(C)
\end{equation*}
is an equivalence. Consider a cone $(p,q,H)$ with vertex $C$. [TODO]
\end{proof}

\begin{thm}\label{thm:pb_fibequiv}
Let $f:A\to B$, and let $g:\prd{a:A}P(a)\to Q(f(a))$ be a fiberwise transformation\index{fiberwise transformation|textit}. The following are equivalent:
\begin{enumerate}
\item The commuting square
\begin{equation*}
\begin{tikzcd}[column sep=large]
\sm{a:A}P(a) \arrow[r,"\total{g}"] \arrow[d,->>] & \sm{b:B}Q(b) \arrow[d,->>] \\
A \arrow[r,swap,"f"] & B
\end{tikzcd}
\end{equation*}
is a pullback square.
\item $g$ is a fiberwise equivalence.
\end{enumerate}
\end{thm}

Sometimes pullbacks are also called \emph{fiber products}\index{fiber products}. The following theorem is why.

\begin{thm}
Consider a pullback square
\begin{equation*}
\begin{tikzcd}
C \arrow[r,"k"] \arrow[d,swap,"h"] & B \arrow[d,"g"] \\
A \arrow[r,swap,"f"] & X.
\end{tikzcd}
\end{equation*}
Then there is an equivalence
\begin{equation*}
\eqv{\fib{f\circ h}{x}}{\fib{f}{x}\times\fib{g}{x}}.
\end{equation*}
\end{thm}

\begin{exercises}
\item Consider a commuting square
\begin{equation*}
\begin{tikzcd}
C \arrow[r,"q"] \arrow[d,swap,"p"] & B \arrow[d,"g"] \\
A \arrow[r,swap,"f"] & X
\end{tikzcd}
\end{equation*}
with $H:f\circ p\htpy g\circ q$. Show that the following are equivalent:
\begin{enumerate}
\item The square is a pullback square.
\item For every type $D$, the commuting square
\begin{equation*}
\begin{tikzcd}
C^D \arrow[r,"q\circ\blank"] \arrow[d,swap,"p\circ\blank"] & B^D \arrow[d,"g\circ\blank"] \\
A^D \arrow[r,swap,"f\circ\blank"] & X^D
\end{tikzcd}
\end{equation*}
is a pullback square.
\end{enumerate}
\item Show that the square\index{identity type!as pullback}
\begin{equation*}
\begin{tikzcd}
(x=y) \arrow[r] \arrow[d] & \unit \arrow[d,"\mathsf{const}_y"] \\
\unit \arrow[r,swap,"\mathsf{const}_x"] & A
\end{tikzcd}
\end{equation*}
is a pullback square.
\item 
\begin{subexenum}
\item Show that for any $f:A\to X$ the commuting square\index{fiber!as pullback}
\begin{equation*}
\begin{tikzcd}
\fib{f}{x} \arrow[d] \arrow[r,"\proj 1"] & A \arrow[d,"f"] \\
\unit \arrow[r,swap,"x"] & X
\end{tikzcd}
\end{equation*}
is a pullback square.
\item Show that for any $B:A\to \UU$ and any $a:A$ the commuting square
\begin{equation*}
\begin{tikzcd}
B(a) \arrow[r] \arrow[d] & \sm{x:A}B(x) \arrow[d,"\proj 1"] \\
\unit \arrow[r,swap,"a"] & A
\end{tikzcd}
\end{equation*}
is a pullback square.
\end{subexenum}
\item \label{lem:pb_pasting}(The pullback pasting lemma.) Consider\index{pullback!pasting lemma} 
\begin{equation*}
\begin{tikzcd}
A \arrow[d,swap,"f"] \arrow[r,"j"] & B \arrow[d,swap,"g"] \arrow[r,"l"] & C \arrow[d,"h"] \\
X \arrow[r,swap,"i"] & Y \arrow[r,swap,"k"] & Z
\end{tikzcd}
\end{equation*}
with homotopies $H:i\circ f\htpy g\circ j$ and $K:k\circ g\htpy h\circ l$ witnessing that the two squares commute, and suppose that the square on the right is a pullback square. Show that the square on the left is a pullback square if and only if the outer rectangle is a pullback square.
\item Given a map $f:A\to X$ define $\delta_f:A\to A\times_X A$ by $x\mapsto (x,x,\refl{f(x)})$. Show that a map $f:A\to X$ is $(k+1)$-truncated if and only if $\delta_f$ is $k$-truncated.
\item Consider a pullback square
\begin{equation*}
\begin{tikzcd}
C \arrow[r,"q"] \arrow[d,swap,"p"] & B \arrow[d,"g"] \\
A \arrow[r,swap,"f"] & X.
\end{tikzcd}
\end{equation*}
Show that if $g$ is $k$-truncated, then so is $p$.\index{truncated!map!pullback stability}
\item Consider a \define{natural transformation of cospans}\index{cospan!natural transformation of}, i.e.~a commuting diagram of the form
\begin{equation*}
\begin{tikzcd}
A \arrow[r] \arrow[d,swap,"f"] & X \arrow[d,swap,"h"] & B \arrow[l] \arrow[d,"g"] \\
A' \arrow[r] & X' & B'. \arrow[l]
\end{tikzcd}
\end{equation*}
Show that the map
\begin{equation*}
(a,b,p)\mapsto (f(a),g(b),\mathsf{ap}_h(p)): A \times_X B \to A'\times_{X'} B'
\end{equation*}
is $n$-truncated if each of the vertical maps is.
\item 
\begin{subexenum}
\item Show that if 
\begin{equation*}
\begin{tikzcd}
P(a) \arrow[r] \arrow[d] & Y(a) \arrow[d] \\
X(a) \arrow[r] & Z(a)
\end{tikzcd}
\end{equation*}
is a pullback square for each $a:A$, then so is
\begin{equation*}
\begin{tikzcd}
\prd{a:A}P(a) \arrow[r] \arrow[d] & \prd{a:A}Y(a) \arrow[d] \\
\prd{a:A}X(a) \arrow[r] & \prd{a:A}Z(a)
\end{tikzcd}
\end{equation*}
\item Show that if
\begin{equation*}
\begin{tikzcd}
C_1 \arrow[r] \arrow[d] & B_1 \arrow[d] & C_2 \arrow[r] \arrow[d] & B_2 \arrow[d] \\
A_1 \arrow[r] & X_1 & A_2 \arrow[r] & X_2
\end{tikzcd}
\end{equation*}
are pullback squares, then so is
\begin{equation*}
\begin{tikzcd}
C_1\times C_2 \arrow[r] \arrow[d] & B_1\times B_2 \arrow[d] \\
A_1 \times A_2 \arrow[r] & X_1\times X_2. 
\end{tikzcd}
\end{equation*}
\end{subexenum}
\item Show that 
\begin{equation*}
\begin{tikzcd}[column sep=8em]
\eqv{A}{B} \arrow[r] \arrow[d] & \unit \arrow[d,"{(\idfunc[A],\idfunc[B])}"] \\
A^B\times B^A \times A^B \arrow[r,swap,"{(h,f,g)\mapsto (h\circ f,f\circ g)}"] & A^A \times B^B
\end{tikzcd}
\end{equation*}
is a pullback square.
\item Consider a diagram of the form
\begin{equation*}
\begin{tikzcd}
A \arrow[d,swap,"f"] & B \arrow[d,"g"] \\
X \arrow[r,swap,"h"] & Y
\end{tikzcd}
\end{equation*}
\begin{subexenum}
\item Construct an equivalence between the type
\begin{equation*}
\prd{x:X}\fib{f}{x}\to\fib{g}{h(x)}
\end{equation*}
of fiberwise transformations between the fibers of $f$ and $g$, and the type
\begin{equation*}
\sm{i:A\to B} h\circ f\htpy g\circ i.
\end{equation*}
\item Now consider additionally $i:A\to B$ and $H:h\circ f\htpy g\circ i$, and let $u:A\to X\times_Y B$ be the unique map for which the diagram
\begin{equation*}
\begin{tikzcd}
A \arrow[ddr,bend right=15,swap,"f"] \arrow[dr,densely dotted,"u"] \arrow[drr,bend left=15,"i"] \\
& X\times_Y B \arrow[d,swap,"\pi_1"] \arrow[r,"\pi_2" near start] & B \arrow[d,"g"] \\
& X \arrow[r,swap,"h"] & Y
\end{tikzcd}
\end{equation*}
commutes. Show that the fiber of $u$ at $(x,b,p):X\times_Y B$ is equivalent to the fiber of
\begin{equation*}
i_x:\fib{f}{x} \to \fib{g}{h(x)}
\end{equation*}
at $(b,p^{-1})$. 
\end{subexenum}
\item Consider a commuting square
\begin{equation*}
\begin{tikzcd}
C \arrow[r] \arrow[d] & A \arrow[d] \\
B \arrow[r] & X.
\end{tikzcd}
\end{equation*}
Show that this square is cartesian if and only if the induced map $C\to A\times_X B$ has a retraction.
\end{exercises}
