\chapter{Homotopy pullbacks}

Suppose we are given a map $f:A\to B$, and type families $P$ over $A$, and $Q$ over $B$.
Then any fiberwise map
\begin{equation*}
g:\prd{x:A}P(x)\to Q(f(x))
\end{equation*}
gives rise to a commuting square
\begin{equation*}
\begin{tikzcd}[column sep=large]
\sm{x:A}P(x) \arrow[r,"{\total[f]{g}}"] \arrow[d,swap,"\proj 1"] & \sm{y:B}Q(y) \arrow[d,"\proj 1"] \\
A \arrow[r,swap,"f"] & B
\end{tikzcd}
\end{equation*}
where $\total[f]{g}$ is defined as $\lam{(x,p)}(f(x),g(x,y))$. 
We will show in \cref{thm:pb_fibequiv} that $g$ is a fiberwise equivalence if and only if this square is a \emph{pullback square}. This generalization of \cref{thm:fib_equiv} is therefore abstracting away from the notion of fiberwise equivalence, and it serves as our motivating theorem to introduce pullbacks. The connection between pullbacks and fiberwise equivalences has an important role in the descent theorem in \cref{chap:descent}.

\section{Cartesian squares}

Recall that a square
\begin{equation*}
\begin{tikzcd}
C \arrow[r,"q"] \arrow[d,swap,"p"] & B \arrow[d,"g"] \\
A \arrow[r,swap,"g"] & X
\end{tikzcd}
\end{equation*}
is said to \define{commute} if there is a homotopy $H:f\circ p\htpy g\circ q$. 
The pullback property is a \emph{universal property} of the upper left corner of a commuting square (in our case $C$), characterizing the maps \emph{into} it.

To describe the universal property of pullbacks we first need to have a closer look at the \emph{anatomy} of commuting squares.

\begin{defn}
A commuting square
\begin{equation*}
\begin{tikzcd}
C \arrow[r,"q"] \arrow[d,swap,"p"] & B \arrow[d,"g"] \\
A \arrow[r,swap,"f"] & X
\end{tikzcd}
\end{equation*}
with $H:f\circ p\htpy g\circ q$ can be dissected into three parts, consisting of a \emph{cospan}, a type, and a \emph{cone}, where
\begin{enumerate}
\item A \define{cospan}\index{cospan} consists of three types $A$, $X$, and $B$, and maps $f:A\to X$ and $g:B\to X$.
\item Given a type $C$, a \define{cone}\index{cone!on a cospan|textbf} on the cospan $A \stackrel{f}{\rightarrow} X \stackrel{g}{\leftarrow} B$ with \define{vertex} $C$\index{vertex!of a cone|textbf} consists of maps $p:C\to A$, $q:C\to B$ and a homotopy $H:f\circ p\htpy g\circ q$. We write\index{cone(C)@{$\mathsf{cone}(\blank)$}|textbf}
\begin{equation*}
\mathsf{cone}(C)\defeq \sm{p:C\to A}{q:C\to B}f\circ p\htpy g\circ q
\end{equation*}
for the type of cones with vertex $C$.
\end{enumerate}
\end{defn}

Given a cone with vertex $C$ on a span $A\stackrel{f}{\rightarrow} X \stackrel{g}{\leftarrow} B$ and a map $h:C'\to C$, we construct a new cone with vertex $C'$ in the following definition.

\begin{defn}
For any cone $(p,q,H)$ with vertex $C$ and any type $C'$, we define a map\index{cone_map@{$\mathsf{cone\usc{}map}$}|textbf}
\begin{equation*}
\mathsf{cone\usc{}map}(p,q,H):(C'\to C)\to\mathsf{cone}(C')
\end{equation*}
by $h\mapsto (p\circ h,q\circ h,H\circ h)$. 
\end{defn}

\begin{defn}
We say that a commuting square
\begin{equation*}
\begin{tikzcd}
C \arrow[r,"q"] \arrow[d,swap,"p"] & B \arrow[d,"g"] \\
A \arrow[r,swap,"f"] & X
\end{tikzcd}
\end{equation*}
with $H:f\circ p\htpy g\circ q$ is a \define{pullback square}\index{pullback square|textbf}, or that it is \define{cartesian}\index{cartesian square|textbf}, if it satisfies the \define{universal property} of pullbacks\index{universal property!of pullbacks}, which asserts that the map
\begin{equation*}
\mathsf{cone\usc{}map}(p,q,H):(C'\to C)\to\mathsf{cone}(C')
\end{equation*}
is an equivalence for every type $C'$. 
\end{defn}

We often indicate the universal property with a diagram as follows:
\begin{equation*}
\begin{tikzcd}
C' \arrow[drr,bend left=15,"{q'}"] \arrow[dr,densely dotted,"h"] \arrow[ddr,bend right=15,swap,"{p'}"] \\
& C \arrow[r,"q"] \arrow[d,swap,"p"] & B \arrow[d,"g"] \\
& A \arrow[r,swap,"f"] & X
\end{tikzcd}
\end{equation*}
since the universal property states that for every cone $(p',q',H')$ with vertex $C'$, the type of pairs $(h,\alpha)$ consisting of $h:C'\to C$ equipped with $\alpha:\mathsf{cone\usc{}map}((p,q,H),h)=(p',q',H')$ is contractible by \cref{thm:contr_equiv}.

In order to see what goes on in the universal property of pullbacks, we need to first characterize the identity type of $\mathsf{cone}(C)$, for any type $C$.

\begin{lem}\label{lem:id_cone}
Let $(p,q,H)$ and $(p',q',H')$ be cones on a cospan $f:A\rightarrow X \leftarrow B:g$, both with vertex $C$. Then the type $(p,q,H)=(p',q',H')$ is equivalent to the type of triples $(K,L,M)$ consisting of
\begin{align*}
K & : p \htpy p' \\
L & : q \htpy q' \\
M & : \ct{H}{(g\cdot L)} \htpy \ct{(f\cdot K)}{H'}
\end{align*}
\end{lem}

\begin{rmk}
Note for $z:C$, the identification $(\ct{H}{(g\cdot L)})(z)$ is the pointwise concatenation
\begin{equation*}
\begin{tikzcd}
f(p(z)) \arrow[r,equals,"H(z)"] & g(q(z)) \arrow[r,equals,"\ap{g}{L(z)}"] &[1.5em] g(q'(z)),
\end{tikzcd}
\end{equation*}
and the identification $(\ct{(f\cdot K)}{H'})(z)$ is the pointwise concatenation
\begin{equation*}
\begin{tikzcd}
f(p(z)) \arrow[r,equals,"\ap{f}{K(z)}"] &[1.5em] f(p'(z)) \arrow[r,equals,"{H'(z)}"] & g(q'(z)).
\end{tikzcd}
\end{equation*}
Therefore the homotopy $M:\ct{H}{(g\cdot L)} \htpy \ct{(f\cdot K)}{H'}$ shows that for each $z:C$, the \emph{square of identifications}
\begin{equation*}
\begin{tikzcd}[column sep=huge]
f(p(z)) \arrow[r,equals,"\ap{f}{K(z)}"] \arrow[d,equals,swap,"H(z)"] & f(p'(z)) \arrow[d,equals,"{H'(z)}"] \\
g(q(z)) \arrow[r,equals,swap,"\ap{g}{L(z)}"] & g(q'(z))
\end{tikzcd}
\end{equation*}
commutes, as witnessed by $M(z)$. Thus each $M(z)$ is a 2-cell in the sense that it is an identifications of identifications. 
\end{rmk}

\begin{proof}[Proof of \cref{lem:id_cone}]
By the fundamental theorem of identity types (\cref{thm:id_fundamental}) and associativity of $\Sigma$-types (\cref{ex:sigma_assoc}) it suffices to show that the type
\begin{equation*}
\sm{p':C\to A}{q':C\to B}{H':f\circ p'\htpy g\circ q}{K:p\htpy p'}{L:q\htpy q'} \ct{H}{(g\cdot L)} \htpy \ct{(f\cdot K)}{H'}
\end{equation*}
is contractible. Now we apply \cref{ex:sigma_swap} repeatedly to see that this type is equivalent to the type
\begin{equation*}
\sm{p':C\to A}{K: p\htpy p'}{q':C\to B}{L: q\htpy q'}{H':f\circ p'\htpy g\circ q} \ct{H}{(g\cdot L)} \htpy \ct{(f\cdot K)}{H'}.
\end{equation*}
The types $\sm{p':C\to A} p\htpy p'$ and $\sm{q':C\to B} q\htpy q'$ are contractible by function extensionality, and  we have
\begin{samepage}
\begin{align*}
(p,\mathsf{htpy.refl}_p) & : \sm{p':C'\to A} p\htpy p' \\
(q,\mathsf{htpy.refl}_q) & : \sm{q':C'\to B} q\htpy q'.
\end{align*}%
\end{samepage}%
Thus we apply \cref{ex:contr_in_sigma} to see that the type of tuples $(p',K,q',L,H',M)$ is equivalent to the type
\begin{equation*}
\sm{H':f\circ p'\htpy g\circ q} \ct{H}{\mathsf{htpy.refl}_{g\circ q}}\htpy \ct{\mathsf{htpy.refl}_{f\circ p}}{H'}.
\end{equation*}
Of course, the type $\ct{H}{\mathsf{htpy.refl}_{g\circ q}}\htpy \ct{\mathsf{htpy.refl}_{f\circ p}}{H'}$ is equivalent to the type $H\htpy H'$, and $\sm{H':f\circ p'\htpy g\circ q} H\htpy H'$ is contractible.
\end{proof}

As a corollary we obtain the following characterization of the universal property of pullbacks.

\begin{thm}\label{lem:pullback_up}
Consider a commuting square
\begin{equation*}
\begin{tikzcd}
C \arrow[r,"q"] \arrow[d,swap,"p"] & B \arrow[d,"g"] \\
A \arrow[r,swap,"f"] & X
\end{tikzcd}
\end{equation*}
with $H:f\circ p\htpy g\circ q$
Then the following are equivalent:
\begin{enumerate}
\item The square is a pullback square.
\item For every type $C'$ and every cone $(p',q',H')$ with vertex $C'$, the type of quadruples $(h,K,L,M)$ consisting of
\begin{align*}
h & : C'\to C \\
K & : p\circ h \htpy p' \\
L & : q\circ h \htpy q' \\
M & : \ct{(H\cdot h)}{(g\cdot L)} \htpy \ct{(f\cdot K)}{H'}
\end{align*}
is contractible.
\end{enumerate}
\end{thm}

\section{The unique existence of pullbacks}

\begin{defn}
Let $f:A\to X$ and $B\to X$ be maps. Then we define
\begin{align*}
A\times_X B & \defeq \sm{x:A}{y:B}f(x)=g(y) \\
\pi_1 & \defeq \proj 1 & & : A\times_X B\to A \\
\pi_2 & \defeq \proj 1\circ\proj 2 & & : A\times_X B\to B\\
\pi_3 & \defeq \proj 2\circ\proj 2 & & : f\circ \pi_1 \htpy g\circ\pi_2.
\end{align*}
The type $A\times_X B$ is called the \define{canonical pullback}\index{canonical pullback} of $f$ and $g$.
\end{defn}

Note that $A\times_X B$ depends on $f$ and $g$, although this dependency is not visible in the notation.

\begin{thm}
Given maps $f:A\to X$ and $g:B\to X$, the commuting square\index{canonical pullback|textit}
\begin{equation*}
\begin{tikzcd}
A\times_X B \arrow[r,"\pi_2"] \arrow[d,swap,"\pi_1"] & B \arrow[d,"g"] \\
A \arrow[r,swap,"f"] & X,
\end{tikzcd}
\end{equation*}
is a pullback square.
\end{thm}

\begin{proof}
Let $C$ be a type. Our goal is to show that the map
\begin{equation*}
\mathsf{cone\usc{}map}(\pi_1,\pi_2,\pi_3): (C\to A\times_X B)\to \mathsf{cone}(C)
\end{equation*}
is an equivalence. 
By double application of \cref{thm:choice} we obtain equivalences
\begin{align*}
(C\to A\times_X B) & \jdeq C\to \sm{x:A}{y:B}f(x)=g(y) \\
& \eqvsym \sm{p:C\to A}\prd{z:C}\sm{y:B} f(p(z))= y \\
& \eqvsym \sm{p:C\to A}{q:C\to B}\prd{z:C} f(p(z))= g(q(z)) \\
& \jdeq \mathsf{cone}(C)
\end{align*}
The composite of these equivalences is the map
\begin{equation*}
\lam{f}(\lam{z}\proj 1(f(z)),\lam{z} \proj 1(\proj 2(f(z))),\lam{z}\proj 2(\proj 2(f(z)))),
\end{equation*}
which is \emph{exactly} the map $\mathsf{cone\usc{}map}(\pi_1,\pi_2,\pi_3)$, and since it is a composite of equivalences it follows that it is itself an equivalence.
\end{proof}

In the following theorem we establish the uniqueness of pullbacks up to equivalence via a \emph{3-for-2 property} for pullbacks\index{pullback!3-for-2 property}.

\begin{thm}\label{thm:pb_3for2}
Consider the squares
\begin{equation*}
\begin{tikzcd}
C \arrow[r,"q"] \arrow[d,swap,"p"] & B \arrow[d,"g"] & {C'} \arrow[r,"{q'}"] \arrow[d,swap,"{p'}"] & B \arrow[d,"g"] \\
A \arrow[r,swap,"f"] & X & A \arrow[r,swap,"f"] & X
\end{tikzcd}
\end{equation*}
with homotopies $H:f\circ p \htpy g\circ q$ and $H':f\circ p'\htpy g\circ q'$.
Furthermore, suppose we have a map $h:C'\to C$ equipped with
\begin{align*}
K & : p\circ h \htpy p' \\
L & : q\circ h \htpy q' \\
M & : \ct{(H\cdot h)}{(g\cdot L)} \htpy \ct{(f\cdot K)}{H'}.
\end{align*}
If any two of the following three properties hold, so does the third:
\begin{samepage}%
\begin{enumerate}
\item $C$ is a pullback.
\item $C'$ is a pullback.
\item $h$ is an equivalence.
\end{enumerate}%
\end{samepage}%
\end{thm}

\begin{proof}
By the characterization of the identity type of $\mathsf{cone}(C')$ given in \cref{lem:id_cone} we obtain an identification
\begin{equation*}
\mathsf{cone\usc{}map}((p,q,H),h)=(p',q',H')
\end{equation*}
from the triple $(K,L,M)$. 
Let $D$ be a type, and let $k:D\to C'$ be a map. We observe that
\begin{align*}
\mathsf{cone\usc{}map}((p,q,H),(h\circ k)) & \jdeq (p\circ (h\circ k),q\circ (h\circ k),H\circ (h\circ k)) \\
& \jdeq ((p\circ h)\circ k,(q\circ h)\circ k, (H\circ h)\circ k) \\
& \jdeq \mathsf{cone\usc{}map}(\mathsf{cone\usc{}map}((p,q,H),h),k) \\
& = \mathsf{cone\usc{}map}((p',q',H'),k).
\end{align*}
Thus we see that the triangle 
\begin{equation*}
\begin{tikzcd}[column sep=-1em]
(D\to C') \arrow[rr,"{h\circ \blank}"] \arrow[dr,swap,"{\mathsf{cone\usc{}map}(p',q',H')}"] & & (D\to C) \arrow[dl,"{\mathsf{cone\usc{}map}(p,q,H)}"] \\
& \mathsf{cone}(D)
\end{tikzcd}
\end{equation*}
commutes. Therefore it follows from the 3-for-2 property of equivalences established in \cref{ex:3_for_2}, that if any two of the following properties hold, then so does the third:
\begin{enumerate}
\item The map $\mathsf{cone\usc{}map}(p,q,H):(D\to C)\to \mathsf{cone}(D)$ is an equivalence,
\item The map $\mathsf{cone\usc{}map}(p',q',H'):(D\to C')\to \mathsf{cone}(D)$ is an equivalence,
\item The map $h\circ\blank : (D\to C')\to (D\to C)$ is an equivalence.
\end{enumerate}
Thus the 3-for-2 property for pullbacks follows once we show that $h$ is an equivalence if and only if $h\circ\blank : (D\to C')\to (D\to C)$ is an equivalence for any type $D$. We establish this in \cref{lem:postcomp_equiv}.
\end{proof}

\begin{lem}\label{lem:postcomp_equiv}
Let $f:X\to Y$ be a map. Then the following are equivalent:
\begin{enumerate}
\item $f$ is an equivalence.
\item For any type $A$, the map $f\circ\blank : X^A\to Y^A$ is an equivalence.
\end{enumerate}
\end{lem}

\begin{proof}
If $f$ is an equivalence, then it is straightforward to see that the map
\begin{equation*}
f^{-1}\circ\blank : Y^A\to X^A
\end{equation*}
is an inverse of $f\circ \blank : X^A \to Y^A$, for any type $A$.

For the converse, we use that the fibers of $f\circ \blank$ are contractible, for any $A$. In particular, the fiber
\begin{equation*}
\fib{f\circ\blank}{\idfunc[Y]}\jdeq \sm{g:Y\to X} f\circ g=\idfunc[Y]
\end{equation*}
is contractible (choosing $A\jdeq Y$). Thus we obtain a function $g:Y\to X$ and a homotopy $G:f\circ g\htpy \idfunc[Y]$.

It remains to be shown that $g$ is a retract of $f$, i.e.~to construct a homotopy $g\circ f\htpy \idfunc[X]$. To see this, we use that the fiber
\begin{equation*}
\fib{f\circ\blank}{f}\jdeq \sm{h:X\to X} f\circ h=f
\end{equation*}
is contractible (choosing $A\jdeq X$). Of course we have $(\idfunc[X],\refl{f})$ in this fiber. However we claim that there also is an identification $p:f\circ (g\circ f)=f$, showing that $(g\circ f,p)$ is in this fiber. To see this, note that
\begin{align*}
f\circ (g\circ f) & \jdeq (f\circ g)\circ f \\
& = \idfunc[Y]\circ f \\
& \jdeq f
\end{align*}
By the contractibility of the fiber we conclude that $(\idfunc[X],\refl{f})=(g\circ f,p)$, so it follows that $\idfunc[X]=g\circ f$.
\end{proof}

Pullbacks are not only unique in the sense that any two pullbacks of the same cospan are equivalent, they are \emph{uniquely unique} in the sense that 

\begin{cor}
Suppose both commuting squares
\begin{equation*}
\begin{tikzcd}
C \arrow[r,"q"] \arrow[d,swap,"p"] & B \arrow[d,"g"] & {C'} \arrow[r,"{q'}"] \arrow[d,swap,"{p'}"] & B \arrow[d,"g"] \\
A \arrow[r,swap,"f"] & X & A \arrow[r,swap,"f"] & X
\end{tikzcd}
\end{equation*}
with homotopies $H:f\circ p \htpy g\circ q$ and $H':f\circ p'\htpy g\circ q'$ are pullback squares.
Then the type of quadruples $(e,K,L,M)$ consisting of an equivalence $e:\eqv{C'}{C}$ equipped with
\begin{align*}
K & : p\circ e \htpy p' \\
L & : q\circ e \htpy q' \\
M & : \ct{(g\cdot L)}{(H\cdot e)} \htpy \ct{(f\cdot K)}{H'}.
\end{align*}
is contractible.
\end{cor}

\begin{proof}
We have seen that the type of quadruples $(h,K,L,M)$ is equivalent to the fiber of $\mathsf{cone\usc{}map}(p,q,H)$ at $(p',q',H')$. By \cref{thm:pb_3for2} it follows that $h$ is an equivalence. Since $\isequiv(h)$ is a proposition (and hence contractible as soon as it is inhabited) it follows that the type of quadruples $(e,K,L,M)$ is contractible. 
\end{proof}

\section{Fiber products}

An important special case of pullbacks occurs when the cospan is of the form
\begin{equation*}
\begin{tikzcd}
A \arrow[r] & \unit & B. \arrow[l]
\end{tikzcd}
\end{equation*}
In this case, the pullback is just the \emph{cartesian product}.

\begin{lem}\label{lem:prod_pb}
Let $A$ and $B$ be types. Then the square
\begin{equation*}
\begin{tikzcd}
A\times B \arrow[r,"\proj 2"] \arrow[d,swap,"\proj 1"] & B \arrow[d,"\mathsf{const}_{\ttt}"] \\
A \arrow[r,swap,"\mathsf{const}_{\ttt}"] & \unit
\end{tikzcd}
\end{equation*}
which commutes by the homotopy $\mathsf{const}_{\refl{\ttt}}$ is a pullback square.
\end{lem}

\begin{proof}
By \cref{thm:pb_3for2} it suffices to construct an equivalence 
\begin{equation*}
e:\eqv{(A\times B)}{(A\times_\unit B)}
\end{equation*}
equipped with homotopies
\begin{align*}
K & : \pi_1\circ e \htpy \proj 1 \\
L & : \pi_2\circ e \htpy \proj 2 \\
M & : \ct{(\pi_3\cdot e)}{(\mathsf{const}_\ttt\cdot L)}\htpy \ct{(\mathsf{const}_\ttt\cdot K)}{\mathsf{const}_{\refl{\ttt}}}.
\end{align*}

The equivalence $e:\eqv{(A\times B)}{(A\times_\unit B)}$ is given by $\lam{(a,b)}(a,b,\refl{\ttt})$. Its inverse is the map $\lam{(a,b,p)}(a,b)$. The homotopies $K$, $L$, and $M$ are given by
\begin{align*}
K & \defeq \lam{(a,b)}\refl{a} \\
L & \defeq \lam{(a,b)}\refl{b} \\
M & \defeq \lam{(a,b)}\refl{\refl{\ttt}}\qedhere
\end{align*}
\end{proof}

The following generalization of \cref{lem:prod_pb} is the reason why pullbacks are sometimes called \define{fiber products}.

\begin{thm}
Let $P$ and $Q$ be families over a type $X$. Then the square
\begin{equation*}
\begin{tikzcd}[column sep=8em]
\sm{x:X}P(x)\times Q(x) \arrow[r,"{\lam{(x,(p,q))}(x,q)}"] \arrow[d,swap,"{\lam{(x,(p,q))}(x,p)}"] & \sm{x:X}Q(x) \arrow[d,"\proj 1"] \\
\sm{x:X}P(x) \arrow[r,swap,"\proj 1"] & X
\end{tikzcd}
\end{equation*}
is a pullback square.
\end{thm}

\begin{proof}
The square commutes by the homotopy
\begin{equation*}
H\defeq \lam{(x,(p,q))}\refl{x}.
\end{equation*}
To show that the stated square is a pullback square we will construct an equivalence
\begin{equation*}
e:\eqv{\Big(\sm{x:X}P(x)\times Q(x)\Big)}{\Big(\sm{x:X}P(x)\Big)\times_X \Big(\sm{x:X}Q(x)\Big)}
\end{equation*}
equipped with homotopies
\begin{align*}
K & : \pi_1\circ e \htpy \lam{(x,(p,q))}(x,p) \\
L & : \pi_2\circ e \htpy \lam{(x,(p,q))}(x,q) \\
M & : \ct{(\pi_3\cdot e)}{(\proj 1\cdot L)} \htpy \ct{(\proj 1\cdot K)}{H}.
\end{align*}
We define $e(x,(p,q))\defeq ((x,p),(x,q),\refl{x})$. The inverse of $e$ is defined as 
\begin{equation*}
\lam{((x,p),(y,q),\alpha)}(y,(\mathsf{tr}_P(\alpha,p),q)).
\end{equation*}
It is straightforward to see that this is indeed an inverse of $e$.

Then we define
\begin{align*}
K & \defeq \lam{(x,(p,q))}\refl{(x,p)} \\
L & \defeq \lam{(x,(p,q))}\refl{(x,q)}.
\end{align*}
Then we have homotopies
\begin{align*}
\proj 1\cdot K & \htpy \lam{(x,(p,q))}\refl{x} \\
\proj 1\cdot L & \htpy \lam{(x,(p,q))}\refl{x} \\
\pi_3\cdot e & \htpy \lam{(x,(p,q))}\refl{x}.
\end{align*}
Therefore the type of $M$ is equivalent to the type of homotopies
\begin{equation*}
\ct{(\lam{(x,(p,q))}\refl{x})}{(\lam{(x,(p,q))}\refl{x})} \htpy \ct{(\lam{(x,(p,q))}\refl{x})}{(\lam{(x,(p,q))}\refl{x})}
\end{equation*}
Here we have the homotopy $\lam{(x,(p,q))}\refl{\refl{x}}$.
\end{proof}

\begin{cor}
For any $f:A\to X$ and $g:B\to X$, the square
\begin{equation*}
\begin{tikzcd}[column sep=8em]
\sm{x:X}\fib{f}{x}\times\fib{g}{y} \arrow[r,"{\lam{(x,((a,p),(b,q)))}b}"] \arrow[d,swap,"{\lam{(x,((a,p),(b,q)))}a}"] & B \arrow[d,"g"]  \\
A \arrow[r,swap,"f"] & X
\end{tikzcd}
\end{equation*}
is a pullback square.
\end{cor}

\section{Fiber sequences}

\begin{lem}\label{lem:fib_pb}
For any function $f:A\to B$, and any $b:B$, consider the square
\begin{equation*}
\begin{tikzcd}[column sep=large]
\fib{f}{b} \arrow[r,"\mathsf{const}_\ttt"] \arrow[d,swap,"\proj 1"] & \unit \arrow[d,"\mathsf{const}_b"] \\
A \arrow[r,swap,"f"] & B
\end{tikzcd}
\end{equation*}
which commutes by $\proj 2 : \prd{t:\fib{f}{b}} f(\proj 1(t))=b$. This is a pullback square.
\end{lem}

\begin{proof}
To see this, note that by \cref{thm:pb_3for2} it suffices to construct an equivalence $e:\eqv{\fib{f}{b}}{A\times_B \unit}$, and homotopies
\begin{align*}
K & : \pi_1\circ e \htpy \proj 1 \\
L & : \pi_2\circ e \htpy \mathsf{const}_\ttt \\
M & : \ct{(\pi_3 \cdot e)}{(\mathsf{const}_b \cdot L)} \htpy \ct{(f \cdot K)}{\proj 2}.
\end{align*}

The equivalence $e:\eqv{\fib{f}{b}}{A\times_B \unit}$ is defined to be the map $\total{\varphi}$, where 
\begin{equation*}
\varphi : \prd{x:A} (f(x)=b)\to \sm{t:\unit} f(x)=b
\end{equation*}
is given by $\varphi_x(p)\defeq (\ttt,p)$. This is a fiberwise equivalence by \cref{ex:contr_in_sigma}, so $e$ is an equivalence by \cref{thm:fib_equiv}. Then we have
\begin{align*}
K\defeq \lam{(a,p)}\refl{a} & : \pi_1\circ e \htpy \proj 1 \\
L \defeq  \lam{(a,p)}\refl{\ttt} & : \pi_2\circ e \htpy \mathsf{const}_\ttt.
\end{align*}
Now we observe that there are homotopies
\begin{align*}
f\cdot K & \htpy \lam{(a,p)} \refl{f(a)} \\
\mathsf{const}_b \cdot L & \htpy \lam{(a,p)}\refl{b} \\
\pi_3\cdot e & \htpy \proj 2.
\end{align*}
Therefore we obtain the homotopy $M$ by constructing a homotopy
\begin{equation*}
\ct{\proj 2}{(\lam{(a,p)}\refl{b})} \htpy \ct{(\lam{(a,p)} \refl{f(a)})}{\proj 2}.
\end{equation*}
We obtain this homotopy from the left and right unit laws of identity types.
\end{proof}

\cref{lem:fib_pb} motivates the following definition of \emph{fiber sequences}, which play an important role in synthetic homotopy theory (and in algebraic topology). 

\begin{defn}
A \define{fiber sequence} consists of types $F$, $E$, and $B$ with \define{base points} $x:F$, $y:E$, and $b:B$, and maps
\begin{equation*}
\begin{tikzcd}
F \arrow[r,"i"] & E \arrow[r,"p"] & B
\end{tikzcd}
\end{equation*}
preserving the base points in the sense that $i(x)=y$ and $p(y)=b$, such that the square
\begin{equation*}
\begin{tikzcd}
F \arrow[r,"i"] \arrow[d] & E \arrow[d,"p"] \\
\unit \arrow[r,swap,"b"] & B
\end{tikzcd}
\end{equation*}
is a pullback square. We often write $F\hookrightarrow E \twoheadrightarrow B$ to indicate that we have a fiber sequence. 

Given a fiber sequence $F\hookrightarrow E\twoheadrightarrow B$, we call $B$ the \define{base space}, $E$ the \define{total space}, and $F$ the \define{fiber}.
\end{defn}

\begin{eg}
For any type family $B$ over $A$ and any $a:A$ the square
\begin{equation*}
\begin{tikzcd}[column sep=large]
B(a) \arrow[d,swap,"{\lam{y}(a,y)}"] \arrow[r,"\mathsf{const}_\ttt"] & \unit \arrow[d,"\lam{\ttt}a"] \\
\sm{x:A}B(x) \arrow[r,swap,"\proj 1"] & A
\end{tikzcd}
\end{equation*}
is a pullback square. 

To see this we have to construct a equivalence $e:\eqv{B(a)}{\fib{\proj 1}{a}}$ equipped with homotopies
\begin{align*}
K & : \proj 1\circ e \htpy \lam{y}(a,y) \\
L & : \mathsf{const}_\ttt \circ e \htpy \mathsf{const}_\ttt \\
M & : \ct{(\proj 2\circ e)}{((\lam{\ttt}a)\cdot L)} \htpy \ct{(\proj 1\cdot K)}{(\lam{y}\refl{a})}
\end{align*}
We define
\begin{equation*}
e \defeq \lam{y}((a,y),\refl{a})
\end{equation*}
This function is an equivalence by \cref{ex:proj_fiber}. For the homotopies we take
\begin{align*}
K & \defeq \lam{y}\refl{(a,y)} \\
L & \defeq \lam{y}\refl{\ttt} \\
M & \defeq \lam{y}\refl{\refl{a}}.
\end{align*}
This concludes the proof that the asserted square is a pullback square.

Thus we see that if we additionally suppose that there is a term $b:B(a)$, then we obtain a fiber sequence
\begin{equation*}
\begin{tikzcd}
B(a) \arrow[r,hookrightarrow] & \sm{x:A}B(x) \arrow[r,->>] & A.
\end{tikzcd}
\end{equation*}
\end{eg}

\section{Fiberwise equivalences}

\begin{lem}\label{lem:pb_subst}
Let $f:A\to B$, and let $Q$ be a type family over $B$. Then the square
\begin{equation*}
\begin{tikzcd}[column sep=6em]
\sm{x:A}Q(f(x)) \arrow[r,"{\lam{(x,q)}(f(x),q)}"] \arrow[d,swap,"\proj 1"] & \sm{y:B}Q(b) \arrow[d,"\proj 1"] \\
A \arrow[r,swap,"f"] & B
\end{tikzcd}
\end{equation*}
commutes by $H\defeq \lam{(x,q)}\refl{f(x)}$. This is a pullback square.
\end{lem}

\begin{proof}
By \cref{thm:pb_3for2} it suffices to construct an equivalence 
\begin{equation*}
e:\eqv{\Big(\sm{x:A}Q(f(x))\Big)}{A\times_B \Big(\sm{y:B}Q(y)\Big)},
\end{equation*}
and homotopies
\begin{align*}
K & : \pi_1\circ e \htpy \proj 1 \\
L & : \pi_2\circ e \htpy \lam{(x,q)}(f(x),q) \\
M & : \ct{(\pi_3 \cdot e)}{(\proj 1 \cdot L)} \htpy \ct{(f \cdot K)}{H}.
\end{align*}
We define $e$ by
The equivalence $e$ is defined by $\Sigma$-induction, taking
\begin{equation*}
e \defeq \lam{(x,q)}(x,(f(x),q),\refl{f(x)}).
\end{equation*}
The inverse of this map is given by $\lam{(x,((y,q),p))}(x,\mathsf{tr}_Q(p^{-1},q))$, and it is straigthforward to see that these maps are indeed mutual inverses.

Now we define 
\begin{align*}
K \defeq \lam{(x,q)}\refl{x} \\
L \defeq \lam{(x,q)}\refl{(f(x),q)}.
\end{align*}
Then we observe that there are homotopies
\begin{align*}
f \cdot K & \htpy \lam{(x,q)} \refl{f(x)} \\
\proj 1 \cdot L & \htpy \lam{(x,q)} \refl{f(x)} \\
\pi_3 \cdot e & \htpy \lam{(x,q)} \refl{f(x)}.
\end{align*}
Therefore the type of $M$ is equivalent to the type of homotopies
\begin{equation*}
\ct{(\lam{(x,q)} \refl{f(x)})}{(\lam{(x,q)} \refl{f(x)})} \htpy \ct{(\lam{(x,q)} \refl{f(x)})}{(\lam{(x,q)} \refl{f(x)})},
\end{equation*}
and here we have the homotopy $\lam{(x,q)}\refl{\refl{f(x)}}$. 
\end{proof}

\begin{thm}\label{thm:pb_fibequiv}
Let $f:A\to B$, and let $g:\prd{a:A}P(a)\to Q(f(a))$ be a fiberwise transformation\index{fiberwise transformation|textit}. The following are equivalent:
\begin{enumerate}
\item The commuting square
\begin{equation*}
\begin{tikzcd}[column sep=large]
\sm{a:A}P(a) \arrow[r,"{\total[f]{g}}"] \arrow[d,->>] & \sm{b:B}Q(b) \arrow[d,->>] \\
A \arrow[r,swap,"f"] & B
\end{tikzcd}
\end{equation*}
is a pullback square.
\item $g$ is a fiberwise equivalence.
\end{enumerate}
\end{thm}

\begin{proof}
Note that we have the map
\begin{equation*}
\total{g}:\Big(\sm{x:A}P(x)\Big)\to\Big(\sm{x:A}Q(f(x))\Big),
\end{equation*}
and homotopies
\begin{align*}
K & : \proj 1\circ \total{g} \htpy \proj 1 \\
L & : (\lam{(x,q)}(f(x),q))\circ \total{g}\htpy \total[f]{g} \\
M & : \ct{((\lam{(x,q)}\refl{f(x)})\cdot \total{g})}{(\proj 1\cdot L)} \htpy \ct{(f\cdot K)}{(\lam{(x,p)}\refl{f(x)})}
\end{align*}
defined by
\begin{align*}
K & \defeq \lam{(x,p)}\refl{x} \\
L & \defeq \lam{(x,p)}\refl{(f(x),g(x,p))} \\
M & \defeq \lam{(x,p)}\refl{f(x)}.
\end{align*}
Therefore we are in the situation of \cref{thm:pb_3for2}, with the commuting squares
\begin{equation*}
\begin{tikzcd}
\sm{x:A}Q(f(x)) \arrow[r] \arrow[d,swap,"\proj 1"] & \sm{y:B}Q(b) \arrow[d,"\proj 1"] &[-1em] \sm{a:A}P(a) \arrow[r,"{\total[f]{g}}"] \arrow[d,->>] &[1.5em] \sm{b:B}Q(b) \arrow[d,->>] \\
A \arrow[r,swap,"f"] & B & A \arrow[r,swap,"f"] & B.
\end{tikzcd}
\end{equation*}
Since the square on the left is a pullback by \cref{lem:pb_subst}, it follows that the square on the right is a pullback if and only if $\total{g}$ is an equivalence. By \cref{thm:fib_equiv} we know that $\total{g}$ is an equivalence if and only if $g$ is a fiberwise equivalence.
\end{proof}

\begin{cor}\label{cor:pb_fibequiv}
Consider a commuting square
\begin{equation*}
\begin{tikzcd}
C \arrow[r,"q"] \arrow[d,swap,"p"] & B \arrow[d,"g"] \\
A \arrow[r,swap,"f"] & X
\end{tikzcd}
\end{equation*}
with $H:f\circ p\htpy g\circ q$. The following are equivalent:
\begin{enumerate}
\item The square is a pullback square.
\item The induced map on fibers
\begin{equation*}
\lam{x}{(z,\alpha)}(q(z),\ct{H(z)^{-1}}{\ap{f}{\alpha}}):\prd{x:A}\fib{p}{x}\to \fib{g}{f(x)}
\end{equation*}
is a fiberwise equivalence.
\end{enumerate}
\end{cor}

\begin{cor}
Consider a pullback square
\begin{equation*}
\begin{tikzcd}
C \arrow[r,"q"] \arrow[d,swap,"p"] & B \arrow[d,"g"] \\
A \arrow[r,swap,"f"] & X.
\end{tikzcd}
\end{equation*}
If $g$ is a $k$-truncated map, then so is $p$.
\end{cor}

\begin{proof}
Since the square is assumed to be a pullback square, it follows from \cref{cor:pb_fibequiv} that for each $x:A$, the fiber $\fib{p}{x}$ is equivalent to the fiber $\fib{g}{f(x)}$, which is $k$-truncated. Since $k$-truncated types are closed under equivalences by \cref{thm:ktype_eqv}, it follows that $p$ is a $k$-truncated map.
\end{proof}

\begin{cor}\label{cor:pb_equiv}
Consider a commuting square
\begin{equation*}
\begin{tikzcd}
C \arrow[r,"q"] \arrow[d,swap,"p"] & B \arrow[d,"g"] \\
A \arrow[r,swap,"f"] & X.
\end{tikzcd}
\end{equation*}
and suppose that $g$ is an equivalence. Then the following are equivalent:
\begin{enumerate}
\item The square is a pullback square.
\item The map $p:C\to A$ is an equivalence.
\end{enumerate}
\end{cor}

\begin{proof}
If the square is a pullback square, then by \cref{thm:pb_fibequiv} the fibers of $p$ are equivalent to the fibers of $g$, which are contractible by \cref{thm:contr_equiv}. Thus it follows that $p$ is a contractible map, and hence that $p$ is an equivalence.

If $p$ is an equivalence, then by \cref{thm:contr_equiv} both $\fib{p}{x}$ and $\fib{g}{f(x)}$ are contractible for any $x:X$. It follows by \cref{ex:contr_equiv} that the induced map $\fib{p}{x}\to\fib{g}{f(x)}$ is an equivalence. Thus we apply \cref{cor:pb_fibequiv} to conclude that the square is a pullback.
\end{proof}

\section{The pullback pasting property}

\begin{thm}\label{thm:pb_pasting}
Consider a commuting diagram of the form
\begin{equation*}
\begin{tikzcd}
A \arrow[r,"k"] \arrow[d,swap,"f"] & B \arrow[r,"l"] \arrow[d,"g"] & C \arrow[d,"h"] \\
X \arrow[r,swap,"i"] & Y \arrow[r,swap,"j"] & Z
\end{tikzcd}
\end{equation*}
with homotopies $H:i\circ f\htpy g\circ k$ and $K:j\circ g\htpy h\circ l$, and the homotopy
\begin{equation*}
\ct{jH}{Kk}:j\circ i\circ f\htpy h\circ l\circ k
\end{equation*}
witnessing that the outer rectangle commutes. Furthermore, suppose that the square on the right is a pullback square. Then the following are equivalent:
\begin{samepage}%
\begin{enumerate}
\item The square on the left is a pullback square.
\item The outer rectangle is a pullback square.
\end{enumerate}%
\end{samepage}%
\end{thm}

\begin{proof}
The commutativity of the two squares induces fiberwise transformations
\begin{align*}
& \prd{x:X}\fib{f}{x}\to \fib{g}{i(x)} \\
& \prd{y:Y}\fib{g}{y}\to \fib{h}{j(y)}.
\end{align*}
By the assumption that the square on the right is a pullback square, it follows from \cref{cor:pb_fibequiv} that the fiberwise transformation
\begin{equation*}
\prd{y:Y}\fib{g}{y}\to\fib{h}{j(y)}
\end{equation*}
is a fiberwise equivalence. Therefore it follows from 3-for-2 property of equivalences that the fiberwise transformation
\begin{equation*}
\prd{x:X}\fib{f}{x}\to\fib{g}{i(x)}
\end{equation*}
is a fiberwise equivalence if and only if the fiberwise transformation
\begin{equation*}
\prd{x:X}\fib{f}{x}\to\fib{h}{j(i(x))}
\end{equation*}
is a fiberwise equivalence. Now the claim follows from one more application of \cref{cor:pb_fibequiv}.
\end{proof}

\section{The disjointness of coproducts}

As an application of the theory of pullbacks, we show that coproducts are disjoint.

\begin{lem}
Let $X$ be a type. Then we have the pullback squares
\begin{equation*}
\begin{tikzcd}
X \arrow[r,"\mathsf{const}_\ttt"] \arrow[d,swap,"\idfunc"] &[2em] \unit \arrow[d,"\mathsf{const}_{\bfalse}"] & \emptyt \arrow[r] \arrow[d] &[2em] \unit \arrow[d,"\mathsf{const}_{\btrue}"] \\
X \arrow[r,swap,"\mathsf{const}_{\bfalse}"] & \bool & X \arrow[r,swap,"\mathsf{const}_{\bfalse}"] & \bool,
\end{tikzcd}
\end{equation*}
and we have similar pullback squares with the roles of $\bfalse$ and $\btrue$ reversed.
\end{lem}

\begin{proof}
For the first square we observe that both squares and the outer rectangle in the diagram
\begin{equation*}
\begin{tikzcd}[column sep=large]
X \arrow[d] \arrow[r] & \unit \arrow[d] \arrow[r] & \unit \arrow[d,"\mathsf{const}_{\bfalse}"] \\
X \arrow[r,swap,"\mathsf{const}_\ttt"] & \unit \arrow[r,swap,"\mathsf{const}_{\bfalse}"] & \bool.
\end{tikzcd}
\end{equation*}
are pullback squares. To see this, recall that the identity type $\bfalse=\bfalse$ is contractible by \cref{ex:eq_bool}. Therefore it follows that the square on the right is a pullback square by \cref{ex:id_pb}. The square on the left is a pullback square by \cref{cor:pb_equiv}. Therefore the outer rectangle is a pullback square by \cref{thm:pb_pasting}.

For the second square we observe that both squares end the outer rectangle in the diagram
\begin{equation*}
\begin{tikzcd}[column sep=large]
\emptyt \arrow[d] \arrow[r] & \emptyt \arrow[d] \arrow[r] & \unit \arrow[d,"\mathsf{const}_{\btrue}"] \\
X \arrow[r,swap,"\mathsf{const}_\ttt"] & \unit \arrow[r,swap,"\mathsf{const}_{\bfalse}"] & \bool.
\end{tikzcd}
\end{equation*}
are pullback squares.
To see this, recall that the identity type $\bfalse=\btrue$ is equivalent to the empty type by \cref{ex:eq_bool}. Therefore it follows that the square on the right is a pullback. It is also straightforward to verify that the square on the left is a pullback. Therefore it follows from \cref{thm:pb_pasting} that the outer rectangle is a pullback.
\end{proof}

\begin{lem}
For any two types $A$ and $B$, the square
\begin{equation*}
\begin{tikzcd}[column sep=huge]
A \arrow[r,"\mathsf{const}_\ttt"] \arrow[d,swap,"\inl"] & \unit \arrow[d,"\mathsf{const}_{\bfalse}"] \\
A+B \arrow[r,swap,"\mathsf{const}_{\bfalse}+\mathsf{const}_{\btrue}" & \bool
\end{tikzcd}
\end{equation*}
is a pullback square.
\end{lem}

\begin{proof}
The square commutes by the homotopy
\begin{equation*}
H\defeq \mathsf{const}_{\refl{\bfalse}}.
\end{equation*}
To see that the asserted square is a pullback square we will apply \cref{thm:pb_3for2} and construct an equivalence
\begin{equation*}
e:\eqv{A}{(A+B)\times_\bool\unit}
\end{equation*}
equipped with homotopies
\begin{align*}
K & : \pi_1\circ e\htpy \inl \\
L & : \pi_2\circ e\htpy \mathsf{const}_\ttt \\
M & : \ct{(\pi_3\cdot e)}{(\mathsf{const}_{\bfalse})}\htpy \ct{((\mathsf{const}_{\bfalse}+\mathfs{const}_{\btrue})\cdot K)}{(\mathsf{const}_{\refl{\bfalse}})}.
\end{align*}
The map $e:A\to (A+B)\times_\bool\unit$ is defined by
\begin{equation*}
\lam{x}(\inl(x),\ttt,\refl{\bfalse}).
\end{equation*}
\end{proof}

\begin{thm}
For any two types $A$ and $B$, the commuting square
\begin{equation*}
\begin{tikzcd}
\emptyt \arrow[r] \arrow[d] & B \arrow[d,"\inr"] \\
A \arrow[r,swap,"\inl"] & A+B
\end{tikzcd}
\end{equation*}
is a pullback square.
\end{thm}

\begin{proof}
Now consider the commuting diagram
\begin{equation*}
\begin{tikzcd}
\emptyt \arrow[d] \arrow[r] & B \arrow[d,"\inr"] \arrow[r] &[5.5em] \unit \arrow[d,"\lam{\ttt}\inr(\ttt)"] \\
A \arrow[r,swap,"\inl"] & A+B \arrow[r,swap,"{\mathsf{const}_{\inl(\ttt)}+\mathsf{const}_{\inr(\ttt)}}"] & \unit + \unit.
\end{tikzcd}
\end{equation*}
By the first observation the outer rectangle is a pullback square. By the second observation the square on the right is a pullback square. Therefore the square on the left is a pullback square by \cref{thm:pb_pasting}.
\end{proof}

\begin{cor}\label{cor:id_coprod}
Let $A$ and $B$ be types. There are equivalences
\begin{align*}
(\inl(x)=\inl(x')) & \eqvsym (x=_A x') \\
(\inl(x)=\inr(y')) & \eqvsym \emptyt \\
(\inr(y)=\inl(x')) & \eqvsym \emptyt \\
(\inr(y)=\inr(y')) & \eqvsym (y=_B y').
\end{align*}
\end{cor}

\begin{exercises}
\item \label{ex:id_pb}\index{identity type!as pullback}
\begin{subexenum}
\item Show that the square
\begin{equation*}
\begin{tikzcd}
(x=y) \arrow[r] \arrow[d] & \unit \arrow[d,"\mathsf{const}_y"] \\
\unit \arrow[r,swap,"\mathsf{const}_x"] & A
\end{tikzcd}
\end{equation*}
is a pullback square.
\item Show that the square
\begin{equation*}
\begin{tikzcd}[column sep=large]
(x=y) \arrow[r,"\mathsf{const}_{x}"] \arrow[d,swap,"\mathsf{const}_\ttt"] & A \arrow[d,"\delta_A"] \\
\unit \arrow[r,swap,"{\ind{\unit}((x,y))}"] & A\times A
\end{tikzcd}
\end{equation*}
is a pullback square, where $\delta_A:A\to A\times A$ is the diagonal of $A$, defined in \cref{ex:diagonal}.
\end{subexenum}
\item In this exercise we give an alternative characterization of the notion of $k$-truncated map, compared to \cref{thm:trunc_ap} Given a map $f:A\to X$ define $\delta_f:A\to A\times_X A$ by $x\mapsto (x,x,\refl{f(x)})$.
\begin{subexenum}
\item Show that the square
\begin{equation*}
\begin{tikzcd}[column sep=large]
\fib{\apfunc{f}}{p} \arrow[r,"\mathsf{const}_x"] \arrow[d,swap,"\mathsf{const}_\ttt"] & A \arrow[d,"\delta_f"] \\
\unit \arrow[r,swap,"{\ind{\unit}((x,y,p))}"] & A\times_X A
\end{tikzcd}
\end{equation*}
is a pullback square, to obtain an equivalence
\begin{equation*}
\eqv{\fib{\delta_f}{(x,y,p)}}{\fib{\apfunc{f}}{p}}
\end{equation*}
for every $x,y:A$ and $p:f(x)=f(y)$.
\item Show that a map $f:A\to X$ is $(k+1)$-truncated if and only if $\delta_f$ is $k$-truncated.
\end{subexenum}
Conclude that $f$ is an embedding if and only if $\delta_f$ is an equivalence.
\item Consider a commuting square
\begin{equation*}
\begin{tikzcd}
C \arrow[r,"q"] \arrow[d,swap,"p"] & B \arrow[d,"g"] \\
A \arrow[r,swap,"f"] & X
\end{tikzcd}
\end{equation*}
with $H:f\circ p\htpy g\circ q$. Show that the following are equivalent:
\begin{enumerate}
\item The square is a pullback square.
\item For every type $D$, the commuting square
\begin{equation*}
\begin{tikzcd}
C^D \arrow[r,"q\circ\blank"] \arrow[d,swap,"p\circ\blank"] & B^D \arrow[d,"g\circ\blank"] \\
A^D \arrow[r,swap,"f\circ\blank"] & X^D
\end{tikzcd}
\end{equation*}
is a pullback square.
\end{enumerate}
\item Consider a commuting square
\begin{equation*}
\begin{tikzcd}
C \arrow[r,"q"] \arrow[d,swap,"p"] & B \arrow[d,"g"] \\
A \arrow[r,swap,"f"] & X
\end{tikzcd}
\end{equation*}
with $H:f\circ p\htpy g\circ q$. Show that the following are equivalent:
\begin{enumerate}
\item The square is a pullback square.
\item The square
\begin{equation*}
\begin{tikzcd}
C \arrow[r,"g\circ q"] \arrow[d,swap,"{\lam{x}(p(x),q(x))}"] & X \arrow[d,"\delta_X"] \\
A\times B \arrow[r,swap,"f\times g"] & X\times X
\end{tikzcd}
\end{equation*}
which commutes by $\lam{z}\mathsf{eq\usc{}pair}(H(z),\refl{g(q(z))})$ is a pullback square.
\end{enumerate}
\item Show that if
\begin{equation*}
\begin{tikzcd}
C_1 \arrow[r] \arrow[d] & B_1 \arrow[d] & C_2 \arrow[r] \arrow[d] & B_2 \arrow[d] \\
A_1 \arrow[r] & X_1 & A_2 \arrow[r] & X_2
\end{tikzcd}
\end{equation*}
are pullback squares, then so is
\begin{equation*}
\begin{tikzcd}
C_1\times C_2 \arrow[r] \arrow[d] & B_1\times B_2 \arrow[d] \\
A_1 \times A_2 \arrow[r] & X_1\times X_2. 
\end{tikzcd}
\end{equation*}
\item Consider for each $i:I$ a pullback square
\begin{equation*}
\begin{tikzcd}
C_i \arrow[r] \arrow[d] & B_i \arrow[d] \\
A_i \arrow[r] & X_i
\end{tikzcd}
\end{equation*}
with $H_i: f_i\circ p_i\htpy g_i\circ q_i$. 
\begin{subexenum}
\item Show that the commuting square
\begin{equation*}
\begin{tikzcd}
\sm{i:I}C_i \arrow[r] \arrow[d] & \sm{i:I}B_i \arrow[d] \\
\sm{i:I}A_i \arrow[r] & \sm{i:I}X_i
\end{tikzcd}
\end{equation*}
is a pullback square.
\item Show that the commuting square
\begin{equation*}
\begin{tikzcd}
\prd{i:I}C_i \arrow[r] \arrow[d] & \prd{i:I}B_i \arrow[d] \\
\prd{i:I}A_i \arrow[r] & \prd{i:I}X_i
\end{tikzcd}
\end{equation*}
is a pullback square.
\end{subexenum}
\item 
\begin{subexenum}
\item Show that 
\begin{equation*}
\begin{tikzcd}[column sep=8em]
\eqv{A}{B} \arrow[r] \arrow[d] & \unit \arrow[d,"{(\idfunc[A],\idfunc[B])}"] \\
A^B\times B^A \times A^B \arrow[r,swap,"{(h,f,g)\mapsto (h\circ f,f\circ g)}"] & A^A \times B^B
\end{tikzcd}
\end{equation*}
is a pullback square.
\item Show that
\begin{equation*}
\begin{tikzcd}[column sep=6em]
\iscontr(A) \arrow[r,"\mathsf{const}_{\ttt}"] \arrow[d,swap,"\proj 1"] & \unit \arrow[d,"{\lam{\ttt}\idfunc[A]}"] \\
A \arrow[r,swap,"{\lam{x}\mathsf{const}_x}"] & A^A
\end{tikzcd}
\end{equation*}
is a pullback square.
\end{subexenum}
%\item Consider a commuting square
%\begin{equation*}
%\begin{tikzcd}
%C \arrow[r] \arrow[d] & A \arrow[d] \\
%B \arrow[r] & X.
%\end{tikzcd}
%\end{equation*}
%Show that this square is cartesian if and only if the induced map $C\to A\times_X B$ has a retraction.
\item Let $B$ be a type family over $A$. Show that the square
\begin{equation*}
\begin{tikzcd}[column sep=6em]
\prd{x:A}B(x) \arrow[r,"{\lam{f}{x}(x,f(x))}"] \arrow[d] & \Big(\sm{x:A}B(x)\Big)^A \arrow[d,"\proj 1\blank\circ"] \\
\unit \arrow[r,swap,"{\lam{\ttt}\idfunc[A]}"] & A^A
\end{tikzcd}
\end{equation*}
is a pullback square. Conclude that the type $\prd{x:A}B(x)$ is equivalent to the type $\mathsf{sec}(\proj 1)$ of sections of the projection map.
%\end{subexenum}
%\item Suppose that the squares
%\begin{equation*}
%\begin{tikzcd}
%C \arrow[r,"q"] \arrow[d,swap,"p"] & B \arrow[d,"g"] & {C'} \arrow[r,"{q'}"] \arrow[d,swap,"{p'}"] & B \arrow[d,"g"] \\
%A \arrow[r,swap,"f"] & X & A \arrow[r,swap,"f"] & X
%\end{tikzcd}
%\end{equation*}
%with homotopies $H:f\circ p \htpy g\circ q$ and $H':f\circ p'\htpy g\circ q'$ are both pullback squares. Show that the type of equivalences $e:\eqv{C'}{C}$ equipped with an identification
%\begin{equation*}
%\mathsf{cone\usc{}map}((p,q,H),e)=(p',q',H')
%\end{equation*}
%is contractible.
\begin{comment}
\item Consider a \define{natural transformation of cospans}\index{cospan!natural transformation of}, i.e.~a commuting diagram of the form
\begin{equation*}
\begin{tikzcd}
A \arrow[r,"f"] \arrow[d,swap,"i"] & X \arrow[d,swap,"j"] & B \arrow[l,swap,"g"] \arrow[d,"k"] \\
A' \arrow[r,swap,"{f'}"] & X' & B'. \arrow[l,"{g'}"]
\end{tikzcd}
\end{equation*}
Show that the map
\begin{equation*}
(a,b,p)\mapsto (i(a),j(b),\mathsf{ap}_k(p)): A \times_X B \to A'\times_{X'} B'
\end{equation*}
is $k$-truncated if each of the vertical maps is.
\end{comment}
\end{exercises}
