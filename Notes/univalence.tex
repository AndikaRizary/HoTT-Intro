\chapter{The univalence axiom}

\section{Type extensionality}

The univalence axiom characterizes the identity type of the universe. Roughly speaking, it asserts that equivalent types are equal. It is considered to be an \emph{extensionality principle}\index{extensionality principle!types} for types. In the following theorem we introduce the univalence axiom and give two more equivalent ways of stating this.

\begin{thm}\label{thm:univalence}
The following are equivalent:
\begin{enumerate}
\item The \define{univalence axiom}\index{univalence axiom|textbf}: for any $A:\UU$ the map\index{equiv_eq@{$\mathsf{equiv\usc{}eq}$}|textbf}
\begin{equation*}
\mathsf{equiv\usc{}eq}\defeq \ind{A=}(\idfunc[A]) : \prd{B:\UU} (\id{A}{B})\to(\eqv{A}{B}).
\end{equation*}
is a fiberwise equivalence.\index{identity type!universe} If this is the case, we write
$\mathsf{eq\usc{}equiv}$\index{eq equiv@{$\mathsf{eq\usc{}equiv}$}|textbf}
for the inverse of $\mathsf{equiv\usc{}eq}$.
\item The type
\begin{equation*}
\sm{B:\UU}\eqv{A}{B}
\end{equation*}
is contractible for each $A:\UU$.
\item The principle of \define{equivalence induction}\index{equivalence induction}\index{induction principle!for equivalences}: for every $A:\UU$ and for every type family
\begin{equation*}
P:\prd{B:\UU} (\eqv{A}{B})\to \mathsf{Type},
\end{equation*}
the map
\begin{equation*}
\Big(\prd{B:\UU}{e:\eqv{A}{B}}P(B,e)\Big)\to P(A,\idfunc[A])
\end{equation*}
given by $f\mapsto f(A,\idfunc[A])$ has a section.
\end{enumerate}
\end{thm}

\begin{proof}
The equivalence of (i) and (ii) is a direct consequence of \cref{thm:id_fundamental}. 
To see that (ii) and (iii) are equivalent, note that we have a commuting triangle
\begin{equation*}
\begin{tikzcd}[column sep=-1em]
\prd{t:\sm{B:\UU}\eqv{A}{B}}P(t) \arrow[rr,"\mathsf{ev\usc{}pair}"] \arrow[dr,"\varphi","{f\mapsto f((A,\idfunc[A]))}"'] & & \prd{B:\UU}{e:\eqv{A}{B}} P((B,e)) \arrow[dl,"{f\mapsto f(A,\idfunc[A])}","\psi"'] \\
& P((A,\idfunc[A]))
\end{tikzcd}
\end{equation*}
The map $\ind{\Sigma}$ has a section. Therefore it follows from \cref{ex:3_for_2} that $\varphi$ has a section if and only if $\psi$ has a section. By \cref{thm:contractible} it follows that $\varphi$ has a section if and only if $\sm{B:\UU}\eqv{A}{B}$ is contractible. 
\end{proof}

From now on we will assume that the univalence axiom holds.

\section{Groups in univalent mathematics}

In this section we exhibit a typical way to use the univalence axiom, showing that isomorphic groups can be identified.
This is an instance of the \emph{structure identity principle}\index{structure identity principle}, which is described in more detail in section 9.8 of \cite{hottbook}.
We will see that in order to establish the fact that isomorphic groups can be identified, it has to be part of the definition of a group that its underlying type is a set. This is an important observation: in many branches of algebra the objects of study are \emph{set-level} structures\footnote{A notable exception is that of categories, which are objects at truncation level $1$, i.e. at the level of \emph{groupoids}. For more on this, see Chapter 9 of \cite{hottbook}.}.

\begin{defn}
A \define{group}\index{group|textbf} $\mathcal{G}$ consists of a type $G$ equipped with
\begin{samepage}
\begin{align*}
p & : \mathsf{is\usc{}set}(G) \\
1 & : G \\
i & : G\to G \\
\mu & : G\to (G\to G),
\end{align*}
\end{samepage}
satisfying the \define{group laws}\index{group laws|textbf}:
\begin{align*}
\mathsf{assoc} & : \prd{x,y,z:G} \mu(\mu(x,y),z)=\mu(x,\mu(y,z)) \\
\mathsf{left\usc{}unit} & : \prd{x:G} \mu(1,x)=x \\
\mathsf{right\usc{}unit} & : \prd{x:G} \mu(x,1)=x \\
\mathsf{left\usc{}inv} & : \prd{x:G} \mu(i(x),x)=1 \\
\mathsf{right\usc{}inv} & : \prd{x:G} \mu(x,i(x))=x.
\end{align*}
The type $\mathsf{Grp}$\index{Grp@{$\mathsf{Grp}$}|textbf} of all small groups is defined as
\begin{align*}
\mathsf{Grp} & \defeq \sm{G:\UU}{p:\mathsf{is\usc{}set}(G)}{1:G}{i:G\to G}{\mu:G\to(G\to G)} \\
& \qquad \Big(\prd{x,y,z:G} \mu(\mu(x,y),z)=\mu(x,\mu(y,z))\Big) \times \\
& \qquad \Big(\prd{x:G} \mu(1,x)=x\Big)\times\Big(\prd{x:G} \mu(x,1)=x\Big) \times \\
& \qquad \Big(\prd{x:G} \mu(i:x,x)=1\Big)\times\Big(\prd{x:G} \mu(x,i(x))=x\Big).
\end{align*}
We will usually write $x^{-1}$ for $i(x)$, and $xy$ for $\mu(x,y)$. The binary operation $\mu$ is also referred to as the \define{group operation}\index{group operation|textbf}.
\end{defn}

\begin{eg}
An important class of examples consists of the \define{loop space}\index{loop space|textbf} $x=x$ of a $1$-type $X$, for any $x:X$. 
We will write $\loopspace{X,x}$ for the loop space of $X$ at $x$. 
Since $X$ is assumed to be a $1$-type, it follows that the type $\loopspace{X,x}$ is a set. Then we have
\begin{align*}
\refl{x} & : \loopspace{X,x} \\
\mathsf{inv} & : \loopspace{X,x} \to \loopspace{X,x} \\
\mathsf{concat} & : \loopspace{X,x} \to (\loopspace{X,x}\to \loopspace{X,x}),
\end{align*}
and these operations satisfy the group laws, since the group laws are just a special case of the groupoid laws for identity types, constructed in \cref{sec:groupoid}.

Using higher inductive types we will show in \cref{chap:image} that \emph{every} group is of this form.
\end{eg}

\begin{eg}
The type $\Z$ of integers\index{Z@{$\Z$}!is a group} can be given the structure of a group, with the group operation being addition. The fact that $\Z$ is a set follows from \cref{thm:eq_nat,ex:set_coprod}. The group laws were shown in \cref{ex:int_group_laws}. 
\end{eg}

\begin{defn}
Let $\mathcal{G}$ and $\mathcal{G}'$ be groups. The type $\mathrm{hom}(\mathcal{G},\mathcal{G}')$ of \define{group homomorphisms}\index{group homomorphism|textbf} from $\mathcal{G}$ to $\mathcal{G}'$ is defined to be the type of pairs $(h,p)$ consisting of
\begin{samepage}
\begin{align*}
h & : G \to G' \\
p & : \prd{x,y:G} f(xy)=f(x)f(y).
\end{align*}
\end{samepage}
\end{defn}

\begin{rmk}
Since preservation of the group operation is a property, we will usually write $h$ for the group homomorphism $(h,p)$.
Moreover, from \cref{thm:subtype} we obtain that the projection map
\begin{equation*}
\proj 1:\mathrm{hom}(\mathcal{G},\mathcal{G}')\to (G\to G')
\end{equation*}
is an embedding. Therefore the equality type $(h,p)=(h',p')$ is equivalent to $h=h'$. In other words, to show that two group homomorphisms are equal it suffices to show that their underlying maps are equal.
\end{rmk}

\begin{lem}
For any two groups $\mathcal{G}$, and $\mathcal{H}$, the type $\mathrm{hom}(\mathcal{G},\mathcal{H})$ is equivalent to the type of quadruples $(f,\alpha,\beta,\gamma)$ consisting of
\begin{align*}
f & : G \to H \\
\alpha & : f(1)=1 \\
\beta & : \prd{x:G}f(x^{-1})=f(x)^{-1} \\
\gamma & : \prd{x,y:G} f(xy)=f(x)f(y).
\end{align*}
\end{lem}

\begin{proof}
It suffices to show that for any group homomorphism $f:\mathrm{hom}(\mathcal{G},\mathcal{H})$, the types $f(1)=1$ and 
\begin{equation*}
\Big(\prd{x:G}f(x^{-1})=f(x)^{-1}\Big)
\end{equation*}
are contractible. Since $G$ is a set, both types are propositions. Therefore it suffices to show they are inhabited. In other words, it suffices to show that any group homomorphism preserves the unit element and inverses. These are just calculations, where each step is an application of a group law:
\begin{align*}
f(1) & = 1f(1)               & f(x^{-1}) & = f(x^{-1})1 \\
     & = (f(1)^{-1}f(1))f(1) &           & = f(x^{-1})(f(x)f(x)^{-1}) \\
     & = f(1)^{-1}(f(1)f(1)) &           & = (f(x^{-1})f(x))f(x)^{-1} \\
     & = f(1)^{-1}f(11)      &           & = f(x^{-1}x)f(x)^{-1} \\
     & = f(1)^{-1}f(1)       &           & = f(1)f(x)^{-1} \\
     & = 1.                  &           & = 1f(x)^{-1} \\
     &                       &           & = f(x)^{-1}.\qedhere
\end{align*}
\end{proof}

\begin{defn}
Let $\mathcal{G}$ be a group. Then the \define{identity homomorphism}\index{group homomorphism!identity homomorphism|textbf} $\idfunc[\mathcal{G}]:\mathrm{hom}(\mathcal{G},\mathcal{G})$ is defined to be the pair $(\idfunc[G],p)$, where 
\begin{equation*}
p(x,y)\defeq \refl{xy}. 
\end{equation*}
\end{defn}

\begin{defn}
Let $h:\mathrm{hom}(\mathcal{G},\mathcal{H})$ and $k:\mathrm{hom}(\mathcal{H},\mathcal{K})$ be group homomorphisms, with proofs $p$ and $q$ that $h$ and $k$ preserve the group operation, respectively. Then we define\index{group homomorphism!composition|textbf}
\begin{equation*}
k\circ h:\mathrm{hom}(\mathcal{G},\mathcal{K})
\end{equation*}
to be the group homomorphism with underlying map $k\circ h$. This map preserves the group operation since
\begin{equation*}
\begin{tikzcd}
k(h(xy)) \arrow[r,equals] & k(h(x)h(y)) \arrow[r,equals] & k(h(x))k(h(y)).
\end{tikzcd}
\end{equation*}
\end{defn}

\begin{defn}
Let $h:\mathrm{hom}(\mathcal{G},\mathcal{H})$ be a group homomorphism. Then $h$ is said to be an \define{isomorphism}\index{group homomorphism!isomorphism}\index{isomorphism!of groups} if there is a group homomorphism $h^{-1}:\mathrm{hom}(\mathcal{H},\mathcal{G})$ such that
\begin{equation*}
h^{-1}\circ h=\idfunc[\mathcal{G}]\qquad\text{and}\qquad h\circ h^{-1}=\idfunc[\mathcal{H}].
\end{equation*}
We write $\mathcal{G}\cong\mathcal{H}$ for the type of all group isomorphisms from $\mathcal{G}$ to $\mathcal{H}$, i.e.
\begin{equation*}
\mathcal{G}\cong\mathcal{H} \defeq \sm{h:\mathrm{hom}(\mathcal{G},\mathcal{H})}{k:\mathrm{hom}(\mathcal{H},\mathcal{G})} (k\circ h = \idfunc[\mathcal{G}])\times (h\circ k=\idfunc[\mathcal{H}]).
\end{equation*}
\end{defn}

\begin{lem}\label{lem:grp_iso}
The type of isomorphisms $\mathcal{G}\cong\mathcal{H}$ is equivalent to the type
\begin{align*}
e & : \eqv{G}{H} \\
\alpha & : e(1)=1 \\
\beta & : \prd{x:G}e(x^{-1})=e(x)^{-1} \\
\gamma & : \prd{x,y:G} e(xy)=e(x)e(y).
\end{align*}
\end{lem}

\begin{proof}
The standard proof showing that if the underlying map $f:G\to H$ of a group homomorphism is invertible then its inverse is again a group homomorphism, also works in type theory. Since being a group homomorphism is a property, it follows that the type of group isomorphism is equivalent to the type of group homomorphisms of which the underlying map has an inverse. By \cref{ex:iso_equiv} it follows that the type 
\begin{equation*}
\sm{f:\mathrm{hom}(\mathcal{G},\mathcal{H})}\mathsf{is\usc{}invertible}(f)
\end{equation*}
of group homomorphism of which the underlying map has an inverse is equivalent to the type
\begin{equation*}
\sm{f:\mathsf{hom}(\mathcal{G},\mathcal{H})}\isequiv(f).
\end{equation*}
of group homomorphisms of which the underlying map is an equivalence.
\end{proof}

\begin{defn}
Let $\mathcal{G}:\mathsf{Grp}$ be a group. We define the map\index{iso_eq@{$\mathsf{iso\usc{}eq}$}|textbf}
\begin{equation*}
\mathsf{iso\usc{}eq} : \prd{\mathcal{H}:\mathsf{Grp}}(\mathcal{G}=\mathcal{H})\to (\mathcal{G}\cong\mathcal{H})
\end{equation*}
by path induction, taking $\refl{\mathcal{G}}$ to $\idfunc[\mathcal{G}]$. Indeed, $\idfunc[\mathcal{G}]$ is a group isomorphism since it is its own inverse.
\end{defn}

\begin{thm}
The fiberwise transformation
\begin{equation*}
\mathsf{iso\usc{}eq} : \prd{\mathcal{G}':\mathsf{Grp}}(\mathcal{G}=\mathcal{G}')\to (\mathcal{G}\cong\mathcal{G}')
\end{equation*}
is a fiberwise equivalence, for any group $\mathcal{G}$.
\end{thm}

\begin{proof}
We will apply \cref{thm:id_fundamental}, and show that the type
\begin{equation*}
\sm{\mathcal{G}':\mathsf{Grp}}\mathcal{G}\cong\mathcal{G}'
\end{equation*}
is contractible.
By \cref{lem:grp_iso} it follows that the total space $\sm{\mathcal{G}':\mathsf{Grp}}(\mathcal{G}\cong\mathcal{G}')$ is equivalent to the type
\begin{samepage}
\begin{align*}
& \sm{G':\UU}{p':\mathsf{is\usc{}set}(G')} \\
& \qquad \sm{1':G}{i':G'\to G'}{\mu':G'\to (G'\to G')}{L':\mathsf{group\usc{}laws}(G',1',i',\mu')} \\
& \qquad \qquad \sm{e:\eqv{G}{G'}} \Big(e(1)=1'\Big) \times \Big(\prd{x:G}e(x^{-1})= i'(e(x))\Big)\times\\
& \qquad \qquad \qquad\Big(\prd{x,y:G}e(xy)=\mu'(e(x),e(y))\Big).
\end{align*}
\end{samepage}%
By the univalence axiom, the type $\sm{G':\UU}\eqv{G}{G'}$ is contractible. Thus we see that the above type is equivalent to
\begin{samepage}
\begin{align*}
& \sm{q:\mathsf{is\usc{}set}(G)}{1':G}{i':G\to G}{\mu':G\to (G\to G)}{L:\mathsf{group\usc{}laws}(G,1',i',\mu')} \\
& \qquad (1=1') \times \Big(\prd{x:G}x^{-1}= i'(x)\Big)\times\Big(\prd{x,y:G}xy=\mu'(x,y)\Big).
\end{align*}
\end{samepage}
Of course, the types
\begin{align*}
& \sm{1':G} 1=1' \\
& \sm{i':G\to G}\prd{x:G}x^{-1}=i'(x) \\
& \sm{\mu':G\to (G\to G)}\Big(\prd{x,y:G}xy=\mu'(x,y)\Big)
\end{align*}
are all contractible. Moreover, being a set is a proposition, and since $G$ is a set the group laws are propositions too. Since $G$ is already assumed to be a set on which the group operations satisfy the group laws, it follows that the types $\mathsf{is\usc{}set}(G)$ and $\mathsf{group\usc{}laws}(G,1,i,\mu)$ are all contractible. This concludes the proof that the total space $\sm{\mathcal{G}':\mathsf{Grp}}\mathcal{G}\cong\mathcal{G}'$ is contractible. 
\end{proof}

\begin{cor}
The type $\mathsf{Grp}$ is a $1$-type.\index{Grp@{$\mathsf{Grp}$}!is a $1$-type|textit}
\end{cor}

\begin{proof}
It is straightforward to see that the type of group isomorphisms $\mathcal{G}\cong\mathcal{H}$ is a set, for any two groups $\mathcal{G}$ and $\mathcal{H}$.
\end{proof}

\section{Equivalence relations}

\begin{defn}\label{defn:eq_rel}
Let $R:A\to (A\to\prop)$ be a binary relation valued in the propositions. We say that $R$ is an \define{($0$-)equivalence relation}\index{equivalence relation|textbf}\index{0-equivalence relation|see {equivalence relation}} if $R$ comes equipped with
\begin{align*}
\rho & : \prd{x:A}R(x,x) \\
\sigma & : \prd{x,y:A} R(x,y)\to R(y,x) \\
\tau & : \prd{x,y,z:A} R(x,y)\to (R(y,z)\to R(x,z)).
\end{align*}
Given an equivalence relation $R:A\to (A\to\prop)$, the \define{equivalence class}\index{equivalence class|textbf} $[x]_R$ of $x:A$ is defined to be
\begin{equation*}
[x]_R\defeq R(x).
\end{equation*}
\end{defn}

\begin{defn}
Let $R:A\to (A\to\prop)$ be a $0$-equivalence relation. 
We define for any $x,y:A$ a map\index{class_eq@{$\mathsf{class\usc{}eq}$}|textbf}
\begin{equation*}
\mathsf{class\usc{}eq}:R(x,y)\to ([x]_R=[y]_R).
\end{equation*}
\end{defn}

\begin{proof}[Construction.]
Let $r:R(x,y)$. By function extensionality, the identity type $R(x)=R(y)$ is equivalent to the type
\begin{equation*}
\prd{z:A}R(x,z)=R(y,z).
\end{equation*}
Let $z:A$. By the univalence axiom, the type $R(x,z)=R(y,z)$ is equivalent to the type
\begin{equation*}
\eqv{R(x,z)}{R(y,z)}.
\end{equation*}
We have the map $\tau_{y,x,z}(\sigma(r)):R(x,z)\to R(y,z)$. Since this is a map between propositions, we only have to construct a map in the converse direction to show that it is an equivalence. The map in the converse direction is just $\tau_{x,y,z}(r):R(y,z)\to R(x,z)$. 
\end{proof}

\begin{thm}\label{thm:equivalence_classes}
Let $R:A\to (A\to\prop)$ be a $0$-equivalence relation. 
Then for any $x,y:A$ the map
\begin{equation*}
\mathsf{class\usc{}eq} : R(x,y)\to ([x]_R=[y]_R)
\end{equation*}
is an equivalence.
\end{thm}

\begin{proof}
By the 3-for-2 property of equivalences, it suffices to show that the map
\begin{equation*}
\lam{r}{z}\tau_{y,x,z}(\sigma(r)) : R(x,y)\to \prd{z:A} \eqv{R(x,z)}{R(y,z)}
\end{equation*}
is an equivalence. Since this is a map between propositions, it suffices to construct a map of type
\begin{equation*}
\Big(\prd{z:A} \eqv{R(x,z)}{R(y,z)}\Big)\to R(x,y).
\end{equation*}
This map is simply $\lam{f} \sigma_{y,x}(f_x(\rho(x)))$. 
\end{proof}

\begin{rmk}
By \cref{thm:equivalence_classes} we can begin to think of the \emph{quotient}\index{quotient} $A/R$ of a type $A$ by an equivalence relation $R$. Classically, the quotient is described as the set of equivalence classes, and \cref{thm:equivalence_classes} establishes that two equivalence classes $[x]_R$ and $[y]_R$ are equal precisely when $x$ and $y$ are related by $R$.

However, the type $A\to\prop$ may contain many more terms than just the $R$-equivalence classes. Therefore we are facing the task of finding a type theoretic description of the smallest subtype of $A\to\prop$ containing the equivalence classes.
Another to think about this is as the \emph{image}\index{image} of $R$ in $A\to \prop$. 
The construction of the (homotopy) image of a map can be done with \emph{higher inductive types}\index{higher inductive type}, which we will do in \cref{chap:image}.
\end{rmk}

The notion of $0$-equivalence relation which we defined in \cref{defn:eq_rel} fits in a hierarchy of `$n$-equivalence relations'\index{n-equivalence relation@{$n$-equivalence relation}}, the study of which is a research topic on its own. However, we already know an example of a relation that should count as an `$\infty$-equivalence relation'\index{infinity-equivalence relation@{$\infty$-equivalence relation}}: the identity type. Analogous to \cref{thm:equivalence_classes}, the following theorem shows that the canonical map
\begin{equation*}
(x=y)\to (\idtypevar{A}(x)=\idtypevar{A}(y))
\end{equation*}
is an equivalence, for any $x,y:A$. In other words, $\idtypevar{A}(x)$ can be thought of as the equivalence class of $x$ with respect to the relation $\idtypevar{A}$.

\begin{thm}
Assuming the univalence axiom on $\UU$, the map
\begin{equation*}
\idtypevar{A}:A\to (A\to\UU)
\end{equation*}
is an embedding, for any type $A:\UU$.\index{identity type!is an embedding|textit}
\end{thm}

\begin{proof}
Let $a:A$. By function extensionality it suffices to show that the canonical map
\begin{equation*}
(a=b)\to \idtypevar{A}(a)\htpy\idtypevar{A}(b)
\end{equation*}
that sends $\refl{a}$ to $\lam{x}\refl{(a=x)}$ is an equivalence for every $b:A$, and by univalence it therefore suffices to show that the canonical map
\begin{equation*}
(a=b)\to \prd{x:A}\eqv{(a=x)}{(b=x)}
\end{equation*}
that sends $\refl{a}$ to $\lam{x}\idfunc[(a=x)]$ is an equivalence for every $b:B$. To do this we employ the type theoretic Yoneda lemma, \autoref{thm:yoneda}.

By the type theoretic Yoneda lemma\index{Yoneda lemma} we have an equivalence
\begin{equation*}
\Big(\prd{x:A} (b=x)\to (a=x)\Big)\to (a=b)
\end{equation*}
given by $\lam{f} f(b,\refl{b})$, for every $b:A$. Note that any fiberwise map $\prd{x:A}(b=x)\to (a=x)$ induces an equivalence of total spaces by \autoref{ex:contr_equiv}, since their total spaces are are both contractible by \autoref{cor:contr_path}. It follows that we have an equivalence
\begin{equation*}
\varphi_b:\Big(\prd{x:A} \eqv{(b=x)}{(a=x)}\Big)\to (a=b)
\end{equation*}
given by $\lam{f} f(b,\refl{b})$, for every $b:A$. 

Note that $\varphi_a(\lam{x}\idfunc[(a=x)])\jdeq\refl{a}$. Therefore it follows by another application of \autoref{thm:yoneda} that the unique family of maps 
\begin{equation*}
\alpha_b:(a=b)\to \Big(\prd{x:A} \eqv{(b=x)}{(a=x)}\Big)
\end{equation*}
that satisfies $\alpha_a(\refl{a})=\lam{x}\idfunc[(a=x)]$ is a fiberwise section of $\varphi$. 
It follows that $\alpha$ is a fiberwise equivalence. Now the proof is completed by reverting the direction of the fiberwise equivalences in the codomain.
\end{proof}

\begin{comment}
\section{Univalence implies function extensionality}
The first application of the univalence axiom was Voevodsky's proof of function extensionality. Just for the purpose of the following theorem we drop our assumption of function extensionality. 

\begin{thm}
The univalence axiom implies function extensionality for small types. 
\end{thm}

\begin{proof}
Note that \cref{thm:funext_wkfunext} also holds when it is restricted to small types. 
Therefore it suffices to show that univalence implies the weak principle of function extensionality.

To see this, we first note that for any equivalence $e:\eqv{X}{Y}$, the post-composition map $e\circ \blank:\eqv{(A\to X)}{(A\to Y)}$. This is is obvious in the case $e\jdeq\idfunc[A]$, so it follows from equivalence induction.

Now suppose that $B:A\to \UU$ is a family of contractible types. Our goal is to show that the product $\prd{x:A}B(x)$ is contractible.
Since each $B(x)$ is contractible, the projection map $\proj 1:\big(\sm{x:A}B(x)\big)\to A$ is an equivalence by \cref{ex:proj_fiber}. It follows from \cref{thm:contr_equiv} that the fibers of
\begin{equation*}
\proj 1\circ\blank : \Big(A\to \sm{x:A}B(x)\Big)\to (A\to A)
\end{equation*}
are contractible. In particular, the fiber at $\idfunc[A]$ is contractible. Therefore it suffices to show that $\prd{x:A}B(x)$ is a retract of $\sm{f:A\to\sm{x:A}B(x)}\proj 1\circ f=\idfunc[A]$. In other words, we will construct
\begin{equation*}
\begin{tikzcd}
\Big(\prd{x:A}B(x)\Big) \arrow[r,"i"] & {\sm{f:A\to\sm{x:A}B(x)}\proj 1\circ f=\idfunc[A]} \arrow[r,"r"] & \Big(\prd{x:A}B(x)\Big),
\end{tikzcd}
\end{equation*}
and a homotopy $r\circ i\htpy \idfunc$. We define
\begin{align*}
i & \defeq \lam{f}(\lam{x}(x,f(x)),\lam{x}\refl{x}) \\
r & \defeq \lam{(f,H)}{x}\mathsf{tr}_B(...,\proj 2(f(x)))
\end{align*}
\end{proof}
\end{comment}

\begin{exercises}
\item \label{ex:tr_ap} Show that for any $P:X\to \UU$ and any $p:x=y$ in $X$, we have\index{equiv_eq@{$\mathsf{equiv\usc{}eq}$}}\index{transport}
\begin{equation*}
\mathsf{equiv\usc{}eq}(\ap{P}{p})\htpy \mathsf{tr}_P(p).
\end{equation*}
\item \label{ex:istrunc_UUtrunc}
\begin{subexenum}
\item Use the univalence axiom to show that the type $\sm{A:\UU}\iscontr(A)$ of all contractible types in $\UU$ is contractible.\index{universe!of contractible types}
\item Use \cref{cor:emb_into_ktype,cor:funtype_trunc,ex:isprop_isequiv} to show that if $A$ and $B$ are $(k+1)$-types, then the type $\eqv{A}{B}$ is also a $(k+1)$-type.\index{equiv@{$\eqv{A}{B}$}!truncatedness}
\item Use univalence to show that the universe of $k$-types\index{universe!of k-types@{of $k$-types}}\index{U leq k@{$\UU^{\leq k}$}|textbf}
\begin{equation*}
\UU^{\leq k}\defeq \sm{X:\UU}\mathsf{is\usc{}trunc}_k(X)
\end{equation*}
is a $(k+1)$-type, for any $k\geq -2$.
\item It follows that the universe of propositions $\UU^{\leq-1}$ is a set. However, show that $\UU^{\leq-1}$ is not a proposition.\index{universe!of propositions}
\item Show that $\eqv{(\eqv{\bool}{\bool})}{\bool}$, and conclude by the univalence axiom that the universe of sets\index{universe!of sets} $\UU^{\leq 0}$ is not a set. 
\end{subexenum}
\item Use the univalence axiom to show that the type $\sm{P:\prop}P$ is contractible.
\item Let $A$ and $B$ be small types. 
\begin{subexenum}
\item Construct an equivalence
\begin{equation*}
\eqv{(A\to (B\to\UU))}{\Big(\sm{S:\UU} (S\to A)\times (S\to B)\Big)}
\end{equation*}
\item We say that a relation $R:A\to (B\to\UU)$ is \define{functional}\index{relation!functional} if it comes equipped with a term of type\index{is_function(R)@{$\mathsf{is\usc{}function}(R)$}|textbf}
\begin{equation*}
\mathsf{is\usc{}function}(R) \defeq \prd{x:A}\iscontr\Big(\sm{y:B}R(x,y)\Big)
\end{equation*}
For any function $f:A\to B$, show that the \define{graph}\index{graph!of a function|textbf} of $f$ 
\begin{equation*}
\mathsf{graph}_f:A\to (B\to \UU)
\end{equation*}
given by $\mathsf{graph}_f(a,b)\defeq (f(a)=b)$ is a functional relation from $A$ to $B$.
\item Construct an equivalence
\begin{equation*}
\eqv{\Big(\sm{R:A\to (B\to\UU)}\mathsf{is\usc{}function}(R)\Big)}{(A\to B)}
\end{equation*}
\item Given a relation $R:A\to (B\to \UU)$ we define the \define{opposite relation}\index{relation!opposite relation|textbf}\index{opposite relation|textbf}\index{op R@{$R^{\mathsf{op}}$}|textbf}
\begin{equation*}
R^{\mathsf{op}} : B\to (A\to\UU)
\end{equation*}
by $R^{\mathsf{op}}(y,x)\defeq R(x,y)$. Construct an equivalence\index{equiv@{$\eqv{A}{B}$}!as relation}
\begin{equation*}
\eqv{\Big(\sm{R:A\to (B\to \UU)}\mathsf{is\usc{}function}(R)\times\mathsf{is\usc{}function}(R^{\mathsf{op}})\Big)}{(\eqv{A}{B})}.
\end{equation*}
\end{subexenum}
\item
  \begin{subexenum}
  \item Show that for any proposition $P$, the type
    \begin{equation*}
      P+\neg P
    \end{equation*}
    is again a proposition. We call a proposition $P$ \define{decidable} if it comes equipped with a term of type $P+\neg P$.
  \item Show that the type of decidable propositions is equivalent to $\bool$.
  \end{subexenum}
\end{exercises}
