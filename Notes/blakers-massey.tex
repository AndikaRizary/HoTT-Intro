% !TEX root = hott_intro.tex

\chapter{The Blakers-Massey theorem}
The Blakers-Massey theorem is a connectivity theorem which can be used to prove the Freudenthal suspension theorem, giving rise to the field of \emph{stable homotopy theory}. It was proven in the setting of homotopy type theory by Lumsdaine et al, and their proof was the first that was given entirely in an elementary way, using only constructions that are invariant under homotopy equivalence. 

Consider a span $A \leftarrow S \rightarrow B$, consisting of an $m$-connected map $f:S\to A$ and an $n$-connected map $g:S\to B$. We take the pushout of this span, and subsequently the pullback of the resulting cospan, as indicated in the diagram
\begin{equation}\label{eq:BM}
\begin{tikzcd}
S \arrow[drr,bend left=15,"g"] \arrow[ddr,bend right=15,swap,"f"] \arrow[dr,densely dotted,"u" near end] \\
& A \times_{(A \sqcup^S B)} B \arrow[r,"\pi_2"] \arrow[d,"\pi_1"] & B \arrow[d,"\inr"] \\
& A \arrow[r,swap,"\inl"] & A \sqcup^S B.
\end{tikzcd}
\end{equation}
The universal property of the pullback determines a unique map $u:S\to A \times_{(A\sqcup^S B)} B$ as indicated.

\begin{thm}[Blakers-Massey]
The map $u:S\to A \times_{(A\sqcup^S B)} B$ of \cref{eq:BM} is $(n+m)$-connected.
\end{thm}

\begin{exercises}
\item Show that if $X$ is $m$-connected and $f:X\to Y$ is $n$-connected, then the map
\begin{equation*}
X \to \fib{m_f}{\ast}
\end{equation*}
where $m_f:Y\to M_f$ is the inclusion of $Y$ into the cofiber of $f$, is $(m+n)$-connected.
\item Suppose that $X$ is a connected type, and let $f:X\to Y$ be a map.
Show that the following are equivalent:
\begin{enumerate}
\item $f$ is $n$-connected.
\item The mapping cone of $f$ is $(n+1)$-connected.
\end{enumerate}
\item Apply the Blakers-Massey theorem to the defining pushout square of the smash product to show that if $A$ and $B$ are $m$- and $n$-connected respectively, then there is a $(m+n+\min(m,n)+2)$-connected map
\begin{equation*}
\join{\loopspace{A}}{\loopspace{B}}\to \loopspace{A \wedge B}.
\end{equation*}
\item Show that the square
\begin{equation*}
\begin{tikzcd}
\unit \arrow[r] \arrow[d] & \bool \arrow[d] \\
X \arrow[r] & X+\unit
\end{tikzcd}
\end{equation*}
is both a pullback and a pushout. Conclude that the result of the Blakers-Massey theorem is not always sharp.
\end{exercises}
