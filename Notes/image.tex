\chapter{The homotopy image of a map}\label{chap:image}
\section{The universal property of the image of a map}

\begin{defn}
Let $f:A\to X$ and $g:B\to X$ be maps. We define
\begin{equation*}
\mathrm{hom}_X(f,g)\defeq\sm{h:A\to B}f\htpy g\circ h.
\end{equation*}
\end{defn}

\begin{rmk}
In other words, a term $(h,H):\mathrm{hom}_X(f,g)$ consists of a map $h:A\to B$ equipped with a homotopy $H:f\htpy g\circ h$ witnessing that the triangle
\begin{equation*}
\begin{tikzcd}[column sep=tiny]
A \arrow[rr,"h"] \arrow[dr,swap,"f"] & & B \arrow[dl,"g"] \\
& X
\end{tikzcd}
\end{equation*}
commutes. Recall from \cref{ex:triangle_fib} that the type $\mathrm{hom}_X(f,g)$ is equivalent to the type
\begin{equation*}
\prd{x:X}\fib{f}{x}\to\fib{g}{x}.
\end{equation*}
\end{rmk}

\begin{lem}
For any $f:A\to X$ and any embedding\index{embedding} $m:B\to X$, the type $\mathrm{hom}_X(f,m)$ is a proposition.
\end{lem}

\begin{proof}
Since propositions are closed under equivalences by \cref{lem:prop_equiv}, it suffices to show that the type
\begin{equation*}
\prd{x:X}\fib{f}{x}\to\fib{m}{x},
\end{equation*}
is a proposition. Recall from \cref{cor:prop_emb} that a map is an embedding if and only if its fibers are propositions.
Thus we see that the type $\prd{x:X}\fib{f}{x}\to\fib{m}{x}$ is a product of propositions, so it is a proposition by \cref{thm:trunc_pi}.
\end{proof}

\begin{defn}
Consider a commuting triangle
\begin{equation*}
\begin{tikzcd}[column sep=small]
A \arrow[rr,"i"] \arrow[dr,swap,"f"] & & B \arrow[dl,"m"] \\
& X
\end{tikzcd}
\end{equation*}
with $I:f\htpy m\circ i$, and where $m$ is an embedding\index{embedding}.
We say that $m$ has the \define{universal property of the image of $f$}\index{universal property!of the image|textit} if the map
\begin{equation*}
(i,I)^\ast : \mathrm{hom}_X(m,m')\to\mathrm{hom}_X(f,m')
\end{equation*}
defined by $(i,I)^\ast(h,H)\defeq (h\circ i,\ct{I}{(i\cdot H)})$,
is an equivalence for every embedding $m':B'\to X$. 
\end{defn}

\begin{rmk}
Since $\mathrm{hom}_X(f,m)$ is a proposition for every $f:A\to X$ and every embedding $m:B\to X$, it follows by \cref{ex:prop_equiv} that the universal property of the image of $f$ is equivalent to the property that the implication
\begin{equation*}
\mathrm{hom}_X(f,m')\to\mathrm{hom}_X(m,m')
\end{equation*}
holds for every embedding $m':B'\to X$. 
\end{rmk}

The homotopy image can be used in many important constructions. In this lecture we discuss two applications: the propositional truncation, and set quotients.

\section{The propositional truncation}
Note that embeddings into the unit type are just propositions. To see this, note that
\begin{align*}
\sm{A:\UU}{f:A\to\unit}\isemb(f)
& \eqvsym \sm{A:\UU}\isemb(\const_\ttt) \\
& \eqvsym \sm{A:\UU}\prd{x:\unit}\isprop(\fib{\const_\ttt}{x}) \\
& \eqvsym \sm{A:\UU}\isprop(\fib{\const_\ttt}{\ttt}) \\
& \eqvsym \sm{A:\UU}\isprop(A).
\end{align*}
Therefore, the universal property of the image of the map $A\to\unit$ is a proposition $P$ satisfying the universal property of the propositional truncation:

\begin{defn}
Let $A$ be a type, and let $P$ be a proposition that comes equipped with a map $\eta:A\to P$. We say that $\eta:A\to P$ satisfies the \define{universal property of propositional truncation}\index{univeral property!of propositional truncation|textit} if for every proposition $Q$, the precomposition map
\begin{equation*}
\blank\circ\eta:(P\to Q)\to (A\to Q)
\end{equation*}
is an equivalence.
\end{defn}

\begin{thm}
Consider a commuting triangle
\begin{equation*}
\begin{tikzcd}[column sep=small]
A \arrow[rr,"i"] \arrow[dr,swap,"f"] & & B \arrow[dl,"m"] \\
& X
\end{tikzcd}
\end{equation*}
with $I:f\htpy m\circ i$, and where $m$ is an embedding. The following are equivalent:
\begin{enumerate}
\item $m$ satisfies the universal property of the image of $f$.
\item for each $x:X$, the proposition $\fib{m}{x}$ satisfies the universal property of the propositional truncation of $\fib{f}{x}$.
\end{enumerate}
\end{thm}


Note that, given a family of propositions $P$ over a type $A$, the type $\sm{a:A}P(a)$ isn't necessarily a proposition. Instead, we think of $\sm{a:A}P(a)$ of the \emph{subtype} of $A$ containing the terms that satisfies $P$. Using the propositional truncation we can assert that there \emph{exists} a term in $A$ that satisfies $P$ without requiring one to construct it. 

\begin{defn}
Let $P:A\to \prop$ be a family of propositions over a type $A$. Then we define
\begin{equation*}
\exists_{(a:A)}P(a)\defeq \brck{\sm{a:A}P(a)}.
\end{equation*}
\end{defn}

Similarly, we can define the disjuction of two propositions $P$ and $Q$ to be the \emph{proposition} $\brck{P+Q}$, which clearly satisfies the universal property of disjunction\footnote{Alternatively, we have shown in \cref{ex:join_propositions} that the join $\join{P}{Q}$ also is a proposition that satisfies the universal property of disjunction.}. In \cref{table:logic} we give an overview of the logical connectives on propositions.

\begin{table}
\caption{\label{table:logic}Logic in type theory}
\begin{center}
\begin{tabular}{ll}
\toprule
\emph{Logical connective} & \emph{Interpretation in HoTT} \\
\midrule
$\top$ & $\unit$ \\
$\bot$ & $\emptyt$ \\
$P\land Q$ & $P\times Q$ \\
$P\lor Q$ & $\brck{P+Q}$ \\
$P\to Q$ & $P\to Q$ \\
$P\leftrightarrow Q$ & $\eqv{P}{Q}$ \\
$\neg P$ & $P\to\emptyt$ \\
$\forall x.P(x)$ & $\prd{x:A}P(x)$ \\
$\exists x.P(x)$ & $\brck{\sm{x:A}P(x)}$ \\
$\exists! x.P(x)$ & $\iscontr(\sm{x:A}P(x))$ \\
\bottomrule
\end{tabular}
\end{center}
\end{table}

\section{Constructing the propositional truncation}
Although technically it is not necessary to construct the propositional truncation before constructing the image of a map, we do so because the construction is simpler in this special case, and yet contains most of the essential ideas.

\begin{lem}\label{lem:extend_join_prop}
Suppose $f:A\to P$, where $A$ is any type, and $P$ is a proposition.
Then the map
\begin{equation*}
(\join{A}{B}\to P)\to (B\to P)
\end{equation*}
given by $h\mapsto h\circ \inr$ is an equivalence, for any type $B$.
\end{lem}

\begin{proof}
Since both types are propositions by \cref{thm:trunc_pi} it suffices to construct a map
\begin{equation*}
(B\to P)\to (\join{A}{B}\to P).
\end{equation*}
Let $g:B\to P$. Then the square
\begin{equation*}
\begin{tikzcd}
A\times B \arrow[r,"\proj 2"] \arrow[d,swap,"\proj 1"] & B \arrow[d,"g"] \\
A \arrow[r,swap,"f"] & P
\end{tikzcd}
\end{equation*}
commutes since $P$ is a proposition. Therefore we obtain a map $\join{A}{B}\to P$ by the universal property of the join.
\end{proof}

The idea of the construction of the propositional truncation is that if we are given a map $f:A\to P$, where $P$ is a proposition, then it extends uniquely along $\inr:A\to \join{A}{A}$ to a map $\join{A}{A}\to P$. This extension again extends uniquely along $\inr:\join{A}{A}\to \join{A}{(\join{A}{A})}$ to a map $\join{A}{(\join{A}{A})}\to P$ and so on, resulting in a diagram of the form
\begin{equation*}
\begin{tikzcd}
A \arrow[dr] \arrow[r,"\inr"] & \join{A}{A} \arrow[d,densely dotted] \arrow[r,"\inr"] & \join{A}{(\join{A}{A})} \arrow[dl,densely dotted] \arrow[r,"\inr"] & \cdots \arrow[dll,densely dotted,bend left=10] \\
& P
\end{tikzcd}
\end{equation*}

\begin{defn}
The \define{join powers} $A^{\ast n}$ of a type $X$ are defined by
\begin{align*}
A^{\ast 0} & \defeq \emptyt \\
A^{\ast 1} & \defeq A \\
A^{\ast (n+1)} & \defeq \join{A}{A^{\ast n}}.
\end{align*}
Furthermore, we define $A^{\ast\infty}$ to be the sequential colimit of the type sequence
\begin{equation*}
\begin{tikzcd}
A^{\ast 0} \arrow[r] & A^{\ast 1} \arrow[r,"\inr"] & A^{\ast 2} \arrow[r,"\inr"] & \cdots.
\end{tikzcd}
\end{equation*}
\end{defn}

Our goal is now to show that $A^{\ast\infty}$ is a proposition and satisfies the universal property of the propositional truncation.

\begin{lem}
Consider a type sequence
\begin{equation*}
\begin{tikzcd}
A_0 \arrow[r,"f_0"] & A_1 \arrow[r,"f_1"] & A_2 \arrow[r,"f_2"] & \cdots
\end{tikzcd}
\end{equation*}
with sequential colimit $A_\infty$, and let $P$ be a proposition. Then the map
\begin{equation*}
\seqin^\ast: (A_\infty\to P)\to \Big(\prd{n:\N}A_n\to P\Big)
\end{equation*}
given by $h\mapsto \lam{n}(h\circ \seqin_n)$ is an equivalence. 
\end{lem}

\begin{proof}
By the universal property of sequential colimits established in \cref{thm:sequential_up} we obtain that $\coconemap$ is an equivalence. Note that we have a commuting triangle
\begin{equation*}
\begin{tikzcd}
& P^{A_\infty} \arrow[dl,swap,"\coconemap"] \arrow[dr,"\seqin^\ast"] \\
\cocone(P) \arrow[rr,swap,"\proj 1"] & & \Big(\prd{n:\N}A_n\to P\Big).
\end{tikzcd}
\end{equation*}
Note that for any $g:\prd{n:\N}A_n\to P$ the type 
\begin{equation*}
\prd{n:\N} g_n\htpy g_{n+1}\circ f_n
\end{equation*}
is a product of contractible types, since $P$ is a proposition. Therefore it is contractible by \cref{thm:funext_wkfunext}, and it follows by \cref{ex:proj_fiber} that the projection is an equivalence. We conclude by the 3-for-2 property of equivalences (\cref{ex:3_for_2}) that $\seqin^\ast$ is an equivalence.
\end{proof}

\begin{lem}\label{lem:infjp_up}
Let $A$ be a type, and let $P$ be a proposition. Then the function
\begin{equation*}
\blank\circ \seqin_0: (A^{\ast\infty}\to P)\to (A\to P)
\end{equation*}
is an equivalence. 
\end{lem}

\begin{proof}
We have the commuting triangle
\begin{equation*}
\begin{tikzcd}
&[-3em] P^{A^{\ast\infty}} \arrow[dl,swap,"\seqin^\ast"] \arrow[dr,"\blank\circ\seqin_0"] \\
\Big(\prd{n:\N}A^{\ast n} \to P\Big) \arrow[rr,swap,"\lam{h}h_0"] & & P^A.
\end{tikzcd}
\end{equation*}
Therefore it suffices to show that the bottom map is an equivalence. Since this is a map between propositions, it suffices to construct a map in the converse direction. Let $f:A\to P$. We will construct a term of type
\begin{equation*}
\prd{n:\N}A^{\ast n} \to P
\end{equation*}
by induction on $n:\N$. The base case is trivial. Given a map $g:A^{\ast n}\to P$, we obtain a map $g:A^{\ast(n+1)}\to P$ by \cref{lem:extend_join_prop}.
\end{proof}

\begin{lem}\label{lem:isprop_infjp}
The type $A^{\ast\infty}$ is a proposition for any type $A$.
\end{lem}

\begin{proof}
By \cref{cor:contr_prop} it suffices to show that $A^{\ast\infty}\to \iscontr(A^{\ast\infty})$, and by \cref{lem:infjp_up} it suffices to show that
\begin{equation*}
A\to \iscontr(A^{\ast\infty}),
\end{equation*}
because $\iscontr(A^{\ast\infty})$ is a proposition by \cref{ex:isprop_istrunc}. 

Let $x:A$. To see that $A^{\ast\infty}$ is contractible it suffices by \cref{ex:seqcolim_contr} to show that $\inr:A^{\ast n}\to A^{\ast(n+1)}$ is homotopic to the constant function $\const_{\inl(x)}$. However, we get a homotopy $\const_{\inl(x)}\htpy \inr$ immediately from the path constructor $\glue$.  
\end{proof}

\begin{thm}
For any type $A$ there is a type $\brck{A}$ that comes equipped with a map $\eta:A\to \brck{A}$, and satisfies the universal property of propositional truncation.
\end{thm}

\begin{proof}
Let $A$ be a type. Then we define $\brck{A}\defeq A^{\ast\infty}$, and we define $\eta\defeq \seqin_0:A\to A^{\ast\infty}$. Then $\brck{A}$ is a proposition by \cref{lem:isprop_infjp}, and $\eta:A\to \brck{A}$ satisfies the universal property of propositional truncation by \cref{lem:infjp_up}.
\end{proof}

\section{The construction of the image of a map}
The image of a map $f:A\to X$ can be defined using the propositional truncation:
\begin{defn}
For any map $f:A\to X$ we define the \define{image}\index{image|textbf} of $f$ to be the type
\begin{equation*}
\im(f) \defeq \sm{x:X}\brck{\fib{f}{x}}
\end{equation*}
and we define the \define{image inclusion} to be the projection $\proj 1 :\im(f)\to X$. 
\end{defn}
However, the construction of the fiberwise join in \cref{ex:fib_join} suggests that we can also define the image of $f$ as the infinite join power $f^{\ast\infty}$, where we repeatedly take the fiberwise join of $f$ with itself. The reasons for defining the image in this way are twofold: we will be able to use this construction to show that the set-quotients of a small type are small, and second, we many interesting types appear in this construction.

\begin{lem}
Consider a map $f:A\to X$, an embedding $m:U\to X$, and $h:\mathrm{hom}_X(f,m)$. Then the map
\begin{equation*}
\mathrm{hom}_X(\join{f}{g},m)\to \mathrm{hom}_X(g,m)
\end{equation*}
is an equivalence for any $g:B\to X$.
\end{lem}

\begin{proof}
Note that both types are propositions, so any equivalence can be used to prove the claim. Thus, we simply calculate
\begin{align*}
\mathrm{hom}_X(\join{f}{g},m) & \eqvsym \prd{x:X}\fib{\join{f}{g}}{x}\to \fib{m}{x} \\
& \eqvsym \prd{x:X}\join{\fib{f}{x}}{\fib{g}{x}}\to\fib{m}{x} \\
& \eqvsym \prd{x:X}\fib{g}{x}\to\fib{m}{x} \\
& \eqvsym \mathrm{hom}_X(g,m).
\end{align*}
The first equivalence holds by \cref{ex:triangle_fib}; the second equivalence holds by \cref{ex:fib_join}, also using \cref{ex:equiv_precomp,lem:postcomp_equiv} where we established that that pre- and postcomposing by an equivalence is an equivalence; the third equivalence holds by \cref{lem:extend_join_prop,lem:postcomp_equiv}; the last equivalence again holds by \cref{ex:triangle_fib}.
\end{proof}

For the construction of the image of $f:A\to X$ we observe that if we are given an embedding $m:U\to X$ and a map $(i,I):\mathrm{hom}_X(f,m)$, then $(i,I)$ extends uniquely along $\inr:A\to \join[X]{A}{A}$ to a map $\mathrm{hom}_X(\join{f}{f},m)$. This extension again extends uniquely along $\inr:\join[X]{A}{A}\to \join[X]{A}{(\join[X]{A}{A})}$ to a map $\mathrm{hom}_X(\join{f}{(\join{f}{f})},m)$ and so on, resulting in a diagram of the form
\begin{equation*}
\begin{tikzcd}
A \arrow[dr] \arrow[r,"\inr"] & \join[X]{A}{A} \arrow[d,densely dotted] \arrow[r,"\inr"] & \join[X]{A}{(\join[X]{A}{A})} \arrow[dl,densely dotted] \arrow[r,"\inr"] & \cdots \arrow[dll,densely dotted,bend left=10] \\
& U
\end{tikzcd}
\end{equation*}

\begin{defn}
Suppose $f:A\to X$ is a map. Then we define the \define{fiberwise join powers} 
\begin{equation*}
f^{\ast n}:A_X^{\ast n} X.
\end{equation*}
\end{defn}

\begin{constr}
Note that the operation $(B,g)\mapsto (\join[X]{A}{B},\join{f}{g})$ defines an endomorphism on the type
\begin{equation*}
\sm{B:\UU}B\to X.
\end{equation*}
We also have $(\emptyt,\ind{\emptyt})$ and $(A,f)$ of this type. For $n\geq 1$ we define
\begin{align*}
A_X^{\ast (n+1)} & \defeq \join[X]{A}{A_X^{\ast n}} \\
f^{\ast (n+1)} & \defeq \join{f}{f^{\ast n}}.\qedhere
\end{align*}
\end{constr}

\begin{defn}
We define $A_X^{\ast\infty}$ to be the sequential colimit of the type sequence
\begin{equation*}
\begin{tikzcd}
A_X^{\ast 0} \arrow[r] & A_X^{\ast 1} \arrow[r,"\inr"] & A_X^{\ast 2} \arrow[r,"\inr"] & \cdots.
\end{tikzcd}
\end{equation*}
Since we have a cocone
\begin{equation*}
\begin{tikzcd}
A_X^{\ast 0} \arrow[r] \arrow[dr,swap,"f^{\ast 0}" near start] & A_X^{\ast 1} \arrow[r,"\inr"] \arrow[d,swap,"f^{\ast 1}" near start] & A_X^{\ast 2} \arrow[r,"\inr"] \arrow[dl,swap,"f^{\ast 2}" xshift=1ex] & \cdots \arrow[dll,bend left=10] \\
& X
\end{tikzcd}
\end{equation*}
we also obtain a map $f^{\ast\infty}:A_X^{\ast\infty}\to X$ by the universal property of $A_X^{\ast\infty}$. 
\end{defn}

\begin{lem}\label{lem:finfjp_up}
Let $f:A\to X$ be a map, and let $m:U\to X$ be an embedding. Then the function
\begin{equation*}
\blank\circ \seqin_0: \mathrm{hom}_X(f^{\ast\infty},m)\to \mathrm{hom}_X(f,m)
\end{equation*}
is an equivalence. 
\end{lem}

\begin{thm}\label{lem:isprop_infjp}
For any map $f:A\to X$, the map $f^{\ast\infty}:A_X^{\ast\infty}\to X$ is an embedding that satisfies the universal property of the image inclusion of $f$.
\end{thm}

\section{Surjective maps}

Another application of the propositional truncation is the notion of surjective map.

\begin{defn}
A map $f:A\to B$ is said to be \define{surjective} if there is a term of type
\begin{equation*}
\issurj(f)\defeq \prd{y:B}\brck{\fib{f}{b}}.
\end{equation*}
\end{defn}

\begin{eg}
Any equivalence is a surjective map, and so is any map that has a section (those are sometimes called \define{split epimorphisms}). Other examples include the base point inclusion $\unit\to\sphere{n}$ for any $n\geq 1$. 
\end{eg}

\begin{thm}\label{thm:surjective}
Consider a commuting triangle
\begin{equation*}
\begin{tikzcd}[column sep=tiny]
A \arrow[rr,"q"] \arrow[dr,swap,"f"] & & B \arrow[dl,"m"] \\
& X
\end{tikzcd}
\end{equation*}
in which $m$ is an embedding. Then $m$ satisfies the universal property of the image of $f$ if and only if $i:A\to B$ is surjective.
\end{thm}

\begin{thm}
Let $f:A\to B$ be a map. The following are equivalent:
\begin{enumerate}
\item $f$ is an equivalence.
\item $f$ is both surjective and an embedding.
\end{enumerate}
\end{thm}

\begin{exercises}
\item Show that
\begin{equation*}
\eqv{\brck{A}}{\prd{P:\prop}(A\to P)\to P}
\end{equation*}
for any type $A:\UU$. This is called the \define{impredicative encoding} of the propositional truncation.
\item For any $B:A\to\UU$, construct an equivalence
\begin{equation*}
\eqv{\Big(\exists_{(a:A)}\brck{B(a)}\Big)}{\brck{\sm{a:A}B(a)}}
\end{equation*}
%\item \label{also}(Mart\'in Escard\'o) For any two propositions $P$ and $Q$, define
%\begin{equation*}
%P\boxplus Q \defeq ((P\to Q)\to Q)\times ((Q\to P)\to P).
%\end{equation*}
%\begin{subexenum}
%\item Show that $P\lor Q\to P\boxplus Q$ and $P\boxplus Q\to\neg(\neg P\land \neg Q)$.
%\end{subexenum}
%\item \label{ex:brck_comp} Formulate the computation rule corresponding to the path constructor $\mu$. That is, compute the type of $\apd{\rec{\brck{\blank}}(f,g)}{\mu(x,y)}$, and find a canonical element in it.
\item Let
\begin{tikzcd}
P_0 \arrow[r] & P_1 \arrow[r] & P_2 \arrow[r] & \cdots
\end{tikzcd}
be a sequence of propositions. Show that
\begin{equation*}
\eqv{\colim_n(P_n)}{\exists_{(n:\N)} P_n}.
\end{equation*}
\item Show that the relation $x,y\mapsto\brck{x=y}$ is an equivalence relation, on any type.
\item Let $f:A\to X$ be a map. Construct an equivalence
\begin{equation*}
\eqv{\Big(\sm{y:\mathsf{join\usc{}power}_X(n,A)}f(x)=f^{\ast n}(y)\Big)}{\Big(\sm{y:A}f(x)=f(y)\Big)^{\ast n}}
\end{equation*}
for any $x:A$.
\item Let $f:A\to B$ be a map. Show that the following are equivalent:
\begin{enumerate}
\item The commuting square
\begin{equation*}
\begin{tikzcd}
A \arrow[d,swap,"f"] \arrow[r] & \brck{A} \arrow[d,"\brck{f}"] \\
B \arrow[r] & \brck{B}.
\end{tikzcd}
\end{equation*}
is a pullback square.
\item There is a term of type $A\to\isequiv(f)$.
\item The commuting square
\begin{equation*}
\begin{tikzcd}
A\times A \arrow[r,"f\times f"] \arrow[d,swap,"\proj 1"] & B \times B \arrow[d,"\proj 1"] \\
A \arrow[r,swap,"f"] & B
\end{tikzcd}
\end{equation*}
is a pullback square. 
\end{enumerate}
\item Consider a pullback square
\begin{equation*}
\begin{tikzcd}
A' \arrow[d,swap,"{f'}"] \arrow[r,"p"] & A \arrow[d,"f"] \\
B' \arrow[r,swap,"q"] & B,
\end{tikzcd}
\end{equation*}
in which $q:B'\to B$ is surjective. Show that if $f':A'\to B'$ is an embedding, then so is $f:A\to B$.
\item Show that a type $A$ is a proposition if and only if the map $\inl:A\to \join{A}{A}$ is an equivalence.
\item Show that $\inl:\brck{A}\to \join{\brck{A}}{A}$ is an equivalence for any type $A$.
\item Consider a family of diagrams of the form
  \begin{equation*}
    \begin{tikzcd}
      A_i \arrow[r] \arrow[d,swap,"{f_i}"] &
        C \arrow[r] \arrow[d,"g"] & X \arrow[d,"h"] \\
      B_i \arrow[r] & D \arrow[r] & Y 
    \end{tikzcd}
  \end{equation*}
  indexed by $i:I$, and assume that the induced map
  \begin{equation*}
    \Big(\sm{i:I}B_i\Big)\to D
  \end{equation*}
  is surjective. Show that the following are equivalent:
  \begin{enumerate}
  \item For each $i:I$ the outer rectangle is a pullback square.
  \item The right square is a pullback square.
  \end{enumerate}
  Hint: By \cref{thm:descent-Sigma} it suffices to prove this equivalence for a sigle diagram of the form
    \begin{equation*}
    \begin{tikzcd}
      A \arrow[r] \arrow[d,swap,"{f}"] &
        C \arrow[r] \arrow[d,swap,"g"] & X \arrow[d,"h"] \\
      B \arrow[r] & D \arrow[r] & Y 
    \end{tikzcd}
    \end{equation*}
  where the map $B \to D$ is assumed to be surjective.
\end{exercises}
