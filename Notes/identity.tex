\chapter{Identity types}\label{chap:identity}
From the perspective of types as proof-relevant propositions, how should we think of \emph{equality} in type theory? Given a type $A$, and two terms $x,y:A$, the equality $\id{x}{y}$ should again be a type. Indeed, we want to \emph{use} type theory to prove equalities. \emph{Dependent} type theory provides us with a convenient setting for this: the equality type $\id{x}{y}$ is dependent on $x,y:A$. 

Then, if $\id{x}{y}$ is to be a type, how should we think of the terms of $\id{x}{y}$. A term $p:\id{x}{y}$ witnesses that $x$ and $y$ are equal terms of type $A$. In other words $p:\id{x}{y}$ is an \emph{identification} of $x$ and $y$. In a proof-relevant world, there might be many terms of type $\id{x}{y}$. I.e., there might be many identifications of $x$ and $y$. And, since $\id{x}{y}$ is itself a type, we can form the type $\id{p}{q}$ for any two identifications $p,q:\id{x}{y}$. That is, since $\id{x}{y}$ is a type, we may also use the type theory to prove things \emph{about} identifications (for instance, that two given such identifications can themselves be identified), and we may use the type theory to perform constructions with them. As we will see shortly, we can give every type a groupoid-like structure.

Clearly, the equality type should not just be any type dependent on $x,y:A$. Then how do we form the equality type, and what ways are there to use identifications in constructions in type theory? The answer to both these questions is that we will form the identity type as an \emph{inductive} type, generated by just a reflexivity term providing an identification of $x$ to itself. The induction principle then provides us with a way of performing constructions with identifications, such as concatenating them, inverting them, and so on. Thus, the identity type is equipped with a reflexivity term, and further possesses the structure that are generated by its induction principle and by the type theory. This inductive construction of the identity type is elegant, beautifully simple, but far from trivial!

The situation where two terms can be identified in possibly more than one way is analogous to the situation in \emph{homotopy theory}, where two points of a space can be connected by possibly more than one \emph{path}. Indeed, for any two points $x,y$ in a space, there is a \emph{space of paths} from $x$ to $y$. Moreover, between any two paths from $x$ to $y$ there is a space of \emph{homotopies} between them, and so on. This leads to the homotopy interpretation of type theory, outlined in \cref{tab:homotopy_interpretation}. The connection between homotopy theory and type theory been made precise by the construction of homotopical models of type theory, and it has led to the fruitful research area of \emph{synthetic homotopy theory}, the subfield of \emph{homotopy type theory} that is the topic of this course.

\begin{table}
\begin{center}
\caption{\label{tab:homotopy_interpretation}The homotopy interpretation\index{Homotopy interpretation}}
\begin{tabular}{ll}
\toprule
\emph{Type theory} &  \emph{Homotopy theory} \\
\midrule
Types  & Spaces \\
Dependent types & Fibrations \\
Terms & Points \\
Dependent pair type & Total space \\
Identity type & Path fibration\\
\bottomrule
\end{tabular}
\end{center}
\end{table}

\section{The inductive definition of identity types}

Let $a:A$ be a term in context $\Gamma$. The \define{identity type}\index{identity type|textbf} of $A$ at $a:A$ is the inductive type family 
\begin{equation*}
\Gamma,x:A\vdash a =_A x~\mathrm{type}
\end{equation*}
with constructor
\begin{equation*}
\Gamma\vdash \refl{a} : a=_A a.
\end{equation*}
The induction principle postulates that for any type family
\begin{equation*}
\Gamma,x:A,p: a=_A x\vdash P(x,p)~\mathrm{type}
\end{equation*}
%we can derive
%\begin{prooftree}
%\AxiomC{$\Gamma,\Delta[a/x,\refl{a}/p]\vdash \alpha : P(a,\refl{a})$}
%\UnaryInfC{$\Gamma,x:A,p:\idtypevar{A}(a,x),\Delta\vdash \mathsf{path\usc{}ind}_a(\alpha,x,p):P(x,p)$}
%\end{prooftree}
there is a term
\begin{equation*}
\mathsf{path\usc{}ind}_a : P(a,\refl{a})\to \prd{x:A}{p:a=_A x}P(x,p)
\end{equation*}
in context $\Gamma$, satisfying the computation rule
\begin{equation*}
\mathsf{path\usc{}ind}_a(y,a,\refl{a})\jdeq y.
\end{equation*}
A term of type $a=_A x$ is also called an \define{identification}\index{identification|textbf} of $a$ with $x$, and sometimes it is called a \define{path}\index{path|textbf} from $a$ to $x$.
The induction principle for identity types is sometimes called \define{identification elimination}\index{identification elimination|textbf} or \define{path induction}\index{path induction|textbf}. We also write $\idtypevar{A}$\index{Id A@{$\idtypevar{A}$}|textbf} for the identity type on $A$. 

We also assume that the universe $\UU$ is closed under identity types, i.e., that there is a map
\begin{equation*}
\check{\mathsf{Id}}:\prd{A:\UU}\mathrm{El}(A)\to\mathrm{El}(A)\to\UU
\end{equation*}
satisfying
\begin{equation*}
\mathrm{El}(\check{\mathsf{Id}}(A,a,x))\jdeq a=_{\mathrm{El}(A)} x.
\end{equation*}

In the following lemma we show that the identity type on $A$ is contained in any reflexive relation on $A$.

\begin{lem}
Let $\Gamma,x:A,y:A\vdash R(x,y)~\mathrm{type}$\index{reflexive relation|textit}\index{relation!reflexive}, and suppose that $R$ is reflexive in the sense that there is a term
\begin{equation*}
\rho:\prd{x:A}R(x,x)
\end{equation*}
Then there is a term of type
\begin{equation*}
\prd{y:A} (x=_A y)\to R(x,y)
\end{equation*}
in context $\Gamma,x:A$.
\end{lem}

\begin{constr}
By weakening the reflexive relation $R$ we obtain
\begin{equation*}
\Gamma,x:A,y:A,\alpha:x=_A y\vdash R(x,y)~\mathrm{type},
\end{equation*}
on which the induction principle is applicable.
Thus we see that by the induction principle for identity types we have a term
\begin{equation*}
\mathsf{path\usc{}ind}_x : R(x,x)\to \prd{y:A}(x=_A y)\to R(x,y)
\end{equation*}
so it suffices to construct a term of type $R(x,x)$, which we have by reflexivity of $R$.
\end{constr}

\section{The groupoid structure of types}\label{sec:groupoid}
We show that identifications can be \emph{concatenated} and \emph{inverted}, which corresponds to the transitivity and symmetry of the identity type. 

Furthermore, we observe that we can iteratively take identity types, i.e., we can take identity types of identity types, 
\begin{equation*}
p =_{(x=_Ay)} q,
\end{equation*}
and so on. In other words, for any two identifications $p,q:x=_A y$, there is a type of identifications of $p$ with $y$. One way to think about this is that the identifications $p,q:x=_A y$ are paths in the type (space) $A$, and an identification of $p$ with $q$ is a \emph{higher path} from $p$ to $q$, i.e., a \emph{homotopy}.

Using the observation that identity types can be iterated we show that concatenation is \emph{associative}, satisfies the left and right \emph{unit laws}, and satisfies the left and right \emph{inverse laws}. These are the \define{groupoid operations} on the identity type.

\begin{defn}\label{defn:id_concat}
Let $A$ be a type. We define the \define{concatenation}\index{concatenation!for identifications}\index{concat@{$\mathsf{concat}$}} operation
\begin{equation*}
\mathsf{concat} : \prd{x,y,z:A} (\id{x}{y})\to(\id{y}{z})\to (\id{x}{z}).
\end{equation*}
We will write $\ct{p}{q}$ for $\mathsf{concat}(p,q)$.
\end{defn}

\begin{constr}
We construct the concatenation operation by path induction. It suffices to construct
\begin{equation*}
\mathsf{concat}(\refl{x}):\prd{z:A} (x=z)\to(x=z).
\end{equation*}
Here we take $\mathsf{concat}(\refl{x})_z \jdeq \idfunc[(x=z)]$. 
Explicitly, the term we have constructed is
\begin{equation*}
\lam{x}\mathsf{path\usc{}ind}_x(\lam{z}\idfunc[(\id{x}{z})]):\prd{x,y:A} (x=y)\to \prd{z:A} (y=z)\to (x=z).
\end{equation*}
To obtain a term of the asserted type we need to swap the order of the arguments $p:x=y$ and $z:A$, using \autoref{ex:swap}.
\end{constr}

\begin{defn}\label{defn:id_inv}
Let $A$ be a type. We define the \define{inverse operation}\index{inverse operation!for identifications|textbf}\index{inv@{$\mathsf{inv}$}|textbf}
\begin{equation*}
\mathsf{inv}:\prd{x,y:A} (x=y)\to (y=x).
\end{equation*}
Most of the time we will write $p^{-1}$ for $\mathsf{inv}(p)$.
\end{defn}

\begin{constr}
We construct the inverse operation by path induction. It suffices to construct
\begin{equation*}
\mathsf{inv}(\refl{x}): x=x,
\end{equation*}
for any $x:A$. Here we take $\mathsf{inv}(\refl{x})\defeq \refl{x}$.
\end{constr}

\begin{defn}\label{defn:id_assoc}
Let $A$ be a type. We define the \define{associativity operation}\index{associativity operation!for identifications|textbf}, which assigns to each $p:x=y$, $q:y=z$, and $r:z=w$ the \define{associator}
\begin{equation*}
\mathsf{assoc}(p,q,r) : \ct{(\ct{p}{q})}{r}=\ct{p}{(\ct{q}{r})}.
\end{equation*}
\end{defn}

\begin{constr}
By identification elimination it suffices to show that
\begin{equation*}
\prd{z:A}{q:x=z}{z':A}{r:z=w} \ct{(\ct{\refl{x}}{q})}{r}= \ct{\refl{x}}{(\ct{q}{r})}.
\end{equation*}
Let $q:x=z$ and $r:z=w$. Note that by the computation rule $\ct{\refl{x}}{q}\jdeq q$, so $\ct{(\ct{\refl{x}}{q})}{r}\jdeq \ct{q}{r}$. Similarly we have $\ct{\refl{x}}{(\ct{q}{r})}\jdeq \ct{q}{r}$. Therefore we can simply take $\refl{\ct{q}{r}}$.
\end{constr}

\begin{defn}\label{defn:id_unit}
Let $A$ be a type. We define the left and right \define{unit law operations}\index{unit law operations!for identifications|textbf}, which assigns to each $p:x=y$ the terms\index{left unit@{$\mathsf{left\usc{}unit}$}|textbf}\index{right unit@{$\mathsf{right\usc{}unit}$}|textbf}
\begin{align*}
\mathsf{left\usc{}unit}(p) & : \ct{\refl{x}}{p}=p \\
\mathsf{right\usc{}unit}(p) & : \ct{p}{\refl{y}}=p,
\end{align*}
respectively.
\end{defn}

\begin{constr}
By identification elimination it suffices to construct
\begin{align*}
\mathsf{left\usc{}unit}(\refl{x}) & : \ct{\refl{x}}{\refl{x}} = \refl{x} \\
\mathsf{right\usc{}unit}(\refl{x}) & : \ct{\refl{x}}{\refl{x}} = \refl{x}.
\end{align*}
In both cases we take $\refl{\refl{x}}$.
\end{constr}

\begin{defn}\label{defn:id_invlaw}
Let $A$ be a type. We define left and right \define{inverse law operations}\index{inverse law operations!for identifications|textbf}\index{left inv@{$\mathsf{left\usc{}inv}$}|textbf}\index{right inv@{$\mathsf{right\usc{}inv}$}|textbf}
\begin{align*}
\mathsf{left\usc{}inv}(p) & : \ct{p^{-1}}{p} = \refl{y} \\
\mathsf{right\usc{}inv}(p) & : \ct{p}{p^{-1}} = \refl{x}.
\end{align*}
\end{defn}

\begin{constr}
By identification elimination it suffices to construct
\begin{align*}
\mathsf{left\usc{}inv}(\refl{x}) & : \ct{\refl{x}^{-1}}{\refl{x}} = \refl{x} \\
\mathsf{right\usc{}inv}(\refl{x}) & : \ct{\refl{x}}{\refl{x}^{-1}} = \refl{x}.
\end{align*}
Using the computation rules we see that
\begin{equation*}
\ct{\refl{x}^{-1}}{\refl{x}}\jdeq \ct{\refl{x}}{\refl{x}}\jdeq\refl{x},
\end{equation*}
so we define $\mathsf{left\usc{}inv}(\refl{x})\defeq \refl{\refl{x}}$. Similarly it follows from the computation rules that
\begin{equation*}
\ct{\refl{x}}{\refl{x}^{-1}} \jdeq \refl{x}^{-1}\jdeq \refl{x}
\end{equation*}
so we again define $\mathsf{right\usc{}inv}(\refl{x})\defeq\refl{\refl{x}}$. 
\end{constr}

\section{The action on paths of functions}

Using the induction principle of the identity type we can show that every function preserves identifications.
In other words, every function sends identified terms to identified terms.
Note that this is a form of continuity for functions in type theory: if there is a path that identifies two points $x$ and $y$ of a type $A$, then there also is a path that identifies the values $f(x)$ and $f(y)$ in the codomain of $f$. 

\begin{defn}\label{defn:ap}
Let $f:A\to B$ be a map. We define the \define{action on paths}\index{action on paths}\index{function!action on paths} of $f$ as an operation\index{ap f@{$\apfunc{f}$}|textbf}
\begin{equation*}
\apfunc{f} : \prd*{x,y:A} (\id{x}{y})\to(\id{f(x)}{f(y)}).
\end{equation*}
Moreover, there are operations\index{ap idfun@{$\mathsf{ap\usc{}idfun}$}|textbf}\index{ap comp@{$\mathsf{ap\usc{}comp}$}|textbf}
\begin{align*}
\mathsf{ap\usc{}idfun}_A & : \prd*{x,y:A}{p:\id{x}{y}} \id{p}{\ap{\idfunc[A]}{p}} \\
\mathsf{ap\usc{}comp}(f,g) & : \prd*{x,y:A}{p:\id{x}{y}} \id{\ap{g}{\ap{f}{p}}}{\ap{g\circ f}{p}}.
\end{align*}
\end{defn}

\begin{constr}
First we define $\apfunc{f}$ by identity elimination, taking
\begin{equation*}
\apfunc{f}(\refl{x})\defeq \refl{f(x)}.
\end{equation*}
Next, we construct $\mathsf{ap\usc{}idfun}_A$ by identity elimination, taking
\begin{equation*}
\mathsf{ap\usc{}idfun}_A(\refl{x}) \defeq \refl{\refl{x}}.
\end{equation*}
Finally, we construct $\mathsf{ap\usc{}comp}(f,g)$ by identity elimination, taking
\begin{equation*}
\mathsf{ap\usc{}comp}(f,g,\refl{x}) \defeq \refl{g(f(x))}.\qedhere
\end{equation*}
\end{constr}

\begin{defn}\label{defn:ap-preserve}
Let $f:A\to B$ be a map. Then there are identifications
\begin{align*}
\mathsf{ap\usc{}refl}(f,x) & : \id{\ap{f}{\refl{x}}}{\refl{f}(x)} \\
\mathsf{ap\usc{}inv}(f,p) & : \id{\ap{f}{p^{-1}}}{\ap{f}{p}^{-1}} \\
\mathsf{ap\usc{}concat}(f,p,q) & : \id{\ap{f}{\ct{p}{q}}}{\ct{\ap{f}{p}}{\ap{f}{q}}}
\end{align*}
for every $p:\id{x}{y}$ and $q:\id{x}{y}$.
\end{defn}

\begin{constr}
To construct $\mathsf{ap\usc{}refl}(f,x)$ we simply observe that ${\ap{f}{\refl{x}}}\jdeq {\refl{f}(x)}$, so we take
\begin{equation*}
\mathsf{ap\usc{}refl}(f,x)\defeq\refl{\refl{f(x)}}.
\end{equation*}
We construct $\mathsf{ap\usc{}inv}(f,p)$ by identification elimination on $p$, taking
\begin{equation*}
\mathsf{ap\usc{}inv}(f,\refl{x}) \defeq \refl{\ap{f}{\refl{x}}}.
\end{equation*}
Finally we construct $\mathsf{ap\usc{}concat}(f,p,q)$ by identification elimination on $p$, taking
\begin{equation*}
\mathsf{ap\usc{}concat}(f,\refl{x},q)  \defeq \refl{\ap{f}{q}}.\qedhere
\end{equation*}
\end{constr}

\section{Transport}

Dependent types also come with an action on paths: the \emph{transport} functions.
Given an identification $p:\id{x}{y}$ in the base type $A$, we can transport any term $b:B(x)$ to the fiber $B(y)$.
The transport functions have many applications, which we will encounter throughout this course.

\begin{defn}
Let $A$ be a type, and let $B$ be a type family over $A$.
We will construct a \define{transport}\index{transport|textbf} operation\index{tr B@{$\mathsf{tr}_B$}|textbf}
\begin{equation*}
\mathsf{tr}_B:\prd*{x,y:A} (\id{x}{y})\to (B(x)\to B(y)).
\end{equation*}
\end{defn}

\begin{constr}
We construct $\mathsf{tr}_B(p)$ by induction on $p:x=_A y$, taking
\begin{equation*}
\mathsf{tr}_B(\refl{x}) \defeq \idfunc[B(x)].\qedhere
\end{equation*}
\end{constr}

Thus we see that type theory cannot distinguish between identified terms $x$ and $y$, because for any type family $B$ over $A$ one gets a term of $B(y)$ as soon as $B(x)$ has a term.

As an application of the transport function we construct the \emph{dependent} action on paths\index{dependent action on paths}\index{function!dependent action on paths} of a dependent function $f:\prd{x:A}B(x)$. Note that for such a dependent function $f$, and an identification $p:\id[A]{x}{y}$, it does not make sense to directly compare $f(x)$ and $f(y)$, since the type of $f(x)$ is $B(x)$ whereas the type of $f(y)$ is $B(y)$, which might not be exactly the same type. However, we can first \emph{transport} $f(x)$ along $p$, so that we obtain the term $\mathsf{tr}_B(p,f(x))$ which is of type $B(y)$. Now we can ask whether it is the case that $\mathsf{tr}_B(p,f(x))=f(y)$. The dependent action on paths of $f$ establishes this identification.

\begin{defn}\label{defn:apd}
Given a dependent function $f:\prd{a:A}B(a)$ and a path $p:\id{x}{y}$ in $A$, we construct a path\index{apd f@{$\apdfunc{f}$}|textbf}
\begin{equation*}
\apd{f}{p} : \id{\mathsf{tr}_B(p,f(x))}{f(y)}.
\end{equation*}
\end{defn}

\begin{constr}
The path $\apd{f}{p}$ is constructed by path induction on $p$. Thus, it suffices to construct a path
\begin{equation*}
\apd{f}{\refl{x}}:\id{\mathsf{tr}_B(\refl{x},f(x))}{f(x)}.
\end{equation*}
Since transporting along $\refl{x}$ is the identity function on $B(x)$, we simply take $\apd{f}{\refl{x}}\defeq\refl{f(x)}$. 
\end{constr}

%\begin{defn}\label{defn:path_lifting}
%Let $A$ be a type, and let $B:A\to\type$ be a type family over $A$.
%We will construct a \define{path lifting} operation
%\begin{equation*}
%\mathsf{lift}^B : \prd*{x,y:A}{p:\id{x}{y}}{b:B(x)} \id{\pairr{x,b}}{\pairr{y,\trans{p}{b}}}.
%\end{equation*}
%\end{defn}
%
%\autoref{defn:path_lifting} gives a way to lift a path $p:x=y$ in the base type of a type family, to a path in the $\Sigma$-type. This, along with the basic groupoid operations developed in \autoref{sec:groupoid}, inspired the \emph{homotopy interpretation} of type theory.

\begin{exercises}
\item \label{ex:trans_concat}Let $B$ be a family over a type $A$. Construct for any two identifications $p:x=_A y$ and $q:y=_A z$, and any $b:B(x)$ an identification
\begin{equation*}
\mathsf{tr}_B(q,\mathsf{tr}_B(p,b))=\mathsf{tr}_B(\ct{p}{q},b).
\end{equation*}
\item \label{ex:inv_assoc}Let $p:\id{x}{y}$ and $q:\id{y}{z}$. Construct an identification
\begin{align*}
\mathsf{inv\usc{}assoc}(p,q):\id{(\ct{p}{q})^{-1}}{\ct{q^{-1}}{p^{-1}}}.
\end{align*}
\item \label{ex:trans_triv}Consider two types $A$ and $B$, and let $p:x=y$ in $A$, and $b:B$. 
\begin{subexenum}
\item Construct an identification
\begin{align*}
\mathsf{tr\usc{}triv}(p,b):\mathsf{tr}_{W_A(B)}(p,b)=b
\end{align*}
where $W_A(B)$ is the family $B$ weakened by $A$.
\item Construct for any $f:A\to B$, an identification 
\begin{equation*}
\apd{f}{p}=\ct{\mathsf{tr\usc{}triv}(p,f(x))}{\mathsf{ap}_f(p)},
\end{equation*}
witnessing that the triangle
\begin{equation*}
\begin{tikzcd}[trim right=(a),column sep=0em]
\mathsf{tr}_{W_A(B)}(p,f(x)) \arrow[dr,equals,swap,"\apd{f}{p}"] \arrow[rr,equals,"{\mathsf{tr\usc{}triv}(p,f(x))}"] & & |[alias=a,right]|f(x) \arrow[dl,equals,"\ap{f}{p}"] \\
& f(y) & \phantom{\mathsf{tr}_{W_A(B)}(p,f(x))}
\end{tikzcd}
\end{equation*}
commutes.
\end{subexenum}
%\item In this exercise we show that the action on paths of a function preserves the groupoid-structure of a type.
%\begin{subexenum}
%\item Construct an identification
%\begin{equation*}
%\mathsf{ap.assoc}(f,p,q,r)
%\end{equation*}
%witnessing that the diagram
%\begin{equation*}
%\begin{tikzcd}[column sep=large]
%\ap{f}{\ct{(\ct{p}{q})}{r}} \arrow[r,equals,"\ap{\apfunc{f}}{\mathsf{assoc}(p,q,r)}"] \arrow[d,swap,equals,"{\mathsf{ap.ct}(f,%\ct{p}{q},r)}"] & \ap{f}{\ct{p}{(\ct{q}{r})}} \arrow[d,equals,"{\mathsf{ap.ct}(f,p,\ct{q}{r})}"] \\ 
%\ct{\ap{f}{\ct{p}{q}}}{\ap{f}{r}} \arrow[dd,equals,near start,"{\mathsf{whisk\usc{}r}(\mathsf{ap.ct}(f,p,q),\ap{f}{r})}"]   & %\ct{\ap{f}{p}}{\ap{f}{\ct{q}{r}}} \arrow[dd,equals,swap,near end,"{\mathsf{whisk\usc{}l}(\ap{f}{p},\mathsf{ap.ct}(f,q,r))}"]  %\\
%\\
%\ct{(\ct{\ap{f}{p}}{\ap{f}{q}})}{\ap{f}{r}} \arrow[r,equals,swap,"{\mathsf{assoc}(\ap{f}{p},\ap{f}{q},\ap{f}{r})}"yshift=-1em] & \ct{\ap{f}{p}}{(\ct{\ap{f}{q}}{\ap{f}{r}})}
%\end{tikzcd}
%\end{equation*}
%commutes.
%\end{subexenum}
\item \label{ex:trans_ap}Let $f:A\to B$ be a map, and consider $p:x=y$ in $A$. 
\begin{subexenum}
\item Construct for any $q:f(x)=b$ in $B$ an identification
\begin{equation*}
\mathsf{tr\usc{}id\usc{}left\usc{}subst}(p,q):\id{\mathsf{tr}_{f(\blank)=b}(p,q)}{\ct{\ap{f}{p}^{-1}}{q}}.
\end{equation*}
\item Similarly, construct for any $q':b=f(x)$ in $B$ an identification
\begin{equation*}
\mathsf{tr\usc{}id\usc{}right\usc{}subst}(p,q'):\id{\mathsf{tr}_{b=f(\blank)}(p,q)}{\ct{q}{\ap{f}{p}}}.
\end{equation*}
\end{subexenum}
\item \label{ex:inv_con}For any $p:x=y$, $q:y=z$, and $r:x=z$, construct maps
\begin{align*}
\mathsf{inv\usc{}con}(p,q,r) & : (\ct{p}{q}=r)\to (q=\ct{p^{-1}}{r}) \\
\mathsf{con\usc{}inv}(p,q,r) & : (\ct{p}{q}=r)\to (p=\ct{r}{q^{-1}}).
\end{align*}
\item Let $A$ be a type, and let $B$ be a type family over $A$.
Construct the \define{path lifting} operation
\begin{equation*}
\mathsf{lift}_B : \prd*{x,y:A}{p:\id{x}{y}}{b:B(x)} \id{\pairr{x,b}}{\pairr{y,\mathsf{tr}_B(p,b)}}.
\end{equation*}
In other words, a path in the \emph{base type} $A$ lifts to a path in the total space $\sm{x:A}B(x)$ for every term over the domain.
\item Consider four consecutive identifications
\begin{equation*}
\begin{tikzcd}
a \arrow[r,equals,"p"] & b \arrow[r,equals,"q"] & c \arrow[r,equals,"r"] & d \arrow[r,equals,"s"] & e
\end{tikzcd}
\end{equation*}
in a type $A$. In this exercise we will show that the \define{Mac Lane pentagon}\index{Mac Lane pentagon} for identifications commutes.
\begin{subexenum}
\item Construct the five identifications $\alpha_1,\ldots,\alpha_5$ in the pentagon
\begin{equation*}
\begin{tikzcd}[column sep=-1.5em]
&[-2em] \ct{(\ct{(\ct{p}{q})}{r})}{s} \arrow[rr,equals,"\alpha_4"] \arrow[dl,equals,swap,"\alpha_1"] & & \ct{(\ct{p}{q})}{(\ct{r}{s})} \arrow[dr,equals,"\alpha_5"] &[-2em] \\
\ct{(\ct{p}{(\ct{q}{r})})}{s} \arrow[drr,equals,swap,"\alpha_2"] & & & & \ct{p}{(\ct{q}{(\ct{r}{s})})}, \\
& & \ct{p}{(\ct{(\ct{q}{r})}{s})} \arrow[urr,equals,swap,"\alpha_3"]
\end{tikzcd}
\end{equation*}
where $\alpha_1$, $\alpha_2$, and $\alpha_3$ run counter-clockwise, and $\alpha_4$ and $\alpha_5$ run clockwise.
\item Show that
\begin{equation*}
\ct{(\ct{\alpha_1}{\alpha_2})}{\alpha_3} = \ct{\alpha_4}{\alpha_5}.
\end{equation*}
\end{subexenum}
\item \label{ex:semi-ring-laws-N}Show that the operations of addition and multiplication on the natural numbers satisfy the following laws:
\begin{align*}
m+(n+k) & =(m+n)+k & m\cdot (n\cdot k) & = (m\cdot n)\cdot k \\
m+0 & = m & m\cdot 1 & = m \\
0+m & = m & 1\cdot m & = m \\
m+n & = n+m & m\cdot n & = n\cdot m\\
& & m\cdot (n+k) & = m\cdot n + m\cdot k.
\end{align*}
\item
  \begin{subexenum}
  \item State Goldbach's conjecture in type theory.
  \item State the twin prime conjecture in type theory.
  \end{subexenum}
\end{exercises}
