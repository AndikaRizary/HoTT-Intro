\chapter{Homotopy pullbacks}

Suppose we are given a map $f:A\to B$, and type families $P$ over $A$, and $Q$ over $B$.
Then any fiberwise map
\begin{equation*}
g:\prd{x:A}P(x)\to Q(f(x))
\end{equation*}
gives rise to a commuting square
\begin{equation*}
\begin{tikzcd}[column sep=large]
\sm{x:A}P(x) \arrow[r,"{\total[f]{g}}"] \arrow[d,swap,"\proj 1"] & \sm{y:B}Q(y) \arrow[d,"\proj 1"] \\
A \arrow[r,swap,"f"] & B
\end{tikzcd}
\end{equation*}
where $\total[f]{g}$ is defined as $\lam{(x,p)}(f(x),g(x,y))$. 
We will show in \cref{thm:pb_fibequiv} that $g$ is a fiberwise equivalence\index{fiberwise equivalence} if and only if this square is a \emph{pullback square}\index{pullback square}. This generalization of \cref{thm:fib_equiv} is therefore abstracting away from the notion of fiberwise equivalence, and it serves as our motivating theorem to introduce pullbacks. The connection between pullbacks and fiberwise equivalences has an important role in the descent theorem\index{descent} in \cref{chap:descent}.

\section{Cartesian squares}

Recall that a square
\begin{equation*}
\begin{tikzcd}
C \arrow[r,"q"] \arrow[d,swap,"p"] & B \arrow[d,"g"] \\
A \arrow[r,swap,"f"] & X
\end{tikzcd}
\end{equation*}
is said to \define{commute}\index{commuting square|textbf} if there is a homotopy $H:f\circ p\htpy g\circ q$. 
The pullback property is a \emph{universal property} of the upper left corner of a commuting square (in our case $C$), characterizing the maps \emph{into} it.

To describe the universal property of pullbacks we first need to have a closer look at the \emph{anatomy} of commuting squares.

\begin{defn}\label{defn:cospan}
A commuting square
\begin{equation*}
\begin{tikzcd}
C \arrow[r,"q"] \arrow[d,swap,"p"] & B \arrow[d,"g"] \\
A \arrow[r,swap,"f"] & X
\end{tikzcd}
\end{equation*}
with $H:f\circ p\htpy g\circ q$ can be dissected into three parts, consisting of a \emph{cospan}, a type, and a \emph{cone}, where
\begin{enumerate}
\item A \define{cospan}\index{cospan|textbf} consists of three types $A$, $X$, and $B$, and maps $f:A\to X$ and $g:B\to X$.
\item Given a type $C$, a \define{cone}\index{cone!on a cospan|textbf} on the cospan $A \stackrel{f}{\rightarrow} X \stackrel{g}{\leftarrow} B$ with \define{vertex} $C$\index{vertex!of a cone|textbf} consists of maps $p:C\to A$, $q:C\to B$ and a homotopy $H:f\circ p\htpy g\circ q$. We write\index{cone(C)@{$\mathsf{cone}(\blank)$}|textbf}
\begin{equation*}
\mathsf{cone}(C)\defeq \sm{p:C\to A}{q:C\to B}f\circ p\htpy g\circ q
\end{equation*}
for the type of cones with vertex $C$.
\end{enumerate}
\end{defn}

Given a cone with vertex $C$ on a span $A\stackrel{f}{\rightarrow} X \stackrel{g}{\leftarrow} B$ and a map $h:C'\to C$, we construct a new cone with vertex $C'$ in the following definition.

\begin{defn}
For any cone $(p,q,H)$ with vertex $C$ and any type $C'$, we define a map\index{cone map@{$\mathsf{cone\usc{}map}$}|textbf}
\begin{equation*}
\mathsf{cone\usc{}map}(p,q,H):(C'\to C)\to\mathsf{cone}(C')
\end{equation*}
by $h\mapsto (p\circ h,q\circ h,H\circ h)$. 
\end{defn}

\begin{defn}
We say that a commuting square
\begin{equation*}
\begin{tikzcd}
C \arrow[r,"q"] \arrow[d,swap,"p"] & B \arrow[d,"g"] \\
A \arrow[r,swap,"f"] & X
\end{tikzcd}
\end{equation*}
with $H:f\circ p\htpy g\circ q$ is a \define{pullback square}\index{pullback square|textbf}, or that it is \define{cartesian}\index{cartesian square|textbf}, if it satisfies the \define{universal property} of pullbacks\index{universal property!of pullbacks}, which asserts that the map
\begin{equation*}
\mathsf{cone\usc{}map}(p,q,H):(C'\to C)\to\mathsf{cone}(C')
\end{equation*}
is an equivalence for every type $C'$. 
\end{defn}

We often indicate the universal property with a diagram as follows:
\begin{equation*}
\begin{tikzcd}
C' \arrow[drr,bend left=15,"{q'}"] \arrow[dr,densely dotted,"h"] \arrow[ddr,bend right=15,swap,"{p'}"] \\
& C \arrow[r,"q"] \arrow[d,swap,"p"] & B \arrow[d,"g"] \\
& A \arrow[r,swap,"f"] & X
\end{tikzcd}
\end{equation*}
since the universal property states that for every cone $(p',q',H')$ with vertex $C'$, the type of pairs $(h,\alpha)$ consisting of $h:C'\to C$ equipped with $\alpha:\mathsf{cone\usc{}map}((p,q,H),h)=(p',q',H')$ is contractible by \cref{thm:contr_equiv}.

In order to see what goes on in the universal property of pullbacks, we need to first characterize the identity type of $\mathsf{cone}(C)$, for any type $C$.

\begin{lem}\label{lem:id_cone}%
\index{identity type!of cone@{of $\mathsf{cone}(C)$}|textit}%
Let $(p,q,H)$ and $(p',q',H')$ be cones on a cospan $f:A\rightarrow X \leftarrow B:g$, both with vertex $C$. Then the type $(p,q,H)=(p',q',H')$ is equivalent to the type of triples $(K,L,M)$ consisting of
\begin{align*}
K & : p \htpy p' \\
L & : q \htpy q' \\
M & : \ct{H}{(g\cdot L)} \htpy \ct{(f\cdot K)}{H'}
\end{align*}
\end{lem}

\begin{rmk}
The homotopy $M$ witnesses that the square
\begin{equation*}
\begin{tikzcd}
f\circ p \arrow[r,"f\cdot K"] \arrow[d,swap,"H"] & f\circ p' \arrow[d,"{H'}"] \\
g\circ q \arrow[r,swap,"g\cdot L"] & g\circ q'
\end{tikzcd}
\end{equation*}
of homotopies commutes. Therefore $M$ is a homotopy of homotopies, and for each $z:C$ the identification $M(z)$ witnesses that the square of identifications
\begin{equation*}
\begin{tikzcd}[column sep=huge]
f(p(z)) \arrow[r,equals,"\ap{f}{K(z)}"] \arrow[d,equals,swap,"H(z)"] & f(p'(z)) \arrow[d,equals,"{H'(z)}"] \\
g(q(z)) \arrow[r,equals,swap,"\ap{g}{L(z)}"] & g(q'(z))
\end{tikzcd}
\end{equation*}
commutes. 
\end{rmk}

\begin{proof}[Proof of \cref{lem:id_cone}]
By the fundamental theorem of identity types (\cref{thm:id_fundamental}) and associativity of $\Sigma$-types (\cref{ex:sigma_assoc}) it suffices to show that the type
\begin{equation*}
\sm{p':C\to A}{q':C\to B}{H':f\circ p'\htpy g\circ q'}{K:p\htpy p'}{L:q\htpy q'} \ct{H}{(g\cdot L)} \htpy \ct{(f\cdot K)}{H'}
\end{equation*}
is contractible. Now we apply \cref{ex:sigma_swap} repeatedly to see that this type is equivalent to the type
\begin{equation*}
\sm{p':C\to A}{K: p\htpy p'}{q':C\to B}{L: q\htpy q'}{H':f\circ p'\htpy g\circ q'} \ct{H}{(g\cdot L)} \htpy \ct{(f\cdot K)}{H'}.
\end{equation*}
The types $\sm{p':C\to A} p\htpy p'$ and $\sm{q':C\to B} q\htpy q'$ are contractible by function extensionality, and  we have
\begin{samepage}
\begin{align*}
(p,\mathsf{htpy\usc{}refl}_p) & : \sm{p':C'\to A} p\htpy p' \\
(q,\mathsf{htpy\usc{}refl}_q) & : \sm{q':C'\to B} q\htpy q'.
\end{align*}%
\end{samepage}%
Thus we apply \cref{ex:contr_in_sigma} to see that the type of tuples $(p',K,q',L,H',M)$ is equivalent to the type
\begin{equation*}
\sm{H':f\circ p'\htpy g\circ q'} \ct{H}{\mathsf{htpy\usc{}refl}_{g\circ q}}\htpy \ct{\mathsf{htpy\usc{}refl}_{f\circ p}}{H'}.
\end{equation*}
Of course, the type $\ct{H}{\mathsf{htpy\usc{}refl}_{g\circ q}}\htpy \ct{\mathsf{htpy\usc{}refl}_{f\circ p}}{H'}$ is equivalent to the type $H\htpy H'$, and $\sm{H':f\circ p\htpy g\circ q} H\htpy H'$ is contractible.
\end{proof}

As a corollary we obtain the following characterization of the universal property of pullbacks.

\begin{thm}\label{thm:pullback_up}
Consider a commuting square
\begin{equation*}
\begin{tikzcd}
C \arrow[r,"q"] \arrow[d,swap,"p"] & B \arrow[d,"g"] \\
A \arrow[r,swap,"f"] & X
\end{tikzcd}
\end{equation*}
with $H:f\circ p\htpy g\circ q$
Then the following are equivalent:\index{universal property!of pullbacks (characterization)|textit}
\begin{enumerate}
\item The square is a pullback square.
\item For every type $C'$ and every cone $(p',q',H')$ with vertex $C'$, the type of quadruples $(h,K,L,M)$ consisting of
\begin{align*}
h & : C'\to C \\
K & : p\circ h \htpy p' \\
L & : q\circ h \htpy q' \\
M & : \ct{(H\cdot h)}{(g\cdot L)} \htpy \ct{(f\cdot K)}{H'}
\end{align*}
is contractible.
\end{enumerate}
\end{thm}

\begin{rmk}
The homotopy $M$ in \cref{thm:pullback_up} witnesses that the square
\begin{equation*}
\begin{tikzcd}
f\circ p\circ h \arrow[r,"f\cdot K"] \arrow[d,swap,"H\cdot h"] & f\circ p' \arrow[d,"{H'}"] \\
g\circ q\circ h \arrow[r,swap,"g\cdot L"] & g\circ q'
\end{tikzcd}
\end{equation*}
of homotopies commutes.
\end{rmk}

\section{The unique existence of pullbacks}

\begin{defn}
Let $f:A\to X$ and $B\to X$ be maps. Then we define
\begin{align*}
A\times_X B & \defeq \sm{x:A}{y:B}f(x)=g(y) \\
\pi_1 & \defeq \proj 1 & & : A\times_X B\to A \\
\pi_2 & \defeq \proj 1\circ\proj 2 & & : A\times_X B\to B\\
\pi_3 & \defeq \proj 2\circ\proj 2 & & : f\circ \pi_1 \htpy g\circ\pi_2.
\end{align*}
The type $A\times_X B$ is called the \define{canonical pullback}\index{canonical pullback|textbf} of $f$ and $g$.
\end{defn}

Note that $A\times_X B$ depends on $f$ and $g$, although this dependency is not visible in the notation.

\begin{thm}
Given maps $f:A\to X$ and $g:B\to X$, the commuting square\index{canonical pullback|textit}
\begin{equation*}
\begin{tikzcd}
A\times_X B \arrow[r,"\pi_2"] \arrow[d,swap,"\pi_1"] & B \arrow[d,"g"] \\
A \arrow[r,swap,"f"] & X,
\end{tikzcd}
\end{equation*}
is a pullback square.
\end{thm}

\begin{proof}
Let $C$ be a type. Our goal is to show that the map
\begin{equation*}
\mathsf{cone\usc{}map}(\pi_1,\pi_2,\pi_3): (C\to A\times_X B)\to \mathsf{cone}(C)
\end{equation*}
is an equivalence. 
By double application of \cref{thm:choice} we obtain equivalences
\begin{align*}
(C\to A\times_X B) & \jdeq C\to \sm{x:A}{y:B}f(x)=g(y) \\
& \eqvsym \sm{p:C\to A}\prd{z:C}\sm{y:B} f(p(z))= y \\
& \eqvsym \sm{p:C\to A}{q:C\to B}\prd{z:C} f(p(z))= g(q(z)) \\
& \jdeq \mathsf{cone}(C)
\end{align*}
The composite of these equivalences is the map
\begin{equation*}
\lam{f}(\lam{z}\proj 1(f(z)),\lam{z} \proj 1(\proj 2(f(z))),\lam{z}\proj 2(\proj 2(f(z)))),
\end{equation*}
which is \emph{exactly} the map $\mathsf{cone\usc{}map}(\pi_1,\pi_2,\pi_3)$, and since it is a composite of equivalences it follows that it is itself an equivalence.
\end{proof}

In the following lemma we establish the uniqueness of pullbacks up to equivalence via a \emph{3-for-2 property} for pullbacks.

\begin{lem}\label{lem:pb_3for2}\index{pullback!3-for-2 property|textit}\index{3-for-2 property!of pullbacks|textit}%
Consider the squares
\begin{equation*}
\begin{tikzcd}
C \arrow[r,"q"] \arrow[d,swap,"p"] & B \arrow[d,"g"] & {C'} \arrow[r,"{q'}"] \arrow[d,swap,"{p'}"] & B \arrow[d,"g"] \\
A \arrow[r,swap,"f"] & X & A \arrow[r,swap,"f"] & X
\end{tikzcd}
\end{equation*}
with homotopies $H:f\circ p \htpy g\circ q$ and $H':f\circ p'\htpy g\circ q'$.
Furthermore, suppose we have a map $h:C'\to C$ equipped with
\begin{align*}
K & : p\circ h \htpy p' \\
L & : q\circ h \htpy q' \\
M & : \ct{(H\cdot h)}{(g\cdot L)} \htpy \ct{(f\cdot K)}{H'}.
\end{align*}
If any two of the following three properties hold, so does the third:
\begin{samepage}%
\begin{enumerate}
\item $C$ is a pullback.
\item $C'$ is a pullback.
\item $h$ is an equivalence.
\end{enumerate}%
\end{samepage}%
\end{lem}

\begin{proof}
By the characterization of the identity type of $\mathsf{cone}(C')$ given in \cref{lem:id_cone} we obtain an identification
\begin{equation*}
\mathsf{cone\usc{}map}((p,q,H),h)=(p',q',H')
\end{equation*}
from the triple $(K,L,M)$. 
Let $D$ be a type, and let $k:D\to C'$ be a map. We observe that
\begin{align*}
\mathsf{cone\usc{}map}((p,q,H),(h\circ k)) & \jdeq (p\circ (h\circ k),q\circ (h\circ k),H\circ (h\circ k)) \\
& \jdeq ((p\circ h)\circ k,(q\circ h)\circ k, (H\circ h)\circ k) \\
& \jdeq \mathsf{cone\usc{}map}(\mathsf{cone\usc{}map}((p,q,H),h),k) \\
& = \mathsf{cone\usc{}map}((p',q',H'),k).
\end{align*}
Thus we see that the triangle 
\begin{equation*}
\begin{tikzcd}[column sep=-1em]
(D\to C') \arrow[rr,"{h\circ \blank}"] \arrow[dr,swap,"{\mathsf{cone\usc{}map}(p',q',H')}"] & & (D\to C) \arrow[dl,"{\mathsf{cone\usc{}map}(p,q,H)}"] \\
& \mathsf{cone}(D)
\end{tikzcd}
\end{equation*}
commutes. Therefore it follows from the 3-for-2 property of equivalences established in \cref{ex:3_for_2}, that if any two of the following properties hold, then so does the third:
\begin{enumerate}
\item The map $\mathsf{cone\usc{}map}(p,q,H):(D\to C)\to \mathsf{cone}(D)$ is an equivalence,
\item The map $\mathsf{cone\usc{}map}(p',q',H'):(D\to C')\to \mathsf{cone}(D)$ is an equivalence,
\item The map $h\circ\blank : (D\to C')\to (D\to C)$ is an equivalence.
\end{enumerate}
Thus the 3-for-2 property for pullbacks follows from the fact that $h$ is an equivalence if and only if $h\circ\blank : (D\to C')\to (D\to C)$ is an equivalence for any type $D$, which was established in \cref{lem:postcomp_equiv}.
\end{proof}

Pullbacks are not only unique in the sense that any two pullbacks of the same cospan are equivalent, they are \emph{uniquely unique}\index{uniquely uniqueness!of pullbacks} in the sense that the type of quadruples $(h,K,L,M)$ as in \cref{lem:pb_3for2} is contractible.

\begin{cor}
Suppose both commuting squares
\begin{equation*}
\begin{tikzcd}
C \arrow[r,"q"] \arrow[d,swap,"p"] & B \arrow[d,"g"] & {C'} \arrow[r,"{q'}"] \arrow[d,swap,"{p'}"] & B \arrow[d,"g"] \\
A \arrow[r,swap,"f"] & X & A \arrow[r,swap,"f"] & X
\end{tikzcd}
\end{equation*}
with homotopies $H:f\circ p \htpy g\circ q$ and $H':f\circ p'\htpy g\circ q'$ are pullback squares.
Then the type of quadruples $(e,K,L,M)$ consisting of an equivalence $e:\eqv{C'}{C}$ equipped with
\begin{align*}
K & : p\circ e \htpy p' \\
L & : q\circ e \htpy q' \\
M & : \ct{(g\cdot L)}{(H\cdot e)} \htpy \ct{(f\cdot K)}{H'}.
\end{align*}
is contractible.
\end{cor}

\begin{proof}
We have seen that the type of quadruples $(h,K,L,M)$ is equivalent to the fiber of $\mathsf{cone\usc{}map}(p,q,H)$ at $(p',q',H')$. By \cref{lem:pb_3for2} it follows that $h$ is an equivalence. Since $\isequiv(h)$ is a proposition by \cref{ex:isprop_isequiv}, and hence contractible as soon as it is inhabited, it follows that the type of quadruples $(e,K,L,M)$ is contractible. 
\end{proof}

\begin{defn}
Given a commuting square
\begin{equation*}
\begin{tikzcd}
C \arrow[r,"q"] \arrow[d,"p"] & B \arrow[d,"g"] \\
A \arrow[r,swap,"f"] & X
\end{tikzcd}
\end{equation*}
with $H:f\circ p \htpy g \circ q$, we define the \define{gap map}\index{gap map|textbf}\index{pullback!gap map|textbf}
\begin{equation*}
\mathsf{gap}(p,q,H):C \to A\times_X B
\end{equation*}
by $\lam{z}(p(z),q(z),H(z))$. Furthermore, we will write\index{is_pullback@{$\mathsf{is\usc{}pullback}$}|textbf}
\begin{equation*}
\mathsf{is\usc{}pullback}(f,g,H)\defeq \isequiv(\mathsf{gap}(p,q,H)).
\end{equation*}
\end{defn}

\begin{thm}\label{thm:is_pullback}
Consider a commuting square
\begin{equation*}
\begin{tikzcd}
C \arrow[r,"q"] \arrow[d,"p"] & B \arrow[d,"g"] \\
A \arrow[r,swap,"f"] & X
\end{tikzcd}
\end{equation*}
with $H:f\circ p \htpy g \circ q$. The following are equivalent:
\begin{enumerate}
\item The square is a pullback square
\item There is a term of type
\begin{equation*}
\mathsf{is\usc{}pullback}(p,q,H)\defeq \isequiv(\mathsf{gap}(p,q,H)).
\end{equation*}
\end{enumerate}
\end{thm}

\begin{proof}
Note that there are homotopies
\begin{align*}
K & : \pi_1\circ \mathsf{gap}(p,q,H) \htpy p \\
L & : \pi_2\circ \mathsf{gap}(p,q,H) \htpy q \\
M & : \ct{(\pi_3\cdot \mathsf{gap}(p,q,H))}{(g\cdot L)} \htpy \ct{(f\cdot K)}{H}.
\end{align*}
given by 
\begin{align*}
K & \defeq \lam{z}\refl{p(z)} \\
L & \defeq \lam{z}\refl{q(z)} \\
M & \defeq \lam{z}\ct{\mathsf{right\usc{}unit}(H(z))}{\mathsf{left\usc{}unit}(H(z))^{-1}}.
\end{align*}
Therefore the claim follows by \cref{lem:pb_3for2}.
\end{proof}

\section{Fiber products}

An important special case of pullbacks occurs when the cospan is of the form
\begin{equation*}
\begin{tikzcd}
A \arrow[r] & \unit & B. \arrow[l]
\end{tikzcd}
\end{equation*}
In this case, the pullback is just the \emph{cartesian product}.

\begin{lem}\label{lem:prod_pb}
Let $A$ and $B$ be types. Then the square
\begin{equation*}
\begin{tikzcd}
A\times B \arrow[r,"\proj 2"] \arrow[d,swap,"\proj 1"] & B \arrow[d,"\mathsf{const}_{\ttt}"] \\
A \arrow[r,swap,"\mathsf{const}_{\ttt}"] & \unit
\end{tikzcd}
\end{equation*}
which commutes by the homotopy $\mathsf{const}_{\refl{\ttt}}$ is a pullback square.\index{cartesian product!as pullback}
\end{lem}

\begin{proof}
By \cref{thm:is_pullback} it suffices to show that
\begin{equation*}
\mathsf{gap}(\proj 1,\proj2,\lam{(a,b)}\refl{\ttt})
\end{equation*}
is an equivalence. Its inverse is the map $\lam{(a,b,p)}(a,b)$.
\end{proof}

The following generalization of \cref{lem:prod_pb} is the reason why pullbacks are sometimes called \define{fiber products}\index{fiber product|textbf}.

\begin{thm}
Let $P$ and $Q$ be families over a type $X$. Then the square
\begin{equation*}
\begin{tikzcd}[column sep=8em]
\sm{x:X}P(x)\times Q(x) \arrow[r,"{\lam{(x,(p,q))}(x,q)}"] \arrow[d,swap,"{\lam{(x,(p,q))}(x,p)}"] & \sm{x:X}Q(x) \arrow[d,"\proj 1"] \\
\sm{x:X}P(x) \arrow[r,swap,"\proj 1"] & X,
\end{tikzcd}
\end{equation*}
which commutes by the homotopy
\begin{equation*}
H\defeq \lam{(x,(p,q))}\refl{x},
\end{equation*}
is a pullback square.
\end{thm}

\begin{proof}
By \cref{thm:is_pullback} it suffices to show that the gap map is an equivalence. The gap map is homotopic to the function
\begin{equation*}
\lam{(x,(p,q))}((x,p),(x,q),\refl{x})
\end{equation*}
is an equivalence. The inverse of this function is the map 
\begin{equation*}
\lam{((x,p),(y,q),\alpha)}(y,(\mathsf{tr}_P(\alpha,p),q)).\qedhere
\end{equation*}
\end{proof}

\begin{cor}
For any $f:A\to X$ and $g:B\to X$, the square
\begin{equation*}
\begin{tikzcd}[column sep=8em]
\sm{x:X}\fib{f}{x}\times\fib{g}{y} \arrow[r,"{\lam{(x,((a,p),(b,q)))}b}"] \arrow[d,swap,"{\lam{(x,((a,p),(b,q)))}a}"] & B \arrow[d,"g"]  \\
A \arrow[r,swap,"f"] & X
\end{tikzcd}
\end{equation*}
is a pullback square.
\end{cor}

\section{Fibers as pullbacks}

\begin{lem}\label{lem:fib_pb}
For any function $f:A\to B$, and any $b:B$, consider the square
\begin{equation*}
\begin{tikzcd}[column sep=large]
\fib{f}{b} \arrow[r,"\mathsf{const}_\ttt"] \arrow[d,swap,"\proj 1"] & \unit \arrow[d,"\mathsf{const}_b"] \\
A \arrow[r,swap,"f"] & B
\end{tikzcd}
\end{equation*}
which commutes by $\proj 2 : \prd{t:\fib{f}{b}} f(\proj 1(t))=b$. This is a pullback square.\index{fiber!as pullback|textit}
\end{lem}

\begin{proof}
By \cref{thm:is_pullback} it suffices to show that the gap map is an equivalence. The gap map is homotopic to the function
\begin{equation*}
\mathsf{total}(\lam{x}{p}(\ttt,p))
\end{equation*}
The map $\lam{x}{p}(\ttt,p)$ is a fiberwise equivalence by \cref{ex:contr_in_sigma}, so it induces an equivalence on total spaces by \cref{thm:fib_equiv}.
\end{proof}

\cref{lem:fib_pb} motivates the following definition of \emph{fiber sequences}, which play an important role in synthetic homotopy theory (and in algebraic topology). 

\begin{defn}
A \define{fiber sequence}\index{fiber sequence|textbf} consists of types $F$, $E$, and $B$ with \define{base points}\index{base point} $x:F$, $y:E$, and $b:B$, and maps
\begin{equation*}
\begin{tikzcd}
F \arrow[r,"i"] & E \arrow[r,"p"] & B
\end{tikzcd}
\end{equation*}
preserving the base points in the sense that $i(x)=y$ and $p(y)=b$, such that the square
\begin{equation*}
\begin{tikzcd}
F \arrow[r,"i"] \arrow[d] & E \arrow[d,"p"] \\
\unit \arrow[r,swap,"b"] & B
\end{tikzcd}
\end{equation*}
is a pullback square. We often write $F\hookrightarrow E \twoheadrightarrow B$ to indicate that we have a fiber sequence. 

Given a fiber sequence $F\hookrightarrow E\twoheadrightarrow B$, we call $B$ the \define{base space}\index{base space!of fiber sequence|textbf}, $E$ the \define{total space}\index{total space!of fiber sequence|textbf}, and $F$ the \define{fiber}\index{fiber!of fiber sequence|textbf}.
\end{defn}

\begin{eg}
For any type family $B$ over $A$ and any $a:A$ the square
\begin{equation*}
\begin{tikzcd}[column sep=large]
B(a) \arrow[d,swap,"{\lam{y}(a,y)}"] \arrow[r,"\mathsf{const}_\ttt"] & \unit \arrow[d,"\lam{\ttt}a"] \\
\sm{x:A}B(x) \arrow[r,swap,"\proj 1"] & A
\end{tikzcd}
\end{equation*}
is a pullback square. 

To see this, note that the gap map is homotopic to the function
\begin{equation*}
e \defeq \lam{y}((a,y),\refl{a}).
\end{equation*}
This function is an equivalence by \cref{ex:proj_fiber}.

Thus we see that if we additionally suppose that there is a term $b:B(a)$, then we obtain a fiber sequence
\begin{equation*}
\begin{tikzcd}
B(a) \arrow[r,hookrightarrow] & \sm{x:A}B(x) \arrow[r,->>] & A.
\end{tikzcd}
\end{equation*}
\end{eg}

\section{Fiberwise equivalences}

\begin{lem}\label{lem:pb_subst}
Let $f:A\to B$, and let $Q$ be a type family over $B$. Then the square
\begin{equation*}
\begin{tikzcd}[column sep=6em]
\sm{x:A}Q(f(x)) \arrow[r,"{\lam{(x,q)}(f(x),q)}"] \arrow[d,swap,"\proj 1"] & \sm{y:B}Q(b) \arrow[d,"\proj 1"] \\
A \arrow[r,swap,"f"] & B
\end{tikzcd}
\end{equation*}
commutes by $H\defeq \lam{(x,q)}\refl{f(x)}$. This is a pullback square.\index{substitution!as pullback|textit}
\end{lem}

\begin{proof}
By \cref{thm:is_pullback} it suffices to show that the gap map is an equivalence. The gap map is homotopic to the function
\begin{equation*}
\lam{(x,q)}(x,(f(x),q),\refl{f(x)}).
\end{equation*}
The inverse of this map is given by $\lam{(x,((y,q),p))}(x,\mathsf{tr}_Q(p^{-1},q))$, and it is straightforward to see that these maps are indeed mutual inverses.
\end{proof}

\begin{thm}\label{thm:pb_fibequiv}
Let $f:A\to B$, and let $g:\prd{a:A}P(a)\to Q(f(a))$ be a fiberwise transformation\index{fiberwise transformation|textit}. The following are equivalent:
\begin{enumerate}
\item The commuting square
\begin{equation*}
\begin{tikzcd}[column sep=large]
\sm{a:A}P(a) \arrow[r,"{\total[f]{g}}"] \arrow[d,->>] & \sm{b:B}Q(b) \arrow[d,->>] \\
A \arrow[r,swap,"f"] & B
\end{tikzcd}
\end{equation*}
is a pullback square.
\item $g$ is a fiberwise equivalence.\index{fiberwise equivalence|textit}
\end{enumerate}
\end{thm}

\begin{proof}
The gap map is homotopic to the composite
\begin{equation*}
\begin{tikzcd}[column sep=large]
\sm{x:A}P(x) \arrow[r,"\total{g}"] & \sm{x:A}Q(f(x)) \arrow[r,"{\mathsf{gap}'}"] & A \times_B \Big(\sm{y:B}Q(y)\Big)
\end{tikzcd}
\end{equation*}
where $\mathsf{gap}'$ is the gap map for the square in \cref{lem:pb_subst}. Since $\mathsf{gap}'$ is an equivalence, it follows by \cref{ex:3_for_2,thm:fib_equiv} that the gap map is an equivalence if and only if $g$ is a fiberwise equivalence.
\end{proof}

\begin{lem}
Consider a commuting square
\begin{equation*}
\begin{tikzcd}
C \arrow[r,"q"] \arrow[d,swap,"p"] & B \arrow[d,"g"] \\
A \arrow[r,swap,"f"] & X
\end{tikzcd}
\end{equation*}
with $H:f\circ p\htpy g\circ q$, and consider the fiberwise transformation
\begin{equation*}
\fibf{(f,q,H)} : \prd{a:A} \fib{p}{a}\to \fib{g}{f(a)}
\end{equation*}
given by $\lam{a}{(c,u)}(q(c),\ct{H(c)^{-1}}{\ap{f}{u}})$. Then there is an equivalence
\begin{equation*}
\eqv{\fib{\mathsf{gap}(p,q,H)}{(a,b,\alpha)}}{\fib{\fibf{(f,q,H)}(a)}{(b,\alpha^{-1})}}
\end{equation*}
\end{lem}

\begin{proof}
To obtain an equivalence of the desired type we simply concatenate known equivalences:
\begin{align*}
\fib{h}{(a,b,\alpha)} & \jdeq \sm{z:C} (p(z),q(z),H(z))=(a,b,\alpha) \\
& \eqvsym \sm{z:C}{u:p(z)=a}{v:q(z)=b}\ct{H(z)}{\ap{g}{v}}=\ct{\ap{f}{u}}{\alpha} \\
& \eqvsym \sm{(z,u):\fib{p}{a}}{v:q(z)=b} \ct{H(z)^{-1}}{\ap{f}{u}}=\ct{\ap{g}{v}}{\alpha^{-1}} \\
& \eqvsym \fib{\varphi(a)}{(b,\alpha^{-1})}\qedhere
\end{align*}
\end{proof}

\begin{cor}\label{cor:pb_fibequiv}
Consider a commuting square
\begin{equation*}
\begin{tikzcd}
C \arrow[r,"q"] \arrow[d,swap,"p"] & B \arrow[d,"g"] \\
A \arrow[r,swap,"f"] & X
\end{tikzcd}
\end{equation*}
with $H:f\circ p\htpy g\circ q$. The following are equivalent:
\begin{enumerate}
\item The square is a pullback square.\index{pullback square!characterized by fiberwise equivalence|textit}
\item The induced map on fibers
\begin{equation*}
\lam{x}{(z,\alpha)}(q(z),\ct{H(z)^{-1}}{\ap{f}{\alpha}}):\prd{x:A}\fib{p}{x}\to \fib{g}{f(x)}
\end{equation*}
is a fiberwise equivalence.
\end{enumerate}
\end{cor}

\begin{cor}\label{cor:pb_trunc}
Consider a pullback square
\begin{equation*}
\begin{tikzcd}
C \arrow[r,"q"] \arrow[d,swap,"p"] & B \arrow[d,"g"] \\
A \arrow[r,swap,"f"] & X.
\end{tikzcd}
\end{equation*}
If $g$ is a $k$-truncated map, then so is $p$. In particular, if $g$ is an embedding then so is $p$.\index{truncated!map!pullbacks of truncated maps|textit}\index{embedding!pullbacks of embeddings|textit}
\end{cor}

\begin{proof}
Since the square is assumed to be a pullback square, it follows from \cref{cor:pb_fibequiv} that for each $x:A$, the fiber $\fib{p}{x}$ is equivalent to the fiber $\fib{g}{f(x)}$, which is $k$-truncated. Since $k$-truncated types are closed under equivalences by \cref{thm:ktype_eqv}, it follows that $p$ is a $k$-truncated map.
\end{proof}

\begin{cor}\label{cor:pb_equiv}
Consider a commuting square
\begin{equation*}
\begin{tikzcd}
C \arrow[r,"q"] \arrow[d,swap,"p"] & B \arrow[d,"g"] \\
A \arrow[r,swap,"f"] & X.
\end{tikzcd}
\end{equation*}
and suppose that $g$ is an equivalence. Then the following are equivalent:
\begin{enumerate}
\item The square is a pullback square.
\item The map $p:C\to A$ is an equivalence.\index{equivalence!pullback of|textit}
\end{enumerate}
\end{cor}

\begin{proof}
If the square is a pullback square, then by \cref{thm:pb_fibequiv} the fibers of $p$ are equivalent to the fibers of $g$, which are contractible by \cref{thm:contr_equiv}. Thus it follows that $p$ is a contractible map, and hence that $p$ is an equivalence.

If $p$ is an equivalence, then by \cref{thm:contr_equiv} both $\fib{p}{x}$ and $\fib{g}{f(x)}$ are contractible for any $x:X$. It follows by \cref{ex:contr_equiv} that the induced map $\fib{p}{x}\to\fib{g}{f(x)}$ is an equivalence. Thus we apply \cref{cor:pb_fibequiv} to conclude that the square is a pullback.
\end{proof}

\begin{thm}\label{thm:pb_fibequiv_complete}
Consider a diagram of the form
\begin{equation*}
\begin{tikzcd}
A \arrow[d,swap,"f"] & B \arrow[d,"g"] \\
X \arrow[r,swap,"h"] & Y.
\end{tikzcd}
\end{equation*}
Then the type of triples $(i,H,p)$ consisting of a map $i:A\to B$, a homotopy $H:h\circ f\htpy g\circ i$, and a term $p$ witnessing that the square
\begin{equation*}
\begin{tikzcd}
A \arrow[d,swap,"f"] \arrow[r,"i"] & B \arrow[d,"g"] \\
X \arrow[r,swap,"h"] & Y.
\end{tikzcd}
\end{equation*}
is a pullback square, is equivalent to the type of fiberwise equivalences
\begin{equation*}
\prd{x:X}\eqv{\fib{f}{x}}{\fib{g}{h(x)}}.
\end{equation*}
\end{thm}

\begin{cor}\label{cor:pb_fibequiv_complete}
Let $h:X\to Y$ be a map, and let $P$ and $Q$ be families over $X$ and $Y$, respectively.
Then the type of triples $(i,H,p)$ consisting of a map 
\begin{equation*}
i:\Big(\sm{x:X}P(x)\Big)\to \Big(\sm{y:Y}Q(y)\Big),
\end{equation*}
a homotopy $H:h\circ \proj 1\htpy \proj 1\circ i$, and a term $p$ witnessing that the square
\begin{equation*}
\begin{tikzcd}
\sm{x:X}P(x) \arrow[d,swap,"\proj 1"] \arrow[r,"i"] & \sm{y:Y}Q(y) \arrow[d,"\proj 1"] \\
X \arrow[r,swap,"h"] & Y.
\end{tikzcd}
\end{equation*}
is a pullback square, is equivalent to the type of fiberwise equivalences
\begin{equation*}
\prd{x:X}\eqv{P(x)}{Q(h(x))}.
\end{equation*}
\end{cor}

\section{The pullback pasting property}

The following theorem is also called the \define{pasting property} of pullbacks.\index{pasting property!of pullbacks|textit}

\begin{thm}\label{thm:pb_pasting}
Consider a commuting diagram of the form
\begin{equation*}
\begin{tikzcd}
A \arrow[r,"k"] \arrow[d,swap,"f"] & B \arrow[r,"l"] \arrow[d,"g"] & C \arrow[d,"h"] \\
X \arrow[r,swap,"i"] & Y \arrow[r,swap,"j"] & Z
\end{tikzcd}
\end{equation*}
with homotopies $H:i\circ f\htpy g\circ k$ and $K:j\circ g\htpy h\circ l$, and the homotopy
\begin{equation*}
\ct{(j\cdot H)}{(K\cdot k)}:j\circ i\circ f\htpy h\circ l\circ k
\end{equation*}
witnessing that the outer rectangle commutes. Furthermore, suppose that the square on the right is a pullback square. Then the following are equivalent:
\begin{samepage}%
\begin{enumerate}
\item The square on the left is a pullback square.
\item The outer rectangle is a pullback square.
\end{enumerate}%
\end{samepage}%
\end{thm}

\begin{proof}
The commutativity of the two squares induces fiberwise transformations
\begin{align*}
& \prd{x:X}\fib{f}{x}\to \fib{g}{i(x)} \\
& \prd{y:Y}\fib{g}{y}\to \fib{h}{j(y)}.
\end{align*}
By the assumption that the square on the right is a pullback square, it follows from \cref{cor:pb_fibequiv} that the fiberwise transformation
\begin{equation*}
\prd{y:Y}\fib{g}{y}\to\fib{h}{j(y)}
\end{equation*}
is a fiberwise equivalence. Therefore it follows from 3-for-2 property of equivalences that the fiberwise transformation
\begin{equation*}
\prd{x:X}\fib{f}{x}\to\fib{g}{i(x)}
\end{equation*}
is a fiberwise equivalence if and only if the fiberwise transformation
\begin{equation*}
\prd{x:X}\fib{f}{x}\to\fib{h}{j(i(x))}
\end{equation*}
is a fiberwise equivalence. Now the claim follows from one more application of \cref{cor:pb_fibequiv}.
\end{proof}

\section{The disjointness of coproducts}

As an application of the theory of pullbacks, we show that coproducts are disjoint.\index{coproduct} In this section we will write
\begin{equation*}
[f,g] : A+B\to X
\end{equation*}
for the unique map satisfying $[f,g](\inl(x))\jdeq f(x)$ and $[f,g](\inr(y))\jdeq g(y)$, where $f:A\to X$ and $g:B\to X$. Furthermore, we will write
\begin{equation*}
f+g\defeq [\inl\circ f,\inr\circ g]:A+B\to X+B
\end{equation*}
for any $f:A\to X$ and $g:B\to Y$.

\begin{lem}\label{lem:pb_bool}
Let $X$ be a type. Then we have the pullback squares
\begin{equation*}
\begin{tikzcd}
X \arrow[r,"\mathsf{const}_\ttt"] \arrow[d,swap,"\idfunc"] &[2em] \unit \arrow[d,"\mathsf{const}_{\bfalse}"] & \emptyt \arrow[r] \arrow[d] &[2em] \unit \arrow[d,"\mathsf{const}_{\btrue}"] \\
X \arrow[r,swap,"\mathsf{const}_{\bfalse}"] & \bool & X \arrow[r,swap,"\mathsf{const}_{\bfalse}"] & \bool,
\end{tikzcd}
\end{equation*}
and we have similar pullback squares with the roles of $\bfalse$ and $\btrue$ reversed.
\end{lem}

\begin{proof}
For the first square we observe that both squares and the outer rectangle in the diagram
\begin{equation*}
\begin{tikzcd}[column sep=large]
X \arrow[d] \arrow[r] & \unit \arrow[d] \arrow[r] & \unit \arrow[d,"\mathsf{const}_{\bfalse}"] \\
X \arrow[r,swap,"\mathsf{const}_\ttt"] & \unit \arrow[r,swap,"\mathsf{const}_{\bfalse}"] & \bool.
\end{tikzcd}
\end{equation*}
are pullback squares. To see this, recall that the identity type $\bfalse=\bfalse$ is contractible by \cref{ex:eq_bool}. Therefore it follows that the square on the right is a pullback square by \cref{ex:id_pb}. The square on the left is a pullback square by \cref{cor:pb_equiv}. Therefore the outer rectangle is a pullback square by \cref{thm:pb_pasting}.

For the second square we observe that both squares end the outer rectangle in the diagram
\begin{equation*}
\begin{tikzcd}[column sep=large]
\emptyt \arrow[d] \arrow[r] & \emptyt \arrow[d] \arrow[r] & \unit \arrow[d,"\mathsf{const}_{\btrue}"] \\
X \arrow[r,swap,"\mathsf{const}_\ttt"] & \unit \arrow[r,swap,"\mathsf{const}_{\bfalse}"] & \bool.
\end{tikzcd}
\end{equation*}
are pullback squares.
To see this, recall that the identity type $\bfalse=\btrue$ is equivalent to the empty type by \cref{ex:eq_bool}. Therefore it follows that the square on the right is a pullback. It is also straightforward to verify that the square on the left is a pullback. Therefore it follows from \cref{thm:pb_pasting} that the outer rectangle is a pullback.
\end{proof}

\begin{lem}\label{lem:inl_pb}
For any two types $A$ and $B$, the squares
\begin{equation*}
\begin{tikzcd}[column sep=6.5em]
A \arrow[r,"\mathsf{const}_\ttt"] \arrow[d,swap,"\inl"] & \unit \arrow[d,"\mathsf{const}_{\bfalse}"] &[-3em] B \arrow[r,"\mathsf{const}_\ttt"] \arrow[d,swap,"\inr"] & \unit \arrow[d,"\mathsf{const}_{\btrue}"] \\
A+B \arrow[r,swap,"{[\mathsf{const}_{\bfalse},\mathsf{const}_{\btrue}]}"] & \bool & A+B \arrow[r,swap,"{[\mathsf{const}_{\bfalse},\mathsf{const}_{\btrue}]}"] & \bool
\end{tikzcd}
\end{equation*}
are pullback squares.
\end{lem}

\begin{proof}
The two cases are similar, so we only give the proof that the left square is a pullback. The left square commutes by the homotopy
\begin{equation*}
H\defeq \mathsf{htpy\usc{}refl}_{\mathsf{const}_{\bfalse}}.
\end{equation*}
To see that the asserted square is a pullback square we use \cref{thm:is_pullback} and show that the gap map is an equivalence. First we note that the gap map is homotopic to the function $e:A\to (A+B)\times_\bool\unit$ is defined by
\begin{equation*}
\lam{x}(\inl(x),\ttt,\refl{\bfalse}).
\end{equation*}
The inverse is defined by the induction principle of coproducts by
\begin{align*}
e^{-1}(\inl(x),t,\alpha) & \defeq x \\
e^{-1}(\inr(y),t,\alpha) & \defeq \ind{\emptyt}(\zeta(\alpha)),
\end{align*}
where $\zeta:\prd{x,y:\bool}(x=y)\to \mathsf{Eq}_\bool(x,y)$ is the canonical map of the identity type of $\bool$ into the observational equality on $\bool$. In the case of $\alpha:\bfalse=\btrue$ we obtain a term of $\mathsf{Eq}_\bool(\bfalse,\btrue)\jdeq \emptyt$. It is immediate from the computation rules that $e^{-1}\circ e\jdeq \idfunc$. 

The homotopy $e\circ e^{-1}\htpy \idfunc$ is again constructed by the induction principle of coproducts. In the $\inl$-case we have $e(e^{-1}(\inl(x),t,\alpha))\jdeq (\inl(x),\ttt,\refl{\bfalse})$. We define the identification
\begin{equation*}
(\inl(x),\ttt,\refl{\bfalse})=(\inl(x),t,\alpha)
\end{equation*}
by singleton induction on $t:\unit$ and $\alpha:\bfalse=\bfalse$ (both of which are terms of contractible types). Thus, it suffices to provide an identification
\begin{equation*}
(\inl(x),\ttt,\refl{\bfalse})=(\inl(x),\ttt,\refl{\bfalse}),
\end{equation*}
which we have by reflexivity. The $\inr$-case is again automatic, since we obtain a term of the empty type from $\alpha:\bfalse=\btrue$. This completes the proof that $e$ is an equivalence.
\end{proof}

\begin{cor}\label{cor:inl_emb}
The maps $\inl:A\to A+B$ and $\inr:B\to A+B$ are embeddings.\index{embedding!coproduct inclusions|textit}
\end{cor}

\begin{proof}
By the pullback squares of \cref{lem:inl_pb} and \cref{cor:pb_trunc} it suffices to show that $\unit\to\bool$ is an embedding. This is \cref{ex:injective}.
\end{proof}

\begin{thm}\label{thm:pb_disjoint}
Coproducts are \define{disjoint}\index{disjointness!of coproducts|textbf} in the sense that for any two types $A$ and $B$, the commuting square
\begin{equation*}
\begin{tikzcd}
\emptyt \arrow[r] \arrow[d] & B \arrow[d,"\inr"] \\
A \arrow[r,swap,"\inl"] & A+B
\end{tikzcd}
\end{equation*}
is a pullback square.
\end{thm}

\begin{proof}
Now consider the commuting diagram
\begin{equation*}
\begin{tikzcd}
\emptyt \arrow[d] \arrow[r] & B \arrow[d,"\inr"] \arrow[r] &[5em] \unit \arrow[d,"\mathsf{const}_\btrue"] \\
A \arrow[r,swap,"\inl"] & A+B \arrow[r,swap,"{[\mathsf{const}_{\bfalse},\mathsf{const}_{\btrue}]}"] & \bool.
\end{tikzcd}
\end{equation*}
By \cref{lem:pb_bool} it follows that the outer rectangle is a pullback square. The square on the right is a pullback square by \cref{lem:inl_pb}. Therefore the square on the left is a pullback square by \cref{thm:pb_pasting}.
\end{proof}

\begin{cor}\label{cor:id_coprod}
Let $A$ and $B$ be types. There are equivalences\index{identity type!of coproducts|textit}
\begin{align*}
(\inl(x)=\inl(x')) & \eqvsym (x=_A x') \\
(\inl(x)=\inr(y')) & \eqvsym \emptyt \\
(\inr(y)=\inl(x')) & \eqvsym \emptyt \\
(\inr(y)=\inr(y')) & \eqvsym (y=_B y').
\end{align*}
\end{cor}

\begin{proof}
The cases
\begin{align*}
(\inl(x)=\inl(x')) & \eqvsym (x=_A x') \\
(\inr(y)=\inr(y')) & \eqvsym (y=_B y').
\end{align*}
follow from \cref{cor:inl_emb} since both $\inl$ and $\inr$ are embeddings. The remaining cases follow from the disjointness of coproducts, proven in \cref{thm:pb_disjoint}.
\end{proof}

\begin{exercises}
\item \label{ex:id_pb}\index{identity type!as pullback}
\begin{subexenum}
\item Show that the square\index{identity type!as pullback}
\begin{equation*}
\begin{tikzcd}
(x=y) \arrow[r] \arrow[d] & \unit \arrow[d,"\mathsf{const}_y"] \\
\unit \arrow[r,swap,"\mathsf{const}_x"] & A
\end{tikzcd}
\end{equation*}
is a pullback square.
\item Show that the square\index{diagonal!of a type!fibers of}
\begin{equation*}
\begin{tikzcd}[column sep=large]
(x=y) \arrow[r,"\mathsf{const}_{x}"] \arrow[d,swap,"\mathsf{const}_\ttt"] & A \arrow[d,"\delta_A"] \\
\unit \arrow[r,swap,"{\mathsf{const}_{(x,y)}}"] & A\times A
\end{tikzcd}
\end{equation*}
is a pullback square, where $\delta_A:A\to A\times A$ is the diagonal of $A$, defined in \cref{ex:diagonal}.
\end{subexenum}
\item \label{ex:trunc_diagonal_map}In this exercise we give an alternative characterization of the notion of $k$-truncated map, compared to \cref{thm:trunc_ap}. Given a map $f:A\to X$ define the \define{diagonal}\index{diagonal!of a map} of $f$ to be the map $\delta_f:A\to A\times_X A$ given by $x\mapsto (x,x,\refl{f(x)})$.
\begin{subexenum}
\item Construct an equivalence
\begin{equation*}
\eqv{\fib{\delta_f}{(x,y,p)}}{\fib{\apfunc{f}}{p}}
\end{equation*}
to show that the square\index{action on paths!fibers of}\index{diagonal!of a map!fibers of}
\begin{equation*}
\begin{tikzcd}[column sep=large]
\fib{\apfunc{f}}{p} \arrow[r,"\mathsf{const}_x"] \arrow[d,swap,"\mathsf{const}_\ttt"] & A \arrow[d,"\delta_f"] \\
\unit \arrow[r,swap,"{\mathsf{const}_(x,y,p)}"] & A\times_X A
\end{tikzcd}
\end{equation*}
is a pullback square, for every $x,y:A$ and $p:f(x)=f(y)$.
\item Show that a map $f:A\to X$ is $(k+1)$-truncated if and only if $\delta_f$ is $k$-truncated.\index{truncated!map!by truncatedness of diagonal}
\end{subexenum}
Conclude that $f$ is an embedding if and only if $\delta_f$ is an equivalence.\index{embedding!diagonal is an equivalence}
\item Consider a commuting square
\begin{equation*}
\begin{tikzcd}
C \arrow[r,"q"] \arrow[d,swap,"p"] & B \arrow[d,"g"] \\
A \arrow[r,swap,"f"] & X
\end{tikzcd}
\end{equation*}
with $H:f\circ p\htpy g\circ q$. Show that the following are equivalent:
\begin{enumerate}
\item The square is a pullback square.
\item For every type $T$, the commuting square
\begin{equation*}
\begin{tikzcd}
C^T \arrow[r,"q\circ\blank"] \arrow[d,swap,"p\circ\blank"] & B^T \arrow[d,"g\circ\blank"] \\
A^T \arrow[r,swap,"f\circ\blank"] & X^T
\end{tikzcd}
\end{equation*}
is a pullback square.
\end{enumerate}
Note: property (ii) is really just a rephrasing of the universal property of pullbacks.\index{pullback square!universal property}
\item \label{ex:pb_diagonal}Consider a commuting square
\begin{equation*}
\begin{tikzcd}
C \arrow[r,"q"] \arrow[d,swap,"p"] & B \arrow[d,"g"] \\
A \arrow[r,swap,"f"] & X
\end{tikzcd}
\end{equation*}
with $H:f\circ p\htpy g\circ q$. Show that the following are equivalent:
\begin{enumerate}
\item The square is a pullback square.
\item The square
\begin{equation*}
\begin{tikzcd}
C \arrow[r,"g\circ q"] \arrow[d,swap,"{\lam{x}(p(x),q(x))}"] & X \arrow[d,"\delta_X"] \\
A\times B \arrow[r,swap,"f\times g"] & X\times X
\end{tikzcd}
\end{equation*}
which commutes by $\lam{z}\mathsf{eq\usc{}pair}(H(z),\refl{g(q(z))})$ is a pullback square.
\end{enumerate}
\item \label{ex:pb_prod}Show that if\index{pullback!cartesian products of pullbacks}
\begin{equation*}
\begin{tikzcd}
C_1 \arrow[r] \arrow[d] & B_1 \arrow[d] & C_2 \arrow[r] \arrow[d] & B_2 \arrow[d] \\
A_1 \arrow[r] & X_1 & A_2 \arrow[r] & X_2
\end{tikzcd}
\end{equation*}
are pullback squares, then so is
\begin{equation*}
\begin{tikzcd}
C_1\times C_2 \arrow[r] \arrow[d] & B_1\times B_2 \arrow[d] \\
A_1 \times A_2 \arrow[r] & X_1\times X_2. 
\end{tikzcd}
\end{equation*}
\item Consider for each $i:I$ a pullback square\index{pullback!Sigma-type of pullbacks@{$\Sigma$-type of pullbacks}}
\begin{equation*}
\begin{tikzcd}
C_i \arrow[r,"q_i"] \arrow[d,swap,"p_i"] & B_i \arrow[d,"g_i"] \\
A_i \arrow[r,swap,"f_i"] & X_i
\end{tikzcd}
\end{equation*}
with $H_i: f_i\circ p_i\htpy g_i\circ q_i$. 
\begin{subexenum}
\item \label{ex:pb_sigma}Show that the square
\begin{equation*}
\begin{tikzcd}[column sep=large]
\sm{i:I}C_i \arrow[r,"\total{q}"] \arrow[d,swap,"\total{p}"] & \sm{i:I}B_i \arrow[d,"\total{g}"] \\
\sm{i:I}A_i \arrow[r,swap,"\total{f}"] & \sm{i:I}X_i
\end{tikzcd}
\end{equation*}
which commutes by the homotopy
\begin{equation*}
\total{H}\defeq \lam{(i,c)}\mathsf{eq\usc{}pair}(\refl{i},H_i(c))
\end{equation*}
is a pullback square.\index{pullback!Pi-type of pullbacks@{$\Pi$-type of pullbacks}}
\item \label{ex:pb_pi}Show that the commuting square
\begin{equation*}
\begin{tikzcd}
\prd{i:I}C_i \arrow[r] \arrow[d] & \prd{i:I}B_i \arrow[d] \\
\prd{i:I}A_i \arrow[r] & \prd{i:I}X_i
\end{tikzcd}
\end{equation*}
is a pullback square.
\end{subexenum}
%\item 
%\begin{subexenum}
%\item Show that \index{equivalence!type of equivalences!as pullback}
%\begin{equation*}
%\begin{tikzcd}[column sep=8em]
%\eqv{A}{B} \arrow[r] \arrow[d] & \unit \arrow[d,"{(\idfunc[A],\idfunc[B])}"] \\
%A^B\times B^A \times A^B \arrow[r,swap,"{(h,f,g)\mapsto (h\circ f,f\circ g)}"] & A^A \times B^B
%\end{tikzcd}
%\end{equation*}
%is a pullback square.
%\item Show that \index{contractible!type of contractibility!as pullback}
%\begin{equation*}
%\begin{tikzcd}[column sep=6em]
%\iscontr(A) \arrow[r,"\mathsf{const}_{\ttt}"] \arrow[d,swap,"\proj 1"] & \unit \arrow[d,"{\lam{\ttt}\idfunc[A]}"] \\
%A \arrow[r,swap,"{\lam{x}\mathsf{const}_x}"] & A^A
%\end{tikzcd}
%\end{equation*}
%is a pullback square.
%\end{subexenum}
%\item Consider a commuting square
%\begin{equation*}
%\begin{tikzcd}
%C \arrow[r] \arrow[d] & A \arrow[d] \\
%B \arrow[r] & X.
%\end{tikzcd}
%\end{equation*}
%Show that this square is cartesian if and only if the induced map $C\to A\times_X B$ has a retraction.
\item \label{ex:pi_sec}Let $B$ be a type family over $A$. Show that the square\index{Pi-type@{$\Pi$-type}!as pullback}
\begin{equation*}
\begin{tikzcd}[column sep=6em]
\prd{x:A}B(x) \arrow[r,"{\lam{f}{x}(x,f(x))}"] \arrow[d] & \Big(\sm{x:A}B(x)\Big)^A \arrow[d,"\proj 1\circ\blank"] \\
\unit \arrow[r,swap,"{\mathsf{const}_{\idfunc[A]}}"] & A^A
\end{tikzcd}
\end{equation*}
is a pullback square. Conclude that the type $\prd{x:A}B(x)$ is equivalent to the type $\mathsf{sec}(\proj 1)$ of sections of the projection map.
\item Consider a pullback square
\begin{equation*}
\begin{tikzcd}
C \arrow[r,"q"] \arrow[d,swap,"p"] & B \arrow[d,"g"] \\
A \arrow[r,swap,"f"] & X,
\end{tikzcd}
\end{equation*}
with $H:f\circ p\htpy g\circ q$, and let $c_1,c_2:C$. Show that the square
\begin{equation*}
\begin{tikzcd}[column sep=8em]
(c_1=c_2) \arrow[r,"\apfunc{q}"] \arrow[d,swap,"\apfunc{p}"] & (q(c_1)=q(c_2)) \arrow[d,"\lam{\beta}\ct{H(c_1)}{\ap{g}{\beta}}"] \\
(p(c_1)=p(c_2)) \arrow[r,swap,"\lam{\alpha}\ct{\ap{f}{\alpha}}{H(c_2)}"] & f(p(c_1))=g(q(c_2)),
\end{tikzcd}
\end{equation*}
which commutes by the naturality of homotopies (\cref{defn:htpy_nat}), is again a pullback square.
%\end{subexenum}
%\item Suppose that the squares
%\begin{equation*}
%\begin{tikzcd}
%C \arrow[r,"q"] \arrow[d,swap,"p"] & B \arrow[d,"g"] & {C'} \arrow[r,"{q'}"] \arrow[d,swap,"{p'}"] & B \arrow[d,"g"] \\
%A \arrow[r,swap,"f"] & X & A \arrow[r,swap,"f"] & X
%\end{tikzcd}
%\end{equation*}
%with homotopies $H:f\circ p \htpy g\circ q$ and $H':f\circ p'\htpy g\circ q'$ are both pullback squares. Show that the type of equivalences $e:\eqv{C'}{C}$ equipped with an identification
%\begin{equation*}
%\mathsf{cone\usc{}map}((p,q,H),e)=(p',q',H')
%\end{equation*}
%is contractible.
\begin{comment}
\item Consider a \define{natural transformation of cospans}\index{cospan!natural transformation of}, i.e.~a commuting diagram of the form
\begin{equation*}
\begin{tikzcd}
A \arrow[r,"f"] \arrow[d,swap,"i"] & X \arrow[d,swap,"j"] & B \arrow[l,swap,"g"] \arrow[d,"k"] \\
A' \arrow[r,swap,"{f'}"] & X' & B'. \arrow[l,"{g'}"]
\end{tikzcd}
\end{equation*}
Show that the map
\begin{equation*}
(a,b,p)\mapsto (i(a),j(b),\mathsf{ap}_k(p)): A \times_X B \to A'\times_{X'} B'
\end{equation*}
is $k$-truncated if each of the vertical maps is.
\end{comment}
\item Suppose that 
\begin{equation*}
\begin{tikzcd}
C \arrow[r,"q"] \arrow[d,swap,"p"] & B \arrow[d,"g"] \\
A \arrow[r,swap,"f"] & X 
\end{tikzcd}
\end{equation*}
with $H:f\circ p\htpy g\circ q$ is a pullback square. Show that the square
\begin{equation*}
\begin{tikzcd}
C \arrow[r,"p"] \arrow[d,swap,"q"] & A \arrow[d,"f"] \\
B \arrow[r,swap,"g"] & X 
\end{tikzcd}
\end{equation*}
with $H^{-1}:g\circ q\htpy f\circ p$ is again a pullback square.
\item \label{ex:pb_fib}Consider a commuting square
\begin{equation*}
\begin{tikzcd}
C \arrow[d,swap,"p"] \arrow[r,"q"] & B \arrow[d,"g"] \\
A \arrow[r,swap,"f"] & X.
\end{tikzcd}
\end{equation*}
with $H:f\circ p\htpy g\circ q$, and let $h:C\to A\times_X B$ be the map given by $h(z)\defeq (p(c),q(c),H(c))$. 
Show that the square
\begin{equation*}
\begin{tikzcd}[column sep=6.5em]
\fib{\mathsf{gap}(p,q,H)}{(a,b,\alpha)} \arrow[d,swap,"\mathsf{const}_{\ttt}"] \arrow[r,"{\lam{(c,\beta)}(c,\ap{\pi_1}{\beta})}"] & \fib{p}{a} \arrow[d,"{\fibf{(f,g,H)}}"] \\
\unit \arrow[r,swap,"\mathsf{const}_{(b,\alpha^{-1})}"] & \fib{g}{f(a)}
\end{tikzcd}
\end{equation*}
\item \label{ex:pb_3by3}Consider a commuting diagram of the form
\begin{equation*}
\begin{tikzcd}
A_0 \arrow[r] \arrow[d] & B_0 \arrow[d] & C_0 \arrow[l] \arrow[d] \\
A_1 \arrow[r] & B_1 & C_1 \arrow[l] \\
A_2 \arrow[u] \arrow[r] & B_2 \arrow[u] & C_2 \arrow[u] \arrow[l]
\end{tikzcd}
\end{equation*}
with homotopies filling the (small) squares. Construct an equivalence
\begin{align*}
& (A_0\times_{B_0} C_0) \times_{(A_1\times_{B_1} C_1)} (A_2\times_{B_2} C_2) \\
& \qquad \eqvsym (A_0\times_{A_1} A_2) \times_{(B_0\times_{B_1} B_2)} (C_0\times_{C_1} C_2).
\end{align*}
This is also known as the \define{3-by-3 lemma}\index{3-by-3 lemma!for pullbacks} for pullbacks.
\end{exercises}
