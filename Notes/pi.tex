\chapter{Dependent function types and the natural numbers}
\chaptermark{\texorpdfstring{$\Pi$}{Π}-types and the natural numbers}

In this lecture we introduce types of functions of which the output can depend on the input. We call such functions \emph{dependent functions}. We will see that ordinary function types are a special case of dependent function types.

We then introduce the type of natural numbers, which is the single most important type. Dependent function types are used to formulate a type theoretic analogue of the induction principle for the natural numbers.

\section{Dependent function types}
Suppose we have
\begin{equation*}
  \Gamma,x:A\vdash B(x)~\mathrm{type}.
\end{equation*}
Then, in the context $\Gamma$, we consider $B$ to be a family of types indexed by $x:A$. Indeed, the judgment $\Gamma,x:A\vdash B(x)~\mathrm{type}$ asserts that in the context $\Gamma$, $B(x)$ is a type for every $x:A$.

We can make similar observations for the judgment
\begin{equation*}
  \Gamma,x:A\vdash b(x):B(x).
\end{equation*}
This time the judgment asserts that in the context $\Gamma$, we have a term $b(x)$ of type $B(x)$ for every $x:A$. Thus, $b$ is an operation that produces a term $b(x):B(x)$ for every $x:A$. When $B$ is viewed as a family over $A$, we might say that $b$ is a \emph{section} of the family $B$. This is analogous to sections of families of sets in set theory. From another point of view we might say that $b$ is a program that takes as imput an $x:A$, and has as output $b(x):B(x)$. A final point of view is that $b$ is a \emph{dependent} function, which is analogous to an ordinary function with the difference being that the type of its output may vary over its input. The idea of \emph{dependent function types} is that for any family $B$ over $A$ in context $\Gamma$, there is a type $\prd{x:A}B(x)$ in context $\Gamma$ of all dependent functions.

In this section we postulate the rules for dependent function types, and henceforth we will assume that they exist. The rules for dependent function types are organized in four groups:
\begin{enumerate}
\item The formation rules, which tell us how we may form dependent function types. Along with the formation rules we will also postulate conversion rules of judgmental type-equality for dependent function types.
\item The introduction rule, tells us how to introduce new terms of dependent function types, and the elimination rule tells us how to use a term of a dependent function type. Along with the introduction rule we postulate conversion rules that assert that our way of introducing functions behaves well with respect to judgmental equality.
\item The elimination rule, which tells us how to use arbitrary terms of dependent function types. We also postulate conversion rules that assert that elimination of functions behaves well with respect to judgmental equality.
\item The computation rules. These rules tell us how the introduction and elimination rules interact, and indeed that every term of a dependent function type behaves as expected: a function.
\end{enumerate}

\subsection{The $\Pi$-formation rule}
\define{Dependent function types}\index{dependent function type}\index{pi-type@{$\Pi$-type}} are formed by the following \define{$\Pi$-formation rule}\index{rule!pi-formation@{$\Pi$-formation}}:
\begin{prooftree}
\AxiomC{$\Gamma,x:A\vdash B(x)~\textrm{type}$}
\RightLabel{$\Pi$}
\UnaryInfC{$\Gamma\vdash \prd{x:A}B(x)~\mathrm{type}$}
\end{prooftree}
With the following conversion rule we postulate that formation of dependent function types respects judgmental equality.
\begin{prooftree}
\AxiomC{$\Gamma\vdash A\jdeq A'~\mathrm{type}$}
\AxiomC{$\Gamma,x:A\vdash B(x)\jdeq B'(x)~\textrm{type}$}
\RightLabel{$\Pi$-eq}
\BinaryInfC{$\Gamma\vdash \prd{x:A}B(x)\jdeq\prd{x:A'}B'(x)~\mathrm{type}$}
\end{prooftree}
Furthermore, when $x'$ is a fresh variable, i.e., which does not occur in the context $\Gamma,x:A$, we also postulate that
\begin{prooftree}
\AxiomC{$\Gamma,x:A\vdash B(x)~\textrm{type}$}
\RightLabel{$\Pi$-$x'/x$}
\UnaryInfC{$\Gamma\vdash \prd{x:A}B(x)\jdeq \prd{x':A}B(x')~\mathrm{type}$}
\end{prooftree}

\begin{rmk}
  One reason why we also postulate the conversion rules is that the situation might arise where we have two equal terms $\Gamma,x:A\vdash b(x)\jdeq b'(x):B(x)$ and a family $\Gamma,x:A,y:B(x)\vdash C(x,y)$. In this situation it follows that $\Gamma,x:A\vdash C(x,b(x))\jdeq C(x,b'(x))~\mathrm{type}$, so it should be the case that
  \begin{equation*}
    \Gamma\vdash \prd{x:A}C(x,b(x))\jdeq\prd{x:A}C(x,b'(x))~\mathrm{type}.
  \end{equation*}
\end{rmk}

\subsection{The $\Pi$-introduction rule}
The introduction rule for dependent function types is also called the $\lambda$-abstraction rule. Recall that dependent functions are formed from terms $b(x)$ of type $B(x)$ in context $\Gamma,x:A$. Therefore \define{$\lambda$-abstraction rule}\index{lambda-abstraction@{$\lambda$-abstraction}}\index{rule!lambda-abstraction@{$\lambda$-abstraction}} is as follows:
\begin{prooftree}
\AxiomC{$\Gamma,x:A \vdash b(x) : B(x)$}
\RightLabel{$\lambda_A$}
\UnaryInfC{$\Gamma\vdash \lam{x}b(x) : \prd{x:A}B(x)$}
\end{prooftree}
\begin{rmk}
  Just like ordinary mathematicians, we will sometimes write $x\mapsto f(x)$ for a function $f$. The map $x\mapsto x^2$ is an example.
\end{rmk}
The $\lambda$-abstraction is also required to respect judgmental equality. Therefore we postulate that
\begin{prooftree}
\AxiomC{$\Gamma,x:A \vdash b(x)\jdeq b'(x) : B(x)$}
\RightLabel{$\lambda_A$-eq}
\UnaryInfC{$\Gamma\vdash \lam{x}b(x)\jdeq \lam{x}b'(x) : \prd{x:A}B(x)$.}
\end{prooftree}

\subsection{The $\Pi$-elimination rule}

The elimination rule for dependent function types provides us with a way to \emph{use} dependent functions. The way to use a dependent function is to apply it to an argument of the domain type. The $\Pi$-elimination rule is therefore also called the \define{evaluation rule}\index{evaluation}\index{rule!evaluation}. It asserts that given a dependent function $f:\prd{x:A}B(x)$ in context $\Gamma$ we obtain a term $f(x)$ of type $B(x)$ in context $\Gamma,x:A$. More formally:
\begin{prooftree}
\AxiomC{$\Gamma\vdash f:\prd{x:A}B(x)$}
\RightLabel{$ev_A$}
\UnaryInfC{$\Gamma,x:A\vdash f(x) : B(x)$}
\end{prooftree}
Again we require that evaluation respects judgmental equality:
\begin{prooftree}
  \AxiomC{$\Gamma\vdash f\jdeq f':\prd{x:A}B(x)$}
  \UnaryInfC{$\Gamma,x:A\vdash f(x)\jdeq f'(x):B(x)$}
\end{prooftree}

\begin{rmk}
  Types of dependent functions with \emph{multiple} arguments can be obtained by iterating the $\Pi$-construction, as follows:
  \begin{prooftree}
    \AxiomC{$\Gamma,x:A,y:B(x)\vdash C(x,y)~\mathrm{type}$}
    \UnaryInfC{$\Gamma,x:A\vdash \prd{y:B(x)}C(x,y)~\mathrm{type}$}
    \UnaryInfC{$\Gamma\vdash\prd{x:A}{y:B(x)}C(x,y)~\mathrm{type}$}
  \end{prooftree}
  Terms $f$ of this type are dependent functions in two variables: by applying $f$ to a variable $x:A$ we obtain a term $f(x):\prd{y:B(x)}C(x,y)$. By further applying $f(x)$ to a term $y:B(x)$ we obtain a term $f(x,y):C(x,y)$.
\end{rmk}

\begin{rmk}
  By the derivation
  \begin{prooftree}
    \AxiomC{$\Gamma\vdash a:A$}
    \AxiomC{$\Gamma\vdash f: \prd{x:A}B(x)$}
    \UnaryInfC{$\Gamma,x:A\vdash f(x):B(x)$}
    \BinaryInfC{$\Gamma\vdash f(a):B(a)$}
  \end{prooftree}
  it follows that the rule
  \begin{prooftree}
    \AxiomC{$\Gamma\vdash a:A$}
    \AxiomC{$\Gamma\vdash f:\prd{x:A}B(x)$}
    \BinaryInfC{$\Gamma\vdash f(a):B(a)$}
  \end{prooftree}
  is admissible.
\end{rmk}


\begin{rmk}\label{rmk:ev_var}
By the derivation
\begin{prooftree}
\AxiomC{$\Gamma,x:A\vdash B~\mathrm{type}$}
\RightLabel{$\Pi_A$}
\UnaryInfC{$\Gamma\vdash\prd{x:A}B(x)~\mathrm{type}$}
\RightLabel{$\delta$}
\UnaryInfC{$\Gamma,f:\prd{x:A}B(x)\vdash f:\prd{x:A}B(x)$}
\RightLabel{$ev$}
\UnaryInfC{$\Gamma,f:\prd{x:A}B(x),x:A\vdash f(x):B(x)$}
\end{prooftree}
it follows that one can also evaluate variables of type $\prd{x:A}B(x)$.
\end{rmk}

\subsection{The $\Pi$-computation rules}
The computation rules for dependent function types postulate that $\lambda$-abstraction rule and the evaluation rule are mutual inverses. Thus we have two computation rules.

First we postulate the \define{$\beta$-rule}\index{beta-rule@{$\beta$-rule}}\index{rule!beta-rule@{$\beta$-rule}}
\begin{prooftree}
\AxiomC{$\Gamma,x:A \vdash b(x) : B(x)$}
\RightLabel{$\beta$}
\UnaryInfC{$\Gamma,x:A \vdash (\lambda y.b(y))(x)\jdeq b(x) : B(x)$.}
\end{prooftree}
Second, we postulate the \define{$\eta$-rule}\index{eta-rule@{$\eta$-rule}}\index{rule!eta-rule@{$\eta$-rule}}
\begin{prooftree}
\AxiomC{$\Gamma\vdash f:\prd{x:A}B(x)$}
\RightLabel{$\eta$}
\UnaryInfC{$\Gamma \vdash \lam{x}f(x) \jdeq f : \prd{x:A}B(x)$}
\end{prooftree}
This completes the specification of dependent function types.

\section{Ordinary function types}
In the case where both $A$ and $B$ are types in context $\Gamma$, we may first weaken $B$ by $A$, and then apply the formation rule for the dependent function type:
\begin{prooftree}
\AxiomC{$\Gamma\vdash A~\textrm{type}$}
\AxiomC{$\Gamma\vdash B~\textrm{type}$}
\BinaryInfC{$\Gamma,x:A\vdash B~\textrm{type}$}
\UnaryInfC{$\Gamma\vdash \prd{x:A}B~\textrm{type}$}
\end{prooftree}
The result is the type of functions that take an argument of type $A$, and return a term of type $B$. In other words, terms of the type $\prd{x:A}B$ are \emph{ordinary} functions from $A$ to $B$. We write $A\to B$ for the \define{type of functions}\index{function type} from $A$ to $B$.

We give a brief summary of the rules specifying ordinary function types, omitting the rules that the asserted operations respect judgmental equality. All of these rules can be derived from the corresponding rules for $\Pi$-types.
\begin{prooftree}
\AxiomC{$\Gamma\vdash A~\textrm{type}$}
\AxiomC{$\Gamma\vdash B~\textrm{type}$}
\RightLabel{$\to$\index{arrow-formation@{$\to$-formation}}\index{rule!arrow-formation@{$\to$-formation}}}
\BinaryInfC{$\Gamma\vdash A\to B~\textrm{type}$}
\end{prooftree}%
\begin{prooftree}
\AxiomC{$\Gamma\vdash B~\textrm{type}$}
\AxiomC{$\Gamma,x:A\vdash b(x):B$}
\RightLabel{$\lambda$\index{lambda-abstraction@{$\lambda$-abstraction}}\index{rule!lambda-abstraction@{$\lambda$-abstraction}}}
\BinaryInfC{$\Gamma\vdash \lam{x}b(x):A\to B$}
\end{prooftree}%
\begin{prooftree}
\AxiomC{$\Gamma\vdash f:A\to B$}
\RightLabel{$ev$\index{rule!evaluation}\index{evaluation}}
\UnaryInfC{$\Gamma,x:A\vdash f(x):B$}
\end{prooftree}%
\begin{prooftree}
\AxiomC{$\Gamma\vdash B~\textrm{type}$}
\AxiomC{$\Gamma,x:A\vdash b(x):B$}
\RightLabel{$\beta$\index{rule!beta-rule@{$\beta$-rule}}\index{beta-rule@{$\beta$-rule}}}
\BinaryInfC{$\Gamma,x:A\vdash(\lam{y}b(y))(x)\jdeq b(x):B$}
\end{prooftree}%
\begin{prooftree}
\AxiomC{$\Gamma\vdash f:A\to B$}
\RightLabel{$\eta$\index{rule!eta-rule@{$\eta$-rule}}\index{eta-rule@{$\eta$-rule}}}
\UnaryInfC{$\Gamma\vdash\lam{x} f(x)\jdeq f:A\to B$}
\end{prooftree}

\begin{rmk}
We also use the exponent notation $B^A$\index{exponent} for the function type $A\to B$.
\end{rmk}

\begin{rmk}
Similar to \cref{rmk:ev_var}, we can derive
\begin{prooftree}
\AxiomC{$\Gamma\vdash A~\mathrm{type}$}
\AxiomC{$\Gamma\vdash B~\mathrm{type}$}
\BinaryInfC{$\Gamma,f:B^A,x:A\vdash f(x):B$}
\end{prooftree}
\end{rmk}

\section{The identity function, composition, and their laws}
\begin{defn}
For any type $A$ in context $\Gamma$, we define the \define{identity function}\index{identity function|textbf} $\idfunc[A]:A\to A$ using the `variable rule'\index{variable rule}\index{rule!variable rule}:
\begin{prooftree}
\AxiomC{$\Gamma\vdash A~\textrm{type}$}
\UnaryInfC{$\Gamma,x:A\vdash x:A$}
\UnaryInfC{$\Gamma\vdash \idfunc[A]\defeq \lam{x}x:A\to A$}
\end{prooftree}
\end{defn}

\begin{defn}
For any three types $A$, $B$, and $C$ in context $\Gamma$, there is a \define{composition}\index{composition!of functions|textbf} operation
\begin{equation*}
\mathsf{comp}:(B\to C)\to ((A\to B)\to (A\to C)),
\end{equation*}
i.e., we can derive
\begin{prooftree}
\AxiomC{$\Gamma\vdash A~\textrm{type}$}
\AxiomC{$\Gamma\vdash B~\textrm{type}$}
\AxiomC{$\Gamma\vdash C~\textrm{type}$}
\TrinaryInfC{$\Gamma\vdash\mathsf{comp}:(B\to C)\to ((A\to B)\to (A\to C))$}
\end{prooftree}
We will write $g\circ f$ for $\mathsf{comp}(g,f)$.
\end{defn}

\begin{constr}
  The idea of the definition is to define $\mathsf{comp}(g,f)$ to be the function $\lam{x}g(f(x))$. The derivation we use to construct $\mathsf{comp}$ is as follows:
  \begin{small}
    \begin{prooftree}
      \AxiomC{$\Gamma\vdash A~\mathrm{type}$}
      \AxiomC{$\Gamma\vdash B~\mathrm{type}$}
      \BinaryInfC{$\Gamma,f:B^A,x:A\vdash f(x):B$}
      \UnaryInfC{$\Gamma,g:C^B,f:B^A,x:A\vdash f(x):B$}
      \AxiomC{$\Gamma\vdash B~\mathrm{type}$}
      \AxiomC{$\Gamma\vdash C~\mathrm{type}$}
      \BinaryInfC{$\Gamma,g:C^B,y:B\vdash g(y):C$}
      \UnaryInfC{$\Gamma,g:C^B,f:B^A,y:B\vdash g(y):C$}
      \UnaryInfC{$\Gamma,g:C^B,f:B^A,x:A,y:B\vdash g(y):C$}
      \BinaryInfC{$\Gamma,g:C^B,f:B^A,x:A\vdash g(f(x)) : C$}
      \UnaryInfC{$\Gamma,g:C^B,f:B^A\vdash \lam{x}g(f(x)):C^A$}
      \UnaryInfC{$\Gamma,g:B\to C\vdash \lam{f}{x}g(f(x)):B^A\to C^A$}
      \UnaryInfC{$\Gamma\vdash\mathsf{comp}\defeq \lam{g}{f}{x}g(f(x)):C^B\to (B^A\to C^A)$}
    \end{prooftree}
  \end{small}
\end{constr}

\begin{lem}
Composition of functions is associative\index{associativity!of function composition}\index{composition!of functions!associativity}, i.e., we can derive
\begin{prooftree}
\AxiomC{$\Gamma\vdash f:A\to B$}
\AxiomC{$\Gamma\vdash g:B\to C$}
\AxiomC{$\Gamma\vdash h:C\to D$}
\TrinaryInfC{$\Gamma \vdash (h\circ g)\circ f\jdeq h\circ(g\circ f):A\to D$}
\end{prooftree}
\end{lem}

\begin{proof}
  The main idea of the proof is that both $((h\circ g)\circ f)(x)$ and $(h\circ (g\circ f))(x)$ evaluate to $h(g(f(x))$, and therefore $(h\circ g)\circ f$ and $h\circ(g\circ f)$ must be judgmentally equal.

  This idea is made formal in the following derivation:
  \begin{prooftree}
    \AxiomC{$\Gamma\vdash f:A\to B$}
    \UnaryInfC{$\Gamma,x:A\vdash f(x):B$}
    \AxiomC{$\Gamma\vdash g:B\to C$}
    \UnaryInfC{$\Gamma,y:B\vdash g(y):C$}
    \BinaryInfC{$\Gamma,x:A\vdash g(f(x)):C$}
    \AxiomC{$\Gamma\vdash h:C\to D$}
    \UnaryInfC{$\Gamma,z:C\vdash h(z):D$}
    \BinaryInfC{$\Gamma,x:A\vdash h(g(f(x))):D$}
    \UnaryInfC{$\Gamma,x:A\vdash h(g(f(x)))\jdeq h(g(f(x))):D$}
    \UnaryInfC{$\Gamma,x:A\vdash (h\circ g)(f(x))\jdeq h((g\circ f)(x)):D$}
    \UnaryInfC{$\Gamma,x:A\vdash ((h\circ g)\circ f)(x)\jdeq (h\circ (g \circ f))(x):D$}
    \UnaryInfC{$\Gamma\vdash (h\circ g)\circ f\jdeq h\circ(g\circ f):A\to D$.}
  \end{prooftree}
\end{proof}

\begin{lem}\label{lem:fun_unit}
Composition of functions satisfies the left and right unit laws\index{left unit law|see {unit laws}}\index{right unit law|see {unit laws}}\index{unit laws!of function composition}, i.e., we can derive
\begin{prooftree}
\AxiomC{$\Gamma\vdash f:A\to B$}
\UnaryInfC{$\Gamma\vdash \idfunc[B]\circ f\jdeq f:A\to B$}
\end{prooftree}
and
\begin{prooftree}
\AxiomC{$\Gamma\vdash f:A\to B$}
\UnaryInfC{$\Gamma\vdash f\circ\idfunc[A]\jdeq f:A\to B$}
\end{prooftree}
\end{lem}

\begin{proof}
The derivation for the left unit law is
\begin{prooftree}
\AxiomC{$\Gamma\vdash f:A\to B$}
\UnaryInfC{$\Gamma,x:A\vdash f(x):B$}
\AxiomC{$\Gamma\vdash B~\mathrm{type}$}
\UnaryInfC{$\Gamma,y:B\vdash \idfunc[B](y)\jdeq y:B$}
\UnaryInfC{$\Gamma,x:A,y:B\vdash \idfunc[B](y)\jdeq y:B$}
\BinaryInfC{$\Gamma,x:A\vdash \idfunc[B](f(x))\jdeq f(x):B$}
\UnaryInfC{$\Gamma,x:A\vdash (\idfunc[B]\circ f)(x)\jdeq f(x):B$}
\UnaryInfC{$\Gamma\vdash \idfunc[B]\circ f\jdeq f:A\to B$}
\end{prooftree}
The right unit law is left as \autoref{ex:fun_right_unit}.
\end{proof}

\section{The natural numbers}
The archetypal example of an inductive type is the type $\N$ of \emph{natural numbers}.

\subsection{The formal specification of $\N$}
The type of \define{natural numbers}\index{natural numbers|see N@{$\N$}} is defined to be a closed type $\nat$\index{N@{$\N$}} equipped with closed terms for a \define{zero term}\index{zero term} $\zeroN:\N$ and a \define{successor function}\index{successor function!of N@{of $\N$}}\index{function!successor on N@{successor on $\N$}} $\succN:\N\to\N$.

Thus, the type $\N$ is formed by the $\N$-formation rule
\begin{prooftree}
  \AxiomC{}
  \RightLabel{$\N$-form}
  \UnaryInfC{$\vdash \N~\mathrm{type}$.}
\end{prooftree}

The introduction rules for $\N$ introduce the zero term and the successor function

\bigskip
\begin{minipage}{.45\textwidth}
  \begin{prooftree}
    \AxiomC{}
    \UnaryInfC{$\vdash \zeroN:\N$}
  \end{prooftree}
\end{minipage}
\begin{minipage}{.45\textwidth}
  \begin{prooftree}
    \AxiomC{}
    \UnaryInfC{$\vdash \succN:\N\to\N$}
  \end{prooftree}
\end{minipage}

\bigskip

To prove properties about the natural numbers, we postulate an \emph{induction principle}\index{induction principle!of N@{of $\N$}} for $\N$. In dependent type theory, however, the induction principle for the natural numbers provides a way to construct \emph{dependent functions} of types depending on the natural numbers.

The induction principle for $\N$ states what one has to do in order to construct a dependent function of type $\prd{n:\N}P(n)$, for any given type family $P$ over $\N$. Just like for the usual induction principle of the natural numbers, there are two things to be constructed: first one has to construct $p_0:P(\zeroN)$, and the second task is to construct a function of type $P(n)\to P(\succN(n))$ for all $n:\N$. 

Therefore the induction principle for $\N$ is as follows:
\begin{prooftree}
  \def\fCenter{\Gamma}
  \Axiom$\fCenter, n:\N\vdash P(n)~\mathrm{type}$
  \noLine
  \UnaryInf$\fCenter\ \vdash p_0:P(\zeroN)$
  \noLine
  \UnaryInf$\fCenter\ \vdash p_S:\prd{n:\N}P(n)\to P(\succN(n))$
  \RightLabel{$\N{-}\mathrm{Ind}$}
  \UnaryInf$\fCenter\ \vdash \ind{\N}(p_0,p_S):\prd{n:\N} P(n)$
\end{prooftree}
Furthermore we require that the dependent function $\ind{\N}(P,p_0,p_S)$ behaves as expected when it is applied to $\zeroN$ or a successor, i.e., with the same hypotheses as for the induction principle we postulate the \define{computation rules}\index{computation rules!of N@{of $\N$}} for $\N$
\begin{prooftree}
\AxiomC{$\cdots$}
%\RightLabel{$\N{-}\mathrm{Comp}(\zeroN)$}
\UnaryInfC{$\Gamma \vdash \ind{\N}(p_0,p_S,\zeroN)\jdeq p_0 : P(\zeroN)$}
\end{prooftree}
\begin{prooftree}
\AxiomC{$\cdots$}
%\RightLabel{$\N{-}\mathrm{Comp}(\succN)$}
\UnaryInfC{$\Gamma, n:\N \vdash  \ind{\N}(p_0,p_S,\succN(n))\jdeq p_S(n,\ind{\N}(p_0,p_S,n)) : P(\succN(n))$}
\end{prooftree}
This completes the formal specification of $\N$.

\begin{rmk}
  We annotate the terms $\zeroN$ and $\succN$ of type $\N$ with their type in the subscript, as a reminder that $\zeroN$ and $\succN$ are declared to be terms of type $\N$, and not of any other type. In the next chapter we will introduce the type $\Z$ of the integers, on which we can also define a zero term $\zeroZ$, and a successor function $\succZ$. These should be distinguished from the terms $\zeroN$ and $\succN$. In general, we will make sure that every term is given a unique name. In libraries of mathematics formalized in a computer proof assistant it is also the case that every type must be given a unique name.
\end{rmk}

\subsection{The definition of addition on $\N$}
Using the induction principle of $\N$ we can perform many familiar constructions. 
For instance, we can define the \define{addition operation}\index{addition!on N@{on $\N$}}\index{function!addition on N@{addition on $\N$}} by induction on $\N$.

\begin{defn}
  We define a function
  \begin{equation*}
    \addN:\N\to (\N\to\N)
  \end{equation*}
  satisfying $\addN(\zeroN,n)\jdeq n$ and $\addN(\succN(m),n)\jdeq\succN(\addN(m,n))$. Usually we will write $n+m$ for $\addN(n,m)$.
\end{defn}

\begin{proof}[Informal construction]
Informally, the definition of addition is as follows. By induction it suffices to construct a function $\mathsf{add\usc{}}\zeroN : \N\to\N$, and a function
\begin{align*}
\mathsf{add\usc{}}\succN(n,f):\N\to\N,
\end{align*}
for every $n:\N$ and every $f:\N\to\N$.

The function $\mathsf{add\usc{}}\zeroN:\N\to\N$ is of course taken to be $\idfunc[\N]$, since the result of adding $0$ to $n$ should be $n$.

Given $n:\N$ and a function $f:\N\to\N$ we define $\mathsf{add\usc{}}\succN(n,f)\defeq \succN\circ f$. The idea is that if $f$ represents adding $m$, then $\mathsf{add\usc{}}\succN(n,f)$ should be adding one more than $f$ did.
\end{proof}

\begin{proof}[Formal derivation]
The derivation for the construction of $\mathsf{add\usc{}}\succN$ looks as follows:
\begin{prooftree}
  \AxiomC{}
  \UnaryInfC{$\succN:\N^\N$}
  \AxiomC{}
  \UnaryInfC{$\vdash\N~\mathrm{type}$}
  \AxiomC{}
  \UnaryInfC{$\vdash\N~\mathrm{type}$}
  \AxiomC{}
  \UnaryInfC{$\vdash\N~\mathrm{type}$}
  \TrinaryInfC{$\vdash \mathsf{comp}:\N^\N\to (\N^\N\to \N^\N)$}
  \BinaryInfC{$\vdash \mathsf{comp}(\succN):\N^\N\to\N^\N$}
  \UnaryInfC{$n:\N\vdash \mathsf{comp}(\succN):\N^\N\to\N^\N$}
  \UnaryInfC{$\vdash \mathsf{add\usc{}}\succN\defeq \lam{n}\mathsf{comp}(\succN):\N\to (\N^\N \to \N^\N)$}
%\BinaryInfC{$\vdash\addN:\ind{\N}(add_0,add_S):\N\to \N^\N$}
\end{prooftree}
We combine this derivation with the induction principle of $\N$ to complete the construction of addition:
\begin{prooftree}
  \AxiomC{$\vdots$}
  \UnaryInfC{$n:\N\vdash \N^\N~\mathrm{type}$}
  \AxiomC{$\vdots$}
  \UnaryInfC{$\vdash \mathsf{add\usc{}}\zeroN\defeq \idfunc[\N]:\N^\N$}
  \AxiomC{$\vdots$}
  \UnaryInfC{$\vdash \mathsf{add\usc{}}\succN:\N\to (\N^\N \to \N^\N)$}
  \TrinaryInfC{$\vdash\addN\jdeq\ind{\N}(\mathsf{add\usc{}}\zeroN,\mathsf{add\usc{}}\succN):\N\to \N^\N$}
\end{prooftree}
The asserted judgmental equalities then hold by the computation rules for $\N$.
\end{proof}

\begin{rmk}
The rules that we provided so far are not sufficient to also conclude that $n+\zeroN\jdeq n$ and $n+ \succN(m)\jdeq \succN(n+m)$. However, once we have introduced the \emph{identity type} in \cref{chap:identity} we will nevertheless be able to \emph{identify} $n+\zeroN$ with $n$, and $n+ \succN(m)$ with $\succN(n+m)$. See \cref{ex:semi-ring-laws-N}. 
\end{rmk}

\begin{exercises}
\item \label{ex:fun_right_unit}Give a derivation for the right unit law of \autoref{lem:fun_unit}.\index{unit laws!for function composition}
\item Show that the rule
\begin{prooftree}
\AxiomC{$\Gamma,x:A \vdash b(x) : B(x)$}
\RightLabel{$\lambda_A$-$x'/x$}
\UnaryInfC{$\Gamma\vdash \lam{x}b(x)\jdeq \lam{x'}b(x') : \prd{x:A}B(x)$}
\end{prooftree}
is admissible for any variable $x'$ that does not occur in the context $\Gamma,x:A$.
\item 
  \begin{subexenum}
  \item Construct the \define{constant function}\index{constant function}\index{function!constant function}
    \begin{prooftree}
      \AxiomC{$\Gamma\vdash A~\textrm{type}$}
      \UnaryInfC{$\Gamma,y:B\vdash \mathsf{const}_y:A\to B$}
    \end{prooftree}
  \item Show that
    \begin{prooftree}
      \AxiomC{$\Gamma\vdash f:A\to B$}
      \UnaryInfC{$\Gamma,z:C\vdash \mathsf{const}_z\circ f\jdeq\mathsf{const}_z : A\to C$}
    \end{prooftree}
  \item Show that
    \begin{prooftree}
      \AxiomC{$\Gamma\vdash A~\textrm{type}$}
      \AxiomC{$\Gamma\vdash g:B\to C$}
      \BinaryInfC{$\Gamma,y:B\vdash g\circ\mathsf{const}_y\jdeq \mathsf{const}_{g(y)}:A\to C$}
    \end{prooftree}
  \end{subexenum}
\item In this exercise we construct some standard functions on the natural numbers.
  \begin{subexenum}
  \item Define the binary \define{min} and \define{max} functions $\min_\N,\max_\N:\N\to(\N\to\N)$.\index{minimum function}\index{maximum function}\index{function!min}\index{function!max}
  \item Define the \define{multiplication}\index{multiplication!on N@{on $\N$}}\index{function!multiplication on N@{multiplication on $\N$}} operation $\mathsf{mul}_\N :\N\to(\N\to\N)$.
  \item Define the \define{power}\index{power function on N@{power function on $\N$}}\index{function!power function on N@{power function on $\N$}} operation $n,m\mapsto m^n$ of type $\N\to (\N\to \N)$.
  \item Define the \define{factorial}\index{factorial function}\index{function!factorial function} function $n\mapsto n!$.
  \item Define the \define{binomial coefficient}\index{binomial coefficients} $\binom{n}{k}$ for any $n,k:\N$, making sure that $\binom{n}{k}\jdeq 0$ when $n<k$.
  \item Define the \define{Fibonacci sequence}\index{Fibonacci sequence} $0,1,1,2,3,5,8,13,\ldots$ as a function $F:\N\to\N$.
  \end{subexenum}
\item In this exercise we generalize the composition operation of non-dependent function types\index{composition!of dependent functions}:
\begin{subexenum}
\item Define a composition operation for dependent function types
\begin{prooftree}
\AxiomC{$\Gamma\vdash f:\prd{x:A}B(x)$}
\AxiomC{$\Gamma\vdash g:\prd{x:A}{y:B(x)} C(x,y)$}
\BinaryInfC{$\Gamma\vdash g\circ' f:\prd{x:A} C(x,f(x))$}
\end{prooftree}
and show that this operation agrees with ordinary composition when it is specialized to non-dependent function types.
\item Show that composition of dependent functions agrees with ordinary composition of functions:
  \begin{prooftree}
    \AxiomC{$\Gamma\vdash f:A\to B$}
    \AxiomC{$\Gamma\vdash g:B\to C$}
    \BinaryInfC{$\Gamma\vdash (\lam{x}g)\circ' f\jdeq g\circ f:A \to C$}
  \end{prooftree}
\item Show that composition of dependent functions is associative.\index{associativity!of dependent function composition}\index{composition!of dependent functions!associativity}
\item Show that composition of dependent functions satisfies the right unit law\index{unit laws!dependent function composition}:
\begin{prooftree}
\AxiomC{$\Gamma\vdash f:\prd{x:A}B(x)$}
\UnaryInfC{$\Gamma\vdash (\lam{x}f)\circ'\idfunc[A]\jdeq f :\prd{x:A}B(x)$}
\end{prooftree}
\item Show that composition of dependent functions satisfies the left unit law\index{unit laws!dependent function composition}:
\begin{prooftree}
\AxiomC{$\Gamma\vdash f:\prd{x:A}B(x)$}
\UnaryInfC{$\Gamma\vdash (\lam{x}\idfunc[B(x)])\circ' f\jdeq f:\prd{x:A}B(x)$}
\end{prooftree}
\end{subexenum}
\item \label{ex:swap}
\begin{subexenum}
\item Given two types $A$ and $B$ in context $\Gamma$, and a type $C$ in context $\Gamma,x:A,y:B$, define the \define{swap function}\index{function!swap}\index{swap function}
\begin{equation*}
\Gamma\vdash \sigma:\Big(\prd{x:A}{y:B}C(x,y)\Big)\to\Big(\prd{y:B}{x:A}C(x,y)\Big)
\end{equation*}
that swaps the order of the arguments.
\item Show that
\begin{equation*}
\Gamma\vdash \sigma\circ\sigma\jdeq\idfunc:\Big(\prd{x:A}{y:B}C(x,y)\Big)\to \Big(\prd{x:A}{y:B}C(x,y)\Big).
\end{equation*}
\end{subexenum}
\end{exercises}
