\chapter{Dependent function types and the natural numbers}
\chaptermark{\texorpdfstring{$\Pi$}{Π}-types and the natural numbers}

\section{Dependent function types}

\define{Dependent function types}\index{dependent function type}\index{pi-type@{$\Pi$-type}} are formed from a type $A$ and a type family $B$ over $A$, i.e.~the \define{$\Pi$-formation rule}\index{rule!pi-formation@{$\Pi$-formation}} is as follows:
\begin{prooftree}
\AxiomC{$\Gamma,x:A\vdash B(x)~\textrm{type}$}
\RightLabel{$\Pi$}
\UnaryInfC{$\Gamma\vdash \prd{x:A}B(x)~\mathrm{type}$}
\end{prooftree}
\begin{prooftree}
\AxiomC{$\Gamma\vdash A\jdeq A'~\mathrm{type}$}
\AxiomC{$\Gamma,x:A\vdash B(x)\jdeq B'(x)~\textrm{type}$}
\RightLabel{$\Pi$-eq}
\BinaryInfC{$\Gamma\vdash \prd{x:A}B(x)\jdeq\prd{x:A'}B'(x)~\mathrm{type}$}
\end{prooftree}
Furthermore, when $x'$ is a fresh variable, i.e. which does not occur in the context $\Gamma,x:A$, we also postulate that
\begin{prooftree}
\AxiomC{$\Gamma,x:A\vdash B(x)~\textrm{type}$}
\RightLabel{$\Pi$-$x'/x$}
\UnaryInfC{$\Gamma\vdash \prd{x:A}B(x)\jdeq \prd{x':A}B(x')~\mathrm{type}$}
\end{prooftree}
The idea of dependent function types is that their terms are functions of which the type of the output depends on the input. In other words, they consist of constructions that provide for every $x:A$ a term $b(x):B(x)$. Dependent functions are formed from terms $b(x)$ of type $B(x)$ in context $\Gamma,x:A$, i.e.~the \define{$\lambda$-abstraction rule}\index{lambda-abstraction@{$\lambda$-abstraction}}\index{rule!lambda-abstraction@{$\lambda$-abstraction}} is as follows:
\begin{prooftree}
\AxiomC{$\Gamma,x:A \vdash b(x) : B(x)$}
\RightLabel{$\lambda_A$}
\UnaryInfC{$\Gamma\vdash \lam{x}b(x) : \prd{x:A}B(x)$}
\end{prooftree}
\begin{prooftree}
\AxiomC{$\Gamma,x:A \vdash b(x)\jdeq b'(x) : B(x)$}
\RightLabel{$\lambda_A$-eq}
\UnaryInfC{$\Gamma\vdash \lam{x}b(x)\jdeq \lam{x}b'(x) : \prd{x:A}B(x)$}
\end{prooftree}
Furthermore, when $x'$ is a fresh variable, we also postulate that
\begin{prooftree}
\AxiomC{$\Gamma,x:A \vdash b(x) : B(x)$}
\RightLabel{$\lambda_A$-$x'/x$}
\UnaryInfC{$\Gamma\vdash \lam{x}b(x)\jdeq \lam{x'}b(x') : \prd{x:A}B(x)$}
\end{prooftree}
There are also rules providing a way to \emph{use} dependent functions.
This is determined by the \define{evaluation rule}\index{evaluation}\index{rule!evaluation}, which asserts that given a dependent function $f:\prd{x:A}B(x)$ in context $\Gamma$ we obtain a term $f(x)$ of type $B(x)$ in context $\Gamma,x:A$. More formally:
\begin{prooftree}
\AxiomC{$\Gamma\vdash f:\prd{x:A}B(x)$}
\RightLabel{$ev_A$}
\UnaryInfC{$\Gamma,x:A\vdash f(x) : B(x)$}
\end{prooftree}
In other words, every term of type $B(x)$ in context $\Gamma,x:A$ determines a term of type $\prd{x:A}B(x)$ in context $\Gamma$, and vice versa. The $\lambda$-abstraction rule and the evaluation rule are mutual inverses: we impose the \define{$\beta$-rule}\index{beta-rule@{$\beta$-rule}}\index{rule!beta-rule@{$\beta$-rule}}
\begin{prooftree}
\AxiomC{$\Gamma,x:A \vdash b(x) : B(x)$}
\RightLabel{$\beta$}
\UnaryInfC{$\Gamma,x:A \vdash (\lambda y.b(y))(x)\jdeq b(x) : B(x)$}
\end{prooftree}
and the \define{$\eta$-rule}\index{eta-rule@{$\eta$-rule}}\index{rule!eta-rule@{$\eta$-rule}}
\begin{prooftree}
\AxiomC{$\Gamma\vdash f:\prd{x:A}B(x)$}
\RightLabel{$\eta$}
\UnaryInfC{$\Gamma \vdash \lam{x}f(x) \jdeq f : \prd{x:A}B(x)$}
\end{prooftree}
This completes the specification of dependent function types.

\begin{rmk}
Types of dependent functions with \emph{multiple} arguments can be obtained by iterating the $\Pi$-construction.
\end{rmk}

\begin{rmk}
Some authors write
\begin{equation*}
(x:A)\to B(x)
\end{equation*}
for the dependent function type $\prd{x:A}B(x)$. 
\end{rmk}

\begin{rmk}\label{rmk:ev_var}
By the derivation
\begin{prooftree}
\AxiomC{$\Gamma,x:A\vdash B~\mathrm{type}$}
\RightLabel{$\Pi_A$}
\UnaryInfC{$\Gamma\vdash\prd{x:A}B(x)~\mathrm{type}$}
\RightLabel{$\delta$}
\UnaryInfC{$\Gamma,f:\prd{x:A}B(x)\vdash f:\prd{x:A}B(x)$}
\RightLabel{$ev$}
\UnaryInfC{$\Gamma,f:\prd{x:A}B(x),x:A\vdash f(x):B(x)$}
\end{prooftree}
it follows that one can also evaluate variables of type $\prd{x:A}B(x)$.
\end{rmk}

\section{Function types}
In the case where both $A$ and $B$ are types in context $\Gamma$, we may first weaken $B$ by $A$, and then apply the formation rule for the dependent function type:
\begin{prooftree}
\AxiomC{$\Gamma\vdash A~\textrm{type}$}
\AxiomC{$\Gamma\vdash B~\textrm{type}$}
\BinaryInfC{$\Gamma,x:A\vdash B~\textrm{type}$}
\UnaryInfC{$\Gamma\vdash \prd{x:A}B~\textrm{type}$}
\end{prooftree}
The result is the type of functions that take an argument of type $A$, and return a term of type $B$. In other words, terms of the type $\prd{x:A}B$ are \emph{ordinary} functions from $A$ to $B$. We write $A\to B$ for the \define{type of functions}\index{function type} from $A$ to $B$.

\begin{prooftree}
\AxiomC{$\Gamma\vdash A~\textrm{type}$}
\AxiomC{$\Gamma\vdash B~\textrm{type}$}
\RightLabel{$\to$}
\BinaryInfC{$\Gamma\vdash A\to B~\textrm{type}$}
\end{prooftree}%
\index{arrow-formation@{$\to$-formation}}\index{rule!arrow-formation@{$\to$-formation}}
\begin{prooftree}
\AxiomC{$\Gamma\vdash B~\textrm{type}$}
\AxiomC{$\Gamma,x:A\vdash b(x):B$}
\RightLabel{$\lambda$}
\BinaryInfC{$\Gamma\vdash \lam{x}b(x):A\to B$}
\end{prooftree}%
\index{lambda-abstraction@{$\lambda$-abstraction}}\index{rule!lambda-abstraction@{$\lambda$-abstraction}}
\begin{prooftree}
\AxiomC{$\Gamma\vdash f:A\to B$}
\RightLabel{$ev$}
\UnaryInfC{$\Gamma,x:A\vdash f(x):B$}
\end{prooftree}%
\index{rule!evaluation}\index{evaluation}
\begin{prooftree}
\AxiomC{$\Gamma\vdash B~\textrm{type}$}
\AxiomC{$\Gamma,x:A\vdash b(x):B$}
\RightLabel{$\beta$}
\BinaryInfC{$\Gamma,x:A\vdash(\lam{y}b(y))(x)\jdeq b(x):B$}
\end{prooftree}%
\index{rule!beta-rule@{$\beta$-rule}}\index{beta-rule@{$\beta$-rule}}
\begin{prooftree}
\AxiomC{$\Gamma\vdash f:A\to B$}
\RightLabel{$\eta$}
\UnaryInfC{$\Gamma\vdash\lam{x} f(x)\jdeq f:A\to B$}
\end{prooftree}
\index{rule!eta-rule@{$\eta$-rule}}\index{eta-rule@{$\eta$-rule}}

\begin{rmk}
We also use the exponent notation $B^A$\index{exponent} for the function type $A\to B$. Furthermore, we maintain the convention that the $\to$ associates to the right, i.e.~when we write $A\to B\to C$, we mean $A\to (B\to C)$.
\end{rmk}

\begin{rmk}
Similar to \cref{rmk:ev_var}, we can derive
\begin{prooftree}
\AxiomC{$\Gamma\vdash A~\mathrm{type}$}
\AxiomC{$\Gamma\vdash B~\mathrm{type}$}
\BinaryInfC{$\Gamma,f:B^A,x:A\vdash f(x):B$}
\end{prooftree}
\end{rmk}

\begin{defn}
For any type $A$ in context $\Gamma$, we define the \define{identity function}\index{identity function|textbf} $\idfunc[A]:A\to A$ using the `variable rule'\index{variable rule}\index{rule!variable rule}:
\begin{prooftree}
\AxiomC{$\Gamma\vdash A~\textrm{type}$}
\UnaryInfC{$\Gamma,x:A\vdash x:A$}
\UnaryInfC{$\Gamma\vdash \idfunc[A]\defeq \lam{x}x:A\to A$}
\end{prooftree}
\end{defn}

\begin{defn}
For any three types $A$, $B$, and $C$ in context $\Gamma$, there is a \define{composition}\index{composition!of functions|textbf} operation
\begin{equation*}
\mathsf{comp}:(B\to C)\to ((A\to B)\to (A\to C)),
\end{equation*}
i.e.~we can derive
\begin{prooftree}
\AxiomC{$\Gamma\vdash A~\textrm{type}$}
\AxiomC{$\Gamma\vdash B~\textrm{type}$}
\AxiomC{$\Gamma\vdash C~\textrm{type}$}
\TrinaryInfC{$\Gamma\vdash\mathsf{comp}:(B\to C)\to ((A\to B)\to (A\to C))$}
\end{prooftree}
We will write $g\circ f$ for $\mathsf{ev}(\mathsf{ev}(\mathsf{comp},g),f)$.
\end{defn}

\begin{constr}
We give the following derivation to define the composition operation:
\begin{small}
\begin{prooftree}
\AxiomC{$\Gamma\vdash A~\mathrm{type}$}
\AxiomC{$\Gamma\vdash B~\mathrm{type}$}
\BinaryInfC{$\Gamma,f:B^A,x:A\vdash f(x):B$}
\UnaryInfC{$\Gamma,g:C^B,f:B^A,x:A\vdash f(x):B$}
\AxiomC{$\Gamma\vdash B~\mathrm{type}$}
\AxiomC{$\Gamma\vdash C~\mathrm{type}$}
\BinaryInfC{$\Gamma,g:C^B,y:B\vdash g(y):C$}
\UnaryInfC{$\Gamma,g:C^B,f:B^A,y:B\vdash g(y):C$}
\UnaryInfC{$\Gamma,g:C^B,f:B^A,x:A,y:B\vdash g(y):C$}
\BinaryInfC{$\Gamma,g:C^B,f:B^A,x:A\vdash g(f(x)) : C$}
\UnaryInfC{$\Gamma,g:C^B,f:B^A\vdash \lam{x}g(f(x)):C^A$}
\UnaryInfC{$\Gamma,g:B\to C\vdash \lam{f}{x}g(f(x)):B^A\to C^A$}
\UnaryInfC{$\Gamma\vdash\mathsf{comp}\defeq \lam{g}{f}{x}g(f(x)):C^B\to (B^A\to C^A)$}
\end{prooftree}
\end{small}
\end{constr}

\begin{lem}
Composition of functions is associative\index{associativity!of function composition}\index{composition!of functions!associativity}, i.e. we can derive
\begin{prooftree}
\AxiomC{$\Gamma\vdash f:A\to B$}
\AxiomC{$\Gamma\vdash g:B\to C$}
\AxiomC{$\Gamma\vdash h:C\to D$}
\TrinaryInfC{$\Gamma \vdash (h\circ g)\circ f\jdeq h\circ(g\circ f):A\to D$}
\end{prooftree}
\end{lem}

\begin{proof}
In the following derivation we prove that composition of functions is associative.
\begin{prooftree}
\AxiomC{$\Gamma\vdash f:A\to B$}
\UnaryInfC{$\Gamma,x:A\vdash \mathsf{ev}(f,x):B$}
\AxiomC{$\Gamma\vdash g:B\to C$}
\UnaryInfC{$\Gamma,y:B\vdash \mathsf{ev}(g,y):C$}
\BinaryInfC{$\Gamma,x:A\vdash \mathsf{ev}(g,\mathsf{ev}(f,x)):C$}
\AxiomC{$\Gamma\vdash h:C\to D$}
\UnaryInfC{$\Gamma,z:C\vdash \mathsf{ev}(h,z):D$}
\BinaryInfC{$\Gamma,x:A\vdash \mathsf{ev}(h,\mathsf{ev}(g,\mathsf{ev}(f,x))):D$}
\UnaryInfC{$\Gamma,x:A\vdash \mathsf{ev}(h,\mathsf{ev}(g,\mathsf{ev}(f,x)))\jdeq\mathsf{ev}(h,\mathsf{ev}(g,\mathsf{ev}(f,x))):D$}
\UnaryInfC{$\Gamma,x:A\vdash \mathsf{ev}(h\circ g,\mathsf{ev}(f,x))\jdeq\mathsf{ev}(h,\mathsf{ev}(g\circ f,x)):D$}
\UnaryInfC{$\Gamma,x:A\vdash \mathsf{ev}((h\circ g)\circ f,x)\jdeq\mathsf{ev}(h\circ(g\circ f),x):D$}
\UnaryInfC{$\Gamma\vdash (h\circ g)\circ f\jdeq h\circ(g\circ f):A\to D$}
\end{prooftree}
\end{proof}

\begin{lem}\label{lem:fun_unit}
Composition of functions satisfies the left and right unit laws\index{left unit law|see {unit laws}}\index{right unit law|see {unit laws}}\index{unit laws!of function composition}, i.e.~we can derive
\begin{prooftree}
\AxiomC{$\Gamma\vdash f:A\to B$}
\UnaryInfC{$\Gamma\vdash \idfunc[B]\circ f\jdeq f:A\to B$}
\end{prooftree}
and
\begin{prooftree}
\AxiomC{$\Gamma\vdash f:A\to B$}
\UnaryInfC{$\Gamma\vdash f\circ\idfunc[A]\jdeq f:A\to B$}
\end{prooftree}
\end{lem}

\begin{proof}
The derivation for the left unit law is
\begin{prooftree}
\AxiomC{$\Gamma\vdash f:A\to B$}
\UnaryInfC{$\Gamma,x:A\vdash \mathsf{ev}(f,x):B$}
\AxiomC{$\Gamma\vdash B~\mathrm{type}$}
\UnaryInfC{$\Gamma,y:B\vdash \mathsf{ev}(\idfunc[B],y)\jdeq y:B$}
\UnaryInfC{$\Gamma,x:A,y:B\vdash \mathsf{ev}(\idfunc[B],y)\jdeq y:B$}
\BinaryInfC{$\Gamma,x:A\vdash \mathsf{ev}(\idfunc[B],\mathsf{ev}(f,x))\jdeq \mathsf{ev}(f,x):B$}
\UnaryInfC{$\Gamma,x:A\vdash \mathsf{ev}(\idfunc[B]\circ f,x)\jdeq \mathsf{ev}(f,x):B$}
\UnaryInfC{$\Gamma\vdash \idfunc[B]\circ f\jdeq f:A\to B$}
\end{prooftree}
The right unit law is left as \autoref{ex:fun_right_unit}.
\end{proof}

\section{The natural numbers}
The archetypal example of an inductive type is the type of \emph{natural numbers}. 
The type of \define{natural numbers}\index{natural numbers|see N@{$\N$}} is defined to be a closed type $\nat$\index{N@{$\N$}} equipped with closed terms for a \define{zero term}\index{zero term} and a \define{successor function}\index{successor function!of N@{of $\N$}}\index{function!successor on N@{successor on $\N$}}
\begin{equation*}
\zeroN:\N \qquad\text{and}\qquad \succN:\N\to\N,
\end{equation*}
To prove properties about the natural numbers, we postulate an \emph{induction principle}\index{induction principle!of N@{of $\N$}} for $\N$. In dependent type theory, however, the induction principle for the natural numbers provides a way to construct \emph{dependent functions} of types depending on the natural numbers. 

The \define{induction principle} for $\N$ states that for every type $P$ in context $\Gamma,n:\N$ one can infer
\begin{prooftree}
\def\fCenter{\Gamma}
\Axiom$\fCenter\ \vdash p_0:P(\zeroN)$
\noLine
\UnaryInf$\fCenter\ \vdash p_S:\prd{n:\N}P(n)\to P(\succN(n))$
\RightLabel{$\N{-}\mathrm{Ind}$}
\UnaryInf$\fCenter\ \vdash \ind{\N}(p_0,p_S):\prd{n:\N} P(n)$
\end{prooftree}
Furthermore we require that the dependent function $\ind{\N}(P,p_0,p_S)$ behaves as expected when it is applied to $\zeroN$ or a successor, i.e.~with the same hypotheses as for the induction principle we postulate the \define{computation rules}\index{computation rules!of N@{of $\N$}} for $\N$
\begin{prooftree}
\AxiomC{$\cdots$}
\RightLabel{$\N{-}\mathrm{Comp}(\zeroN)$}
\UnaryInfC{$\Gamma \vdash \ind{\N}(p_0,p_S,\zeroN)\jdeq p_0 : P(\zeroN)$}
\end{prooftree}
\begin{prooftree}
\AxiomC{$\cdots$}
\RightLabel{$\N{-}\mathrm{Comp}(\succN)$}
\UnaryInfC{$\Gamma, n:\N \vdash  \ind{\N}(p_0,p_S,\succN(n))\jdeq p_S(n,\ind{\N}(p_0,p_S,n)) : P(\succN(n))$}
\end{prooftree}

Using the induction principle of $\N$ we can perform many familiar constructions. 
For instance, we can define the \define{addition operation}\index{addition!on N@{on $\N$}}\index{function!addition on N@{addition on $\N$}}
\begin{equation*}
\addN:\N\to (\N\to\N)
\end{equation*}
by induction.

Informally, the definition of addition is as follows. By induction it suffices to construct a function $add_0 : \N\to\N$, and a function
\begin{align*}
add_S(f):\N\to\N,
\end{align*}
assuming $n:\N$ and $f:\N\to\N$. 
The function $add_0:\N\to\N$ is of course taken to be $\idfunc[\N]$, since it has to add nothing. Given a function $f:\N\to\N$ we define $add_S(f)$ to be $\lam{m}\succN(f(m))$, simply adding one to $f$ pointwise.

The derivation for the construction of $add_S$ looks as follows:
\begin{prooftree}
\AxiomC{}
\UnaryInfC{$\vdash\N~\mathrm{type}$}
\AxiomC{}
\UnaryInfC{$\vdash\N~\mathrm{type}$}
\AxiomC{}
\UnaryInfC{$\vdash\N~\mathrm{type}$}
\BinaryInfC{$f:\N^\N,m:\N\vdash f(m):\N$}
\AxiomC{}
\UnaryInfC{$\vdash \succN:\N\to\N$}
\UnaryInfC{$n:\N\vdash \succN(n):\N$}
\UnaryInfC{$f:\N^\N,m:\N,n:\N\vdash \succN(n):\N$}
\BinaryInfC{$f:\N^\N,m:\N\vdash \succN(f(m)):\N$}
\BinaryInfC{$n:\N,f:\N^\N,m:\N\vdash \succN(f(m)):\N$}
\UnaryInfC{$\vdash add_S\defeq \lam{n}{f}{m}\succN(f(m)):\N\to \N^\N \to \N^\N$}
%\BinaryInfC{$\vdash\addN:\ind{\N}(add_0,add_S):\N\to \N^\N$}
\end{prooftree}
We combine this derivation with the induction principle of $\N$ to complete the construction of addition:
\begin{prooftree}
\AxiomC{}
\UnaryInfC{$\vdash add_0\defeq \idfunc[\N]:\N^\N$}
\AxiomC{}
\UnaryInfC{$\vdash add_S:\N\to \N^\N \to \N^\N$}
\BinaryInfC{$\vdash\addN\jdeq\ind{\N}(add_0,add_S):\N\to \N^\N$}
\end{prooftree}
Usually we will write $n+m$ for $\addN(n,m)$. By the computation rules we have
\begin{align*}
\zeroN+m & \jdeq m \\
\succN(n)+m & \jdeq \succN(n+m)
\end{align*}
for any $n,m:\N$. 

\begin{rmk}
The rules that we provided so far are not sufficient to also conclude that $n+\zeroN\jdeq n$ and $n+ \succN(m)\jdeq \succN(n+m)$. However, once we have introduced the \emph{identity type} we will nevertheless be able to \emph{identify} $n+\zeroN$ with $n$, and $n+ \succN(m)$ with $\succN(n+m)$. 
\end{rmk}

\begin{rmk}
  We annotate the terms $\zeroN$ and $\succN$ of type $\N$ with their type in the subscript, to remind ourselves that $\zeroN$ and $\succN$ are declared to be terms of type $\N$, and not of any other type. In the next chapter we will introduce the type $\Z$ of the integers, on which we can also define a zero term $\zeroZ$, and a successor function $\succZ$. These should be distinguished from the terms $\zeroN$ and $\succN$. In general, we will make sure that every term is given a unique name. In libraries of mathematics formalized in a computer proof assistant it is also the case that every type must be given a unique name.
  
  One difference between set theory and type theory is that every well-formed term is specified along with its type and with its context. One way of looking at this is that there are three sorts in type theory: contexts, types, and terms. On the other hand, there is only one sort in set theory: sets! Sets are governed by the elementhood relation: the formula $x\in y$ is a well-formed formula of set theory for any two sets $x$ and $y$. In particular, set theory is a theory stated in first order logic.

  In type theory we only have sentences of the form $\Gamma \vdash a : A$, asserting that in context $\Gamma$, $a$ is a term of type $A$. Type theory does not make use of an ambient first order logic. Instead, type theory is its own system of inference rules, as we have seen in \cref{ch:dtt}.
\end{rmk}

\begin{exercises}
\item \label{ex:fun_right_unit}Give a derivation for the right unit law of \autoref{lem:fun_unit}.\index{unit laws!for function composition}
\item In this exercise we generalize the composition operation of non-dependent function types\index{composition!of dependent functions}:
\begin{subexenum}
\item Define a composition operation for dependent function types
\begin{prooftree}
\AxiomC{$\Gamma\vdash f:\prd{x:A}B(x)$}
\AxiomC{$\Gamma,x:A\vdash g:\prd{y:B} C(x,y)$}
\BinaryInfC{$\Gamma\vdash g\circ f:\prd{x:A} C(x,\mathsf{ev}(f,x))$}
\end{prooftree}
and show that this operation agrees with ordinary composition when it is specialized to non-dependent function types.
\item Show that composition of dependent functions is associative.\index{associativity!of dependent function composition}\index{composition!of dependent functions!associativity}
\item Show that composition of dependent functions satisfies the right unit law\index{unit laws!dependent function composition}:
\begin{prooftree}
\AxiomC{$\Gamma\vdash f:\prd{x:A}B(x)$}
\UnaryInfC{$\Gamma\vdash f\circ\idfunc[A]\jdeq f :\prd{x:A}B(x)$}
\end{prooftree}
\item Show that composition of dependent functions satisfies the left unit law\index{unit laws!dependent function composition}:
\begin{prooftree}
\AxiomC{$\Gamma\vdash f:\prd{x:A}B(x)$}
\UnaryInfC{$\Gamma\vdash \idfunc[B]\circ f\jdeq f:\prd{x:A}B(x)$}
\end{prooftree}
\end{subexenum}
\item 
\begin{subexenum}
\item Construct the \define{constant function}\index{constant function}\index{function!constant function}
\begin{prooftree}
\AxiomC{$\Gamma\vdash A~\textrm{type}$}
\UnaryInfC{$\Gamma,y:B\vdash \mathsf{const}_y:A\to B$}
\end{prooftree}
\item Show that
\begin{prooftree}
\AxiomC{$\Gamma\vdash f:A\to B$}
\UnaryInfC{$\Gamma,z:C\vdash \mathsf{const}_z\circ f\jdeq\mathsf{const}_z : A\to C$}
\end{prooftree}
\item Show that
\begin{prooftree}
\AxiomC{$\Gamma\vdash A~\textrm{type}$}
\AxiomC{$\Gamma\vdash g:B\to C$}
\BinaryInfC{$\Gamma,y:B\vdash g\circ\mathsf{const}_y\jdeq \mathsf{const}_{\mathsf{ev}(g,y)}:A\to C$}
\end{prooftree}
\end{subexenum}
\item \label{ex:swap}
\begin{subexenum}
\item Given two types $A$ and $B$ in context $\Gamma$, and a type $C$ in context $\Gamma,x:A,y:B$, define the \define{swap function}\index{function!swap}\index{swap function}
\begin{equation*}
\Gamma\vdash \sigma:\Big(\prd{x:A}{y:B}C(x,y)\Big)\to\Big(\prd{y:B}{x:A}C(x,y)\Big)
\end{equation*}
that swaps the order of the arguments.
\item Show that
\begin{equation*}
\Gamma\vdash \sigma\circ\sigma\jdeq\idfunc:\Big(\prd{x:A}{y:B}C(x,y)\Big)\to \Big(\prd{x:A}{y:B}C(x,y)\Big).
\end{equation*}
\end{subexenum}
\item 
\begin{subexenum}
\item Define the \define{multiplication}\index{multiplication!on N@{on $\N$}}\index{function!multiplication on N@{multiplication on $\N$}} operation $\mathsf{mul}:\N\to(\N\to\N)$.
\item Define the \define{power}\index{power function on N@{power function on $\N$}}\index{function!power function on N@{power function on $\N$}} operation $n,m\mapsto m^n$ of type $\N\to (\N\to \N)$.
\item Define the \define{factorial}\index{factorial function}\index{function!factorial function} function $n\mapsto n!$. 
\end{subexenum}
\item Define the binary $\min$ and $\max$ functions $\mathsf{min},\mathsf{max}:\N\to(\N\to\N)$.\index{minimum function}\index{maximum function}\index{function!min}\index{function!max}
\item Construct a function
\begin{equation*}
\indN:P(\zeroN)\to \Big(\Big(\prd{n:\nat}P(n)\to P(\succN(n))\Big)\to \prd{n:\nat}P(n)\Big),
\end{equation*}
in context $\Gamma$, for every type $P$ in context $\Gamma,n:\N$, and show that the computation rules
\begin{align*}
\indN(p_0,p_S,\zeroN) & \jdeq p_0 \\
\indN(p_0,p_S,\succN(n)) & \jdeq p_S(n,\ind{\nat}(p_0,p_S,n))
\end{align*}
hold. \emph{Note: this is more an exercise in type theoretical bookkeeping than an exercise about the natural numbers.}
\end{exercises}
