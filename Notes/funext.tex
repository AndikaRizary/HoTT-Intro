\chapter{Function extensionality}

A significant part of the development of homotopy type theory involves answering the following basic questions:
\begin{enumerate}
\item What is the identity type of a given type $A$?
\item What is the total space of a type family $B$ over $A$?
\item What are the fibers of a given map $f:A\to B$?
\item What does transporting a point in a given type family $B$ over $A$ do?
\end{enumerate}
We have already characterized the identity types of $\Sigma$-types as a $\Sigma$-type of identity types (\cref{thm:eq_sigma}), of $\emptyt$ and $\unit$ since both are propositions (\cref{eg:prop_contr}), of natural numbers $\N$ as observational equality (\cref{thm:eq_nat}), and also of $\bool$ as observational equality (\cref{ex:eq_bool}). In this section we will discuss the identity type of $\Pi$-types.

\section{Equivalent forms of function extensionality}
\begin{thm}\label{thm:funext_wkfunext}
The following are equivalent:\index{function extensionality}
\begin{enumerate}
\item The principle of \define{homotopy induction}\index{homotopy induction}: for every $f:\prd{x:A}B(x)$ and every type family
\begin{equation*}
\Gamma,g:\prd{x:A}B(x), H:f\htpy g \vdash P(g,H)~\mathrm{type},
\end{equation*}
the map
\begin{equation*}
\Big(\prd{g:\prd{x:A}B(x)}{H:f\htpy g}P(g,H)\Big)\to P(f,\mathsf{htpy\usc{}refl}_f)
\end{equation*}
given by $s\mapsto s(f,\mathsf{htpy\usc{}refl}_f)$ has a section.
\item The \define{function extensionality principle}\index{function extensionality}: For every type family $B$ over $A$, and any two dependent functions $f,g:\prd{x:A}B(x)$, the canonical map\index{htpy_eq@{$\mathsf{htpy\usc{}eq}$}|textbf}
\begin{equation*}
\mathsf{htpy\usc{}eq}(f,g) : (\id{f}{g})\to (f\htpy g)
\end{equation*}
by path induction (sending $\refl{f}$ to $\lam{x}\refl{f(x)}$) is an equivalence. We will write $\mathsf{eq\usc{}htpy}$\index{eq_htpy@{$\mathsf{eq\usc{}htpy}$}} for its inverse.
\item The \define{weak function extensionality principle}\index{weak function extensionality} holds: For every type family $B$ over $A$ one has\index{contractible!weak function extensionality}
\begin{equation*}
\Big(\prd{x:A}\iscontr(B(x))\Big)\to\iscontr\Big(\prd{x:A}B(x)\Big).
\end{equation*}
\end{enumerate}
\end{thm}

\begin{proof}
To show that homotopy induction implies function extensionality, note that from homotopy induction and $\Sigma$-induction combined, we obtain that for any type family $P$ over $\sm{g:\prd{x:A}B(x)}f\htpy g$, the evaluation map
\begin{equation*}
\Big(\prd{t:\sm{g:\prd{x:A}B(x)}f\htpy g} P(t)\Big)\to P((f,\mathsf{htpy\usc{}refl}_f))
\end{equation*}
has a section. In other words, the type $\sm{g:\prd{x:A}B(x)}f\htpy g$ satisfies singleton induction and therefore it is contractible. Now we conclude by \cref{thm:id_fundamental} that the canonical map $(f=g)\to (f\htpy g)$ is an equivalence, i.e.~that function extensionality holds.

Conversely, to prove homotopy induction from function extensionality we again note that by function extensionality and \cref{thm:id_fundamental} the type 
\begin{equation*}
\sm{g:\prd{x:A}B(x)}f\htpy g
\end{equation*}
is contractible. Therefore it satisfies singleton induction, so homotopy induction follows.

To show that function extensionality implies weak function extensionality, suppose that each $B(a)$ is contractible with center of contraction $c(a)$ and contraction $C_a:\prd{y:B(a)}c(a)=y$. Then we take $c\defeq \lam{a}c(a)$ to be the center of contraction of $\prd{x:A}B(x)$. To construct the contraction we have to define a term of type
\begin{equation*}
\prd{f:\prd{x:A}B(x)}c=f.
\end{equation*}
Let $f:\prd{x:A}B(x)$. By function extensionality we have a map $(c\htpy f)\to (c=f)$, so it suffices to construct a term of type $c\htpy f$. Here we take $\lam{a}C_a(f(a))$. This completes the proof that function extensionality implies weak function extensionality.

To prove function extensionality from weak function extensionality, observe that it suffices by \autoref{thm:id_fundamental} to show that
\begin{equation*}
\sm{g:\prd{x:A}B(x)}f\htpy g
\end{equation*}
is contractible.

Since the type $\sm{b:B(x)}f(x)=b$ is contractible for each $x:X$, it follows by our assumption of weak function extensionality that the type $\prd{x:A}\sm{b:B(x)}f(x)=b$ is contractible. By \autoref{ex:contr_retr} it therefore suffices to show that
\begin{equation*}
\sm{g:\prd{x:A}B(x)}f\htpy g
\end{equation*}
is a retract of the type $\prd{x:A}\sm{b:B(x)}f(x)=b$. We have the functions
\begin{align*}
i & \defeq \ind{\Sigma}(\lam{g}{H}{x}\pairr{g(x),H(x)}) \\
r & \defeq \lam{p}\pairr{\lam{x}\proj 1(p(x)),\lam{x}\proj 2(p(x))}.
\end{align*}
It remains to show that $r\circ i\htpy \idfunc$. This homotopy is constructed by $\Sigma$-induction. Let $g:\prd{x:A}B(x)$ and let $H:f\htpy g$. Then we have
\begin{align*}
r(i(g,H)) & \jdeq r(\lam{x}\pairr{g(x),H(x)}) \\
& \jdeq \pairr{\lam{x}g(x),\lam{x}H(x)} \\
& \jdeq \pairr{g,H}.
\end{align*}
In other words, the homotopy $r\circ i\htpy \idfunc$ is given by $\ind{\Sigma}(\lam{g}{H}\refl{(\pairr{g,H})})$. 
\end{proof}

\begin{comment}
\begin{rmk}
Since we assumed the $\eta$-rule for $\Sigma$-types, we also have
\begin{align*}
\mathsf{pi\usc{}sigma}(\mathsf{sigma\usc{}pi}(p)) & \jdeq \mathsf{pi\usc{}sigma}(\pairr{\lam{x}\proj 1(p(x)),\lam{x}\proj 2(p(x))}) \\
& \jdeq \lam{x}\pairr{\proj 1(p(x)),\proj 2(p(x))} \\
& \jdeq \lam{x} p(x) \\
& \jdeq p.
\end{align*}
Therefore, the types $\sum_g f\htpy g$ and $\prod_x\sum_b f(x)=b$ are actually \emph{judgmentally isomorphic}. 
\end{rmk}
\end{comment}

\begin{comment}
\begin{thm}
Assume weak function extensionality. Then for any type $A$, the type $\iscontr(A)$ is a proposition.
\end{thm}

\begin{proof}
Let $A$ be a contractible type with center of contraction $c$, and contraction $C$. We want to show that
\begin{equation*}
\prd{d:A}{D:\prd{x:A}d=x} (c,C)=(d,D).
\end{equation*}
Since $A$ is contractible, we may proceed by singleton induction. Thus, it suffices to show that
\begin{equation*}
\prd{D:\prd{x:A}c=x}(c,C)=(c,D).
\end{equation*}
Now recall from \cref{ex:prop_contr} it follows that for any $x:A$ the type $c=x$ is contractible. By the principle of weak function extensionality it follows that the type $\prd{x:A}c=x$ is contractible. Therefore we may proceed by singleton induction on $D$. It suffices to show that
\begin{equation*}
(c,C)=(c,C),
\end{equation*}
which we have by reflexivity.
\end{proof}
\end{comment}

\begin{thm}\label{thm:trunc_pi}
Assume function extensionality. Then for any type family $B$ over $A$ one has\index{truncated}
\begin{equation*}
\Big(\prd{x:A}\istrunc{k}(B(x))\Big)\to \istrunc{k}\Big(\prd{x:A}B(x)\Big).
\end{equation*}
\end{thm}

\begin{proof}
The theorem is proven by induction on $k\geq -2$. The base case is just the weak function extensionality principle, which was shown to follow from function extensionality in \autoref{thm:funext_wkfunext}.

For the inductive hypothesis, assume that the $k$-types are closed under dependent function types. Assume that $B$ is a family of $(k+1)$-types. By function extensionality, the type $f=g$ is equivalent to $f\htpy g$ for any two dependent functions $f,g:\prd{x:A}B(x)$. Now observe that $f\htpy g$ is a dependent product of $k$-types, and therefore it is an $k$-type by our inductive hypotheses. Therefore, it follows by \autoref{thm:ktype_eqv} that $f=g$ is an $k$-type, and hence that $\prd{x:A}B(x)$ is an $(k+1)$-type.
\end{proof}

\begin{cor}\label{cor:funtype_trunc}
Suppose $B$ is a $k$-type. Then $A\to B$ is also a $k$-type, for any type $A$.
\end{cor}

The following theorem is sometimes called the \define{type theoretic principle of choice}. This terminology comes from the point of view of propositions as types, where the $\Sigma$-type has the role of the existential quantifier, and the $\Pi$-type has the role of the universal quantifier.

\begin{thm}\label{thm:choice}
Let $C(x,y)$ be a type in context $\Gamma,x:A,y:B(x)$. Then the map
\begin{equation*}
\varphi:\Big(\prd{x:A}\sm{y:B(x)}C(x,y)\Big)\to \Big(\sm{f:\prd{x:A}B(x)}\prd{x:A}C(x,f(x))\Big)
\end{equation*}
given by $\lam{h}(\lam{x}\proj 1(h(x)),\lam{x}\proj 2(h(x)))$ is an equivalence.
\end{thm}

\begin{proof}
The map $\psi$ in the converse direction is defined by
\begin{equation*}
\psi\defeq \ind{\Sigma}(\lam{f}{g}{x}(f(x),g(x))).
\end{equation*}
We need to define homotopies $\psi\circ\varphi\htpy \idfunc$ and $\varphi\circ\psi\htpy \idfunc$. 

For the first homotopy, let $h:\prd{x:A}\sm{y:B(x)}C(x,y)$. Then we have
\begin{align*}
\psi(\varphi(h)) & \jdeq \psi(\lam{x}\proj 1(h(x)),\lam{x}\proj 2(h(x))) \\
& \jdeq \lam{x}(\proj 1(h(x)),\proj 2(h(x))).
\end{align*}
Note that for each $x:A$ we have an identification
\begin{equation*}
(\proj 1(h(x)),\proj 2(h(x)))=h(x).
\end{equation*}
Therefore we obtain a \emph{homotopy} $\psi(\varphi(h))\htpy h$, but this suffices because function extensionality now provides us with an identification $\psi(\varphi(h))=h$.

The second homotopy is constructed by $\Sigma$-induction. Let $f:\prd{x:A}B(x)$ and let $g:\prd{x:A}C(x,f(x))$. 
Then we have
\begin{align*}
\varphi(\psi(f,g)) & \jdeq \varphi(\lam{x}(f(x),g(x))) \\
& \jdeq (\lam{x}f(x),\lam{x}g(x)) \\
& \jdeq (f,g)
\end{align*}
where the last judgmental equality holds by the $\eta$-rule for $\Pi$-types. In other words, the homotopy $\varphi\circ\psi\htpy \idfunc$ is given by
\begin{equation*}
\ind{\Sigma}(\lam{f}{g}\refl{(f,g)}).\qedhere
\end{equation*}
\end{proof}

\begin{cor}
For type $A$ and any type family $C$ over $B$, the map
\begin{equation*}
\Big(\sm{f:A\to B} \prd{x:A}C(f(x))\Big)\to\Big(A\to\sm{y:B}C(x)\Big)
\end{equation*}
given by $\lam{(f,g)}{x}(f(x),g(x))$ is an equivalence.
\end{cor}

\section{Universal properties}
The function extensionality principle allows us to prove \emph{universal properties}: characterizations of all maps out of (or into) a given type. Universal properties characterize a type up to equivalence. In the following theorem we prove the universal property of dependent pair types.

\begin{thm}
Let $B$ be a type family over $A$, and let $X$ be a type. Then the map
\begin{equation*}
\mathsf{ev\usc{}pair}:\Big(\Big(\sm{x:A}B(x)\Big)\to X\Big)\to \Big(\prd{x:A}(B(x)\to X)\Big)
\end{equation*}
given by $f\mapsto\lam{a}{b}f((a,b))$ is an equivalence.
\end{thm}

This theorem justifies that we usually write $f(a,b)$ rather than $f((a,b))$, since $f:\big(\sm{x:A}B(x)\big)\to X$ is uniquely determined by its action on terms of the form $(a,b)$.

\begin{proof}
The map in the converse direction is simply
\begin{equation*}
\ind{\Sigma} : \Big(\prd{x:A}(B(x)\to X)\Big)\to \Big(\Big(\sm{x:A}B(x)\Big)\to X\Big).
\end{equation*}
By the computation rules for $\Sigma$-types we have
\begin{equation*}
\lam{f}\refl{f}:\mathsf{ev\usc{}pair}\circ\ind{\Sigma}\htpy\idfunc
\end{equation*}

To show that $\ind{\Sigma}\circ\mathsf{ev\usc{}pair}\htpy\idfunc$ we will also apply function extensionality. Thus, it suffices to show that $\ind{\Sigma}(\lam{x}{y}f((x,y)))=f$. We apply function extensionality again, so it suffices to show that
\begin{equation*}
\prd{t:\sm{x:A}B(x)}\ind{\Sigma}\big(\lam{x}{y}f((x,y))\big)(t)=f(t).
\end{equation*}
We obtain this homotopy by another application of $\Sigma$-induction. 
\end{proof}

\begin{cor}\label{cor:times_up_out}
Let $A$, $B$, and $X$ be types. Then the map
\begin{equation*}
\mathsf{ev\usc{}pair}: (A\times B \to X)\to (A\to (B\to X))
\end{equation*}
given by $f\mapsto\lam{a}{b}f((a,b))$ is an equivalence.
\end{cor}

The universal property of identity types is sometimes called the \emph{type theoretical Yoneda lemma}: families of maps out of the identity type are uniquely determined by their action on the reflexivity identification.

\begin{thm}\label{thm:yoneda}
Let $B$ be a type family over $A$, and let $a:A$. Then the map
\begin{equation*}
\mathsf{ev\usc{}refl}:\Big(\prd{x:A} (a=x)\to B(x)\Big)\to B(a)
\end{equation*}
given by $\lam{f} f(a,\refl{a})$ is an equivalence. 
\end{thm}

\begin{proof}
The inverse $\varphi$ is defined by path induction, taking $b:B(a)$ to the function $f$ satisfying $f(a,\refl{a})\jdeq b$. It is immediate that $\mathsf{ev\usc{}refl}\circ\varphi\htpy \idfunc$.

To see that $\varphi\circ \mathsf{ev\usc{}refl}\htpy\idfunc$, let $f:\prd{x:A}(a=x)\to B(x)$. To show that $\varphi(f(a,\refl{a}))=f$ we use function extensionality (twice), so it suffices to show that
\begin{equation*}
\prd{x:A}{p:a=x} \varphi(f(a,\refl{a}),x,p)=f(x,p).
\end{equation*}
This follows by path induction on $p$, since $\varphi(f(a,\refl{a}),a,\refl{a})\jdeq f(a,\refl{a})$.
\end{proof}

\section{Composing with equivalences}

We show in this section that a map $f:A\to B$ is an equivalence if and only if for any type $X$ the precomposition map 
\begin{equation*}
\blank\circ f: (B\to X)\to (A\to X)
\end{equation*}
is an equivalence. Moreover, we will show in \cref{ex:equiv_precomp} that the `dependent version' of this statement also holds: a map $f:A\to B$ is an equivalence if and only if for any type family $P$ over $B$, the precomposition map
\begin{equation*}
\blank\circ f: \Big(\prd{y:B}P(y)\Big)\to\Big(\prd{x:A}P(f(x))\Big)
\end{equation*}
is an equivalence.

\begin{thm}\label{ex:equiv_precomp}
For any map $f:A\to B$, the following are equivalent:
\begin{enumerate}
\item $f$ is an equivalence.
\item For any type family $P$ over $B$ the map
\begin{equation*}
\Big(\prd{y:B}P(y)\Big)\to\Big(\prd{x:A}P(f(x))\Big)
\end{equation*}
given by $s\mapsto s\circ f$ is an equivalence.
\item For any type $X$ the map
\begin{equation*}
(B\to X)\to (A\to X)
\end{equation*}
given by $g\mapsto g\circ f$ is an equivalence. 
\end{enumerate}
\end{thm}

\begin{proof}
To show that (i) implies (ii), we first recall from \cref{thm:half_adj} that any equivalence is also a half-adjoint equivalence, so $f$ comes equipped with
\begin{align*}
g & : B \to A\\
G & : f\circ g \htpy \idfunc[B] \\
H & : g\circ f \htpy \idfunc[A] \\
K & : G\cdot f \htpy f\cdot H.
\end{align*}
Then we define the inverse of $\blank\circ f$ to be the map
\begin{equation*}
\varphi:\Big(\prd{x:A}P(f(x))\Big)\to\Big(\prd{y:B}P(y)\Big)
\end{equation*}
given by $s\mapsto \lam{y}\mathsf{tr}_P(G(y),sg(y))$. 

To see that $\varphi$ is a section of $\blank\circ f$, let $s:\prd{x:A}P(f(x))$. By function extensionality it suffices to construct a homotopy $\varphi(s)\circ f\htpy s$. In other words, we have to show that
\begin{equation*}
\mathsf{tr}_P(G(f(x)),s(g(f(x)))=s(x)
\end{equation*}
for any $x:A$. Now we use the additional homotopy $K$ from our assumption that $f$ is a half-adjoint equivalence. Since we have $K(x):G(f(x))=\ap{f}{H(x)}$ it suffices to show that
\begin{equation*}
\mathsf{tr}_P(\ap{f}{H(x)},sgf(x))=s(x).
\end{equation*}
A simple path-induction argument yields that
\begin{equation*}
\mathsf{tr}_P(\ap{f}{p})\htpy \mathsf{tr}_{P\circ f}(p)
\end{equation*}
for any path $p:x=y$ in $A$, so it suffices to construct an identification
\begin{equation*}
\mathsf{tr}_{P\circ f}(H(x),sgf(x))=s(x).
\end{equation*}
We have such an identification by $\apd{s}{H(x)}$.

To see that $\varphi$ is a retraction of $\blank\circ f$, let $s:\prd{y:B}P(y)$. By function extensionality it suffices to construct a homotopy $\varphi(s\circ f)\htpy s$. In other words, we have to show that
\begin{equation*}
\mathsf{tr}_P(G(y),sfg(y))=s(y)
\end{equation*}
for any $y:B$. We have such an identification by $\apd{s}{G(y)}$. This completes the proof that (iii) implies (iv).

Note that (v) is an immediate consequence of (iv), since we can just choose $P$ to be the constant family $X$.

It remains to show that (v) implies (i). Suppose that
\begin{equation*}
\blank\circ f:(B\to X)\to (A\to X)
\end{equation*}
is an equivalence for every type $X$. Then its fibers are contractible by \cref{thm:contr_equiv}. In particular, choosing $X\jdeq A$ we see that the fiber
\begin{equation*}
\fib{\blank\circ f}{\idfunc[A]}\jdeq \sm{h:B\to A}h\circ f=\idfunc[A]
\end{equation*}
is contractible. Thus we obtain a function $h:B\to A$ and a homotopy $H:h\circ f\htpy\idfunc[A]$ showing that $h$ is a retraction of $f$. We will show that $h$ is also a section of $f$. To see this, we use that the fiber
\begin{equation*}
\fib{\blank\circ f}{f}\jdeq \sm{i:B\to B} i\circ f=f
\end{equation*}
is contractible (choosing $X\jdeq B$). 
Of course we have $(\idfunc[B],\refl{f})$ in this fiber. However we claim that there also is an identification $p:(f\circ h)\circ f=f$, showing that $(f\circ h,p)$ is in this fiber, because
\begin{align*}
(f\circ h)\circ f & \jdeq f\circ (h\circ f) \\
& = f\circ \idfunc[A] \\
& \jdeq f
\end{align*}
Now we conclude by the contractibility of the fiber that $(\idfunc[B],\refl{f})=(f\circ h,p)$. In particular we obtain that $\idfunc[B]=f\circ h$, showing that $h$ is a section of $f$.
\end{proof}

\begin{exercises}
\item Show that the functions
\begin{align*}
\mathsf{htpy\usc{}inv} & : (f \htpy g) \to (g \htpy f) \\
\mathsf{htpy\usc{}concat}(H) & : (g \htpy h) \to (f \htpy h) \\
\mathsf{htpy\usc{}concat}'(K) & : (f \htpy g) \to (f \htpy h)
\end{align*}
are equivalences for every $f,g,h : \prd{x:A}B(x)$. Here, $\mathsf{htpy\usc{}concat}'(K)$ is the function defined by $H\mapsto \ct{H}{K}$.
\item \label{ex:isprop_istrunc}
\begin{subexenum}
\item Show that for any type $A$ the type $\iscontr(A)$ is a proposition. %There's an easy proof using double singleton induction. This is a nice application of weak funext.
\item Show that for any type $A$ and any $k\geq-2$, the type $\istrunc{k}(A)$ is a proposition.
\end{subexenum}
\item \label{lem:postcomp_equiv}
Let $f:X\to Y$ be a map. Show that the following are equivalent:
\begin{enumerate}
\item $f$ is an equivalence.\index{equivalence!post-composition|textit}
\item For any type $A$, the map $f\circ\blank : X^A\to Y^A$ is an equivalence.
\end{enumerate}
\item \label{ex:isprop_isequiv}Let $f:A\to B$ be a function.
\begin{subexenum}
\item Show that if $f$ is an equivalence, then the type $\sm{g:B\to A}f\circ g\htpy \idfunc$ of sections of $f$ is contractible.
\item Show that if $f$ is an equivalence, then the type $\sm{h:B\to A}h\circ f\htpy \idfunc$ of retractions of $f$ is contractible.
\item Show that $\isequiv(f)$ is a proposition.
\item Use \cref{ex:prop_equiv,ex:isprop_istrunc} to show that $\isequiv(f)\eqvsym \iscontr(f)$.
\end{subexenum}
Conclude that $\eqv{A}{B}$ is a subtype of $A\to B$, and in particular that the map $\proj 1 : (\eqv{A}{B})\to (A\to B)$ is an embedding.
\item \label{ex:prop_equiv}
\begin{subexenum}
\item Let $P$ and $Q$ be propositions. Show that
\begin{equation*}
\eqv{(P\leftrightarrow Q)}{(\eqv{P}{Q})}.
\end{equation*}
\item Show that $P$ is a proposition if and only if $P\to P$ is contractible.
\end{subexenum}
\item Show that $\mathsf{path\usc{}split}(f)$ and $\mathsf{half\usc{}adj}(f)$ are propositions for any map $f:A\to B$. 
%\item Let $B$ and $C$ be type families over $A$, suppose that $p:\id{a}{a'}$ in $A$, and consider two functions $f:B(a)\to C(a)$ and $g:B(a')\to C(a')$.
%\begin{subexenum}
%\item Show that the square
%\begin{equation*}
%\begin{tikzcd}
%B(a) \arrow[r,"f"] \arrow[d,swap,"\mathsf{tr}_B(p)"] & C(a) \arrow[d,"\mathsf{tr}_C(p)"] \\
%B(a') \arrow[r,swap,"g"] & C(a')
%\end{tikzcd}
%\end{equation*}
%commutes for every homotopy $H:\mathsf{tr}_{B(x)\to C(x)}(p,f)\htpy g$. In other words, construct a function of type
%\begin{equation*} 
%\Big(\mathsf{tr}_{B(x)\to C(x)}(p,f)\htpy g\Big)\to \Big(\mathsf{tr}_C(p)\circ f\htpy g\circ \mathsf{tr}_B(p)\Big)
%\end{equation*}
%\item Show that this map is an equivalence.
%\end{subexenum}
\item \label{ex:idfunc_autohtpy}Construct for any type $A$ an equivalence
\begin{equation*}
\eqv{\mathsf{is\usc{}invertible}(\idfunc[A])}{\Big(\idfunc[A]\htpy\idfunc[A]\Big)}.
\end{equation*}
\item 
\begin{subexenum}
\item Show that the map
\begin{equation*}
(\emptyt \to X)\to \unit
\end{equation*}
given by $\lam{f}\ttt$ is an equivalence. 
\item Show that any type $Y$ satisfying the property that the map 
\begin{equation*}
\lam{f}\ttt: (Y\to X)\to \unit
\end{equation*} 
is an equivalence for any $X$, is itself equivalent to $\emptyt$.
\end{subexenum}
\item 
\begin{subexenum}
\item Show that the map
\begin{equation*}
(A+B\to X)\to \Big((A\to X)\times (B\to X)\Big)
\end{equation*}
given by $f\mapsto (f\circ\inl,f\circ\inr)$ is an equivalence.
\item Show that any type $Y$ that is equipped with functions $i:A\to Y$ and $j:B\to Y$ and satisfies the condition that the map
\begin{equation*}
\lam{f}(f\circ i,f\circ j): (Y\to X)\to \Big((A\to X)\times (B\to X)\Big)
\end{equation*}
is an equivalence for any $X$, is itself equivalent to $A+B$.
\end{subexenum}
\item 
\begin{subexenum}
\item Show that the map
\begin{equation*}
(\unit\to X)\to X
\end{equation*}
given by $\lam{f}f(\ttt)$ is an equivalence. 
\item Show that a type $C$ is contractible if and only if it comes equipped with a term $c:C$ and satisfies the condition that the map
\begin{equation*}
\lam{f}f(c) :(C\to X)\to X
\end{equation*}
is an equivalence for any $X$.
\end{subexenum}
\item \label{ex:sec_retr}Consider a commuting triangle 
\begin{equation*}
\begin{tikzcd}[column sep=tiny]
A \arrow[rr,"h"] \arrow[dr,swap,"f"] & & B \arrow[dl,"g"] \\
& X
\end{tikzcd}
\end{equation*}
with $H:f\htpy g\circ h$.
\begin{subexenum}
\item Show that if $h$ has a section, then $\mathsf{sec}(g)$ is a retract of $\mathsf{sec}(f)$.
\item Show that if $g$ has a retraction, then $\mathsf{retr}(h)$ is a retract of $\mathsf{sec}(f)$.
\end{subexenum}
\item \label{ex:equiv_pi}Let $e_i:\eqv{A_i}{B_i}$ be an equivalence for every $i:I$. Show that the map
\begin{equation*}
\lam{f}{i}e_i\circ f:\Big(\prd{i:I}A_i\Big)\to\Big(\prd{i:I}B_i\Big)
\end{equation*}
is an equivalence.
\item \label{ex:triangle_fib}Consider a diagram of the form
\begin{equation*}
\begin{tikzcd}[column sep=tiny]
A \arrow[dr,swap,"f"] & & B \arrow[dl,"g"] \\
& X
\end{tikzcd}
\end{equation*}
\begin{subexenum}
\item Show that the type $\sm{h:A\to B} f\htpy g\circ h$ is equivalent to the type of fiberwise transformations
\begin{equation*}
\prd{x:X}\fib{f}{x}\to\fib{g}{x}.
\end{equation*}
\item Show that the type $\sm{h:\eqv{A}{B}} f\htpy g\circ h$ is equivalent to the type of fiberwise equivalences
\begin{equation*}
\prd{x:X}\fib{f}{x}\eqvsym\fib{g}{x}.
\end{equation*}
\end{subexenum}
\item \label{ex:sq_fib}Consider a diagram of the form
\begin{equation*}
\begin{tikzcd}
A \arrow[d,swap,"f"] & B \arrow[d,"g"] \\
X \arrow[r,swap,"h"] & Y.
\end{tikzcd}
\end{equation*}
Show that the type $\sm{i:A\to B}h\circ f\htpy g\circ i$ is equivalent to the type of fiberwise transformations
\begin{equation*}
\prd{x:X}\fib{f}{x}\to\fib{g}{h(x)}.
\end{equation*}
%\item Show that the type $\sm{i:\eqv{A}{B}}h\circ f\htpy g\circ i$ is equivalent to the type of fiberwise equivalences
%\begin{equation*}
%\prd{x:X}\fib{f}{x}\eqvsym\fib{g}{h(x)}.
%\end{equation*}
\item \label{ex:iso_equiv}Let $A$ and $B$ be sets. Show that type type $\eqv{A}{B}$ of equivalences from $A$ to $B$ is equivalent to the type $A\cong B$ of \define{isomorphisms} from $A$ to $B$, i.e. the type of quadruples $(f,g,H,K)$ consisting of
\begin{align*}
f & : A\to B \\
g & : B\to A \\
H & : f\circ g = \idfunc[B] \\
K & : g\circ f = \idfunc[A].
\end{align*}
\end{exercises}
