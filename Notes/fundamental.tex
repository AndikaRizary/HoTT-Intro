% !TEX root = hott_intro.tex

\chapter{The fundamental theorem of identity types}\label{chap:fundamental}
\chaptermark{The fundamental theorem}

In many situations in homotopy type theory it is important to know what the identity types are. For example, we have used a characterization of the identity types of the fibers of a map in order to conclude that any equivalence is a contractible map. Therefore it is a routine task to give for any given type of interest, a characterization of its identity type. The fundamental theorem of identity types is our main tool to carry out such characterizations. To name a few applications, we will show in \cref{thm:eq_nat} that the identity type of the natural numbers is equivalent to its observational equality, and we will show in \cref{thm:eq-circle} that the loop space of the circle is equivalent to $\Z$.

In order to prove the fundamental theorem of identity types, we first need the basic fact that a family of maps is a family of equivalences if and only if it induces an equivalence on total spaces. This fact will also be used in many other situations, most notably in the characterization of pullback squares in \cref{cor:pb_fibequiv}.

Our first application of the fundamental theorem of identity types in the present lecture is a simple proof that any equivalence is an embedding. Embeddings are maps that induce equivalences on identity types, i.e., they are the homotopical analogue of injective maps. In our second application we characterize the identity types of coproducts.

\section{Families of equivalences}
Consider a family
\begin{equation*}
f : \prd{x:A}B(x)\to C(x)
\end{equation*}
of maps.

\begin{defn}
We define the map
\begin{equation*}
\total{f}:\sm{x:A}B(x)\to\sm{x:A}C(x).
\end{equation*}
by $\lam{(x,y)}(x,f(x,y))$.
\end{defn}

\begin{lem}\label{lem:fib_total}
  For any family of maps $f:\prd{x:A}B(x)\to C(x)$ and any $t:\sm{x:A}C(x)$,
  there is an equivalence
  \begin{equation*}
    \eqv{\fib{\total{f}}{t}}{\fib{f(\proj 1(t))}{\proj 2(t)}}.
  \end{equation*}
\end{lem}

\begin{proof}
  For any $p:\fib{\total{f}}{t}$ we define $\varphi(t,p):\fib{\proj 1(t)}{\proj 2(t)}$ by $\Sigma$-induction on $p$. Therefore it suffices to define $\varphi(t,(s,\alpha)):\fib{\proj 1(t)}{\proj 2 (t)}$ for any $s:\sm{x:A}B(x)$ and $\alpha:\total{f}(s)=t$. Now we proceed by path induction on $\alpha$, so it suffices to define $\varphi(\total{f}(s),(s,\refl{})):\fib{f(\proj 1(\total{f}(s)))}{\proj 2(\total{f}(s))}$. Finally, we use $\Sigma$-induction on $s$ once more, so it suffices to define
  \begin{equation*}
    \varphi((x,f(x,y)),((x,y),\refl{})):\fib{f(x)}{f(x,y)}.
  \end{equation*}
  Now we take as our definition
  \begin{equation*}
    \varphi((x,f(x,y)),((x,y),\refl{}))\defeq(y,\refl{}).
  \end{equation*}

  For the proof that this map is an equivalence we construct a map
  \begin{equation*}
    \psi(t) : \fib{f(\proj 1(t))}{\proj 2(t)}\to\fib{\total{f}}{t}
  \end{equation*}
  equipped with homotopies $G(t):\varphi(t)\circ\psi(t)\htpy\idfunc$ and $H(t):\psi(t)\circ\varphi(t)\htpy\idfunc$. In each of these definitions we use $\Sigma$-induction and path induction all the way through, until an obvious choice of definition becomes apparent. We define $\psi(t)$, $G(t)$, and $H(t)$ as follows:
  \begin{align*}
    \psi((x,f(x,y)),(y,\refl{})) & \defeq ((x,y),\refl{}) \\
    G((x,f(x,y)),(y,\refl{})) & \defeq \refl{} \\
    H((x,f(x,y)),((x,y),\refl{})) & \defeq \refl{}.\qedhere
  \end{align*}
\end{proof}

\begin{thm}\label{thm:fib_equiv}
Let $f:\prd{x:A}B(x)\to C(x)$ be a family of maps. The following are equivalent:
\begin{enumerate}
\item For each $x:A$, the map $f(x)$ is an equivalence. In this case we say that $f$ is a \define{family of equivalences}.
\item The map $\total{f}:\sm{x:A}B(x)\to\sm{x:A}C(x)$ is an equivalence.
\end{enumerate}
\end{thm}

\begin{proof}
By \cref{thm:equiv_contr,thm:contr_equiv} it suffices to show that $f(x)$ is a contractible map for each $x:A$, if and only if $\total{f}$ is a contractible map. Thus, we will show that $\fib{f(x)}{c}$ is contractible if and only if $\fib{\total{f}}{x,c}$ is contractible, for each $x:A$ and $c:C(x)$. However, by \cref{lem:fib_total} these types are equivalent, so the result follows by \cref{ex:contr_equiv}.
\end{proof}

Now consider the situation where we have a map $f:A\to B$, and a family $C$ over $B$. Then we have the map
\begin{equation*}
  \lam{(x,z)}(f(x),z):\sm{x:A}C(f(x))\to\sm{y:B}C(y).
\end{equation*}
We claim that this map is an equivalence when $f$ is an equivalence. The technique to prove this claim is the same as the technique we used in \cref{thm:fib_equiv}: first we note that the fibers are equivalent to the fibers of $f$, and then we use the fact that a map is an equivalence if and only if its fibers are contractible to finish the proof.

\begin{lem}\label{lem:total-equiv-base-equiv}
  Consider an equivalence $e:A\simeq B$, and let $C$ be a type family over $B$. Then the map
  \begin{equation*}
    \sigma_f(C) \defeq\lam{(x,z)}(f(x),z):\sm{x:A}C(f(x))\to\sm{y:B}C(y)
  \end{equation*}
  is an equivalence.
\end{lem}

\begin{proof}
  We claim that for each $t:\sm{y:B}C(y)$ there is an equivalence
  \begin{equation*}
    \fib{\sigma_f(C)}{t}\simeq \fib{f}{\proj 1(t)}.
  \end{equation*}
  We prove this by constructing
  \begin{align*}
    \varphi(t) & : \fib{\sigma_f(C)}{t}\to\fib{f}{\proj 1 (t)} \\
    \psi(t) & : \fib{f}{\proj 1(t)} \to\fib{\sigma_f(C)}{t} \\
    G(t) & : \varphi\circ\psi\htpy\idfunc\\
    H(t) & : \psi\circ\varphi\htpy\idfunc.
  \end{align*}
  The construction of these functions and homotopies is by using $\Sigma$-induction and path induction all the way through, just as in the proof of \cref{lem:fib_total}. We list the definitions
  \begin{align*}
    \varphi((f(x),z),((x,z),\refl{})) & \defeq (x,\refl{}) \\
    \psi((f(x),z),(x,\refl{})) & \defeq ((x,z),\refl{}) \\
    G((f(x),z),(x,\refl{})) & \defeq \refl{} \\
    H((f(x),z),((x,z),\refl{})) & \defeq \refl{}.
  \end{align*}
  Now the claim follows, since we see that $\varphi$ is a contractible map if and only if $f$ is a contractible map.
\end{proof}

We now combine \cref{thm:fib_equiv,lem:total-equiv-base-equiv}.

\begin{defn}
  Consider a map $f:A\to B$ and a family of maps
  \begin{equation*}
    g:\prd{x:A}C(x)\to D(f(x)),
  \end{equation*}
  where $C$ is a type family over $A$, and $D$ is a type family over $B$. In this situation we also say that $g$ is a \define{family of maps over $f$}. Then we define
  \begin{equation*}
    \total[f]{g}:\sm{x:A}C(x)\to\sm{y:B}D(y)
  \end{equation*}
  by $\total[f]{g}(x,z)\defeq (f(x),g(x,z))$.
\end{defn}

\begin{thm}
  Suppose that $g$ is a family of maps over $f$, and suppose that $f$ is an equivalence. Then the following are equivalent:
  \begin{enumerate}
  \item The family of maps $g$ over $f$ is a family of equivalences.
  \item The map $\total[f]{g}$ is an equivalence.
  \end{enumerate}
\end{thm}

\begin{proof}
  Note that we have a commuting triangle
  \begin{equation*}
    \begin{tikzcd}
      \sm{x:A}C(x) \arrow[rr,"{\total[f]{g}}"] \arrow[dr,swap,"\total{g}"]& & \sm{y:B}D(y) \\
      & \sm{x:A}D(f(x)) \arrow[ur,swap,"{\lam{(x,z)}(f(x),z)}"]
    \end{tikzcd}
  \end{equation*}
  By the assumption that $f$ is an equivalence, it follows that the map $\sm{x:A}D(f(x))\to \sm{y:B}D(y)$ is an equivalence. Therefore it follows that $\total[f]{g}$ is an equivalence if and only if $\total{g}$ is an equivalence. Now the claim follows, since $\total{g}$ is an equivalence if and only if $g$ if a family of equivalences.
\end{proof}

\section{Identity systems}

The fundamental theorem (\cref{thm:id_fundamental}) tells us when, for a type family $B$ over $A$ and a fixed $a:A$, there is a family of equivalences $\prd{x:A}(a=x)\simeq B(x)$. In other words, it tells us when a family $B$ is a characterization of the identity type of $A$.

\begin{thm}\label{thm:id_fundamental}
Let $A$ be a type with $a:A$, and let $B$ be be a type family over $A$ with $b:B(a)$.
Then  the following are logically equivalent for any family of maps
\begin{equation*}
  f:\prd{x:A}(a=x)\to B(x).
\end{equation*}
\begin{enumerate}
\item The family of maps $f$ is a family of equivalences.
\item The total space
\begin{equation*}
\sm{x:A}B(x)
\end{equation*}
is contractible.
\end{enumerate}
In particular the canonical family of maps
\begin{equation*}
\mathsf{path\usc{}ind}_a(b):\prd{x:A} (a=x)\to B(x)
\end{equation*}
is a family of equivalences if and only if $\sm{x:A}B(x)$ is contractible.
\end{thm}

\begin{proof}
By \cref{thm:fib_equiv} it follows that the family of maps $\mathsf{path\usc{}ind}_a(b)$ is a family of equivalences if and only if it induces an equivalence
\begin{equation*}
\eqv{\Big(\sm{x:A}a=x\Big)}{\Big(\sm{x:A}B(x)\Big)}
\end{equation*}
on total spaces. We have that $\sm{x:A}a=x$ is contractible. Now it follows by \cref{ex:contr_equiv}, applied in the case
\begin{equation*}
\begin{tikzcd}
\sm{x:A}a=x \arrow[rr,"\total{\mathsf{path\usc{}ind}_a(b)}"] \arrow[dr,swap,"\eqvsym"] & & \sm{x:A}B(x) \arrow[dl] \\
& \unit
\end{tikzcd}
\end{equation*}
that $\total{\mathsf{path\usc{}ind}_a(b)}$ is an equivalence if and only if $\sm{x:A}B(x)$ is contractible.
\end{proof}

As could be expected, when $B$ is equivalent to the identity type of $A$ we can also prove a variant of path induction for $B$.

\begin{defn}
  Let $B$ be a type family over $A$, and let $a:A$ and $b:B(a)$ be given. We say that $B$ is an \define{(unary) identity system} if for every family $P$ indexed by $x:A$ and $y:B(x)$, the map
  \begin{equation*}
    \Big(\prd{x:A}{y:B(x)}P(x,y)\Big)\to P(a,\refl{a})
  \end{equation*}
  has a section.
\end{defn}

\begin{thm}
  Let $B$ be a type family over $A$, and let $a:A$ and $b:B(a)$ be given. Then the following are equivalent:
  \begin{enumerate}
  \item The family $B$ is an identity system.
  \item The total space of $B$ is contractible.
  \end{enumerate}
\end{thm}

\begin{proof}
  Note that we have the following commuting triangle
  \begin{equation*}
    \begin{tikzcd}
      \prd{t:\sm{x:A}B(x)}P(t) \arrow[rr,"\mathsf{ev\usc{}pair}"] \arrow[dr,swap,"{\mathsf{ev\usc{}pt}(a,b)}"] & & \prd{x:A}{y:B(x)}P(x,y) \arrow[dl,"{\lam{f}f(a,b)}"] \\
      & P(a,b)
    \end{tikzcd}
  \end{equation*}
  In this diagram the top map has a section. Therefore it follows by \cref{ex:3_for_2} that the left map has a section if and only if the right map has a section. Notice that the left map has a section for all $P$ if and only if $\sm{x:A}B(x)$ satisfies singleton induction, which is by \cref{thm:contractible} equivalent to $\sm{x:A}B(x)$ being contractible.
\end{proof}

\section{Embeddings}
As an application of the fundamental theorem we show that equivalences are embeddings. The notion of embedding is the homotopical analogue of the set theoretic notion of injective map.

\begin{defn}
An \define{embedding}\index{embedding|textbf} is a map $f:A\to B$ satisfying the property that
\begin{equation*}
\apfunc{f}:(\id{x}{y})\to(\id{f(x)}{f(y)})
\end{equation*}
is an equivalence for every $x,y:A$. We write $\mathsf{is\usc{}emb}(f)$ for the type of witnesses that $f$ is an embedding.
\end{defn}

Another way of phrasing the following statement is that equivalent types have equivalent identity types.

\begin{thm}
\label{cor:emb_equiv} 
Any equivalence is an embedding.\index{embedding!equivalences are embeddings|textit}\index{equivalence!is an embedding|textit}
\end{thm}

\begin{proof}
Let $e:\eqv{A}{B}$ be an equivalence, and let $x:A$. Our goal is to show that
\begin{equation*}
\apfunc{e} : (\id{x}{y})\to (\id{e(x)}{e(y)})
\end{equation*}
is an equivalence for every $y:A$. By \cref{thm:id_fundamental} it suffices to show that 
\begin{equation*}
\sm{y:A}e(x)=e(y)
\end{equation*}
is contractible for every $y:A$. Now observe that there is an equivalence
\begin{samepage}
\begin{align*}
\sm{y:A}e(x)=e(y) & \eqvsym \sm{y:A}e(y)=e(x) \\
& \jdeq \fib{e}{e(x)}
\end{align*}
\end{samepage}
by \cref{thm:fib_equiv}, since for each $y:A$ the map
\begin{equation*}
\mathsf{inv} : (e(x)=e(y))\to (e(y)= e(x))
\end{equation*}
is an equivalence by \cref{ex:equiv_grpd_ops}.
The fiber $\fib{e}{e(x)}$ is contractible by \cref{thm:contr_equiv}, so it follows by \cref{ex:contr_equiv} that the type $\sm{y:A}e(x)=e(y)$ is indeed contractible.
\end{proof}

\begin{comment}
As a first application of the fundamental theorem, we compute the identity type of a coproduct.

\begin{defn}
Let $A$ and $B$ be types in $\UU$. We construct equivalences
\begin{align*}
(\id[A+B]{\inl(x)}{\inl(x')}) & \eqvsym (\id[A]{x}{x'}) \\
(\id[A+B]{\inl(x)}{\inr(y')}) & \eqvsym \emptyt \\
(\id[A+B]{\inr(y)}{\inl(x')}) & \eqvsym \emptyt \\
(\id[A+B]{\inr(y)}{\inr(y')}) & \eqvsym (\id[B]{y}{y'}).
\end{align*}
\end{defn}

\begin{constr}
We define by double induction on the disjoint sum the binary relation
\begin{equation*}
E : (A+B)\to (A+B)\to\UU
\end{equation*}
given by
\begin{align*}
E({\inl(x)},{\inl(x')}) & \defeq \id[A]{x}{x'} \\
E({\inl(x)},{\inr(y')}) & \defeq \emptyt \\
E({\inr(y)},{\inl(x')}) & \defeq \emptyt \\
E({\inr(y)},{\inr(y')}) & \defeq (\id[B]{y}{y'}).
\end{align*}
Moreover, we have a term $\rho:\prd{s:A+B}E(s,s)$ defined by $\rho(\inl(x))\defeq\refl{x}$ and $\rho(\inr(y))\defeq\refl{y}$.

Our goal is to construct an equivalence $\eqv{(\id{s}{t})}{E(s,t)}$ for any $s,t:A+B$. 
By \cref{thm:id_fundamental} it suffices to show that for any $s:A+B$, the type
\begin{equation*}
\sm{t:A+B}E(s,t)
\end{equation*}
is contractible. The center of contraction is taken to be $\pairr{s,\rho(s)}$, so it remains to construct the contraction
\begin{equation*}
\prd{t:A+B}{e:E(s,t)} \pairr{s,\rho(s)}=\pairr{t,e}.
\end{equation*}
This is done by induction on $s$ and $t$, so we have to show that
\begin{align*}
& \prd{x':A}{p:x=x'} \pairr{\inl(x),\refl{x}}=\pairr{x',p} \\
& \prd{y':A}{q:\emptyt} \pairr{\inl(x),\refl{x}}=\pairr{y',q} \\
& \prd{x':A}{q:\emptyt} \pairr{\inr(y),\refl{y}}=\pairr{x',q} \\
& \prd{y':A}{p:y=y'} \pairr{\inr(y),\refl{y}}=\pairr{y',p}.
\end{align*}
The first and fourth case are easily shown by path induction on $p$, and the second and third case are easily shown by induction on the empty type.
\end{constr}
\end{comment}

\section{Disjointness of coproducts}

To give a second application of the fundamental theorem of identity types, we characterize the identity types of coproducts. Our goal in this section is to prove the following theorem.

\begin{thm}\label{thm:id-coprod-compute}
Let $A$ and $B$ be types. Then there are equivalences
\begin{align*}
(\inl(x)=\inl(x')) & \eqvsym (x = x')\\
(\inl(x)=\inr(y')) & \eqvsym \emptyt \\
(\inr(y)=\inl(x')) & \eqvsym \emptyt \\
(\inr(y)=\inr(y')) & \eqvsym (y=y')
\end{align*}
for any $x,x':A$ and $y,y':B$.
\end{thm}

In order to prove \cref{thm:id-coprod-compute}, we first define
a binary relation $\mathsf{Eq\usc{}coprod}_{A,B}$ on the coproduct $A+B$.

\begin{defn}
Let $A$ and $B$ be types. We define 
\begin{equation*}
\mathsf{Eq\usc{}coprod}_{A,B} : (A+B)\to (A+B)\to\UU
\end{equation*}
by double induction on the coproduct, postulating
\begin{align*}
\mathsf{Eq\usc{}coprod}_{A,B}(\inl(x),\inl(x')) & \defeq (x=x') \\
\mathsf{Eq\usc{}coprod}_{A,B}(\inl(x),\inr(y')) & \defeq \emptyt \\
\mathsf{Eq\usc{}coprod}_{A,B}(\inr(y),\inl(x')) & \defeq \emptyt \\
\mathsf{Eq\usc{}coprod}_{A,B}(\inr(y),\inr(y')) & \defeq (y=y')
\end{align*}
The relation $\mathsf{Eq\usc{}coprod}_{A,B}$ is also called the \define{observational equality of coproducts}\index{observational equality!of coproducts}.
\end{defn}

\begin{lem}
The observational equality relation $\mathsf{Eq\usc{}coprod}_{A,B}$ on $A+B$ is reflexive, and therefore there is a map
\begin{equation*}
\mathsf{Eq\usc{}coprod\usc{}eq}:\prd{s,t:A+B} (s=t)\to \mathsf{Eq\usc{}coprod}_{A,B}(s,t)
\end{equation*}
\end{lem}

\begin{constr}
The reflexivity term $\rho$ is constructed by induction on $t:A+B$, using
\begin{align*}
\rho(\inl(x))\defeq \refl{\inl(x)}  & : \mathsf{Eq\usc{}coprod}_{A,B}(\inl(x)) \\
\rho(\inr(y))\defeq \refl{\inr(y)} & : \mathsf{Eq\usc{}coprod}_{A,B}(\inr(y)).\qedhere
\end{align*}
\end{constr}

To show that $\mathsf{Eq\usc{}coprod\usc{}eq}$ is a family of equivalences, we will use the fundamental theorem, \cref{thm:id_fundamental}. Moreover, we will use the functoriality of coproducts (established in \cref{ex:coproduct_functor}), along with the following facts about $\Sigma$-types, coproducts, and the empty type:
\begin{align*}
\sm{t:A+B}P(t) & \eqvsym \Big(\sm{x:A}P(\inl(x))\Big)+\Big(\sm{y:B}P(\inr(y))\Big)\\
\sm{x:A}\emptyt & \eqvsym \emptyt \\
A+\emptyt & \eqvsym A.
\end{align*}
All of these equivalences are straightforward to construct, so we leave them as an exercise to the reader. 

\begin{lem}\label{lem:is-contr-total-eq-coprod}
For any $s:A+B$ the total space
\begin{equation*}
\sm{t:A+B}\mathsf{Eq\usc{}coprod}_{A,B}(s,t)
\end{equation*}
is contractible.
\end{lem}

\begin{proof}
We will do the proof by induction on $s$. The two cases are similar, so we only show that the total space
\begin{equation*}
\sm{t:A+B}\mathsf{Eq\usc{}coprod}_{A,B}(\inl(x),t)
\end{equation*}
is contractible. Note that we have equivalences
\begin{samepage}
\begin{align*}
& \sm{t:A+B}\mathsf{Eq\usc{}coprod}_{A,B}(\inl(x),t) \\
& \eqvsym \Big(\sm{x':A}\mathsf{Eq\usc{}coprod}_{A,B}(\inl(x),\inl(x'))\Big)+\Big(\sm{y':B}\mathsf{Eq\usc{}coprod}_{A,B}(\inl(x),\inr(y'))\Big) \\
& \eqvsym \Big(\sm{x':A}x=x'\Big)+\Big(\sm{y':B}\emptyt\Big) \\
& \eqvsym \Big(\sm{x':A}x=x'\Big)+\emptyt \\
& \eqvsym \sm{x':A}x=x'.
\end{align*}%
\end{samepage}%
The latter type is contractible by \cref{thm:total_path}.
\end{proof}

\begin{proof}[Proof of \cref{thm:id-coprod-compute}]
The proof is now concluded with an application of \cref{thm:id_fundamental}, using \cref{lem:is-contr-total-eq-coprod}.
\end{proof}

\begin{exercises}
\item Show that the map $\emptyt\to A$ is an embedding for every type $A$.
\item Consider an equivalence $e:A\simeq B$. Construct an equivalence
  \begin{equation*}
    (e(x)=y)\simeq(x=e^{-1}(y))
  \end{equation*}
  for every $x:A$ and $y:B$.
\item Show that 
\begin{equation*}
(f\htpy g)\to (\mathsf{is\usc{}emb}(f)\leftrightarrow\mathsf{is\usc{}emb}(g))
\end{equation*}
for any $f,g:A\to B$.
\item \label{ex:emb_triangle}Consider a commuting triangle
\begin{equation*}
\begin{tikzcd}[column sep=tiny]
A \arrow[rr,"h"] \arrow[dr,swap,"f"] & & B \arrow[dl,"g"] \\
& X
\end{tikzcd}
\end{equation*}
with $H:f\htpy g\circ h$. 
\begin{subexenum}
\item Suppose that $g$ is an embedding. Show that $f$ is an embedding if and only if $h$ is an embedding.
\item Supose that $h$ is an equivalence. Show that $f$ is an embedding if and only if $g$ is an embedding.
\end{subexenum}
\item Show that for any two types $A$ and $B$, the coproduct inclusion maps $\inl:A \to A + B$ and $\inr : B \to A + B$ are embeddings.
\item \label{ex:is-equiv-is-equiv-functor-coprod}Consider two maps $f:A\to A'$ and $g:B \to B'$.
  \begin{subexenum}
  \item Show that if the map
    \begin{equation*}
      f+g:(A+B)\to (A'+B')
    \end{equation*}
    is an equivalence, then so are both $f$ and $g$ (this is the converse of \cref{ex:coproduct_functor_equivalence}).
  \item Show that $f+g$ is an embedding if and only if both $f$ and $g$ are embeddings.
  \end{subexenum}
\item \label{ex:htpy_total} 
\begin{subexenum}
\item Let $f,g:\prd{x:A}B(x)\to C(x)$ be two families of maps. Show that
\begin{equation*}
\Big(\prd{x:A}f(x)\htpy g(x)\Big)\to \Big(\total{f}\htpy \total{g}\Big). 
\end{equation*}
\item Let $f:\prd{x:A}B(x)\to C(x)$ and let $g:\prd{x:A}C(x)\to D(x)$. Show that
\begin{equation*}
\total{\lam{x}g(x)\circ f(x)}\htpy \total{g}\circ\total{f}.
\end{equation*}
\item For any family $B$ over $A$, show that
\begin{equation*}
\total{\lam{x}\idfunc[B(x)]}\htpy\idfunc.
\end{equation*}
\end{subexenum}
\item \label{ex:id_fundamental_retr}Let $a:A$, and let $B$ be a type family over $A$. 
\begin{subexenum}
\item Use \cref{ex:htpy_total,ex:contr_retr} to show that if each $B(x)$ is a retract of $\id{a}{x}$, then $B(x)$ is equivalent to $\id{a}{x}$ for every $x:A$.
\item Conclude that for any family of maps
\begin{equation*}
f : \prd{x:A} (a=x) \to B(x),
\end{equation*}
if each $f(x)$ has a section, then $f$ is a family of equivalences.
\end{subexenum}
\item Use \cref{ex:id_fundamental_retr} to show that for any map $f:A\to B$, if
\begin{equation*}
\apfunc{f} : (x=y) \to (f(x)=f(y))
\end{equation*}
has a section for each $x,y:A$, then $f$ is an embedding.
\item \label{ex:path-split}We say that a map $f:A\to B$ is \define{path-split}\index{path-split|textbf} if $f$ has a section, and for each $x,y:A$ the map
\begin{equation*}
\apfunc{f}(x,y):(x=y)\to (f(x)=f(y))
\end{equation*}
also has a section. We write $\mathsf{path\usc{}split}(f)$\index{path_split(f)@{$\mathsf{path\usc{}split}(f)$}|textbf} for the type
\begin{equation*}
\mathsf{sec}(f)\times\prd{x,y:A}\mathsf{sec}(\apfunc{f}(x,y)).
\end{equation*}
Show that for any map $f:A\to B$ the following are equivalent:
\begin{enumerate}
\item The map $f$ is an equivalence.
\item The map $f$ is path-split.
\end{enumerate}
\begin{comment}
\item \label{ex:eqv_sigma_mv}Consider a map
\begin{equation*}
f:A \to \sm{y:B}C(y).
\end{equation*}
\begin{subexenum}
\item Construct a family of maps
\begin{equation*}
f':\prd{y:B} \fib{\proj 1\circ f}{y}\to C(y).
\end{equation*}
\item Construct an equivalence
\begin{equation*}
\eqv{\fib{f'(b)}{c}}{\fib{f}{(b,c)}}
\end{equation*}
for every $(b,c):\sm{y:B}C(y)$.
\item Conclude that the following are equivalent:
\begin{enumerate}
\item $f$ is an equivalence.
\item $f'$ is a family of equivalences.
\end{enumerate}
\end{subexenum}
\item \label{ex:coh_intro}Consider a type $A$ with base point $a:A$, and let $B$ be a type family on $A$ that implies the identity type, i.e., there is a term
\begin{equation*}
\alpha : \prd{x:A} B(x)\to (a=x).
\end{equation*}
Show that the \define{coherence reduction map}
\begin{equation*}
\mathsf{coh\usc{}red} : \Big(\sm{y:B(a)}\alpha(a,y)=\refl{a}\Big) \to \Big(\sm{x:A}B(x)\Big)
\end{equation*}
defined by $\lam{(y,q)}(a,y)$ is an equivalence.
\end{comment}
\item \label{ex:fiber_trans}Consider a triangle
\begin{equation*}
\begin{tikzcd}[column sep=small]
A \arrow[rr,"h"] \arrow[dr,swap,"f"] & & B \arrow[dl,"g"] \\
& X
\end{tikzcd}
\end{equation*}
with a homotopy $H:f\htpy g\circ h$ witnessing that the triangle commutes. 
\begin{subexenum}
\item Construct a family of maps
\begin{equation*}
\mathsf{fib\usc{}triangle}(h,H):\prd{x:X}\fib{f}{x}\to\fib{g}{x},
\end{equation*}
for which the square
\begin{equation*}
\begin{tikzcd}[column sep=8em]
\sm{x:X}\fib{f}{x} \arrow[r,"\total{\mathsf{fib\usc{}triangle}(h,H)}"] \arrow[d] & \sm{x:X}\fib{g}{x} \arrow[d] \\
A \arrow[r,swap,"h"] & B
\end{tikzcd}
\end{equation*}
commutes, where the vertical maps are as constructed in \cref{ex:fib_replacement}.
\item Show that $h$ is an equivalence if and only if $\mathsf{fib\usc{}triangle}(h,H)$ is a family of equivalences.
\end{subexenum}
\begin{comment}
\item Let $f:A\to B$ be a map, and let $s,t : \fib{f}{b}$. Consider the function
\begin{equation*}
\varphi : (s=t)\to \fib{\apfunc{f}}{\ct{\proj 2(s)}{\proj 2(t)^{-1}}}
\end{equation*}
given by $\varphi(\refl{s})=(\refl{\proj 1(s)},\mathsf{right\usc{}inv}(\proj 2(s))^{-1})$. Show that this map is an equivalence. Conclude that for any $q:f(x)=f(y)$ we have an equivalence
\begin{equation*}
((x,q)=(y,\refl{f(y)})) \simeq \fib{\apfunc{f}}{q}.
\end{equation*}
\item Construct an equivalence 
\begin{equation*}
\eqv{\big(\sm{x:A}f(x)=y\big)}{\big(\sm{x:A}y=f(x)\big)}.
\end{equation*}
%Conclude that $\sm{x:A}a=x$ is contractible for any $a:A$.
\end{comment}
\end{exercises}
