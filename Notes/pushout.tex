\chapter{Homotopy pushouts}

We can use higher inductive types\index{higher inductive types} to attach cells\index{attaching cells} to types.
For example, when we are given a type $A$, and we have a map $f:\sphere{1}\to A$ describing a circle\index{circle} in $A$.
Then we can form a new type $A'$ in which we attach a disc by `gluing' the boundary of the disc to the circle in $A$.
Using higher inductive types, this process of attaching a disc works as follows:
\begin{enumerate}
\item First we add all the points of $A$ to $A'$, i.e.~$A'$ comes equipped with a map
\begin{equation*}
i : A \to A'
\end{equation*}
\item Next, we add a new point, which is to be thought of as the center of the disc that we're attaching. In other words, $A'$ comes equipped with
\begin{equation*}
\mathrm{pt} : A'
\end{equation*}
\item Finally, for each point $x$ on the circle we add a path from the center of the disc to $i(f(x))$. In other words, $A'$ comes equipped with a path constructor
\begin{equation*}
r:\prd{x:\sphere{1}} \mathrm{pt}=i(f(x)).
\end{equation*}
\end{enumerate}
Moreover, since we're only attaching a disc to $A$ along $f$, we suppose that $A'$ satisfies an induction principle with respect to the constructors $i$, $\mathrm{pt}$, and $r$. 

The process of attaching a disc to a type $A$ along a map $f:\sphere{1}\to A$ can be generalized, so that we will also be able to attach cells of different shapes to a type. This generalization is called homotopy pushouts. Homotopy pushouts are dual to homotopy pullbacks. However, unlike pullbacks we will \emph{assume} that pushouts exist by postulating rules for higher inductive types. For the purpose of this course, the only higher inductive types that we add to our type theory are the pushouts. Some of the more exotic higher inductive types, including the Cauchy real numbers, are described in \cite{hottbook}.

\section{Pushouts as higher inductive types}

The idea of pushouts is to glue two types $A$ and $B$ together using a mediating type $S$ and maps $f:S\to A$ and $g:S\to B$. In other words, we start with a diagram of the form
\begin{equation*}
\begin{tikzcd}
A & S \arrow[l,swap,"f"] \arrow[r,"g"] & B.
\end{tikzcd}
\end{equation*}
We call such a triple $\mathcal{S}\jdeq (S,f,g)$ a \define{span}\index{span} from $A$ to $B$.
A span from $A$ to $B$ can be thought of as a relation\index{relation} from $A$ to $B$, relating $f(x)$ to $g(x)$ for any $x:S$.
Indeed, an equivalence between the type of all spans and the type of relations from $A$ to $B$ is established in \cref{ex:span_rel}.

Given a span $\mathcal{S}$ from $A$ to $B$, we form the higher inductive type $A \sqcup^{\mathcal{S}} B$. It comes equipped with the following constructors\index{inl@{$\inl$}!for pushouts}\index{inr@{$\inr$}!for pushouts}\index{glue@{$\glue$}}
\begin{align*}
\inl & : A \to A \sqcup^{\mathcal{S}} B \\
\inr & : B \to A \sqcup^{\mathcal{S}} B \\
\glue & : \prd{x:S} \inl(f(x))=\inr(g(x))
\end{align*}
and we require that it satisfies an induction principle and computation rules.

To see what the induction principle has to be, consider first a dependent function $s:\prd{x:A\sqcup^{\mathcal{S}}B}P(x)$. When we evaluate this function at the constructors, we obtain
\begin{align*}
s\circ \inl & : \prd{a:A} P(\inl(a)) \\
s\circ \inr & : \prd{b:B} P(\inr(b)) \\
\apdfunc{s}\circ \glue & : \prd{x:S} \mathsf{tr}_P(\glue(x),s(f(x)))=s(g(x)).
\end{align*}

\begin{defn}
Consider a span $\mathcal{S}\jdeq (S,f,g)$ from $A$ to $B$, and let $P$ be a family over $A\sqcup^{\mathcal{S}} B$. The \define{dependent action on generators}\index{dependent action on generators!for pushouts|textbf} is defined to be the map\index{dgen_S@{$\mathsf{dgen}_{\mathcal{S}}$}|textbf}
\begin{align*}
\mathsf{dgen}_{\mathcal{S}}^P & : \Big(\prd{x:A\sqcup^{\mathcal{S}} B} P(x)\Big) \to \Big(\sm{f': \prd{a:A}P(\inl(a))}{g':\prd{b:B}P(\inr(b))}\Big.\\
& \qquad\qquad\qquad\qquad\qquad\qquad \Big.\prd{x:S} \mathsf{tr}_P(\glue(x),f'(f(x)))=g'(g(x))\Big).
\end{align*}
given by $s\mapsto (s\circ\inl,s\circ\inr,\apdfunc{s}\circ\glue)$.
\end{defn}

We can now fully specify homotopy pushouts.

\begin{defn}
Given a span $\mathcal{S}\jdeq (S,f,g)$, the \define{(homotopy) pushout}\index{pushout|textbf} $A\sqcup^{\mathcal{S}} B$ of $\mathcal{S}$ is defined to be the higher inductive\index{higher inductive types} type equipped with\index{inl@{$\inl$}!for pushouts|textbf}\index{inr@{$\inr$}!for pushouts|textbf}\index{glue@{$\glue$}|textbf}
\begin{align*}
\inl & : A \to A \sqcup^{\mathcal{S}} B \\
\inr & : B \to A \sqcup^{\mathcal{S}} B \\
\glue & : \prd{x:S} \inl(f(x))=\inr(g(x)),
\end{align*}
satisfying the \define{induction principle} for pushouts\index{induction principle!for pushouts|textbf}, which asserts that for each type family $P$ over $A\sqcup^{\mathcal{S}} B$ the map $\mathsf{dgen}_{\mathcal{S}}^P$ has a section.
\end{defn}

\begin{rmk}
The induction principle of the pushout $A\sqcup^{\mathcal{S}} B$ provides us with a dependent function
\begin{equation*}
\ind{\mathcal{S}}(f',g',G) : \prd{x:A\sqcup^{\mathcal{S}} B} P(x),
\end{equation*}
for every
\begin{align*}
f' & : \prd{a:A}P(\inl(a)) \\
g' & : \prd{b:B}P(\inr(b)) \\
G & : \prd{x:S} \mathsf{tr}_P(\glue(x),f'(f(x)))=g'(g(x))
\end{align*}
Moreover, the function $\ind{\mathcal{S}}(f',g',G)$ comes equipped with an identification
\begin{equation*}
\mathsf{dgen}_{\mathcal{S}}(\ind{\mathcal{S}}(f',g',G))=(f',g',G).
\end{equation*}
Writing $s\defeq \ind{\mathcal{S}}(f',g',G)$, we see that such an identification between triples is equivalently described by a triple $(H,K,L)$ consisting of
\begin{align*}
H : s\circ \inl \htpy f' \\
K : s\circ\inr \htpy g' 
\end{align*}
and a homotopy $L$ witnessing that the square
\begin{equation*}
\begin{tikzcd}[column sep=8em]
\mathsf{tr}_{P}(\glue(x),s(\inl(f(x)))) \arrow[r,equals,"\ap{\mathsf{tr}_P(\glue(x))}{H(x)}"] \arrow[d,equals,swap,"\apd{s}{\glue(x)}"] & \mathsf{tr}_{P}(\glue(x),f'(f(x))) \arrow[d,equals,"G(x)"] \\
s(\inr(g(x))) \arrow[r,equals,swap,"K(x)"] & g'(g(x))
\end{tikzcd}
\end{equation*}
commutes, for every $x:S$. These are the \define{computation rules} for pushouts\index{computation rules!for pushouts}.
\end{rmk}

\begin{comment}
The \define{formation rule} for pushouts simply states that for any span $\mathcal{S}\defeq (S,f,g)$ from $A$ to $B$, a type $A\sqcup^{\mathcal{S}} B$ can be formed. We call $A\sqcup^{\mathcal{S}} B$ the \define{canonical pushout} of $\mathcal{S}$. 

\begin{prooftree}
\AxiomC{$\Gamma\vdash f:S\to A$}
\AxiomC{$\Gamma\vdash g:S\to B$}
\BinaryInfC{$\Gamma\vdash A\sqcup^{\mathcal{S}} B~\mathrm{type}$}
\end{prooftree}

The \define{introduction rules} for pushouts provide ways to construct terms of the type $A\sqcup^{\mathcal{S}} B$, and ways to identify some of those.
\begin{prooftree}
\AxiomC{$\Gamma\vdash f:S\to A$}
\AxiomC{$\Gamma\vdash g:S\to B$}
\BinaryInfC{$\Gamma\vdash \inl : A \to A\sqcup^{\mathcal{S}} B$}
\end{prooftree}

\begin{prooftree}
\AxiomC{$\Gamma\vdash f:S\to A$}
\AxiomC{$\Gamma\vdash g:S\to B$}
\BinaryInfC{$\Gamma\vdash \inr : B \to A\sqcup^{\mathcal{S}} B$}
\end{prooftree}

\begin{prooftree}
\AxiomC{$\Gamma\vdash f:S\to A$}
\AxiomC{$\Gamma\vdash g:S\to B$}
\BinaryInfC{$\Gamma\vdash \glue : \inl\circ f \htpy \inr\circ g$}
\end{prooftree}
We assume that $A\sqcup^{\mathcal{S}} B$ is span inductive in the sense of \autoref{thm:pushout_up}. Moreover, if $A$, $B$, and $S$ are types in $\UU$, then we assume that also $A\sqcup^{\mathcal{S}} B$ is in $\UU$. In other words, we assume that the universe is \emph{closed under pushouts}.
\end{comment}

\section{Examples of pushouts}
Many interesting types can be defined as homotopy pushouts. 

\begin{defn}
Let $X$ be a type. We define the \define{suspension}\index{suspension|textbf} $\susp X$\index{SX@{$\susp X$}|textbf} of $X$ to be the pushout of the span
\begin{equation*}
\begin{tikzcd}
X \arrow[r] \arrow[d] & \unit \arrow[d,"\inr"] \\
\unit \arrow[r,swap,"\inl"] & \susp X 
\end{tikzcd}
\end{equation*}
\end{defn}

\begin{defn}
We define the \define{$n$-sphere}\index{n-sphere@{$n$-sphere}|textbf} $\sphere{n}$\index{Sn@{$\sphere{n}$}|textbf} for any $n:\N$ by induction on $n$, by taking
\begin{align*}
\sphere{0} & \defeq \bool \\
\sphere{n+1} & \defeq \susp{\sphere{n}}.
\end{align*}
\end{defn}

\begin{defn}
Given a map $f:A\to B$, we define the \define{cofiber}\index{cofiber|textbf} $\mathsf{cofib}_f$\index{cofib_f@{$\mathsf{cofib}_f$}|textbf} of $f$ as the pushout
\begin{equation*}
\begin{tikzcd}
A \arrow[r,"f"] \arrow[d] & B \arrow[d,"\inr"] \\
\unit \arrow[r,swap,"\inl"] & \mathsf{cofib}_f. 
\end{tikzcd}
\end{equation*}
The cofiber of a map is sometimes also called the \define{mapping cone}\index{mapping cone|textbf}.
\end{defn}

\begin{eg}
The suspension $\susp X$ of $X$ is the cofiber of the map $X\to \unit$.\index{suspension!as cofiber} 
\end{eg}

\begin{defn}
We define the \define{join}\index{join} $\join{X}{Y}$\index{join X Y@{$\join{X}{Y}$}|textbf} of $X$ and $Y$ to be the pushout 
\begin{equation*}
\begin{tikzcd}
X\times Y \arrow[r,"\proj 2"] \arrow[d,swap,"\proj 1"] & Y \arrow[d,"\inr"] \\
X \arrow[r,swap,"\inl"] & X \ast Y. 
\end{tikzcd}
\end{equation*}
\end{defn}

\begin{defn}
Suppose $A$ and $B$ are pointed types, with base points $a_0$ and $b_0$, respectively. The \define{(binary) wedge}\index{wedge@(binary) wedge|textbf} $A\vee B$ of $A$ and $B$ is defined as the pushout
\begin{equation*}
\begin{tikzcd}
\bool \arrow[r] \arrow[d] & A+B \arrow[d] \\
\unit \arrow[r] & A\vee B.
\end{tikzcd}
\end{equation*}
\end{defn}

\begin{defn}
Given a type $I$, and a family of pointed types $A$ over $i$, with base points $a_0(i)$. We define the \define{(indexed) wedge}\index{wedge@{(indexed) wedge}|textbf} $\bigvee_{(i:I)}A_i$ as the pushout
\begin{equation*}
\begin{tikzcd}[column sep=huge]
I \arrow[d] \arrow[r,"{\lam{i}(i,a_0(i))}"] & \sm{i:I}A_i \arrow[d] \\
\unit \arrow[r] & \bigvee_{(i:I)} A_i.
\end{tikzcd}
\end{equation*}
\end{defn}

\begin{comment}
\begin{defn}
Let $X$ and $Y$ be types with base points $x_0$ and $y_0$, respectively.
We define the \define{wedge} $X\lor Y$ of $X$ and $Y$ to be the pushout
\begin{equation*}
\begin{tikzcd}[column sep=8em]
\bool \arrow[r,"{\ind{\bool}(\inl(x_0),\inr(y_0))}"] \arrow[d,swap,"\mathsf{const}_\ttt"] & X+Y \arrow[d,"\inr"] \\
\unit \arrow[r,swap,"\inl"] & X\lor Y
\end{tikzcd}
\end{equation*}
\end{defn}

\begin{defn}
Let $X$ and $Y$ be types with base points $x_0$ and $y_0$, respectively.
We define a map
\begin{equation*}
\mathsf{wedge\usc{}incl} : X \lor Y \to X\times Y.
\end{equation*}
as the unique map obtained from the commutative square
\begin{equation*}
\begin{tikzcd}[column sep=8em]
\bool \arrow[r,"{\ind{\bool}(\inl(x_0),\inr(y_0))}"] \arrow[d,swap,"\mathsf{const}_\ttt"] & X+Y \arrow[d,"{\ind{X+Y}(\lam{x}\pairr{x,y_0},\lam{y}\pairr{x_0,y})}"] \\
\unit \arrow[r,swap,"\lam{t}\pairr{x_0,y_0}"] & X\times Y.
\end{tikzcd}
\end{equation*}
\end{defn}

\begin{defn}
We define the \define{smash product} $X\wedge Y$ of $X$ and $Y$ to be the pushout
\begin{equation*}
\begin{tikzcd}[column sep=huge]
X\lor Y \arrow[r,"\mathsf{wedge\usc{}incl}"] \arrow[d,swap,"\mathsf{const}_\ttt"] & X\times Y \arrow[d,"\inr"] \\
\unit \arrow[r,swap,"\inl"] & X\wedge Y.
\end{tikzcd}
\end{equation*}
\end{defn}
\end{comment}

\section{The universal property of pushouts}

\begin{defn}
Consider a span $\mathcal{S}\jdeq (S,f,g)$ from $A$ to $B$, and let $X$ be a type.
A \define{cocone}\index{cocone|textbf} with vertex $X$ on $\mathcal{S}$ is a triple $(i,j,H)$ consisting of maps $i:A\to X$ and $j:B\to X$, and a homotopy $H:i\circ f\htpy j\circ g$ witnessing that the square
\begin{equation*}
\begin{tikzcd}
S \arrow[r,"g"] \arrow[d,swap,"f"] & B \arrow[d,"j"] \\
A \arrow[r,swap,"i"] & X
\end{tikzcd}
\end{equation*}
commutes.
We write $\mathsf{cocone}_{\mathcal{S}}(X)$\index{cocone_S(X)@{$\mathsf{cocone}_{\mathcal{S}}(X)$}|textbf} for the type of cocones on $\mathcal{S}$ with vertex $X$.
\end{defn}

\begin{defn}
Consider a cocone $(i,j,H)$ with vertex $X$ on the span $\mathcal{S}\jdeq (S,f,g)$, as indicated in the following commuting square
\begin{equation*}
\begin{tikzcd}
S \arrow[r,"g"] \arrow[d,swap,"f"] & B \arrow[d,"j"] \\
A \arrow[r,swap,"i"] & X.
\end{tikzcd}
\end{equation*}
For every type $Y$, we define the map\index{cocone_map@{$\mathsf{cocone\usc{}map}$}|textbf}
\begin{equation*}
\mathsf{cocone\usc{}map}(i,j,H):(X\to Y)\to \mathsf{cocone}(Y)
\end{equation*}
by $f\mapsto (f\circ i,f\circ j,f\cdot H)$.
\end{defn}

\begin{defn}
A commuting square
\begin{equation*}
\begin{tikzcd}
S \arrow[r,"g"] \arrow[d,swap,"f"] & B \arrow[d,"j"] \\
A \arrow[r,swap,"i"] & X.
\end{tikzcd}
\end{equation*}
with $H:i\circ f \htpy j\circ g$ is said to be a \define{(homotopy) pushout square}\index{pushout square} if the cocone $(i,j,H)$ with vertex $X$ on the span $\mathcal{S}\jdeq (S,f,g)$
satisfies the \define{universal property of pushouts}\index{universal property!of pushouts|textbf}, which asserts that the map
\begin{equation*}
\mathsf{cocone\usc{}map}(i,j,H):(X\to Y)\to \mathsf{cocone}(Y)
\end{equation*}
is an equivalence for any type $Y$. Sometimes pushout squares are also called \define{cocartesian squares}\index{cocartesian square|textbf}.
\end{defn}

\begin{comment}
\begin{rmk}
Given a pushout square
\begin{equation*}
\begin{tikzcd}
S \arrow[r,"g"] \arrow[d,swap,"f"] & B \arrow[d,"j"] \\
A \arrow[r,swap,"i"] & X,
\end{tikzcd}
\end{equation*}
we can view the cocone $(i,j,H)$ as \emph{structure} on $X$, in the sense that $X$ comes equipped with
\begin{align*}
i & : A\to X \\
j & : B\to X \\
H & : \prd{s:S} i(f(s))=j(g(s)).
\end{align*}
As we will see in \cref{thm:pushout_up}, the type $X$ is a pushout precisely when it satisfies an \emph{induction principle} formulated in terms of $(i,j,H)$. However, the homotopy $H$ provides \emph{path constructors} of $X$. 

The induction principle of pushouts is formulated with respect to families $P$ over $X$, and provides a way to construct sections of $P$. Note that from any section $s:\prd{x:X}P(x)$ we obtain
\begin{align*}
s\circ i & : \prd{a:A}P(i(a)) \\
s\circ j & : \prd{b:B}P(j(b)) \\
s\cdot H & : \prd{x:S}\mathsf{tr}_P(H(x),s(i(x)))=s(j(x)).
\end{align*}
It will be useful to write
\begin{equation*}
i' \htpy_H j' \defeq \prd{s:S} \mathsf{tr}_P(H(s),i'(f(s)))=j'(g(s))
\end{equation*}
for the type of $s\cdot H$. Thus we see that there is a map
\begin{equation*}
\Big(\prd{x:X}P(x)\Big)\to \sm{i':\prd{a:A}P(i(a))}{j':\prd{b:B}P(j(b))} i'\htpy_H j'
\end{equation*}
given by $s\mapsto (s\circ i,s\circ j,s\cdot H)$.
\end{rmk}
\end{comment}

\begin{lem}\label{lem:cocone_pb}
For any span $\mathcal{S}\jdeq (S,f,g)$ from $A$ to $B$, and any type $X$ the square\index{cocone_S(X)@{$\mathsf{cocone}_{\mathcal{S}}(X)$}!as a pullback|textit}
\begin{equation*}
\begin{tikzcd}
\mathsf{cocone}_{\mathcal{S}}(X) \arrow[r,"\pi_2"] \arrow[d,swap,"\pi_1"] & X^B \arrow[d,"\blank\circ g"] \\
X^A \arrow[r,swap,"\blank\circ f"] & X^S,
\end{tikzcd}
\end{equation*}
which commutes by the homotopy $\pi_3' \defeq\lam{(i,j,H)} \mathsf{eq\usc{}htpy}(H)$, is a pullback square.
\end{lem}

\begin{proof}
The gap map $\mathsf{cocone}_{\mathcal{S}}(X)\to X^A\times_{X^S} X^B$ is the function 
\begin{equation*}
\lam{(i,j,H)}(i,j,\mathsf{eq\usc{}htpy}(H)).
\end{equation*}
This is an equivalence by \cref{thm:fib_equiv}, since it is the induced map on total spaces of the fiberwise equivalence $\mathsf{eq\usc{}htpy}$. Therefore, the square is a pullback square by \cref{thm:is_pullback}.
\end{proof}

In the following theorem we establish an alternative characterization of the universal property of pushouts.
\begin{thm}\label{thm:pushout_up}
Consider a commuting square\index{universal property!of pushouts|textit}
\begin{equation*}
\begin{tikzcd}
S \arrow[r,"g"] \arrow[d,swap,"f"] & B \arrow[d,"j"] \\
A \arrow[r,swap,"i"] & X,
\end{tikzcd}
\end{equation*}
with $H:i\circ f\htpy j\circ g$. The following are equivalent:
\begin{enumerate}
\item The square is a pushout square.
\item The square
\begin{equation*}
\begin{tikzcd}
T^X \arrow[r,"\blank\circ j"] \arrow[d,swap,"\blank\circ i"] & T^B \arrow[d,"\blank\circ g"] \\
T^A \arrow[r,swap,"\blank\circ f"] & T^S
\end{tikzcd}
\end{equation*}
which commutes by the homotopy
\begin{equation*}
\lam{h} \mathsf{eq\usc{}htpy}(h\cdot H)
\end{equation*}
is a pullback square, for every type $T$.
%\item The type $X$ satisfies \define{span induction} for the span $A\leftarrow S \rightarrow B$, in the sense that for any type family $P$ over $X$, the map
%\begin{equation*}
%\Big(\prd{x:X}P(x)\Big)\to \Big(\sm{i':\prd{a:A}P(i(a))}{j':\prd{b:B}P(j(b))} i'\htpy_H j'\Big)
%\end{equation*}
%given by $s\mapsto (s\circ i,s\circ j,s\cdot H)$ has a section.
\end{enumerate}
\end{thm}

\begin{proof}
It is straightforward to verify that the triangle
\begin{equation*}
\begin{tikzcd}
& T^X \arrow[dl,swap,"{\mathsf{cocone\usc{}map}(i,j,H)}"] \arrow[dr,"{\mathsf{gap}(\blank\circ i,\blank\circ j, \mathsf{eq\usc{}htpy}(\blank\cdot H))}"] \\
\mathsf{cocone}(T) \arrow[rr,swap,"{\mathsf{gap}(i,j,\mathsf{eq\usc{}htpy}(H))}"] & & T^A \times_{T^S} T^B
\end{tikzcd}
\end{equation*}
commutes. Since the bottom map is an equivalence by \cref{lem:cocone_pb}, it follows that if either one of the remaining maps is an equivalence, so is the other. The claim now follows by \cref{thm:is_pullback}.
\end{proof}

\begin{eg}\label{eg:circle_pushout}
By \autoref{ex:circle_up_pushout} and the second characterization of pushouts in \autoref{thm:pushout_up} it follows that the circle is a pushout\index{circle!S1 equiv susp 2@{$\eqv{\sphere{1}}{\susp\bool}$}|textit}
\begin{equation*}
\begin{tikzcd}
\bool \arrow[r] \arrow[d] & \unit \arrow[d] \\
\unit \arrow[r] & \sphere{1}.
\end{tikzcd}
\end{equation*}
In other words, $\eqv{\sphere{1}}{\susp{\bool}}$. 
\end{eg}

\begin{thm}\label{thm:pushout}
Consider a span $\mathcal{S}\jdeq (S,f,g)$ from $A$ to $B$. Then the square
\begin{equation*}
\begin{tikzcd}
S \arrow[r,"g"] \arrow[d,swap,"f"] & B \arrow[d,"\inr"] \\
A \arrow[r,swap,"\inl"] & A \sqcup^{\mathcal{S}} B
\end{tikzcd}
\end{equation*}
is a pushout square.\index{pushout!universal property|textit}
\end{thm}

\begin{proof}
Let $X$ be a type. Our goal is to show that the map
\begin{equation*}
\mathsf{cocone\usc{}map}(\inl,\inr,\glue):(A\sqcup^{\mathcal{S}} B \to X)\to \mathsf{cocone}_{\mathcal{S}}(X)
\end{equation*}
is an equivalence. For notational breveity we will just write $\mathsf{gen}_{\mathcal{S}}$\index{gen_S@{$\mathsf{gen}_{\mathcal{S}}$}} for $\mathsf{cocone\usc{}map}_{\mathcal{S}}(\inl,\inr,\glue)$, because $\mathsf{cocone\usc{}map}_{\mathcal{S}}(\inl,\inr,\glue)$ is just the action on generators.

We first note that by \cref{ex:trans_triv} there is a commuting triangle
\begin{equation*}
\begin{tikzcd}[column sep=0]
& X^{A\sqcup^{\mathcal{S}} B} \arrow[dl,swap,"\mathsf{gen}_{\mathcal{S}}"] \arrow[dr,"\mathsf{dgen}_{\mathcal{S}}"] \\
\mathsf{cocone}_{\mathcal{S}}(X) \arrow[rr,"\eqvsym"] & & \mathsf{cocone}'_{\mathcal{S}}(X)
\end{tikzcd}
\end{equation*}
where we write
\begin{align*}
\mathsf{cocone}'_{\mathcal{S}}(X) & : \Big(\sm{f': A\to X}{g':A\to X}\Big.\\
& \qquad\qquad\Big.\prd{x:S} \mathsf{tr}_{W_{A\sqcup^{\mathcal{S}}B}(X)}(\glue(x),f'(f(x)))=g'(g(x))\Big).
\end{align*}
By the induction principle for $A\sqcup^{\mathcal{S}} B$ we have a section $\ind{\mathcal{S}}$ of $\mathsf{dgen}_{\mathcal{S}}$. Thus we obtain a section $\rec{\mathcal{S}}$ of $\mathsf{gen}_{\mathcal{S}}$. Our goal is now to show that $\rec{\mathcal{S}}$ is also a retraction of $\mathsf{gen}_{\mathcal{S}}$. We establish in \cref{lem:pushout_up_htpy} that
\begin{equation*}
(\mathsf{gen}_{\mathcal{S}}(\rec{\mathcal{S}}(\mathsf{gen}_{\mathcal{S}}(h)))= \mathsf{gen}_{\mathcal{S}}(h))
\to (\rec{\mathcal{S}}(\mathsf{gen}_{\mathcal{S}}(h))= h)
\end{equation*}
Then we obtain that $\rec{\mathcal{S}}$ is a retraction of $\mathsf{gen}_{\mathcal{S}}$ by using this implication and the fact that $\rec{\mathcal{S}}$ is a section of $\mathsf{gen}_{\mathsf{S}}$.
\end{proof}

\begin{lem}\label{lem:pushout_up_htpy}
Let $h,h':A\sqcup^{\mathcal{S}}B\to X$ be two functions. Then we have
\begin{equation*}
(\mathsf{gen}_{\mathcal{S}}(h)=\mathsf{gen}_{\mathcal{S}}(h'))\to (h=h').
\end{equation*}
\end{lem}

\begin{proof}
Suppose we have $\mathsf{gen}_{\mathcal{S}}(h)=\mathsf{gen}_{\mathcal{S}}(h')$. This type of equalities between triples is equivalent to the type of triples $(K,L,M)$ consisting of
\begin{align*}
K & : h\circ \inl \htpy h'\circ \inl \\
L & : h\circ \inr \htpy h'\circ \inr,
\end{align*}
and a homotopy $M$ witnessing that the square 
\begin{equation*}
\begin{tikzcd}
h\circ \inl\circ f \arrow[r,"{K\cdot f}"] \arrow[d,swap,"{h\cdot\glue}"] & h'\circ\inl\circ f \arrow[d,"{h'\cdot\glue}"] \\
h\circ \inr\circ f \arrow[r,swap,"{L\cdot g}"] & h'\circ\inr\circ g
\end{tikzcd}
\end{equation*}
of homotopies commutes. By function extensionality, our goal is equivalent to constructing a homotopy (i.e.~a dependent function) of type
\begin{equation*}
\prd{t:A\sqcup^{\mathcal{S}} B} f(t)=g(t).
\end{equation*}
We will construct such a function by the induction principle for $A\sqcup^{\mathcal{S}} B$. Therefore it suffices to construct
\begin{align*}
K & : h\circ \inl \htpy h'\circ \inl \\
L & : h\circ \inr \htpy h'\circ \inr \\
M' & : \mathsf{tr}_{E_{h,h'}}(\glue,K)=L
\end{align*}
The type of $M'$ is equivalent to the type of $M$, so we obtain the requested structure from our assumptions.
\end{proof}

\begin{comment}
\begin{cor}
Consider two commuting squares
\begin{equation*}
\begin{tikzcd}
S \arrow[r,"g"] \arrow[d,swap,"f"] & B \arrow[d,"j"] & S \arrow[r,"g"] \arrow[d,swap,"f"] & B \arrow[d,"{j'}"] \\
A \arrow[r,swap,"i"] & X & A \arrow[r,swap,"{i'}"] & {X'}
\end{tikzcd}
\end{equation*}
with homotopies $H:i\circ f\htpy j\circ g$ and $H':i'\circ f\htpy j'\circ g$. Furthermore, consider a map
\begin{equation*}
h:X\to X'
\end{equation*}
equipped with
\begin{align*}
K & : h\circ i\htpy i' \\
L & : h\circ j\htpy j' \\
M & : \ct{(h\cdot H)}{(L\cdot g)} \htpy \ct{(K\cdot f)}{H'}.
\end{align*}
If any two of the following three properties hold, then so does the third:
\begin{enumerate}
\item $X$ is a pushout.
\item $X'$ is a pushout.
\item $h$ is an equivalence.
\end{enumerate}
\end{cor}
\end{comment}

As a basic application we establish the universal property of suspensions.
\begin{cor}
Let $X$ and $Y$ be types. Then the map\index{universal property!of suspensions|textit}
\begin{equation*}
(\susp{X}\to Y)\to \sm{y,y':Y} X\to (y=y')
\end{equation*}
given by $f\mapsto (f(\inl(\ttt)),f(\inr(\ttt)),\ap{f}{\glue(\blank)})$ is an equivalence.
\end{cor}

\begin{proof}
We have equivalences
\begin{align*}
(\susp{X}\to Y) & \eqvsym \sm{y,y':\unit \to Y} X\to (y(\ttt)=y'(\ttt)) \\
& \eqvsym \sm{y,y':Y} X\to (y=y').\qedhere
\end{align*}
\end{proof}

\section{The pasting property for pushouts}
\begin{thm}\label{thm:pushout_pasting}
Consider the following configuration of commuting squares:\index{pushout!pasting property|textit}\index{pasting property!for pushouts|textit}
\begin{equation*}
\begin{tikzcd}
A \arrow[r,"i"] \arrow[d,swap,"f"] & B \arrow[r,"k"] \arrow[d,swap,"g"] & C \arrow[d,"h"] \\
X \arrow[r,swap,"j"] & Y \arrow[r,swap,"l"] & Z
\end{tikzcd}
\end{equation*}
with homotopies $H:j\circ f\htpy g\circ i$ and $K:l\circ g\htpy h\circ k$, and suppose that the square on the left is a pushout square. 
Then the square on the right is a pushout square if and only if the outer rectangle is a pushout square.
\end{thm}

\begin{proof}
Let $T$ be a type. Taking the exponent $T^{(\blank)}$ of the entire diagram of the statement of the theorem, we obtain the following commuting diagram
\begin{equation*}
\begin{tikzcd}
T^Z \arrow[r,"\blank\circ l"] \arrow[d,swap,"\blank\circ h"] & T^Y \arrow[d,swap,"\blank\circ g"] \arrow[r,"\blank\circ j"] & T^X \arrow[d,"\blank\circ f"] \\
T^C \arrow[r,swap,"\blank\circ k"] & T^B \arrow[r,swap,"\blank\circ i"] & T^A.
\end{tikzcd}
\end{equation*}
By the assumption that $Y$ is the pushout of $B\leftarrow A \rightarrow X$, it follows that the square on the right is a pullback square. It follows by \autoref{thm:pb_pasting} that the rectangle on the left is a pullback if and only if the outer rectangle is a pullback. Thus the statement follows by the second characterization in \autoref{thm:pushout_up}.
\end{proof}

\begin{lem}
Consider a map $f:A\to B$. Then the cofiber of the map $\inr:B\to \mathsf{cofib}_f$ is equivalent to the suspension $\susp{A}$ of $A$. 
\end{lem}

\begin{exercises}
\item \label{ex:span_rel}Use \cref{thm:choice,thm:fam_proj,cor:times_up_out} to show that the type 
\begin{equation*}
\mathsf{span}(A,B)\defeq \sm{S:\UU} (S\to A)\times (S\to B)
\end{equation*}
of small spans from $A$ to $B$ is equivalent to the type $A\to (B\to\UU)$ of small relations from $A$ to $B$.
\item \label{ex:pushout_equiv}Use \cref{thm:pushout_up,cor:pb_equiv,ex:equiv_precomp} to show that for any commuting square
\begin{equation*}
\begin{tikzcd}
S \arrow[r,"g"] \arrow[d,swap,"f","{\eqvsym}"'] & B \arrow[d,"j"] \\
A \arrow[r,swap,"i"] & C
\end{tikzcd}
\end{equation*} 
where $f$ is an equivalence, the square is a pushout square if and only if $j:B\to C$ is an equivalence.
Use this observation to conclude the following:
\begin{enumerate}
\item If $X$ is contractible, then $\susp X$ is contractible.
\item The cofiber of any equivalence is contractible.
\item The cofiber of a point in $B$ (i.e.~of a map of the type $\unit\to B$) is equivalent to $B$.
\item There is an equivalence $\eqv{X}{\join{\emptyt}{X}}$.
\item If $X$ is contractible, then $\join{X}{Y}$ is contractible. 
\item If $A$ is contractible, then there is an equivalence $\eqv{A\vee B}{B}$ for any pointed type $B$.
\end{enumerate}
\item \label{ex:join_propositions}Let $P$ and $Q$ be propositions.
\begin{subexenum}
\item Show that $\join{P}{Q}$ satisfies the \emph{universal property of disjunction}, i.e.~that for any proposition $R$, the map
\begin{equation*}
(\join{P}{Q}\to R)\to (P\to R)\times (Q\to R)
\end{equation*}
given by $f\mapsto (f\circ \inl,f\circ \inr)$, is an equivalence.
\item Use the proposition $R\defeq\iscontr(\join{P}{Q})$ to show that $\join{P}{Q}$ is again a proposition.
\end{subexenum}
\item Let $Q$ be a proposition, and let $A$ be a type. Show that the following are equivalent:
\begin{subexenum}
\item The map $(Q\to A)\to(\emptyt\to A)$ is an equivalence.
\item The type $A^Q$ is contractible.
\item There is a term of type $Q\to\iscontr(A)$.
\item The map $\inr:A\to \join{Q}{A}$ is an equivalence.
\end{subexenum}
\item Let $P$ be a proposition. Show that $\susp P$ is a set, with an equivalence
\begin{equation*}
\eqv{\Big(\inl(\ttt)=\inr(\ttt)\Big)}{P}.
\end{equation*}
\item Show that $\eqv{A\sqcup^{\mathcal{S}} B}{B\sqcup^{\mathcal{S}^{\mathsf{op}}} A}$, where $\mathcal{S^{\mathsf{op}}}\defeq (S,g,f)$ is the \define{opposite span} of $\mathcal{S}$. 
\item Use \cref{ex:pb_pi} to show that if
\begin{equation*}
\begin{tikzcd}
S \arrow[r] \arrow[d] & Y \arrow[d] \\
X \arrow[r] & Z
\end{tikzcd}
\end{equation*}
is a pushout square, then so is
\begin{equation*}
\begin{tikzcd}
A\times S \arrow[r] \arrow[d] & A\times Y \arrow[d] \\
A\times X \arrow[r] & A\times Z
\end{tikzcd}
\end{equation*}
for any type $A$.
\item Use \cref{ex:pb_prod} to show that if
\begin{equation*}
\begin{tikzcd}
S_1 \arrow[r] \arrow[d] & Y_1 \arrow[d] & S_2 \arrow[r] \arrow[d] & Y_2 \arrow[d] \\
X_1 \arrow[r] & Z_1 & X_2 \arrow[r] & Z_2
\end{tikzcd}
\end{equation*}
are pushout squares, then so is
\begin{equation*}
\begin{tikzcd}
S_1+S_2 \arrow[r] \arrow[d] & Y_1+ Y_2 \arrow[d] \\
X_1 +X_2 \arrow[r] & Z_1+Z_2. 
\end{tikzcd}
\end{equation*}
\item 
\begin{subexenum}
\item Consider a span $(S,f,g)$ from $A$ to $B$. Use \cref{ex:pb_diagonal} to show that the square
\begin{equation*}
\begin{tikzcd}[column sep=large]
S+S \arrow[d,swap,"{f+g}"] \arrow[r,"{[\idfunc,\idfunc]}"] & S \arrow[d,"{\inr\circ g}"] \\
A+B \arrow[r,swap,"{[\inl,\inr]}"] & A\sqcup^\mathcal{S} B
\end{tikzcd}
\end{equation*}
is again a pushout square.
\item Show that $\eqv{\susp X}{\join{\bool}{X}}$.
\end{subexenum}
\item Consider a commuting triangle
\begin{equation*}
\begin{tikzcd}[column sep=tiny]
A \arrow[rr,"h"] \arrow[dr,swap,"f"] & & B \arrow[dl,"g"] \\
& X
\end{tikzcd}
\end{equation*}
with $H:f\htpy g\circ h$. 
\begin{subexenum}
\item Construct a map $\mathsf{cofib}_{(h,H)}: \mathsf{cofib}_{g}\to \mathsf{cofib}_f$.
\item Use \cref{ex:pb_fib} to show that $\eqv{\mathsf{cofib}_{\mathsf{cofib}(h,H)}}{\mathsf{cofib}_h}$.
\end{subexenum}
\item \label{ex:sphere_null}Use \cref{ex:circle_connected} to show that for $n\geq 0$, $X$ is an $n$-type if and only if the map
\begin{equation*}
\lam{x}\mathsf{const}_x : X \to (\sphere{n+1}\to X)
\end{equation*}
is an equivalence.
\item 
\begin{subexenum}
\item Construct for every $f:X\to Y$ a function
\begin{equation*}
\susp f : \susp X\to \susp Y.
\end{equation*}
\item Show that if $f\htpy g$, then $\susp f \htpy \susp g$. 
\item Show that $\susp \idfunc[X]\htpy\idfunc[\susp X]$
\item Show that
\begin{equation*}
\susp(g\circ f)\htpy (\susp g)\circ (\susp f).
\end{equation*}
for any $f:X\to Y$ and $g:Y\to Z$.
\end{subexenum}
\item Consider a commuting diagram of the form
\begin{equation*}
\begin{tikzcd}
A_0 & B_0 \arrow[l] \arrow[r] & C_0 \\
A_1 \arrow[u] \arrow[d] & B_1 \arrow[u] \arrow[r] \arrow[d] \arrow[l] & C_1 \arrow[u] \arrow[d] \\
A_2 & B_2 \arrow[l] \arrow[r] & C_2
\end{tikzcd}
\end{equation*}
with homotopies filling the (small) squares. Use \cref{ex:pb_3by3} to construct an equivalence
\begin{align*}
& (A_0\sqcup^{B_0} C_0) \sqcup^{(A_1\sqcup^{B_1} C_1)} (A_2 \sqcup^{B_2} C_2) \\
& \qquad \eqvsym (A_0 \sqcup^{A_1} A_2) \sqcup^{(B_0\sqcup^{B_1} B_2)} (C_0\sqcup^{C_1} C_2).
\end{align*}
This is known as the \define{3-by-3 lemma}\index{3-by-3 lemma!for pushouts} for pushouts.
\item 
\begin{subexenum}
\item Let $I$ be a type, and let $A$ be a family over $I$. Construct an equivalence
\begin{equation*}
\eqv{\Big(\bigvee\nolimits_{(i:I)}\susp A_i\Big)}{\susp\Big(\bigvee\nolimits_{(i:I)}A_i\Big)}.
\end{equation*}
\item Show that for any type $X$ there is an equivalence
\begin{equation*}
\eqv{\Big(\bigvee\nolimits_{(x:X)}\bool\Big)}{X+\unit}.
\end{equation*}
\item Construct an equivalence
\begin{equation*}
\eqv{\susp(\mathsf{Fin}(n+1))}{\bigvee\nolimits_{(i:\mathsf{Fin}(n))}\sphere{1}}.
\end{equation*}
\end{subexenum}
\item Show that $\eqv{\join{\mathsf{Fin}(n+1)}{\mathsf{Fin}(m+1)}}{\bigvee\nolimits_{(i:\mathsf{Fin}(n\cdot m))}\sphere{1}}$, for any $n,m:\N$. 
\end{exercises}
