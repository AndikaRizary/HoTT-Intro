% !TEX root = hott_intro.tex

\chapter{Contractible types and contractible maps}
\chaptermark{Contractible types and maps}

In this lecture we introduce the notions of contractible types and maps. A contractible type is a type which has, up to identification, only one term. In other words, a contractible type is a type that comes equipped with a point, and an identification of this point with any point.

We may think of contractible types as singletons up to homotopy, and indeed we show that the unit type is an example of a contractible type. Moreover, we show that contractible typs satisfy an induction principle that is very similar to the induction principle of the unit type, provided that we formulate the computation rule using the identity type rather than postulating a judgmental computation rule. Another example of a contractible type is the total space of the family of identifications with a fixed starting point.

We then introduce the notion of fiber of a map, which is the type theoretic analogue of the pre-image of a map, and we say that a map in contractible if all its fibers are contractible. Thus, a map is contractible if the preimage at any point in the codomain is a singleton. This condition is of course analogous to the set theoretic notion of bijective map, which suggests that on the type theoretical side of things a map should be contractible if and only if it is an equivalence.

The forward direction of this claim is straightforward, and we prove this direction immediately in \cref{thm:equiv_contr}. The converse direction can be done directly, but it is certainly more involved. Therefore we prepare the proof of the converse direction by first characterizing the identity type of a fiber of a map. Then we show that any equivalence $e$ can be given the structure of a invertible map with an additional coherence relating the homotopies
\begin{equation*}
  e\circ e^{-1}\htpy \idfunc,\qquad\text{and}\qquad e^{-1}\circ e\htpy\idfunc,
\end{equation*}
and finally we use these observations in \cref{thm:contr_equiv} to conclude that the fibers of any equivalence must be contractible.

\section{Contractible types}

\begin{defn}
  We say that a type $A$ is \define{contractible}\index{contractible!type|textbf} if it comes equipped with a term of type
  \begin{equation*}
    \iscontr(A) \defeq \sm{c:A}\prd{x:A}c=x.
  \end{equation*}
  Given a term $(c,C):\iscontr(A)$, we call $c:A$ the \define{center of contraction}\index{center of contraction|textbf} of $A$, and we call $C:\prd{x:A}c=x$ the \define{contraction}\index{contraction} of $A$.
\end{defn}

\begin{rmk}
Suppose $A$ is a contractible type with center of contraction $c$ and contraction $C$. Then the type of $C$ is (judgmentally) equal to the type
\begin{equation*}
\mathsf{const}_c\htpy\idfunc[A].
\end{equation*}
In other words, the contraction $C$ is a \emph{homotopy} from the constant function to the identity function.
\end{rmk}

\begin{defn}
  Suppose $A$ comes equipped with a term $a:A$. Then we say that $A$ satisfies \define{singleton induction}\index{singleton induction|textbf} if for every type family $B$ over $A$, the map
\begin{equation*}
\Big(\prd{x:A}B(x)\Big)\to B(a)
\end{equation*}
given by $f\mapsto f(a)$ has a section. In other words, we have a function and a homotopy
\begin{align*}
\mathsf{sing\usc{}ind}_{a} & : B(a)\to \prd{x:A}B(x) \\
\mathsf{sing\usc{}comp}_{a} & : \prd{b:B(a)} \mathsf{sing\usc{}ind}_{A,a}(b,a)=b.
\end{align*}
\end{defn}

\begin{rmk}
  Note that the "computation rule" in the singleton induction for $A$ is stated using an \emph{identification} rather than as a judgmental equality. 
\end{rmk}

\begin{thm}\label{thm:contractible}
Let $A$ be a type. The following are equivalent:
\begin{enumerate}
\item The type $A$ is contractible.
\item The type $A$ comes equipped with a term $a:A$, and satisfies signleton induction.
\end{enumerate}
\end{thm}

\begin{proof}
Suppose $A$ is contractible with center of contraction $c$ and contraction $C$. 
First we observe that, without loss of generality, we may assume that $C$ comes equipped with an identification $p:C(c)=\refl{c}$.
To see this, note that we can always define a new contraction $C'$ by
\begin{equation*}
C'(x)\defeq\ct{C(c)^{-1}}{C(x)},
\end{equation*}
which satisfies the requirement by the left inverse law, constructed in \cref{defn:id_invlaw}.

To show that $A$ satisfies singleton induction let $B$ be a type family over $A$ equipped with $b:B(a)$. We define $\mathsf{sing\usc{}ind}(b):\prd{x:A}B(x)$ by $\lam{x}\mathsf{tr}_B(C(x),b)$. To see that $\mathsf{sing\usc{}ind}(c)=b$ note that we have
\begin{equation*}
\begin{tikzcd}
\mathsf{tr}_B(C(c),b) \arrow[r,equals,"\ap{\lam{\omega}\mathsf{tr}_B(\omega,b)}{p}"] &[4em] \mathsf{tr}_B(\refl{c},b) \arrow[r,equals,"\refl{b}"] & b.
\end{tikzcd}
\end{equation*}
This completes the proof that $A$ satisfies singleton induction.

For the converse, suppose that $a:A$ and that $A$ satisfies singleton induction. Our goal is to show that $A$ is contractible. For the center of contraction we take the term $a:A$. By singleton induction applied to $B(x)\defeq a=x$ we have the map 
\begin{equation*}
\mathsf{sing\usc{}ind}_{A,a} : a=a \to \prd{x:A}a=x.
\end{equation*}
Therefore $\mathsf{sing\usc{}ind}_{A,a}(\refl{a})$ is a contraction.
\end{proof}

\begin{eg}
By definition the unit type\index{unit type!contractibility} $\unit$ satisfies singleton induction, so it is contractible.
\end{eg}

\begin{thm}\label{thm:total_path}
For any $x:A$, the type
\begin{equation*}
\sm{y:A}x=y
\end{equation*}
is contractible.\index{identity type!contractibility of total space|textit}
\end{thm}

In the following proof we stress the analogy of path induction with singleton elimination, as we did in class. An alternative proof can be found in Lemma 3.11.8 of the HoTT book \cite{hottbook}.

\begin{proof}
We will prove the statement by showing that $\sm{y:A}x=y$ satisfies singleton induction, applying \cref{thm:contractible}.

We have the term $(x,\refl{x}):\sm{y:A}x=y$. Thus we need to show that for any type family $B$ over $\sm{x:A}x=y$, the map
\begin{equation*}
\Big(\prd{t:\sm{x:A}x=y}B(t)\Big)\to B((x,\refl{x}))
\end{equation*}
has a section. Note that we have the composite of maps
\begin{equation*}
\begin{tikzcd}[column sep=large]
B((x,\refl{x})) \arrow[r,"\mathsf{path\usc{}ind}_x"] & \prd{y:A}{p:x=y}B((y,p)) \arrow[r,"\ind{\Sigma}"] & \prd{t:\sm{y:A}x=y}B(t)
\end{tikzcd}
\end{equation*}
which we take as our definition of $\mathsf{sing\usc{}ind}$. 
Moreover, by the computation rules we have
\begin{equation*}
\ind{\Sigma}(\mathsf{path\usc{}ind}_x(b),(x,\refl{x}))\jdeq b.
\end{equation*}
Thus, for $\mathsf{sing\usc{}comp}$ we simply take $\lam{b}\refl{b}$.
\end{proof}

\section{Contractible maps}
\begin{defn}
Let $f:A\to B$ be a function, and let $b:B$. The \define{fiber}\index{fiber|textbf}\index{homotopy fiber|see {fiber}} of $f$ at $b$ is defined to be the type
\begin{equation*}
\fib{f}{b}\defeq\sm{a:A}f(a)=b.
\end{equation*}
\end{defn}

In other words, the fiber of $f$ at $b$ is the type of $a:A$ that get mapped by $f$ to $b$.
One may think of the fiber as a type theoretic version of the pre-image\index{pre-image|see {fiber}} of a point.

It will be useful to have a characterization of the identity type of a fiber, so we will make such a characterization immediately.

  \begin{defn}
    Let $f:A \to B$ be a map, and let $(x,p),(x',p'):\fib{f}{y}$ for some $y:B$.
    Then we define
    \begin{equation*}
      \mathsf{Eq\usc{}fib}_f((x,p),(x',p'))\defeq \sm{\alpha:x=x'}p=\ct{\ap{f}{\alpha}}{p'}
    \end{equation*}
    The relation $\mathsf{Eq\usc{}fib}_f:\fib{f}{y}\to\fib{f}{y}\to\UU$ is a reflexive relation, since we have
    \begin{equation*}
      \lam{(x,p)}(\refl{x},\refl{p}):\prd{(x,p):\fib{f}{y}}\mathsf{Eq\usc{}fib}_f((x,p),(x,p)).
    \end{equation*}
  \end{defn}

  \begin{lem}
    Consider a map $f:A\to B$ and let $y:B$. The canonical map
    \begin{equation*}
      ((x,p)=(x',p'))\to\mathsf{Eq\usc{}fib}_f((x,p),(x',p'))
    \end{equation*}
    induced by the reflexivity of $\mathsf{Eq\usc{}fib}_f$ is an equivalence for any $(x,p),(x',p'):\fib{f}{y}$.
  \end{lem}

  \begin{proof}
    The converse map
    \begin{equation*}
      \mathsf{Eq\usc{}fib}_f((x,p),(x',p'))\to ((x,p)=(x',p'))
    \end{equation*}
    is easily defined by $\Sigma$-induction, and then path induction twice. The homotopies witnessing that this converse map is indeed a right inverse as well as a left inverse is similarly constructed by induction.
  \end{proof}

  Now we arrive at the notion of contractible map.

\begin{defn}
We say that a function $f:A\to B$ is \define{contractible}\index{contractible!map|textbf} if there is a term of type
\begin{equation*}
\iscontr(f)\defeq\prd{b:B}\iscontr(\fib{f}{b}).
\end{equation*}
\end{defn}

\begin{thm}\label{thm:equiv_contr}
Any contractible map is an equivalence.\index{contractible!map!is an equivalence|textit}
\end{thm}

\begin{proof}
Let $f:A\to B$ be a contractible map. Using the center of contraction of each $\fib{f}{y}$, we obtain a term of type
\begin{align*}
\lam{y}\pairr{g(y),G(y)}:\prd{y:B}\fib{f}{y}.
\end{align*}
Thus, we get map $g:B\to A$, and a homotopy $G:\prd{y:B} f(g(y))=y$. In other words, we get a section of $f$.

It remains to construct a retraction of $f$. Taking $g$ as our retraction, we have to show that $\prd{x:A} g(f(x))=x$. Note that we get an identification $p:f(g(f(x)))=f(x)$ since $g$ is a section of $f$. It follows that $(g(f(x)),p):\fib{f}{f(x)}$. Moreover, since $\fib{f}{f(x)}$ is contractible we get an identification $q:\pairr{g(f(x)),p}=\pairr{x,\refl{f(x)}}$. The base path $\ap{\proj 1}{q}$ of this identification is an identification of type $g(f(x))=x$, as desired.
\end{proof}

\section{Equivalences are contractible maps}

In \cref{thm:contr_equiv} we will show the converse to \autoref{thm:equiv_contr}, that any equivalence is a contractible map. We will do this in two steps.

First we introduce a new notion of \emph{coherently invertible map}, for which we can easily show that such maps have contractible fibers. Then we show that any equivalence is a coherently invertible map.

  Recall that an invertible map is a map $f:A\to B$ equipped with $g:B\to A$ and homotopies
  \begin{equation*}
    G : f\circ g \htpy \idfunc\qquad\text{and}\qquad H:g\circ f\htpy \idfunc.
  \end{equation*}
  Then we observe that both $G \cdot f$ and $f \cdot H$ are homotopies of the same type
  \begin{equation*}
    f\circ g\circ f \htpy f.
  \end{equation*}
  A coherently invertible map is an invertible map for which there is a further homotopy $G \cdot f\htpy f\cdot H$.

  \begin{defn}
    Consider a map $f:A\to B$. We say that $f$ is \define{coherently invertible} if it comes equipped with
    \begin{align*}
      g & : B \to A \\
      G & : f \circ g \htpy \idfunc \\
      H & : g \circ f \htpy \idfunc \\
      K & : G \cdot f \htpy f \cdot H.
    \end{align*}
    We will write $\mathsf{is\usc{}coh\usc{}invertible}(f)$ for the type of quadruples $(g,G,H,K)$.
  \end{defn}

  Although we will encounter the notion of coherently invertible map on some further occasions, the following lemma is our main motivation for considering it.

  \begin{lem}\label{lem:contr-inv}
    Any coherently invertible map has contractible fibers.
  \end{lem}

  \begin{proof}
    Consider a map $f:A\to B$ equipped with
    \begin{align*}
      g & : B \to A \\
      G & : f \circ g \htpy \idfunc \\
      H & : g \circ f \htpy \idfunc \\
      K & : G \cdot f \htpy f \cdot H,
    \end{align*}
    and let $y:B$. Our goal is to show that $\fib{f}{y}$ is contractible. For the center of contraction we take $(g(y),G(y))$. In order to construct a contraction, it suffices to construct a term of type
    \begin{equation*}
      \prd{x:A}{p:f(x)=y}\mathsf{Eq\usc{}fib}_f((g(y),G(y)),(x,p)).
    \end{equation*}
    By path induction on $p:f(x)=y$ it suffices to construct a term of type
    \begin{equation*}
      \prd{x:A}\mathsf{Eq\usc{}fib}_f((g(f(x)),G(f(x))),(x,\refl{f(x)})).
    \end{equation*}
    By definition of $\mathsf{Eq\usc{}fib}_f$, we have to construct a term of type
    \begin{equation*}
      \prd{x:A}\sm{\alpha:g(f(x))=x}G(f(x))=\ct{\ap{f}{\alpha}}{\refl{f(x)}}.
    \end{equation*}
    Such a term is constructed as $\lam{x}(H(x),K'(x))$, where the homotopy $H:g\circ f\htpy \idfunc$ is given by assumption, and the homotopy
    \begin{align*}
      K' & : \prd{x:A}G(f(x))=\ct{\ap{f}{H(x)}}{\refl{f(x)}}
    \end{align*}
    is defined as
    \begin{equation*}
      K'\defeq \ct{K}{\mathsf{htpy\usc{}right\usc{}unit}(f\cdot H)^{-1}}.\qedhere
    \end{equation*}
  \end{proof}

  Our next goal is to show that for any map $f:A\to B$ equipped with
  \begin{equation*}
    g:B\to A,\qquad G:f\circ g \htpy \idfunc,\qquad\text{and}\qquad H:g\circ f\htpy \idfunc,
  \end{equation*}
  we can improve the homotopy $G$ to a new homotopy $G':f\circ g\htpy \idfunc$ for which there is a further homotopy
  \begin{equation*}
    f\cdot H\htpy G'\cdot f.
  \end{equation*}
  Note that this situation is analogous to the situation in the proof of \cref{thm:contractible}, where we improved the contraction $C$ so that it satisfied $C(c)=\refl{}$. The extra coherence $f\cdot H\htpy G'\cdot f$ is then used in the proof that the fibers of an equivalence are contractible.

\begin{defn}\label{defn:htpy_nat}\index{homotopy!naturality|textbf}
Let $f,g:A\to B$ be functions, and consider $H:f\htpy g$ and $p:x=y$ in $A$. We define identification
\begin{equation*}
\mathsf{htpy\usc{}nat}(H,p) \defeq  :\ct{\ap{f}{p}}{H(y)}=\ct{H(x)}{\ap{g}{p}}
\end{equation*}
witnessing that the square
\begin{equation*}
\begin{tikzcd}
f(x) \arrow[r,equals,"H(x)"] \arrow[d,equals,swap,"\ap{f}{p}"] & g(x) \arrow[d,equals,"\ap{g}{p}"] \\
f(y) \arrow[r,equals,swap,"H(y)"] & g(y)
\end{tikzcd}
\end{equation*}
commutes. This square is also called the \define{naturality square} of the homotopy $H$ at $p$.
\end{defn}

\begin{constr}
  By path induction on $p$ it suffices to construct an identification
  \begin{equation*}
    \ct{\ap{f}{\refl{x}}}{H(x)}=\ct{H(x)}{\ap{g}{\refl{x}}}
  \end{equation*}
  since $\ap{f}{\refl{x}}\jdeq \refl{f(x)}$ and $\ap{g}{\refl{x}}\jdeq\refl{g(x)}$, and since $\ct{\refl{f(x)}}{H(x)}\jdeq H(x)$, we see that the path $\mathsf{right\usc{}unit}(H(x))^{-1}$ is of the asserted type.
\end{constr}

\begin{defn}\label{defn:retraction_swap}
Consider $f:A\to A$ and $H: f\htpy \idfunc[A]$. We construct an identification $H(f(x))=\ap{f}{H(x)}$, for any $x:A$.
\end{defn}

\begin{constr}
By the naturality of homotopies with respect to identifications the square
\begin{equation*}
\begin{tikzcd}[column sep=large]
ff(x) \arrow[d,swap,equals,"\ap{f}{H(x)}"] \arrow[r,equals,"H(f(x))"] & f(x) \arrow[d,equals,"H(x)"] \\
f(x) \arrow[r,swap,equals,"H(x)"] & x
\end{tikzcd}
\end{equation*}
commutes. This gives the desired identification $H(f(x))=\ap{f}{H(x)}$.
\end{constr}

\begin{lem}\label{lem:coherently-invertible}
  Let $f:A\to B$ be a map, and consider $(g,G,H):\mathsf{has\usc{}inverse}(f)$. Then there is a homotopy $G':f\circ g\htpy \idfunc$ equipped with a further homotopy
  \begin{equation*}
    K : G'\cdot f \htpy f\cdot H.
  \end{equation*}
  Thus we obtain a map $\mathsf{has\usc{}inverse}(f)\to\mathsf{is\usc{}coh\usc{}invertible}(f)$.
\end{lem}

\begin{proof}
  For each $y:B$, we construct the identification $G'(y)$ as the concatenation
  \begin{equation*}
    \begin{tikzcd}
      fg(y) \arrow[r,equals,"{G(fg(y))}^{-1}"] &[2.5em] fgfg(y) \arrow[r,equals,"\ap{f}{H(g(y))}"] &[2.5em] fg(y) \arrow[r,equals,"G(y)"] & y.
\end{tikzcd}
  \end{equation*}
  In order to construct a homotopy $G'\cdot f\htpy f\cdot H$, it suffices to show that the square
  \begin{equation*}
    \begin{tikzcd}[column sep=8em]
      fgfgf(x) \arrow[r,equals,"{G(fgf(x))}"] \arrow[d,equals,swap,"\ap{f}{H(gf(x))}"] & fgf(x) \arrow[d,equals,"\ap{f}{H(x)}"] \\
      fgf(x) \arrow[r,equals,swap,"G(f(x))"] & f(x)
    \end{tikzcd}
  \end{equation*}
  commutes for every $x:A$.
  Recall from \autoref{defn:retraction_swap} that we have $H(gf(x))=\ap{gf}{H(x)}$. Using this identification, we see that it suffices to show that the square
  \begin{equation*}
    \begin{tikzcd}[column sep=8em]
      fgfgf(x) \arrow[r,equals,"G(fgf(x))"] \arrow[d,equals,swap,"\ap{fgf}{H(x)}"] & fgf(x) \arrow[d,equals,"\ap{f}{H(x)}"] \\
      fgf(x) \arrow[r,equals,swap,"G(f(x))"] & f(x)
    \end{tikzcd}
  \end{equation*}
  commutes. Now we observe that this is just a naturality square the homotopy $Gf:fgf\htpy f$, which commutes by \autoref{defn:htpy_nat}.
\end{proof}

Now we put the pieces together to conclude that any equivalence has contractible fibers.

\begin{thm}\label{thm:contr_equiv}
Any equivalence is a contractible map.\index{equivalence!is a contractible map|textit}
\end{thm}

\begin{proof}
  We have seen in \cref{lem:contr-inv} that any coherently invertible map is a contractible map. Moreover, any equivalence has the structure of an invertible map by \cref{thm:inv_equiv}, and any invertible map is coherently invertible by \cref{lem:coherently-invertible}.
\end{proof}

\begin{cor}\label{cor:contr_path}
Let $A$ be a type, and let $a:A$. Then the type
\begin{equation*}
\sm{x:A}x=a
\end{equation*}
is contractible.
\end{cor}

\begin{proof}
By \autoref{thm:id_equiv}, the identity function is an equivalence. Therefore, the fibers of the identity function are contractible by \autoref{thm:contr_equiv}. Note that $\sm{x:A}x=a$ is exactly the fiber of $\idfunc[A]$ at $a:A$.
\end{proof}

\begin{comment}
\begin{proof}
We have the term $(a,\refl{a}):\sm{x:A}a=x$, which we take for the center of contraction. To construct the contraction, we have to show that
\begin{equation*}
\prd{p:\sm{x:A}a=x} (a,\refl{a})=p.
\end{equation*}
By the induction principle for dependent pair types it suffices to construct a term of type
\begin{equation*}
\prd{x:A}{p:a=x} (a,\refl{a})=(x,p)
\end{equation*}
Note that we may proceed here by path induction on $p$. That is, it suffices to consider the case $p\jdeq\refl{a}$, and show that $(a,\refl{a})=(a,\refl{a})$. Here we choose $\refl{(a,\refl{a})}$.
\end{proof}
\end{comment}

\begin{exercises}
\item \label{ex:prop_contr}Show that if $A$ is contractible, then for any $x,y:A$ the identity type $x=y$ is also contractible.\index{contractible!type!identity types of}
\item \label{ex:contr_retr}Suppose that $A$ is a retract of $B$. Show that\index{contractible!type!retracts of}
  \begin{equation*}
    \iscontr(B)\to\iscontr(A).
  \end{equation*}
\item \label{ex:contr_equiv}
  \begin{subexenum}
  \item Show that for any type $A$, the map $\mathsf{const}_\ttt : A\to \unit$ is an equivalence if and only if $A$ is contractible.\index{contractible!type!equivalence with $\unit$}
  \item Apply \cref{ex:3_for_2} to show that for any map $f:A\to B$, if any two of the three assertions\index{contractible!type!three@{3-for-2}}
    \begin{enumerate}
    \item $A$ is contractible
    \item $B$ is contractible
    \item $f$ is an equivalence
    \end{enumerate}
    hold, then so does the third.
  \end{subexenum}
\item Show that for any two types $A$ and $B$, the following are equivalent:
  \begin{enumerate}
  \item Both $A$ and $B$ are contractible.
  \item The type $A\times B$ is contractible.
  \end{enumerate}
\item \label{ex:contr_in_sigma} Let $C$ be a contractible type with center of contraction $c:C$. Furthermore, let $B$ be a type family over $C$. Show that the map $b\mapsto\pairr{c,b}:B(c)\to\sm{x:C}B(x)$ is an equivalence.
\item \label{ex:proj_fiber}Let $B$ be a family of types over $A$, and consider the projection map 
  \begin{equation*}
    \proj 1 : \big(\sm{x:A}B(x)\big)\to A.
  \end{equation*}
  Show that for any $a:A$, the map
  \begin{equation*}
    \lam{((x,y),p)} \mathsf{tr}_B(p,y) : \fib{\proj 1}{a} \to B(a),
  \end{equation*}
  is an equivalence. Conclude that $\proj 1$ is an equivalence if and only if each $B(a)$ is contractible.
\item \label{ex:fib_replacement}Construct for any map $f:A\to B$ an equivalence $e:\eqv{A}{\sm{y:B}\fib{f}{y}}$ and a homotopy $H:f\htpy \proj 1\circ e$ witnessing that the triangle
  \begin{equation*}
    \begin{tikzcd}[column sep=0em]
      A \arrow[rr,"e"] \arrow[dr,swap,"f"] & & \sm{y:B}\fib{f}{y} \arrow[dl,"\proj 1"] \\
      \phantom{\sm{y:B}\fib{f}{y}} & B
    \end{tikzcd}
  \end{equation*}
  commutes. The projection $\proj 1 : (\sm{y:B}\fib{f}{y})\to B$ is sometimes also called the \define{fibrant replacement}\index{fibrant replacement} of $f$.
\end{exercises}
