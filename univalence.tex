\chapter{The univalence axiom}

\section{Type extensionality}

The univalence axiom asserts that equivalent types are equal. It is considered to be an \emph{extensionality principle} for types. In the following theorem we give several equivalent ways of stating this.

\begin{thm}\label{thm:univalence}
The following are equivalent:
\begin{enumerate}
\item The \define{univalence axiom}: for any $A:\UU$ the map
\begin{equation*}
\mathsf{equiv\usc{}eq}\defeq \mathsf{eq\usc{}ind}(\idfunc[A]) : \prd{B:\UU} (\id{A}{B})\to(\eqv{A}{B}).
\end{equation*}
is a fiberwise equivalence.
\item The type
\begin{equation*}
\sm{B:\UU}\eqv{A}{B}
\end{equation*}
is contractible for each $A:\UU$.
\item The principle of \define{equivalence induction}: for every $A:\UU$ and for every type family
\begin{equation*}
P:\prd{B:\UU} (\eqv{A}{B})\to \mathsf{Type},
\end{equation*}
the map
\begin{equation*}
\Big(\prd{B:\UU}{e:\eqv{A}{B}}P(B,e)\Big)\to P(A,\idfunc[A])
\end{equation*}
given by $f\mapsto f(A,\idfunc[A])$ has a section.
\end{enumerate}
\end{thm}

From now on we will assume that the univalence axiom holds.

\section{Consequences of the univalence axiom}

The first application of the univalence axiom was Voevodsky's proof of function extensionality.

\begin{thm}
The univalence axiom implies function extensionality.
\end{thm}

\begin{proof}
By \autoref{thm:funext_wkfunext} it suffices to show that univalence implies the weak principle of function extensionality.

To see this, we first note that post-composing with an equivalence $\eqv{X}{Y}$ gives an equivalence $\eqv{(A\to X)}{(A\to Y)}$. This follows immediately by equivalence induction, \autoref{thm:univalence}
\end{proof}

\begin{thm}
For any type $X:\UU$ the map
\begin{equation*}
\Big(\sm{A:\UU}A\to X\Big)\to \Big(X\to\UU\Big)
\end{equation*}
given by $(A,f)\mapsto \mathsf{fib}_f$ is an equivalence. 
\end{thm}

\begin{thm}
For any map $f:A\to B$ in $\UU$, the square
\begin{equation*}
\begin{tikzcd}[column sep=10em]
A \arrow[d,swap,"f"] \arrow[r,"a\mapsto{(\fib{f}{f(a)},(a,\refl{f(a)}))}"] & \sm{X:\UU}X \arrow[d,"\proj 1"] \\
B \arrow[r,swap,"\mathsf{fib}_{f}"] & \UU
\end{tikzcd}
\end{equation*}
is a pullback square.
\end{thm}

\begin{thm}
Assuming the univalence axiom on $\UU$, the map
\begin{equation*}
\idtypevar{X}:X\to (X\to\UU)
\end{equation*}
is an embedding, for any type $X:\UU$.
\end{thm}

\begin{proof}
Let $a:A$. By function extensionality it suffices to show that the canonical map
\begin{equation*}
(a=b)\to \idtypevar{A}(a)\htpy\idtypevar{A}(b)
\end{equation*}
that sends $\refl{a}$ to $\lam{x}\refl{(a=x)}$ is an equivalence for every $b:A$, and by univalence it therefore suffices to show that the canonical map
\begin{equation*}
(a=b)\to \prd{x:A}\eqv{(a=x)}{(b=x)}
\end{equation*}
that sends $\refl{a}$ to $\lam{x}\idfunc[(a=x)]$ is an equivalence for every $b:B$. To do this we employ the type theoretic Yoneda lemma, \autoref{thm:yoneda}.

By the type theoretic Yoneda lemma we have an equivalence
\begin{equation*}
\Big(\prd{x:A} (b=x)\to (a=x)\Big)\to (a=b)
\end{equation*}
given by $\lam{f} f(b,\refl{b})$, for every $b:A$. Note that any fiberwise map $\prd{x:A}(b=x)\to (a=x)$ induces an equivalence of total spaces by \autoref{ex:contr_equiv}, since their total spaces are are both contractible by \autoref{cor:contr_path}. It follows that we have an equivalence
\begin{equation*}
\varphi_b:\Big(\prd{x:A} \eqv{(b=x)}{(a=x)}\Big)\to (a=b)
\end{equation*}
given by $\lam{f} f(b,\refl{b})$, for every $b:A$. 

Note that $\varphi_a(\lam{x}\idfunc[(a=x)])\jdeq\refl{a}$. Therefore it follows by another application of \autoref{thm:yoneda} that the unique family of maps 
\begin{equation*}
\alpha_b:(a=b)\to \Big(\prd{x:A} \eqv{(b=x)}{(a=x)}\Big)
\end{equation*}
that satisfies $\alpha_a(\refl{a})=\lam{x}\idfunc[(a=x)]$ is a fiberwise section of $\varphi$. 
It follows that $\alpha$ is a fiberwise equivalence. Now the proof is completed by reverting the direction of the fiberwise equivalences in the codomain.
\end{proof}

\section{Locally small types}
It is a trivial observation, but nevertheless of fundamental importance, that by the univalence axiom the identity types of $\UU$ are equivalent to types in $\UU$, namely $\eqv{(A=B)}{(\eqv{A}{B})}$. In other words, the universe is \emph{locally small}.

\begin{defn}
A type $X$ is said to be \define{locally small} if for every $x,y:X$ there is a type $x='y:\UU$ and an equivalence
\begin{equation*}
\eqv{(x=y)}{(x='y)}.
\end{equation*}
\end{defn}

\begin{exercises}
\item \label{ex:tr_ap} Show that for any $P:X\to \UU$ and any $p:x=y$ in $X$, we have
\begin{equation*}
\mathsf{equiv\usc{}eq}(\ap{P}{p})=\mathsf{tr}^P(p).
\end{equation*}
\item Use the univalence axiom to show that the type $\sm{A:\UU}\iscontr(A)$ of all contractible types in $\UU$ is contractible.
\item Use the univalence axiom to show that the type $\sm{P:\prop}P$ is contractible.
\item Show that $\eqv{(\eqv{\bool}{\bool})}{\bool}$, and conclude by the univalence axiom that the universe is not a set.
\end{exercises}
