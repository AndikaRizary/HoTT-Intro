% arara: makechapters: {items: [syllabus, dtt, pi, inductive, identity, equivalences, contractible, fundamental, hierarchy, funext, pullback, univalence]}

\documentclass[11pt]{memoir} %[ebook,10pt,oneside]
  
\usepackage{hott}

\title{Introduction to homotopy type theory}
\author{Egbert Rijke}
\date{Carnegie Mellon University\\Pittsburgh PA\\Spring 2018}%\\Version: \today}
%\address{Carnegie Mellon University}
%\email{erijke@andrew.cmu.edu}

\pretitle{\begin{center}\textsc\bgroup\LARGE}
\posttitle{\egroup\end{center}\vspace{2cm}}
\preauthor{\begin{center}\textsc\bgroup\Large}\postauthor{\egroup\end{center}\vfill}
\predate{\begin{center}\textsc\bgroup}{\postdate{\egroup\end{center}}


  % The following is to avoid overfull hboxes in the table of contents.
% https://tex.stackexchange.com/questions/49887/overfull-hbox-warning-for-toc-entries-when-using-memoir-documentclass
%\renewcommand*{\cftdotsep}{1}
\setpnumwidth{2em}
\setrmarg{3em}
\setlength{\cftchapternumwidth}{2em}
\setlength{\cftsectionindent}{2em}
\setlength{\cftsectionnumwidth}{2em}
\setlength{\cftsubsectionindent}{4em}
\setlength{\cftsubsectionnumwidth}{3em}

% We number sections independently of chapters, and subsections will be
% numbered too.

\counterwithout{section}{chapter}
\settocdepth{subsection}

% We set up the exercise environment, which produces list environment in a new
% unnumbered subsection that also gets mentioned in the table of contents.

\newlist{exenum}{enumerate}{1}
\setlist[exenum]{noitemsep,label=\thesection.\arabic*}
  %,ref=\thechapter.\arabic*}

\crefname{exenumi}{Exercise}{Exercises}
  
\newlist{subexenum}{enumerate}{1}
\setlist[subexenum]{noitemsep,label=(\alph*),ref=\theexenumi.\alph*}
\crefname{subexenumi}{Exercise}{Exercises}
  
\newenvironment{exercises}
{%
\subsection*{Exercises}%
\addcontentsline{toc}{subsection}{Exercises}%
\sectionmark{Exercises}%
\begin{exenum}}
{%
\end{exenum}}
      
\addbibresource{bibliography.bib}

\makeindex

\begin{document}

\frontmatter

\begin{titlingpage}
  \maketitle 
\end{titlingpage}

\tableofcontents

%\chapter{Introduction}

\emph{These are notes and exercise sets for the course Introduction to Homotopy Type Theory, taught at Carnegie Mellon University in the spring semester of 2018.} 

\bigskip
\noindent Homotopy Type Theory (HoTT) is an emerging field of mathematics and computer science that extends Martin-Löf's dependent type theory by the addition of the univalence axiom and higher inductive types. In HoTT we think of types as spaces, dependent types as fibrations, and of the identity types as path spaces.

\begin{figure}
\makebox[\textwidth][c]{%
\begin{tikzpicture}
  [small mindmap,
  every node/.style={concept, execute at begin node=\hskip0pt},
  concept color = black!20,
  grow cyclic,
  level 1/.append style = {level distance = 5cm, sibling angle = 90}, 
  level 2/.append style={level distance = 2.5cm, sibling angle = 45}]
  \node [root concept] {\scshape Homotopy Type Theory}
    child { node {\scshape Computer proof assistants}
      child { node {\tiny Automath, Coq, Agda, Lean, \ldots}}
      child { node {Four color theorem}}
      child { node {Feit-Thompson theorem}}
      child { node {Kepler conjecture}}
      child { node {Formal abstracts project}}}
    child { node {\scshape Dependent type theories}
      child { node {$\lambda$-calculus}}
      child { node {\tiny Calculus of inductive constructions}}
      child { node {Cubical type theory}}
      child { node {Cohesive type theory}}}
    child { node {\scshape Category theory}
      child { node {The groupoid model}}
      child { node {Categorical semantics}}
      child { node {Realizability}}
      child { node {Toposes}}
      child { node {$n$-categories}}
      child { node {Higher topos theory}}}
    child { node {\scshape Homotopy Theory}
      child { node {Quillen model categories}}
      child { node {Homotopy groups}}
      child { node {Homology and cohomology}}
      child { node {Spectral sequences}}
      child { node {Stable homotopy theory}}
      child { node {The simplicial model}}};
\end{tikzpicture}}

\end{figure}

In the present lecture we do not commit to either point of view, and simply present a deductive system of dependent type theory. It must be noted that \emph{types} and \emph{terms} are primitive notions in this system. Therefore, we will only learn what a type is by learning in what ways types may be used. The semantics of type theory is currently an active field of research (part of homotopy type theory), and could be the topic of an entire other course.

We start at the very beginning, describing the deductive system of dependent type theory is in \cref{ch:dtt} without any type forming operations. Then we gradually introduce the standard type forming operations until we give Martin-L\"of's inductive definition of identity types. 

We cover the basics --- including equivalences, the univalence axiom, and higher inductive types --- and we introduce the student to the subfield of homotopy type theory that is sometimes called \emph{synthetic homotopy theory}. Early on, students will get acquainted with the basic techniques that are used in homotopy type theory to characterize the identity types of various classes of types, and once higher inductive types are introduced students will get acquainted with the descent property that can be used to construct type families over higher inductive types in order to prove properties about them.

\section{The Curry-Howard correspondence}
%Dependent type theory is designed to reflect closely on actual mathematical practice and is compatible with classical logic. The foundational issue that isomorphic objects may have wildly different encodings in set-theoretic language, complicating the verification of mathematics, is addressed in type theory, where objects can only ever be defined up to equivalence. Despite the fact that dependent type theory is of constructive nature, it is important to note that type theory is not anti-classical: at the loss of certain properties of constructive type theory constructivists may care about, the axiom of choice may be assumed in type theory and it is in fact consistent with the univalence axiom. This may be helpful to obtain some classical results in type theory.

%One of the important properties that dependent type theory has (when the axiom of choice is not assumed) is that  

%From a logical point of view, type theory can be seen as a deductive system for constructive logic, in which types are propositions of which the constituents are precisely its proofs. In the view of Heyting, `to know the meaning of a proposition is to know which constructions can be considered as proofs of that proposition'. For instance, a proof of the proposition $A\to B$ is an algorithm that transforms proofs of $A$ into proofs of $B$.

From a syntactic point of view, type theory is a just a deductive system, or a language with enough structure to encode (most) mathematical practice. If one thinks of type theory as a deductive system, then it is natural to think of types as propositions. The terms of a type are then its proofs. However, one important difference between types and propositions is that types may have different terms, whereas propositions are completely determined by their truth value, and therefore do not have intrinsic structure beyond their provability. In other words, if there are two proofs of a given proposition $P$, then these two proofs are never regarded as distinct elements of that proposition (although they might be distinct in a syntactic sense). Nevertheless, the analogy between types and propositions holds up quite well, and is made precise in the \define{Curry-Howard correspondence}, see \cref{table:ch}.

The phenomenon that types may have distinct terms is known as \define{proof-relevance}: to construct a term of a given type with a certain property it often matters how that term is constructed. This is of course no different in mathematical practice. For example, every now and then one encounters in a mathematical exposition a proposition that of the form `structures $A$ and $B$ are isomorphic', with the isomorphism being constructed in the proof. Here it matters of course how that isomorphism is constructed, and that specific isomorphism might even be used later on. Thus, the idea of proof-relevance is nothing new.

Since types may possess many terms, one might observe that there are also formal similarities between types and sets. Indeed, a set is completely determined by how one can give an element of that set, in a similar way that a proposition is determined by how one can give a proof of that proposition. The Curry-Howard correspondence also provides a translation between types and sets.

An important difference between type theory and set theory, which makes type theory more useful as a language for formalizing mathematical constructions, is that the theory of types is itself a deductive system, whereas the theory of sets is formulated on a \emph{separate} deductive system: first order logic. Moreover, one may extract programs from proofs: a proof of the existence of an object with a certain property yields a construction of that object together with a proof that the constructed object indeed satisfies the stated property.

\begin{table}\label{table:ch}
\caption{The Curry-Howard correspondence}
\begin{center}
\begin{tabular}{lll}
\toprule
\emph{First order logic} & \emph{Set theory} & \emph{Type theory}\\
\midrule
Propositions & Sets & Types\\
Predicates & Families of sets & Dependent types\\
Proofs & Elements & Terms \\
$\top$ & $\{\emptyset\}$ & $\unit$\\
$\bot$ & $\emptyset$ & $\emptyt$ \\
$P \land Q$ & $A \times B$ & $A \times B$ \\
$P \vee Q$ & $A \sqcup B$ & $A + B$ \\
$\exists x.P(x)$ & $\coprod_{i\in I}A_i$ & $\sm{x:A}B(x)$ \\
$\forall x.P(x)$ & $\prod_{i\in I}A_i$ & $\prd{x:A}B(x)$\\
\bottomrule
\end{tabular}
\end{center}
\end{table}

\section{Types in mathematical practice}


To illustrate the concept of type dependency, let us have a closer look at the anatomy of the following purposefully simple lemma.

\begin{lem}\label{lem:unit}
Given a binary operation $\mu:A\times A\to A$ on a set $A$, any $u_l\in A$ satisfying satisfying the left unit law $\mu(u_l,x)=x$, and any $u_r\in A$ satisfying the right unit law $\mu(x,u_r)=x$, one has $u_l=u_r$. 
\end{lem}

\begin{proof}
Since $u_l$ is a left unit, we have in particular $u_l=\mu(u_l,u_r)$. Furthermore, since $u_r$ is a right unit we have in particular $\mu(u_l,u_r)=u_r$. Thus, we have $u_l=\mu(u_l,u_r)=u_r$. 
\end{proof}

\begin{samepage}
By the hypotheses of \cref{lem:unit}, we start the proof with the following set of presuppositions:
\begin{align*}
A & : \mathbf{Set} \\
\mu & : A\times A\to A \\
u_l & : A \\
p & : \forall x.\,\mu(u_l,x)=x\\
u_r & : A \\
q & : \forall x.\,\mu(x,u_r)=x,
\end{align*}
and the task is to show that $u_l=u_r$.
\end{samepage}

This list of assumptions is called the context of our proof, and the goal $u_l=u_r$ is a type in this context. 
Note that $\mathbf{Set}$ is a type in the empty context (where no assumptions are made), $A\times A\to A$ is a type in the context $A:\mathbf{Set}$, also $A$ is a type in the context $A:\mathrm{Set}$, and $\forall x.\,\mu(u_l,x)=x$ is a type in context $A:\mathbf{Set},\mu:A\times A\to A,u_l:A$, and so on.
In principle, one could give such a finite list of presumed structure for any mathematical text at any position in the text.

More generally, \define{contexts} are lists of `typed' variable declarations. By `typed' we mean that any variable is assigned a (unique) type. A context is always finite, and the variables in a context can have any type, possibly depending on variables that have been declared previously. In our example, the variable $p:\forall x.\,\mu(u_l,x)=x$ depends on $A:\mathbf{Set}$, $\mu:A\times A\to A$, and $u_l:A$. 


\chapter{Syllabus}

\section{Essential course information}
\begin{center}
\begin{tabular}{ll}
\emph{Course title} & Introduction to Homotopy Type Theory \\
\emph{Instructor} & Egbert Rijke \\
& Department of Philosophy \\
& Carnegie Mellon University \\
\emph{Course number} & 80-518, 80-818 \\
\emph{Semester} & Spring 2018 \\
\emph{Website} & \url{http://www.andrew.cmu.edu/user/erijke/hott/} \\
\emph{Lecture room} & Baker Hal 150 \\
\emph{Meeting time} & Tue/Thu 12:00 - 1:20 \\
\emph{Email} & \href{mailto:erijke@andrew.cmu.edu}{erijke@andrew.cmu.edu} \\
\emph{Instructor's office} & Baker Hall 148 \\
\emph{Office Hours} & Mon/Wed 5:00 - 6:00, or by appointment
\end{tabular}
\end{center}

\section{Course material}

We will roughly follow the book \emph{Homotopy Type Theory: Univalent foundation of mathematics} \cite{hottbook}, of which a PDF is freely available.

Some of the later results of synthetic homotopy theory can only be found in recent research papers. We will also use the PhD thesis of Guillaume Brunerie \cite{BruneriePhD} as a resource.

\section{Organization}

Each session will consist of two parts: a 50 minute lecture and 30 minutes in which students present solutions to exercises provided with the previous lecture. These presentations are intended to be short (roughly 5 minutes) and focused to the problem at hand. Problem sets will be posted below with the lecture synopses.

Students are expected to:
\begin{enumerate}
\item Present a solution when they are asked to do so (usually a week in advance). Graduate students will be asked to present more often than undergraduate students.
\item Per lecture, either correct a somewhat substantial mistake made by the instructor, or hand in a written solution for one exercise of their choice. Written solutions are to be handed in at the start of the next lecture for an A, or at the start of the next lecture after that for a B. Collaborations are encouraged, but solutions must be handed in individually. Presenting students hand in a written solution for the exercise they are asked to present.
\end{enumerate}

Hints for the exercises will be presented by the instructor during office hours, a day before they have to be handed in. 


\chapter{Introduction}

To include introduction:
\begin{enumerate}
\item What are types in mathematics. Dependent types and dependent functions are everywhere in mathematics.
\item Why univalent foundations. Why should homotopy be in the foundation of mathematics
\item Constructive nature of homotopy type theory. Discuss differences with set theory.
\item What this course is about
\item How to use this book
\item Formal type theory versus informal type theory
\item Mention the formalization
\end{enumerate}

\begin{rmk}
  One difference between set theory and type theory is that every well-formed term is specified along with its type and with its context. One way of looking at this is that there are three sorts in type theory: contexts, types, and terms. On the other hand, there is only one sort in set theory: sets. Sets are governed by the elementhood relation: the formula $x\in y$ is a well-formed formula of set theory for any two sets $x$ and $y$. In particular, for a given set $x$, the formula $x\in y$ can be true for many sets $y$, which is very different to the situation in type theory, where every term is assigned a unique type.
  
  Another important difference between set theory and type theory is that set theory is formulated in the language of first order logic, whereas type theory is its own deductive system, not making use of any ambient logic. We will see in the present chapter and in the next few chapters what this deductive system looks like.
\end{rmk}

\mainmatter

\renewcommand{\thechapter}{\Roman{chapter}}
\setsecnumdepth{subsection}

\chapter{Martin-L\"of's dependent type theory}
\section{Dependent type theory}
\label{ch:dtt}

Type theory is a system of inference rules that can be combined to make \emph{derivations}. In these derivations, the goal is often to construct a term of a certain type. Such terms can be functions if the type of the constructed term is a function type; proofs of properties if the type of the constructed term is a proposition; and identification if the type of the constructed term is an identity type, and so on. In many respects, a type is just a collection of mathematical objects and constructing terms of them is the everyday mathematical task or challenge. The system of inference rules that we call type theory offers a principled way of engaging in mathematical activity.

\subsection{Judgments and contexts in type theory}\label{sec:judgments}

An \define{inference rule} is an expression of the form
\begin{prooftree}
  \AxiomC{$\mathcal{H}_1$\quad $\mathcal{H}_2$ \quad \dots \quad $\mathcal{H}_n$}
  \UnaryInfC{$\mathcal{C}$}
\end{prooftree}
containing above the horizontal line a finite list $\mathcal{H}_1$, $\mathcal{H}_2$, \dots, $\mathcal{H}_n$ of \emph{judgments} for the hypotheses, and below the horizontal line a single judgment $\mathcal{C}$ for the conclusion. A very simple example that we will encounter in \cref{ch:pi} when we introduce function types, is the inference rule
\begin{prooftree}
  \AxiomC{$\Gamma\vdash a:A$}
  \AxiomC{$\Gamma\vdash f:A\to B$}
  \BinaryInfC{$\Gamma\vdash f(a):B$}
\end{prooftree}
This rule asserts that in any context $\Gamma$ we may use a term $a:A$ and a function $f:A\to B$ to obtain a term $f(a):B$. Each of the expressions
\begin{align*}
  \Gamma & \vdash a :A \\
  \Gamma & \vdash f : A \to B \\
  \Gamma & \vdash f(a):B
\end{align*}
are examples of judgments. There are four kinds of judgments in type theory:
\begin{enumerate}
\item \emph{$A$ is a (well-formed) \define{type}\index{well-formed type}\index{type} in context $\Gamma$.}\index{judgment!type in context} The symbolic expression for this judgment is
  \begin{equation*}
    \Gamma\vdash A~\mathrm{type}
  \end{equation*}
\item \emph{$A$ and $B$ are \define{judgmentally equal types} in context $\Gamma$.}\index{judgment!equal types in context}\index{judgmental equality!of types} The symbolic expression for this judgment is
  \begin{equation*}
    \Gamma\vdash A \jdeq B~\mathrm{type}
  \end{equation*}
\item \emph{$a$ is a (well-formed) \define{term}\index{well-formed term}\index{term} of type $A$ in context $\Gamma$.}\index{judgment!term of a type in context} The symbolic expression for this judgment is
  \begin{equation*}
    \Gamma \vdash a:A
  \end{equation*}
\item \emph{$a$ and $b$ are \define{judgmentally equal terms} of type $A$ in context $\Gamma$.}\index{judgment!equal terms of a type in context}\index{judgmental equality!of terms} The symbolic expression for this judgment is
  \begin{equation*}
    \Gamma\vdash a\jdeq b:A
  \end{equation*}
\end{enumerate}
Thus we see that any judgment is of the form $\Gamma\vdash\mathcal{J}$, consisting of a context $\Gamma$ and an expression $\mathcal{J}$ asserting that $A$ is a type, that $A$ and $B$ are equal types, that $a$ is a term of type $A$, or that $a$ and $b$ are equal terms of type $A$. A \define{context}\index{context|textbf} is an expression of the form
\begin{equation}\label{eq:context}
x_1:A_1,~x_2:A_2(x_1),~\ldots,~x_n:A_n(x_1,\ldots,x_{n-1})
\end{equation}
satisfying the condition that for each $1\leq k\leq n$ we can derive, using the inference rules of type theory, that
\begin{equation}\label{eq:context-condition}
  x_1:A_1,~x_2:A_2(x_1),~\ldots,~x_{k-1}:A_{k-1}(x_1,\ldots,x_{k-2})\vdash A_k(x_1,\ldots,x_{k-1})~\mathrm{type}.
\end{equation}
In other words, to check that an expression of the form \cref{eq:context} is a context, one starts on the left and works their way to the right verifying that each variable $x_k$ is assigned a well-formed type. We say that a context as in \cref{eq:context} \define{declares the variables}\index{variable declaration} $x_1,\ldots,x_n$. We may use variable names other than $x_1,\ldots,x_n$, as long as no variable is declared more than once.

Note that the context of length $0$ satisfies the requirement in \cref{eq:context-condition} vacuously. This context is called the \define{empty context}\index{context!empty context|textbf}\index{empty context|textbf}. An expression of the form $x_1:A_1$ is a context if and only if $A_1$ is a well-formed type in the empty context. Such types are called \define{closed types}\index{closed type|textbf}. We will soon encounter the type $\N$ of natural numbers, which is an example of a closed type. There is also the notion of \define{closed term}\index{closed term|textbf}, which is simply a term in the empty context. The next case is that an expression of the form $x_1:A_1,~x_2:A_2(x_1)$ is a context if and only if $A_1$ is a well-formed type in the empty context, and $A_2(x_1)$ is a well-formed type in context $x_1:A_1$, and so on.

It is a feature of \emph{dependent} type theory that all judgments are context-dependent, and indeed that even the types of the variables may depend on any previously declared variables. For example, when we introduce the \emph{identity type} in \cref{chap:identity}, we make full use of the machinery of dependent types, as is clear from how they are introduced:
\begin{prooftree}
  \AxiomC{$\Gamma\vdash A~\mathrm{type}$}
  \UnaryInfC{$\Gamma,x:A,y:A\vdash x=y~\mathrm{type}$}
\end{prooftree}
This rule asserts that given a type $A$ in context $\Gamma$, we may form a type $x=y$ in context $\Gamma,x:A,y:A$. Note that in order to know that the expression $\Gamma,x:A,y:A$ is indeed a well-formed context, we need to know that $A$ is a well-formed type in context $\Gamma,x:A$. This is an instance of \emph{weakening}, which we will describe shortly.

In the situation where we have
\begin{equation*}
  \Gamma,x:A\vdash B(x)~\mathrm{type},
\end{equation*}
we say that $B$ is a \define{family} of types over $A$ (or indexed by $A$) in context $\Gamma$. Similarly, in the situation where we have
\begin{equation*}
  \Gamma,x:A\vdash b(x):B(x),
\end{equation*}
we say that $b$ is a \define{section} of the family $B$ over $A$ in context $\Gamma$. Note that $A$, $B$, and $b$ also depend on the variables declared in the context $\Gamma$, even though we have not explicitly mentioned them. It is common practice to not mention every variable in the context $\Gamma$ in such situations.

\subsection{Inference rules}\label{sec:rules}

In this section we present the basic inference rules of dependent type theory. Those rules are valid to be used in any type theoretic derivation. There are only four sets of inference rules:
\begin{enumerate}
\item Rules for judgmental equality 
\item Rules for substitution
\item Rules for weakening
\item The ``variable rule''
\end{enumerate}

\subsubsection*{Judgmental equality}

In this set of inference rules we ensure that judgmental equality (both on types and on terms) are equivalence relations, and we make sure that in any context $\Gamma$, we can change the type of any variable to a judgmentally equal type.

\begin{samepage}
The rules postulating that judgmental equality on types and on terms is an equivalence relation are as follows\index{judgmental equality!equivalence relation}:
\begin{center}
%\begin{small}
\begin{minipage}{.2\textwidth}
\begin{prooftree}
\AxiomC{$\Gamma\vdash A~\textrm{type}$}
\UnaryInfC{$\Gamma\vdash A\jdeq A~\textrm{type}$}
\end{prooftree}
\end{minipage}
\begin{minipage}{.25\textwidth}
\begin{prooftree}
\AxiomC{$\Gamma\vdash A\jdeq A'~\textrm{type}$}
\UnaryInfC{$\Gamma\vdash A'\jdeq A~\textrm{type}$}
\end{prooftree}
\end{minipage}
\begin{minipage}{.5\textwidth}
\begin{prooftree}
\AxiomC{$\Gamma\vdash A\jdeq A'~\textrm{type}$}
\AxiomC{$\Gamma\vdash A'\jdeq A''~\textrm{type}$}
\BinaryInfC{$\Gamma\vdash A\jdeq A''~\textrm{type}$}
\end{prooftree}
\end{minipage}
\\*
\bigskip
\begin{minipage}{.2\textwidth}
\begin{prooftree}
\AxiomC{$\Gamma\vdash a:A$}
\UnaryInfC{$\Gamma\vdash a\jdeq a : A$}
\end{prooftree}
\end{minipage}
\begin{minipage}{.25\textwidth}
\begin{prooftree}
\AxiomC{$\Gamma\vdash a\jdeq a':A$}
\UnaryInfC{$\Gamma\vdash a'\jdeq a: A$}
\end{prooftree}
\end{minipage}
\begin{minipage}{.5\textwidth}
\begin{prooftree}
\AxiomC{$\Gamma\vdash a\jdeq a' : A$}
\AxiomC{$\Gamma\vdash a'\jdeq a'': A$}
\BinaryInfC{$\Gamma\vdash a\jdeq a'': A$}
\end{prooftree}
\end{minipage}
%\end{small}
\end{center}
\end{samepage}

\bigskip
Apart from the rules postulating that judgmental equality is an equivalence relation, there are also \define{variable conversion rules}\index{judgmental equality!conversion rules}\index{variable conversion rules}\index{conversion rule!variable}\index{rule!variable conversion}.
Informally, these are rules stating that if $A$ and $A'$ are judgmentally equal types in context $\Gamma$, then any valid judgment in context $\Gamma,x:A$ is also a valid judgment in context $\Gamma,x:A'$. In other words: we can convert the type of a variable to a judgmentally equal type.

The first variable conversion rule states that
\begin{prooftree}
\AxiomC{$\Gamma\vdash A\jdeq A'~\textrm{type}$}
\AxiomC{$\Gamma,x:A,\Delta\vdash B(x)~\mathrm{type}$}
\BinaryInfC{$\Gamma,x:A',\Delta\vdash B(x)~\mathrm{type}$}
\end{prooftree}
In this conversion rule, the context of the form $\Gamma,x:A,\Delta$ is just any extension of the context $\Gamma,x:A$.

Similarly, there are variable conversion rules for judgmental equality of types, for terms, and for judgmental equality of terms. To avoid having to state essentially the same rule four times, we state all four variable conversion rules at once using a \emph{generic judgment} $\mathcal{J}$, which can be any of the four kinds of judgments.
\begin{prooftree}
\AxiomC{$\Gamma\vdash A\jdeq A'~\textrm{type}$}
\AxiomC{$\Gamma,x:A,\Delta\vdash \mathcal{J}$}
\BinaryInfC{$\Gamma,x:A',\Delta\vdash \mathcal{J}$}
\end{prooftree}
An analogous \emph{term conversion rule}\index{term conversion rule}\index{conversion rule!term}\index{rule!term conversion}, stated in \cref{ex:term_conversion}, converting the type of a term to a judgmentally equal type, is derivable using the rules for substitution and weakening, and the variable rule.

\subsubsection*{Substitution}
If we are given a term $a:A$ in context $\Gamma$, then for any type $B$ in context $\Gamma,x:A,\Delta$ we can form the type $B[a/x]$ in context $\Gamma,\Delta[a/x]$, where $B[a/x]$ is an abbreviation for
\begin{equation*}
B(x_1,\ldots,x_{n-1},a(x_1,\ldots,x_{n-1}),x_{n+1},\ldots,x_{n+m-1})
\end{equation*}
This definition of substituting $a$ for $x$ is understood to be recursive over the length of $\Delta$. Similarly we obtain for any term $b:B$ in context $\Gamma,x:A,\Delta$ a term $b[a/x]:B[a/x]$. The \define{substitution rule}\index{substitution}\index{rule!substitution} asserts that substitution preserves well-formedness and judgmental equality of types and terms:
\begin{prooftree}
\AxiomC{$\Gamma\vdash a:A$}
\AxiomC{$\Gamma,x:A,\Delta\vdash \mathcal{J}$}
\RightLabel{$S_a$}
\BinaryInfC{$\Gamma,\Delta[a/x]\vdash \mathcal{J}[a/x]$}
\end{prooftree}
Furthermore, we postulate that substitution by judgmentally equal terms results in judgmentally equal types
\begin{prooftree}
\AxiomC{$\Gamma\vdash a\jdeq a':A$}
\AxiomC{$\Gamma,x:A,\Delta\vdash B~\mathrm{type}$}
\BinaryInfC{$\Gamma,\Delta[a/x]\vdash B[a/x]\jdeq B[a'/x]~\mathrm{type}$}
\end{prooftree}
and it also results in judgmentally equal terms
\begin{prooftree}
\AxiomC{$\Gamma\vdash a\jdeq a':A$}
\AxiomC{$\Gamma,x:A,\Delta\vdash b:B$}
\BinaryInfC{$\Gamma,\Delta[a/x]\vdash b[a/x]\jdeq b[a'/x]:B[a/x]$}
\end{prooftree}
When $B$ is a family of types over $A$ and $a:A$, we also say that $B[a/x]$ is the \define{fiber}\index{family!fiber of}\index{fiber!of a family} of $B$ at $a$. Often we write $B(a)$ for $B[a/x]$.

\subsubsection*{Weakening}
If we are given a type $A$ in context $\Gamma$, then any judgment made in a longer context $\Gamma,\Delta$ can also be made in the context $\Gamma,x:A,\Delta$, for a fresh variable $x$. The \define{weakening rule}\index{weakening}\index{rule!weakening} asserts that weakening by a type $A$ in context preserves well-formedness and judgmental equality of types and terms.
\begin{prooftree}
\AxiomC{$\Gamma\vdash A~\textrm{type}$}
\AxiomC{$\Gamma,\Delta\vdash \mathcal{J}$}
\RightLabel{$W_A$}
\BinaryInfC{$\Gamma,x:A,\Delta \vdash \mathcal{J}$}
\end{prooftree}
This process of expanding the context by a fresh variable of type $A$ is called \define{weakening (by $A$)}.

In the simplest situation where weakening applies, we have two types $A$ and $B$ in context $\Gamma$. Then we can weaken $B$ by $A$ as follows
\begin{prooftree}
  \AxiomC{$\Gamma\vdash A~\textrm{type}$}
  \AxiomC{$\Gamma\vdash B~\textrm{type}$}
  \RightLabel{$W_A$}
  \BinaryInfC{$\Gamma,x:A\vdash B~\mathrm{type}$}
\end{prooftree}
in order to form the type $B$ in context $\Gamma,x:A$. The type $B$ in context $\Gamma,x:A$ is called the \define{constant family}\index{family!constant family}\index{constant family} $B$, or the \define{trivial family}\index{family!trivial family}\index{trivial family} $B$.

\subsubsection*{The variable rule}
If we are given a type $A$ in context $\Gamma$, then we can weaken $A$ by itself to obtain that $A$ is a type in context $\Gamma,x:A$. The \define{variable rule}\index{variable rule}\index{rule!variable rule|textbf} now asserts that $x$ is a well-formed term of type $A$ in context $\Gamma,x:A$.
\begin{prooftree}
\AxiomC{$\Gamma\vdash A~\textrm{type}$}
\RightLabel{$\delta_A$}
\UnaryInfC{$\Gamma,x:A\vdash x:A$}
\end{prooftree}
One of the reasons for including the variable rule is that it provides an \emph{identity function}\index{identity function} on the type $A$ in context $\Gamma$.

\begin{comment}
\bigskip
\begin{minipage}{.45\textwidth}
\begin{prooftree}
\AxiomC{$\Gamma\vdash A~\textrm{type}$}
\AxiomC{$\Gamma,\Delta\vdash B~\textrm{type}$}
\RightLabel{$W_A$}
\BinaryInfC{$\Gamma,x:A,\Delta \vdash B~\textrm{type}$}
\end{prooftree}
\end{minipage}\hfill
\begin{minipage}{.45\textwidth}
\begin{prooftree}
\AxiomC{$\Gamma\vdash A~\textrm{type}$}
\AxiomC{$\Gamma,\Delta\vdash b:B$}
\RightLabel{$W_A$}
\BinaryInfC{$\Gamma,x:A,\Delta \vdash b:B$}
\end{prooftree}
\end{minipage}

\noindent
\begin{prooftree}
\AxiomC{$\Gamma\vdash A~\textrm{type}$}
\RightLabel{$\delta_A$}
\UnaryInfC{$\Gamma,x:A\vdash x:A$}
\end{prooftree}

\noindent
\begin{minipage}{.5\textwidth}
\begin{prooftree}
\AxiomC{$\Gamma\vdash a:A$}
\AxiomC{$\Gamma,x:A,\Delta\vdash B~\textrm{type}$}
\RightLabel{$S_a$}
\BinaryInfC{$\Gamma,\Delta[a/x]\vdash B[a/x]~\textrm{type}$}
\end{prooftree}
\end{minipage}\hfill
\begin{minipage}{.5\textwidth}
\begin{prooftree}
\AxiomC{$\Gamma\vdash a:A$}
\AxiomC{$\Gamma,x:A,\Delta\vdash b:B$}
\RightLabel{$S_a$}
\BinaryInfC{$\Gamma,\Delta[a/x]\vdash b[a/x] : B[a/x]$}
\end{prooftree}
\end{minipage}

\bigskip
\end{comment}

\subsection{Derivations}\label{sec:derivations}

A derivation in type theory is a tree in which each node is a valid rule of inference. We give two examples of derivations: a derivation showing that any variable can be changed to a fresh one, and a derivation showing that any two variables that do not depend on one another can be swapped in order.

Thus, we will see some examples of new inference rules that can be derived using the rules of type theory. Such inference rules are called \define{admissible}. Since derivations tend to get long and unwieldy, we declare that admissible inference rules are also valid to be used in derivations.

\subsubsection*{Changing variables}

Variables can always be changed to fresh variables. We show that this is the case by showing that the inference rule
\begin{prooftree}
  \AxiomC{$\Gamma,x:A,\Delta\vdash \mathcal{J}$}
  \RightLabel{$x'/x$}
  \UnaryInfC{$\Gamma,x':A,\Delta[x'/x]\vdash \mathcal{J}[x'/x]$}
\end{prooftree}
is admissible, where $x'$ is a variable that does not occur in the context $\Gamma,x:A,\Delta$. 

Indeed, we have the following derivation using substitution, weakening, and the variable rule:
\begin{prooftree}
  \AxiomC{$\Gamma\vdash A~\mathrm{type}$}
  \RightLabel{$\delta_A$}
  \UnaryInfC{$\Gamma,x':A\vdash x':A$}
  \AxiomC{$\Gamma\vdash A~\mathrm{type}$}
  \AxiomC{$\Gamma,x:A,\Delta\vdash \mathcal{J}$}
  \RightLabel{$W_A$}
  \BinaryInfC{$\Gamma,x':A,x:A,\Delta\vdash \mathcal{J}$}
  \RightLabel{$S_{x'}$}
  \BinaryInfC{$\Gamma,x':A,\Delta[x'/x]\vdash \mathcal{J}[x'/x]$}
\end{prooftree}
In this derivation it is the application of the weakening rule where we have to check that $x'$ does not occur in the context $\Gamma,x:A,\Delta$.

\subsubsection*{Interchanging variables}

The \define{interchange rule}\index{rule!interchange}\index{interchange rule} states that if we have two types $A$ and $B$ in context $\Gamma$, and we make a judgment in context $\Gamma,x:A,y:B,\Delta$, then we can make that same judgment in context $\Gamma,y:B,x:A,\Delta$ where the order of $x:A$ and $y:B$ is swapped. More formally, the interchange rule is the following inference rule
\begin{prooftree}
\AxiomC{$\Gamma\vdash B~\textrm{type}$}
\AxiomC{$\Gamma,x:A,y:B,\Delta\vdash \mathcal{J}$}
\BinaryInfC{$\Gamma,y:B,x:A,\Delta\vdash \mathcal{J}$}
\end{prooftree}
Just as the rule for changing variables, we claim that the interchange rule is an admissible rule.

The idea of the derivation for the interchange rule is as follows: If we have a judgment
\begin{equation*}
  \Gamma,x:A,y:B,\Delta\vdash\mathcal{J},
\end{equation*}
then we can change the variable $y$ to a fresh variable $y'$ and weaken the judgment to obtain the judgment
\begin{equation*}
  \Gamma,y:B,x:A,y':B,\Delta[y'/y]\vdash\mathcal{J}[y'/y].
\end{equation*}
Now we can substitute $y$ for $y'$ to obtain the desired judgment $\Gamma,y:B,x:A,\Delta\vdash\mathcal{J}$. The formal derivation is as follows:
%\begin{small}
\begin{prooftree}
\AxiomC{$\Gamma\vdash B~\textrm{type}$}
\RightLabel{$\delta_B$}
\UnaryInfC{$\Gamma,y:B\vdash y:B$}
\RightLabel{$W_{W_B(A)}$}
\UnaryInfC{$\Gamma,y:B,x:A\vdash y:B$}
\AxiomC{$\Gamma,x:A,y:B,\Delta\vdash \mathcal{J}$}
\RightLabel{$y'/y$}
\UnaryInfC{$\Gamma,x:A,y':B,\Delta[y'/y]\vdash \mathcal{J}[y'/y]$}
\RightLabel{$W_B$}
\UnaryInfC{$\Gamma,y:B,x:A,y':B,\Delta[y'/y]\vdash \mathcal{J}[y'/y]$}
\RightLabel{$S_{W_A(y)}$}
\BinaryInfC{$\Gamma,y:B,x:A,\Delta\vdash \mathcal{J}$}
\end{prooftree}
%\end{small}


\begin{comment}
For $A\in T_n$ we define $T_{n+k+1}(A):= \{B\in T_{n+k}\mid \mathrm{ft}^{k+1}(B)=A\}$. Similarly we define $\tilde{T}_{n+k+1}(A):=\{b\in\tilde{T}_{n+k+1}\mid\mathrm{ft}^{k+1}(\partial(b))=A\}$. For any closed type $A$ we maintain the convention that $T_{k}(\mathrm{ft}(A)):= T_k$.
\begin{enumerate}
\item For all $A\in T_n$, and all $k\in\N$, \define{weakening} operations
\begin{align*}
W_A & : T_{n+k}(\mathrm{ft}(A)) \to T_{n+k+1}(A) \\
\tilde{W}_A & : \tilde{T}_{n+k}(\mathrm{ft}(A))\to \tilde{T}_{n+k+1}(A)
\end{align*}
subject to the conditions $\mathrm{ft}(W_A(B))=W_A(\mathrm{ft}(B))$ if $B\in T_{n+k}$ with $k\geq 1$, and $\partial(\tilde{W}_A(b))=W_A(\partial(b))$.
\item For all $A\in T_n$ a term $\delta_A\in \tilde{T}_{n+1}$ satisfying $\partial(\delta_A)=W_A(A)$. 
\item For all $a\in \tilde{T}_n$ satisfying $\partial(a)=A$, and all $k\in\N$, \define{substitution} operations
\begin{align*}
S_a & : \{B\in T_{n+k+1}\mid \mathrm{ft}^{k+1}(B)= A\}\to T_k \\
\tilde{S}_a & : \{b\in \tilde{T}_{n+k+1}\mid \mathrm{ft}^{k+1}(\partial(b))=A\}\to \tilde{T}_{n+k}
\end{align*}
subject to the conditions $\mathrm{ft}(S_a(B))=\mathrm{ft}(A)$ if $B\in T_{n+1}$, $\mathrm{ft}(S_a(B))=S_a(\mathrm{ft}(B))$ if $B\in T_{n+k+1}$ with $k\geq 1$, and $\partial(\tilde{S}_a(b))=S_a(\partial(b))$.
\end{enumerate}
\end{comment}

%\subsection{Axioms for weakening, substitution, and the diagonal}
\begin{comment}
\begin{prooftree}
\AxiomC{$\Gamma\vdash A~\textrm{type}$}
  \AxiomC{$\Gamma,x:A,\Delta\vdash B~\textrm{type}$}
    \AxiomC{$\Gamma,x:A,\Delta,y:B,E\vdash C~\textrm{type}$}
\TrinaryInfC{$\Gamma,\Delta[a/x],E[b/y][a/x]\vdash C[b/y][a/x]\jdeq C[a/x][b[a/x]/y']~\textrm{type}$}
\end{prooftree}
\begin{prooftree}
\AxiomC{$\Gamma\vdash A~\textrm{type}$}
  \AxiomC{$\Gamma,x:A,\Delta\vdash B~\textrm{type}$}
    \AxiomC{$\Gamma,x:A,\Delta,y:B,E\vdash c:C$}
\TrinaryInfC{$\Gamma,\Delta[a/x],E[b/y][a/x]\vdash c[b/y][a/x]\jdeq c[a/x][b[a/x]/y']:C[b/y][a/x]$}
\end{prooftree}
\begin{prooftree}
\AxiomC{$\Gamma\vdash a:A$}
  \AxiomC{$\Gamma,\Delta\vdash B~\textrm{type}$}
\RightLabel{($x$ not free in $B$)}
\BinaryInfC{$\Gamma,\Delta\vdash B[a/x]\jdeq B~\textrm{type}$}
\end{prooftree}
\end{comment}


\begin{comment}
\subsection{An algebraic presentation of dependent type theory}

%Let us write $T_n$ for the set of well-formed contexts of length $n$, for $n>1$. Then any well-formed context of length $n+1$ is of the form $\Gamma,x:A$, where $\Gamma$ is a well-formed context of length $n$. Thus we see that there are maps $\eft:T_{n+1}\to T_n$ for $n>1$. Similarly, if we write $\tilde{T}_n$ for the set of all terms of a type in a context of length $n-1$, we see that there is a map $\tilde{T}_n\to T_n$. The following picture emerges:
%\begin{equation*}
%\begin{tikzcd}
%\tilde{T}_3 \arrow[dr,"\partial"] & \vdots \arrow[d,"\mathrm{ft}"] \\
%\tilde{T}_2 \arrow[dr,"\partial"] & T_3 \arrow[d,"\mathrm{ft}"] \\
%\tilde{T}_1 \arrow[dr,"\partial"] & T_2 \arrow[d,"\mathrm{ft}"] \\
%& T_1
%\end{tikzcd}
%\end{equation*}

Observe that given a type $A$ in context $\Gamma$ and a type $B$ in context $\Gamma,\Delta$ we can weaken twice by first weakening by $B$ and then by $A$, as indicated in the following derivation:
\begin{prooftree}
\AxiomC{$\Gamma\vdash A~\textrm{type}$}
\AxiomC{$\Gamma,\Delta\vdash B~\textrm{type}$}
  \AxiomC{$\Gamma,\Delta,\mathrm{E}\vdash \mathcal{J}$}
\BinaryInfC{$\Gamma,\Delta,y:B,\mathrm{E}\vdash \mathcal{J}$}
\BinaryInfC{$\Gamma,x:A,\Delta,y:B,\mathrm{E}\vdash \mathcal{J}$}
\end{prooftree}
However, we can also first weaken by $A$, and then by `$B$ weakened by $A$', as indicated in the following derivation:
\begin{prooftree}
\AxiomC{$\Gamma\vdash A~\textrm{type}$}
  \AxiomC{$\Gamma,\Delta\vdash B~\textrm{type}$}
\BinaryInfC{$\Gamma,x:A,\Delta\vdash B~\textrm{type}$}
  \AxiomC{$\Gamma\vdash A~\textrm{type}$}
    \AxiomC{$\Gamma,\Delta,\mathrm{E}\vdash \mathcal{J}$}
  \BinaryInfC{$\Gamma,x:A,\Delta,\mathrm{E}\vdash \mathcal{J}$}
\BinaryInfC{$\Gamma,x:A,\Delta,y:B,\mathrm{E}\vdash \mathcal{J}$}
\end{prooftree}
For the end result it does not matter in what order the two weakenings are performed. We can express this conveniently as an equation:
\begin{equation*}
W_A(W_B(\mathcal{J}))\jdeq W_{W_A(B)}(W_A(\mathcal{J})).
\end{equation*}
The complete list of rules (in alphabetic order) is
\begin{align*}
S_a(\delta_B) & \jdeq \delta_{S_a(B)} \\
S_a(\delta_A) & \jdeq a \\
S_a(S_b(\mathcal{J})) & \jdeq S_{S_a(b)}(S_a(\mathcal{J})) \\
S_a(W_A(\mathcal{J})) & \jdeq \mathcal{J} \\
S_a(W_B(\mathcal{J})) & \jdeq W_{S_a(B)}(S_a(\mathcal{J})) \\
S_{\delta_A}(W_{W_A}(\mathcal{J})) & \jdeq \mathcal{J} \\
W_A(\delta_B) & \jdeq \delta_{W_A(B)} \\
W_A(S_b(\mathcal{J})) & \jdeq S_{W_A(b)}(W_A(\mathcal{J})) \\
W_A(W_B(\mathcal{J})) & \jdeq W_{W_A(B)}(W_A(\mathcal{J}))
\end{align*}
Here $A$ is a type in context $\Gamma$ and $a$ is a term of type $A$, $B$ is a type in context $\Gamma,x:A,\Delta$ and $b$ is a term of type $B$.
\end{comment}

%\begin{rmk}
%To avoid overly long proof trees maintain as a convention that every derivation with hypotheses $\mathcal{H}_1,\ldots,\mathcal{H}_n$ and conclusion $\mathcal{C}$ can be used as a rule
%\begin{prooftree}
%\AxiomC{$\mathcal{H}_1$}
%\AxiomC{$\cdots$}
%\AxiomC{$\mathcal{H}_n$}
%\TrinaryInfC{$\mathcal{C}$}
%\end{prooftree}
%in later derivations.
%\end{rmk}

\begin{exercises}
\item \label{ex:term_conversion}Give a derivation for the following conversion rule\index{term conversion rule}\index{term conversion rule}\index{rule!term conversion}\index{term conversion rule}\index{conversion rule!term}:
\begin{prooftree}
\AxiomC{$\Gamma\vdash A\jdeq A'~\textrm{type}$}
\AxiomC{$\Gamma\vdash a:A$}
\BinaryInfC{$\Gamma\vdash a:A'$}
\end{prooftree}
\item Consider a type $A$ in context $\Gamma$. In this exercise we establish a connection between types in context $\Gamma,x:A$, and uniform choices of types $B_a$, where $a$ ranges over terms of $A$ in a uniform way. A similar connection is made for terms.
  \begin{subexenum}
  \item We define a \define{uniform family} over $A$ to consist of a type
    \begin{equation*}
      \Delta,\Gamma\vdash B_a~\mathrm{type}
    \end{equation*}
    for every context $\Delta$, and every term $\Delta,\Gamma\vdash a:A$, subject to the condition that one can derive
    \begin{prooftree}
      \AxiomC{$\Delta\vdash d:D$}
      \AxiomC{$\Delta,y:D,\Gamma\vdash a:A$}
      \BinaryInfC{$\Delta,\Gamma\vdash B_a[d/y]\jdeq B_{a[d/y]}~\mathrm{type}$}
    \end{prooftree}
    Define a bijection between types in context $\Gamma,x:A$ and uniform families over $A$. 
  \item Consider a type $\Gamma,x:A\vdash B$. We define a \define{uniform term} of $B$ over $A$ to consist of a type
    \begin{equation*}
      \Delta,\Gamma\vdash b_a:B[a/x]~\mathrm{type}
    \end{equation*}
    for every context $\Delta$, and every term $\Delta,\Gamma\vdash a:A$, subject to the condition that one can derive
    \begin{prooftree}
      \AxiomC{$\Delta\vdash d:D$}
      \AxiomC{$\Delta,y:D,\Gamma\vdash a:A$}
      \BinaryInfC{$\Delta,\Gamma\vdash b_a[d/y]\jdeq b_{a[d/y]}:B[a/x][d/y]$}
    \end{prooftree}
    Define a bijection between terms of $B$ in context $\Gamma,x:A$ and uniform terms of $B$ over $A$. 
  \end{subexenum}
\end{exercises}

\section{Dependent function types}
\label{ch:pi}

\index{Pi-type@{$\Pi$-type}|see {dependent function type}}
\index{dependent function type|(}
A fundamental concept in dependent type theory is that of a dependent function. A dependent function is a function of which the type of the output may depend on the input. They are a generalization of ordinary functions, because an ordinary function $f:A\to B$ is a function of which the output $f(x)$ has type $B$ regardless of the value of $x$.

\subsection{Dependent function types}
Consider a section $b$ of a family $B$ over $A$ in context $\Gamma$, i.e.,
\begin{equation*}
  \Gamma,x:A\vdash b(x):B(x).
\end{equation*}
From one point of view, such a section $b$ is an operation, or a program\index{program}, that takes as input $x:A$ and produces a term $b(x):B(x)$. From a more mathematical point of view we see $b$ as a choice of an element of each $B(x)$. In other words, we may see $b$ as a function that takes $x:A$ to $b(x):B(x)$. Note that the type $B(x)$ of the output is dependent on $x:A$. In this section we postulate rules for the \emph{type} of all such dependent functions: whenever $B$ is a family over $A$ in context $\Gamma$, there is a type
\begin{equation*}
  \prd{x:A}B(x)
\end{equation*}
in context $\Gamma$, consisting of all the dependent functions of which the output at $x:A$ has type $B(x)$. There are four principal rules for $\Pi$-types:
\begin{enumerate}
\item The formation rule, which tells us how we may form dependent function types.
\item The introduction rule, which tells us how to introduce new terms of dependent function types.
\item The elimination rule, which tells us how to use arbitrary terms of dependent function types.
\item The computation rules, which tell us how the introduction and elimination rules interact. These computation rules guarantee that every term of a dependent function type behaves as expected: as a dependent function.
\end{enumerate}
In the cases of the formation rule, the introduction rule, and the elimination rule, we will also provide conversion rules that ensure that all the constructions respect judgmental equality.

\subsubsection{The $\Pi$-formation rule}
\index{dependent function type!formation rule}
\define{Dependent function types} are formed by the following \define{$\Pi$-formation rule}\index{rules!for dependent function types!formation}:
\begin{prooftree}
\AxiomC{$\Gamma,x:A\vdash B(x)~\textrm{type}$}
\RightLabel{$\Pi$.}
\UnaryInfC{$\Gamma\vdash \prd{x:A}B(x)~\type$}
\end{prooftree}
With the following conversion rule we postulate that formation of dependent function types respects judgmental equality of types:
\index{rules!for dependent function types!conversion}
\index{dependent function type!conversion rule}
\begin{prooftree}
\AxiomC{$\Gamma\vdash A\jdeq A'~\type$}
\AxiomC{$\Gamma,x:A\vdash B(x)\jdeq B'(x)~\textrm{type}$}
\RightLabel{$\Pi$-eq.}
\BinaryInfC{$\Gamma\vdash \prd{x:A}B(x)\jdeq\prd{x:A'}B'(x)~\type$}
\end{prooftree}
Furthermore, when $x'$ is a fresh variable, i.e., which does not occur in the context $\Gamma,x:A$, we also postulate that
\index{rules!for dependent function types!change of bound variable}
\index{dependent function type!change of bound variable}
\begin{prooftree}
\AxiomC{$\Gamma,x:A\vdash B(x)~\textrm{type}$}
\RightLabel{$\Pi$-$x'/x$.}
\UnaryInfC{$\Gamma\vdash \prd{x:A}B(x)\jdeq \prd{x':A}B(x')~\type$}
\end{prooftree}

\subsubsection{The $\Pi$-introduction rule}
The introduction rule%
\index{dependent function type!introduction rule|see {$\lambda$-abstraction}}
for dependent function types is also called the $\lambda$-abstraction rule. Recall that dependent functions are formed from terms $b(x)$ of type $B(x)$ in context $\Gamma,x:A$.
Therefore the \define{$\lambda$-abstraction rule}
\index{lambda-abstraction@{$\lambda$-abstraction}}
\index{rules!for dependent function types!lambda-abstraction@{$\lambda$-abstraction}}
\index{dependent function type!lambda-abstraction@{$\lambda$-abstraction}}
is as follows:
\begin{prooftree}
  \AxiomC{$\Gamma,x:A \vdash b(x) : B(x)$}
  \RightLabel{$\lambda$}
  \UnaryInfC{$\Gamma\vdash \lam{x}b(x) : \prd{x:A}B(x)$}
\end{prooftree}

Just like ordinary mathematicians, we will sometimes write $x\mapsto f(x)$ for a function $f$. The map $n\mapsto n^2$ is an example. The $\lambda$-abstraction is also required to respect judgmental equality. Therefore we postulate the \define{$\lambda$-conversion rule},
\index{rules!for dependent function types!lambda-conversion@{$\lambda$-conversion}}
\index{lambda-conversion@{$\lambda$-conversion}}
\index{dependent function type!lambda-conversion@{$\lambda$-conversion}}
which asserts that
\begin{prooftree}
  \AxiomC{$\Gamma,x:A \vdash b(x)\jdeq b'(x) : B(x)$}
  \RightLabel{$\lambda$-eq.}
  \UnaryInfC{$\Gamma\vdash \lam{x}b(x)\jdeq \lam{x}b'(x) : \prd{x:A}B(x)$}
\end{prooftree}

\subsubsection{The $\Pi$-elimination rule}

\index{dependent function type!elimination rule|see {evaluation}}
The elimination rule for dependent function types provides us with a way to \emph{use} dependent functions. The way to use a dependent function is to apply it to an argument of the domain type. The $\Pi$-elimination rule is therefore also called the \define{evaluation rule}\index{evaluation}\index{rules!for dependent function types!evaluation}\index{dependent function type!evaluation}. It asserts that given a dependent function $f:\prd{x:A}B(x)$ in context $\Gamma$ we obtain a term $f(x)$ of type $B(x)$ in context $\Gamma,x:A$. More formally:
\begin{prooftree}
\AxiomC{$\Gamma\vdash f:\prd{x:A}B(x)$}
\RightLabel{$ev$}
\UnaryInfC{$\Gamma,x:A\vdash f(x) : B(x)$}
\end{prooftree}
Again we require that evaluation respects judgmental equality:
\begin{prooftree}
  \AxiomC{$\Gamma\vdash f\jdeq f':\prd{x:A}B(x)$}
  \UnaryInfC{$\Gamma,x:A\vdash f(x)\jdeq f'(x):B(x)$}
\end{prooftree}

\subsubsection{The $\Pi$-computation rules}

\index{dependent function type!computation rules|see {$\beta$- and $\eta$-rules}}
The computation rules for dependent function types postulate that $\lambda$-abstraction rule and the evaluation rule are mutual inverses. Thus we have two computation rules.

First we postulate the \define{$\beta$-rule}\index{beta-rule@{$\beta$-rule}!for Pi-types@{for $\Pi$-types}}\index{rules!for dependent function types!beta-rule@{$\beta$-rule}}\index{dependent function type!beta-rule@{$\beta$-rule}}
\begin{prooftree}
\AxiomC{$\Gamma,x:A \vdash b(x) : B(x)$}
\RightLabel{$\beta$.}
\UnaryInfC{$\Gamma,x:A \vdash (\lambda y.b(y))(x)\jdeq b(x) : B(x)$}
\end{prooftree}
Second, we postulate the \define{$\eta$-rule}\index{eta-rule@{$\eta$-rule}!for Pi-types@{for $\Pi$-types}}\index{rules!for dependent function types!eta-rule@{$\eta$-rule}}\index{dependent function type!eta-rule@{$\eta$-rule}}
\begin{prooftree}
\AxiomC{$\Gamma\vdash f:\prd{x:A}B(x)$}
\RightLabel{$\eta$.}
\UnaryInfC{$\Gamma \vdash \lam{x}f(x) \jdeq f : \prd{x:A}B(x)$}
\end{prooftree}
This completes the specification of dependent function types.

\subsection{Ordinary function types}
In the case where both $A$ and $B$ are types in context $\Gamma$, we may first weaken $B$ by $A$, and then apply the formation rule for the dependent function type:
\begin{prooftree}
\AxiomC{$\Gamma\vdash A~\textrm{type}$}
\AxiomC{$\Gamma\vdash B~\textrm{type}$}
\BinaryInfC{$\Gamma,x:A\vdash B~\textrm{type}$}
\UnaryInfC{$\Gamma\vdash \prd{x:A}B~\textrm{type}$}
\end{prooftree}
The result is the type of functions that take an argument of type $A$, and return a term of type $B$. In other words, terms of the type $\prd{x:A}B$ are \emph{ordinary} functions from $A$ to $B$. We write $A\to B$\index{A arrow B@{$A\to B$}|see {function type}} for the \define{type of functions}\index{function type} from $A$ to $B$. Sometimes we will also write $B^A$\index{B^A@{$B^A$}|see {function type}} for the type $A\to B$.

We give a brief summary of the rules specifying ordinary function types, omitting the conversion rules. All of these rules can be derived easily from the corresponding rules for $\Pi$-types.\index{rules!for function types}
\begin{prooftree}
\AxiomC{$\Gamma\vdash A~\textrm{type}$}
\AxiomC{$\Gamma\vdash B~\textrm{type}$}
\RightLabel{$\to$}
\BinaryInfC{$\Gamma\vdash A\to B~\textrm{type}$}
\end{prooftree}%
\begin{center}
\begin{minipage}{.45\textwidth}
\begin{prooftree}
\AxiomC{$\Gamma\vdash B~\textrm{type}$}
\AxiomC{$\Gamma,x:A\vdash b(x):B$}
\RightLabel{$\lambda$}
\BinaryInfC{$\Gamma\vdash \lam{x}b(x):A\to B$}
\end{prooftree}%
\end{minipage}
\begin{minipage}{.45\textwidth}
\begin{prooftree}
\AxiomC{$\Gamma\vdash f:A\to B$}
\RightLabel{$ev$}
\UnaryInfC{$\Gamma,x:A\vdash f(x):B$}
\end{prooftree}%
\end{minipage}
\end{center}
\begin{center}
\begin{minipage}{.45\textwidth}
\begin{prooftree}
\AxiomC{$\Gamma\vdash B~\textrm{type}$}
\AxiomC{$\Gamma,x:A\vdash b(x):B$}
\RightLabel{$\beta$}
\BinaryInfC{$\Gamma,x:A\vdash(\lam{y}b(y))(x)\jdeq b(x):B$}
\end{prooftree}%
\end{minipage}
\begin{minipage}{.45\textwidth}
\begin{prooftree}
\AxiomC{$\Gamma\vdash f:A\to B$}
\RightLabel{$\eta$}
\UnaryInfC{$\Gamma\vdash\lam{x} f(x)\jdeq f:A\to B$}
\end{prooftree}
\end{minipage}
\end{center}

\subsection{The identity function, composition, and their laws}
\begin{defn}
For any type $A$ in context $\Gamma$, we define the \define{identity function}\index{identity function}\index{function type!identity function} $\idfunc[A]:A\to A$\index{id A@{$\idfunc[A]$}} using the variable rule:
\begin{prooftree}
\AxiomC{$\Gamma\vdash A~\textrm{type}$}
\UnaryInfC{$\Gamma,x:A\vdash x:A$}
\UnaryInfC{$\Gamma\vdash \idfunc[A]\defeq \lam{x}x:A\to A$}
\end{prooftree}
\end{defn}

Note that we have used the symbol $\jdeq$ in the conclusion to define the identity function. A judgment of the form $\Gamma\vdash a\defeq b:A$ should be read as "$b$ is a well-defined term of type $A$ in context $\Gamma$, and we will refer to it as $a$".

\begin{defn}
For any three types $A$, $B$, and $C$ in context $\Gamma$, there is a \define{composition}\index{function type!composition}\index{composition!of functions} operation
\begin{equation*}
\comp:(B\to C)\to ((A\to B)\to (A\to C)),
\end{equation*}
i.e., we can derive
\begin{prooftree}
\AxiomC{$\Gamma\vdash A~\textrm{type}$}
\AxiomC{$\Gamma\vdash B~\textrm{type}$}
\AxiomC{$\Gamma\vdash C~\textrm{type}$}
\TrinaryInfC{$\Gamma\vdash\comp:(B\to C)\to ((A\to B)\to (A\to C))$}
\end{prooftree}
We will write $g\circ f$\index{g composed with f@{$g\circ f$}} for $\comp(g,f)$\index{comp(g,f)@{$\comp(g,f)$}}.
\end{defn}

\begin{constr}
  The idea of the definition is to define $\comp(g,f)$ to be the function $\lam{x}g(f(x))$. The derivation we use to construct $\comp$ is as follows:
  \begin{prooftree}
    \AxiomC{$\Gamma\vdash A~\type$}
    \AxiomC{$\Gamma\vdash B~\type$}
    \BinaryInfC{$\Gamma,f:B^A,x:A\vdash f(x):B$}
    \UnaryInfC{$\Gamma,g:C^B,f:B^A,x:A\vdash f(x):B$}
    \AxiomC{$\Gamma\vdash B~\type$}
    \AxiomC{$\Gamma\vdash C~\type$}
    \BinaryInfC{$\Gamma,g:C^B,y:B\vdash g(y):C$}
    \UnaryInfC{$\Gamma,g:C^B,f:B^A,y:B\vdash g(y):C$}
    \UnaryInfC{$\Gamma,g:C^B,f:B^A,x:A,y:B\vdash g(y):C$}
    \BinaryInfC{$\Gamma,g:C^B,f:B^A,x:A\vdash g(f(x)) : C$}
    \UnaryInfC{$\Gamma,g:C^B,f:B^A\vdash \lam{x}g(f(x)):C^A$}
    \UnaryInfC{$\Gamma,g:B\to C\vdash \lam{f}\lam{x}g(f(x)):B^A\to C^A$}
    \UnaryInfC{$\Gamma\vdash\comp\defeq \lam{g}\lam{f}\lam{x}g(f(x)):C^B\to (B^A\to C^A)$}
  \end{prooftree}
\end{constr}

The rules of function types can be used to derive the laws of a category\index{category laws!for functions} for functions, i.e., we can derive that function composition is associative and that the identity function satisfies the unit laws. In the remainder of this section we will give these derivations.

\begin{lem}
Composition of functions is associative\index{associativity!of function composition}\index{composition!of functions!associativity}, i.e., we can derive
\begin{prooftree}
\AxiomC{$\Gamma\vdash f:A\to B$}
\AxiomC{$\Gamma\vdash g:B\to C$}
\AxiomC{$\Gamma\vdash h:C\to D$}
\TrinaryInfC{$\Gamma \vdash (h\circ g)\circ f\jdeq h\circ(g\circ f):A\to D$}
\end{prooftree}
\end{lem}

\begin{proof}
  The main idea of the proof is that both $((h\circ g)\circ f)(x)$ and $(h\circ (g\circ f))(x)$ evaluate to $h(g(f(x))$, and therefore $(h\circ g)\circ f$ and $h\circ(g\circ f)$ must be judgmentally equal. This idea is made formal in the following derivation:
  \begin{prooftree}
    \AxiomC{$\Gamma\vdash f:A\to B$}
    \UnaryInfC{$\Gamma,x:A\vdash f(x):B$}
    \AxiomC{$\Gamma\vdash g:B\to C$}
    \UnaryInfC{$\Gamma,y:B\vdash g(y):C$}
    \UnaryInfC{$\Gamma,x:A,y:B\vdash g(y):C$}
    \BinaryInfC{$\Gamma,x:A\vdash g(f(x)):C$}
    \AxiomC{$\Gamma\vdash h:C\to D$}
    \UnaryInfC{$\Gamma,z:C\vdash h(z):D$}
    \UnaryInfC{$\Gamma,x:A,z:C\vdash h(z):D$}
    \BinaryInfC{$\Gamma,x:A\vdash h(g(f(x))):D$}
    \UnaryInfC{$\Gamma,x:A\vdash h(g(f(x)))\jdeq h(g(f(x))):D$}
    \UnaryInfC{$\Gamma,x:A\vdash (h\circ g)(f(x))\jdeq h((g\circ f)(x)):D$}
    \UnaryInfC{$\Gamma,x:A\vdash ((h\circ g)\circ f)(x)\jdeq (h\circ (g \circ f))(x):D$}
    \UnaryInfC{$\Gamma\vdash (h\circ g)\circ f\jdeq h\circ(g\circ f):A\to D$.}
  \end{prooftree}
\end{proof}

\begin{lem}\label{lem:fun_unit}
Composition of functions satisfies the left and right unit laws\index{left unit law|see {unit laws}}\index{right unit law|see {unit laws}}\index{unit laws!for function composition}\index{composition!of functions!unit laws}, i.e., we can derive
\begin{prooftree}
\AxiomC{$\Gamma\vdash f:A\to B$}
\UnaryInfC{$\Gamma\vdash \idfunc[B]\circ f\jdeq f:A\to B$}
\end{prooftree}
and
\begin{prooftree}
\AxiomC{$\Gamma\vdash f:A\to B$}
\UnaryInfC{$\Gamma\vdash f\circ\idfunc[A]\jdeq f:A\to B$}
\end{prooftree}
\end{lem}

\begin{proof}
The derivation for the left unit law is
\begin{prooftree}
\AxiomC{$\Gamma\vdash f:A\to B$}
\UnaryInfC{$\Gamma,x:A\vdash f(x):B$}
\AxiomC{$\Gamma\vdash B~\type$}
\UnaryInfC{$\Gamma,y:B\vdash \idfunc[B](y)\jdeq y:B$}
\UnaryInfC{$\Gamma,x:A,y:B\vdash \idfunc[B](y)\jdeq y:B$}
\BinaryInfC{$\Gamma,x:A\vdash \idfunc[B](f(x))\jdeq f(x):B$}
\UnaryInfC{$\Gamma,x:A\vdash (\idfunc[B]\circ f)(x)\jdeq f(x):B$}
\UnaryInfC{$\Gamma\vdash \idfunc[B]\circ f\jdeq f:A\to B$}
\end{prooftree}
The right unit law is left as \cref{ex:fun_right_unit}.
\end{proof}

\begin{exercises}
\item \label{ex:fun_right_unit}Give a derivation for the right unit law of \cref{lem:fun_unit}.\index{unit laws!for function composition}
\item Show that the rule
\begin{prooftree}
\AxiomC{$\Gamma,x:A \vdash b(x) : B(x)$}
\RightLabel{$\lambda$-$x'/x$}
\UnaryInfC{$\Gamma\vdash \lam{x}b(x)\jdeq \lam{x'}b(x') : \prd{x:A}B(x)$}
\end{prooftree}
is admissible for any variable $x'$ that does not occur in the context $\Gamma,x:A$.
\item 
  \begin{subexenum}
  \item Construct the \define{constant function}\index{constant function}\index{function!constant function}\index{const x@{$\const_x$}}\index{function!const@{$\const$}}
    \begin{prooftree}
      \AxiomC{$\Gamma\vdash A~\textrm{type}$}
      \UnaryInfC{$\Gamma,y:B\vdash \const_y:A\to B$}
    \end{prooftree}
  \item Show that
    \begin{prooftree}
      \AxiomC{$\Gamma\vdash f:A\to B$}
      \UnaryInfC{$\Gamma,z:C\vdash \const_z\circ f\jdeq\const_z : A\to C$}
    \end{prooftree}
  \item Show that
    \begin{prooftree}
      \AxiomC{$\Gamma\vdash A~\textrm{type}$}
      \AxiomC{$\Gamma\vdash g:B\to C$}
      \BinaryInfC{$\Gamma,y:B\vdash g\circ\const_y\jdeq \const_{g(y)}:A\to C$}
    \end{prooftree}
  \end{subexenum}
\item In this exercise we generalize the composition operation of non-dependent function types\index{composition!of dependent functions}:
\begin{subexenum}
\item Define a composition operation for dependent function types
\begin{prooftree}
\AxiomC{$\Gamma\vdash f:\prd{x:A}B(x)$}
\AxiomC{$\Gamma\vdash g:\prd{x:A}\prd{y:B(x)} C(x,y)$}
\BinaryInfC{$\Gamma\vdash g\circ' f:\prd{x:A} C(x,f(x))$}
\end{prooftree}
and show that this operation agrees with ordinary composition when it is specialized to non-dependent function types.
\item Show that composition of dependent functions agrees with ordinary composition of functions:
  \begin{prooftree}
    \AxiomC{$\Gamma\vdash f:A\to B$}
    \AxiomC{$\Gamma\vdash g:B\to C$}
    \BinaryInfC{$\Gamma\vdash (\lam{x}g)\circ' f\jdeq g\circ f:A \to C$}
  \end{prooftree}
\item Show that composition of dependent functions is associative.\index{associativity!of dependent function composition}\index{composition!of dependent functions!associativity}
\item Show that composition of dependent functions satisfies the right unit law\index{unit laws!dependent function composition}:
\begin{prooftree}
\AxiomC{$\Gamma\vdash f:\prd{x:A}B(x)$}
\UnaryInfC{$\Gamma\vdash (\lam{x}f)\circ'\idfunc[A]\jdeq f :\prd{x:A}B(x)$}
\end{prooftree}
\item Show that composition of dependent functions satisfies the left unit law\index{unit laws!dependent function composition}\index{composition!of dependent functions!unit laws}:
\begin{prooftree}
\AxiomC{$\Gamma\vdash f:\prd{x:A}B(x)$}
\UnaryInfC{$\Gamma\vdash (\lam{x}\idfunc[B(x)])\circ' f\jdeq f:\prd{x:A}B(x)$}
\end{prooftree}
\end{subexenum}
\item \label{ex:swap}
\begin{subexenum}
\item Given two types $A$ and $B$ in context $\Gamma$, and a type $C$ in context $\Gamma,x:A,y:B$, define the \define{swap function}\index{function!swap}\index{swap function}
\begin{equation*}
\Gamma\vdash \sigma:\Big(\prd{x:A}\prd{y:B}C(x,y)\Big)\to\Big(\prd{y:B}\prd{x:A}C(x,y)\Big)
\end{equation*}
that swaps the order of the arguments.
\item Show that
\begin{equation*}
\Gamma\vdash \sigma\circ\sigma\jdeq\idfunc:\Big(\prd{x:A}\prd{y:B}C(x,y)\Big)\to \Big(\prd{x:A}\prd{y:B}C(x,y)\Big).
\end{equation*}
\end{subexenum}
\end{exercises}
\index{dependent function type|)}

\chapter{Inductive types}

Analogous to the type of natural numbers, many types can be specified as inductive types. In this lecture we introduce some further examples of inductive types: the unit type, the empty type, the booleans, coproducts, dependent pair types, and cartesian products. We also introduce the type of integers.

\section{The idea of general inductive types}

Just like the type of natural numbers, other inductive types are also specified by their \emph{constructors}, an \emph{induction principle}, and their \emph{computation rules}: 
\begin{enumerate}
\item The constructors tell what structure the inductive type comes equipped with. There may any finite number of constructors, even no constructors at all, in the specification of an inductive type. 
\item The induction principle specifies the data that should be provided in order to construct a section of an arbitrary type family over the inductive type. 
\item The computation rules assert that the inductively defined section agrees on the constructors with the data that was used to define the section. Thus, there is a computation rule for every constructor.
\end{enumerate}
The induction principle and computation rules can be generated automatically once the constructors are specified, but it goes beyond the scope of our course to describe general inductive types.
%For a more general treatment of inductive types, we refer to Chapter 5 of \cite{hottbook}.


\section{The unit type}
A straightforward example of an inductive type is the \emph{unit type}, which has just one constructor. 
Its induction principle is analogous to just the base case of induction on the natural numbers.

\begin{defn}
We define the \define{unit type}\index{1@{$\unit$}|see {unit type}}\index{unit type|textbf} to be a closed type $\unit$ equipped with a closed term\index{unit type!star@{$\ttt$}}
\begin{equation*}
\ttt:\unit,
\end{equation*}
satisfying the induction principle\index{induction principle!of unit type} that for any type family $\Gamma,x:\unit\vdash P(x)~\mathrm{type}$, there is a term
\begin{equation*}
\indunit : P(\ttt)\to\prd{x:\unit}P(x)
\end{equation*}
in context $\Gamma$ for which the computation rule\index{computation rules!of unit type}
\begin{equation*}
\indunit(p,\ttt) \jdeq p
\end{equation*}
holds. Sometimes we write $\lam{\ttt}p$ for $\indunit(p)$.
\end{defn}

Using the induction principle for the unit type, one can construct in the context $x:A$ a function $\mathsf{pt}_x:\unit\to A$. Indeed, we have
\begin{equation*}
  \indunit : A \to (\unit\to A).
\end{equation*}

\section{The empty type}
The empty type is a degenerate example of an inductive type. It does \emph{not} come equipped with any constructors, and therefore there are also no computation rules. The induction principle merely asserts that any type family has a section. In other words: if we assume the empty type has a term, then we can prove anything.

\begin{defn}
We define the \define{empty type}\index{0@{$\emptyt$}|see {empty type}}\index{empty type|textbf} to be a type $\emptyt$ satisfying the induction principle\index{induction principle!of empty type} that for any type family $P:\emptyt\to\type$, there is a term
\begin{equation*}
\indempty : \prd{x:\emptyt}P(x).
\end{equation*}
\end{defn}
In particular, there is a function
\begin{equation*}
  \emptyt\to A
\end{equation*}
for any type $A$, by the induction principle for the empty type. Using the empty type we can also define \emph{negation}. The idea is that if $A$ is false (i.e., has no terms), then from $A$ follows everything.

\begin{defn}
For any type $A$, we define $\neg A\defeq A\to \emptyt$.\index{negation!of a type}\index{not ($\neg$)|see {negation, of a type}}
\end{defn}

\section{The booleans}
\begin{defn}
We define the \define{booleans}\index{booleans|textbf}\index{2@{$\bool$}|see {booleans}} to be a type $\bool$ that comes equipped with
\begin{align*}
\bfalse & : \bool \\
\btrue & : \bool
\end{align*}
satisfying the induction principle\index{induction principle!of booleans} that for any type family $P:\bool\to\type$, there is a term
\begin{equation*}
\indbool : P(\bfalse)\to \Big(P(\btrue)\to \prd{x:\bool}P(x)\Big)
\end{equation*}
for which the computation rules\index{computation rules!of booleans}
\begin{align*}
\indbool(p_0,p_1,\bfalse) & \jdeq p_0 \\
\indbool(p_0,p_1,\btrue) & \jdeq p_1
\end{align*}
hold.
\end{defn}

\section{Coproducts}
\begin{defn}
Let $A$ and $B$ be types. We define the \define{coproduct}\index{coproduct}\index{disjoint sum|see {coproduct}} $A+B$\index{plus ($+$)|see {coproduct}} to be a type that comes equipped with
\begin{align*}
\inl & : A \to A+B \\
\inr & : B \to A+B
\end{align*}
satisfying the induction principle\index{induction principle!of coproduct} that for any type family $P:(A+B)\to\type$, there is a term
\begin{equation*}
\ind{+} : \Big(\prd{x:A}P(\inl(x))\Big)\to\Big(\prd{y:B}P(\inr(y))\Big)\to\prd{z:A+B}P(z)
\end{equation*}
for which the computation rules\index{computation rules!of coproduct}
\begin{align*}
\ind{+}(f,g,\inl(x)) & \jdeq f(x) \\
\inr{+}(f,g,\inr(y)) & \jdeq g(y)
\end{align*}
hold. Sometimes we write $[f,g]$ for $\ind{+}(f,g)$.
\end{defn}

The coproduct of two types is sometimes also called the \define{disjoint sum}.
When one thinks of types as propositions, then the coproduct plays the role of the disjunction.\index{propositions as types!disjunction}
To construct a term of type $A+B$ you first have to decide whether it is of the form $\inl$ or $\inr$, and then you construct a term of $A$ or $B$ accordingly. Of course, this is to be contrasted with the \emph{double negation translation}\index{double negation translation!disjunction} of the disjunction, which is read as `not neither $A$ nor $B$'. 

\section{Dependent pair types}
The \emph{dependent pair type}\index{dependent pair type|see {$\Sigma$-type}} (or $\Sigma$-type) can be thought of as a `type indexed' disjoint sum.
However, this intuition for the dependent pair type can be counterproductive once we start to do homotopy theory in type theory.
It is better to think of the $\Sigma$-type as the total space of a family of types depending continuously on a base type, just like one can have a family of spaces depending continuously on a base space (i.e., a fibration).

\begin{defn}
Let $A$ be a type in context $\Gamma$, and let $\Gamma,x:A\vdash B(x)~\mathrm{type}$ be a type family over $A$.
The \define{dependent pair type}\index{Sigma type@{$\Sigma$-type}|textbf} is defined to be the inductive type $\sm{x:A}B(x)$ in context $\Gamma$ equipped with a \define{pairing function}\index{pairing function}
\begin{equation*}
\pairr{\blank,\blank}:\prd{x:A} \Big(B(x)\to \sm{y:A}B(y)\Big).
\end{equation*}
The induction principle\index{induction principle|of Sigma types@{of $\Sigma$-types}} for $\sm{x:A}B(x)$ asserts that for every type family 
\begin{equation*}
\Gamma,p:\sm{x:A}B(x)\vdash P(p)~\mathrm{type}
\end{equation*}
one has
\begin{equation*}
\ind{\Sigma}:\Big(\prd{x:A}{y:B(x)}P(\pairr{x,y})\Big)\to\Big(\prd{p:\sm{x:A}B(x)}P(p)\Big).
\end{equation*}
satisfying the computation rule\index{computation rules!of Sigma types@{of $\Sigma$-types}}
\begin{equation*}
\ind{\Sigma}(f,\pairr{x,y})\jdeq f(x,y).
\end{equation*}
Most of the time we write $\lam{(x,y)}f(x,y)$ for $\ind{\Sigma}(\lam{x}{y}f(x,y))$. 
\end{defn}

\begin{rmk}
Some authors write $(x:A)\times B(x)$ for the dependent pair type $\sm{x:A}B(x)$. 
\end{rmk}

\begin{defn}
Given a type $A$ and a type family $B$ over $A$, the \define{first projection map}\index{first projection map|textbf}\index{projection maps!first projection|textbf}
\begin{equation*}
\proj 1:\Big(\sm{x:A}B(x)\Big)\to A
\end{equation*}
is defined by induction as
\begin{equation*}
\proj 1\defeq \lam{(x,y)}x.
\end{equation*}
The \define{second projection map}\index{second projection map|textbf}\index{projection map!second projection|textbf} is a dependent function
\begin{equation*}
\prd{p:\sm{x:A}B(x)} B(\proj 1(p))
\end{equation*}
defined by induction as
\begin{equation*}
\proj 2\defeq \lam{(x,y)}y.
\end{equation*}
By the computation rule we have
\begin{align*}
\proj 1 \pairr{x,y} & \jdeq x \\
\proj 2 \pairr{x,y} & \jdeq y.
\end{align*}
\end{defn}

When one thinks of types as propositions, then the $\Sigma$-type has the r\^{o}le of the existential quantification.

\section{Cartesian products}
A special case of the $\Sigma$-type occurs when the $B$ is a type in context $\Gamma$ weakened by $A$ (i.e., $B$ is not actually depending on $A$). In this case, a term of $\sm{x:A}B$ is given as a pair consisting of a term of $A$ and a term of $B$. Thus, $\sm{x:A}B$ is the \emph{(cartesian) product} $A\times B$. Since the cartesian product is so common (just like $A\to B$ is a common special case of the dependent product), we provide its specification along with the induction principle for cartesian products.

\begin{defn}
Let $A$ and $B$ be types in context $\Gamma$. The \define{(cartesian) product}\index{cartesian product|textbf}\index{product!of types|textbf} of $A$ and $B$ is defined as the inductive type $A\times B$\index{times ($\times$)|see {cartesian product}} with constructor
\begin{equation*}
\pairr{\blank,\blank}:A\to (B\to A\times B).
\end{equation*}
The induction principle\index{induction principle!of cartesian products} for $A\times B$ asserts that for any type family $P$ over $A\times B$, one has
\begin{equation*}
\ind{\times} : \Big(\prd{x:A}{y:B}P(\pairr{\blank,\blank})\Big)\to\Big(\prd{p:A\times B} P(p)\Big)
\end{equation*}
satisfying the computation rule\index{computation rules!of cartesian product} that
\begin{align*}
\ind{\times}(f,x,y) & \jdeq f(x,y).
\end{align*}
\end{defn}

The projection maps are defined similarly to the projection maps of $\Sigma$-types. When one thinks of types as propositions\index{propositions as types!conjunction}, then $A\times B$ is interpreted as the conjunction of $A$ and $B$.

\section{The type of integers}

\begin{defn}
We define the \define{integers}\index{integers|see Z@{$\Z$}} to be the type $\Z\defeq\nat+(\unit+\nat)$, and we write
\begin{align*}
\mathsf{neg} & \defeq \inl & &  : \N \to \Z \\
-1 & \defeq \mathsf{neg}(0) & & : \Z \\
0 & \defeq \inr(\ttt) & & : \Z \\
\mathsf{pos} & \defeq \inr\circ\inr & & : \N\to \Z \\
1 & \defeq \mathsf{pos}(0) & & : \Z.
\end{align*}
\end{defn}

In the following lemma we derive an alternative induction principle\index{induction principle!of Z@{of $\Z$}} for $\Z$, which makes it easier to make definitions.
\begin{lem}
\label{lem:Z_ind}
For any $\Gamma,k:\Z\vdash P(k)~\mathrm{type}$ we have
\begin{prooftree}
\Axiom$\fCenter\Gamma\vdash p_{-1}:P(-1)$
\noLine
\UnaryInf$\fCenter\Gamma\vdash p_{-S} : \prd{n:\N}P(\mathsf{neg}(n))\to P(\mathsf{neg}(S(n)))$
\noLine
\UnaryInf$\fCenter\Gamma\vdash p_0: P(0)$
\noLine
\UnaryInf$\fCenter\Gamma\vdash p_{1}:P(1)$
\noLine
\UnaryInf$\fCenter\Gamma\vdash p_{-S} : \prd{n:\N}P(\mathsf{pos}(n))\to P(\mathsf{pos}(S(n)))$
\UnaryInf$\fCenter\Gamma\vdash\ind{\Z}(p_{-1},p_{-S},p_{0},p_{1},p_{S}):\prd{k:\Z}P(k)$
\end{prooftree}
The term $\ind{\Z}(p_{-1},p_{-S},p_{0},p_{1},p_{S})$ furthermore satisfies the following computation rules:
\begin{align*}
\ind{\Z}(p_{-1},p_{-S},p_{0},p_{1},p_{S},-1) & \jdeq p_{-1} \\
\ind{\Z}(p_{-1},p_{-S},p_{0},p_{1},p_{S},\mathsf{neg}(S(n))) & \jdeq p_{-S}(n,\ind{\Z}(p_{-1},p_{-S},p_{0},p_{1},p_{S},\mathsf{neg}(n))) \\
\ind{\Z}(p_{-1},p_{-S},p_{0},p_{1},p_{S},0) & \jdeq p_0 \\
\ind{\Z}(p_{-1},p_{-S},p_{0},p_{1},p_{S},1) & \jdeq p_1 \\
\ind{\Z}(p_{-1},p_{-S},p_{0},p_{1},p_{S},\mathsf{pos}(S(n))) & \jdeq p_{S}(n,\ind{\Z}(p_{-1},p_{-S},p_{0},p_{1},p_{S},\mathsf{pos}(n))).
\end{align*}
\end{lem}

As an application we define the successor function on the integers.

\begin{defn}
We define the \define{successor function}\index{successor function!on Z@{on $\Z$}|textbf} on the integers $S_\Z:\Z\to\Z$.
\end{defn}

\begin{constr}
We apply the induction principle of \autoref{lem:Z_ind}, taking
\begin{align*}
S_\Z(-1) & \defeq 0 \\
S_\Z(\mathsf{neg}(S(n))) & \defeq \mathsf{neg}(n) \\
S_\Z(0) & \defeq 1 \\
S_\Z(1) & \defeq \mathsf{pos}(S(1)) \\
S_\Z(\mathsf{pos}(S(n))) & \defeq \mathsf{pos}(S(S(n))).
\end{align*}
\end{constr}

\section{Overview of the inductive types}
\begin{center}
\begin{tabular}{llll}
\toprule
name & type & constructors \\
\midrule
\define{natural numbers} & $\N$ & $0:\N$ \\
& & $S:\N\to \N$ & \\
\define{empty type} & $\emptyt$ & {\color{black!20}(no constructors)}\\
\define{unit type} & $\unit$ & $\ttt:\unit$ \\
\define{booleans} & $\bool$ & $\bfalse:\bool$ \\
& & $\btrue : \bool$ \\
\define{coproduct} & $A+B$ & $\inl : A \to A+B$ \\
& & $\inr : B\to A+B$ & \\
\define{product} & $A\times B$ & $\pairr{\blank,\blank}:A\to (B\to A\times B)$ \\
\define{$\Sigma$-type} & $\sm{x:A}B(x)$ & $\pairr{\blank,\blank}:\prd{y:A} \big(B(y)\to \sm{x:A}B(x)\big)$ \\
\bottomrule
\end{tabular}
\end{center}

\begin{exercises}
\item Let $A$ be a type.
  \begin{subexenum}
  \item Show that $(A+\neg A)\to(\neg\neg A\to A)$.
  \item Show that $\neg\neg\neg A \to \neg A$.
  \end{subexenum}
\item \label{ex:one_plus_one} Show that $\unit+\unit$ satisfies the same induction principle\index{induction principle!of booleans} as $\bool$, i.e., define
\begin{align*}
t_0 & : \unit + \unit \\
t_1 & : \unit + \unit,
\end{align*}
and show that for every $\Gamma,t:\unit+\unit\vdash P(t)~\mathrm{type}$ there is a dependent function of type
\begin{align*}
\ind{\unit+\unit}:P(t_0)\to \Big(P(t_1)\to \prd{t:\unit+\unit}P(t)\Big)
\end{align*}
satisfying
\begin{align*}
\ind{\unit+\unit}(p_0,p_1,t_0) & \jdeq p_0 \\
\ind{\unit+\unit}(p_0,p_1,t_1) & \jdeq p_1.
\end{align*}
In other words, \emph{type theory cannot distinguish between $\bool$ and $\unit+\unit$.}
\item \label{ex:int_pred}\index{predecessor function|textbf}Define the predecessor function $\mathsf{pred}:\Z\to \Z$.
\item \label{ex:int_group_ops}\index{group operations!on Z@{on $\Z$}}Define operations $k,l\mapsto k+l:\Z\to\Z\to\Z$ and $k\mapsto -k:\Z\to \Z$.
\item Construct a function $F:\Z\to\Z$ that extends the Fibonacci sequence to the left
  \begin{equation*}
    \ldots,5,-3,2,-1,1,0,1,1,2,3,5,8,13,\ldots
  \end{equation*}
  in the expected way.
\end{exercises}

\chapter{The universe and type-valued relations}

In this lecture we introduce type theoretic \emph{universes}. Universes are types $\UU$ of which the terms again represent types. In other words, the universe comes equipped with a type family, which we call the \emph{universal type family}.

One reason for introducting universes is that it enables us to define new type families over inductive types, using their induction principle. Indeed, since the universe is itself a type, we can use the induction principle to eliminate into the universe. Then we obtain a type family over the inductive type by the universal type family.

We use this way of defining type families to define many familiar relations over $\N$, such as $\leq$ and $<$. We also introduce a relation called \emph{observational equality} $\mathsf{Eq}_\N$ on $\N$, which we can think of as equality of $\N$. This relation is reflexive, symmetric, and transitive, and moreover it is the least reflexive relation. Furthermore, one of the most important aspects of observational equality $\mathsf{Eq}_\N$ on $\N$ is that $\mathsf{Eq}_\N(m,n)$ is a type for every $m,n:\N$, unlike judgmental equality. Therefore we can use type theory to reason about observational equality on $\N$. Indeed, in the exercises we show that some very elementary mathematics can already be done at this early stage in our development of type theory.

\section{The universe}
The induction principle for inductive types can be used to prove universal quantifications. 
However, it would also be nice if we could construct \emph{new type families} over inductive types, using their induction principles.
To be able to do this, we introduce a \emph{universe}, a type of which the terms represent types. The idea is that the universe $\UU$ comes equipped with a type family $\mathrm{El}$, so that for each $X:\UU$ we have an associated type $\mathrm{El}(X)$, the type of \emph{elements} of $X$. 

We assume there is a closed type $\UU$\index{U@{$\UU$}|textbf} called the \define{universe}\index{universe|textbf}, and a type family $\mathrm{El}$\index{El@{$\mathrm{El}$}} over $\UU$ called the \define{universal family}\index{universal family}\index{family!universal family|textbf}.
\begin{center}
\begin{minipage}{.4\textwidth}
\begin{prooftree}
\AxiomC{}
\UnaryInfC{$\vdash\UU~\mathrm{type}$}
\end{prooftree}
\end{minipage}\quad
\begin{minipage}{.4\textwidth}
\begin{prooftree}
\AxiomC{}
\UnaryInfC{$X:\UU \vdash \mathrm{El}(X)~\mathrm{type}$}
\end{prooftree}
\end{minipage}
\end{center}

We postulate that the universe is closed under the type constructors, by the following rules:
\begin{enumerate}
\item The universe is closed under $\Pi$-types
\begin{prooftree}
\AxiomC{$\Gamma\vdash A:\UU$}
\AxiomC{$\Gamma\vdash B:\mathrm{El}(A)\to\UU$}
\BinaryInfC{$\Gamma \vdash \check{\Pi}(A,B):\UU$}
\end{prooftree}
\begin{prooftree}
\AxiomC{$\Gamma\vdash A:\UU$}
\AxiomC{$\Gamma\vdash B:\mathrm{El}(A)\to\UU$}
\BinaryInfC{$\Gamma \vdash \mathrm{El}(\check{\Pi}(A,B))\jdeq \prd{x:\mathrm{El}(A)}\mathrm{El}(B(x))~\mathrm{type}$}
\end{prooftree}
\item The type of natural numbers is in the universe
\begin{prooftree}
\AxiomC{}
\UnaryInfC{$\vdash \check{\N}:\UU$}
\end{prooftree}
\begin{prooftree}
\AxiomC{}
\UnaryInfC{$\vdash \mathrm{El}(\check{\N})\jdeq \mathbb{N}~\mathrm{type}$}
\end{prooftree}
\item Similarly we postulate that the universe contains the empty type, the unit type, the booleans, coproducts, products, and $\Sigma$-types. These closure properties of the universe are given concisely in \cref{tab:universe}.
\begin{table}
\begin{center}
\caption{\label{tab:universe}Closure properties of the universe}
\begin{tabular}{lll}
\toprule
Premises & Type encoding\index{type encoding@{type encoding $\check{A}$}} in $\UU$ & Type of elements $\mathrm{El}(\blank)$ \\
\midrule
$A:\UU,B:\mathrm{El}(A)\to\UU$ & $\check{\Pi}(A,B)$ & $\prd{x:\mathrm{El}(A)}\mathrm{El}(B(x))$ \\
& $\check{\nat}$ & $\nat$ \\
& $\check{\emptyt}$ &  $\emptyt$ \\
& $\check{\unit}$ &  $\unit$ \\
& $\check{\bool}$ &  $\bool$ \\
$A,B:\UU$ & $A\mathbin{\check{+}}B$ &  $\mathrm{El}(A)+\mathrm{El}(B)$ \\
$A,B:\UU$ & $A\mathbin{\check{\times}}B$ &  $\mathrm{El}(A)\times\mathrm{El}(B)$ \\
$A:\UU,B:\mathrm{El}(A)\to\UU$ & $\check{\Sigma}(A,B)$ & $\sm{x:\mathrm{El}(A)}\mathrm{El}(B(x))$ \\
\bottomrule
\end{tabular}
\end{center}
\end{table}
\end{enumerate}

\begin{defn}
We say that a type $A$ in context $\Gamma$ is \define{small}\index{small type|textbf} if it occurs in the universe, i.e., if there is a term $\check{A}:\UU$ in context $\Gamma$ such that $\Gamma\vdash\mathrm{El}(\check{A})\jdeq A~\mathrm{type}$.
\end{defn}

In particular, if $A$ is a small type in context $\Gamma$ and $B$ is a small type in context $\Gamma,x:A$, then $\prd{x:A}B(x)$ is again a small type in context $\Gamma$.

\begin{defn}
Let $A$ be a type in context $\Gamma$. A \define{family of small types}\index{family of small types|textbf} over $A$ is defined to be a map
\begin{equation*}
B:A\to\UU
\end{equation*}
\end{defn}

\begin{rmk}
If $A$ is small, we usually write simply $A$ for $\check{A}$ and also $A$ for $\mathrm{El}(\check{A})$. In other words, by $A:\UU$ we mean that $A$ is a small type. 
\end{rmk}

\begin{eg}
One important way to use the universe is to define types of \define{structured types}\index{structured types}. We give some examples:
\begin{enumerate}
\item The type of small \define{pointed types}\index{pointed types} is defined as
\begin{equation*}
\UU_\ast\defeq \sm{A:\UU}A,
\end{equation*}
\item The type of small \define{graphs}\index{graphs} is defined as the type
\begin{equation*}
\mathsf{Gph}_\UU \defeq \sm{A:\UU} A\to (A\to \UU),
\end{equation*}
\item The type of small \define{reflexive graphs}\index{reflexive graphs} is defined as the type
\begin{equation*}
\mathsf{rGph}_\UU \defeq \sm{A:\UU}{R:A\to (A\to \UU)}\prd{a:A}R(a,a).
\end{equation*}
\end{enumerate}
Once we have introduced the \emph{identity types} we will also be able to state the types of groups, rings, and many other structured types. However, when doing so one has to be cautious to make sure that the underlying type is in the level of sets, in the hierarchy of homotopical complexity of types.
\end{eg}

Another important way to use the universe is to \emph{define} new type families by induction. For example, we can define the finite types as family over the natural numbers.

\begin{defn}\label{defn:fin}
We define the type family $\mathsf{Fin}:\N\to\UU$ of finite types\index{Fin@{$\mathsf{Fin}$}|textbf}\index{finite types|textbf} by induction on $\N$\index{family!of finite types}, taking
\begin{align*}
\mathsf{Fin}(0) & \defeq \emptyt \\
\mathsf{Fin}(n+1) & \defeq \mathsf{Fin}(n)+\unit
\end{align*}
\end{defn}

\section{Observational equality on \texorpdfstring{$\N$}{ℕ}}
A second example of this kind is the notion of \emph{observational equality} on the natural numbers.

\begin{defn}\label{defn:obs_nat}
We define the \define{observational equality}\index{observational equality!on N@{on $\N$}} on $\N$ as binary relation $\mathrm{Eq}_\N:\N\to(\N\to\UU)$\index{Eq_N@{$\mathrm{Eq}_\N$}|textbf} satisfying
\begin{align*}
\mathrm{Eq}_\N(0,0) & \jdeq \unit & \mathrm{Eq}_\N(S(n),0) & \jdeq \emptyt \\
\mathrm{Eq}_\N(0,S(n)) & \jdeq \emptyt & \mathrm{Eq}_\N(S(n),S(m)) & \jdeq \mathrm{Eq}_\N(n,m).
\end{align*}
\end{defn}

\begin{constr}
We define $\mathrm{Eq}_\N$ by double induction on $\N$. By the first application of induction it suffices to provide
\begin{align*}
E_0 & : \N\to\UU \\
E_S & : \N\to (\N\to\UU)\to(\N\to\UU)
\end{align*}
We define $E_0$ by induction, taking $E_{00}\defeq \unit$ and $E_{0S}(n,X,m)\defeq \emptyt$. The resulting family $E_0$ satisfies
\begin{align*}
E_0(0) & \jdeq \unit \\
E_0(S(n)) & \jdeq \emptyt.
\end{align*} 
We define $E_S$ by induction, taking $E_{S0}\defeq \emptyt$ and $E_{S0}(n,X,m)\defeq X(m)$. The resulting family $E_S$ satisfies
\begin{align*}
E_S(n,X,0) & \jdeq \emptyt \\
E_S(n,X,S(m)) & \jdeq X(m) 
\end{align*}
Therefore we have by the computation rule for the first induction that the judgmental equality
\begin{align*}
\mathrm{Eq}_\N(0,m) & \jdeq E_0(m) \\
\mathrm{Eq}_\N(S(n),m) & \jdeq E_S(n,\mathrm{Eq}_\N(n),m)
\end{align*}
holds, from which the judgmental equalities in the statement of the definition follow.
\end{constr}

We can also define observational equality for many other kinds of types, such as $\bool$ or $\Z$. In each of these cases, what sets the observational equality apart from other relations is that it is the \emph{least} reflexive relation. However, the definitions of observational equality in the various situations are always very specific to the type they are defined on. In the next chapter we introduce the \emph{identity type}, which is an equality type defined uniformly for all types. In \cref{chap:hierarchy} we will show that the observational equality on $\N$ is equivalent to the identity type on $\N$.

\section{How many universes are there?}

\begin{exercises}
\item Construct a function
\begin{equation*}
\check{\Pi}:\prd{A:\UU} (\mathrm{El}(A)\to\UU)\to \UU
\end{equation*}
such that
\begin{equation*}
\mathrm{El}(\check{\Pi}(A,B))\jdeq \prd{x:\mathrm{El}(A)}\mathrm{El}(B(x))
\end{equation*}
holds for every $A:\UU$ and $B:\mathrm{El}(A)\to\UU$. 

\emph{A similar exercise can be posed for $\Sigma$ and $+$ (and for $\to$ and $\times$ as special cases of $\Pi$ and $\Sigma$).}
\item \label{ex:obs_nat_eqrel}Show that observational equality on $\N$\index{observational equality!on N@{on $\N$}!is an equivalence relation} is an equivalence relation\index{equivalence relation!observational equality on N@{observational equality on $\N$}}, i.e., construct terms of the following types:
\begin{align*}
& \prd{n:\N} \mathrm{Eq}_\N(n,n) \\
& \prd{n,m:\N} \mathrm{Eq}_\N(n,m)\to \mathrm{Eq}_\N(m,n) \\
& \prd{n,m,l:\N} \mathrm{Eq}_\N(n,m)\to (\mathrm{Eq}_\N(m,l)\to \mathrm{Eq}_\N(n,l)).
\end{align*}
\item \label{ex:obs_nat_least}\index{observational equality!on N@{on $\N$}!is least reflexive relation}Let $R$ be a reflexive binary relation\index{reflexive relation}\index{relation!reflexive} on $\N$, i.e., $R$ is of type $\N\to (\N\to\UU)$ and comes equipped with a term $\rho:\prd{n:\N}R(n,n)$. Show that
\begin{equation*}
\prd{n,m:\N} \mathrm{Eq}_\N(n,m)\to R(n,m).
\end{equation*}
\item \index{observational equality!on N@{on $\N$}!is preserved by functions}Show that every function $f:\N\to \N$ preserves observational equality in the sense that
\begin{equation*}
\prd{n,m:\N} \mathrm{Eq}_\N(n,m)\to \mathrm{Eq}_\N(f(n),f(m)).
\end{equation*}
\emph{Hint: to get the inductive step going the induction hypothesis has to be strong enough. Construct by double induction a term of type}
\begin{equation*}
\prd{n,m:\N}{f:\N\to\N} \mathrm{Eq}_\N(n,m)\to \mathrm{Eq}_\N(f(n),f(m)),
\end{equation*}
\emph{and pull out the universal quantification over $f:\N\to\N$ by \autoref{ex:swap}.}
\item 
  \begin{subexenum}
  \item Define the \define{order relations}\index{relation!order}\index{order relation} $\leq$ and $<$ on $\N$.
  \item Show that $\leq$ is reflexive and that $<$ is \define{anti-reflexive}\index{anti-reflexive}\index{relation!anti-reflexive}, i.e., that $\neg(n<n)$. 
  \item Show that both $\leq$ and $<$ are transitive, and that $n<S(n)$.
  \item Show that $k\leq \min(m,n)$ holds if and only if both $k\leq m$ and $k\leq n$ hold, and show that $\max(m,n)\leq k$ holds if and only if both $m\leq k$ and $n\leq k$ hold.
\end{subexenum}
\item \label{ex:obs_bool}\index{observational equality!on 2@{on $\bool$}}
\begin{subexenum}
\item Define observational equality $\mathrm{Eq}_\bool$\index{Eq_bool@{$\mathrm{Eq}_\bool$}|textbf} on the booleans.
\item Show that $\mathrm{Eq}_\bool$ is reflexive.\index{observational equality!on 2@{on $\bool$}!is reflexive}
\item Show that for any reflexive relation $R:\bool\to(\bool\to \UU)$ one has\index{observational equality!on 2@{on $\bool$}!is least reflexive relation}
\begin{equation*}
\prd{x,y:\bool} \mathrm{Eq}_\bool(x,y)\to R(x,y).
\end{equation*}
\end{subexenum}
\item \label{ex:int_order}
\begin{subexenum}
\item Define the order relations\index{relation!order}\index{order relation} $\leq$ and $<$ on and $\Z$.
\item For $k:\Z$, consider the type $\Z_{\geq k}\defeq \sm{n:\Z}n\geq k$. Construct
\begin{align*}
b_k & : \Z_{\geq k} \\
s_k & : \Z_{\geq k}\to\Z_{\geq k},
\end{align*}
and show that $\Z_{\geq k}$ satisfies the induction principle of the natural numbers\index{induction principle!of N@{of $\N$}}:
\begin{equation*}
\ind{\Z_{\geq k}} : P(b_k)\to \Big(\prd{n:\Z_{\geq k}} P(n)\to P(s_k(n))\Big)\to \Big(\prd{n:\Z_{\geq k}} P(n)\Big)
\end{equation*}
\end{subexenum}
\item
  \begin{subexenum}
  \item Show that $\N$ satisfies \define{strong induction}, i.e., construct for any type family $P$ over $\N$ a function of type
    \begin{equation*}
      P(\zeroN) \to \Big(\prd{k:\N}\Big(\prd{m:\N} (m\leq k) \to P(m)\Big)\to P(\succN(k))\Big) \to \prd{n:\N}P(n).
    \end{equation*}
  \item Show that $\N$ satisfies \define{ordinal induction}, i.e., construct for any type family $P$ over $\N$ a function of type
    \begin{equation*}
      \Big(\prd{k:\N} \Big(\prd{m:\N} (m< k) \to P(m)\Big)\to P(k)\Big) \to \prd{n:\N}P(n).
    \end{equation*}
  \end{subexenum}
\end{exercises}

\chapter{Identity types}\label{chap:identity}
From the perspective of types as proof-relevant propositions, how should we think of \emph{equality} in type theory? Given a type $A$, and two terms $x,y:A$, the equality $\id{x}{y}$ should again be a type. Indeed, we want to \emph{use} type theory to prove equalities. \emph{Dependent} type theory provides us with a convenient setting for this: the equality type $\id{x}{y}$ is dependent on $x,y:A$. 

Then, if $\id{x}{y}$ is to be a type, how should we think of the terms of $\id{x}{y}$. A term $p:\id{x}{y}$ witnesses that $x$ and $y$ are equal terms of type $A$. In other words $p:\id{x}{y}$ is an \emph{identification} of $x$ and $y$. In a proof-relevant world, there might be many terms of type $\id{x}{y}$. I.e., there might be many identifications of $x$ and $y$. And, since $\id{x}{y}$ is itself a type, we can form the type $\id{p}{q}$ for any two identifications $p,q:\id{x}{y}$. That is, since $\id{x}{y}$ is a type, we may also use the type theory to prove things \emph{about} identifications (for instance, that two given such identifications can themselves be identified), and we may use the type theory to perform constructions with them. As we will see shortly, we can give every type a groupoid-like structure.

Clearly, the equality type should not just be any type dependent on $x,y:A$. Then how do we form the equality type, and what ways are there to use identifications in constructions in type theory? The answer to both these questions is that we will form the identity type as an \emph{inductive} type, generated by just a reflexivity term providing an identification of $x$ to itself. The induction principle then provides us with a way of performing constructions with identifications, such as concatenating them, inverting them, and so on. Thus, the identity type is equipped with a reflexivity term, and further possesses the structure that are generated by its induction principle and by the type theory. This inductive construction of the identity type is elegant, beautifully simple, but far from trivial!

The situation where two terms can be identified in possibly more than one way is analogous to the situation in \emph{homotopy theory}, where two points of a space can be connected by possibly more than one \emph{path}. Indeed, for any two points $x,y$ in a space, there is a \emph{space of paths} from $x$ to $y$. Moreover, between any two paths from $x$ to $y$ there is a space of \emph{homotopies} between them, and so on. This leads to the homotopy interpretation of type theory, outlined in \cref{tab:homotopy_interpretation}. The connection between homotopy theory and type theory been made precise by the construction of homotopical models of type theory, and it has led to the fruitful research area of \emph{synthetic homotopy theory}, the subfield of \emph{homotopy type theory} that is the topic of this course.

\begin{table}
\begin{center}
\caption{\label{tab:homotopy_interpretation}The homotopy interpretation\index{Homotopy interpretation}}
\begin{tabular}{ll}
\toprule
\emph{Type theory} &  \emph{Homotopy theory} \\
\midrule
Types  & Spaces \\
Dependent types & Fibrations \\
Terms & Points \\
Dependent pair type & Total space \\
Identity type & Path fibration\\
\bottomrule
\end{tabular}
\end{center}
\end{table}

\section{The inductive definition of identity types}

\begin{defn}
  Consider a type $A$ and let $a:A$. Then we define the \define{identity type}\index{identity type|textbf} of $A$ at $a$ as an inductive family of types $a =_A x$ indexed by $x:A$, of which the constructor is
  \begin{equation*}
    \refl{a}:a=_Aa.
  \end{equation*}
  The induction principle of the identity type postulates that for any family of types $P(x,p)$ indexed by $x:A$ and $p:a=_A x$, there is a function
  \begin{equation*}
    \mathsf{path\usc{}ind}_a:P(a,\refl{a}) \to \prd{x:A}{p:a=_A x} P(x,p)
  \end{equation*}
  that satisfies $\mathsf{path\usc{}ind}_a(p,a,\refl{a})\jdeq p$.

  A term of type $a=_A x$ is also called an \define{identification}\index{identification|textbf} of $a$ with $x$, and sometimes it is called a \define{path}\index{path|textbf} from $a$ to $x$.
The induction principle for identity types is sometimes called \define{identification elimination}\index{identification elimination|textbf} or \define{path induction}\index{path induction|textbf}. We also write $\idtypevar{A}$\index{Id A@{$\idtypevar{A}$}|textbf} for the identity type on $A$, and often we write $a=x$ for the type of identifications of $a$ with $x$, omitting reference to the ambient type $A$.
\end{defn}

\begin{rmk}
  We see that the identity type is not just an inductive type, like the inductive types $\N$, $\emptyt$, and $\unit$ for example, but it is and inductive \emph{family} of types. Even though we have a type $a=_A x$ for any $x:A$, the constructor only provides a term $\refl{a}:a=_A a$, identifying $a$ with itself. The induction principle then asserts that in order to prove something about all identifications of $a$ with some $x:A$, it suffices to prove this assertion about $\refl{a}$ only. We will see in the next sections that this induction principle is strong enough to derive many familiar facts about equality, namely that it is a symmetric and transitive relation, and that all functions preserve equality.
\end{rmk}

\begin{rmk}
  Since the identity types require getting used to, we provide the formal rules
  for identity types. The identity type is formed by the formation rule:
  \begin{prooftree}
    \AxiomC{$\Gamma\vdash a:A$}
    \UnaryInfC{$\Gamma,x:A\vdash a=_A x~\mathrm{type}$}
  \end{prooftree}
  The constructor of the identity type is then given by the introduction rule:
  \begin{prooftree}
    \AxiomC{$\Gamma\vdash a:A$}
    \UnaryInfC{$\Gamma\vdash \refl{a}:a=_A a$}
  \end{prooftree}
  The induction principle is now given by the elimination rule:
  \begin{prooftree}
    \AxiomC{$\Gamma\vdash a:A$}
    \AxiomC{$\Gamma,x:A,p:a=_A x\vdash P(x,p)~\mathrm{type}$}
    \BinaryInfC{$\Gamma\vdash \mathsf{path\usc{}ind}_a:P(a,\refl{a})\to\prd{x:A}{p:a=_A x}P(x,p)$}
  \end{prooftree}
  And finally the computation rule is:
  \begin{prooftree}
    \AxiomC{$\Gamma\vdash a:A$}
    \AxiomC{$\Gamma,x:A,p:a=_A x\vdash P(x,p)~\mathrm{type}$}
    \BinaryInfC{$\Gamma\vdash \mathsf{path\usc{}ind}_a(p,a,\refl{a})\jdeq p : P(a,\refl{a})$}
  \end{prooftree}
  Furthermore we postulate that any universe is closed under identity types, i.e., that there is a function
  \begin{equation*}
    \check{\mathsf{Id}}:\prd{A:\UU}\mathrm{El}(A)\to\mathrm{El}(A)\to\UU
  \end{equation*}
  satisfying
  \begin{equation*}
    \mathrm{El}(\check{\mathsf{Id}}(A,a,x))\jdeq a=_{\mathrm{El}(A)} x.
  \end{equation*}
\end{rmk}

\begin{rmk}
  One might wonder whether it is also possible to form the identity type at a \emph{variable} of type $A$, rather than at a term. This is certainly possible: since we can form the identity type in \emph{any} context, we can form the identity type at a variable $x:A$ as follows:
  \begin{prooftree}
    \AxiomC{$\Gamma,x:A\vdash x:A$}
    \UnaryInfC{$\Gamma,x:A,y:A\vdash x=_A y~\mathrm{type}$}
  \end{prooftree}
  In this way we obtain the 'binary' identity type. Its constructor is then also indexed by $x:A$. We have the following introduction rule
  \begin{prooftree}
    \AxiomC{$\Gamma,x:A\vdash x:A$}
    \UnaryInfC{$\Gamma,x:A\vdash \refl{x}:x=_A x$}
  \end{prooftree}
  and similarly we have elimination and computation rules.
\end{rmk}
 

\begin{comment}
In the following lemma we show that the identity type on $A$ is contained in any reflexive relation on $A$.

\begin{lem}
Let $\Gamma,x:A,y:A\vdash R(x,y)~\mathrm{type}$\index{reflexive relation|textit}\index{relation!reflexive}, and suppose that $R$ is reflexive in the sense that there is a term
\begin{equation*}
\rho:\prd{x:A}R(x,x)
\end{equation*}
Then there is a term of type
\begin{equation*}
\prd{y:A} (x=_A y)\to R(x,y)
\end{equation*}
in context $\Gamma,x:A$.
\end{lem}

\begin{constr}
By weakening the reflexive relation $R$ we obtain
\begin{equation*}
\Gamma,x:A,y:A,\alpha:x=_A y\vdash R(x,y)~\mathrm{type},
\end{equation*}
on which the induction principle is applicable.
Thus we see that by the induction principle for identity types we have a term
\begin{equation*}
\mathsf{path\usc{}ind}_x : R(x,x)\to \prd{y:A}(x=_A y)\to R(x,y)
\end{equation*}
so it suffices to construct a term of type $R(x,x)$, which we have by reflexivity of $R$.
\end{constr}
\end{comment}

\section{The groupoid structure of types}\label{sec:groupoid}
We show that identifications can be \emph{concatenated} and \emph{inverted}, which corresponds to the transitivity and symmetry of the identity type. 

Furthermore, we observe that we can iteratively take identity types, i.e., we can take identity types of identity types, 
\begin{equation*}
p =_{(x=_Ay)} q,
\end{equation*}
and so on. In other words, for any two identifications $p,q:x=_A y$, there is a type of identifications of $p$ with $y$. One way to think about this is that the identifications $p,q:x=_A y$ are paths in the type (space) $A$, and an identification of $p$ with $q$ is a \emph{higher path} from $p$ to $q$, i.e., a \emph{homotopy}.

Using the observation that identity types can be iterated we show that concatenation is \emph{associative}, satisfies the left and right \emph{unit laws}, and satisfies the left and right \emph{inverse laws}. These are the \define{groupoid operations} on the identity type.

\begin{defn}\label{defn:id_concat}
Let $A$ be a type. We define the \define{concatenation}\index{concatenation!for identifications}\index{concat@{$\mathsf{concat}$}} operation
\begin{equation*}
\mathsf{concat} : \prd{x,y,z:A} (\id{x}{y})\to(\id{y}{z})\to (\id{x}{z}).
\end{equation*}
We will write $\ct{p}{q}$ for $\mathsf{concat}(p,q)$.
\end{defn}

\begin{constr}
We construct the concatenation operation by path induction. It suffices to construct
\begin{equation*}
\mathsf{concat}(\refl{x}):\prd{z:A} (x=z)\to(x=z).
\end{equation*}
Here we take $\mathsf{concat}(\refl{x})_z \jdeq \idfunc[(x=z)]$. 
Explicitly, the term we have constructed is
\begin{equation*}
\lam{x}\mathsf{path\usc{}ind}_x(\lam{z}\idfunc[(\id{x}{z})]):\prd{x,y:A} (x=y)\to \prd{z:A} (y=z)\to (x=z).
\end{equation*}
To obtain a term of the asserted type we need to swap the order of the arguments $p:x=y$ and $z:A$, using \cref{ex:swap}.
\end{constr}

\begin{defn}\label{defn:id_inv}
Let $A$ be a type. We define the \define{inverse operation}\index{inverse operation!for identifications|textbf}\index{inv@{$\mathsf{inv}$}|textbf}
\begin{equation*}
\mathsf{inv}:\prd{x,y:A} (x=y)\to (y=x).
\end{equation*}
Most of the time we will write $p^{-1}$ for $\mathsf{inv}(p)$.
\end{defn}

\begin{constr}
We construct the inverse operation by path induction. It suffices to construct
\begin{equation*}
\mathsf{inv}(\refl{x}): x=x,
\end{equation*}
for any $x:A$. Here we take $\mathsf{inv}(\refl{x})\defeq \refl{x}$.
\end{constr}

\begin{defn}\label{defn:id_assoc}
Let $A$ be a type. We define the \define{associativity operation}\index{associativity operation!for identifications|textbf}, which assigns to each $p:x=y$, $q:y=z$, and $r:z=w$ the \define{associator}
\begin{equation*}
\mathsf{assoc}(p,q,r) : \ct{(\ct{p}{q})}{r}=\ct{p}{(\ct{q}{r})}.
\end{equation*}
\end{defn}

\begin{constr}
By identification elimination it suffices to show that
\begin{equation*}
\prd{z:A}{q:x=z}{z':A}{r:z=w} \ct{(\ct{\refl{x}}{q})}{r}= \ct{\refl{x}}{(\ct{q}{r})}.
\end{equation*}
Let $q:x=z$ and $r:z=w$. Note that by the computation rule $\ct{\refl{x}}{q}\jdeq q$, so $\ct{(\ct{\refl{x}}{q})}{r}\jdeq \ct{q}{r}$. Similarly we have $\ct{\refl{x}}{(\ct{q}{r})}\jdeq \ct{q}{r}$. Therefore we can simply take $\refl{\ct{q}{r}}$.
\end{constr}

\begin{defn}\label{defn:id_unit}
Let $A$ be a type. We define the left and right \define{unit law operations}\index{unit law operations!for identifications|textbf}, which assigns to each $p:x=y$ the terms\index{left unit@{$\mathsf{left\usc{}unit}$}|textbf}\index{right unit@{$\mathsf{right\usc{}unit}$}|textbf}
\begin{align*}
\mathsf{left\usc{}unit}(p) & : \ct{\refl{x}}{p}=p \\
\mathsf{right\usc{}unit}(p) & : \ct{p}{\refl{y}}=p,
\end{align*}
respectively.
\end{defn}

\begin{constr}
By identification elimination it suffices to construct
\begin{align*}
\mathsf{left\usc{}unit}(\refl{x}) & : \ct{\refl{x}}{\refl{x}} = \refl{x} \\
\mathsf{right\usc{}unit}(\refl{x}) & : \ct{\refl{x}}{\refl{x}} = \refl{x}.
\end{align*}
In both cases we take $\refl{\refl{x}}$.
\end{constr}

\begin{defn}\label{defn:id_invlaw}
Let $A$ be a type. We define left and right \define{inverse law operations}\index{inverse law operations!for identifications|textbf}\index{left inv@{$\mathsf{left\usc{}inv}$}|textbf}\index{right inv@{$\mathsf{right\usc{}inv}$}|textbf}
\begin{align*}
\mathsf{left\usc{}inv}(p) & : \ct{p^{-1}}{p} = \refl{y} \\
\mathsf{right\usc{}inv}(p) & : \ct{p}{p^{-1}} = \refl{x}.
\end{align*}
\end{defn}

\begin{constr}
By identification elimination it suffices to construct
\begin{align*}
\mathsf{left\usc{}inv}(\refl{x}) & : \ct{\refl{x}^{-1}}{\refl{x}} = \refl{x} \\
\mathsf{right\usc{}inv}(\refl{x}) & : \ct{\refl{x}}{\refl{x}^{-1}} = \refl{x}.
\end{align*}
Using the computation rules we see that
\begin{equation*}
\ct{\refl{x}^{-1}}{\refl{x}}\jdeq \ct{\refl{x}}{\refl{x}}\jdeq\refl{x},
\end{equation*}
so we define $\mathsf{left\usc{}inv}(\refl{x})\defeq \refl{\refl{x}}$. Similarly it follows from the computation rules that
\begin{equation*}
\ct{\refl{x}}{\refl{x}^{-1}} \jdeq \refl{x}^{-1}\jdeq \refl{x}
\end{equation*}
so we again define $\mathsf{right\usc{}inv}(\refl{x})\defeq\refl{\refl{x}}$. 
\end{constr}

\section{The action on paths of functions}

Using the induction principle of the identity type we can show that every function preserves identifications.
In other words, every function sends identified terms to identified terms.
Note that this is a form of continuity for functions in type theory: if there is a path that identifies two points $x$ and $y$ of a type $A$, then there also is a path that identifies the values $f(x)$ and $f(y)$ in the codomain of $f$. 

\begin{defn}\label{defn:ap}
Let $f:A\to B$ be a map. We define the \define{action on paths}\index{action on paths}\index{function!action on paths} of $f$ as an operation\index{ap f@{$\apfunc{f}$}|textbf}
\begin{equation*}
\apfunc{f} : \prd*{x,y:A} (\id{x}{y})\to(\id{f(x)}{f(y)}).
\end{equation*}
Moreover, there are operations\index{ap idfun@{$\mathsf{ap\usc{}idfun}$}|textbf}\index{ap comp@{$\mathsf{ap\usc{}comp}$}|textbf}
\begin{align*}
\mathsf{ap\usc{}idfun}_A & : \prd*{x,y:A}{p:\id{x}{y}} \id{p}{\ap{\idfunc[A]}{p}} \\
\mathsf{ap\usc{}comp}(f,g) & : \prd*{x,y:A}{p:\id{x}{y}} \id{\ap{g}{\ap{f}{p}}}{\ap{g\circ f}{p}}.
\end{align*}
\end{defn}

\begin{constr}
First we define $\apfunc{f}$ by identity elimination, taking
\begin{equation*}
\apfunc{f}(\refl{x})\defeq \refl{f(x)}.
\end{equation*}
Next, we construct $\mathsf{ap\usc{}idfun}_A$ by identity elimination, taking
\begin{equation*}
\mathsf{ap\usc{}idfun}_A(\refl{x}) \defeq \refl{\refl{x}}.
\end{equation*}
Finally, we construct $\mathsf{ap\usc{}comp}(f,g)$ by identity elimination, taking
\begin{equation*}
\mathsf{ap\usc{}comp}(f,g,\refl{x}) \defeq \refl{g(f(x))}.\qedhere
\end{equation*}
\end{constr}

\begin{defn}\label{defn:ap-preserve}
Let $f:A\to B$ be a map. Then there are identifications
\begin{align*}
\mathsf{ap\usc{}refl}(f,x) & : \id{\ap{f}{\refl{x}}}{\refl{f}(x)} \\
\mathsf{ap\usc{}inv}(f,p) & : \id{\ap{f}{p^{-1}}}{\ap{f}{p}^{-1}} \\
\mathsf{ap\usc{}concat}(f,p,q) & : \id{\ap{f}{\ct{p}{q}}}{\ct{\ap{f}{p}}{\ap{f}{q}}}
\end{align*}
for every $p:\id{x}{y}$ and $q:\id{x}{y}$.
\end{defn}

\begin{constr}
To construct $\mathsf{ap\usc{}refl}(f,x)$ we simply observe that ${\ap{f}{\refl{x}}}\jdeq {\refl{f}(x)}$, so we take
\begin{equation*}
\mathsf{ap\usc{}refl}(f,x)\defeq\refl{\refl{f(x)}}.
\end{equation*}
We construct $\mathsf{ap\usc{}inv}(f,p)$ by identification elimination on $p$, taking
\begin{equation*}
\mathsf{ap\usc{}inv}(f,\refl{x}) \defeq \refl{\ap{f}{\refl{x}}}.
\end{equation*}
Finally we construct $\mathsf{ap\usc{}concat}(f,p,q)$ by identification elimination on $p$, taking
\begin{equation*}
\mathsf{ap\usc{}concat}(f,\refl{x},q)  \defeq \refl{\ap{f}{q}}.\qedhere
\end{equation*}
\end{constr}

\section{Transport}

Dependent types also come with an action on paths: the \emph{transport} functions.
Given an identification $p:\id{x}{y}$ in the base type $A$, we can transport any term $b:B(x)$ to the fiber $B(y)$.
The transport functions have many applications, which we will encounter throughout this course.

\begin{defn}
Let $A$ be a type, and let $B$ be a type family over $A$.
We will construct a \define{transport}\index{transport|textbf} operation\index{tr B@{$\mathsf{tr}_B$}|textbf}
\begin{equation*}
\mathsf{tr}_B:\prd*{x,y:A} (\id{x}{y})\to (B(x)\to B(y)).
\end{equation*}
\end{defn}

\begin{constr}
We construct $\mathsf{tr}_B(p)$ by induction on $p:x=_A y$, taking
\begin{equation*}
\mathsf{tr}_B(\refl{x}) \defeq \idfunc[B(x)].\qedhere
\end{equation*}
\end{constr}

Thus we see that type theory cannot distinguish between identified terms $x$ and $y$, because for any type family $B$ over $A$ one gets a term of $B(y)$ as soon as $B(x)$ has a term.

As an application of the transport function we construct the \emph{dependent} action on paths\index{dependent action on paths}\index{function!dependent action on paths} of a dependent function $f:\prd{x:A}B(x)$. Note that for such a dependent function $f$, and an identification $p:\id[A]{x}{y}$, it does not make sense to directly compare $f(x)$ and $f(y)$, since the type of $f(x)$ is $B(x)$ whereas the type of $f(y)$ is $B(y)$, which might not be exactly the same type. However, we can first \emph{transport} $f(x)$ along $p$, so that we obtain the term $\mathsf{tr}_B(p,f(x))$ which is of type $B(y)$. Now we can ask whether it is the case that $\mathsf{tr}_B(p,f(x))=f(y)$. The dependent action on paths of $f$ establishes this identification.

\begin{defn}\label{defn:apd}
Given a dependent function $f:\prd{a:A}B(a)$ and a path $p:\id{x}{y}$ in $A$, we construct a path\index{apd f@{$\apdfunc{f}$}|textbf}
\begin{equation*}
\apd{f}{p} : \id{\mathsf{tr}_B(p,f(x))}{f(y)}.
\end{equation*}
\end{defn}

\begin{constr}
The path $\apd{f}{p}$ is constructed by path induction on $p$. Thus, it suffices to construct a path
\begin{equation*}
\apd{f}{\refl{x}}:\id{\mathsf{tr}_B(\refl{x},f(x))}{f(x)}.
\end{equation*}
Since transporting along $\refl{x}$ is the identity function on $B(x)$, we simply take $\apd{f}{\refl{x}}\defeq\refl{f(x)}$. 
\end{constr}

%\begin{defn}\label{defn:path_lifting}
%Let $A$ be a type, and let $B:A\to\type$ be a type family over $A$.
%We will construct a \define{path lifting} operation
%\begin{equation*}
%\mathsf{lift}^B : \prd*{x,y:A}{p:\id{x}{y}}{b:B(x)} \id{\pairr{x,b}}{\pairr{y,\trans{p}{b}}}.
%\end{equation*}
%\end{defn}
%
%\cref{defn:path_lifting} gives a way to lift a path $p:x=y$ in the base type of a type family, to a path in the $\Sigma$-type. This, along with the basic groupoid operations developed in \cref{sec:groupoid}, inspired the \emph{homotopy interpretation} of type theory.

\section{Identity systems}

Many types come equipped with a reflexive relation that possesses a similar
structure as the identity type. The observational equality on the natural
numbers is such an example. We have see that it is a reflexive, symmetric, and
transitive relation, and moreover it is contained in any other reflexive
relation. As we will see in this section, the observational equality on the
natural numbers even satisfies a variant of path induction, so in type theory
we really can't distinguish between the identity type and observational
equality.

In this section we will introduce \emph{identity systems}. Those are reflexive
relations on a type that satisfy path induction. Identity systems are useful
to \emph{characterize} the identity type of a given type. Characterizing
identity types is one of the core activities in homotopy type theory, because
in order to fully understand a type one has to understand in what ways two
terms of this type can be identified.

\begin{defn}
  Let $B$ be a type family over $A$, and let $a:A$ and $b:B(a)$ be given. We say that $B$ is an \define{(unary) identity system} if for every family of types
  $P(x,y)$ indexed by $x:A$ and $y:B(x)$, there is a function map
  \begin{equation*}
    \mathsf{ind\usc{}id\usc{}sys}:P(a,b)\to \prd{x:A}{y:B(x)}P(x,y)
  \end{equation*}
  equipped with an identification
  \begin{equation*}
    \mathsf{ind\usc{}id\usc{}sys}(p,a,b)=p.
  \end{equation*}
  for every $p:P(a,b)$.
\end{defn}

\begin{rmk}
  Note that in the definition of identity systems we require an identification for the `computation rule', rather than a judgmental equality. 
\end{rmk}

\begin{thm}
  For any natural number $n$, the family $\mathsf{Eq}_\N(n,m)$ indexed by $m:\N$ is an identity system.
\end{thm}

\begin{proof}
  Our first goal is to construct a function
  \begin{equation*}
    \mathsf{ind\usc{}Eq}_\N:P(n,\mathsf{refl\usc{}Eq}_\N(n))\to\prd{x:\N}{e:\EqN(n,x)}P(x,e)
  \end{equation*}
  for any family of types $P(x,e)$ indexed by $x:\N$ and $e:\mathsf{Eq}_\N(n,x)$. The construction on $\mathsf{ind\usc{}Eq}_\N$ is by induction on $n:\N$ and $x:\N$. However, before we start the inductive argument we make sure to rearrange the order of the variables so that the inductive hypothesis will be strong enough: by induction on $n,m:\N$ we will construct for any
  \begin{align*}
    e & :\mathsf{Eq}_\N(n,m) \\
    P & :\prd{x:\N}\mathsf{Eq}_\N(n,x)\to \UU \\
    p & :P(n,\mathsf{refl\usc{}Eq}_\N(n)),
  \end{align*}
  a term $\mathsf{ind\usc{}Eq}_\N(p,m,e):P(m,e)$. There are four cases to consider:

  In the case where $n\jdeq 0$ and $m\jdeq 0$, the type $\mathsf{Eq}_\N(n,m)$ is the unit type. We proceed by induction on $e$, and we note that $\mathsf{refl\usc{}Eq}_\N(0)\jdeq\ttt$. Therefore we take as our definition
  \begin{equation*}
    \mathsf{ind\usc{}Eq}_\N(p,0,\ttt)\defeq p:P(0,\mathsf{refl\usc{}Eq}_\N(0)).
  \end{equation*}

  In the case where $m\jdeq 0$ and $n\jdeq \succN(n')$, the type $\mathsf{Eq}_\N(m,n)$ is empty, so we proceed by the induction principle of the empty type and we are done.

  Similarly in the case where $m\jdeq\succN(m')$ and $n\jdeq 0$, the type $\mathsf{Eq}_\N(m,n)$ is empty, so we proceed by the induction principle of the empty type and we are done.

  In the inductive steps for both $m$ and $n$ we use that
  \begin{equation*}
    \mathsf{Eq}_\N(\succN(m),\succN(n))\jdeq \mathsf{Eq}_\N(m,n).
  \end{equation*}
  The inductive hypothesis gives for every $e'$, $R'$, and $\rho'$ a term of type $R'(m,n,e')$. Therefore we simply apply the inductive hypothesis to the case where
  \begin{align*}
    R'(x,y,e) & \defeq R(\succN(x),\succN(y),e) \\
    \rho'(x) & \defeq \rho(\succN(x)),
  \end{align*}
  to obtain a term of type $R'(m,n,e)$. This completes the construction of $\mathsf{ind\usc{}Eq}_\N$.
  
  Finally, we have to construct an identification
  \begin{equation*}
    \mathsf{comp\usc{}Eq}_\N(p,n):\mathsf{ind\usc{}Eq}_\N(p,n,\mathsf{refl\usc{}Eq}_\N(n))= p
  \end{equation*}
  for any $p:P(n,\mathsf{refl\usc{}Eq}_\N(n))$. This is again by induction on $n:\N$.
\end{proof}

\begin{exercises}
\item
  \begin{subexenum}
  \item State Goldbach's conjecture in type theory.
  \item State the twin prime conjecture in type theory.
  \end{subexenum}
\item \label{ex:inv_assoc}Show that the operation inverting paths distributes over the concatenation operation, i.e., construct an identification
  \begin{align*}
    \mathsf{distributive\usc{}inv\usc{}concat}(p,q):\id{(\ct{p}{q})^{-1}}{\ct{q^{-1}}{p^{-1}}}.
  \end{align*}
  for any $p:\id{x}{y}$ and $q:\id{y}{z}$.
\item \label{ex:inv_con}For any $p:x=y$, $q:y=z$, and $r:x=z$, construct maps
  \begin{align*}
    \mathsf{inv\usc{}con}(p,q,r) & : (\ct{p}{q}=r)\to (q=\ct{p^{-1}}{r}) \\
    \mathsf{con\usc{}inv}(p,q,r) & : (\ct{p}{q}=r)\to (p=\ct{r}{q^{-1}}).
  \end{align*}
\item Let $B$ be a type family over $A$, and consider a path $p:\id{x}{x'}$ in $A$. Construct for any $y:B(x)$ a path
  \begin{equation*}
    \mathsf{lift}_B(p,y) : \id{\pairr{x,y}}{\pairr{x',\mathsf{tr}_B(p,y)}}.
  \end{equation*}
  In other words, a path in the \emph{base type} $A$ \emph{lifts} to a path in the total space $\sm{x:A}B(x)$ for every term over the domain, analogous to the path lifting property for fibrations in homotopy theory.
\item \label{ex:semi-ring-laws-N}Show that the operations of addition and multiplication on the natural numbers satisfy the following laws:
  \begin{align*}
    m+(n+k) & =(m+n)+k & m\cdot (n\cdot k) & = (m\cdot n)\cdot k \\
    m+0 & = m & m\cdot 1 & = m \\
    0+m & = m & 1\cdot m & = m \\
    m+n & = n+m & m\cdot n & = n\cdot m\\
    & & m\cdot (n+k) & = m\cdot n + m\cdot k.
  \end{align*}
\item Consider four consecutive identifications
  \begin{equation*}
    \begin{tikzcd}
      a \arrow[r,equals,"p"] & b \arrow[r,equals,"q"] & c \arrow[r,equals,"r"] & d \arrow[r,equals,"s"] & e
    \end{tikzcd}
  \end{equation*}
  in a type $A$. In this exercise we will show that the \define{Mac Lane pentagon}\index{Mac Lane pentagon} for identifications commutes.
  \begin{subexenum}
  \item Construct the five identifications $\alpha_1,\ldots,\alpha_5$ in the pentagon
    \begin{equation*}
      \begin{tikzcd}[column sep=-1.5em]
        &[-2em] \ct{(\ct{(\ct{p}{q})}{r})}{s} \arrow[rr,equals,"\alpha_4"] \arrow[dl,equals,swap,"\alpha_1"] & & \ct{(\ct{p}{q})}{(\ct{r}{s})} \arrow[dr,equals,"\alpha_5"] &[-2em] \\
        \ct{(\ct{p}{(\ct{q}{r})})}{s} \arrow[drr,equals,swap,"\alpha_2"] & & & & \ct{p}{(\ct{q}{(\ct{r}{s})})}, \\
        & & \ct{p}{(\ct{(\ct{q}{r})}{s})} \arrow[urr,equals,swap,"\alpha_3"]
      \end{tikzcd}
    \end{equation*}
    where $\alpha_1$, $\alpha_2$, and $\alpha_3$ run counter-clockwise, and $\alpha_4$ and $\alpha_5$ run clockwise.
  \item Show that
    \begin{equation*}
      \ct{(\ct{\alpha_1}{\alpha_2})}{\alpha_3} = \ct{\alpha_4}{\alpha_5}.
    \end{equation*}
  \end{subexenum}
\item Show that observational equality on the booleans is an identity system.
\end{exercises}

%\item In this exercise we show that the action on paths of a function preserves the groupoid-structure of a type.
%\begin{subexenum}
%\item Construct an identification
%\begin{equation*}
%\mathsf{ap.assoc}(f,p,q,r)
%\end{equation*}
%witnessing that the diagram
%\begin{equation*}
%\begin{tikzcd}[column sep=large]
%\ap{f}{\ct{(\ct{p}{q})}{r}} \arrow[r,equals,"\ap{\apfunc{f}}{\mathsf{assoc}(p,q,r)}"] \arrow[d,swap,equals,"{\mathsf{ap.ct}(f,%\ct{p}{q},r)}"] & \ap{f}{\ct{p}{(\ct{q}{r})}} \arrow[d,equals,"{\mathsf{ap.ct}(f,p,\ct{q}{r})}"] \\ 
%\ct{\ap{f}{\ct{p}{q}}}{\ap{f}{r}} \arrow[dd,equals,near start,"{\mathsf{whisk\usc{}r}(\mathsf{ap.ct}(f,p,q),\ap{f}{r})}"]   & %\ct{\ap{f}{p}}{\ap{f}{\ct{q}{r}}} \arrow[dd,equals,swap,near end,"{\mathsf{whisk\usc{}l}(\ap{f}{p},\mathsf{ap.ct}(f,q,r))}"]  %\\
%\\
%\ct{(\ct{\ap{f}{p}}{\ap{f}{q}})}{\ap{f}{r}} \arrow[r,equals,swap,"{\mathsf{assoc}(\ap{f}{p},\ap{f}{q},\ap{f}{r})}"yshift=-1em] & \ct{\ap{f}{p}}{(\ct{\ap{f}{q}}{\ap{f}{r}})}
%\end{tikzcd}
%\end{equation*}
%commutes.
%\end{subexenum}

\begin{comment}
\item \label{ex:trans_triv}Consider two types $A$ and $B$, and let $p:x=y$ in $A$, and $b:B$. 
  \begin{subexenum}
  \item Construct an identification
    \begin{align*}
      \mathsf{tr\usc{}triv}(p,b):\mathsf{tr}_{W_A(B)}(p,b)=b
    \end{align*}
    where $W_A(B)$ is the family $B$ weakened by $A$.
  \item Construct for any $f:A\to B$, an identification 
    \begin{equation*}
      \apd{f}{p}=\ct{\mathsf{tr\usc{}triv}(p,f(x))}{\mathsf{ap}_f(p)},
    \end{equation*}
    witnessing that the triangle
    \begin{equation*}
      \begin{tikzcd}[trim right=(a),column sep=0em]
        \mathsf{tr}_{W_A(B)}(p,f(x)) \arrow[dr,equals,swap,"\apd{f}{p}"] \arrow[rr,equals,"{\mathsf{tr\usc{}triv}(p,f(x))}"] & & |[alias=a,right]|f(x) \arrow[dl,equals,"\ap{f}{p}"] \\
        & f(y) & \phantom{\mathsf{tr}_{W_A(B)}(p,f(x))}
      \end{tikzcd}
    \end{equation*}
    commutes.
  \end{subexenum}
\item \label{ex:trans_ap}Let $f:A\to B$ be a map, and consider $p:x=y$ in $A$. 
  \begin{subexenum}
  \item Construct for any $q:f(x)=b$ in $B$ an identification
    \begin{equation*}
      \mathsf{tr\usc{}id\usc{}left\usc{}subst}(p,q):\id{\mathsf{tr}_{f(\blank)=b}(p,q)}{\ct{\ap{f}{p}^{-1}}{q}}.
    \end{equation*}
  \item Similarly, construct for any $q':b=f(x)$ in $B$ an identification
    \begin{equation*}
      \mathsf{tr\usc{}id\usc{}right\usc{}subst}(p,q'):\id{\mathsf{tr}_{b=f(\blank)}(p,q)}{\ct{q}{\ap{f}{p}}}.
    \end{equation*}
  \end{subexenum}
\end{comment}

\section{The universe and type-valued relations}

In this lecture we introduce type theoretic \emph{universes}. Universes are types that consist of types. In other words, a universe is a type $\UU$ that comes equipped with a type family $\mathsf{Ty}$ over $\UU$, and for any $X:\UU$ we think of $X$ as an \emph{encoding} of the type $\mathsf{Ty}(X)$. We call this type family the \emph{universal type family}.

There are several reasons to equip type theory with universes. One reason is that it enables us to define new type families over inductive types, using their induction principle. For example, since the universe is itself a type, we can use the induction principle of $\bool$ to obtain a map $P:\bool\to\UU$ from any two terms $X_0,X_1:\UU$. Then we obtain a type family over $\bool$ by substituting $P$ into the universal type family:
\begin{equation*}
  x:\bool\vdash \mathsf{Ty}(P(x))~\mathrm{type}
\end{equation*}
satisfying $\mathsf{Ty}(P(0_\bool))\jdeq \mathsf{Ty}(X_0)$ and $\mathsf{Ty}(P(1_\bool))\jdeq \mathsf{Ty}(X_1)$.

We use this way of defining type families to define many familiar relations over $\N$, such as $\leq$ and $<$. We also introduce a relation called \emph{observational equality} $\mathsf{Eq}_\N$ on $\N$, which we can think of as equality of $\N$. This relation is reflexive, symmetric, and transitive, and moreover it is the least reflexive relation. Furthermore, one of the most important aspects of observational equality $\mathsf{Eq}_\N$ on $\N$ is that $\mathsf{Eq}_\N(m,n)$ is a type for every $m,n:\N$, unlike judgmental equality. Therefore we can use type theory to reason about observational equality on $\N$. Indeed, in the exercises we show that some very elementary mathematics can already be done at this early stage in our development of type theory.

A second reason to introduce universes is that it allows us to define many types of types equipped with structure. One of the most important examples is the type of groups, which is the type of types equipped with the group operations satisfying the group laws, and for which the underlying type is a set. We won't discuss the condition for a type to be a set until \cref{chap:hierarchy}, so the definition of groups in type theory will be given much later. Therefore we illustrate this use of the universe by giving simpler examples: pointed types, graphs, and reflexive graphs.

One of the aspects that make universes useful is that they are postulated to be closed under all the type constructors. For example, if we are given $X:\UU$ and $P:\mathsf{Ty}(X)\to \UU$, then the universe is equipped with a term
\begin{equation*}
  \check{\Sigma}(X,P):\UU
\end{equation*}
satisfying the judgmental equality $\mathsf{Ty}(\check{\Sigma}(X,P)\jdeq\sm{x:\mathsf{Ty}(X)}\mathsf{Ty}(P(x))$. We will similarly assume that any universe is closed under $\Pi$-types and the other ways of forming types. However, there is an important restriction: it would be inconsistent to assume that the universe is contained in itself. One way of thinking about this is that universes are types of \emph{small} types, and it cannot be the case that the universe is small with respect to itself. We address this problem by assuming that there are many universes: enough universes so that any type family can be obtained by substituting into the universal type family of some universe.

\subsection{Type theoretic universes}
The induction principle for inductive types can be used to prove universal quantifications. 
However, it would also be nice if we could construct \emph{new type families} over inductive types, using their induction principles.
To be able to do this, we introduce a \emph{universe}, a type of which the terms represent types. The idea is that the universe $\UU$ comes equipped with a type family $\mathsf{Ty}$, so that for each $X:\UU$ we have an associated type $\mathsf{Ty}(X)$, the type of \emph{elements} of $X$. 

We assume there is a closed type $\UU$\index{U@{$\UU$}|textbf} called the \define{universe}\index{universe|textbf}, and a type family $\mathsf{Ty}$\index{El@{$\mathsf{Ty}$}} over $\UU$ called the \define{universal family}\index{universal family}\index{family!universal family|textbf}.
\begin{center}
\begin{minipage}{.4\textwidth}
\begin{prooftree}
\AxiomC{}
\UnaryInfC{$\vdash\UU~\mathrm{type}$}
\end{prooftree}
\end{minipage}\quad
\begin{minipage}{.4\textwidth}
\begin{prooftree}
\AxiomC{}
\UnaryInfC{$X:\UU \vdash \mathsf{Ty}(X)~\mathrm{type}$}
\end{prooftree}
\end{minipage}
\end{center}

We postulate that the universe is closed under the type constructors, by the following rules:
\begin{enumerate}
\item The universe is closed under $\Pi$-types
\begin{prooftree}
\AxiomC{$\Gamma\vdash A:\UU$}
\AxiomC{$\Gamma\vdash B:\mathsf{Ty}(A)\to\UU$}
\BinaryInfC{$\Gamma \vdash \check{\Pi}(A,B):\UU$}
\end{prooftree}
\begin{prooftree}
\AxiomC{$\Gamma\vdash A:\UU$}
\AxiomC{$\Gamma\vdash B:\mathsf{Ty}(A)\to\UU$}
\BinaryInfC{$\Gamma \vdash \mathsf{Ty}(\check{\Pi}(A,B))\jdeq \prd{x:\mathsf{Ty}(A)}\mathsf{Ty}(B(x))~\mathrm{type}$}
\end{prooftree}
\item The type of natural numbers is in the universe
\begin{prooftree}
\AxiomC{}
\UnaryInfC{$\vdash \check{\N}:\UU$}
\end{prooftree}
\begin{prooftree}
\AxiomC{}
\UnaryInfC{$\vdash \mathsf{Ty}(\check{\N})\jdeq \mathbb{N}~\mathrm{type}$}
\end{prooftree}
\item Similarly we postulate that the universe contains the empty type, the unit type, the booleans, coproducts, products, and $\Sigma$-types. These closure properties of the universe are given concisely in \cref{tab:universe}.
\begin{table}
\begin{center}
\caption{\label{tab:universe}Closure properties of the universe}
\begin{tabular}{lll}
\toprule
Premises & Type encoding\index{type encoding@{type encoding $\check{A}$}} in $\UU$ & Type of elements $\mathsf{Ty}(\blank)$ \\
\midrule
$A:\UU,B:\mathsf{Ty}(A)\to\UU$ & $\check{\Pi}(A,B)$ & $\prd{x:\mathsf{Ty}(A)}\mathsf{Ty}(B(x))$ \\
& $\check{\nat}$ & $\nat$ \\
& $\check{\emptyt}$ &  $\emptyt$ \\
& $\check{\unit}$ &  $\unit$ \\
& $\check{\bool}$ &  $\bool$ \\
$A,B:\UU$ & $A\mathbin{\check{+}}B$ &  $\mathsf{Ty}(A)+\mathsf{Ty}(B)$ \\
$A,B:\UU$ & $A\mathbin{\check{\times}}B$ &  $\mathsf{Ty}(A)\times\mathsf{Ty}(B)$ \\
$A:\UU,B:\mathsf{Ty}(A)\to\UU$ & $\check{\Sigma}(A,B)$ & $\sm{x:\mathsf{Ty}(A)}\mathsf{Ty}(B(x))$ \\
\bottomrule
\end{tabular}
\end{center}
\end{table}
\end{enumerate}

\begin{defn}
We say that a type $A$ in context $\Gamma$ is \define{small}\index{small type|textbf} if it occurs in the universe, i.e., if there is a term $\check{A}:\UU$ in context $\Gamma$ such that $\Gamma\vdash\mathsf{Ty}(\check{A})\jdeq A~\mathrm{type}$.
\end{defn}

In particular, if $A$ is a small type in context $\Gamma$ and $B$ is a small type in context $\Gamma,x:A$, then $\prd{x:A}B(x)$ is again a small type in context $\Gamma$.

\begin{defn}
Let $A$ be a type in context $\Gamma$. A \define{family of small types}\index{family of small types|textbf} over $A$ is defined to be a map
\begin{equation*}
B:A\to\UU
\end{equation*}
\end{defn}

\begin{rmk}
If $A$ is small, we usually write simply $A$ for $\check{A}$ and also $A$ for $\mathsf{Ty}(\check{A})$. In other words, by $A:\UU$ we mean that $A$ is a small type. 
\end{rmk}

\begin{eg}
One important way to use the universe is to define types of \define{structured types}\index{structured types}. We give some examples:
\begin{enumerate}
\item The type of small \define{pointed types}\index{pointed types} is defined as
\begin{equation*}
\UU_\ast\defeq \sm{A:\UU}A,
\end{equation*}
\item The type of small \define{graphs}\index{graphs} is defined as the type
\begin{equation*}
\mathsf{Gph}_\UU \defeq \sm{A:\UU} A\to (A\to \UU),
\end{equation*}
\item The type of small \define{reflexive graphs}\index{reflexive graphs} is defined as the type
\begin{equation*}
\mathsf{rGph}_\UU \defeq \sm{A:\UU}{R:A\to (A\to \UU)}\prd{a:A}R(a,a).
\end{equation*}
\end{enumerate}
Once we have introduced the \emph{identity types} we will also be able to state the types of groups, rings, and many other structured types. However, when doing so one has to be cautious to make sure that the underlying type is in the level of sets, in the hierarchy of homotopical complexity of types.
\end{eg}

\subsection{Defining families and relations using a universe}
Another important way to use the universe is to \emph{define} new type families by induction. For example, we can define the finite types as family over the natural numbers.

\begin{defn}\label{defn:fin}
We define the type family $\mathsf{Fin}:\N\to\UU$ of finite types\index{Fin@{$\mathsf{Fin}$}|textbf}\index{finite types|textbf} by induction on $\N$\index{family!of finite types}, taking
\begin{align*}
\mathsf{Fin}(0) & \defeq \emptyt \\
\mathsf{Fin}(n+1) & \defeq \mathsf{Fin}(n)+\unit
\end{align*}
\end{defn}

A second example of this kind is the notion of \emph{observational equality} on the natural numbers.

\begin{defn}\label{defn:obs_nat}
We define the \define{observational equality}\index{observational equality!on N@{on $\N$}} on $\N$ as binary relation $\mathsf{Eq}_\N:\N\to(\N\to\UU)$\index{Eq_N@{$\mathsf{Eq}_\N$}|textbf} satisfying
\begin{align*}
\mathsf{Eq}_\N(0,0) & \jdeq \unit & \mathsf{Eq}_\N(S(n),0) & \jdeq \emptyt \\
\mathsf{Eq}_\N(0,S(n)) & \jdeq \emptyt & \mathsf{Eq}_\N(S(n),S(m)) & \jdeq \mathsf{Eq}_\N(n,m).
\end{align*}
\end{defn}

\begin{constr}
We define $\mathsf{Eq}_\N$ by double induction on $\N$. By the first application of induction it suffices to provide
\begin{align*}
E_0 & : \N\to\UU \\
E_S & : \N\to (\N\to\UU)\to(\N\to\UU)
\end{align*}
We define $E_0$ by induction, taking $E_{00}\defeq \unit$ and $E_{0S}(n,X,m)\defeq \emptyt$. The resulting family $E_0$ satisfies
\begin{align*}
E_0(0) & \jdeq \unit \\
E_0(S(n)) & \jdeq \emptyt.
\end{align*} 
We define $E_S$ by induction, taking $E_{S0}\defeq \emptyt$ and $E_{S0}(n,X,m)\defeq X(m)$. The resulting family $E_S$ satisfies
\begin{align*}
E_S(n,X,0) & \jdeq \emptyt \\
E_S(n,X,S(m)) & \jdeq X(m) 
\end{align*}
Therefore we have by the computation rule for the first induction that the judgmental equality
\begin{align*}
\mathsf{Eq}_\N(0,m) & \jdeq E_0(m) \\
\mathsf{Eq}_\N(S(n),m) & \jdeq E_S(n,\mathsf{Eq}_\N(n),m)
\end{align*}
holds, from which the judgmental equalities in the statement of the definition follow.
\end{constr}

We can also define observational equality for many other kinds of types, such as $\bool$ or $\Z$. In each of these cases, what sets the observational equality apart from other relations is that it is the \emph{least} reflexive relation. However, the definitions of observational equality in the various situations are always very specific to the type they are defined on. In the next chapter we introduce the \emph{identity type}, which is an equality type defined uniformly for all types. In \cref{chap:hierarchy} we will show that the observational equality on $\N$ is equivalent to the identity type on $\N$.

\subsection{Negations and decidability}
Using the empty type we can define the \emph{negation} of a type. The idea is that if $A$ is false (i.e., has no terms), then from $A$ follows everything.

\begin{defn}
For any type $A$, we define $\neg A\defeq A\to \emptyt$.\index{negation!of a type}\index{not ($\neg$)|see {negation, of a type}}
\end{defn}

Note that $\neg A$ is the type of functions from $A$ to $\emptyt$. Therefore one can construct a term of type $\neg A$ by constructing a term $f(x):\emptyt$ using $x:A$. In other words, to construct a term of type $\neg A$ one assumes $A$ and derives a contradiction. This proof technique is called \define{proof of negation}.

Proof of negation is not to be confused with \emph{proof by contradiction}. In type theory there is no way of obtaining a term of type $A$ from a term of type $(A\to \emptyt)\to\emptyt$. Even stronger: we will use the univalence axiom to obtain a term of type
\begin{equation*}
  \neg\Big(\prd{A:\UU}((A\to\emptyt)\to\emptyt)\to A\Big).
\end{equation*}
In other words, it would be \emph{inconsistent} to admit proofs by contradiction as a valid way of constructing terms of general types. Nevertheless it turns out that a restricted form of proof by contradiction is still consistent with the univalence axiom.

\begin{exercises}
\item Let $A$ be a type.
  \begin{subexenum}
  \item Show that $(A+\neg A)\to(\neg\neg A\to A)$.
  \item Show that $\neg\neg\neg A \to \neg A$.
  \end{subexenum}
\item Construct a function
  \begin{equation*}
    \check{\Pi}:\prd{A:\UU} (\mathsf{Ty}(A)\to\UU)\to \UU
  \end{equation*}
  such that
  \begin{equation*}
    \mathsf{Ty}(\check{\Pi}(A,B))\jdeq \prd{x:\mathsf{Ty}(A)}\mathsf{Ty}(B(x))
  \end{equation*}
  holds for every $A:\UU$ and $B:\mathsf{Ty}(A)\to\UU$. 
  
  \emph{A similar exercise can be posed for $\Sigma$ and $+$ (and for $\to$ and $\times$ as special cases of $\Pi$ and $\Sigma$).}
\item \label{ex:obs_nat_eqrel}Show that observational equality on $\N$\index{observational equality!on N@{on $\N$}!is an equivalence relation} is an equivalence relation\index{equivalence relation!observational equality on N@{observational equality on $\N$}}, i.e., construct terms of the following types:
  \begin{align*}
    & \prd{n:\N} \EqN(n,n) \\
    & \prd{n,m:\N} \EqN(n,m)\to \EqN(m,n) \\
    & \prd{n,m,l:\N} \EqN(n,m)\to (\EqN(m,l)\to \EqN(n,l)).
  \end{align*}
\item \label{ex:obs_nat_least}\index{observational equality!on N@{on $\N$}!is least reflexive relation}Let $R$ be a reflexive binary relation\index{reflexive relation}\index{relation!reflexive} on $\N$, i.e., $R$ is of type $\N\to (\N\to\UU)$ and comes equipped with a term $\rho:\prd{n:\N}R(n,n)$. Show that
  \begin{equation*}
    \prd{n,m:\N} \EqN(n,m)\to R(n,m).
  \end{equation*}
\item \index{observational equality!on N@{on $\N$}!is preserved by functions}Show that every function $f:\N\to \N$ preserves observational equality in the sense that
  \begin{equation*}
    \prd{n,m:\N} \EqN(n,m)\to \EqN(f(n),f(m)).
  \end{equation*}
  \emph{Hint: to get the inductive step going the induction hypothesis has to be strong enough. Construct by double induction a term of type}
  \begin{equation*}
    \prd{n,m:\N}{f:\N\to\N} \EqN(n,m)\to \EqN(f(n),f(m)),
  \end{equation*}
  \emph{and pull out the universal quantification over $f:\N\to\N$ by \cref{ex:swap}.}
\item 
  \begin{subexenum}
  \item Define the \define{order relations}\index{relation!order}\index{order relation} $\leq$ and $<$ on $\N$.
  \item Show that $\leq$ is reflexive and that $<$ is \define{anti-reflexive}\index{anti-reflexive}\index{relation!anti-reflexive}, i.e., that $\neg(n<n)$. 
  \item Show that both $\leq$ and $<$ are transitive, and that $n<S(n)$.
  \item Show that $k\leq \min(m,n)$ holds if and only if both $k\leq m$ and $k\leq n$ hold, and show that $\max(m,n)\leq k$ holds if and only if both $m\leq k$ and $n\leq k$ hold.
  \end{subexenum}
\item \label{ex:obs_bool}\index{observational equality!on 2@{on $\bool$}}
  \begin{subexenum}
  \item Define observational equality $\mathsf{Eq}_\bool$\index{Eq_bool@{$\mathsf{Eq}_\bool$}|textbf} on the booleans.
  \item Show that $\mathsf{Eq}_\bool$ is reflexive.\index{observational equality!on 2@{on $\bool$}!is reflexive}
  \item Show that for any reflexive relation $R:\bool\to(\bool\to \UU)$ one has\index{observational equality!on 2@{on $\bool$}!is least reflexive relation}
    \begin{equation*}
      \prd{x,y:\bool} \mathsf{Eq}_\bool(x,y)\to R(x,y).
    \end{equation*}
  \end{subexenum}
\item \label{ex:int_order}
  \begin{subexenum}
  \item Define the order relations\index{relation!order}\index{order relation} $\leq$ and $<$ on and $\Z$.
  \item For $k:\Z$, consider the type $\Z_{\geq k}\defeq \sm{n:\Z}n\geq k$. Construct
    \begin{align*}
      b_k & : \Z_{\geq k} \\
      s_k & : \Z_{\geq k}\to\Z_{\geq k},
    \end{align*}
    and show that $\Z_{\geq k}$ satisfies the induction principle of the natural numbers\index{induction principle!of N@{of $\N$}}:
    \begin{equation*}
      \ind{\Z_{\geq k}} : P(b_k)\to \Big(\prd{n:\Z_{\geq k}} P(n)\to P(s_k(n))\Big)\to \Big(\prd{n:\Z_{\geq k}} P(n)\Big)
    \end{equation*}
  \end{subexenum}
\item
  \begin{subexenum}
  \item Show that $\N$ satisfies \define{strong induction}, i.e., construct for any type family $P$ over $\N$ a function of type
    \begin{equation*}
      P(\zeroN) \to \Big(\prd{k:\N}\Big(\prd{m:\N} (m\leq k) \to P(m)\Big)\to P(\succN(k))\Big) \to \prd{n:\N}P(n).
    \end{equation*}
  \item Show that $\N$ satisfies \define{ordinal induction}, i.e., construct for any type family $P$ over $\N$ a function of type
    \begin{equation*}
      \Big(\prd{k:\N} \Big(\prd{m:\N} (m< k) \to P(m)\Big)\to P(k)\Big) \to \prd{n:\N}P(n).
    \end{equation*}
  \end{subexenum}
\end{exercises}


\chapter{Basic concepts of type theory}
\chapter{Equivalences}

\section{Homotopies}
In homotopy type theory, a homotopy is just a pointwise equality between two functions $f$ and $g$.

\begin{defn}
Let $f,g:\prd{x:A}P(x)$ be two dependent functions. The type of \define{homotopies}\index{homotopy|textbf} from $f$ to $g$ is defined as
\begin{equation*}
f\htpy g \defeq \prd{x:A} f(x)=g(x).
\end{equation*}
\end{defn}

Since we formulated homotopies using dependent functions, we may also consider homotopies \emph{between}\index{homotopy!iterated} homotopies, and further homotopies between those higher homotopies. 
Explicitly, if $H,K:f\htpy g$, then the type $H\htpy K$ of homotopies is just the type
\begin{equation*}
\prd{x:A} H(x)=K(x).
\end{equation*}

In the following definition we define the groupoid-like structure of homotopies. Note that we implement the groupoid-laws as \emph{homotopies} rather than as identifications.

\begin{defn}\index{groupoid laws!of homotopies|textbf}
For any dependent type $B:A\to\type$ there are operations
\begin{align*}
& \mathsf{htpy.refl} & & : \prd{f:\prd{x:A}B(x)}f\htpy f \\
& \mathsf{htpy.inv} & & : \prd*{f,g:\prd{x:A}B(x)} (f\htpy g)\to(g\htpy f) \\
& \mathsf{htpy.concat} & & : \prd*{f,g,h:\prd{x:A}B(x)} (f\htpy g)\to (g\htpy h)\to (f\htpy h).
\end{align*}
We will write $H^{-1}$ for $\mathsf{htpy.inv}(H)$, and $\ct{H}{K}$ for $\mathsf{htpy.concat}(H,K)$. 

Furthermore, we define
\begin{align*}
& \mathsf{htpy.assoc}(H,K,L) & & : \ct{(\ct{H}{K})}{L}\htpy\ct{H}{(\ct{K}{L})} \\
& \mathsf{htpy.left\usc{}unit}(H) & & : \ct{\mathsf{htpy.refl}_f}{H}\htpy H \\
& \mathsf{htpy.right\usc{}unit}(H) & & : \ct{H}{\mathsf{htpy.refl}_g}\htpy H \\
& \mathsf{htpy.left\usc{}inv}(H) & & : \ct{H^{-1}}{H} \htpy \mathsf{htpy.refl}_g \\
& \mathsf{htpy.right\usc{}inv}(H) & & : \ct{H}{H^{-1}} \htpy \mathsf{htpy.refl}_f
\end{align*}
for any $H:f\htpy g$, $K:g\htpy h$ and $L:h\htpy i$, where $f,g,h,i:\prd{x:A}B(x)$.
\end{defn}

\begin{constr}
We define
\begin{align*}
\mathsf{htpy.refl}(f) & \defeq \lam{x} \refl{f(x)} \\
\mathsf{htpy.inv}(H) & \defeq \lam{x} H(x)^{-1} \\
\mathsf{htpy.concat}(H,K) & \defeq \lam{x}\ct{H(x)}{K(x)},
\end{align*}
where $H:f\htpy g$ and $K:g\htpy h$ are homotopies. Furthermore, we define
\begin{align*}
\mathsf{htpy.assoc}(H,K,L) & \defeq \lam{x}\mathsf{assoc}(H(x),K(x),L(x)) \\
\mathsf{htpy.left\usc{}unit}(H) & \defeq \lam{x}\mathsf{left\usc{}unit}(H(x)) \\
\mathsf{htpy.right\usc{}unit}(H) & \defeq \lam{x}\mathsf{right\usc{}unit}(H(x)) \\
\mathsf{htpy.left\usc{}inv}(H) & \defeq \lam{x}\mathsf{left\usc{}inv}(H(x)) \\
\mathsf{htpy.right\usc{}inv}(H) & \defeq \lam{x}\mathsf{right\usc{}inv}(H(x)).\qedhere
\end{align*}
\end{constr}


Apart from the groupoid operations and their laws, we will occasionally need \emph{whiskering} operations.

\begin{defn}
We define the following \define{whiskering}\index{homotopy!whiskering operations|textbf}\index{whiskering operations!of homotopies|textbf} operations on homotopies:
\begin{enumerate}
\item Suppose $H:f\htpy g$ for two functions $f,g:A\to B$, and let $h:B\to C$. We define
\begin{equation*}
hH\defeq \lam{x}\ap{h}{H(x)}:h\circ f\htpy h\circ g.
\end{equation*}
\item Suppose $f:A\to B$ and $H:g\htpy h$ for two functions $g,h:B\to C$. We define
\begin{equation*}
Hf\defeq\lam{x}H(f(x)):h\circ f\htpy g\circ f.
\end{equation*}
\end{enumerate}
\end{defn}

\section{Bi-invertible maps}
\begin{defn}
Let $f:A\to B$ be a function. We say that $f$ has a \define{section}\index{section!of a map|textbf} if there is a term of type
\begin{equation*}
\mathsf{sec}(f) \defeq \sm{g:B\to A} f\circ g\htpy \idfunc[B].
\end{equation*}
Dually, we say that $f$ has a \define{retraction}\index{retraction} if there is a term of type
\begin{equation*}
\mathsf{retr}(f) \defeq \sm{h:B\to A} h\circ f\htpy \idfunc[A].
\end{equation*}
If $f$ has a retraction, we also say that $A$ is a \define{retract}\index{retract!of a type} of $B$.
We say that a function $f:A\to B$ is an \define{equivalence}\index{equivalence|textbf}\index{bi-invertible map|see {equivalence}} if it has both a section and a retraction, i.e.~if it comes equipped with a term of type
\begin{equation*}
\isequiv(f)\defeq\mathsf{sec}(f)\times\mathsf{retr}(f).
\end{equation*}
We will write $\eqv{A}{B}$ for the type $\sm{f:A\to B}\isequiv(f)$.
\end{defn}

\begin{rmk}
An equivalence, as we defined it here, can be thought of as a \define{bi-invertible} map, since it comes equipped with a separate left and right inverse. Explicitly, if $f$ is an equivalence, then there are
\begin{align*}
g & : B\to A & h & : B\to A \\
G & : f\circ g \htpy \idfunc[B] & H & : h\circ f \htpy \idfunc[A].
\end{align*}
Clearly, if $f$ is \define{invertible}\index{invertible map} in the sense that it comes equipped with a function $g:B\to A$ such that $f\circ g\htpy\idfunc[B]$ and $g\circ f\htpy\idfunc[A]$, then $f$ is an equivalence.
\end{rmk}

\begin{defn}\label{defn:inv_equiv}
Any equivalence can be given the structure of an invertible map.\index{equivalence!invertibility of}
\end{defn}

\begin{constr}
First we construct for any equivalence $f$ with right inverse $g$ and left inverse $h$ a homotopy $K:g\htpy h$. For any $y:B$, we have 
\begin{equation*}
\begin{tikzcd}[column sep=huge]
g(y) \arrow[r,equals,"H(g(y))^{-1}"] & hfg(y) \arrow[r,equals,"\ap{h}{G(y)}"] & h(y).
\end{tikzcd}
\end{equation*} 
Therefore we define $K\defeq \ct{(Hg)^{-1}}{hG}$.
from which we obtain a homotopy $K:g\htpy h$.
This allows us to show that $g$ is also a left inverse of $f$. For $x:A$ we have the identification
\begin{equation*}
\begin{tikzcd}[column sep=large]
gf(x) \arrow[r,equals,"K(f(x))"] & hf(x) \arrow[r,equals,"H(x)"] & x.
\end{tikzcd}\qedhere
\end{equation*}
\end{constr}

\begin{thm}\label{thm:id_equiv}
For any type $A$, the identity function $\idfunc[A]$ is an equivalence.\index{identity function!is an equivalence|textit}
\end{thm}

\begin{proof}
The identity function is trivially its own section and its own retraction.
\end{proof}

\begin{eg}
Let $A$ and $B$ be types in context $\Gamma$. 
For any $\Gamma,x:A,y:B\vdash C(x,y)~\mathrm{type}$, the map
\begin{equation*}
\Big(\prd{x:A}{y:B}C(x,y)\Big)\to\Big(\prd{y:B}{x:A}C(x,y)\Big)
\end{equation*}
is an equivalence by \cref{ex:swap}.\index{swap function!is an equivalence|textit}
\end{eg}

\section{The identity type of a \texorpdfstring{$\Sigma$-}{dependent pair }type}
\begin{thm}\label{thm:eq_sigma}
Let $B$ be a type family over $A$, and let $\pairr{x,y},\pairr{x',y'}:\sm{x:A}B(x)$. Then the map
\begin{equation*}
\mathsf{eq\usc{}pair} : \Big(\sm{p:x=x'}\id{\mathsf{tr}_B(p,y)}{y'}\Big)\to(\id{\pairr{x,y}}{\pairr{x',y'}})
\end{equation*}
defined by double path induction by sending $\pairr{\refl{x},\refl{y}}$ to $\refl{\pairr{x,y}}$ is an equivalence.\index{Sigma type@{$\Sigma$-type}!identity types of|textit}
\end{thm}

\begin{proof}
Before we construct a map in the converse direction, we construct a slightly more general map
\begin{equation*}
\mathsf{pair\usc{}eq}:(z=z')\to \sm{p:\proj 1(z)=\proj 1(z')} \mathsf{tr}(p,\proj 2(z))=\proj 2(z').
\end{equation*}
for every $z,z':\sm{x:A}B(x)$. This map can be constructed by path induction, sending $\refl{z}$ to $(\refl{\proj 1(z)},\refl{\proj 2(z)})$. 

We first construct an identification $\mathsf{pair\usc{}eq}(\mathsf{eq\usc{}pair}(p,q))=\pairr{p,q}$ for each $\pairr{p,q}:\sm{p:x=x'}\id{\mathsf{tr}_B(p,y)}{y'}$. We proceed by path induction on $p$, followed by path induction on $q$. Our goal is now to construct a term of type
\begin{equation*}
\mathsf{pair\usc{}eq}(\mathsf{eq\usc{}pair}\pairr{\refl{x},\refl{y}})=\pairr{\refl{x},\refl{y}}
\end{equation*}
By the definition of $\mathsf{eq\usc{}pair}$ we have $\mathsf{eq\usc{}pair}\pairr{\refl{x},\refl{y}}\jdeq \refl{\pairr{x,y}}$, and by the definition of $\mathsf{pair\usc{}eq}$ we have $\mathsf{pair\usc{}eq}(\refl{\pairr{x,y}})\jdeq\pairr{\refl{x},\refl{y}}$. Thus we may take $\refl{\pairr{\refl{x},\refl{y}}}$ to complete the construction of the homotopy $\mathsf{pair\usc{}eq}\circ\mathsf{eq\usc{}pair}\htpy\idfunc$.

Next show that $\mathsf{eq\usc{}pair}(\mathsf{pair\usc{}eq}(p))=p$ for each $p:\pairr{x,y}=\pairr{x',y'}$.
To do this, note that by $\Sigma$-induction, the map $\mathsf{eq\usc{}pair}$ induces a map
\begin{equation*}
\mathsf{eq\usc{}pair} : \Big(\sm{p:\proj 1(z)=\proj 1 (z')}\id{\mathsf{tr}_B(p,\proj 2(z))}{\proj 2(z')}\Big)\to(\id{z}{z'})
\end{equation*}
Now we can use identification elimination to show that $\mathsf{eq\usc{}pair}(\mathsf{pair\usc{}eq}(p))=p$ for any $p:z=z'$.
It suffices to construct an identification 
\begin{equation*}
\mathsf{eq\usc{}pair}\pairr{\refl{\proj 1(z)},\refl{\proj 2(z)}}=\refl{z}.
\end{equation*}
Now we proceed by $\Sigma$-induction on $z:\sm{x:A}B(x)$, so it suffices to construct an identification
\begin{equation*}
\mathsf{eq\usc{}pair}\pairr{\refl{x},\refl{y}}=\refl{(x,y)}.
\end{equation*}
Since $\mathsf{eq\usc{}pair}\pairr{\refl{x},\refl{y}}$ computes to $\refl{(x,y)}$, we may simply take $\refl{\refl{(x,y)}}$.
\end{proof}

\begin{exercises}
\item \label{ex:equiv_grpd_ops}Show that $\mathsf{inv}:(\id{x}{y})\to(\id{y}{x})$, $\mathsf{concat}(p):(\id{y}{z})\to(\id{x}{z})$, and $\mathsf{tr}_B(p):B(x)\to B(y)$ are equivalences. What are their inverses?
\item \label{ex:htpy_equiv}\index{equivalence!homotopic maps} Consider two functions $f,g:A\to B$ and a homotopy $H:f\htpy g$. Then
\begin{equation*}
\isequiv(f)\leftrightarrow\isequiv(g).
\end{equation*}
\item \label{ex:3_for_2}\index{equivalence!three@{3-for-2 property}} (The 3-for-2 property) Let $f:A\to B$ and $g:B\to C$ be functions. Show that if any two of the functions
\begin{equation*}
f,\qquad g,\qquad g\circ f
\end{equation*}
are equivalences, then so is the third.
\item \label{ex:neg_equiv} Show that the negation function on the booleans is an equivalence. Also show that for any function $f:\bool\to\bool$, if $f(\bfalse)=f(\btrue)$ then $f$ is \emph{not} an equivalence.\index{negation function!is an equivalence}
\item \label{ex:succ_equiv} Show that the successor function on the integers is an equivalence.\index{successor function!on Z@{on $\Z$}!is an equivalence}
\item Construct a equivalences $\eqv{A+B}{B+A}$ and $\eqv{A\times B}{B\times A}$.\index{coproduct!is symmetric}
\item \label{ex:retr_id} Consider a section-retraction pair
\begin{equation*}
\begin{tikzcd}
A \arrow[r,"i"] & B \arrow[r,"r"] & A,
\end{tikzcd}
\end{equation*}
with $H:r\circ i\htpy \idfunc$. Show that $\id{x}{y}$ is a retract of $\id{i(x)}{i(y)}$.\index{retract!identity types of}
\item \label{ex:sigma_assoc}\index{Sigma type@{$\Sigma$-type}!associativity of}Let $B:A\to \type$, and let $C:\prd{x:A}B(a)\to\type$. Construct an equivalence
\begin{equation*}
\mathsf{\Sigma.assoc}:\eqv{\Big(\sm{p:\sm{x:A}B(x)}C(\proj 1 p,\proj 2 p)\Big)}{\Big(\sm{x:A}\sm{y:B(x)}C(x,y)\Big)}
\end{equation*}
\item \label{ex:int_group_laws}\index{Z@{$\Z$}!group laws} To define all the proof terms involved in showing that the integers form an abelian group is fairly involved. We suggest to show first that $\Z$ is a retract of $\N\times\N$\index{Z@{$\Z$}!as retract of $\N\times\N$} in a way that is compatible with addition.
\begin{subexenum}
\item Define the map $\Z\to\N\times\N$ by
\begin{align*}
\mathsf{neg}(k) & \mapsto (0,k) \\
0 & \mapsto (0,0) \\
\mathsf{pos}(k) & \mapsto (k,0)
\end{align*}
for $k:\N$. Construct a retraction $r:\N\times \N\to \Z$ of this map, in such a way that 
\begin{equation*}
r((m+m',n+n'))=r(m,n)+r(m',n')
\end{equation*}
for any $m,n:\N$.
\item Use this retraction to show that the operations $k,l\mapsto k+l$ and $k\mapsto -k$ on the integers defined in \autoref{ex:int_group_ops} satisfy the group laws:
\begin{align*}
k+(l+m) & = (k+l)+m \\
k+0 & = k \\
0+k & = k \\
k+(-k) & = 0 \\
(-k) + k & = 0,
\end{align*}
and that $k+l=l+k$, making $\pairr{\Z,0,+,-}$ into an abelian group.
\end{subexenum}
\item \label{ex:int_ptd_auto} Given a type $X$ with a base point $x_0$ and an equivalence $e:\eqv{X}{X}$, construct a map $f:\Z\to X$ for which $f(0)=x_0$ and $f\circ\mathsf{succ} \htpy e\circ f$. 
\end{exercises}

\chapter{Contractible types and contractible maps}
\chaptermark{Contractible types and maps}

\section{Contractible types}
\begin{defn}
A type $A$ is said to be \define{contractible} if there is a term of type
\begin{equation*}
\iscontr(A) \defeq \sm{a:A}\prd{x:A}a=x.
\end{equation*}
Given a term $(c,C):\iscontr(A)$, we call $c:A$ the \define{center of contraction} of $A$, and we call $C:\prd{x:A}a=x$ the \define{contraction} of $A$.
\end{defn}

\begin{rmk}
Suppose $A$ is a contractible type with center of contraction $c$ and contraction $C$. Then the type of $C$ is (judgmentally) equal to the type
\begin{equation*}
\mathsf{const}_a\htpy\idfunc[A].
\end{equation*}
In other words, the contraction $C$ is a \emph{homotopy} from the constant function to the identity function.
\end{rmk}

\begin{thm}
The unit type $\unit$ is contractible.
\end{thm}

\begin{proof}
The unit type comes equipped with a point $\ttt:\unit$. Now we have to construct the contraction, which is of type $\prd{x:\unit}\ttt=x$. We do this by induction on $x$, so we only have to provide a term of type $\ttt=\ttt$. Here we can just take $\refl{\ttt}$. 
\end{proof}

\section{Contractible maps}
\begin{defn}
Let $f:A\to B$ be a function, and let $b:B$. The \define{fiber} of $f$ at $b$ is defined to be the type
\begin{equation*}
\fib{f}{b}\defeq\sm{a:A}f(a)=b.
\end{equation*}
\end{defn}

\begin{defn}
We say that a function $f:A\to B$ is \define{contractible} if there is a term of type
\begin{equation*}
\iscontr(f)\defeq\prd{b:B}\iscontr(\fib{f}{b}).
\end{equation*}
\end{defn}

\begin{thm}\label{thm:equiv_contr}
Any contractible map is an equivalence.
\end{thm}

\begin{proof}
Let $f:A\to B$ be a contractible map. Using the center of contraction of each $\fib{f}{y}$, we obtain a term of type
\begin{align*}
\lam{y}\pairr{g(y),G(y)}:\prd{y:B}\fib{f}{y}.
\end{align*}
Thus, we get map $g:B\to A$, and a homotopy $G:\prd{y:B} f(g(y))=y$. In other words, we get a section of $f$.

It remains to construct a retraction of $f$. Taking $g$ as our retraction, we have to show that $\prd{x:A} g(f(x))=x$. Note that we get an identification $p:f(g(f(x)))=f(x)$ since $g$ is a section of $f$. Moreover, since $\fib{f}{f(x)}$ is contractible we get an identification $q:\pairr{g(f(x)),p}=\pairr{x,\refl{f(x)}}$. The base path of this identification is an identification of type $g(f(x))=x$, as desired.
\end{proof}

\section{Equivalences are contractible maps}

In this section we will show the converse to \autoref{thm:equiv_contr}, that equivalences are contractible maps. Before we do so, we will establish some useful constructions on homotopies and section-retraction pairs.

\begin{defn}\label{defn:htpy_nat}
Let $f,g:A\to B$ be functions, and consider $H:f\htpy g$ and $p:x=y$ in $A$. We will construct an identification
\begin{align*}
\mathsf{htpy\usc{}nat}(H,p) & :\ct{H(x)}{\ap{g}{p}}=\ct{\ap{f}{p}}{H(y)}
\end{align*}
witnessing that the square
\begin{equation*}
\begin{tikzcd}
f(x) \arrow[r,equals,"H(x)"] \arrow[d,equals,swap,"\ap{f}{p}"] & g(x) \arrow[d,equals,"\ap{g}{p}"] \\
f(y) \arrow[r,equals,swap,"H(y)"] & g(y)
\end{tikzcd}
\end{equation*}
commutes.
\end{defn}

\begin{defn}\label{defn:retraction_swap}
Let $f:A\to B$, $g:B\to A$, and let $H:g\circ f\htpy \idfunc[A]$ be a homotopy witnessing that $g$ is a retraction of $f$. We construct an identification $H(gf(x))=\ap{gf}{H(x)}$, for any $x:A$.
\end{defn}

\begin{constr}
By the naturality of homotopies with respect to identifications the square
\begin{equation*}
\begin{tikzcd}[column sep=large]
gfgf(x) \arrow[d,swap,equals,"\ap{gf}{H(x)}"] \arrow[r,equals,"H(gf(x))"] & gf(x) \arrow[d,equals,"H(x)"] \\
gf(x) \arrow[r,swap,equals,"H(x)"] & x
\end{tikzcd}
\end{equation*}
commutes. This gives the desired identification $H(gf(x))=\ap{gf}{H(x)}$.
\end{constr}

\begin{thm}\label{thm:contr_equiv}
Any equivalence is a contractible map.
\end{thm}

\begin{proof}
Since every equivalence has the structure of an invertible map by \autoref{defn:inv_equiv}, it suffices to show that any invertible map is contractible.

Let $f:A\to B$ be a map, with $g:B\to A$, $G:f\circ g\htpy\idfunc[B]$, and $H:h\circ f\htpy \idfunc[A]$.
We have for any $y:B$ the term $\pairr{g(y),G(y)}:\fib{f}{y}$. However, as our center of contraction we take
$\pairr{g(y),\epsilon(y)}$, where
\begin{equation*}
\varepsilon(y) \defeq \ct{\ap{fg}{G(y)}^{-1}}{\ap{f}{H(g(y))}}{G(y)}.
\end{equation*}
Now it remains to construct the contraction. Let $x:A$, and let $p:f(x)=y$.
Since $p:f(x)=y$ has a free endpoint, we can apply path induction on it. 
Our goal is now to construct an identification
\begin{equation*}
\pairr{g(f(x)),\varepsilon(f(x))}=\pairr{x,\refl{f(x)}}.
\end{equation*}
We will construct an identification of the form $\mathsf{eq\usc{}pair}(H(x),\nameless)$,
so it remains to construct an identification of type
\begin{equation*}
\trans{H(x)}{\varepsilon(f(x))}=\refl{f(x)}.
\end{equation*}
Using \autoref{ex:trans_ap} we see that it suffices to show that the square
\begin{equation*}
\begin{tikzcd}[column sep=8em]
fgfgf(x) \arrow[r,equals,"\ap{fg}{G(f(x))}"] \arrow[d,equals,swap,"\ap{f}{H(gf(x))}"] & fgf(x) \arrow[d,equals,"\ap{f}{H(x)}"] \\
fgf(x) \arrow[r,equals,swap,"G(f(x))"] & f(x)
\end{tikzcd}
\end{equation*}
commutes. Recall from \autoref{defn:retraction_swap} that we have $H(gf(x))=\ap{gf}{H(x)}$ and $\ap{fg}{G(y)}=G(fg(y))$. Using these two identifications and \autoref{ex:ap_ap}, we see that it suffices to show that the square
\begin{equation*}
\begin{tikzcd}[column sep=8em]
fgfgf(x) \arrow[r,equals,"G(fgf(x))"] \arrow[d,equals,swap,"\ap{fgf}{H(x)}"] & fgf(x) \arrow[d,equals,"\ap{f}{H(x)}"] \\
fgf(x) \arrow[r,equals,swap,"G(f(x))"] & f(x)
\end{tikzcd}
\end{equation*}
commutes. However, this is just a naturality square of homotopies, which commutes by \autoref{defn:htpy_nat}.
\end{proof}

\begin{cor}
Let $A$ be a type, and let $a:A$. Then the type
\begin{equation*}
\sm{x:A}x=a
\end{equation*}
is contractible.
\end{cor}

\begin{proof}
By \autoref{thm:id_equiv}, the identity function is an equivalence. Therefore, the fibers of the identity function are contractible by \autoref{thm:contr_equiv}. Note that $\sm{x:A}x=a$ is exactly the fiber of $\idfunc[A]$ at $a:A$.
\end{proof}

\begin{comment}
\begin{proof}
We have the term $(a,\refl{a}):\sm{x:A}a=x$, which we take for the center of contraction. To construct the contraction, we have to show that
\begin{equation*}
\prd{p:\sm{x:A}a=x} (a,\refl{a})=p.
\end{equation*}
By the induction principle for dependent pair types it suffices to construct a term of type
\begin{equation*}
\prd{x:A}{p:a=x} (a,\refl{a})=(x,p)
\end{equation*}
Note that we may proceed here by path induction on $p$. That is, it suffices to consider the case $p\jdeq\refl{a}$, and show that $(a,\refl{a})=(a,\refl{a})$. Here we choose $\refl{(a,\refl{a})}$.
\end{proof}
\end{comment}

\begin{exercises}
\item Construct an equivalence 
\begin{equation*}
\eqv{\big(\sm{x:A}f(x)=y\big)}{\big(\sm{x:A}y=f(x)\big)}.
\end{equation*}
Conclude that $\sm{x:A}a=x$ is contractible for any $a:A$.
\item \label{ex:contr_retr}Suppose that $A$ is a retract of $B$. Show that
\begin{equation*}
\iscontr(B)\to\iscontr(A).
\end{equation*}
\item \label{ex:contr_equiv}Show that for any type $A$, the map $\mathsf{const}_\ttt : A\to \unit$ is contractible if and only if $A$ is contractible. Conclude that for any map $f:A\to B$, if any two of the three assertions
\begin{enumerate}
\item $A$ is contractible
\item $B$ is contractible
\item $f$ is an equivalence
\end{enumerate}
hold, then so does the third.
\item \label{ex:contr_ind} Let $C$ be a contractible type with center of contraction $c:C$. Furthermore, let $B:C\to\type$ be a type family. 
\begin{subexenum}
\item Show that the map $b\mapsto\pairr{c,b}:B(c)\to\sm{x:C}B(x)$ is an equivalence.
\item Construct a section
\begin{equation*}
\mathsf{contr\usc{}ind}:B(c)\to \prd{x:C}B(x),
\end{equation*}
of the map $\mathsf{ev}_c\defeq\lam{f}f(c)$.
\item Construct a homotopy $\mathsf{contr\usc{}ind}(\mathsf{ev}_c(f))\htpy f$ for each $f:\prd{x:C}B(x)$.
\end{subexenum}
\end{exercises}

% !TEX root = hott_intro.tex

\section{The fundamental theorem of identity types}\label{chap:fundamental}
\sectionmark{The fundamental theorem}

In many situations in homotopy type theory it is important to know what the identity types are. For example, we have used a characterization of the identity types of the fibers of a map in order to conclude that any equivalence is a contractible map. Therefore it is a routine task to give for any given type of interest, a characterization of its identity type. The fundamental theorem of identity types is our main tool to carry out such characterizations. To name a few applications, we will show in \cref{thm:eq_nat} that the identity type of the natural numbers is equivalent to its observational equality, and we will show in \cref{thm:eq-circle} that the loop space of the circle is equivalent to $\Z$.

In order to prove the fundamental theorem of identity types, we first need the basic fact that a family of maps is a family of equivalences if and only if it induces an equivalence on total spaces. This fact will also be used in many other situations, most notably in the characterization of pullback squares in \cref{cor:pb_fibequiv}.

Our first application of the fundamental theorem of identity types in the present lecture is a simple proof that any equivalence is an embedding. Embeddings are maps that induce equivalences on identity types, i.e., they are the homotopical analogue of injective maps. In our second application we characterize the identity types of coproducts.

\subsection{Families of equivalences}
Consider a family
\begin{equation*}
f : \prd{x:A}B(x)\to C(x)
\end{equation*}
of maps.

\begin{defn}
We define the map
\begin{equation*}
\total{f}:\sm{x:A}B(x)\to\sm{x:A}C(x).
\end{equation*}
by $\lam{(x,y)}(x,f(x,y))$.
\end{defn}

\begin{lem}\label{lem:fib_total}
  For any family of maps $f:\prd{x:A}B(x)\to C(x)$ and any $t:\sm{x:A}C(x)$,
  there is an equivalence
  \begin{equation*}
    \eqv{\fib{\total{f}}{t}}{\fib{f(\proj 1(t))}{\proj 2(t)}}.
  \end{equation*}
\end{lem}

\begin{proof}
  For any $p:\fib{\total{f}}{t}$ we define $\varphi(t,p):\fib{\proj 1(t)}{\proj 2(t)}$ by $\Sigma$-induction on $p$. Therefore it suffices to define $\varphi(t,(s,\alpha)):\fib{\proj 1(t)}{\proj 2 (t)}$ for any $s:\sm{x:A}B(x)$ and $\alpha:\total{f}(s)=t$. Now we proceed by path induction on $\alpha$, so it suffices to define $\varphi(\total{f}(s),(s,\refl{})):\fib{f(\proj 1(\total{f}(s)))}{\proj 2(\total{f}(s))}$. Finally, we use $\Sigma$-induction on $s$ once more, so it suffices to define
  \begin{equation*}
    \varphi((x,f(x,y)),((x,y),\refl{})):\fib{f(x)}{f(x,y)}.
  \end{equation*}
  Now we take as our definition
  \begin{equation*}
    \varphi((x,f(x,y)),((x,y),\refl{}))\defeq(y,\refl{}).
  \end{equation*}

  For the proof that this map is an equivalence we construct a map
  \begin{equation*}
    \psi(t) : \fib{f(\proj 1(t))}{\proj 2(t)}\to\fib{\total{f}}{t}
  \end{equation*}
  equipped with homotopies $G(t):\varphi(t)\circ\psi(t)\htpy\idfunc$ and $H(t):\psi(t)\circ\varphi(t)\htpy\idfunc$. In each of these definitions we use $\Sigma$-induction and path induction all the way through, until an obvious choice of definition becomes apparent. We define $\psi(t)$, $G(t)$, and $H(t)$ as follows:
  \begin{align*}
    \psi((x,f(x,y)),(y,\refl{})) & \defeq ((x,y),\refl{}) \\
    G((x,f(x,y)),(y,\refl{})) & \defeq \refl{} \\
    H((x,f(x,y)),((x,y),\refl{})) & \defeq \refl{}.\qedhere
  \end{align*}
\end{proof}

\begin{thm}\label{thm:fib_equiv}
Let $f:\prd{x:A}B(x)\to C(x)$ be a family of maps. The following are equivalent:
\begin{enumerate}
\item For each $x:A$, the map $f(x)$ is an equivalence. In this case we say that $f$ is a \define{family of equivalences}.
\item The map $\total{f}:\sm{x:A}B(x)\to\sm{x:A}C(x)$ is an equivalence.
\end{enumerate}
\end{thm}

\begin{proof}
By \cref{thm:equiv_contr,thm:contr_equiv} it suffices to show that $f(x)$ is a contractible map for each $x:A$, if and only if $\total{f}$ is a contractible map. Thus, we will show that $\fib{f(x)}{c}$ is contractible if and only if $\fib{\total{f}}{x,c}$ is contractible, for each $x:A$ and $c:C(x)$. However, by \cref{lem:fib_total} these types are equivalent, so the result follows by \cref{ex:contr_equiv}.
\end{proof}

Now consider the situation where we have a map $f:A\to B$, and a family $C$ over $B$. Then we have the map
\begin{equation*}
  \lam{(x,z)}(f(x),z):\sm{x:A}C(f(x))\to\sm{y:B}C(y).
\end{equation*}
We claim that this map is an equivalence when $f$ is an equivalence. The technique to prove this claim is the same as the technique we used in \cref{thm:fib_equiv}: first we note that the fibers are equivalent to the fibers of $f$, and then we use the fact that a map is an equivalence if and only if its fibers are contractible to finish the proof.

\begin{lem}\label{lem:total-equiv-base-equiv}
  Consider an equivalence $e:A\simeq B$, and let $C$ be a type family over $B$. Then the map
  \begin{equation*}
    \sigma_f(C) \defeq\lam{(x,z)}(f(x),z):\sm{x:A}C(f(x))\to\sm{y:B}C(y)
  \end{equation*}
  is an equivalence.
\end{lem}

\begin{proof}
  We claim that for each $t:\sm{y:B}C(y)$ there is an equivalence
  \begin{equation*}
    \fib{\sigma_f(C)}{t}\simeq \fib{f}{\proj 1(t)}.
  \end{equation*}
  We prove this by constructing
  \begin{align*}
    \varphi(t) & : \fib{\sigma_f(C)}{t}\to\fib{f}{\proj 1 (t)} \\
    \psi(t) & : \fib{f}{\proj 1(t)} \to\fib{\sigma_f(C)}{t} \\
    G(t) & : \varphi\circ\psi\htpy\idfunc\\
    H(t) & : \psi\circ\varphi\htpy\idfunc.
  \end{align*}
  The construction of these functions and homotopies is by using $\Sigma$-induction and path induction all the way through, just as in the proof of \cref{lem:fib_total}. We list the definitions
  \begin{align*}
    \varphi((f(x),z),((x,z),\refl{})) & \defeq (x,\refl{}) \\
    \psi((f(x),z),(x,\refl{})) & \defeq ((x,z),\refl{}) \\
    G((f(x),z),(x,\refl{})) & \defeq \refl{} \\
    H((f(x),z),((x,z),\refl{})) & \defeq \refl{}.
  \end{align*}
  Now the claim follows, since we see that $\varphi$ is a contractible map if and only if $f$ is a contractible map.
\end{proof}

We now combine \cref{thm:fib_equiv,lem:total-equiv-base-equiv}.

\begin{defn}
  Consider a map $f:A\to B$ and a family of maps
  \begin{equation*}
    g:\prd{x:A}C(x)\to D(f(x)),
  \end{equation*}
  where $C$ is a type family over $A$, and $D$ is a type family over $B$. In this situation we also say that $g$ is a \define{family of maps over $f$}. Then we define
  \begin{equation*}
    \total[f]{g}:\sm{x:A}C(x)\to\sm{y:B}D(y)
  \end{equation*}
  by $\total[f]{g}(x,z)\defeq (f(x),g(x,z))$.
\end{defn}

\begin{thm}
  Suppose that $g$ is a family of maps over $f$, and suppose that $f$ is an equivalence. Then the following are equivalent:
  \begin{enumerate}
  \item The family of maps $g$ over $f$ is a family of equivalences.
  \item The map $\total[f]{g}$ is an equivalence.
  \end{enumerate}
\end{thm}

\begin{proof}
  Note that we have a commuting triangle
  \begin{equation*}
    \begin{tikzcd}[column sep=0]
      \sm{x:A}C(x) \arrow[rr,"{\total[f]{g}}"] \arrow[dr,swap,"\total{g}"]& & \sm{y:B}D(y) \\
      & \sm{x:A}D(f(x)) \arrow[ur,swap,"{\lam{(x,z)}(f(x),z)}"]
    \end{tikzcd}
  \end{equation*}
  By the assumption that $f$ is an equivalence, it follows that the map $\sm{x:A}D(f(x))\to \sm{y:B}D(y)$ is an equivalence. Therefore it follows that $\total[f]{g}$ is an equivalence if and only if $\total{g}$ is an equivalence. Now the claim follows, since $\total{g}$ is an equivalence if and only if $g$ if a family of equivalences.
\end{proof}

\subsection{The fundamental theorem}

The fundamental theorem of identity types (\cref{thm:id_fundamental}) tells us when, for a type family $B$ over $A$ and a fixed $a:A$, there is a family of equivalences $\prd{x:A}(a=x)\simeq B(x)$. In other words, it tells us when a family $B$ is a characterization of the identity type of $A$.

The most important implication in the fundamental theorem is that (ii) implies (i). Occasionally we will also use the third equivalent statement. We note that the fundamental theorem also appears as Theorem 5.8.4 in \cite{hottbook}.

\begin{thm}\label{thm:id_fundamental}
Let $A$ be a type with $a:A$, and let $B$ be be a type family over $A$ with $b:B(a)$.
Then  the following are logically equivalent for any family of maps
\begin{equation*}
  f:\prd{x:A}(a=x)\to B(x).
\end{equation*}
\begin{enumerate}
\item The family of maps $f$ is a family of equivalences.
\item The total space
\begin{equation*}
\sm{x:A}B(x)
\end{equation*}
is contractible.
\item The family $B$ is an identity system.
\end{enumerate}
In particular the canonical family of maps
\begin{equation*}
\mathsf{path\usc{}ind}_a(b):\prd{x:A} (a=x)\to B(x)
\end{equation*}
is a family of equivalences if and only if $\sm{x:A}B(x)$ is contractible.
\end{thm}

\begin{proof}
  First we show that (i) and (ii) are equivalent.
  By \cref{thm:fib_equiv} it follows that the family of maps $\mathsf{path\usc{}ind}_a(b)$ is a family of equivalences if and only if it induces an equivalence
  \begin{equation*}
    \eqv{\Big(\sm{x:A}a=x\Big)}{\Big(\sm{x:A}B(x)\Big)}
  \end{equation*}
  on total spaces. We have that $\sm{x:A}a=x$ is contractible. Now it follows by \cref{ex:contr_equiv}, applied in the case
  \begin{equation*}
    \begin{tikzcd}[column sep=3em]
      \sm{x:A}a=x \arrow[rr,"\total{\mathsf{path\usc{}ind}_a(b)}"] \arrow[dr,swap,"\eqvsym"] & & \sm{x:A}B(x) \arrow[dl] \\
      & \unit & \phantom{\sm{x:A}a=x}
    \end{tikzcd}
  \end{equation*}
  that $\total{\mathsf{path\usc{}ind}_a(b)}$ is an equivalence if and only if $\sm{x:A}B(x)$ is contractible.

  Now we show that (ii) and (iii) are equivalent. Note that we have the following commuting triangle
  \begin{equation*}
    \begin{tikzcd}[column sep=0]
      \prd{t:\sm{x:A}B(x)}P(t) \arrow[rr,"\mathsf{ev\usc{}pair}"] \arrow[dr,swap,"{\mathsf{ev\usc{}pt}(a,b)}"] & & \prd{x:A}{y:B(x)}P(x,y) \arrow[dl,"{\lam{f}f(a,b)}"] \\
      \phantom{\prd{x:A}{y:B(x)}P(x,y)} & P(a,b)
    \end{tikzcd}
  \end{equation*}
  In this diagram the top map has a section. Therefore it follows by \cref{ex:3_for_2} that the left map has a section if and only if the right map has a section. Notice that the left map has a section for all $P$ if and only if $\sm{x:A}B(x)$ satisfies singleton induction, which is by \cref{thm:contractible} equivalent to $\sm{x:A}B(x)$ being contractible.
\end{proof}

\subsection{Embeddings}
As an application of the fundamental theorem we show that equivalences are embeddings. The notion of embedding is the homotopical analogue of the set theoretic notion of injective map.

\begin{defn}
An \define{embedding}\index{embedding|textbf} is a map $f:A\to B$ satisfying the property that
\begin{equation*}
\apfunc{f}:(\id{x}{y})\to(\id{f(x)}{f(y)})
\end{equation*}
is an equivalence for every $x,y:A$. We write $\mathsf{is\usc{}emb}(f)$ for the type of witnesses that $f$ is an embedding.
\end{defn}

Another way of phrasing the following statement is that equivalent types have equivalent identity types.

\begin{thm}
\label{cor:emb_equiv} 
Any equivalence is an embedding.\index{embedding!equivalences are embeddings|textit}\index{equivalence!is an embedding|textit}
\end{thm}

\begin{proof}
Let $e:\eqv{A}{B}$ be an equivalence, and let $x:A$. Our goal is to show that
\begin{equation*}
\apfunc{e} : (\id{x}{y})\to (\id{e(x)}{e(y)})
\end{equation*}
is an equivalence for every $y:A$. By \cref{thm:id_fundamental} it suffices to show that 
\begin{equation*}
\sm{y:A}e(x)=e(y)
\end{equation*}
is contractible for every $y:A$. Now observe that there is an equivalence
\begin{samepage}
\begin{align*}
\sm{y:A}e(x)=e(y) & \eqvsym \sm{y:A}e(y)=e(x) \\
& \jdeq \fib{e}{e(x)}
\end{align*}
\end{samepage}
by \cref{thm:fib_equiv}, since for each $y:A$ the map
\begin{equation*}
\mathsf{inv} : (e(x)=e(y))\to (e(y)= e(x))
\end{equation*}
is an equivalence by \cref{ex:equiv_grpd_ops}.
The fiber $\fib{e}{e(x)}$ is contractible by \cref{thm:contr_equiv}, so it follows by \cref{ex:contr_equiv} that the type $\sm{y:A}e(x)=e(y)$ is indeed contractible.
\end{proof}

\begin{comment}
As a first application of the fundamental theorem, we compute the identity type of a coproduct.

\begin{defn}
Let $A$ and $B$ be types in $\UU$. We construct equivalences
\begin{align*}
(\id[A+B]{\inl(x)}{\inl(x')}) & \eqvsym (\id[A]{x}{x'}) \\
(\id[A+B]{\inl(x)}{\inr(y')}) & \eqvsym \emptyt \\
(\id[A+B]{\inr(y)}{\inl(x')}) & \eqvsym \emptyt \\
(\id[A+B]{\inr(y)}{\inr(y')}) & \eqvsym (\id[B]{y}{y'}).
\end{align*}
\end{defn}

\begin{constr}
We define by double induction on the disjoint sum the binary relation
\begin{equation*}
E : (A+B)\to (A+B)\to\UU
\end{equation*}
given by
\begin{align*}
E({\inl(x)},{\inl(x')}) & \defeq \id[A]{x}{x'} \\
E({\inl(x)},{\inr(y')}) & \defeq \emptyt \\
E({\inr(y)},{\inl(x')}) & \defeq \emptyt \\
E({\inr(y)},{\inr(y')}) & \defeq (\id[B]{y}{y'}).
\end{align*}
Moreover, we have a term $\rho:\prd{s:A+B}E(s,s)$ defined by $\rho(\inl(x))\defeq\refl{x}$ and $\rho(\inr(y))\defeq\refl{y}$.

Our goal is to construct an equivalence $\eqv{(\id{s}{t})}{E(s,t)}$ for any $s,t:A+B$. 
By \cref{thm:id_fundamental} it suffices to show that for any $s:A+B$, the type
\begin{equation*}
\sm{t:A+B}E(s,t)
\end{equation*}
is contractible. The center of contraction is taken to be $\pairr{s,\rho(s)}$, so it remains to construct the contraction
\begin{equation*}
\prd{t:A+B}{e:E(s,t)} \pairr{s,\rho(s)}=\pairr{t,e}.
\end{equation*}
This is done by induction on $s$ and $t$, so we have to show that
\begin{align*}
& \prd{x':A}{p:x=x'} \pairr{\inl(x),\refl{x}}=\pairr{x',p} \\
& \prd{y':A}{q:\emptyt} \pairr{\inl(x),\refl{x}}=\pairr{y',q} \\
& \prd{x':A}{q:\emptyt} \pairr{\inr(y),\refl{y}}=\pairr{x',q} \\
& \prd{y':A}{p:y=y'} \pairr{\inr(y),\refl{y}}=\pairr{y',p}.
\end{align*}
The first and fourth case are easily shown by path induction on $p$, and the second and third case are easily shown by induction on the empty type.
\end{constr}
\end{comment}

\subsection{Disjointness of coproducts}

To give a second application of the fundamental theorem of identity types, we characterize the identity types of coproducts. Our goal in this section is to prove the following theorem.

\begin{thm}\label{thm:id-coprod-compute}
Let $A$ and $B$ be types. Then there are equivalences
\begin{align*}
(\inl(x)=\inl(x')) & \eqvsym (x = x')\\
(\inl(x)=\inr(y')) & \eqvsym \emptyt \\
(\inr(y)=\inl(x')) & \eqvsym \emptyt \\
(\inr(y)=\inr(y')) & \eqvsym (y=y')
\end{align*}
for any $x,x':A$ and $y,y':B$.
\end{thm}

In order to prove \cref{thm:id-coprod-compute}, we first define
a binary relation $\mathsf{Eq\usc{}coprod}_{A,B}$ on the coproduct $A+B$.

\begin{defn}
Let $A$ and $B$ be types. We define 
\begin{equation*}
\mathsf{Eq\usc{}coprod}_{A,B} : (A+B)\to (A+B)\to\UU
\end{equation*}
by double induction on the coproduct, postulating
\begin{align*}
\mathsf{Eq\usc{}coprod}_{A,B}(\inl(x),\inl(x')) & \defeq (x=x') \\
\mathsf{Eq\usc{}coprod}_{A,B}(\inl(x),\inr(y')) & \defeq \emptyt \\
\mathsf{Eq\usc{}coprod}_{A,B}(\inr(y),\inl(x')) & \defeq \emptyt \\
\mathsf{Eq\usc{}coprod}_{A,B}(\inr(y),\inr(y')) & \defeq (y=y')
\end{align*}
The relation $\mathsf{Eq\usc{}coprod}_{A,B}$ is also called the \define{observational equality of coproducts}\index{observational equality!of coproducts}.
\end{defn}

\begin{lem}
The observational equality relation $\mathsf{Eq\usc{}coprod}_{A,B}$ on $A+B$ is reflexive, and therefore there is a map
\begin{equation*}
\mathsf{Eq\usc{}coprod\usc{}eq}:\prd{s,t:A+B} (s=t)\to \mathsf{Eq\usc{}coprod}_{A,B}(s,t)
\end{equation*}
\end{lem}

\begin{constr}
The reflexivity term $\rho$ is constructed by induction on $t:A+B$, using
\begin{align*}
\rho(\inl(x))\defeq \refl{\inl(x)}  & : \mathsf{Eq\usc{}coprod}_{A,B}(\inl(x)) \\
\rho(\inr(y))\defeq \refl{\inr(y)} & : \mathsf{Eq\usc{}coprod}_{A,B}(\inr(y)).\qedhere
\end{align*}
\end{constr}

To show that $\mathsf{Eq\usc{}coprod\usc{}eq}$ is a family of equivalences, we will use the fundamental theorem, \cref{thm:id_fundamental}. Moreover, we will use the functoriality of coproducts (established in \cref{ex:coproduct_functor}), along with the following facts about $\Sigma$-types, coproducts, and the empty type:
\begin{align*}
\sm{t:A+B}P(t) & \eqvsym \Big(\sm{x:A}P(\inl(x))\Big)+\Big(\sm{y:B}P(\inr(y))\Big)\\
\sm{x:A}\emptyt & \eqvsym \emptyt \\
A+\emptyt & \eqvsym A.
\end{align*}
All of these equivalences are straightforward to construct, so we leave them as an exercise to the reader. 

\begin{lem}\label{lem:is-contr-total-eq-coprod}
For any $s:A+B$ the total space
\begin{equation*}
\sm{t:A+B}\mathsf{Eq\usc{}coprod}_{A,B}(s,t)
\end{equation*}
is contractible.
\end{lem}

\begin{proof}
We will do the proof by induction on $s$. The two cases are similar, so we only show that the total space
\begin{equation*}
\sm{t:A+B}\mathsf{Eq\usc{}coprod}_{A,B}(\inl(x),t)
\end{equation*}
is contractible. Note that we have equivalences
\begin{samepage}
\begin{align*}
& \sm{t:A+B}\mathsf{Eq\usc{}coprod}_{A,B}(\inl(x),t) \\
& \eqvsym \Big(\sm{x':A}\mathsf{Eq\usc{}coprod}_{A,B}(\inl(x),\inl(x'))\Big)+\Big(\sm{y':B}\mathsf{Eq\usc{}coprod}_{A,B}(\inl(x),\inr(y'))\Big) \\
& \eqvsym \Big(\sm{x':A}x=x'\Big)+\Big(\sm{y':B}\emptyt\Big) \\
& \eqvsym \Big(\sm{x':A}x=x'\Big)+\emptyt \\
& \eqvsym \sm{x':A}x=x'.
\end{align*}%
\end{samepage}%
The latter type is contractible by \cref{thm:total_path}.
\end{proof}

\begin{proof}[Proof of \cref{thm:id-coprod-compute}]
The proof is now concluded with an application of \cref{thm:id_fundamental}, using \cref{lem:is-contr-total-eq-coprod}.
\end{proof}

\begin{exercises}
\item
  \begin{subexenum}
  \item Show that the map $\emptyt\to A$ is an embedding for every type $A$.
  \item Show that $\succN:\N\to\N$ is an embedding.
  \item Show that $\inl:A\to A+B$ and $\inr:B\to A+B$ are embeddings for any two types $A$ and $B$.
  \end{subexenum}
\item Consider an equivalence $e:A\simeq B$. Construct an equivalence
  \begin{equation*}
    (e(x)=y)\simeq(x=e^{-1}(y))
  \end{equation*}
  for every $x:A$ and $y:B$.
\item Show that 
\begin{equation*}
(f\htpy g)\to (\mathsf{is\usc{}emb}(f)\leftrightarrow\mathsf{is\usc{}emb}(g))
\end{equation*}
for any $f,g:A\to B$.
\item \label{ex:emb_triangle}Consider a commuting triangle
\begin{equation*}
\begin{tikzcd}[column sep=tiny]
A \arrow[rr,"h"] \arrow[dr,swap,"f"] & & B \arrow[dl,"g"] \\
& X
\end{tikzcd}
\end{equation*}
with $H:f\htpy g\circ h$. 
\begin{subexenum}
\item Suppose that $g$ is an embedding. Show that $f$ is an embedding if and only if $h$ is an embedding.
\item Suppose that $h$ is an equivalence. Show that $f$ is an embedding if and only if $g$ is an embedding.
\end{subexenum}
\item \label{ex:is-equiv-is-equiv-functor-coprod}Consider two maps $f:A\to A'$ and $g:B \to B'$.
  \begin{subexenum}
  \item Show that if the map
    \begin{equation*}
      f+g:(A+B)\to (A'+B')
    \end{equation*}
    is an equivalence, then so are both $f$ and $g$ (this is the converse of \cref{ex:coproduct_functor_equivalence}).
  \item Show that $f+g$ is an embedding if and only if both $f$ and $g$ are embeddings.
  \end{subexenum}
\item \label{ex:htpy_total} 
\begin{subexenum}
\item Let $f,g:\prd{x:A}B(x)\to C(x)$ be two families of maps. Show that
\begin{equation*}
\Big(\prd{x:A}f(x)\htpy g(x)\Big)\to \Big(\total{f}\htpy \total{g}\Big). 
\end{equation*}
\item Let $f:\prd{x:A}B(x)\to C(x)$ and let $g:\prd{x:A}C(x)\to D(x)$. Show that
\begin{equation*}
\total{\lam{x}g(x)\circ f(x)}\htpy \total{g}\circ\total{f}.
\end{equation*}
\item For any family $B$ over $A$, show that
\begin{equation*}
\total{\lam{x}\idfunc[B(x)]}\htpy\idfunc.
\end{equation*}
\end{subexenum}
\item \label{ex:id_fundamental_retr}Let $a:A$, and let $B$ be a type family over $A$. 
\begin{subexenum}
\item Use \cref{ex:htpy_total,ex:contr_retr} to show that if each $B(x)$ is a retract of $\id{a}{x}$, then $B(x)$ is equivalent to $\id{a}{x}$ for every $x:A$.
\item Conclude that for any family of maps
\begin{equation*}
f : \prd{x:A} (a=x) \to B(x),
\end{equation*}
if each $f(x)$ has a section, then $f$ is a family of equivalences.
\end{subexenum}
\item Use \cref{ex:id_fundamental_retr} to show that for any map $f:A\to B$, if
\begin{equation*}
\apfunc{f} : (x=y) \to (f(x)=f(y))
\end{equation*}
has a section for each $x,y:A$, then $f$ is an embedding.
\item \label{ex:path-split}We say that a map $f:A\to B$ is \define{path-split}\index{path-split|textbf} if $f$ has a section, and for each $x,y:A$ the map
\begin{equation*}
\apfunc{f}(x,y):(x=y)\to (f(x)=f(y))
\end{equation*}
also has a section. We write $\mathsf{path\usc{}split}(f)$\index{path_split(f)@{$\mathsf{path\usc{}split}(f)$}|textbf} for the type
\begin{equation*}
\mathsf{sec}(f)\times\prd{x,y:A}\mathsf{sec}(\apfunc{f}(x,y)).
\end{equation*}
Show that for any map $f:A\to B$ the following are equivalent:
\begin{enumerate}
\item The map $f$ is an equivalence.
\item The map $f$ is path-split.
\end{enumerate}
\begin{comment}
\item \label{ex:eqv_sigma_mv}Consider a map
\begin{equation*}
f:A \to \sm{y:B}C(y).
\end{equation*}
\begin{subexenum}
\item Construct a family of maps
\begin{equation*}
f':\prd{y:B} \fib{\proj 1\circ f}{y}\to C(y).
\end{equation*}
\item Construct an equivalence
\begin{equation*}
\eqv{\fib{f'(b)}{c}}{\fib{f}{(b,c)}}
\end{equation*}
for every $(b,c):\sm{y:B}C(y)$.
\item Conclude that the following are equivalent:
\begin{enumerate}
\item $f$ is an equivalence.
\item $f'$ is a family of equivalences.
\end{enumerate}
\end{subexenum}
\item \label{ex:coh_intro}Consider a type $A$ with base point $a:A$, and let $B$ be a type family on $A$ that implies the identity type, i.e., there is a term
\begin{equation*}
\alpha : \prd{x:A} B(x)\to (a=x).
\end{equation*}
Show that the \define{coherence reduction map}
\begin{equation*}
\mathsf{coh\usc{}red} : \Big(\sm{y:B(a)}\alpha(a,y)=\refl{a}\Big) \to \Big(\sm{x:A}B(x)\Big)
\end{equation*}
defined by $\lam{(y,q)}(a,y)$ is an equivalence.
\end{comment}
\item \label{ex:fiber_trans}Consider a triangle
\begin{equation*}
\begin{tikzcd}[column sep=small]
A \arrow[rr,"h"] \arrow[dr,swap,"f"] & & B \arrow[dl,"g"] \\
& X
\end{tikzcd}
\end{equation*}
with a homotopy $H:f\htpy g\circ h$ witnessing that the triangle commutes. 
\begin{subexenum}
\item Construct a family of maps
\begin{equation*}
\mathsf{fib\usc{}triangle}(h,H):\prd{x:X}\fib{f}{x}\to\fib{g}{x},
\end{equation*}
for which the square
\begin{equation*}
\begin{tikzcd}[column sep=8em]
\sm{x:X}\fib{f}{x} \arrow[r,"\total{\mathsf{fib\usc{}triangle}(h,H)}"] \arrow[d] & \sm{x:X}\fib{g}{x} \arrow[d] \\
A \arrow[r,swap,"h"] & B
\end{tikzcd}
\end{equation*}
commutes, where the vertical maps are as constructed in \cref{ex:fib_replacement}.
\item Show that $h$ is an equivalence if and only if $\mathsf{fib\usc{}triangle}(h,H)$ is a family of equivalences.
\end{subexenum}
\begin{comment}
\item Let $f:A\to B$ be a map, and let $s,t : \fib{f}{b}$. Consider the function
\begin{equation*}
\varphi : (s=t)\to \fib{\apfunc{f}}{\ct{\proj 2(s)}{\proj 2(t)^{-1}}}
\end{equation*}
given by $\varphi(\refl{s})=(\refl{\proj 1(s)},\mathsf{right\usc{}inv}(\proj 2(s))^{-1})$. Show that this map is an equivalence. Conclude that for any $q:f(x)=f(y)$ we have an equivalence
\begin{equation*}
((x,q)=(y,\refl{f(y)})) \simeq \fib{\apfunc{f}}{q}.
\end{equation*}
\item Construct an equivalence 
\begin{equation*}
\eqv{\big(\sm{x:A}f(x)=y\big)}{\big(\sm{x:A}y=f(x)\big)}.
\end{equation*}
%Conclude that $\sm{x:A}a=x$ is contractible for any $a:A$.
\end{comment}
\end{exercises}

\chapter{The hierarchy of homotopical complexity}
\chaptermark{Homotopical complexity}
%Not all types have interesting higher groupoid structure. For example, we will see below that two natural numbers can only be equal in at most one way. Voevodsky articulated a useful notion to detect the homotopical complexity of types, which allows us to distinguish between contractible types (also called \emph{$(-2)$-types}), \emph{propositions} (also called \emph{$(-1)$-types}), \emph{sets} (\emph{$0$-types}), and \emph{$k$-types} for higher $k$.

%We will see [later] that there are types that are not $k$-types for any $k$.

\section{Propositions and subtypes}

\begin{defn}
A type $A$ is said to be a \define{proposition} if there is a term of type
\begin{equation*}
\isprop(A)\defeq\prd{x,y:A}\iscontr(x=y).
\end{equation*}
\end{defn}

\begin{eg}\label{eg:prop_contr}
Any contractible type is a proposition by \cref{ex:prop_contr}. However, propositions do not need to be inhabited: the empty type is also a proposition, since
\begin{equation*}
\prd{x,y:\emptyt}\iscontr(x=y)
\end{equation*}
follows from the induction principle of the empty type.
\end{eg}

In the following lemma we prove that in order to show that a type $A$ is a proposition, it suffices to show that any two terms of $A$ are equal. In other words, propositions are types with \define{proof irrelevance}.

\begin{lem}\label{lem:isprop_eq}
Let $A$ be a type. Then we have
\begin{equation*}
\isprop(A)\leftrightarrow \Big(\prd{x,y:A}x=y\Big).
\end{equation*}
\end{lem}

\begin{proof}
Suppose $A$ is a proposition. By taking the center of contraction of $\id{x}{y}$ for each $x,y:A$ we obtain a term of type $\prd{x,y:A}\id{x}{y}$.

Now suppose that $A$ is a type equipped with $H:\prd{x,y:A}\id{x}{y}$. Then we take $\ct{H(x,x)^{-1}}{H(x,y)}$ as the center of contraction of $\id{x}{y}$. To construct the contraction
\begin{equation*}
\prd{p:\id{x}{y}} \ct{H(x,x)^{-1}}{H(x,y)}=p
\end{equation*}
we proceed by path induction. Our goal is to show that
\begin{equation*}
\ct{H(x,x)^{-1}}{H(x,x)}=\refl{x}.\qedhere
\end{equation*}
\end{proof}

By proof irrelevance it follows that propositions are contractible as soon as they are inhabited.

\begin{cor}\label{cor:contr_prop}
For any proposition $P$ we have $P\to\iscontr(P)$.
\end{cor}

In type theory terms always come equipped with their types, i.e.~they never appear in isolation. This is useful from the perspective that terms are programs with a certain specification, but as a consequence we cannot consider subtypes in the same way as set theorists have subsets. Our definition of subtype is therefore considerably different:

\begin{defn}
A type family $B$ over $A$ is said to be a \define{subtype} of $A$ if for each $x:A$ the type $B(x)$ is a proposition.
\end{defn}

We will show in \cref{thm:subtype} that a type family $B$ over $A$ is a subtype of $A$ if and only if the projection map $\proj 1:\big(\sm{x:A}B(x)\big)\to A$ is an embedding.

\begin{comment}
\begin{samepage}
\begin{thm}\label{thm:subtype}
Let $B$ be a type family over $A$. The following are equivalent:
\begin{enumerate}
\item The family $B$ over $A$ is a \define{subtype} of $A$, in the sense that for each $x:A$ the type $B(x)$ is a proposition.
\item The projection map
\begin{equation*}
\proj 1 : \Big(\sm{x:A}B(x)\Big)\to A
\end{equation*}
is an embedding. 
\end{enumerate}
\end{thm}
\end{samepage}

\begin{proof}
First assume that $B(x)$ is a proposition for each $x:A$. Our goal is to show that
\begin{equation*}
\apfunc{\proj 1} : (\id{s}{t})\to (\id{\proj 1(s)}{\proj 1(t)})
\end{equation*}
is an equivalence for every $s,t:\sm{x:A}B(x)$. By $\Sigma$-induction on $s$ and \autoref{thm:id_fundamental} it suffices to show that the type
\begin{equation*}
\sm{t:\sm{x:A}B(x)} \id{a}{\proj 1(t)}
\end{equation*}
is contractible, for any $a:A$ and $b:B(a)$. 
For the center of contraction we take $\pairr{\pairr{a,b},\refl{a}}$. 
The contraction is constructed by applying $\Sigma$-induction twice, by which it suffices to construct a term of type
\begin{equation*}
\prd{x:A}{y:B(x)}{p:\id{a}{x}} \pairr{\pairr{a,b},\refl{a}}=\pairr{\pairr{x,y},p}.
\end{equation*}
This term is constructed by path induction on $p$, so it suffices to construct a term of type
\begin{equation*}
\prd{y:B(a)} \pairr{\pairr{a,b},\refl{a}}=\pairr{\pairr{a,y},\refl{a}}
\end{equation*}
However, the proposition $B(a)$ is contractible by \cref{cor:contr_prop}, since we have $b:B(a)$. Therefore we may proceed by singleton induction, so it suffices to construct an identification of type
\begin{equation*}
\pairr{\pairr{a,b},\refl{a}}=\pairr{\pairr{a,b},\refl{a}},
\end{equation*}
which we have by reflexivity. This completes the proof that if each $B(x)$ is a proposition, then the projection map $\proj 1 : \big(\sm{x:A}B(x)\big)\to A$ is an embedding.

For the converse, assume that the projection map is an embedding, and let $x:A$. Our goal is to show that $B(x)$ is a proposition. By \cref{lem:isprop_eq} it suffices to show that
\begin{equation*}
\prd{x:A}{y,z:B(x)} \id{y}{z}
\end{equation*}
Let $y,z:B(x)$. By our assumption that the projection map is an embedding we have an equivalence
\begin{equation*}
\eqv{(\id{\pairr{x,y}}{\pairr{x,z}})}{(\id{x}{x})}
\end{equation*}
In particular, we obtain an identification $p:\id{\pairr{x,y}}{\pairr{x,z}}$ which comes equipped with an identification $q:\ap{\proj 1}{p}=\refl{x}$. Now it follows that
\begin{equation*}
\begin{tikzcd}[column sep=huge]
y \arrow[r,equals,"\apfunc{\mathsf{tr}_B(\blank,y)}(q)"] & \mathsf{tr}_B(p,y) \arrow[r,equals,"\apd{\proj 2}{p}"] & z,
\end{tikzcd}
\end{equation*}
where $\apdfunc{\proj 2}$ is the \emph{dependent} action on paths of the dependent function $\proj 2:\prd{t:\sm{x:A}B(x)} B(\proj 1(t))$, constructed in \cref{defn:apd}.
\end{proof}

\begin{cor}
Let $f:A\to B$ be a map. The following are equivalent:
\begin{enumerate}
\item For each $y:B$, the fiber $\fib{f}{y}$ is a proposition. 
\item $f$ is an embedding.
\end{enumerate}
\end{cor}

\begin{proof}
By \cref{ex:fib_replacement} there is a commuting triangle
\begin{equation*}
\begin{tikzcd}[column sep=large]
A \arrow[rr,"\lam{a}\pairr{f(a),\pairr{a,\refl{f(a)}}}"] \arrow[dr,swap,"f"] & & \sm{y:B}\fib{f}{y} \arrow[dl,"\proj 1"] \\
& B
\end{tikzcd}
\end{equation*}
in which the top map is an equivalence. Thus it follows from \autoref{ex:emb_triangle} that $f$ is an embedding if and only if $\proj 1:\big(\sm{y:B}\fib{f}{y}\big)\to B$ is an embedding. Now the claim follows from \cref{thm:subtype}.
\end{proof}
\end{comment}

\section{Sets}

\begin{defn}
A type $A$ is said to be a \define{set} if there is a term of type
\begin{equation*}
\isset(A)\defeq \prd{x,y:A}\isprop(\id{x}{y}).
\end{equation*}
\end{defn}

\begin{lem}
A type $A$ is a set if and only if it satisfies \define{axiom K}, which asserts that
\begin{equation*}
\prd{x:A}{p:\id{x}{x}}\id{\refl{x}}{p}.
\end{equation*}
\end{lem}

\begin{proof}
If $A$ is a set, then $\id{x}{x}$ is a proposition, so any two of its elements are equal. 
This implies axiom $K$. 

For the converse, if $A$ satisfies axiom $K$, then for any $p,q:\id{x}{y}$ we have $\id{\ct{p}{q^{-1}}}{\refl{x}}$, and hence $\id{p}{q}$. This shows that $\id{x}{y}$ is a proposition, and hence that $A$ is a set.
\end{proof}

\begin{lem}\label{lem:prop_to_id}
Let $A$ be a type, and let $R:A\to A\to\UU$ be a binary relation on $A$ satisfying
\begin{enumerate}
\item Each $R(x,y)$ is a proposition,
\item $R$ is reflexive, as witnessed by $\rho:\prd{x:A}R(x,x)$.
\end{enumerate}
Then any fiberwise map
\begin{equation*}
\prd{x,y:A}R(x,y)\to (\id{x}{y})
\end{equation*}
is a fiberwise equivalence. Consequently, if there is such a fiberwise map, then $A$ is a set.
\end{lem}

\begin{proof}
Let $f:\prd{x,y:A}R(x,y)\to(\id{x}{y})$. 
Since $R$ is assumed to be reflexive, we also have a fiberwise transformation
\begin{equation*}
\rec{x=}(\rho(x)):\prd{y:A}(\id{x}{y})\to R(x,y).
\end{equation*}
Since each $R(x,y)$ is assumed to be a proposition, it therefore follows that each $R(x,y)$ is a retract of $\id{x}{y}$. We conclude by \autoref{ex:id_fundamental_retr} that for each $x,y:A$, the map $f(x,y):R(x,y)\to(\id{x}{y})$ must be an equivalence. 
\end{proof}

\begin{thm}\label{thm:eq_nat}
The type of natural numbers is a set.
\end{thm}

\begin{proof}
We will apply \cref{lem:prop_to_id}. Note that the observational equality $\mathrm{Eq}_\N:\N\to(\N\to\UU)$ on $\N$ (\cref{defn:obs_nat}) is a reflexive relation by \autoref{ex:obs_nat_eqrel}, and moreover that $\mathrm{Eq}_\N(n,m)$ is a proposition for every $n,m:\N$ (proof by double induction).
Therefore it suffices to show that
\begin{equation*}
\prd{m,n:\nat}\mathrm{Eq}_\N(m,n)\to (\id{m}{n}).
\end{equation*}
This follows from the fact that observational equality is the \emph{least} reflexive relation, which was shown in \cref{ex:obs_nat_least}.
\end{proof}

\begin{comment}
\begin{thm}[Hedberg]\label{thm:dec_eq}
Any type with decidable equality is a set.
\end{thm}

\begin{proof}
Let $A$ be a type, and let $d:\prd{x,y:A}(\id{x}{y})+\neg(\id{x}{y})$ be the witness that $A$ has decidable equality.
We first construct a reflexive binary relation $E:A\to A\to\type$ such that each $E(x,y)$ is a proposition.
For every $x,y:A$, we first define a type family $E'(x,y):((\id{x}{y})+\neg(\id{x}{y}))\to\type$ by
\begin{align*}
E'(x,y,\inl(p)) & \defeq \unit \\
E'(x,y,\inr(p)) & \defeq \emptyt.
\end{align*}
Note that $E'(x,y,q)$ is a proposition for each $x,y:A$ and $q:(\id{x}{y})+\neg(\id{x}{y})$. 
Now we set $E(x,y)\defeq E'(x,y,d(x,y))$. Then $E$ is clearly reflexive, and a family of propositions.
Therefore it remains to show that $E$ implies identity. 

Since $E$ is defined as an instance of $E'$, it suffices to construct a term of type
\begin{equation*}
\prd{x,y:A}{q:(\id{x}{y})+\neg(\id{x}{y})} E'(q)\to (\id{x}{y}). 
\end{equation*}
By induction of disjoint sums, it suffices to construct terms of types
\begin{align*}
& \prd{x,y:A}{p:\id{x}{y}} \unit\to (\id{x}{y}) \\
& \prd{x,y:A}{p:\neg(\id{x}{y})} \emptyt\to (\id{x}{y}).
\end{align*}
In the first case, we take $\lam{x}{y}{p}{t}p$, and the second case is by induction on the empty type.
\end{proof}
\end{comment}

\section{General truncation levels}
\begin{defn}
We define $\istrunc{} : \Z_{\geq-2}\to\UU\to\UU$ by induction on $k:\Z_{\geq -2}$, taking
\begin{align*}
\istrunc{-2}(A) & \defeq \iscontr(A) \\
\istrunc{k+1}(A) & \defeq \prd{x,y:A}\istrunc{k}(\id{x}{y}).\qedhere
\end{align*}
For any type $A$, we say that $A$ is \define{$k$-truncated}, or a \define{$k$-type}, if there is a term of type $\istrunc{k}(A)$. We say that a map $f:A\to B$ is $k$-truncated if its fibers are $k$-truncated.
\end{defn}

%For the rest of this section, let $k:\Z_{\geq-2}$.

\begin{thm}
If $A$ is a $k$-type, then $A$ is also a $(k+1)$-type.
\end{thm}

\begin{proof}
We have seen in \cref{eg:prop_contr} that contractible types are propositions. This proves the base case.
For the inductive step, note that if any $k$-type is also a $(k+1)$-type, then any $(k+1)$-type is a $(k+2)$-type, since its identity types are $k$-types and therefore $(k+1)$-types.
\end{proof}

\begin{thm}\label{thm:ktype_eqv}
If $e:\eqv{A}{B}$ is an equivalence, and $B$ is a $k$-type, then so is $A$.
\end{thm}

\begin{proof}
We have seen in \autoref{ex:contr_equiv} that if $B$ is contractible and $e:\eqv{A}{B}$ is an equivalence, then $A$ is also contractible. This proves the base case.

For the inductive step, assume that the $k$-types are stable under equivalences, and consider $e:\eqv{A}{B}$ where $B$ is a $(k+1)$-type. In \autoref{cor:emb_equiv} we have seen that
\begin{equation*}
\apfunc{e}:(\id{x}{y})\to(\id{e(x)}{e(y)})
\end{equation*}
is an equivalence for any $x,y$. Note that $\id{e(x)}{e(y)}$ is a $k$-type, so by the induction hypothesis it follows that $\id{x}{y}$ is a $k$-type. This proves that $A$ is a $(k+1)$-type.
\end{proof}

\begin{cor}
If $f:A\to B$ is an embedding, and $B$ is a $(k+1)$-type, then so is $A$.
\end{cor}

\begin{proof}
By the assumption that $f$ is an embedding, the action on paths
\begin{equation*}
\apfunc{f}:(\id{x}{y})\to (\id{f(x)}{f(y)})
\end{equation*}
is an equivalence for every $x,y:A$. Since $B$ is assumed to be a $(k+1)$-type, it follows that $f(x)=f(y)$ is a $k$-type for every $x,y:A$. Therefore we conclude by \cref{thm:ktype_eqv} that $\id{x}{y}$ is a $k$-type for every $x,y:A$. In other words, $A$ is a $(k+1)$-type.
\end{proof}

In the following definition we generalize the notion of contractible map.

\begin{defn}
We say that a map $f:A\to B$ is \define{$k$-truncated} if for each $y:B$ the fiber $\fib{f}{y}$ is $k$-truncated.
\end{defn}

\begin{thm}
Let $B$ be a type family over $A$. Then the following are equivalent:
\begin{enumerate}
\item For each $x:A$ the type $B(x)$ is $k$-truncated.
\item The projection map
\begin{equation*}
\proj 1 : \Big(\sm{x:A}B(x)\Big)\to A
\end{equation*}
is $k$-truncated.
\end{enumerate}
\end{thm}

\begin{proof}
By \cref{ex:fib_replacement,ex:fiber_trans} we obtain equivalences
\begin{equation*}
\eqv{B(x)}{\fib{\proj 1}{x}}
\end{equation*}
for every $x:A$. Therefore the claim follows from \cref{thm:ktype_eqv}.
\end{proof}

\begin{thm}\label{thm:trunc_ap}
Let $f:A\to B$ be a map. The following are equivalent:
\begin{enumerate}
\item The map $f$ is $(k+1)$-truncated.
\item For each $x,y:A$, the map
\begin{equation*}
\apfunc{f} : (x=y)\to (f(x)=f(y))
\end{equation*}
is $k$-truncated. 
\end{enumerate}
\end{thm}

\begin{proof}
First we show that for any $s,t:\fib{f}{b}$ there is an equivalence
\begin{equation*}
\eqv{(s=t)}{\fib{\apfunc{f}}{\ct{\proj 2(s)}{\proj 2(t)^{-1}}}}
\end{equation*}
We do this by $\Sigma$-induction on $s$ and $t$, and then we calculate using \cref{ex:trans_ap} and basic manipulations of identifications that
\begin{align*}
(\pairr{x,p}=\pairr{y,q}) & \eqvsym \sm{r:x=y} \mathsf{tr}_{f(\blank)=b}(r,p)=q \\
& \eqvsym \sm{r:x=y} \ct{\ap{f}{r}^{-1}}{p}=q \\
& \eqvsym \sm{r:x=y} \ap{f}{r}=\ct{p}{q^{-1}} \\
& \jdeq \fib{\apfunc{f}}{\ct{p}{q^{-1}}}.
\end{align*}
By these equivalences, it follows that if $\apfunc{f}$ is $k$-truncated, then for each $s,t:\fib{f}{b}$ the identity type $s=t$ is equivalent to a $k$-truncated type, and therefore we obtain by \cref{thm:ktype_eqv} that $f$ is $(k+1)$-truncated.

For the converse, note that we have equivalences
\begin{align*}
\fib{\apfunc{f}}{p} & \eqvsym ((x,p)=(y,\refl{f(y)})).
\end{align*}
Therefore it follows that if $f$ is $(k+1)$-truncated, then the identity type $(x,p)=(y,\refl{f(y)})$ in $\fib{f}{f(y)}$ is $k$-truncated for any $p:f(x)=f(y)$, and therefore $\fib{\apfunc{f}}{p}$ is $k$-truncated by \cref{thm:ktype_eqv}. 
\end{proof}

\begin{cor}
A map is an embedding if and only if its fibers are propositions.
\end{cor}

\begin{cor}\label{thm:subtype}
A type family $B$ over $A$ is a subtype if and only if the projection map
\begin{equation*}
\proj 1 : \Big(\sm{x:A}B(x)\Big)\to A
\end{equation*}
is an embedding.
\end{cor}

\begin{thm}
Let $f:\prd{x:A}B(x)\to C(x)$ be a fiberwise transformation. Then the following are equivalent:
\begin{enumerate}
\item For each $x:A$ the map $f(x)$ is $k$-truncated.
\item The induced map 
\begin{equation*}
\total{f}:\Big(\sm{x:A}B(x)\Big)\to\Big(\sm{x:A}C(x)\Big)
\end{equation*}
is $k$-truncated.
\end{enumerate}
\end{thm}

\begin{proof}
This follows directly from \cref{lem:fib_total,thm:ktype_eqv}.
\end{proof}

\begin{comment}
\begin{proof}
By \autoref{ex:contr_retr} it follows that if $A$ is a retract of a contractible type, then $A$ is contractible.
For the inductive step, suppose that the $k$-types are closed under retracts, and consider a section-retraction pair
\begin{equation*}
\begin{tikzcd}
A \arrow[r,"i"] & B \arrow[r,"r"] & A,
\end{tikzcd}
\end{equation*}
with $H:r\circ i\htpy \idfunc$, where $B$ is a $(k+1)$-type.
By the induction hypothesis it suffices to show that for any $x,y:A$, the function $\apfunc{i}:(\id{x}{y})\to (\id{i(x)}{i(y)})$ has a retraction.
The retraction $\varphi:(\id{i(x)}{i(y)})\to(\id{x}{y})$ is defined as
\begin{equation*}
\varphi \defeq \lam{q} \ct{H(x)^{-1}}{\ap{r}{q}}{H(y)}
\end{equation*}
To see that $\varphi(\ap{i}{p})=p$, we have to show that the square
\begin{equation*}
\begin{tikzcd}
r(i(x)) \arrow[d,equals,swap,"\ap{r}{q}"] \arrow[r,equals,"H(x)"] & x \arrow[d,equals,"p"] \\
r(i(y)) \arrow[r,equals,swap,"H(y)"] & y
\end{tikzcd}
\end{equation*}
commutes. This square commutes by the naturality of homotopies, proven in \autoref{ex:htpy_nat}.
\end{proof}
\end{comment}

\begin{exercises}
\item \label{ex:diagonal}Let $A$ be a type, and let the \define{diagonal} of $A$ be the map $\delta_A:A\to A\times A$ given by $\lam{x}(x,x)$. 
\begin{subexenum}
\item Show that
\begin{equation*}
{\isequiv(\delta_A)}\leftrightarrow{\isprop(A)}.
\end{equation*}
\item Construct an equivalence $\eqv{\fib{\delta_A}{(x,y)}}{(x=y)}$ for any $x,y:A$.
\item Show that $A$ is $(k+1)$-truncated if and only if $\delta_A:A\to A\times A$ is $k$-truncated.
\end{subexenum}
\item \label{ex:istrunc_sigma}
\begin{subexenum}
\item Let $B$ be a type family over $A$. Show that if $A$ is a $k$-type, and $B(x)$ is a $k$-type for each $x:A$, then so is $\sm{x:A}B(x)$. Hint: for the base case, use \cref{ex:contr_in_sigma,ex:contr_equiv}.
\item Show that if $A$ and $B$ are $k$-types, then so is $A\times B$.
\end{subexenum}
\item \label{ex:eq_bool}Show that $\bool$ is a set by applying \cref{lem:prop_to_id} with the observational equality on $\bool$ defined in \cref{ex:obs_bool}.
\item Show that for any two sets $A$ and $B$, the disjoint sum $A+B$ is again a set.
\item \label{ex:hedberg}(Hedberg's theorem) A type $A$ is said to have \define{decidable equality} if there is a term of type
\begin{equation*}
\prd{x,y:A} (\id{x}{y})+\neg(\id{x}{y}).
\end{equation*}
For any type $A$, and every $x,y:A$, consider the type family $D(x,y):((\id{x}{y})+\neg(\id{x}{y}))\to\UU$ given by
\begin{align*}
D(x,y,\inl(p)) & \defeq \unit \\
D(x,y,\inr(p)) & \defeq \emptyt.
\end{align*}
Use $D$ to show that any type with decidable equality is a set.
\item Show that $\nat$ and $\bool$ have decidable equality, as defined in \autoref{ex:hedberg}.
\item Show that if $A$ and $B$ have decidable equality, then so do $A+B$ and $A\times B$.
\item Use \autoref{ex:contr_retr,ex:retr_id} to show that if $A$ is a retract of a $k$-type $B$, then $A$ is also a $k$-type.
\end{exercises}

\section{Elementary number theory}

In this chapter our goal is to show how to use the language of type theory to do some elementary number theory. In particular, we define the greatest common divisor of any two numbers, and we use the trial division algorithm to show that being a prime is decidable.

\subsection{Decidability}

A common way of reasoning in mathematics is via a proof by contradiction: ``in order to show that $P$ holds we show that it cannot be the case that $P$ doesn't hold". There are no inference rules in type theory that allow us to obtain a term of type $P$ from a term of type $\neg\neg P$. However, for some propositions $P$ one can construct a function $\neg\neg P \to P$. The \emph{decidable propositions} from a class of such propositions $P$ for which we can show $\neg\neg P \to P$.

\begin{defn}
  A type $A$ is said to be decidable if it comes equipped with a term of type
  \begin{equation*}
    \mathsf{is\usc{}decidable}(A)\defeq A+\neg A.
  \end{equation*}
\end{defn}

\begin{eg}
  The types $\unit$ and $\emptyt$ are decidable. Indeed, we have
  \begin{align*}
    \inl(\ttt) & :\mathsf{is\usc{}decidable}(\unit) \\
    \inr(\idfunc) & : \mathsf{is\usc{}decidable}(\emptyt).\qedhere
  \end{align*}
  Any type $A$ equipped with a point $a:A$ is decidable.
\end{eg}

\begin{lem}
  For each $m,n:\N$, the types $\EqN(m,n)$, $m\leq n$ and $m<n$ are decidable.
\end{lem}

\begin{proof}
  The proofs in each of the three cases is similar, so we only show that $\EqN(m,n)$ is decidable for each $m,n:\N$. This is done by induction on $m$ and $n$. Note that the types
  \begin{align*}
    \EqN(\zeroN,\zeroN) & \jdeq \unit \\
    \EqN(\zeroN,\succN(n)) & \jdeq \emptyt \\
    \EqN(\succN(m),\zeroN) & \jdeq \emptyt 
  \end{align*}
  are all decidable. Moreover, the type $\EqN(\succN(m),\succN(n))\jdeq \EqN(m,n)$ is decidable by the inductive hypothesis.
\end{proof}

Typically we are mostly interested in decidability of propositions. However, we have defined the notion of decidability for general types. Therefore it is the case that the type natural numbers, or indeed any type $A$ that comes equipped with a point $a:A$, is decidable. One reason for defining decidability in this generality is that we can now formulate a theorem that shows that if the identity types of a given type are all decidable, then that type must be a set.

\begin{defn}
  We say that a type $A$ has decidable equality if the identity type $x=y$ is decidable for every $x,y:A$. 
\end{defn}

\begin{lem}
  Equality on the natural numbers is decidable.
\end{lem}

\begin{proof}
  We use the canonical family of maps
  \begin{equation*}
    f : \prd{n:\N} (m= n) \to \mathsf{Eq}_\N(m,n)
  \end{equation*}
  to write the decision algorithm
  \begin{equation*}
    d:\prd{m,n:\N} (m=n)+\neg(m=n).
  \end{equation*}
  by induction on $m,n:\N$. 
\end{proof}

We have already shown in \cref{thm:eq_nat} that the type of natural numbers is a set. In fact, any type with decidable equality is a set.

\begin{thm}[Hedberg]
  Any type with decidable equality is a set.
\end{thm}

\begin{proof}
  Let $A$ be a type, and let
  \begin{equation*}
    d:\prd{x,y:A}(x=y)+\neg(x=y).
  \end{equation*}
  Consider the type family $D(x,y):((\id{x}{y})+\neg(\id{x}{y}))\to\UU$ given by
  \begin{align*}
    D(x,y,\inl(p)) & \defeq \unit \\
    D(x,y,\inr(p)) & \defeq \emptyt.
  \end{align*}
  We use $D$ to show that any type with decidable equality is a set.
\end{proof}

\begin{lem}
  Suppose that $A$ and $B$ are types with decidable equality. Then the coproduct $A+B$ also has decidable equality.
\end{lem}

\begin{cor}
  The type $\Z$ has decidable equality.
\end{cor}

\begin{cor}
  For any $n:\N$ the type $\mathsf{Fin}(n)$ has decidable equality. 
\end{cor}

It should be noted, however, that if $P$ is a decidable subset of $\N$, then it is not necessarily the case that the proposition $\prd{n:\N}P(n)$ is decidable. Intuitively, this is because an algorithm that checks case by case whether $P(n)$ holds only halts when it finds an $n$ for which $P(n)$ doesn't hold. The best we get is the following

\begin{thm}
  Let $n:\N$, and let $P$ be a decidable subset of $\mathsf{Fin}(n)$. Then the proposition
  \begin{equation*}
    \prd{i:\mathsf{Fin}(n)}P(i)
  \end{equation*}
  is decidable. 
\end{thm}

\begin{lem}
  For any decidable subtype $P$ of $\mathsf{Fin}(n)$ there is a function
  \begin{equation*}
    \neg\neg\Big(\sm{i:\mathsf{Fin}(n)}P(i)\Big)\to
    \Big(\sm{i:\mathsf{Fin}(n)}P(i)\Big).
  \end{equation*}
  In other words, any non-empty decidable subset of $\mathsf{Fin}(n)$ is inhabited.
\end{lem}

\subsection{The pigeonhole principle}

The pigeonhole principle states that if we place more than $n$ balls in $n$ bags, then at least one bag will contain more than one ball. In this section we will give a type theoretical proof of the pigeonhole principle.

\begin{thm}\label{thm:pigeonhole}
  For any $m,n:\N$ and any function $f:\mathsf{Fin}(m)\to\mathsf{Fin}(n)$, if $m>n$, then there is an $i:\mathsf{Fin}(n)$ which is in the image of more than one point in $\mathsf{Fin}(m)$.
\end{thm}

\begin{proof}
  The pigeonhole principle is proven by induction on $m,n:\N$. In the base case for $m$ we immediately obtain a contradiction from the assumption that $m>n$. For the inductive step on $m$ and the base case for $n$, we note that $\mathsf{Fin}(\succN(m))\jdeq \mathsf{Fin}(m)+\unit$ and $\mathsf{Fin}(\zeroN)\jdeq \empty$. Therefore $f:\mathsf{Fin}(\succN(m))\to\mathsf{Fin}(\zeroN)$ is a function from a pointed type to the empty type, which gives us a contradiction.

  It remains to give the inductive step for $n$. Let $i\defeq f(\inr(\ttt)):\mathsf{Fin}(\succN(n))$. Since the ordering relation $<$ on $\N$ is decidable, we can decide whether $i$ is in the image of more than one point in $\mathsf{Fin}(m)$. If this is the case, this completes the proof. If this is not the case, note that we have a commuting square
  \begin{equation*}
    \begin{tikzcd}
      \mathsf{Fin}(m) \arrow[r,"{f'}"] \arrow[d,swap,"\inl"] & \mathsf{Fin}(n) \arrow[d,"{\hat{i}}"] \\
      \mathsf{Fin}(\succN(m)) \arrow[r,swap,"f"] & \mathsf{Fin}(\succN(n))
    \end{tikzcd}
  \end{equation*}
  where $\hat{i}$ is the inclusion that omits the value $i$. The function $f':\mathsf{Fin}(m)\to\mathsf{Fin}(n)$ is defined 

  Now note that both the left and right maps in this square are embeddings, and that by the induction hypothesis the pigeonhole principle applies to the function $f':\mathsf{Fin}(m)\to\mathsf{Fin}(n)$. 
\end{proof}

\begin{cor}\label{cor:pigeonhole}
  Given $m>n$, no function $\mathsf{Fin}(m)\to\mathsf{Fin}(n)$ is an embedding.
\end{cor}

It is straightforward to see that the statements of \cref{thm:pigeonhole,cor:pigeonhole} are equivalent, and one might argue that the statement of \cref{cor:pigeonhole} is the more `type theoretical way' of phrasing the pigeonhole principle. However, the relation to counting the number of points that get mapped to 

\begin{thm}\label{thm:generalized-pigeonhole}
  For any $m,n:\N$ and any function $f:\mathsf{Fin}(m)\to\mathsf{Fin}(n)$, if $m>kn$ for some $k:\N$, then there is an $i:\mathsf{Fin}(n)$ which is in the image of more than $k$ points in $\mathsf{Fin}(m)$. 
\end{thm}

\subsection{Defining the greatest common divisor}

\begin{lem}
  For any $d,n:\N$, the type $d\mid n$ is a decidable proposition.
\end{lem}

\begin{lem}
  For any decidable subtype of $\mathsf{Fin}(n)$, if it contains a number $i:\mathsf{Fin}(n)$, then it contains both a minimal and maximal element.
\end{lem}

\begin{defn}
  For any two natural numbers $m,n$ we define the \define{greatest common divisor} $\gcd(m,n)$, which satisfies the following two properties:
  \begin{enumerate}
  \item We have both $\gcd(m,n)\mid m$ and $\gcd(m,n)\mid n$.
  \item For any $d:\N$ we have $d\mid \gcd(m,n)$ if and only if both $d\mid m$ and $d\mid n$ hold.
  \end{enumerate}
\end{defn}

\subsection{The trial division primality test}

\begin{thm}
  For any $n:\N$, the proposition $\mathsf{is\usc{}prime}(n)$ is decidable.
\end{thm}

\subsection{The infinitude of primes}

\begin{thm}
  There are infinitely many primes.
\end{thm}

Some further ideas to include in this chapter:
\begin{enumerate}
\item If $2^n-1$ is prime, then $n$ is prime.
\item Fermat's little theorem.
\end{enumerate}

\begin{exercises}
\item Show that $\mathsf{is\usc{}decidable}(P)$ is a proposition, for any proposition $P$.
\item
  \begin{subexenum}
  \item Show that $\nat$ and $\bool$ have decidable equality. Hint: to show that $\mathbb{N}$ has decidable equality, show first that the successor function is injective.
  \item Show that if $A$ and $B$ have decidable equality, then so do $A+B$ and $A\times B$. Conclude that $\Z$ has decidable equality.
  \item Show that if $A$ is a retract of a type $B$ with decidable equality, then $A$ also has decidable equality.
  \end{subexenum}
\item Define the prime-counting function $\pi:\N\to\N$.
\item (The Cantor-Schr\"oder-Bernstein theorem) Let $X$ and $Y$ be two sets with decidable equality, and consider two maps $f:X\to Y$ and $g:Y\to X$, both of which we assume to be injective. Construct an equivalence $X\simeq Y$.
\item For any $k:\Z$, define a function $i\mapsto i+k \mod n$ of type $\mathsf{Fin}(n)\to\mathsf{Fin}(n)$. Show that this function is an equivalence.
  \item For any $k:\Z$, define a function $i\mapsto i\cdot k \mod n$ of type $\mathsf{Fin}(n)\to\mathsf{Fin}(n)$. Show that this function is an equivalence if and only if $\gcd(n,k)=1$.
\end{exercises}


\chapter{Univalent mathematics}
\section{Function extensionality}
\label{chap:funext}

\subsection{Equivalent forms of function extensionality}
\begin{defn}
  The \define{axiom of function extensionality}\index{function extensionality}\index{identity type!of a Pi-type@{of a $\Pi$-type}}\index{extensionality principle!for functions} asserts that for any type family $B$ over $A$, and any two dependent functions $f,g:\prd{x:A}B(x)$, the canonical map\index{htpy-eq@{$\htpyeq$}}\index{htpy-eq@{$\htpyeq$}!is an equivalence}\index{is an equivalence!htpy-eq@{$\htpyeq$}}
  \begin{equation*}
    \mathsf{htpy\usc{}eq}:(f=g)\to (f\htpy g)
  \end{equation*}
  that sends $\refl{f}$ to $\mathsf{htpy\usc{}refl}_{f}$ is an equivalence. We will write $\mathsf{eq\usc{}htpy}$\index{eq-htpy@{$\mathsf{eq\usc{}htpy}$}} for its inverse, if it is assumed to exist.
\end{defn}

In other words, the axiom of function extensionality asserts that for any two dependent functions $f,g:\prd{x:A}B(x)$, the type of identifications $f=g$ is equivalent to the type of homotopies $f\htpy g$ from $f$ to $g$. By the fundamental theorem of identity types (\cref{thm:id_fundamental})\index{fundamental theorem of identity types} there are three equivalent ways of asserting function extensionality. In the following theorem we state one further equivalent condition.

\begin{thm}\label{thm:funext_wkfunext}
The following are equivalent:\index{function extensionality}
\begin{enumerate}
\item The axion of function extensionality.
\item For any type family $B$ over $A$ and any dependent function $f:\prd{x:A}B(x)$, the total space
  \begin{equation*}
    \sm{g:\prd{x:A}B(x)}f\htpy g
  \end{equation*}
  is contractible.
\item
  The principle of \define{homotopy induction}\index{homotopy induction}\index{induction principle!for homotopies}:
  for any type family $B$ over $A$, any dependent function $f:\prd{x:A}B(x)$, and any family of types $P(g,H)$ indexed by $g:\prd{x:A}B(x)$ and $H:f\htpy g$, the evaluation function
\begin{equation*}
\Big(\prd{g:\prd{x:A}B(x)}{H:f\htpy g}P(g,H)\Big)\to P(f,\mathsf{htpy\usc{}refl}_f)
\end{equation*}
given by $s\mapsto s(f,\mathsf{htpy\usc{}refl}_f)$ has a section.
\item The \define{weak function extensionality principle}\index{weak function extensionality} holds: For every type family $B$ over $A$ one has\index{contractible!weak function extensionality}
\begin{equation*}
\Big(\prd{x:A}\iscontr(B(x))\Big)\to\iscontr\Big(\prd{x:A}B(x)\Big).
\end{equation*}
\end{enumerate}
\end{thm}

\begin{proof}
The fact that function extensionality is equivalent to (ii) and (iii) follows directly from \cref{thm:id_fundamental}.

To show that function extensionality implies weak function extensionality, suppose that each $B(a)$ is contractible with center of contraction $c(a)$ and contraction $C_a:\prd{y:B(a)}c(a)=y$. Then we take $c\defeq \lam{a}c(a)$ to be the center of contraction of $\prd{x:A}B(x)$. To construct the contraction we have to define a term of type
\begin{equation*}
\prd{f:\prd{x:A}B(x)}c=f.
\end{equation*}
Let $f:\prd{x:A}B(x)$. By function extensionality we have a map $(c\htpy f)\to (c=f)$, so it suffices to construct a term of type $c\htpy f$. Here we take $\lam{a}C_a(f(a))$. This completes the proof that function extensionality implies weak function extensionality.

In the remaining part of the proof, we will show that weak function extensionality implies that the type
\begin{equation*}
\sm{g:\prd{x:A}B(x)}f\htpy g
\end{equation*}
is contractible for any $f:\prd{x:A}B(x)$. In order to do this, we first note that we have a section-retraction pair
\begin{equation*}
  \begin{tikzcd}[column sep=small]
    \Big(\sm{g:\prd{x:A}B(x)}f\htpy g\Big) \arrow[r,"i"] & \Big(\prd{x:A}\sm{b:B(x)}f(x)=b\Big) \arrow[r,"r"] & \Big(\sm{g:\prd{x:A}B(x)}f\htpy g\Big).
  \end{tikzcd}
\end{equation*}
Here we have the functions
\begin{align*}
i & \defeq \lam{(g,H)}{x}(g(x),H(x)) \\
r & \defeq \lam{p}\pairr{\lam{x}\proj 1(p(x)),\lam{x}\proj 2(p(x))}.
\end{align*}
Their composite is homotopic to the identity function by the computation rule for $\Sigma$-types and the $\eta$-rule for $\Pi$-types:
\begin{align*}
r(i(g,H)) & \jdeq r(\lam{x}\pairr{g(x),H(x)}) \\
& \jdeq \pairr{\lam{x}g(x),\lam{x}H(x)} \\
& \jdeq \pairr{g,H}.
\end{align*}
Now we observe that the type $\prd{x:A}\sm{b:B(x)}f(x)=b$ is a product of contractible types, so it is contractible by our assumption of the weak function extensionality principle. The claim therefore follows, since retracts of contractible types are contractible by \cref{ex:contr_retr}\index{contractible!retract of}.
\end{proof}

For the remainder of this chapter we will assume that the function extensionality axiom holds\index{axiom!function extensionality}. In \cref{thm:funext-univalence} we will derive function extensionality from the univalence axiom. 

As a first application of the function extensionality axiom we generalize the weak function extensionality axiom to $k$-types.

\begin{thm}\label{thm:trunc_pi}\index{k-type@{$k$-type}}
Assume function extensionality. Then for any type family $B$ over $A$ one has\index{truncated type!closed under Pi@{closed under $\Pi$}}
\begin{equation*}
\Big(\prd{x:A}\istrunc{k}(B(x))\Big)\to \istrunc{k}\Big(\prd{x:A}B(x)\Big).
\end{equation*}
\end{thm}

\begin{proof}
The theorem is proven by induction on $k\geq -2$. The base case is just the weak function extensionality principle\index{weak function extensionality}, which was shown to follow from function extensionality in \cref{thm:funext_wkfunext}.

For the inductive hypothesis, assume that the $k$-types are closed under dependent function types. Assume that $B$ is a family of $(k+1)$-types. By function extensionality, the type $f=g$ is equivalent to $f\htpy g$ for any two dependent functions $f,g:\prd{x:A}B(x)$. Now observe that $f\htpy g$ is a dependent product of $k$-types, and therefore it is an $k$-type by our inductive hypotheses. Therefore, it follows by \cref{thm:ktype_eqv} that $f=g$ is an $k$-type, and hence that $\prd{x:A}B(x)$ is an $(k+1)$-type.
\end{proof}

\begin{cor}\label{cor:funtype_trunc}\index{truncated type!closed under exponentials}
Suppose $B$ is a $k$-type. Then $A\to B$ is also a $k$-type, for any type $A$.
\end{cor}

\subsection{The type theoretic principle of choice}

The type theoretic principle of choice\index{type theoretic choice} asserts that $\Pi$ distributes over $\Sigma$\index{distributivity!of Pi over Sigma@{of $\Pi$ over $\Sigma$}}. More precisely, it asserts that the canonical map\index{choice@{$\mathsf{choice}$}}
\begin{equation*}
  \mathsf{choice}:\Big(\prd{x:A}\sm{y:B(x)}C(x,y)\Big)\to \Big(\sm{f:\prd{x:A}B(x)}\prd{x:A}C(x,f(x))\Big)
\end{equation*}
given by $\lam{h}(\proj 1(h(x)),\proj 2(h(x)))$, is an equivalence. In order to see this as a principle of choice, one can view the left hand side as the type of functions $h$ that pick for every $x:A$ a term $y:B(x)$ equipped with a term of type $C(x,y)$. The function $\mathsf{choice}$ then constructs a dependent function $f:\prd{x:A}B(x)$ equipped with a term of type $\prd{x:A}C(x,f(x))$. In this section we show that the map $\mathsf{choice}$ is an equivalence, and we use this to characterize the identity of any dependent function type $\prd{x:A}B(x)$ in terms of any characterization of the identity types of the individual types $B(x)$.

\begin{thm}\label{thm:choice}
Consider a family of types $C(x,y)$ indexed by $x:A$ and $y:B(x)$. Then the map
\begin{equation*}
  \mathsf{choice}:\Big(\prd{x:A}\sm{y:B(x)}C(x,y)\Big)\to \Big(\sm{f:\prd{x:A}B(x)}\prd{x:A}C(x,f(x))\Big)
\end{equation*}
given by $\lam{h}(\proj 1(h(x)),\proj 2(h(x)))$ is an equivalence.
\end{thm}

\begin{proof}
  We define the map\index{choice-1@{$\mathsf{choice}^{-1}$}}
  \begin{equation*}
    \mathsf{choice}^{-1}:\Big(\sm{f:\prd{x:A}B(x)}\prd{x:A}C(x,f(x))\Big)\to\Big(\prd{x:A}\sm{y:B(x)}C(x,y)\Big)
  \end{equation*}
  by $\lam{(f,g)}{x}(f(x),g(x))$. Then we have to construct homotopies
  \begin{equation*}
    \mathsf{choice}\circ\mathsf{choice}^{-1}\htpy\idfunc,\qquad\text{and}\qquad
    \mathsf{choice}^{-1}\circ\mathsf{choice}\htpy\idfunc.
  \end{equation*}
  For the first homotopy it suffices to construct an identification
  \begin{equation*}
    \mathsf{choice}(\mathsf{choice}^{-1}(f,g))=(f,g)
  \end{equation*}
  for any $f:\prd{x:A}B(x)$ and any $g:\prd{x:A}C(x,f(x))$. We compute the left-hand side as follows:
  \begin{align*}
    \mathsf{choice}(\mathsf{choice}^{-1}(f,g))
    & \jdeq \mathsf{choice}(\lam{x}(f(x),g(x))) \\
    & \jdeq (\lam{x}f(x),\lam{x}g(x)).
  \end{align*}
  By the $\eta$-rule it follows that $f\jdeq \lam{x}f(x)$ and $g\jdeq\lam{x}g(x)$. Therefore we have the identification
  \begin{equation*}
    \refl{(f,g)}:\mathsf{choice}(\mathsf{choice}^{-1}(f,g))=(f,g).
  \end{equation*}
  This completes the construction of the first homotopy.

  For the second homotopy we have to construct an identification
  \begin{equation*}
    \mathsf{choice}^{-1}(\mathsf{choice}(h))=h
  \end{equation*}
  for any $h:\prd{x:A}\sm{y:B(x)}C(x,y)$. We compute the left-hand side as follows:
  \begin{align*}
    \mathsf{choice}^{-1}(\mathsf{choice}(h))
    & \jdeq \mathsf{choice}^{-1}(\lam{x}\proj 1(h(x)),(\lam{x}\proj 2(h(x)))) \\
    & \jdeq \lam{x}(\proj 1(h(x)),\proj 2(h(x)))
  \end{align*}
  However, it is \emph{not} the case that $(\proj 1(h(x)),\proj 2(h(x)))\jdeq h(x)$ for any $h:\prd{x:A}\sm{y:B(x)}C(x,y)$. Nevertheless, we have the identification
  \begin{equation*}
    \mathsf{eq\usc{}pair}(\refl{},\refl{}):(\proj 1(h(x)),\proj 2(h(x)))= h(x).
  \end{equation*}
  Therefore we obtain the required homotopy by function extensionality:
  \begin{equation*}
    \lam{h}\mathsf{eq\usc{}htpy}(\lam{x}\mathsf{eq\usc{}pair}(\refl{\proj 1(h(x))},\refl{\proj 2(h(x))})):\mathsf{choice}^{-1}\circ\mathsf{choice}\htpy\idfunc.\qedhere
  \end{equation*}
\end{proof}

\begin{cor}
For type $A$ and any type family $C$ over $B$, the map
\begin{equation*}
\Big(\sm{f:A\to B} \prd{x:A}C(f(x))\Big)\to\Big(A\to\sm{y:B}C(x)\Big)
\end{equation*}
given by $\lam{(f,g)}{x}(f(x),g(x))$ is an equivalence.
\end{cor}

\begin{rmk}
  The type theoretic choice principle can be used to derive the binomial theorem\index{binomial theorem}. We give an informal argument of how this goes. Recall that the bimomial theorem asserts that
  \begin{equation*}
    (n+m)^k=\sum_{l=0}^k\binom{k}{l}n^l m^{k-l}
  \end{equation*}
  for any three natural numbers $k,m,n$.

  Consider the types $A\defeq\mathsf{Fin}(k)$\index{Fin@{$\mathsf{Fin}$}}, $B\defeq\mathsf{Fin}(n)$ and $C\defeq\mathsf{Fin}(m)$. Then we can define the type family $P:\bool\to\UU$ given by
  \begin{align*}
    P(\btrue) & \defeq B \\
    P(\bfalse) & \defeq C.
  \end{align*}
  Now, the type theoretic principle of choice gives us an equivalence
  \begin{equation*}
    \Big(\prd{x:A}\sm{t:\bool}P(t)\Big)\simeq \Big(\sm{f:A\to\bool}\prd{x:A}P(f(x))\Big).
  \end{equation*}
  Now we note that the type $(f(x)=1)+(f(x)=0)$ is contractible for any $f:A\to\bool$ and $x:A$. Therefore we have equivalences
  \begin{align*}
    \sm{f:A\to\bool}\prd{x:A}P(f(x) & \simeq
    \sm{f:A\to\bool}\prd{x:A}{t:(f(x)=1)+(f(x)=0)}P(f(x)) \\
    & \simeq \sm{f:A\to\bool}(\fib{f}{1}\to B)\times (\fib{f}{0}\to C)
  \end{align*}
  Now we note that, because there are $\binom{k}{l}$\index{binomial coefficient} ways to choose a subset of $l$ elements of $A$, there are
  \begin{equation*}
    \sum_{l=0}^k\binom{k}{l}n^l m^{k-l}
  \end{equation*}
  elements in the above type.
\end{rmk}

\subsection{Universal properties}
The function extensionality principle allows us to prove \emph{universal properties}. Universal properties are characterizations of all maps out of or into a given type, so they are very important. Among other applications, universal properties characterize a type up to equivalence. In the following theorem we prove the universal property of dependent pair types.

\begin{thm}\index{universal property!Sigma-types@{$\Sigma$-types}}\index{Sigma-type@{$\Sigma$-type}!universal property}
Let $B$ be a type family over $A$, and let $X$ be a type. Then the map\index{ev-pair@{$\mathsf{ev\usc{}pair}$}}
\begin{equation*}
\mathsf{ev\usc{}pair}:\Big(\Big(\sm{x:A}B(x)\Big)\to X\Big)\to \Big(\prd{x:A}(B(x)\to X)\Big)
\end{equation*}
given by $f\mapsto\lam{a}{b}f(a,b)$ is an equivalence.
\end{thm}

\begin{proof}
The map in the converse direction is simply
\begin{equation*}
\ind{\Sigma} : \Big(\prd{x:A}(B(x)\to X)\Big)\to \Big(\Big(\sm{x:A}B(x)\Big)\to X\Big).
\end{equation*}
By the computation rules for $\Sigma$-types we have
\begin{equation*}
\lam{f}\refl{f}:\mathsf{ev\usc{}pair}\circ\ind{\Sigma}\htpy\idfunc
\end{equation*}

To show that $\ind{\Sigma}\circ\mathsf{ev\usc{}pair}\htpy\idfunc$ we will also apply function extensionality. Thus, it suffices to show that $\ind{\Sigma}(\lam{x}{y}f((x,y)))=f$. We apply function extensionality again, so it suffices to show that
\begin{equation*}
\prd{t:\sm{x:A}B(x)}\ind{\Sigma}\big(\lam{x}{y}f((x,y))\big)(t)=f(t).
\end{equation*}
We obtain this homotopy by another application of $\Sigma$-induction. 
\end{proof}

\begin{cor}\label{cor:times_up_out}\index{universal property!cartesian product}\index{cartesian product!universal property}
Let $A$, $B$, and $X$ be types. Then the map\index{ev-pair@{$\mathsf{ev\usc{}pair}$}}
\begin{equation*}
\mathsf{ev\usc{}pair}: (A\times B \to X)\to (A\to (B\to X))
\end{equation*}
given by $f\mapsto\lam{a}{b}f((a,b))$ is an equivalence.
\end{cor}

The universal property of identity types is sometimes called the \emph{type theoretical Yoneda lemma}\index{Yoneda lemma (type theoretical)}: families of maps out of the identity type are uniquely determined by their action on the reflexivity identification.

\begin{thm}\label{thm:yoneda}\index{universal property!identity type}\index{identity type!universal property}
Let $B$ be a type family over $A$, and let $a:A$. Then the map\index{ev-refl@{$\mathsf{ev\usc{}refl}$}}
\begin{equation*}
\mathsf{ev\usc{}refl}:\Big(\prd{x:A} (a=x)\to B(x)\Big)\to B(a)
\end{equation*}
given by $\lam{f} f(a,\refl{a})$ is an equivalence. 
\end{thm}

\begin{proof}
The inverse $\varphi$ is defined by path induction, taking $b:B(a)$ to the function $f$ satisfying $f(a,\refl{a})\jdeq b$. It is immediate that $\mathsf{ev\usc{}refl}\circ\varphi\htpy \idfunc$.

To see that $\varphi\circ \mathsf{ev\usc{}refl}\htpy\idfunc$, let $f:\prd{x:A}(a=x)\to B(x)$. To show that $\varphi(f(a,\refl{a}))=f$ we use function extensionality (twice), so it suffices to show that
\begin{equation*}
\prd{x:A}{p:a=x} \varphi(f(a,\refl{a}),x,p)=f(x,p).
\end{equation*}
This follows by path induction on $p$, since $\varphi(f(a,\refl{a}),a,\refl{a})\jdeq f(a,\refl{a})$.
\end{proof}

\subsection{Composing with equivalences}

We show in this section that a map $f:A\to B$ is an equivalence if and only if for any type $X$ the precomposition map\index{precomposition map}
\begin{equation*}
\blank\circ f: (B\to X)\to (A\to X)
\end{equation*}
is an equivalence. Moreover, we will show in \cref{ex:equiv_precomp} that the `dependent version' of this statement also holds: a map $f:A\to B$ is an equivalence if and only if for any type family $P$ over $B$, the precomposition map
\begin{equation*}
\blank\circ f: \Big(\prd{y:B}P(y)\Big)\to\Big(\prd{x:A}P(f(x))\Big)
\end{equation*}
is an equivalence.

\begin{thm}\label{ex:equiv_precomp}\index{equivalence!precomposition}
For any map $f:A\to B$, the following are equivalent:
\begin{enumerate}
\item $f$ is an equivalence.
\item For any type family $P$ over $B$ the map
\begin{equation*}
\Big(\prd{y:B}P(y)\Big)\to\Big(\prd{x:A}P(f(x))\Big)
\end{equation*}
given by $h\mapsto h\circ f$ is an equivalence.
\item For any type $X$ the map
\begin{equation*}
(B\to X)\to (A\to X)
\end{equation*}
given by $g\mapsto g\circ f$ is an equivalence. 
\end{enumerate}
\end{thm}

\begin{proof}
To show that (i) implies (ii), we first recall from \cref{lem:coherently-invertible} that any equivalence is also coherently invertible\index{coherently invertible}. Therefore $f$ comes equipped with
\begin{align*}
g & : B \to A\\
G & : f\circ g \htpy \idfunc[B] \\
H & : g\circ f \htpy \idfunc[A] \\
K & : G\cdot f \htpy f\cdot H.
\end{align*}
Then we define the inverse of $\blank\circ f$ to be the map
\begin{equation*}
\varphi:\Big(\prd{x:A}P(f(x))\Big)\to\Big(\prd{y:B}P(y)\Big)
\end{equation*}
given by $h\mapsto \lam{y}\mathsf{tr}_P(G(y),h(g(y)))$. 

To see that $\varphi$ is a section of $\blank\circ f$, let $h:\prd{x:A}P(f(x))$. By function extensionality it suffices to construct a homotopy $\varphi(h)\circ f\htpy h$. In other words, we have to show that
\begin{equation*}
\mathsf{tr}_P(G(f(x)),h(g(f(x)))=h(x)
\end{equation*}
for any $x:A$. Now we use the additional homotopy $K$ from our assumption that $f$ is coherently invertible. Since we have $K(x):G(f(x))=\ap{f}{H(x)}$ it suffices to show that
\begin{equation*}
\mathsf{tr}_P(\ap{f}{H(x)},hgf(x))=h(x).
\end{equation*}
A simple path-induction argument yields that
\begin{equation*}
\mathsf{tr}_P(\ap{f}{p})\htpy \mathsf{tr}_{P\circ f}(p)
\end{equation*}
for any path $p:x=y$ in $A$, so it suffices to construct an identification
\begin{equation*}
\mathsf{tr}_{P\circ f}(H(x),hgf(x))=h(x).
\end{equation*}
We have such an identification by $\apd{h}{H(x)}$.

To see that $\varphi$ is a retraction of $\blank\circ f$, let $h:\prd{y:B}P(y)$. By function extensionality it suffices to construct a homotopy $\varphi(h\circ f)\htpy h$. In other words, we have to show that
\begin{equation*}
\mathsf{tr}_P(G(y),hfg(y))=h(y)
\end{equation*}
for any $y:B$. We have such an identification by $\apd{h}{G(y)}$. This completes the proof that (i) implies (ii).

Note that (iii) is an immediate consequence of (ii), since we can just choose $P$ to be the constant family $X$.

It remains to show that (iii) implies (i). Suppose that
\begin{equation*}
\blank\circ f:(B\to X)\to (A\to X)
\end{equation*}
is an equivalence for every type $X$. Then its fibers are contractible by \cref{thm:contr_equiv}. In particular, choosing $X\jdeq A$ we see that the fiber
\begin{equation*}
\fib{\blank\circ f}{\idfunc[A]}\jdeq \sm{h:B\to A}h\circ f=\idfunc[A]
\end{equation*}
is contractible. Thus we obtain a function $h:B\to A$ and a homotopy $H:h\circ f\htpy\idfunc[A]$ showing that $h$ is a retraction of $f$. We will show that $h$ is also a section of $f$. To see this, we use that the fiber
\begin{equation*}
\fib{\blank\circ f}{f}\jdeq \sm{i:B\to B} i\circ f=f
\end{equation*}
is contractible (choosing $X\jdeq B$). 
Of course we have $(\idfunc[B],\refl{f})$ in this fiber. However we claim that there also is an identification $p:(f\circ h)\circ f=f$, showing that $(f\circ h,p)$ is in this fiber, because
\begin{align*}
(f\circ h)\circ f & \jdeq f\circ (h\circ f) \\
& = f\circ \idfunc[A] \\
& \jdeq f
\end{align*}
Now we conclude by the contractibility of the fiber that $(\idfunc[B],\refl{f})=(f\circ h,p)$. In particular we obtain that $\idfunc[B]=f\circ h$, showing that $h$ is a section of $f$.
\end{proof}

\begin{exercises}
\item Show that the functions\index{htpy-inv@{$\htpyinv$}!is an equivalence}\index{htpy-concat@{$\htpyconcat$}!is a family of equivalences}\index{htpy-concat'@{$\htpyconcat'$}!is a family of equivalences}\index{is an equivalence!htpy-inv@{$\htpyinv$}}\index{is an equivalence!htpy-concat(H)@{$\htpyconcat(H)$}}\index{is an equivalence!htpy-concat'(K)@{$\htpyconcat'(K)$}}
\begin{align*}
\htpyinv & : (f \htpy g) \to (g \htpy f) \\
\htpyconcat(H) & : (g \htpy h) \to (f \htpy h) \\
\htpyconcat'(K) & : (f \htpy g) \to (f \htpy h)
\end{align*}
are equivalences for every $f,g,h : \prd{x:A}B(x)$. Here, $\htpyconcat'(K)$ is the function defined by $H\mapsto \ct{H}{K}$.
\item \label{ex:isprop_istrunc}
\begin{subexenum}
\item Show that for any type $A$ the type $\iscontr(A)$ is a proposition\index{is-cont(A)r@{$\iscontr(A)$}!is a proposition}\index{is contractible!is a property}. %There's an easy proof using double singleton induction. This is a nice application of weak funext.
\item Show that for any type $A$ and any $k\geq-2$, the type $\istrunc{k}(A)$ is a proposition.\index{istrunc@{$\istrunc{k}$}!is a proposition}
\end{subexenum}
\item \label{lem:postcomp_equiv}
Let $f:X\to Y$ be a map. Show that the following are equivalent:
\begin{enumerate}
\item $f$ is an equivalence.\index{equivalence!postcomposition}
\item The map $f\circ\blank : X^A\to Y^A$ is an equivalence for every type $A$.
\end{enumerate}
\item \label{ex:isprop_isequiv}Let $f:A\to B$ be a function.
\begin{subexenum}
\item Show that if $f$ is an equivalence, then the type $\sm{g:B\to A}f\circ g\htpy \idfunc$ of sections of $f$ is contractible.
\item Show that if $f$ is an equivalence, then the type $\sm{h:B\to A}h\circ f\htpy \idfunc$ of retractions of $f$ is contractible.
\item Show that $\isequiv(f)$ is a proposition.\index{isequiv@{$\isequiv$}!is a proposition}
\item Use \cref{ex:prop_equiv,ex:isprop_istrunc} to show that $\isequiv(f)\eqvsym \iscontr(f)$.\index{isequiv@{$\isequiv$}!isequiv iscontr@{$\isequiv(f)\eqvsym\iscontr(f)$}}
\end{subexenum}
Conclude that $\eqv{A}{B}$ is a subtype of $A\to B$, and in particular that the map $\proj 1 : (\eqv{A}{B})\to (A\to B)$ is an embedding.
\item \label{ex:prop_equiv}
\begin{subexenum}
\item \label{ex:equiv-bi-implication}Let $P$ and $Q$ be propositions. Show that\index{bi-implication}
\begin{equation*}
\eqv{(P\leftrightarrow Q)}{(\eqv{P}{Q})}.
\end{equation*}
\item Show that $P$ is a proposition if and only if $P\to P$ is contractible.
\end{subexenum}
\item Show that $\mathsf{path\usc{}split}(f)$\index{path-split!is a proposition} and $\mathsf{is\usc{}coh\usc{}invertible}(f)$\index{coherently invertible!is a proposition} are propositions for any map $f:A\to B$. Conclude that we have equivalences\index{isequiv@{$\isequiv$}!isequiv path-split@{$\isequiv(f)\eqvsym\mathsf{path\usc{}split}(f)$}}\index{isequiv@{$\isequiv$}!isequiv iscohinvertible@{$\isequiv(f)\eqvsym\mathsf{is\usc{}coh\usc{}invertible}(f)$}}
  \begin{equation*}
    \isequiv(f) \eqvsym \mathsf{path\usc{}split}(f) \eqvsym \mathsf{is\usc{}coh\usc{}invertible}(f).
  \end{equation*}
%\item Let $B$ and $C$ be type families over $A$, suppose that $p:\id{a}{a'}$ in $A$, and consider two functions $f:B(a)\to C(a)$ and $g:B(a')\to C(a')$.
%\begin{subexenum}
%\item Show that the square
%\begin{equation*}
%\begin{tikzcd}
%B(a) \arrow[r,"f"] \arrow[d,swap,"\mathsf{tr}_B(p)"] & C(a) \arrow[d,"\mathsf{tr}_C(p)"] \\
%B(a') \arrow[r,swap,"g"] & C(a')
%\end{tikzcd}
%\end{equation*}
%commutes for every homotopy $H:\mathsf{tr}_{B(x)\to C(x)}(p,f)\htpy g$. In other words, construct a function of type
%\begin{equation*} 
%\Big(\mathsf{tr}_{B(x)\to C(x)}(p,f)\htpy g\Big)\to \Big(\mathsf{tr}_C(p)\circ f\htpy g\circ \mathsf{tr}_B(p)\Big)
%\end{equation*}
%\item Show that this map is an equivalence.
%\end{subexenum}
\item \label{ex:idfunc_autohtpy}Construct for any type $A$ an equivalence\index{isinvertible@{$\mathsf{is\usc{}invertible}$}}
\begin{equation*}
\eqv{\mathsf{is\usc{}invertible}(\idfunc[A])}{\Big(\idfunc[A]\htpy\idfunc[A]\Big)}.
\end{equation*}
Note: We will use this fact in \cref{ex:is_invertible_id_S1} to show that there
are types for which $\mathsf{is\usc{}invertible}(\idfunc[A])\not\eqvsym\isequiv(\idfunc[A])$.
\item 
\begin{subexenum}
\item Show that the type\index{universal property!empty type}\index{empty type!universal property}
\begin{equation*}
\prd{t:\emptyt}P(t)
\end{equation*}
is contractible for any $P:\emptyt\to \UU$.
\item Show that for any type $X$ the following are equivalent:
  \begin{enumerate}
  \item the unique map $\emptyt \to X$ is an equivalence.
  \item The type $Y^X$ is contractible for any type $Y$.
  \end{enumerate}
\end{subexenum}
\item Consider two types $A$ and $B$.\index{universal property!coproduct}\index{coproduct!universal property}
\begin{subexenum}
\item Show that the map\index{ev-inl-inr@{$\mathsf{ev\usc{}inl\usc{}inr}$}}
\begin{equation*}
  \mathsf{ev\usc{}inl\usc{}inr}: \Big(\prd{t:A+B}P(t)\Big) \to
  \Big(\prd{x:A}P(\inl(x))\Big)\times\Big(\prd{y:B}P(\inr(y))\Big)
\end{equation*}
given by $f\mapsto (f\circ\inl,f\circ\inr)$ is an equivalence.
\item Show that the following are equivalent for any type $X$ equipped with maps $i:A\to X$ and $j:B\to X$:
  \begin{enumerate}
  \item The map $\mathsf{ind\usc{}coprod}(i,j) :A+B\to X$ is an equivalence.
  \item For any type $Y$, the map
    \begin{equation*}
      \lam{f}(f\circ i,f\circ j):(X\to Y)\to (A\to Y)\times (B \to Y)
    \end{equation*}
    is an equivalence.
  \end{enumerate}
\end{subexenum}
\item 
\begin{subexenum}
\item Show that the map\index{universal property!unit type}\index{unit type!universal property}\index{ev-pt@{$\mathsf{ev\usc{}pt}$}}
\begin{equation*}
\Big(\prd{t:\unit}P(t)\Big)\to P(\ttt)
\end{equation*}
given by $\lam{f}f(\ttt)$ is an equivalence. 
\item Consider a type $X$ equipped with a point $x:X$. Show that the following are equivalent: 
\begin{enumerate}
\item The map $\mathsf{ind\usc{}unit}(x):\unit\to X$ is an equivalence (i.e., $X$ is contractible).
\item For any type $Y$ the map
\begin{equation*}
\lam{f}f(x) : (X\to Y)\to Y
\end{equation*}
is an equivalence.
\end{enumerate}
\end{subexenum}
\item \label{ex:sec_retr}Consider a commuting triangle 
\begin{equation*}
\begin{tikzcd}[column sep=tiny]
A \arrow[rr,"h"] \arrow[dr,swap,"f"] & & B \arrow[dl,"g"] \\
& X
\end{tikzcd}
\end{equation*}
with $H:f\htpy g\circ h$.
\begin{subexenum}
\item Show that if $h$ has a section, then $\mathsf{sec}(g)$ is a retract of $\mathsf{sec}(f)$.
\item Show that if $g$ has a retraction, then $\mathsf{retr}(h)$ is a retract of $\mathsf{sec}(f)$.
\end{subexenum}
\item \label{ex:equiv_pi}Let $e_i:\eqv{A_i}{B_i}$ be an equivalence for every $i:I$. Show that the map
\begin{equation*}
\lam{f}{i}e_i\circ f:\Big(\prd{i:I}A_i\Big)\to\Big(\prd{i:I}B_i\Big)
\end{equation*}
is an equivalence.
\item \label{ex:triangle_fib}Consider a diagram of the form
\begin{equation*}
\begin{tikzcd}[column sep=tiny]
A \arrow[dr,swap,"f"] & & B \arrow[dl,"g"] \\
& X
\end{tikzcd}
\end{equation*}
\begin{subexenum}
\item Show that the type $\sm{h:A\to B} f\htpy g\circ h$ is equivalent to the type of families of maps
\begin{equation*}
\prd{x:X}\fib{f}{x}\to\fib{g}{x}.
\end{equation*}
\item Show that the type $\sm{h:\eqv{A}{B}} f\htpy g\circ h$ is equivalent to the type of families of equivalences
\begin{equation*}
\prd{x:X}\fib{f}{x}\eqvsym\fib{g}{x}.
\end{equation*}
\end{subexenum}
\item \label{ex:sq_fib}Consider a diagram of the form
\begin{equation*}
\begin{tikzcd}
A \arrow[d,swap,"f"] & B \arrow[d,"g"] \\
X \arrow[r,swap,"h"] & Y.
\end{tikzcd}
\end{equation*}
Show that the type $\sm{i:A\to B}h\circ f\htpy g\circ i$ is equivalent to the type of families of maps
\begin{equation*}
\prd{x:X}\fib{f}{x}\to\fib{g}{h(x)}.
\end{equation*}
Note: In \cref{thm:pb_fibequiv_complete} we will characterize the type of families of equivalences $\prd{x:X}\fib{f}{x}\simeq\fib{g}{x}$.
%\item Show that the type $\sm{i:\eqv{A}{B}}h\circ f\htpy g\circ i$ is equivalent to the type of families of equivalences
%\begin{equation*}
%\prd{x:X}\fib{f}{x}\eqvsym\fib{g}{h(x)}.
%\end{equation*}
\item \label{ex:iso_equiv}Let $A$ and $B$ be sets. Show that type type $\eqv{A}{B}$ of equivalences from $A$ to $B$ is equivalent to the type $A\cong B$ of \define{isomorphisms}\index{isomorphism}\index{set!isomorphism} from $A$ to $B$, i.e., the type of quadruples $(f,g,H,K)$ consisting of
\begin{align*}
f & : A\to B \\
g & : B\to A \\
H & : f\circ g = \idfunc[B] \\
K & : g\circ f = \idfunc[A].
\end{align*}
\item \label{ex:pi_sec}Let $B$ be a type family over $A$, and consider the maps
  \begin{align*}
    \proj 1 & : \sm{x:A} B(x)\to A \\
    \proj1 \circ \blank & : \Big(\sm{x:A} B(x)\Big)^A \to A^A.
  \end{align*}
  Construct equivalences
  \begin{equation*}
    \Big(\prd{x:A}B(x)\Big) \eqvsym \mathsf{sec}(\proj 1) \eqvsym \fib{\proj 1 \circ\blank}{\idfunc[A]}.
  \end{equation*}
\item Suppose that $A:I\to \UU$ is a type family over a set $I$ with decidable equality. Show that
  \begin{equation*}
    \Big(\prd{i:I}\iscontr(A_i)\Big)\leftrightarrow \iscontr\Big(\prd{i:I}A_i\Big).
  \end{equation*}
\item Construct equivalences
  \begin{align*}
    \mathsf{Fin}(n^m) & \simeq (\mathsf{Fin}(m)\to\mathsf{Fin}(n)) \\
    \mathsf{Fin}(n!) & \simeq (\mathsf{Fin}(n)\simeq\mathsf{Fin}(n)).
  \end{align*}
\end{exercises}

\section{The univalence axiom}

\subsection{Type extensionality}

The univalence axiom characterizes the identity type of the universe. Roughly speaking, it asserts that equivalent types are equal. It is considered to be an \emph{extensionality principle}\index{extensionality principle!types} for types. In the following theorem we introduce the univalence axiom and give two more equivalent ways of stating this.

\begin{thm}\label{thm:univalence}
The following are equivalent:
\begin{enumerate}
\item The \define{univalence axiom}\index{univalence axiom|textbf}: for any $A:\UU$ the map\index{equiv_eq@{$\mathsf{equiv\usc{}eq}$}|textbf}
\begin{equation*}
\mathsf{equiv\usc{}eq}\defeq \mathsf{path\usc{}ind}_A(\idfunc[A]) : \prd{B:\UU} (\id{A}{B})\to(\eqv{A}{B}).
\end{equation*}
is a family of equivalences.\index{identity type!universe} If this is the case, we write
$\mathsf{eq\usc{}equiv}$\index{eq equiv@{$\mathsf{eq\usc{}equiv}$}|textbf}
for the inverse of $\mathsf{equiv\usc{}eq}$.
\item The type
\begin{equation*}
\sm{B:\UU}\eqv{A}{B}
\end{equation*}
is contractible for each $A:\UU$.
\item The principle of \define{equivalence induction}\index{equivalence induction}\index{induction principle!for equivalences}: for every $A:\UU$ and for every type family
\begin{equation*}
P:\prd{B:\UU} (\eqv{A}{B})\to \mathsf{Type},
\end{equation*}
the map
\begin{equation*}
\Big(\prd{B:\UU}{e:\eqv{A}{B}}P(B,e)\Big)\to P(A,\idfunc[A])
\end{equation*}
given by $f\mapsto f(A,\idfunc[A])$ has a section.
\end{enumerate}
\end{thm}

\begin{proof}
The equivalence of (i) and (ii) is a direct consequence of \cref{thm:id_fundamental}. 
To see that (ii) and (iii) are equivalent, note that we have a commuting triangle
\begin{equation*}
\begin{tikzcd}[column sep=-1em]
\prd{t:\sm{B:\UU}\eqv{A}{B}}P(t) \arrow[rr,"\mathsf{ev\usc{}pair}"] \arrow[dr,"\varphi","{f\mapsto f((A,\idfunc[A]))}"'] & & \prd{B:\UU}{e:\eqv{A}{B}} P((B,e)) \arrow[dl,"{f\mapsto f(A,\idfunc[A])}","\psi"'] \\
& P((A,\idfunc[A]))
\end{tikzcd}
\end{equation*}
The map $\ind{\Sigma}$ has a section. Therefore it follows from \cref{ex:3_for_2} that $\varphi$ has a section if and only if $\psi$ has a section. By \cref{thm:contractible} it follows that $\varphi$ has a section if and only if $\sm{B:\UU}\eqv{A}{B}$ is contractible. 
\end{proof}

From now on we will assume that the univalence axiom holds.

\subsection{Groups in univalent mathematics}

In this section we exhibit a typical way to use the univalence axiom, showing that isomorphic groups can be identified.
This is an instance of the \emph{structure identity principle}\index{structure identity principle}, which is described in more detail in section 9.8 of \cite{hottbook}.
We will see that in order to establish the fact that isomorphic groups can be identified, it has to be part of the definition of a group that its underlying type is a set. This is an important observation: in many branches of algebra the objects of study are \emph{set-level} structures\footnote{A notable exception is that of categories, which are objects at truncation level $1$, i.e., at the level of \emph{groupoids}. For more on this, see Chapter 9 of \cite{hottbook}.}.

\begin{defn}
A \define{group}\index{group|textbf} $\mathcal{G}$ consists of a type $G$ equipped with
\begin{samepage}
\begin{align*}
p & : \mathsf{is\usc{}set}(G) \\
1 & : G \\
i & : G\to G \\
\mu & : G\to (G\to G),
\end{align*}
\end{samepage}
satisfying the \define{group laws}\index{group laws|textbf}:
\begin{align*}
\mathsf{assoc} & : \prd{x,y,z:G} \mu(\mu(x,y),z)=\mu(x,\mu(y,z)) \\
\mathsf{left\usc{}unit} & : \prd{x:G} \mu(1,x)=x \\
\mathsf{right\usc{}unit} & : \prd{x:G} \mu(x,1)=x \\
\mathsf{left\usc{}inv} & : \prd{x:G} \mu(i(x),x)=1 \\
\mathsf{right\usc{}inv} & : \prd{x:G} \mu(x,i(x))=x.
\end{align*}
The type $\mathsf{Grp}$\index{Grp@{$\mathsf{Grp}$}|textbf} of all small groups is defined as
\begin{align*}
\mathsf{Grp} & \defeq \sm{G:\UU}{p:\mathsf{is\usc{}set}(G)}{1:G}{i:G\to G}{\mu:G\to(G\to G)} \\
& \qquad \Big(\prd{x,y,z:G} \mu(\mu(x,y),z)=\mu(x,\mu(y,z))\Big) \times \\
& \qquad \Big(\prd{x:G} \mu(1,x)=x\Big)\times\Big(\prd{x:G} \mu(x,1)=x\Big) \times \\
& \qquad \Big(\prd{x:G} \mu(i:x,x)=1\Big)\times\Big(\prd{x:G} \mu(x,i(x))=x\Big).
\end{align*}
We will usually write $x^{-1}$ for $i(x)$, and $xy$ for $\mu(x,y)$. The binary operation $\mu$ is also referred to as the \define{group operation}\index{group operation|textbf}.
\end{defn}

\begin{eg}
An important class of examples consists of the \define{loop space}\index{loop space|textbf} $x=x$ of a $1$-type $X$, for any $x:X$. 
We will write $\loopspace{X,x}$ for the loop space of $X$ at $x$. 
Since $X$ is assumed to be a $1$-type, it follows that the type $\loopspace{X,x}$ is a set. Then we have
\begin{align*}
\refl{x} & : \loopspace{X,x} \\
\mathsf{inv} & : \loopspace{X,x} \to \loopspace{X,x} \\
\mathsf{concat} & : \loopspace{X,x} \to (\loopspace{X,x}\to \loopspace{X,x}),
\end{align*}
and these operations satisfy the group laws, since the group laws are just a special case of the groupoid laws for identity types, constructed in \cref{sec:groupoid}.

Using higher inductive types we will show in \cref{chap:image} that \emph{every} group is of this form.
\end{eg}

\begin{eg}
The type $\Z$ of integers\index{Z@{$\Z$}!is a group} can be given the structure of a group, with the group operation being addition. The fact that $\Z$ is a set follows from \cref{thm:eq_nat,ex:set_coprod}. The group laws were shown in \cref{ex:int_group_laws}. 
\end{eg}

\begin{defn}
Let $\mathcal{G}$ and $\mathcal{G}'$ be groups. The type $\mathrm{hom}(\mathcal{G},\mathcal{G}')$ of \define{group homomorphisms}\index{group homomorphism|textbf} from $\mathcal{G}$ to $\mathcal{G}'$ is defined to be the type of pairs $(h,p)$ consisting of
\begin{samepage}
\begin{align*}
h & : G \to G' \\
p & : \prd{x,y:G} f(xy)=f(x)f(y).
\end{align*}
\end{samepage}
\end{defn}

\begin{rmk}
Since preservation of the group operation is a property, we will usually write $h$ for the group homomorphism $(h,p)$.
Moreover, from \cref{thm:subtype} we obtain that the projection map
\begin{equation*}
\proj 1:\mathrm{hom}(\mathcal{G},\mathcal{G}')\to (G\to G')
\end{equation*}
is an embedding. Therefore the equality type $(h,p)=(h',p')$ is equivalent to $h=h'$. In other words, to show that two group homomorphisms are equal it suffices to show that their underlying maps are equal.
\end{rmk}

\begin{lem}
For any two groups $\mathcal{G}$, and $\mathcal{H}$, the type $\mathrm{hom}(\mathcal{G},\mathcal{H})$ is equivalent to the type of quadruples $(f,\alpha,\beta,\gamma)$ consisting of
\begin{align*}
f & : G \to H \\
\alpha & : f(1)=1 \\
\beta & : \prd{x:G}f(x^{-1})=f(x)^{-1} \\
\gamma & : \prd{x,y:G} f(xy)=f(x)f(y).
\end{align*}
\end{lem}

\begin{proof}
It suffices to show that for any group homomorphism $f:\mathrm{hom}(\mathcal{G},\mathcal{H})$, the types $f(1)=1$ and 
\begin{equation*}
\Big(\prd{x:G}f(x^{-1})=f(x)^{-1}\Big)
\end{equation*}
are contractible. Since $G$ is a set, both types are propositions. Therefore it suffices to show they are inhabited. In other words, it suffices to show that any group homomorphism preserves the unit element and inverses. These are just calculations, where each step is an application of a group law:
\begin{align*}
f(1) & = 1f(1)               & f(x^{-1}) & = f(x^{-1})1 \\
     & = (f(1)^{-1}f(1))f(1) &           & = f(x^{-1})(f(x)f(x)^{-1}) \\
     & = f(1)^{-1}(f(1)f(1)) &           & = (f(x^{-1})f(x))f(x)^{-1} \\
     & = f(1)^{-1}f(11)      &           & = f(x^{-1}x)f(x)^{-1} \\
     & = f(1)^{-1}f(1)       &           & = f(1)f(x)^{-1} \\
     & = 1.                  &           & = 1f(x)^{-1} \\
     &                       &           & = f(x)^{-1}.\qedhere
\end{align*}
\end{proof}

\begin{defn}
Let $\mathcal{G}$ be a group. Then the \define{identity homomorphism}\index{group homomorphism!identity homomorphism|textbf} $\idfunc[\mathcal{G}]:\mathrm{hom}(\mathcal{G},\mathcal{G})$ is defined to be the pair $(\idfunc[G],p)$, where 
\begin{equation*}
p(x,y)\defeq \refl{xy}. 
\end{equation*}
\end{defn}

\begin{defn}
Let $h:\mathrm{hom}(\mathcal{G},\mathcal{H})$ and $k:\mathrm{hom}(\mathcal{H},\mathcal{K})$ be group homomorphisms, with proofs $p$ and $q$ that $h$ and $k$ preserve the group operation, respectively. Then we define\index{group homomorphism!composition|textbf}
\begin{equation*}
k\circ h:\mathrm{hom}(\mathcal{G},\mathcal{K})
\end{equation*}
to be the group homomorphism with underlying map $k\circ h$. This map preserves the group operation since
\begin{equation*}
\begin{tikzcd}
k(h(xy)) \arrow[r,equals] & k(h(x)h(y)) \arrow[r,equals] & k(h(x))k(h(y)).
\end{tikzcd}
\end{equation*}
\end{defn}

\begin{defn}
Let $h:\mathrm{hom}(\mathcal{G},\mathcal{H})$ be a group homomorphism. Then $h$ is said to be an \define{isomorphism}\index{group homomorphism!isomorphism}\index{isomorphism!of groups} if there is a group homomorphism $h^{-1}:\mathrm{hom}(\mathcal{H},\mathcal{G})$ such that
\begin{equation*}
h^{-1}\circ h=\idfunc[\mathcal{G}]\qquad\text{and}\qquad h\circ h^{-1}=\idfunc[\mathcal{H}].
\end{equation*}
We write $\mathcal{G}\cong\mathcal{H}$ for the type of all group isomorphisms from $\mathcal{G}$ to $\mathcal{H}$, i.e.,
\begin{equation*}
\mathcal{G}\cong\mathcal{H} \defeq \sm{h:\mathrm{hom}(\mathcal{G},\mathcal{H})}{k:\mathrm{hom}(\mathcal{H},\mathcal{G})} (k\circ h = \idfunc[\mathcal{G}])\times (h\circ k=\idfunc[\mathcal{H}]).
\end{equation*}
\end{defn}

\begin{lem}\label{lem:grp_iso}
The type of isomorphisms $\mathcal{G}\cong\mathcal{H}$ is equivalent to the type
\begin{align*}
e & : \eqv{G}{H} \\
\alpha & : e(1)=1 \\
\beta & : \prd{x:G}e(x^{-1})=e(x)^{-1} \\
\gamma & : \prd{x,y:G} e(xy)=e(x)e(y).
\end{align*}
\end{lem}

\begin{proof}
The standard proof showing that if the underlying map $f:G\to H$ of a group homomorphism is invertible then its inverse is again a group homomorphism, also works in type theory. Since being a group homomorphism is a property, it follows that the type of group isomorphism is equivalent to the type of group homomorphisms of which the underlying map has an inverse. By \cref{ex:iso_equiv} it follows that the type 
\begin{equation*}
\sm{f:\mathrm{hom}(\mathcal{G},\mathcal{H})}\mathsf{is\usc{}invertible}(f)
\end{equation*}
of group homomorphism of which the underlying map has an inverse is equivalent to the type
\begin{equation*}
\sm{f:\mathsf{hom}(\mathcal{G},\mathcal{H})}\isequiv(f).
\end{equation*}
of group homomorphisms of which the underlying map is an equivalence.
\end{proof}

\begin{defn}
Let $\mathcal{G}:\mathsf{Grp}$ be a group. We define the map\index{iso_eq@{$\mathsf{iso\usc{}eq}$}|textbf}
\begin{equation*}
\mathsf{iso\usc{}eq} : \prd{\mathcal{H}:\mathsf{Grp}}(\mathcal{G}=\mathcal{H})\to (\mathcal{G}\cong\mathcal{H})
\end{equation*}
by path induction, taking $\refl{\mathcal{G}}$ to $\idfunc[\mathcal{G}]$. Indeed, $\idfunc[\mathcal{G}]$ is a group isomorphism since it is its own inverse.
\end{defn}

\begin{thm}
The family of maps
\begin{equation*}
\mathsf{iso\usc{}eq} : \prd{\mathcal{G}':\mathsf{Grp}}(\mathcal{G}=\mathcal{G}')\to (\mathcal{G}\cong\mathcal{G}')
\end{equation*}
is a family of equivalences, for any group $\mathcal{G}$.
\end{thm}

\begin{proof}
We will apply \cref{thm:id_fundamental}, and show that the type
\begin{equation*}
\sm{\mathcal{G}':\mathsf{Grp}}\mathcal{G}\cong\mathcal{G}'
\end{equation*}
is contractible.
By \cref{lem:grp_iso} it follows that the total space $\sm{\mathcal{G}':\mathsf{Grp}}(\mathcal{G}\cong\mathcal{G}')$ is equivalent to the type
\begin{samepage}
\begin{align*}
& \sm{G':\UU}{p':\mathsf{is\usc{}set}(G')} \\
& \qquad \sm{1':G}{i':G'\to G'}{\mu':G'\to (G'\to G')}{L':\mathsf{group\usc{}laws}(G',1',i',\mu')} \\
& \qquad \qquad \sm{e:\eqv{G}{G'}} \Big(e(1)=1'\Big) \times \Big(\prd{x:G}e(x^{-1})= i'(e(x))\Big)\times\\
& \qquad \qquad \qquad\Big(\prd{x,y:G}e(xy)=\mu'(e(x),e(y))\Big).
\end{align*}
\end{samepage}%
By the univalence axiom, the type $\sm{G':\UU}\eqv{G}{G'}$ is contractible. Thus we see that the above type is equivalent to
\begin{samepage}
\begin{align*}
& \sm{q:\mathsf{is\usc{}set}(G)}{1':G}{i':G\to G}{\mu':G\to (G\to G)}{L:\mathsf{group\usc{}laws}(G,1',i',\mu')} \\
& \qquad (1=1') \times \Big(\prd{x:G}x^{-1}= i'(x)\Big)\times\Big(\prd{x,y:G}xy=\mu'(x,y)\Big).
\end{align*}
\end{samepage}
Of course, the types
\begin{align*}
& \sm{1':G} 1=1' \\
& \sm{i':G\to G}\prd{x:G}x^{-1}=i'(x) \\
& \sm{\mu':G\to (G\to G)}\Big(\prd{x,y:G}xy=\mu'(x,y)\Big)
\end{align*}
are all contractible. Moreover, being a set is a proposition, and since $G$ is a set the group laws are propositions too. Since $G$ is already assumed to be a set on which the group operations satisfy the group laws, it follows that the types $\mathsf{is\usc{}set}(G)$ and $\mathsf{group\usc{}laws}(G,1,i,\mu)$ are all contractible. This concludes the proof that the total space $\sm{\mathcal{G}':\mathsf{Grp}}\mathcal{G}\cong\mathcal{G}'$ is contractible. 
\end{proof}

\begin{cor}
The type $\mathsf{Grp}$ is a $1$-type.\index{Grp@{$\mathsf{Grp}$}!is a $1$-type|textit}
\end{cor}

\begin{proof}
It is straightforward to see that the type of group isomorphisms $\mathcal{G}\cong\mathcal{H}$ is a set, for any two groups $\mathcal{G}$ and $\mathcal{H}$.
\end{proof}

\subsection{Equivalence relations}

\begin{defn}\label{defn:eq_rel}
Let $R:A\to (A\to\prop)$ be a binary relation valued in the propositions. We say that $R$ is an \define{($0$-)equivalence relation}\index{equivalence relation|textbf}\index{0-equivalence relation|see {equivalence relation}} if $R$ comes equipped with
\begin{align*}
\rho & : \prd{x:A}R(x,x) \\
\sigma & : \prd{x,y:A} R(x,y)\to R(y,x) \\
\tau & : \prd{x,y,z:A} R(x,y)\to (R(y,z)\to R(x,z)).
\end{align*}
Given an equivalence relation $R:A\to (A\to\prop)$, the \define{equivalence class}\index{equivalence class|textbf} $[x]_R$ of $x:A$ is defined to be
\begin{equation*}
[x]_R\defeq R(x).
\end{equation*}
\end{defn}

\begin{defn}
Let $R:A\to (A\to\prop)$ be a $0$-equivalence relation. 
We define for any $x,y:A$ a map\index{class_eq@{$\mathsf{class\usc{}eq}$}|textbf}
\begin{equation*}
\mathsf{class\usc{}eq}:R(x,y)\to ([x]_R=[y]_R).
\end{equation*}
\end{defn}

\begin{proof}[Construction.]
Let $r:R(x,y)$. By function extensionality, the identity type $R(x)=R(y)$ is equivalent to the type
\begin{equation*}
\prd{z:A}R(x,z)=R(y,z).
\end{equation*}
Let $z:A$. By the univalence axiom, the type $R(x,z)=R(y,z)$ is equivalent to the type
\begin{equation*}
\eqv{R(x,z)}{R(y,z)}.
\end{equation*}
We have the map $\tau_{y,x,z}(\sigma(r)):R(x,z)\to R(y,z)$. Since this is a map between propositions, we only have to construct a map in the converse direction to show that it is an equivalence. The map in the converse direction is just $\tau_{x,y,z}(r):R(y,z)\to R(x,z)$. 
\end{proof}

\begin{thm}\label{thm:equivalence_classes}
Let $R:A\to (A\to\prop)$ be a $0$-equivalence relation. 
Then for any $x,y:A$ the map
\begin{equation*}
\mathsf{class\usc{}eq} : R(x,y)\to ([x]_R=[y]_R)
\end{equation*}
is an equivalence.
\end{thm}

\begin{proof}
By the 3-for-2 property of equivalences, it suffices to show that the map
\begin{equation*}
\lam{r}{z}\tau_{y,x,z}(\sigma(r)) : R(x,y)\to \prd{z:A} \eqv{R(x,z)}{R(y,z)}
\end{equation*}
is an equivalence. Since this is a map between propositions, it suffices to construct a map of type
\begin{equation*}
\Big(\prd{z:A} \eqv{R(x,z)}{R(y,z)}\Big)\to R(x,y).
\end{equation*}
This map is simply $\lam{f} \sigma_{y,x}(f_x(\rho(x)))$. 
\end{proof}

\begin{rmk}
By \cref{thm:equivalence_classes} we can begin to think of the \emph{quotient}\index{quotient} $A/R$ of a type $A$ by an equivalence relation $R$. Classically, the quotient is described as the set of equivalence classes, and \cref{thm:equivalence_classes} establishes that two equivalence classes $[x]_R$ and $[y]_R$ are equal precisely when $x$ and $y$ are related by $R$.

However, the type $A\to\prop$ may contain many more terms than just the $R$-equivalence classes. Therefore we are facing the task of finding a type theoretic description of the smallest subtype of $A\to\prop$ containing the equivalence classes.
Another to think about this is as the \emph{image}\index{image} of $R$ in $A\to \prop$. 
The construction of the (homotopy) image of a map can be done with \emph{higher inductive types}\index{higher inductive type}, which we will do in \cref{chap:image}.
\end{rmk}

The notion of $0$-equivalence relation which we defined in \cref{defn:eq_rel} fits in a hierarchy of `$n$-equivalence relations'\index{n-equivalence relation@{$n$-equivalence relation}}, the study of which is a research topic on its own. However, we already know an example of a relation that should count as an `$\infty$-equivalence relation'\index{infinity-equivalence relation@{$\infty$-equivalence relation}}: the identity type. Analogous to \cref{thm:equivalence_classes}, the following theorem shows that the canonical map
\begin{equation*}
(x=y)\to (\idtypevar{A}(x)=\idtypevar{A}(y))
\end{equation*}
is an equivalence, for any $x,y:A$. In other words, $\idtypevar{A}(x)$ can be thought of as the equivalence class of $x$ with respect to the relation $\idtypevar{A}$.

\begin{thm}
Assuming the univalence axiom on $\UU$, the map
\begin{equation*}
\idtypevar{A}:A\to (A\to\UU)
\end{equation*}
is an embedding, for any type $A:\UU$.\index{identity type!is an embedding|textit}
\end{thm}

\begin{proof}
Let $a:A$. By function extensionality it suffices to show that the canonical map
\begin{equation*}
(a=b)\to \idtypevar{A}(a)\htpy\idtypevar{A}(b)
\end{equation*}
that sends $\refl{a}$ to $\lam{x}\refl{(a=x)}$ is an equivalence for every $b:A$, and by univalence it therefore suffices to show that the canonical map
\begin{equation*}
(a=b)\to \prd{x:A}\eqv{(a=x)}{(b=x)}
\end{equation*}
that sends $\refl{a}$ to $\lam{x}\idfunc[(a=x)]$ is an equivalence for every $b:B$. To do this we employ the type theoretic Yoneda lemma, \cref{thm:yoneda}.

By the type theoretic Yoneda lemma\index{Yoneda lemma} we have an equivalence
\begin{equation*}
\Big(\prd{x:A} (b=x)\to (a=x)\Big)\to (a=b)
\end{equation*}
given by $\lam{f} f(b,\refl{b})$, for every $b:A$. Note that any family of maps $\prd{x:A}(b=x)\to (a=x)$ induces an equivalence of total spaces by \cref{ex:contr_equiv}, since their total spaces are are both contractible by \cref{cor:contr_path}. It follows that we have an equivalence
\begin{equation*}
\varphi_b:\Big(\prd{x:A} \eqv{(b=x)}{(a=x)}\Big)\to (a=b)
\end{equation*}
given by $\lam{f} f(b,\refl{b})$, for every $b:A$. 

Note that $\varphi_a(\lam{x}\idfunc[(a=x)])\jdeq\refl{a}$. Therefore it follows by another application of \cref{thm:yoneda} that the unique family of maps 
\begin{equation*}
\alpha_b:(a=b)\to \Big(\prd{x:A} \eqv{(b=x)}{(a=x)}\Big)
\end{equation*}
that satisfies $\alpha_a(\refl{a})=\lam{x}\idfunc[(a=x)]$ is a family of sections of $\varphi$. 
It follows that $\alpha$ is a family of equivalences. Now the proof is completed by reverting the direction of the family of equivalences in the codomain.
\end{proof}

\begin{comment}
\subsection{Univalence implies function extensionality}
The first application of the univalence axiom was Voevodsky's proof of function extensionality. Just for the purpose of the following theorem we drop our assumption of function extensionality. 

\begin{thm}
The univalence axiom implies function extensionality for small types. 
\end{thm}

\begin{proof}
Note that \cref{thm:funext_wkfunext} also holds when it is restricted to small types. 
Therefore it suffices to show that univalence implies the weak principle of function extensionality.

To see this, we first note that for any equivalence $e:\eqv{X}{Y}$, the post-composition map $e\circ \blank:\eqv{(A\to X)}{(A\to Y)}$. This is is obvious in the case $e\jdeq\idfunc[A]$, so it follows from equivalence induction.

Now suppose that $B:A\to \UU$ is a family of contractible types. Our goal is to show that the product $\prd{x:A}B(x)$ is contractible.
Since each $B(x)$ is contractible, the projection map $\proj 1:\big(\sm{x:A}B(x)\big)\to A$ is an equivalence by \cref{ex:proj_fiber}. It follows from \cref{thm:contr_equiv} that the fibers of
\begin{equation*}
\proj 1\circ\blank : \Big(A\to \sm{x:A}B(x)\Big)\to (A\to A)
\end{equation*}
are contractible. In particular, the fiber at $\idfunc[A]$ is contractible. Therefore it suffices to show that $\prd{x:A}B(x)$ is a retract of $\sm{f:A\to\sm{x:A}B(x)}\proj 1\circ f=\idfunc[A]$. In other words, we will construct
\begin{equation*}
\begin{tikzcd}
\Big(\prd{x:A}B(x)\Big) \arrow[r,"i"] & {\sm{f:A\to\sm{x:A}B(x)}\proj 1\circ f=\idfunc[A]} \arrow[r,"r"] & \Big(\prd{x:A}B(x)\Big),
\end{tikzcd}
\end{equation*}
and a homotopy $r\circ i\htpy \idfunc$. We define
\begin{align*}
i & \defeq \lam{f}(\lam{x}(x,f(x)),\lam{x}\refl{x}) \\
r & \defeq \lam{(f,H)}{x}\mathsf{tr}_B(...,\proj 2(f(x)))
\end{align*}
\end{proof}
\end{comment}

\begin{exercises}
\item \label{ex:tr_ap} Show that for any $P:X\to \UU$ and any $p:x=y$ in $X$, we have\index{equiv_eq@{$\mathsf{equiv\usc{}eq}$}}\index{transport}
\begin{equation*}
\mathsf{equiv\usc{}eq}(\ap{P}{p})\htpy \mathsf{tr}_P(p).
\end{equation*}
\item \label{ex:istrunc_UUtrunc}
\begin{subexenum}
\item Use the univalence axiom to show that the type $\sm{A:\UU}\iscontr(A)$ of all contractible types in $\UU$ is contractible.\index{universe!of contractible types}
\item Use \cref{cor:emb_into_ktype,cor:funtype_trunc,ex:isprop_isequiv} to show that if $A$ and $B$ are $(k+1)$-types, then the type $\eqv{A}{B}$ is also a $(k+1)$-type.\index{equiv@{$\eqv{A}{B}$}!truncatedness}
\item Use univalence to show that the universe of $k$-types\index{universe!of k-types@{of $k$-types}}\index{U leq k@{$\UU^{\leq k}$}|textbf}
\begin{equation*}
\UU^{\leq k}\defeq \sm{X:\UU}\mathsf{is\usc{}trunc}_k(X)
\end{equation*}
is a $(k+1)$-type, for any $k\geq -2$.
\item It follows that the universe of propositions $\UU^{\leq-1}$ is a set. However, show that $\UU^{\leq-1}$ is not a proposition.\index{universe!of propositions}
\item Show that $\eqv{(\eqv{\bool}{\bool})}{\bool}$, and conclude by the univalence axiom that the universe of sets\index{universe!of sets} $\UU^{\leq 0}$ is not a set. 
\end{subexenum}
\item Use the univalence axiom to show that the type $\sm{P:\prop}P$ is contractible.
\item Let $A$ and $B$ be small types. 
\begin{subexenum}
\item Construct an equivalence
\begin{equation*}
\eqv{(A\to (B\to\UU))}{\Big(\sm{S:\UU} (S\to A)\times (S\to B)\Big)}
\end{equation*}
\item We say that a relation $R:A\to (B\to\UU)$ is \define{functional}\index{relation!functional} if it comes equipped with a term of type\index{is_function(R)@{$\mathsf{is\usc{}function}(R)$}|textbf}
\begin{equation*}
\mathsf{is\usc{}function}(R) \defeq \prd{x:A}\iscontr\Big(\sm{y:B}R(x,y)\Big)
\end{equation*}
For any function $f:A\to B$, show that the \define{graph}\index{graph!of a function|textbf} of $f$ 
\begin{equation*}
\mathsf{graph}_f:A\to (B\to \UU)
\end{equation*}
given by $\mathsf{graph}_f(a,b)\defeq (f(a)=b)$ is a functional relation from $A$ to $B$.
\item Construct an equivalence
\begin{equation*}
\eqv{\Big(\sm{R:A\to (B\to\UU)}\mathsf{is\usc{}function}(R)\Big)}{(A\to B)}
\end{equation*}
\item Given a relation $R:A\to (B\to \UU)$ we define the \define{opposite relation}\index{relation!opposite relation|textbf}\index{opposite relation|textbf}\index{op R@{$R^{\mathsf{op}}$}|textbf}
\begin{equation*}
R^{\mathsf{op}} : B\to (A\to\UU)
\end{equation*}
by $R^{\mathsf{op}}(y,x)\defeq R(x,y)$. Construct an equivalence\index{equiv@{$\eqv{A}{B}$}!as relation}
\begin{equation*}
\eqv{\Big(\sm{R:A\to (B\to \UU)}\mathsf{is\usc{}function}(R)\times\mathsf{is\usc{}function}(R^{\mathsf{op}})\Big)}{(\eqv{A}{B})}.
\end{equation*}
\end{subexenum}
\item
  \begin{subexenum}
  \item Show that for any proposition $P$, the type
    \begin{equation*}
      P+\neg P
    \end{equation*}
    is again a proposition. We call a proposition $P$ \define{decidable} if it comes equipped with a term of type $P+\neg P$.
  \item Show that the type of decidable propositions is equivalent to $\bool$.
  \end{subexenum}
\end{exercises}

% !TEX root = hott_intro.tex

\chapter{The circle}

We have seen inductive types, in which we describe a type by its constructors and an induction principle that allows us to construct sections of dependent types. Inductive types are freely generated by their constructors, which describe how we can construct their terms. 

However, many familiar constructions in algebra involve the construction of algebras by generators and relations. 
For example, the free abelian group with two generators is described as the group with generators $x$ and $y$, and the relation $xy=yx$. 

In this chapter we introduce higher inductive types, where we follow a similar idea: to allow in the specification of inductive types not only \emph{point constructors}, but also \emph{path constructors} that give us relations between the point constructors. 
The ideas behind the definition of higher inductive types are introduced by studying the simplest non-trivial example: the \emph{circle}.
Moreover, we show that the loop space of the circle is equivalent to $\mathbb{Z}$ by constructing the universal cover of the circle as an application of the univalence axiom. 

\section{The induction principle of the circle}
The \emph{circle}\index{circle} is defined as a higher inductive type\index{higher inductive type} $\sphere{1}$\index{S 1@{$\sphere{1}$}} that comes equipped with\index{base@{$\base$}}\index{loop@{$\lloop$}}
\begin{align*}
\base & : \sphere{1} \\
\lloop & : \id{\base}{\base}.
\end{align*}
Just like for ordinary inductive types, the induction principle for higher inductive types provides us with a way of constructing sections of dependent types. However, we need to take the \emph{path constructor}\index{path constructor} $\lloop$ into account in the induction principle. 

By applying a section $f:\prd{t:\sphere{1}}P(t)$ to the base point of the circle, we obtain a term $f(\base):P(\base)$. Moreover, using the dependent action on paths\index{dependent action on paths} of $f$ of \autoref{defn:apd} we also obtain for any dependent function $f:\prd{t:\sphere{1}}P(t)$ a path
\begin{align*}
\apd{f}{\lloop} & : \id{\mathsf{tr}_P(\lloop,f(\base))}{f(\base)}
\end{align*}
in the fiber $P(\base)$.

\begin{defn}
Let $P$ be a type family over the circle. The \define{dependent action on generators}\index{dependent action on generators!for the circle|textbf} is the map\index{dgen_S1@{$\mathsf{dgen}_{\sphere{1}}$}|textbf}
\begin{equation}\label{eq:dgen_circle}
\mathsf{dgen}_{\sphere{1}}:\Big(\prd{t:\sphere{1}}P(t)\Big)\to\Big(\sm{y:P(\base)}\id{\mathsf{tr}_P(\lloop,y)}{y}\Big)
\end{equation}
given by $\mathsf{dgen}_{\sphere{1}}(f)\defeq\pairr{f(\base),\apd{f}{\lloop}}$.
\end{defn}

We now give the full specification of the circle.

\begin{defn}
The \define{circle}\index{circle|textbf} is a type $\sphere{1}$\index{S 1@{$\sphere{1}$}} that comes equipped with\index{base@{$\base$}}\index{loop@{$\lloop$}}
\begin{align*}
\base & : \sphere{1} \\
\lloop & : \id{\base}{\base},
\end{align*}
and satisfies the \define{induction principle of the circle}\index{induction principle!of the circle}, which provides for each type family $P$ over $\sphere{1}$ a map
\begin{equation*}
\ind{\sphere{1}}:\Big(\sm{y:P(\base)}\id{\mathsf{tr}_P(\lloop,y)}{y}\Big)\to \Big(\prd{t:\sphere{1}}P(t)\Big),
\end{equation*}
and a homotopy witnessing that $\ind{\sphere{1}}$ is a section of $\mathsf{dgen}_{\sphere{1}}$
\begin{equation*}
\mathsf{dgen}_{\sphere{1}}\circ \ind{\sphere{1}}\htpy \idfunc
\end{equation*}
for the computation rule\index{computation rules!of the circle}.
\end{defn}

\begin{rmk}
The induction principle of the circle provides us with a dependent function $f:\prd{t:\sphere{1}}P(t)$ equipped with an identification
\begin{equation*}
(f(\base),\apd{f}{\lloop})=(x,p),
\end{equation*}
for any $x : P(\base)$ and $p : \mathsf{tr}_P(\lloop,x)=x$. By \cref{thm:eq_sigma} the identification
$(f(\base),\apd{f}{\lloop})=(x,p)$ is equivalently described as a pair of identifications
\begin{samepage}
\begin{align*}
\alpha & : f(\base)= x \\
\beta & : \mathsf{tr}(\alpha,\apd{f}{\lloop}) = p.
\end{align*}\end{samepage}%
Here, the transport is taken with respect to the family $x\mapsto \mathsf{tr}_P(\lloop,x)=x$. 

The identity type $\mathsf{tr}(\alpha,\apd{f}{\lloop}) = p$ is equivalent to the type
\begin{equation*}
\ct{\apd{f}{\lloop}}{\alpha}=\ct{\mathsf{ap}_{\mathsf{tr}_P(\lloop)}(\alpha)}{p}.
\end{equation*}
Indeed, such an equivalence can be constructed by path induction, because types reduce to the type $\apd{f}{\lloop}=p$ when $\alpha\jdeq\refl{f(x)}$. Therefore we obtain from the computation rule of the circle an identification $\alpha:f(\base)=x$, and an identification
\begin{equation*}
\beta':\ct{\apd{f}{\lloop}}{\alpha}=\ct{\mathsf{ap}_{\mathsf{tr}_P(\lloop)}(\alpha)}{p}
\end{equation*}
witnessing that the square
\begin{equation*}
\begin{tikzcd}[column sep=huge]
\mathsf{tr}_P(\lloop,f(\base)) \arrow[d,equals,swap,"\apd{f}{\lloop}"] \arrow[r,equals,"\ap{\mathsf{tr}_P(\lloop)}{\alpha}"] & \mathsf{tr}_P(\lloop,x) \arrow[d,equals,"p"] \\
f(\base) \arrow[r,equals,swap,"\alpha"] & x
\end{tikzcd}
\end{equation*}
commutes.
\end{rmk}

\section{The universal property of the circle}

In the following theorem we establish the \define{universal property}\index{universal property!of the circle} of the circle. The proof requires \cref{lem:circle_up_htpy,lem:circle_up_tr_compute}, which we state after we encounter their application.

\begin{thm}\label{thm:circle_up} 
For each type $X$, the \define{action on generators}\index{action on generators!for the circle}\index{gen_S1@{$\mathsf{gen}_{\sphere{1}}$}|textbf}
\begin{equation*}
\mathsf{gen}_{\sphere{1}}:(\sphere{1}\to X)\to \sm{x:X}x=x
\end{equation*}
given by $f\mapsto (f(\base),\ap{f}{\lloop})$ is an equivalence.
\end{thm}

\begin{proof}
Let $x:X$ and let $p:x=x$. By \cref{ex:trans_triv} we have an identification 
\begin{equation*}
\mathsf{tr\usc{}triv}(\lloop,x):\mathsf{tr}_{W_{\sphere{1}}X}(\lloop,x)=x,
\end{equation*}
from which we obtain a fiberwise equivalence
\begin{equation*}
\varphi : \prd{x:X} (x=x) \to (\mathsf{tr}_{W_{\sphere{1}}X}(\lloop,x)=x)
\end{equation*}
given by $p\mapsto \ct{\mathsf{tr\usc{}triv}(\lloop,x)}{p}$.
Moreover, for any $f:A\to B$, and any $p:x=y$ there is an identification $\ct{\mathsf{tr\usc{}triv}(p,f(x))}{\mathsf{ap}_f(p)}=\apd{f}{p}$, so it follows that the triangle
\begin{equation*}
\begin{tikzcd}[column sep=0]
& (\sphere{1}\to X) \arrow[dl,swap,"\mathsf{gen}_{\sphere{1}}"] \arrow[dr,swap,"\mathsf{dgen}_{\sphere{1}}" near start] \\
\sm{x:X}x=x \arrow[rr,"\total{\varphi}"',"\eqvsym"] & & \sm{x:X} \mathsf{tr}_{W_{\sphere{1}}X}(\lloop,x)=x \arrow[ul,densely dotted,bend right=15,swap,"\ind{\sphere{1}}"]
\end{tikzcd}
\end{equation*}
commutes, and the map $\total{\varphi}$ is a fiberwise equivalence by \cref{thm:fib_equiv}. Since the triangle commutes and $\ind{\sphere{1}}$ is a section of $\mathsf{dgen}_{\sphere{1}}$, it follows that the composite
\begin{equation*}
\rec{\sphere{1}}\defeq \ind{\sphere{1}}\circ \total{\varphi}
\end{equation*}
is a section of $\mathsf{gen}_{\sphere{1}}$. Therefore it remains to show that $\rec{\sphere{1}}$ is also a retraction of $\mathsf{gen}_{\sphere{1}}$, i.e.~we have to show that for every $f:\sphere{1}\to X$ there is an identification
\begin{equation*}
\rec{\sphere{1}}(\mathsf{gen}_{\sphere{1}}(f))=f.
\end{equation*}
In \cref{lem:circle_up_htpy} below we establish that
\begin{equation*}
(\mathsf{gen}_{\sphere{1}}(\rec{\sphere{1}}(\mathsf{gen}_{\sphere{1}}(f)))=\mathsf{gen}_{\sphere{1}}(f))\to (\rec{\sphere{1}}(\mathsf{gen}_{\sphere{1}}(f))=f).
\end{equation*}
We get an identification $\mathsf{gen}_{\sphere{1}}(\rec{\sphere{1}}(\mathsf{gen}_{\sphere{1}}(f)))=\mathsf{gen}_{\sphere{1}}(f)$ from the fact that $\rec{\sphere{1}}$ is a section of $\mathsf{gen}_{\sphere{1}}$.
\end{proof}

\begin{lem}\label{lem:circle_up_htpy}
Let $f,g:\sphere{1}\to X$ be two dependent functions. Then there is a map
\begin{equation*}
(\mathsf{gen}_{\sphere{1}}(f)=\mathsf{gen}_{\sphere{1}}(g))\to (f=g)
\end{equation*}
\end{lem}

\begin{proof}
Let $p:\mathsf{gen}_{\sphere{1}}(f)=\mathsf{gen}_{\sphere{1}}(g)$. By function extensionality, it suffices to show that $f\htpy g$. However, since $f\htpy g$ is just the type $\prd{t:\sphere{1}}f(t)=g(t)$, we can construct such a homotopy by $\sphere{1}$-induction. Thus, it suffices to construct a term of type
\begin{equation*}
\sm{p:f(\base)=g(\base)} \mathsf{tr}_{E_{f,g}}(\lloop,p)=p, 
\end{equation*}
where $E_{f,g}$ is the family over $\sphere{1}$ given by $t\mapsto f(t)=g(t)$.

We claim that it suffices to construct for each $p:f(\base)=g(\base)$ an equivalence
\begin{equation*}
\Big(\mathsf{tr}_{E_{f,g}}(\lloop,p)=p\Big)\eqvsym\Big(\mathsf{tr}_{L}(p,\ap{f}{\lloop})=\ap{g}{\lloop}\Big),
\end{equation*}
where $L$ is the family over $X$ given by $x\mapsto x=x$. 
To see that this suffices, we note that such a fiberwise equivalence induces an equivalence on total spaces, and the total space
\begin{align*}
\sm{p:f(\base)=g(\base)} \mathsf{tr}_{L}(p,\ap{f}{\lloop})=\ap{g}{\lloop},
\end{align*}
and is equivalent to $\mathsf{gen}(f)=\mathsf{gen}(g)$, of which we have assumed a term.

The asserted fiberwise equivalence that we need for this proof to go through requires a sufficient generalization so that it can be constructed by path induction, so it is established separately in \cref{lem:circle_up_tr_compute} below.
\end{proof}

\begin{comment}
Consider $f,g:\sphere{1}\to X$ with a homotopy $H:f\htpy g$. Then we have $H(\base):f(\base)=g(\base)$, and the square
\begin{equation*}
\begin{tikzcd}[column sep=large]
f(\base) \arrow[r,equals,"H(\base)"] \arrow[d,swap,equals,"\ap{f}{\lloop}"] & g(\base) \arrow[d,equals,"\ap{g}{\lloop}"] \\
f(\base) \arrow[r,equals,swap,"H(\base)"] & g(\base)
\end{tikzcd}
\end{equation*}
commutes by the naturality of homotopies, established in \cref{defn:htpy_nat}\index{naturality!of homotopies}. In the following lemma we will relate such squares in two ways to a transport, by generalizing the above situation sufficiently so that path induction becomes applicable. We will use these computations of transports to establish the universal property of the circle. 
\end{comment}

With the following lemma we complete the proof of the universal property of the circle. 

\begin{samepage}%
\begin{lem}\label{lem:circle_up_tr_compute} ~
\begin{enumerate}
\item Let $f,g:A \to B$, and let $E_{f,g}$ be the family over $A$ given by 
\begin{equation*}
E_{f,g}(x)\defeq f(x)=g(x).
\end{equation*}
Then for any $p:x=x'$ in $A$ there is an equivalence
\begin{equation*}
\eqv{(\mathsf{tr}_{E_{f,g}}(p,q)=q')}{(\ct{\ap{f}{p}}{q'}=\ct{q}{\ap{g}{p}})}.
\end{equation*}
for any $q:f(x)=g(x)$ and $q':f(x')=g(x')$. In other words, there is an identification $\mathsf{tr}_{E_{f,g}}(p,q)=q'$ if and only if the square
\begin{equation*}
\begin{tikzcd}
f(x) \arrow[r,equals,"q"] \arrow[d,equals,swap,"\ap{f}{p}"] & g(x) \arrow[d,equals,"\ap{g}{p}"] \\
f(x') \arrow[r,equals,swap,"{q'}"] & g(x') 
\end{tikzcd}
\end{equation*}
commutes.
\item Let $L$ be the family over $B$ given by $L(y)\defeq y=y$, and let $q:y=y'$ be an identification in $B$. Then there is an equivalence
\begin{equation*}
\eqv{(\mathsf{tr}_L(q,p)=p')}{(\ct{q}{p'}=\ct{p}{q})}. 
\end{equation*}
for any $p:y=y$ and $p':y'=y'$. In other words, there is an identification $\mathsf{tr}_L(q,p)=p'$ if and only if the square
\begin{equation*}
\begin{tikzcd}
y \arrow[r,equals,"p"] \arrow[d,swap,equals,"q"] & y \arrow[d,equals,"q"] \\
y' \arrow[r,equals,swap,"{p'}"] & y'
\end{tikzcd}
\end{equation*}
commutes.
\item Let $f,g:A \to B$, let $p:x=x$ be a loop in $A$, and let $q:f(x)=g(x)$. Then there is an equivalence
\begin{equation*}
\eqv{(\mathsf{tr}_{E_{f,g}}(p,q)=q)}{(\mathsf{tr}_L(q,\ap{f}{p})=\ap{g}{p}).}
\end{equation*}
\end{enumerate}
\end{lem}
\end{samepage}%

\begin{proof}
The first claim follows by path induction on $p$, and the second claim follows by path induction on $q$. The third claim follows by combining the first two, since the types on both sides are equivalent to the type
\begin{equation*}
\ct{\ap{f}{p}}{q}=\ct{q}{\ap{g}{p}}
\end{equation*}
of witnesses that the square
\begin{equation*}
\begin{tikzcd}[column sep=large]
f(x) \arrow[r,equals,"q"] \arrow[d,swap,equals,"\ap{f}{p}"] & g(x) \arrow[d,equals,"\ap{g}{p}"] \\
f(x) \arrow[r,equals,swap,"q"] & g(x)
\end{tikzcd}
\end{equation*}
commutes.
\end{proof}

\section{Families over the circle}

The type of small families over $\sphere{1}$ is just the function type $\sphere{1}\to\UU$, so in fact we may use the universal property of the circle to construct small dependent types over the circle. 
By the universal property, small type families over $\sphere{1}$ are equivalently described as pairs $(X,p)$ consisting of a type $X:\UU$ and an identification $p:X=X$.
This is where the univalence axiom\index{univalence axiom!families over $\sphere{1}$} comes in. By the map
\begin{equation*}
\mathsf{eq\usc{}equiv}_{X,X}:(\eqv{X}{X})\to (X=X)
\end{equation*}
it suffices to provide an equivalence $\eqv{X}{X}$.

\begin{defn}\label{defn:circle_descent}
Consider a type $X$ and every equivalence $e:\eqv{X}{X}$.
We will construct a dependent type $\mathcal{D}(X,e):\sphere{1}\to\UU$ with an equivalence $x\mapsto x_{\mathcal{D}}:\eqv{X}{\mathcal{D}(X,e,\base)}$ for which the square
\begin{equation*}
\begin{tikzcd}
X \arrow[r,"\eqvsym"] \arrow[d,swap,"e"] & \mathcal{D}(X,e,\base) \arrow[d,"\mathsf{tr}_{\mathcal{D}(X,e)}(\lloop)"] \\
X \arrow[r,swap,"\eqvsym"] & \mathcal{D}(X,e,\base)
\end{tikzcd}
\end{equation*}
commutes. We also write $d\mapsto d_{X}$ for the inverse of this equivalence, so that the relations
\begin{samepage}%
\begin{align*}
(x_{\mathcal{D}})_X & =x & (e(x)_{\mathcal{D}}) & = \mathsf{tr}_{\mathcal{D}(X,e)}(\lloop,x_{\mathcal{D}}) \\
(d_X)_{\mathcal{D}} & =d & (\mathsf{tr}_{\mathcal{D}(X,e)}(d))_X & = e(d_X)
\end{align*}
\end{samepage}%
hold.

The type $\sm{X:\UU}\eqv{X}{X}$ is also called the type of \define{descent data}\index{descent data!for the circle|textbf} for the circle.
\end{defn}

\begin{constr}
By \autoref{ex:tr_ap} we have an identification
\begin{equation*}
\mathsf{equiv\usc{}eq}(\ap{P}{\lloop})=\mathsf{tr}_P(\lloop)
\end{equation*}
for each dependent type $P:\sphere{1}\to\UU$. Therefore we see that the triangle\index{desc_S1@{$\mathsf{desc}_{\sphere{1}}$}}
\begin{equation*}
\begin{tikzcd}
& (\sphere{1}\to \UU) \arrow[dl,swap,"\mathsf{gen}_{\sphere{1}}"] \arrow[dr,"\mathsf{desc}_{\sphere{1}}"] \\
\sm{X:\UU}X=X \arrow[rr,swap,"\total{\lam{X}\mathsf{equiv\usc{}eq}_{X,X}}"] & & \sm{X:\UU}\eqv{X}{X}
\end{tikzcd}
\end{equation*}
commutes, where the map $\mathsf{desc}_{\sphere{1}}$ is given by $P\mapsto\pairr{P(\base),\mathsf{tr}_P(\lloop)}$ and the bottom map is an equivalence by the univalence axiom and \cref{thm:fib_equiv}.
Now it follows by the 3-for-2 property that $\mathsf{desc}_{\sphere{1}}$ is an equivalence, since $\mathsf{gen}_{\sphere{1}}$ is an equivalence by \cref{thm:circle_up}.
This means that for every type $X$ and every $e:\eqv{X}{X}$ there is a type family $\mathcal{D}(X,e):\sphere{1}\to\UU$ such that
\begin{equation*}
\pairr{\mathcal{D}(X,e,\base),\mathsf{tr}_{\mathcal{D}(X,e)}(\lloop)}=\pairr{X,e}.
\end{equation*}
Equivalently, we have $p:\id{\mathcal{D}(X,e,\base)}{X}$ and $\mathsf{tr}(p,{\mathsf{tr}_{\mathcal{D}(X,e)}(\lloop)})=e$. Thus, we obtain $\mathsf{equiv\usc{}eq}(p):\eqv{\mathcal{D}(X,e,\base)}{X}$, for which the square
\begin{equation*}
\begin{tikzcd}[column sep=huge]
\mathcal{D}(X,e,\base)\arrow[r,"\mathsf{equiv\usc{}eq}(p)"] \arrow[d,swap,"\mathsf{tr}_{\mathcal{D}(X,e)}(\lloop)"] & X \arrow[d,"e"] \\
\mathcal{D}(X,e,\base)\arrow[r,swap,"\mathsf{equiv\usc{}eq}(p)"] & X
\end{tikzcd}
\end{equation*}
commutes.
\end{constr}

\begin{comment}
\begin{defn}\label{defn:fiber_sequence}
A \define{fiber sequence} 
\begin{equation*}
F \hookrightarrow E \twoheadrightarrow B
\end{equation*}
consists of a \define{base type} $B$ with a base point $b_0$ and a dependent type $P:B\to\type$, a type $F$ called the \define{fiber} with an equivalence $\eqv{P(b_0)}{F}$, and a type $E$ called the \define{total space} with a map $p:E\to B$ and an equivalence $e:\eqv{(\sm{b:B}P(b))}{E}$ such that the triangle
\begin{equation*}
\begin{tikzcd}
\Big(\sm{b:B}P(b)\Big) \arrow[rr,"e"] \arrow[dr,swap,"\proj 1"] & & E \arrow[dl,"p"] \\
& B
\end{tikzcd}
\end{equation*}
commutes.
\end{defn}
\end{comment}

\section{The fundamental cover of the circle}

The \emph{fundamental cover}\index{fundamental cover!of the circle} of the circle is a family of sets over the circle with contractible total space.
Classically, the fundamental cover is described as a map $\mathbb{R}\to\sphere{1}$ that winds the real line around the circle.
In homotopy type theory there is no analogue of such a construction.

Recall from \cref{ex:succ_equiv} that the successor function $\mathsf{succ}:\Z\to \Z$ is an equivalence. Its inverse is the predecessor function defined in \cref{ex:int_pred}. 

\begin{defn}
The \define{fundamental cover}\index{fundamental cover!of the circle|textbf} of the circle is the dependent type $\mathcal{E}_{\sphere{1}}\defeq\mathcal{D}(\Z,\mathsf{succ}):\sphere{1}\to\UU$.\index{Z@{$\Z$}!fundamental cover of S1@{fundamental cover of $\sphere{1}$}}\index{E_S1@{$\mathcal{E}_{\sphere{1}}$}|textbf}
\end{defn}

The picture of the fundamental cover is that of a helix\index{helix} over the circle.

\begin{lem}
For any $k:\Z$, there is an identification
\begin{equation*}
s_k:(\base,k_{\mathcal{E}})=(\base,\mathsf{succ}(k)_{\mathcal{E}})
\end{equation*}
in the total space $\sm{t:\sphere{1}}\mathcal{E}(t)$.
\end{lem}

\begin{proof}
By \cref{thm:eq_sigma} it suffices to show that
\begin{equation*}
\prd{k:\Z} \sm{\alpha:\base=\base} \mathsf{tr}_{\mathcal{E}}(\alpha,k_{\mathcal{E}})= \mathsf{succ}(k)_{\mathcal{E}}.
\end{equation*}
We just take $\alpha\defeq\lloop$. Then we have $\mathsf{tr}_{\mathcal{E}}(\alpha,k_{\mathcal{E}})= \mathsf{succ}(k)_{\mathcal{E}}$ by the commuting square provided in the definition of $\mathcal{E}$.
\end{proof}

Our goal in this section is to show that the total space of the fundamental cover is contractible. We will use the following elimination principle for the integers.

\begin{lem}
Let $B$ be a family over $\Z$, equipped with a term $b_0:B(0)$, and an equivalence
\begin{equation*}
e_k : B(k)\eqvsym B(\mathsf{succ}(k))
\end{equation*}
for each $k:\Z$. Then there is a dependent function $f:\prd{k:\Z}B(k)$ equipped with identifications $f(0)=b_0$ and
\begin{equation*}
f(\mathsf{succ}(k))=e_k(f(k))
\end{equation*}
for any $k:\Z$.
\end{lem}

\begin{proof}
The map is defined using the induction principle for the integers, stated in \cref{lem:Z_ind}. First we take
\begin{align*}
f(-1) & \defeq e^{-1}(b_0) \\
f(0) & \defeq b_0 \\
f(1) & \defeq e(b_0).
\end{align*}
For the induction step on the negative integers we use
\begin{equation*}
\lam{n}e_{\mathsf{neg}(S(n))}^{-1} : \prd{n:\N} B(\mathsf{neg}(n))\to B(\mathsf{neg}(S(n)))
\end{equation*}
For the induction step on the positive integers we use
\begin{equation*}
\lam{n}e(\mathsf{pos}(n)) : \prd{n:\N} B(\mathsf{pos}(n))\to B(\mathsf{pos}(S(n))).
\end{equation*}
The computation rules follow in a straightforward way from the computation rules of $\Z$-induction and the fact that $e^{-1}$ is an inverse of $e$. 
\end{proof}

\begin{eg}
For any type $A$, we obtain a map $f:\Z\to A$ from any $x:A$ and any equivalence $e:\eqv{A}{A}$, such that $f(0)=x$ and the square
\begin{equation*}
\begin{tikzcd}
\Z \arrow[d,swap,"\mathsf{succ}"] \arrow[r,"f"] & A \arrow[d,"e"] \\
\Z \arrow[r,swap,"f"] & A
\end{tikzcd}
\end{equation*}
commutes. In particular, if we take $A\jdeq (x=x)$ for some $x:X$, then for any $p:x=x$ we have the equivalence $\lam{q}\ct{p}{q}:(x=x)\to (x=x)$. This equivalence induces the map
\begin{equation*}
k\mapsto p^k : \Z \to (x=x).
\end{equation*}
\end{eg}

\begin{thm}\label{thm:circle_fundamental}
The total space $\sm{t:\sphere{1}}\mathcal{E}(t)$ of the fundamental cover of $\sphere{1}$ is contractible.
\end{thm}

\begin{proof}
We show that the total space satisfies singleton induction (i.e.~we apply \cref{thm:contractible}). Let $P$ be a family over the total space of the fundamental cover, and let $p_0:P(\base,0_{\mathcal{E}})$. Our goal is to construct a term of type
\begin{equation*}
\prd{t:\sphere{1}}{x:\mathcal{E}(t)} P(t,x).
\end{equation*}
We do this by induction. For the base case we must construct a term of type
\begin{equation*}
\prd{k:\Z}P(\base,k_{\mathcal{E}}).
\end{equation*}
Since we have the identifications $s_k: (\base,k_{\mathcal{E}})=(\base,\mathsf{succ}(k)_{\mathcal{E}})$, we have the equivalences
\begin{equation*}
\mathsf{tr}_P(s_k) : \eqv{P(\base,k_{\mathcal{E}})}{P(\base,\mathsf{succ}(k)_{\mathcal{E}})}
\end{equation*}
for each $k:\Z$. Thus we obtain a dependent function $f:\prd{x:\mathcal{E}(\base)}P(\base,x)$ satisfying $f(0_{\mathcal{E}})=p_0$ and $f(\mathsf{succ}(k)_{\mathcal{E}})=\mathsf{tr}_P(s_k,f(k_{\mathcal{E}}))$, for each $k:\Z$. 

For the loop case we must show that
\begin{equation*}
\mathsf{tr}_Q(\lloop,f)=f,
\end{equation*}
where $Q$ is the family over $\sphere{1}$ given by $Q(t)\defeq \prd{x:\mathcal{E}(t)} P(t,x)$. By function extensionality it suffices to construct a homotopy, and the transport along $\lloop$ in $Q$ computes as
\begin{equation*}
\mathsf{tr}_Q(\lloop,f)(k_{\mathcal{E}})= \mathsf{tr}_P(s_k,f(\mathsf{succ}^{-1}(k)_{\mathcal{E}})). 
\end{equation*}
Therefore the following computation completes the proof:
\begin{align*}
\mathsf{tr}_Q(\lloop,f)(k_{\mathcal{E}})
& = \mathsf{tr}_P(s_k,f(\mathsf{succ}^{-1}(k)_{\mathcal{E}})) \\
& = f(\mathsf{succ}(\mathsf{succ}^{-1}(k))_{\mathcal{E}}) \\
& = f(k_{\mathcal{E}}).\qedhere
\end{align*}
\end{proof}

\begin{cor}\label{cor:circle_loopspace}
We have a fiber sequence
\begin{equation*}
\Z \hookrightarrow \unit \twoheadrightarrow \sphere{1}.
\end{equation*}
In other words: the loop space $\loopspace{\sphere{1}}$ of the circle is equivalent to $\Z$.\index{identity type!of the circle}
\end{cor}

\begin{proof}
This follows from \cref{thm:circle_fundamental} by an application of \cref{thm:id_fundamental}.
\end{proof}

\begin{cor}
The circle is a $1$-type and it is not a $0$-type.\index{circle!is a 1-type@{is a $1$-type}|textit}
\end{cor}

\begin{exercises}
\item \label{ex:circle_up_pushout}Show that
\begin{equation*}
\begin{tikzcd}[column sep=huge]
X^{\sphere{1}} \arrow[r,"\blank\circ\mathsf{const}_{\base}"] \arrow[d,swap,"\blank\circ\mathsf{const}_{\base}"] & X^\unit \arrow[d,"\blank\circ\mathsf{const}_{\ttt}"] \\
X^\unit \arrow[r,swap,"\blank\circ\mathsf{const}_{\ttt}"] & X^\bool
\end{tikzcd}
\end{equation*}
is a pullback square for each type $X$.
\item \label{ex:circle_dup}In this exercise we establish the \emph{dependent universal property} of the circle, analogous to the proof of \cref{thm:circle_up}.
\begin{subexenum}
\item Let $f,g:\prd{x:A}B(x)$, and let $E_{f,g}$ be the family over $A$ given by 
\begin{equation*}
E_{f,g}(x)\defeq f(x)=g(x).
\end{equation*}
Construct for any $p:x=x'$ in $A$ an equivalence
\begin{equation*}
\eqv{(\mathsf{tr}_{E_{f,g}}(p,q)=q')}{(\ct{\apd{f}{p}}{q'}=\ct{\ap{\mathsf{tr}_B(p)}{q}}{\apd{g}{p}})}.
\end{equation*}
for any $q:f(x)=g(x)$ and $q':f(x')=g(x')$.
\item Let $B$ be a family over $A$, and for $l:x=_A x$ let $L_x$ be the family over $B(x)$ given by 
\begin{equation*}
L_x(y)\defeq \mathsf{tr}_B(l,y)=y.
\end{equation*}
Furthermore, let $q:y=y'$ be an identification in $B(x)$. 
Construct an equivalence
\begin{equation*}
\eqv{(\mathsf{tr}_{L_x}(q,p)=p')}{(\ct{\ap{\mathsf{tr}_B(l)}{q}}{p'}=\ct{p}{q})}. 
\end{equation*}
for any $p:\mathsf{tr}_B(l,y)=y$ and $p':\mathsf{tr}_B(l,y')=y'$.
\item Let $f,g:\prd{x:A}B(x)$, let $p:x=x$ be a loop in $A$, and let $q:f(x)=g(x)$. 
Construct an equivalence
\begin{equation*}
\eqv{(\mathsf{tr}_{E_{f,g}}(p,q)=q)}{(\mathsf{tr}_{L_x}(q,\apd{f}{p})=\apd{g}{p}).}
\end{equation*}
\item Show that for any $f,g:\prd{t:\sphere{1}}P(t)$ there is a function
\begin{equation*}
\Big(\mathsf{dgen}_{\sphere{1}}(f)=\mathsf{dgen}_{\sphere{1}}(g)\Big)\to (f=g).
\end{equation*}
\item Show that for any type family $P$ over $\sphere{1}$, the \emph{dependent action on generators}
\begin{equation*}
\Big(\prd{t:\sphere{1}}P(t)\Big)\to \sm{u:P(\base)}\mathsf{tr}_P(\lloop,u)=u
\end{equation*}
is an equivalence.
\end{subexenum}
\item Let $P:\sphere{1}\to\prop$ be a family of propositions over the circle. Show that
\begin{equation*}
P(\base)\to\prd{t:\sphere{1}}P(t).
\end{equation*}
In this sense the circle is \emph{connected}.
\item Show that
\begin{equation*}
\prd{x,y:\sphere{1}}\neg\neg(x=y).
\end{equation*}
\item Use the fundamental cover of the circle to show that
\begin{equation*}
\neg\Big(\prd{t:\sphere{1}}\base=t\Big).
\end{equation*}
\item \label{ex:circle_constant}
Show that for any type $X$ and any $x:X$, the map
\begin{equation*}
\rec{\sphere{1}}(x,\refl{x}):\sphere{1}\to X
\end{equation*}
is homotopic to the constant map $\mathsf{const}_x$.
\item \label{ex:circle_degk}
\begin{subexenum}
\item Show that for every $x:X$, we have an equivalence
\begin{equation*}
\eqv{\Big(\sm{f:\sphere{1}\to X}f(\base)= x \Big)}{(x=x)}
\end{equation*}
\item Show that for every $t:\sphere{1}$, we have an equivalence
\begin{equation*}
\eqv{\Big(\sm{f:\sphere{1}\to \sphere{1}}f(\base)= t \Big)}{\Z}
\end{equation*}
The base point preserving map $f:\sphere{1}\to\sphere{1}$ corresponding to $k:\Z$ is called the \define{degree $k$ map} on the circle, and is denoted by $\mathsf{deg}(k)$.
\item Show that for every $t:\sphere{1}$, we have an equivalence
\begin{equation*}
\eqv{\Big(\sm{e:\eqv{\sphere{1}}{\sphere{1}}}e(\base)= t \Big)}{\bool}
\end{equation*}
\end{subexenum}
\item \label{ex:circle_double_cover} The \define{(twisted) double cover} of the circle is defined as the type family $\mathcal{T}\defeq\mathcal{D}(\bool,\mathsf{neg}):\sphere{1}\to\UU$, where $\mathsf{neg}:\eqv{\bool}{\bool}$ is the negation equivalence of \autoref{ex:neg_equiv}.
\begin{subexenum}
\item Show that $\neg(\prd{t:\sphere{1}}\mathcal{T}(t))$.
\item Construct an equivalence $e:\eqv{\sphere{1}}{\sm{t:\sphere{1}}\mathcal{T}(t)}$ for which the triangle
\begin{equation*}
\begin{tikzcd}[column sep=tiny]
\sphere{1} \arrow[rr,"e"] \arrow[dr,swap,"\mathsf{deg}(2)"] & & \sm{t:\sphere{1}}\mathcal{T}(t) \arrow[dl,"\proj 1"] \\
\phantom{\sm{t:\sphere{1}}\mathcal{T}(t)} & \sphere{1}
\end{tikzcd}
\end{equation*}
commutes.
\end{subexenum}
\item \label{ex:circle_connected}
\begin{subexenum}
\item Show that a type $X$ is a set if and only if the map
\begin{equation*}
\lam{x}{t} x : X \to (\sphere{1}\to X)
\end{equation*}
is an equivalence.
\item Show that a type $X$ is a set if and only if the map
\begin{equation*}
\lam{f}f(\base) : (\sphere{1}\to X)\to X
\end{equation*}
is an equivalence.
\end{subexenum}
\item Show that $\eqv{(\eqv{\sphere{1}}{\sphere{1}})}{\sphere{1}+\sphere{1}}$. Conclude that a univalent universe containing a circle is not a $1$-type.
\item Show that any retract of the circle is equivalent to the circle.
\item \label{ex:is_invertible_id_S1}
\begin{subexenum}
\item Construct a fiberwise equivalence
\begin{equation*}
\prd{t:\sphere{1}} \big(\eqv{(t=t)}{\Z}\big).
\end{equation*}
\item Use \cref{ex:circle_connected} to show that $\eqv{(\idfunc[\sphere{1}]\htpy\idfunc[\sphere{1}])}{\Z}$.
\item Use \cref{ex:idfunc_autohtpy} to show that
\begin{equation*}
\eqv{\mathsf{is\usc{}invertible}(\idfunc[\sphere{1}])}{\Z},
\end{equation*}
and conclude that ${\mathsf{is\usc{}invertible}}(\idfunc[\sphere{1}])\not\simeq{\isequiv(\idfunc[\sphere{1}])}$. 
\end{subexenum}
\end{exercises}


\chapter{Homotopy pullbacks and pushouts}
% !TEX root = hott_intro.tex

\chapter{Homotopy pullbacks}

Suppose we are given a map $f:A\to B$, and type families $P$ over $A$, and $Q$ over $B$.
Then any fiberwise map
\begin{equation*}
g:\prd{x:A}P(x)\to Q(f(x))
\end{equation*}
gives rise to a commuting square
\begin{equation*}
\begin{tikzcd}[column sep=large]
\sm{x:A}P(x) \arrow[r,"{\total[f]{g}}"] \arrow[d,swap,"\proj 1"] & \sm{y:B}Q(y) \arrow[d,"\proj 1"] \\
A \arrow[r,swap,"f"] & B
\end{tikzcd}
\end{equation*}
where $\total[f]{g}$ is defined as $\lam{(x,p)}(f(x),g(x,y))$. 
We will show in \cref{thm:pb_fibequiv} that $g$ is a fiberwise equivalence\index{fiberwise equivalence} if and only if this square is a \emph{pullback square}\index{pullback square}. This generalization of \cref{thm:fib_equiv} is therefore abstracting away from the notion of fiberwise equivalence, and it serves as our motivating theorem to introduce pullbacks. The connection between pullbacks and fiberwise equivalences has an important role in the descent theorem\index{descent} in \cref{chap:descent}.

\section{Cartesian squares}

Recall that a square
\begin{equation*}
\begin{tikzcd}
C \arrow[r,"q"] \arrow[d,swap,"p"] & B \arrow[d,"g"] \\
A \arrow[r,swap,"f"] & X
\end{tikzcd}
\end{equation*}
is said to \define{commute}\index{commuting square|textbf} if there is a homotopy $H:f\circ p\htpy g\circ q$. 
The pullback property is a \emph{universal property} of the upper left corner of a commuting square (in our case $C$), characterizing the maps \emph{into} it.

To describe the universal property of pullbacks we first need to have a closer look at the \emph{anatomy} of commuting squares.

\begin{defn}\label{defn:cospan}
A commuting square
\begin{equation*}
\begin{tikzcd}
C \arrow[r,"q"] \arrow[d,swap,"p"] & B \arrow[d,"g"] \\
A \arrow[r,swap,"f"] & X
\end{tikzcd}
\end{equation*}
with $H:f\circ p\htpy g\circ q$ can be dissected into three parts, consisting of a \emph{cospan}, a type, and a \emph{cone}, where
\begin{enumerate}
\item A \define{cospan}\index{cospan|textbf} consists of three types $A$, $X$, and $B$, and maps $f:A\to X$ and $g:B\to X$.
\item Given a type $C$, a \define{cone}\index{cone!on a cospan|textbf} on the cospan $A \stackrel{f}{\rightarrow} X \stackrel{g}{\leftarrow} B$ with \define{vertex} $C$\index{vertex!of a cone|textbf} consists of maps $p:C\to A$, $q:C\to B$ and a homotopy $H:f\circ p\htpy g\circ q$. We write\index{cone(C)@{$\mathsf{cone}(\blank)$}|textbf}
\begin{equation*}
\mathsf{cone}(C)\defeq \sm{p:C\to A}{q:C\to B}f\circ p\htpy g\circ q
\end{equation*}
for the type of cones with vertex $C$.
\end{enumerate}
\end{defn}

Given a cone with vertex $C$ on a span $A\stackrel{f}{\rightarrow} X \stackrel{g}{\leftarrow} B$ and a map $h:C'\to C$, we construct a new cone with vertex $C'$ in the following definition.

\begin{defn}
For any cone $(p,q,H)$ with vertex $C$ and any type $C'$, we define a map\index{cone map@{$\mathsf{cone\usc{}map}$}|textbf}
\begin{equation*}
\mathsf{cone\usc{}map}(p,q,H):(C'\to C)\to\mathsf{cone}(C')
\end{equation*}
by $h\mapsto (p\circ h,q\circ h,H\circ h)$. 
\end{defn}

\begin{defn}
We say that a commuting square
\begin{equation*}
\begin{tikzcd}
C \arrow[r,"q"] \arrow[d,swap,"p"] & B \arrow[d,"g"] \\
A \arrow[r,swap,"f"] & X
\end{tikzcd}
\end{equation*}
with $H:f\circ p\htpy g\circ q$ is a \define{pullback square}\index{pullback square|textbf}, or that it is \define{cartesian}\index{cartesian square|textbf}, if it satisfies the \define{universal property} of pullbacks\index{universal property!of pullbacks}, which asserts that the map
\begin{equation*}
\mathsf{cone\usc{}map}(p,q,H):(C'\to C)\to\mathsf{cone}(C')
\end{equation*}
is an equivalence for every type $C'$. 
\end{defn}

We often indicate the universal property with a diagram as follows:
\begin{equation*}
\begin{tikzcd}
C' \arrow[drr,bend left=15,"{q'}"] \arrow[dr,densely dotted,"h"] \arrow[ddr,bend right=15,swap,"{p'}"] \\
& C \arrow[r,"q"] \arrow[d,swap,"p"] & B \arrow[d,"g"] \\
& A \arrow[r,swap,"f"] & X
\end{tikzcd}
\end{equation*}
since the universal property states that for every cone $(p',q',H')$ with vertex $C'$, the type of pairs $(h,\alpha)$ consisting of $h:C'\to C$ equipped with $\alpha:\mathsf{cone\usc{}map}((p,q,H),h)=(p',q',H')$ is contractible by \cref{thm:contr_equiv}.

In order to see what goes on in the universal property of pullbacks, we need to first characterize the identity type of $\mathsf{cone}(C)$, for any type $C$.

\begin{lem}\label{lem:id_cone}%
\index{identity type!of cone@{of $\mathsf{cone}(C)$}|textit}%
Let $(p,q,H)$ and $(p',q',H')$ be cones on a cospan $f:A\rightarrow X \leftarrow B:g$, both with vertex $C$. Then the type $(p,q,H)=(p',q',H')$ is equivalent to the type of triples $(K,L,M)$ consisting of
\begin{align*}
K & : p \htpy p' \\
L & : q \htpy q' \\
M & : \ct{H}{(g\cdot L)} \htpy \ct{(f\cdot K)}{H'}
\end{align*}
\end{lem}

\begin{rmk}
The homotopy $M$ witnesses that the square
\begin{equation*}
\begin{tikzcd}
f\circ p \arrow[r,"f\cdot K"] \arrow[d,swap,"H"] & f\circ p' \arrow[d,"{H'}"] \\
g\circ q \arrow[r,swap,"g\cdot L"] & g\circ q'
\end{tikzcd}
\end{equation*}
of homotopies commutes. Therefore $M$ is a homotopy of homotopies, and for each $z:C$ the identification $M(z)$ witnesses that the square of identifications
\begin{equation*}
\begin{tikzcd}[column sep=huge]
f(p(z)) \arrow[r,equals,"\ap{f}{K(z)}"] \arrow[d,equals,swap,"H(z)"] & f(p'(z)) \arrow[d,equals,"{H'(z)}"] \\
g(q(z)) \arrow[r,equals,swap,"\ap{g}{L(z)}"] & g(q'(z))
\end{tikzcd}
\end{equation*}
commutes. 
\end{rmk}

\begin{proof}[Proof of \cref{lem:id_cone}]
By the fundamental theorem of identity types (\cref{thm:id_fundamental}) and associativity of $\Sigma$-types (\cref{ex:sigma_assoc}) it suffices to show that the type
\begin{equation*}
\sm{p':C\to A}{q':C\to B}{H':f\circ p'\htpy g\circ q'}{K:p\htpy p'}{L:q\htpy q'} \ct{H}{(g\cdot L)} \htpy \ct{(f\cdot K)}{H'}
\end{equation*}
is contractible. Now we apply \cref{ex:sigma_swap} repeatedly to see that this type is equivalent to the type
\begin{equation*}
\sm{p':C\to A}{K: p\htpy p'}{q':C\to B}{L: q\htpy q'}{H':f\circ p'\htpy g\circ q'} \ct{H}{(g\cdot L)} \htpy \ct{(f\cdot K)}{H'}.
\end{equation*}
The types $\sm{p':C\to A} p\htpy p'$ and $\sm{q':C\to B} q\htpy q'$ are contractible by function extensionality, and  we have
\begin{samepage}
\begin{align*}
(p,\mathsf{htpy\usc{}refl}_p) & : \sm{p':C'\to A} p\htpy p' \\
(q,\mathsf{htpy\usc{}refl}_q) & : \sm{q':C'\to B} q\htpy q'.
\end{align*}%
\end{samepage}%
Thus we apply \cref{ex:contr_in_sigma} to see that the type of tuples $(p',K,q',L,H',M)$ is equivalent to the type
\begin{equation*}
\sm{H':f\circ p'\htpy g\circ q'} \ct{H}{\mathsf{htpy\usc{}refl}_{g\circ q}}\htpy \ct{\mathsf{htpy\usc{}refl}_{f\circ p}}{H'}.
\end{equation*}
Of course, the type $\ct{H}{\mathsf{htpy\usc{}refl}_{g\circ q}}\htpy \ct{\mathsf{htpy\usc{}refl}_{f\circ p}}{H'}$ is equivalent to the type $H\htpy H'$, and $\sm{H':f\circ p\htpy g\circ q} H\htpy H'$ is contractible.
\end{proof}

As a corollary we obtain the following characterization of the universal property of pullbacks.

\begin{thm}\label{thm:pullback_up}
Consider a commuting square
\begin{equation*}
\begin{tikzcd}
C \arrow[r,"q"] \arrow[d,swap,"p"] & B \arrow[d,"g"] \\
A \arrow[r,swap,"f"] & X
\end{tikzcd}
\end{equation*}
with $H:f\circ p\htpy g\circ q$
Then the following are equivalent:\index{universal property!of pullbacks (characterization)|textit}
\begin{enumerate}
\item The square is a pullback square.
\item For every type $C'$ and every cone $(p',q',H')$ with vertex $C'$, the type of quadruples $(h,K,L,M)$ consisting of
\begin{align*}
h & : C'\to C \\
K & : p\circ h \htpy p' \\
L & : q\circ h \htpy q' \\
M & : \ct{(H\cdot h)}{(g\cdot L)} \htpy \ct{(f\cdot K)}{H'}
\end{align*}
is contractible.
\end{enumerate}
\end{thm}

\begin{rmk}
The homotopy $M$ in \cref{thm:pullback_up} witnesses that the square
\begin{equation*}
\begin{tikzcd}
f\circ p\circ h \arrow[r,"f\cdot K"] \arrow[d,swap,"H\cdot h"] & f\circ p' \arrow[d,"{H'}"] \\
g\circ q\circ h \arrow[r,swap,"g\cdot L"] & g\circ q'
\end{tikzcd}
\end{equation*}
of homotopies commutes.
\end{rmk}

\section{The unique existence of pullbacks}

\begin{defn}
Let $f:A\to X$ and $B\to X$ be maps. Then we define
\begin{align*}
A\times_X B & \defeq \sm{x:A}{y:B}f(x)=g(y) \\
\pi_1 & \defeq \proj 1 & & : A\times_X B\to A \\
\pi_2 & \defeq \proj 1\circ\proj 2 & & : A\times_X B\to B\\
\pi_3 & \defeq \proj 2\circ\proj 2 & & : f\circ \pi_1 \htpy g\circ\pi_2.
\end{align*}
The type $A\times_X B$ is called the \define{canonical pullback}\index{canonical pullback|textbf} of $f$ and $g$.
\end{defn}

Note that $A\times_X B$ depends on $f$ and $g$, although this dependency is not visible in the notation.

\begin{thm}
Given maps $f:A\to X$ and $g:B\to X$, the commuting square\index{canonical pullback|textit}
\begin{equation*}
\begin{tikzcd}
A\times_X B \arrow[r,"\pi_2"] \arrow[d,swap,"\pi_1"] & B \arrow[d,"g"] \\
A \arrow[r,swap,"f"] & X,
\end{tikzcd}
\end{equation*}
is a pullback square.
\end{thm}

\begin{proof}
Let $C$ be a type. Our goal is to show that the map
\begin{equation*}
\mathsf{cone\usc{}map}(\pi_1,\pi_2,\pi_3): (C\to A\times_X B)\to \mathsf{cone}(C)
\end{equation*}
is an equivalence. 
By double application of \cref{thm:choice} we obtain equivalences
\begin{align*}
(C\to A\times_X B) & \jdeq C\to \sm{x:A}{y:B}f(x)=g(y) \\
& \eqvsym \sm{p:C\to A}\prd{z:C}\sm{y:B} f(p(z))= y \\
& \eqvsym \sm{p:C\to A}{q:C\to B}\prd{z:C} f(p(z))= g(q(z)) \\
& \jdeq \mathsf{cone}(C)
\end{align*}
The composite of these equivalences is the map
\begin{equation*}
\lam{f}(\lam{z}\proj 1(f(z)),\lam{z} \proj 1(\proj 2(f(z))),\lam{z}\proj 2(\proj 2(f(z)))),
\end{equation*}
which is \emph{exactly} the map $\mathsf{cone\usc{}map}(\pi_1,\pi_2,\pi_3)$, and since it is a composite of equivalences it follows that it is itself an equivalence.
\end{proof}

In the following lemma we establish the uniqueness of pullbacks up to equivalence via a \emph{3-for-2 property} for pullbacks.

\begin{lem}\label{lem:pb_3for2}\index{pullback!3-for-2 property|textit}\index{3-for-2 property!of pullbacks|textit}%
Consider the squares
\begin{equation*}
\begin{tikzcd}
C \arrow[r,"q"] \arrow[d,swap,"p"] & B \arrow[d,"g"] & {C'} \arrow[r,"{q'}"] \arrow[d,swap,"{p'}"] & B \arrow[d,"g"] \\
A \arrow[r,swap,"f"] & X & A \arrow[r,swap,"f"] & X
\end{tikzcd}
\end{equation*}
with homotopies $H:f\circ p \htpy g\circ q$ and $H':f\circ p'\htpy g\circ q'$.
Furthermore, suppose we have a map $h:C'\to C$ equipped with
\begin{align*}
K & : p\circ h \htpy p' \\
L & : q\circ h \htpy q' \\
M & : \ct{(H\cdot h)}{(g\cdot L)} \htpy \ct{(f\cdot K)}{H'}.
\end{align*}
If any two of the following three properties hold, so does the third:
\begin{samepage}%
\begin{enumerate}
\item $C$ is a pullback.
\item $C'$ is a pullback.
\item $h$ is an equivalence.
\end{enumerate}%
\end{samepage}%
\end{lem}

\begin{proof}
By the characterization of the identity type of $\mathsf{cone}(C')$ given in \cref{lem:id_cone} we obtain an identification
\begin{equation*}
\mathsf{cone\usc{}map}((p,q,H),h)=(p',q',H')
\end{equation*}
from the triple $(K,L,M)$. 
Let $D$ be a type, and let $k:D\to C'$ be a map. We observe that
\begin{align*}
\mathsf{cone\usc{}map}((p,q,H),(h\circ k)) & \jdeq (p\circ (h\circ k),q\circ (h\circ k),H\circ (h\circ k)) \\
& \jdeq ((p\circ h)\circ k,(q\circ h)\circ k, (H\circ h)\circ k) \\
& \jdeq \mathsf{cone\usc{}map}(\mathsf{cone\usc{}map}((p,q,H),h),k) \\
& = \mathsf{cone\usc{}map}((p',q',H'),k).
\end{align*}
Thus we see that the triangle 
\begin{equation*}
\begin{tikzcd}[column sep=-1em]
(D\to C') \arrow[rr,"{h\circ \blank}"] \arrow[dr,swap,"{\mathsf{cone\usc{}map}(p',q',H')}"] & & (D\to C) \arrow[dl,"{\mathsf{cone\usc{}map}(p,q,H)}"] \\
& \mathsf{cone}(D)
\end{tikzcd}
\end{equation*}
commutes. Therefore it follows from the 3-for-2 property of equivalences established in \cref{ex:3_for_2}, that if any two of the following properties hold, then so does the third:
\begin{enumerate}
\item The map $\mathsf{cone\usc{}map}(p,q,H):(D\to C)\to \mathsf{cone}(D)$ is an equivalence,
\item The map $\mathsf{cone\usc{}map}(p',q',H'):(D\to C')\to \mathsf{cone}(D)$ is an equivalence,
\item The map $h\circ\blank : (D\to C')\to (D\to C)$ is an equivalence.
\end{enumerate}
Thus the 3-for-2 property for pullbacks follows from the fact that $h$ is an equivalence if and only if $h\circ\blank : (D\to C')\to (D\to C)$ is an equivalence for any type $D$, which was established in \cref{lem:postcomp_equiv}.
\end{proof}

Pullbacks are not only unique in the sense that any two pullbacks of the same cospan are equivalent, they are \emph{uniquely unique}\index{uniquely uniqueness!of pullbacks} in the sense that the type of quadruples $(h,K,L,M)$ as in \cref{lem:pb_3for2} is contractible.

\begin{cor}
Suppose both commuting squares
\begin{equation*}
\begin{tikzcd}
C \arrow[r,"q"] \arrow[d,swap,"p"] & B \arrow[d,"g"] & {C'} \arrow[r,"{q'}"] \arrow[d,swap,"{p'}"] & B \arrow[d,"g"] \\
A \arrow[r,swap,"f"] & X & A \arrow[r,swap,"f"] & X
\end{tikzcd}
\end{equation*}
with homotopies $H:f\circ p \htpy g\circ q$ and $H':f\circ p'\htpy g\circ q'$ are pullback squares.
Then the type of quadruples $(e,K,L,M)$ consisting of an equivalence $e:\eqv{C'}{C}$ equipped with
\begin{align*}
K & : p\circ e \htpy p' \\
L & : q\circ e \htpy q' \\
M & : \ct{(g\cdot L)}{(H\cdot e)} \htpy \ct{(f\cdot K)}{H'}.
\end{align*}
is contractible.
\end{cor}

\begin{proof}
We have seen that the type of quadruples $(h,K,L,M)$ is equivalent to the fiber of $\mathsf{cone\usc{}map}(p,q,H)$ at $(p',q',H')$. By \cref{lem:pb_3for2} it follows that $h$ is an equivalence. Since $\isequiv(h)$ is a proposition by \cref{ex:isprop_isequiv}, and hence contractible as soon as it is inhabited, it follows that the type of quadruples $(e,K,L,M)$ is contractible. 
\end{proof}

\begin{defn}
Given a commuting square
\begin{equation*}
\begin{tikzcd}
C \arrow[r,"q"] \arrow[d,"p"] & B \arrow[d,"g"] \\
A \arrow[r,swap,"f"] & X
\end{tikzcd}
\end{equation*}
with $H:f\circ p \htpy g \circ q$, we define the \define{gap map}\index{gap map|textbf}\index{pullback!gap map|textbf}
\begin{equation*}
\mathsf{gap}(p,q,H):C \to A\times_X B
\end{equation*}
by $\lam{z}(p(z),q(z),H(z))$. Furthermore, we will write\index{is_pullback@{$\mathsf{is\usc{}pullback}$}|textbf}
\begin{equation*}
\mathsf{is\usc{}pullback}(f,g,H)\defeq \isequiv(\mathsf{gap}(p,q,H)).
\end{equation*}
\end{defn}

\begin{thm}\label{thm:is_pullback}
Consider a commuting square
\begin{equation*}
\begin{tikzcd}
C \arrow[r,"q"] \arrow[d,"p"] & B \arrow[d,"g"] \\
A \arrow[r,swap,"f"] & X
\end{tikzcd}
\end{equation*}
with $H:f\circ p \htpy g \circ q$. The following are equivalent:
\begin{enumerate}
\item The square is a pullback square
\item There is a term of type
\begin{equation*}
\mathsf{is\usc{}pullback}(p,q,H)\defeq \isequiv(\mathsf{gap}(p,q,H)).
\end{equation*}
\end{enumerate}
\end{thm}

\begin{proof}
Note that there are homotopies
\begin{align*}
K & : \pi_1\circ \mathsf{gap}(p,q,H) \htpy p \\
L & : \pi_2\circ \mathsf{gap}(p,q,H) \htpy q \\
M & : \ct{(\pi_3\cdot \mathsf{gap}(p,q,H))}{(g\cdot L)} \htpy \ct{(f\cdot K)}{H}.
\end{align*}
given by 
\begin{align*}
K & \defeq \lam{z}\refl{p(z)} \\
L & \defeq \lam{z}\refl{q(z)} \\
M & \defeq \lam{z}\ct{\mathsf{right\usc{}unit}(H(z))}{\mathsf{left\usc{}unit}(H(z))^{-1}}.
\end{align*}
Therefore the claim follows by \cref{lem:pb_3for2}.
\end{proof}

\section{Fiber products}

An important special case of pullbacks occurs when the cospan is of the form
\begin{equation*}
\begin{tikzcd}
A \arrow[r] & \unit & B. \arrow[l]
\end{tikzcd}
\end{equation*}
In this case, the pullback is just the \emph{cartesian product}.

\begin{lem}\label{lem:prod_pb}
Let $A$ and $B$ be types. Then the square
\begin{equation*}
\begin{tikzcd}
A\times B \arrow[r,"\proj 2"] \arrow[d,swap,"\proj 1"] & B \arrow[d,"\mathsf{const}_{\ttt}"] \\
A \arrow[r,swap,"\mathsf{const}_{\ttt}"] & \unit
\end{tikzcd}
\end{equation*}
which commutes by the homotopy $\mathsf{const}_{\refl{\ttt}}$ is a pullback square.\index{cartesian product!as pullback}
\end{lem}

\begin{proof}
By \cref{thm:is_pullback} it suffices to show that
\begin{equation*}
\mathsf{gap}(\proj 1,\proj2,\lam{(a,b)}\refl{\ttt})
\end{equation*}
is an equivalence. Its inverse is the map $\lam{(a,b,p)}(a,b)$.
\end{proof}

The following generalization of \cref{lem:prod_pb} is the reason why pullbacks are sometimes called \define{fiber products}\index{fiber product|textbf}.

\begin{thm}
Let $P$ and $Q$ be families over a type $X$. Then the square
\begin{equation*}
\begin{tikzcd}[column sep=8em]
\sm{x:X}P(x)\times Q(x) \arrow[r,"{\lam{(x,(p,q))}(x,q)}"] \arrow[d,swap,"{\lam{(x,(p,q))}(x,p)}"] & \sm{x:X}Q(x) \arrow[d,"\proj 1"] \\
\sm{x:X}P(x) \arrow[r,swap,"\proj 1"] & X,
\end{tikzcd}
\end{equation*}
which commutes by the homotopy
\begin{equation*}
H\defeq \lam{(x,(p,q))}\refl{x},
\end{equation*}
is a pullback square.
\end{thm}

\begin{proof}
By \cref{thm:is_pullback} it suffices to show that the gap map is an equivalence. The gap map is homotopic to the function
\begin{equation*}
\lam{(x,(p,q))}((x,p),(x,q),\refl{x})
\end{equation*}
is an equivalence. The inverse of this function is the map 
\begin{equation*}
\lam{((x,p),(y,q),\alpha)}(y,(\mathsf{tr}_P(\alpha,p),q)).\qedhere
\end{equation*}
\end{proof}

\begin{cor}
For any $f:A\to X$ and $g:B\to X$, the square
\begin{equation*}
\begin{tikzcd}[column sep=8em]
\sm{x:X}\fib{f}{x}\times\fib{g}{y} \arrow[r,"{\lam{(x,((a,p),(b,q)))}b}"] \arrow[d,swap,"{\lam{(x,((a,p),(b,q)))}a}"] & B \arrow[d,"g"]  \\
A \arrow[r,swap,"f"] & X
\end{tikzcd}
\end{equation*}
is a pullback square.
\end{cor}

\section{Fibers as pullbacks}

\begin{lem}\label{lem:fib_pb}
For any function $f:A\to B$, and any $b:B$, consider the square
\begin{equation*}
\begin{tikzcd}[column sep=large]
\fib{f}{b} \arrow[r,"\mathsf{const}_\ttt"] \arrow[d,swap,"\proj 1"] & \unit \arrow[d,"\mathsf{const}_b"] \\
A \arrow[r,swap,"f"] & B
\end{tikzcd}
\end{equation*}
which commutes by $\proj 2 : \prd{t:\fib{f}{b}} f(\proj 1(t))=b$. This is a pullback square.\index{fiber!as pullback|textit}
\end{lem}

\begin{proof}
By \cref{thm:is_pullback} it suffices to show that the gap map is an equivalence. The gap map is homotopic to the function
\begin{equation*}
\mathsf{total}(\lam{x}{p}(\ttt,p))
\end{equation*}
The map $\lam{x}{p}(\ttt,p)$ is a fiberwise equivalence by \cref{ex:contr_in_sigma}, so it induces an equivalence on total spaces by \cref{thm:fib_equiv}.
\end{proof}

\begin{cor}
For any type family $B$ over $A$ and any $a:A$ the square
\begin{equation*}
\begin{tikzcd}[column sep=large]
B(a) \arrow[d,swap,"{\lam{y}(a,y)}"] \arrow[r,"\mathsf{const}_\ttt"] & \unit \arrow[d,"\lam{\ttt}a"] \\
\sm{x:A}B(x) \arrow[r,swap,"\proj 1"] & A
\end{tikzcd}
\end{equation*}
is a pullback square.
\end{cor}

\begin{proof}
  To see this, note that the triangle
  \begin{equation*}
    \begin{tikzcd}[column sep=0]
      B(a) \arrow[rr,"{\lam{b}((a,b),\refl{a})}"] \arrow[dr,swap,"\mathsf{gap}"] & & \fib{\proj 1}{a} \arrow[dl,"\mathsf{gap}"] \\
      & \Big(\sm{x:A}B(x)\Big)\times_A\unit.
    \end{tikzcd}
  \end{equation*}
  Since the top map is an equivalence by \cref{ex:fib_replacement}, and the map on the right is an equivalence by \cref{lem:fib_pb}, it follows that the map on the left is an equivalence. The claim follows.
\end{proof}

\section{Fiberwise equivalences}

\begin{lem}\label{lem:pb_subst}
Let $f:A\to B$, and let $Q$ be a type family over $B$. Then the square
\begin{equation*}
\begin{tikzcd}[column sep=6em]
\sm{x:A}Q(f(x)) \arrow[r,"{\lam{(x,q)}(f(x),q)}"] \arrow[d,swap,"\proj 1"] & \sm{y:B}Q(b) \arrow[d,"\proj 1"] \\
A \arrow[r,swap,"f"] & B
\end{tikzcd}
\end{equation*}
commutes by $H\defeq \lam{(x,q)}\refl{f(x)}$. This is a pullback square.\index{substitution!as pullback|textit}
\end{lem}

\begin{proof}
By \cref{thm:is_pullback} it suffices to show that the gap map is an equivalence. The gap map is homotopic to the function
\begin{equation*}
\lam{(x,q)}(x,(f(x),q),\refl{f(x)}).
\end{equation*}
The inverse of this map is given by $\lam{(x,((y,q),p))}(x,\mathsf{tr}_Q(p^{-1},q))$, and it is straightforward to see that these maps are indeed mutual inverses.
\end{proof}

\begin{thm}\label{thm:pb_fibequiv}
Let $f:A\to B$, and let $g:\prd{a:A}P(a)\to Q(f(a))$ be a fiberwise transformation\index{fiberwise transformation|textit}. The following are equivalent:
\begin{enumerate}
\item The commuting square
\begin{equation*}
\begin{tikzcd}[column sep=large]
\sm{a:A}P(a) \arrow[r,"{\total[f]{g}}"] \arrow[d,->>] & \sm{b:B}Q(b) \arrow[d,->>] \\
A \arrow[r,swap,"f"] & B
\end{tikzcd}
\end{equation*}
is a pullback square.
\item $g$ is a fiberwise equivalence.\index{fiberwise equivalence|textit}
\end{enumerate}
\end{thm}

\begin{proof}
The gap map is homotopic to the composite
\begin{equation*}
\begin{tikzcd}[column sep=large]
\sm{x:A}P(x) \arrow[r,"\total{g}"] & \sm{x:A}Q(f(x)) \arrow[r,"{\mathsf{gap}'}"] & A \times_B \Big(\sm{y:B}Q(y)\Big)
\end{tikzcd}
\end{equation*}
where $\mathsf{gap}'$ is the gap map for the square in \cref{lem:pb_subst}. Since $\mathsf{gap}'$ is an equivalence, it follows by \cref{ex:3_for_2,thm:fib_equiv} that the gap map is an equivalence if and only if $g$ is a fiberwise equivalence.
\end{proof}

Our goal is now to extend \cref{thm:pb_fibequiv} to
arbitrary pullback squares. Note that every commuting
square
\begin{equation*}
\begin{tikzcd}
A \arrow[d,swap,"f"] \arrow[r,"h"] & B \arrow[d,"g"] \\
X \arrow[r,swap,"i"] & Y
\end{tikzcd}
\end{equation*}
with $H: i\circ f ~ g \circ h$ induces a map
\begin{equation*}
\mathsf{fib\usc{}sq} : \prd{x:X} \fib{f}{x} \to \fib{g}{f(x)}
\end{equation*}
on the fibers, by
\begin{equation*}
\mathsf{fib\usc{}sq}(x,(a,p))\defeq (h(a),\ct{H(a)^{-1}}{\ap{i}{p}}).
\end{equation*}

\begin{thm}\label{cor:pb_fibequiv}
Consider a commuting square
\begin{equation*}
\begin{tikzcd}
A \arrow[d,swap,"f"] \arrow[r,"h"] & B \arrow[d,"g"] \\
X \arrow[r,swap,"i"] & Y
\end{tikzcd}
\end{equation*}
with $H: i\circ f ~ g \circ h$. The following are equivalent:
\begin{enumerate}
\item The square is a pullback square.\index{pullback square!characterized by fiberwise equivalence|textit}
\item The induced map on fibers
\begin{equation*}
\mathsf{fib\usc{}sq} : \prd{x:X} \fib{f}{x} \to \fib{g}{f(x)}
\end{equation*}
is a fiberwise equivalence.
\end{enumerate}
\end{thm}

\begin{proof}
First we observe that the square
\begin{equation*}
\begin{tikzcd}[column sep=huge]
\sm{x:X}\fib{f}{x} \arrow[d,swap,"\eqvsym"] \arrow[r,"\total{\mathsf{fib\usc{}sq}}"] &
\sm{x:X}\fib{g}{f(x)} \arrow[d,"\total{\total{\mathsf{inv}}}"] \\
A \arrow[r,swap,"\mathsf{gap}"] & X \times_Y B
\end{tikzcd}
\end{equation*}
commutes. To construct such a homotopy, we need to construct an identification
\begin{equation*}
(f(a),h(a),H(a))=(x,h(a),(\ct{H(a)^{-1}}{\ap{i}{p}})^{-1})
\end{equation*}
for every $x : X$, $a : A$, and $p : f(a) = x$. This is shown by path induction on $p : f(a)=x$. Thus, it suffices to show that
\begin{equation*}
(f(a),h(a),H(a))=(f(a),h(a),(\ct{H(a)^{-1}}{\refl{i(f(a))}})^{-1}),
\end{equation*}
which is a routine exercise. 

Now we note that the left and right maps in this square are both equivalences. Therefore it follows that the top map is an equivalence if and only if the bottom map is. The claim now follows by \cref{thm:fib_equiv}.
\end{proof}

\begin{cor}\label{cor:pb_trunc}
Consider a pullback square
\begin{equation*}
\begin{tikzcd}
C \arrow[r,"q"] \arrow[d,swap,"p"] & B \arrow[d,"g"] \\
A \arrow[r,swap,"f"] & X.
\end{tikzcd}
\end{equation*}
If $g$ is a $k$-truncated map, then so is $p$. In particular, if $g$ is an embedding then so is $p$.\index{truncated!map!pullbacks of truncated maps|textit}\index{embedding!pullbacks of embeddings|textit}
\end{cor}

\begin{proof}
Since the square is assumed to be a pullback square, it follows from \cref{cor:pb_fibequiv} that for each $x:A$, the fiber $\fib{p}{x}$ is equivalent to the fiber $\fib{g}{f(x)}$, which is $k$-truncated. Since $k$-truncated types are closed under equivalences by \cref{thm:ktype_eqv}, it follows that $p$ is a $k$-truncated map.
\end{proof}

\begin{cor}\label{cor:pb_equiv}
Consider a commuting square
\begin{equation*}
\begin{tikzcd}
C \arrow[r,"q"] \arrow[d,swap,"p"] & B \arrow[d,"g"] \\
A \arrow[r,swap,"f"] & X.
\end{tikzcd}
\end{equation*}
and suppose that $g$ is an equivalence. Then the following are equivalent:
\begin{enumerate}
\item The square is a pullback square.
\item The map $p:C\to A$ is an equivalence.\index{equivalence!pullback of|textit}
\end{enumerate}
\end{cor}

\begin{proof}
If the square is a pullback square, then by \cref{thm:pb_fibequiv} the fibers of $p$ are equivalent to the fibers of $g$, which are contractible by \cref{thm:contr_equiv}. Thus it follows that $p$ is a contractible map, and hence that $p$ is an equivalence.

If $p$ is an equivalence, then by \cref{thm:contr_equiv} both $\fib{p}{x}$ and $\fib{g}{f(x)}$ are contractible for any $x:X$. It follows by \cref{ex:contr_equiv} that the induced map $\fib{p}{x}\to\fib{g}{f(x)}$ is an equivalence. Thus we apply \cref{cor:pb_fibequiv} to conclude that the square is a pullback.
\end{proof}

\begin{thm}\label{thm:pb_fibequiv_complete}
Consider a diagram of the form
\begin{equation*}
\begin{tikzcd}
A \arrow[d,swap,"f"] & B \arrow[d,"g"] \\
X \arrow[r,swap,"h"] & Y.
\end{tikzcd}
\end{equation*}
Then the type of triples $(i,H,p)$ consisting of a map $i:A\to B$, a homotopy $H:h\circ f\htpy g\circ i$, and a term $p$ witnessing that the square
\begin{equation*}
\begin{tikzcd}
A \arrow[d,swap,"f"] \arrow[r,"i"] & B \arrow[d,"g"] \\
X \arrow[r,swap,"h"] & Y.
\end{tikzcd}
\end{equation*}
is a pullback square, is equivalent to the type of fiberwise equivalences
\begin{equation*}
\prd{x:X}\eqv{\fib{f}{x}}{\fib{g}{h(x)}}.
\end{equation*}
\end{thm}

\begin{cor}\label{cor:pb_fibequiv_complete}
Let $h:X\to Y$ be a map, and let $P$ and $Q$ be families over $X$ and $Y$, respectively.
Then the type of triples $(i,H,p)$ consisting of a map 
\begin{equation*}
i:\Big(\sm{x:X}P(x)\Big)\to \Big(\sm{y:Y}Q(y)\Big),
\end{equation*}
a homotopy $H:h\circ \proj 1\htpy \proj 1\circ i$, and a term $p$ witnessing that the square
\begin{equation*}
\begin{tikzcd}
\sm{x:X}P(x) \arrow[d,swap,"\proj 1"] \arrow[r,"i"] & \sm{y:Y}Q(y) \arrow[d,"\proj 1"] \\
X \arrow[r,swap,"h"] & Y.
\end{tikzcd}
\end{equation*}
is a pullback square, is equivalent to the type of fiberwise equivalences
\begin{equation*}
\prd{x:X}\eqv{P(x)}{Q(h(x))}.
\end{equation*}
\end{cor}

\section{The pullback pasting property}

The following theorem is also called the \define{pasting property} of pullbacks.\index{pasting property!of pullbacks|textit}

\begin{thm}\label{thm:pb_pasting}
Consider a commuting diagram of the form
\begin{equation*}
\begin{tikzcd}
A \arrow[r,"k"] \arrow[d,swap,"f"] & B \arrow[r,"l"] \arrow[d,"g"] & C \arrow[d,"h"] \\
X \arrow[r,swap,"i"] & Y \arrow[r,swap,"j"] & Z
\end{tikzcd}
\end{equation*}
with homotopies $H:i\circ f\htpy g\circ k$ and $K:j\circ g\htpy h\circ l$, and the homotopy
\begin{equation*}
\ct{(j\cdot H)}{(K\cdot k)}:j\circ i\circ f\htpy h\circ l\circ k
\end{equation*}
witnessing that the outer rectangle commutes. Furthermore, suppose that the square on the right is a pullback square. Then the following are equivalent:
\begin{samepage}%
\begin{enumerate}
\item The square on the left is a pullback square.
\item The outer rectangle is a pullback square.
\end{enumerate}%
\end{samepage}%
\end{thm}

\begin{proof}
The commutativity of the two squares and the outer rectangle induces a commuting triangle
\begin{equation*}
\begin{tikzcd}[column sep=tiny]
\fib{f}{x} \arrow[rr,"\mathsf{fib\usc{}sq}_{(f,k,H)}(x)"] \arrow[dr,swap,"\mathsf{fib\usc{}sq}_{f,l\circ k,\ct{(j\cdot H)}{(K\cdot k)}}(x)"] & & \fib{g}{i(x)} \arrow[dl,"\mathsf{fib\usc{}sq}_{(g,l,K)}(i(x))"] \\
& \fib{h}{j(i(x))}.
\end{tikzcd}
\end{equation*}
A homotopy witnessing that the triangle commutes is constructed by a routine calculation.

Since the triangle commutes, and since the map $\mathsf{fib\usc{}sq}_{(g,l,K)}(i(x))$ is an equivalence for each $x:X$ by \cref{cor:pb_fibequiv}, it follows
by the 3-for-2 property of equivalences that for each $x:X$ the top map in the triangle is an equivalence if and only if the left map is an equivalence.
The claim now follows by a second application of \cref{cor:pb_fibequiv}.
\end{proof}

\section{The disjointness of coproducts}

As an application of the theory of pullbacks, we show that coproducts are disjoint.\index{coproduct} In this section we will write
\begin{equation*}
[f,g] : A+B\to X
\end{equation*}
for the unique map satisfying $[f,g](\inl(x))\jdeq f(x)$ and $[f,g](\inr(y))\jdeq g(y)$, where $f:A\to X$ and $g:B\to X$. Furthermore, we will write
\begin{equation*}
f+g\defeq [\inl\circ f,\inr\circ g]:A+B\to X+Y
\end{equation*}
for any $f:A\to X$ and $g:B\to Y$.

\begin{lem}\label{lem:pb_bool}
Let $X$ be a type. Then we have the pullback squares
\begin{equation*}
\begin{tikzcd}
X \arrow[r,"\mathsf{const}_\ttt"] \arrow[d,swap,"\idfunc"] &[2em] \unit \arrow[d,"\mathsf{const}_{\bfalse}"] & \emptyt \arrow[r] \arrow[d] &[2em] \unit \arrow[d,"\mathsf{const}_{\btrue}"] \\
X \arrow[r,swap,"\mathsf{const}_{\bfalse}"] & \bool & X \arrow[r,swap,"\mathsf{const}_{\bfalse}"] & \bool,
\end{tikzcd}
\end{equation*}
and we have similar pullback squares with the roles of $\bfalse$ and $\btrue$ reversed.
\end{lem}

\begin{proof}
For the first square we observe that both squares and the outer rectangle in the diagram
\begin{equation*}
\begin{tikzcd}[column sep=large]
X \arrow[d] \arrow[r] & \unit \arrow[d] \arrow[r] & \unit \arrow[d,"\mathsf{const}_{\bfalse}"] \\
X \arrow[r,swap,"\mathsf{const}_\ttt"] & \unit \arrow[r,swap,"\mathsf{const}_{\bfalse}"] & \bool.
\end{tikzcd}
\end{equation*}
are pullback squares. To see this, recall that the identity type $\bfalse=\bfalse$ is contractible by \cref{ex:eq_bool}. Therefore it follows that the square on the right is a pullback square by \cref{ex:id_pb}. The square on the left is a pullback square by \cref{cor:pb_equiv}. Therefore the outer rectangle is a pullback square by \cref{thm:pb_pasting}.

For the second square we observe that both squares end the outer rectangle in the diagram
\begin{equation*}
\begin{tikzcd}[column sep=large]
\emptyt \arrow[d] \arrow[r] & \emptyt \arrow[d] \arrow[r] & \unit \arrow[d,"\mathsf{const}_{\btrue}"] \\
X \arrow[r,swap,"\mathsf{const}_\ttt"] & \unit \arrow[r,swap,"\mathsf{const}_{\bfalse}"] & \bool.
\end{tikzcd}
\end{equation*}
are pullback squares.
To see this, recall that the identity type $\bfalse=\btrue$ is equivalent to the empty type by \cref{ex:eq_bool}. Therefore it follows that the square on the right is a pullback. It is also straightforward to verify that the square on the left is a pullback. Therefore it follows from \cref{thm:pb_pasting} that the outer rectangle is a pullback.
\end{proof}

\begin{lem}\label{lem:inl_pb}
For any two types $A$ and $B$, the squares
\begin{equation*}
\begin{tikzcd}[column sep=6.5em]
A \arrow[r,"\mathsf{const}_\ttt"] \arrow[d,swap,"\inl"] & \unit \arrow[d,"\mathsf{const}_{\bfalse}"] &[-3em] B \arrow[r,"\mathsf{const}_\ttt"] \arrow[d,swap,"\inr"] & \unit \arrow[d,"\mathsf{const}_{\btrue}"] \\
A+B \arrow[r,swap,"{[\mathsf{const}_{\bfalse},\mathsf{const}_{\btrue}]}"] & \bool & A+B \arrow[r,swap,"{[\mathsf{const}_{\bfalse},\mathsf{const}_{\btrue}]}"] & \bool
\end{tikzcd}
\end{equation*}
are pullback squares.
\end{lem}

\begin{proof}
The two cases are similar, so we only give the proof that the left square is a pullback. The left square commutes by the homotopy
\begin{equation*}
H\defeq \mathsf{htpy\usc{}refl}_{\mathsf{const}_{\bfalse}}.
\end{equation*}
To see that the asserted square is a pullback square we use \cref{thm:is_pullback} and show that the gap map is an equivalence. First we note that the gap map is homotopic to the function $e:A\to (A+B)\times_\bool\unit$ is defined by
\begin{equation*}
\lam{x}(\inl(x),\ttt,\refl{\bfalse}).
\end{equation*}
The inverse is defined by the induction principle of coproducts by
\begin{align*}
e^{-1}(\inl(x),t,\alpha) & \defeq x \\
e^{-1}(\inr(y),t,\alpha) & \defeq \ind{\emptyt}(\zeta(\alpha)),
\end{align*}
where $\zeta:\prd{x,y:\bool}(x=y)\to \mathsf{Eq}_\bool(x,y)$ is the canonical map of the identity type of $\bool$ into the observational equality on $\bool$. In the case of $\alpha:\bfalse=\btrue$ we obtain a term of $\mathsf{Eq}_\bool(\bfalse,\btrue)\jdeq \emptyt$. It is immediate from the computation rules that $e^{-1}\circ e\jdeq \idfunc$. 

The homotopy $e\circ e^{-1}\htpy \idfunc$ is again constructed by the induction principle of coproducts. In the $\inl$-case we have $e(e^{-1}(\inl(x),t,\alpha))\jdeq (\inl(x),\ttt,\refl{\bfalse})$. We define the identification
\begin{equation*}
(\inl(x),\ttt,\refl{\bfalse})=(\inl(x),t,\alpha)
\end{equation*}
by singleton induction on $t:\unit$ and $\alpha:\bfalse=\bfalse$ (both of which are terms of contractible types). Thus, it suffices to provide an identification
\begin{equation*}
(\inl(x),\ttt,\refl{\bfalse})=(\inl(x),\ttt,\refl{\bfalse}),
\end{equation*}
which we have by reflexivity. The $\inr$-case is again automatic, since we obtain a term of the empty type from $\alpha:\bfalse=\btrue$. This completes the proof that $e$ is an equivalence.
\end{proof}

\begin{cor}\label{cor:inl_emb}
The maps $\inl:A\to A+B$ and $\inr:B\to A+B$ are embeddings.\index{embedding!coproduct inclusions|textit}
\end{cor}

\begin{proof}
By the pullback squares of \cref{lem:inl_pb} and \cref{cor:pb_trunc} it suffices to show that $\unit\to\bool$ is an embedding. This is \cref{ex:injective}.
\end{proof}

\begin{thm}\label{thm:pb_disjoint}
Coproducts are \define{disjoint}\index{disjointness!of coproducts|textbf} in the sense that for any two types $A$ and $B$, the commuting square
\begin{equation*}
\begin{tikzcd}
\emptyt \arrow[r] \arrow[d] & B \arrow[d,"\inr"] \\
A \arrow[r,swap,"\inl"] & A+B
\end{tikzcd}
\end{equation*}
is a pullback square.
\end{thm}

\begin{proof}
Now consider the commuting diagram
\begin{equation*}
\begin{tikzcd}
\emptyt \arrow[d] \arrow[r] & B \arrow[d,"\inr"] \arrow[r] &[5em] \unit \arrow[d,"\mathsf{const}_\btrue"] \\
A \arrow[r,swap,"\inl"] & A+B \arrow[r,swap,"{[\mathsf{const}_{\bfalse},\mathsf{const}_{\btrue}]}"] & \bool.
\end{tikzcd}
\end{equation*}
By \cref{lem:pb_bool} it follows that the outer rectangle is a pullback square. The square on the right is a pullback square by \cref{lem:inl_pb}. Therefore the square on the left is a pullback square by \cref{thm:pb_pasting}.
\end{proof}

\begin{cor}\label{cor:id_coprod}
Let $A$ and $B$ be types. There are equivalences\index{identity type!of coproducts|textit}
\begin{align*}
(\inl(x)=\inl(x')) & \eqvsym (x=_A x') \\
(\inl(x)=\inr(y')) & \eqvsym \emptyt \\
(\inr(y)=\inl(x')) & \eqvsym \emptyt \\
(\inr(y)=\inr(y')) & \eqvsym (y=_B y').
\end{align*}
\end{cor}

\begin{proof}
The cases
\begin{align*}
(\inl(x)=\inl(x')) & \eqvsym (x=_A x') \\
(\inr(y)=\inr(y')) & \eqvsym (y=_B y').
\end{align*}
follow from \cref{cor:inl_emb} since both $\inl$ and $\inr$ are embeddings. The remaining cases follow from the disjointness of coproducts, proven in \cref{thm:pb_disjoint}.
\end{proof}

\begin{exercises}
\item \label{ex:id_pb}\index{identity type!as pullback}
\begin{subexenum}
\item Show that the square\index{identity type!as pullback}
\begin{equation*}
\begin{tikzcd}
(x=y) \arrow[r] \arrow[d] & \unit \arrow[d,"\mathsf{const}_y"] \\
\unit \arrow[r,swap,"\mathsf{const}_x"] & A
\end{tikzcd}
\end{equation*}
is a pullback square.
\item Show that the square\index{diagonal!of a type!fibers of}
\begin{equation*}
\begin{tikzcd}[column sep=large]
(x=y) \arrow[r,"\mathsf{const}_{x}"] \arrow[d,swap,"\mathsf{const}_\ttt"] & A \arrow[d,"\delta_A"] \\
\unit \arrow[r,swap,"{\mathsf{const}_{(x,y)}}"] & A\times A
\end{tikzcd}
\end{equation*}
is a pullback square, where $\delta_A:A\to A\times A$ is the diagonal of $A$, defined in \cref{ex:diagonal}.
\end{subexenum}
\item \label{ex:trunc_diagonal_map}In this exercise we give an alternative characterization of the notion of $k$-truncated map, compared to \cref{thm:trunc_ap}. Given a map $f:A\to X$ define the \define{diagonal}\index{diagonal!of a map} of $f$ to be the map $\delta_f:A\to A\times_X A$ given by $x\mapsto (x,x,\refl{f(x)})$.
\begin{subexenum}
\item Construct an equivalence
\begin{equation*}
\eqv{\fib{\delta_f}{(x,y,p)}}{\fib{\apfunc{f}}{p}}
\end{equation*}
to show that the square\index{action on paths!fibers of}\index{diagonal!of a map!fibers of}
\begin{equation*}
\begin{tikzcd}[column sep=large]
\fib{\apfunc{f}}{p} \arrow[r,"\mathsf{const}_x"] \arrow[d,swap,"\mathsf{const}_\ttt"] & A \arrow[d,"\delta_f"] \\
\unit \arrow[r,swap,"{\mathsf{const}_(x,y,p)}"] & A\times_X A
\end{tikzcd}
\end{equation*}
is a pullback square, for every $x,y:A$ and $p:f(x)=f(y)$.
\item Show that a map $f:A\to X$ is $(k+1)$-truncated if and only if $\delta_f$ is $k$-truncated.\index{truncated!map!by truncatedness of diagonal}
\end{subexenum}
Conclude that $f$ is an embedding if and only if $\delta_f$ is an equivalence.\index{embedding!diagonal is an equivalence}
\item Consider a commuting square
\begin{equation*}
\begin{tikzcd}
C \arrow[r,"q"] \arrow[d,swap,"p"] & B \arrow[d,"g"] \\
A \arrow[r,swap,"f"] & X
\end{tikzcd}
\end{equation*}
with $H:f\circ p\htpy g\circ q$. Show that the following are equivalent:
\begin{enumerate}
\item The square is a pullback square.
\item For every type $T$, the commuting square
\begin{equation*}
\begin{tikzcd}
C^T \arrow[r,"q\circ\blank"] \arrow[d,swap,"p\circ\blank"] & B^T \arrow[d,"g\circ\blank"] \\
A^T \arrow[r,swap,"f\circ\blank"] & X^T
\end{tikzcd}
\end{equation*}
is a pullback square.
\end{enumerate}
Note: property (ii) is really just a rephrasing of the universal property of pullbacks.\index{pullback square!universal property}
\item \label{ex:pb_diagonal}Consider a commuting square
\begin{equation*}
\begin{tikzcd}
C \arrow[r,"q"] \arrow[d,swap,"p"] & B \arrow[d,"g"] \\
A \arrow[r,swap,"f"] & X
\end{tikzcd}
\end{equation*}
with $H:f\circ p\htpy g\circ q$. Show that the following are equivalent:
\begin{enumerate}
\item The square is a pullback square.
\item The square
\begin{equation*}
\begin{tikzcd}
C \arrow[r,"g\circ q"] \arrow[d,swap,"{\lam{x}(p(x),q(x))}"] & X \arrow[d,"\delta_X"] \\
A\times B \arrow[r,swap,"f\times g"] & X\times X
\end{tikzcd}
\end{equation*}
which commutes by $\lam{z}\mathsf{eq\usc{}pair}(H(z),\refl{g(q(z))})$ is a pullback square.
\end{enumerate}
\item \label{ex:pb_prod}Show that if\index{pullback!cartesian products of pullbacks}
\begin{equation*}
\begin{tikzcd}
C_1 \arrow[r] \arrow[d] & B_1 \arrow[d] & C_2 \arrow[r] \arrow[d] & B_2 \arrow[d] \\
A_1 \arrow[r] & X_1 & A_2 \arrow[r] & X_2
\end{tikzcd}
\end{equation*}
are pullback squares, then so is
\begin{equation*}
\begin{tikzcd}
C_1\times C_2 \arrow[r] \arrow[d] & B_1\times B_2 \arrow[d] \\
A_1 \times A_2 \arrow[r] & X_1\times X_2. 
\end{tikzcd}
\end{equation*}
\item Consider for each $i:I$ a pullback square\index{pullback!Sigma-type of pullbacks@{$\Sigma$-type of pullbacks}}
\begin{equation*}
\begin{tikzcd}
C_i \arrow[r,"q_i"] \arrow[d,swap,"p_i"] & B_i \arrow[d,"g_i"] \\
A_i \arrow[r,swap,"f_i"] & X_i
\end{tikzcd}
\end{equation*}
with $H_i: f_i\circ p_i\htpy g_i\circ q_i$. 
\begin{subexenum}
\item \label{ex:pb_sigma}Show that the square
\begin{equation*}
\begin{tikzcd}[column sep=large]
\sm{i:I}C_i \arrow[r,"\total{q}"] \arrow[d,swap,"\total{p}"] & \sm{i:I}B_i \arrow[d,"\total{g}"] \\
\sm{i:I}A_i \arrow[r,swap,"\total{f}"] & \sm{i:I}X_i
\end{tikzcd}
\end{equation*}
which commutes by the homotopy
\begin{equation*}
\total{H}\defeq \lam{(i,c)}\mathsf{eq\usc{}pair}(\refl{i},H_i(c))
\end{equation*}
is a pullback square.\index{pullback!Pi-type of pullbacks@{$\Pi$-type of pullbacks}}
\item \label{ex:pb_pi}Show that the commuting square
\begin{equation*}
\begin{tikzcd}
\prd{i:I}C_i \arrow[r] \arrow[d] & \prd{i:I}B_i \arrow[d] \\
\prd{i:I}A_i \arrow[r] & \prd{i:I}X_i
\end{tikzcd}
\end{equation*}
is a pullback square.
\end{subexenum}
%\item 
%\begin{subexenum}
%\item Show that \index{equivalence!type of equivalences!as pullback}
%\begin{equation*}
%\begin{tikzcd}[column sep=8em]
%\eqv{A}{B} \arrow[r] \arrow[d] & \unit \arrow[d,"{(\idfunc[A],\idfunc[B])}"] \\
%A^B\times B^A \times A^B \arrow[r,swap,"{(h,f,g)\mapsto (h\circ f,f\circ g)}"] & A^A \times B^B
%\end{tikzcd}
%\end{equation*}
%is a pullback square.
%\item Show that \index{contractible!type of contractibility!as pullback}
%\begin{equation*}
%\begin{tikzcd}[column sep=6em]
%\iscontr(A) \arrow[r,"\mathsf{const}_{\ttt}"] \arrow[d,swap,"\proj 1"] & \unit \arrow[d,"{\lam{\ttt}\idfunc[A]}"] \\
%A \arrow[r,swap,"{\lam{x}\mathsf{const}_x}"] & A^A
%\end{tikzcd}
%\end{equation*}
%is a pullback square.
%\end{subexenum}
%\item Consider a commuting square
%\begin{equation*}
%\begin{tikzcd}
%C \arrow[r] \arrow[d] & A \arrow[d] \\
%B \arrow[r] & X.
%\end{tikzcd}
%\end{equation*}
%Show that this square is cartesian if and only if the induced map $C\to A\times_X B$ has a retraction.
\item \label{ex:pi_sec}Let $B$ be a type family over $A$. Show that the square\index{Pi-type@{$\Pi$-type}!as pullback}
\begin{equation*}
\begin{tikzcd}[column sep=6em]
\prd{x:A}B(x) \arrow[r,"{\lam{f}{x}(x,f(x))}"] \arrow[d] & \Big(\sm{x:A}B(x)\Big)^A \arrow[d,"\proj 1\circ\blank"] \\
\unit \arrow[r,swap,"{\mathsf{const}_{\idfunc[A]}}"] & A^A
\end{tikzcd}
\end{equation*}
is a pullback square. Conclude that the type $\prd{x:A}B(x)$ is equivalent to the type $\mathsf{sec}(\proj 1)$ of sections of the projection map.
\item Consider a pullback square
\begin{equation*}
\begin{tikzcd}
C \arrow[r,"q"] \arrow[d,swap,"p"] & B \arrow[d,"g"] \\
A \arrow[r,swap,"f"] & X,
\end{tikzcd}
\end{equation*}
with $H:f\circ p\htpy g\circ q$, and let $c_1,c_2:C$. Show that the square
\begin{equation*}
\begin{tikzcd}[column sep=8em]
(c_1=c_2) \arrow[r,"\apfunc{q}"] \arrow[d,swap,"\apfunc{p}"] & (q(c_1)=q(c_2)) \arrow[d,"\lam{\beta}\ct{H(c_1)}{\ap{g}{\beta}}"] \\
(p(c_1)=p(c_2)) \arrow[r,swap,"\lam{\alpha}\ct{\ap{f}{\alpha}}{H(c_2)}"] & f(p(c_1))=g(q(c_2)),
\end{tikzcd}
\end{equation*}
which commutes by the naturality of homotopies (\cref{defn:htpy_nat}), is again a pullback square.
%\end{subexenum}
%\item Suppose that the squares
%\begin{equation*}
%\begin{tikzcd}
%C \arrow[r,"q"] \arrow[d,swap,"p"] & B \arrow[d,"g"] & {C'} \arrow[r,"{q'}"] \arrow[d,swap,"{p'}"] & B \arrow[d,"g"] \\
%A \arrow[r,swap,"f"] & X & A \arrow[r,swap,"f"] & X
%\end{tikzcd}
%\end{equation*}
%with homotopies $H:f\circ p \htpy g\circ q$ and $H':f\circ p'\htpy g\circ q'$ are both pullback squares. Show that the type of equivalences $e:\eqv{C'}{C}$ equipped with an identification
%\begin{equation*}
%\mathsf{cone\usc{}map}((p,q,H),e)=(p',q',H')
%\end{equation*}
%is contractible.
\begin{comment}
\item Consider a \define{natural transformation of cospans}\index{cospan!natural transformation of}, i.e.~a commuting diagram of the form
\begin{equation*}
\begin{tikzcd}
A \arrow[r,"f"] \arrow[d,swap,"i"] & X \arrow[d,swap,"j"] & B \arrow[l,swap,"g"] \arrow[d,"k"] \\
A' \arrow[r,swap,"{f'}"] & X' & B'. \arrow[l,"{g'}"]
\end{tikzcd}
\end{equation*}
Show that the map
\begin{equation*}
(a,b,p)\mapsto (i(a),j(b),\mathsf{ap}_k(p)): A \times_X B \to A'\times_{X'} B'
\end{equation*}
is $k$-truncated if each of the vertical maps is.
\end{comment}
\item Suppose that 
\begin{equation*}
\begin{tikzcd}
C \arrow[r,"q"] \arrow[d,swap,"p"] & B \arrow[d,"g"] \\
A \arrow[r,swap,"f"] & X 
\end{tikzcd}
\end{equation*}
with $H:f\circ p\htpy g\circ q$ is a pullback square. Show that the square
\begin{equation*}
\begin{tikzcd}
C \arrow[r,"p"] \arrow[d,swap,"q"] & A \arrow[d,"f"] \\
B \arrow[r,swap,"g"] & X 
\end{tikzcd}
\end{equation*}
with $H^{-1}:g\circ q\htpy f\circ p$ is again a pullback square.
\item \label{ex:pb_fib}Consider a commuting square
\begin{equation*}
\begin{tikzcd}
C \arrow[d,swap,"p"] \arrow[r,"q"] & B \arrow[d,"g"] \\
A \arrow[r,swap,"f"] & X.
\end{tikzcd}
\end{equation*}
with $H:f\circ p\htpy g\circ q$, and let $h:C\to A\times_X B$ be the map given by $h(z)\defeq (p(c),q(c),H(c))$. 
Show that the square
\begin{equation*}
\begin{tikzcd}[column sep=6.5em]
\fib{\mathsf{gap}(p,q,H)}{(a,b,\alpha)} \arrow[d,swap,"\mathsf{const}_{\ttt}"] \arrow[r,"{\lam{(c,\beta)}(c,\ap{\pi_1}{\beta})}"] & \fib{p}{a} \arrow[d,"{\fibf{(f,g,H)}}"] \\
\unit \arrow[r,swap,"\mathsf{const}_{(b,\alpha^{-1})}"] & \fib{g}{f(a)}
\end{tikzcd}
\end{equation*}
\item \label{ex:pb_3by3}Consider a commuting diagram of the form
\begin{equation*}
\begin{tikzcd}
A_0 \arrow[r] \arrow[d] & B_0 \arrow[d] & C_0 \arrow[l] \arrow[d] \\
A_1 \arrow[r] & B_1 & C_1 \arrow[l] \\
A_2 \arrow[u] \arrow[r] & B_2 \arrow[u] & C_2 \arrow[u] \arrow[l]
\end{tikzcd}
\end{equation*}
with homotopies filling the (small) squares. Construct an equivalence
\begin{align*}
& (A_0\times_{B_0} C_0) \times_{(A_1\times_{B_1} C_1)} (A_2\times_{B_2} C_2) \\
& \qquad \eqvsym (A_0\times_{A_1} A_2) \times_{(B_0\times_{B_1} B_2)} (C_0\times_{C_1} C_2).
\end{align*}
This is also known as the \define{3-by-3 lemma}\index{3-by-3 lemma!for pullbacks} for pullbacks.
\end{exercises}

\chapter{Homotopy pushouts}

We can use higher inductive types\index{higher inductive types} to attach cells\index{attaching cells} to types.
For example, when we are given a type $A$, and we have a map $f:\sphere{1}\to A$ describing a circle\index{circle} in $A$.
Then we can form a new type $A'$ in which we attach a disc by `gluing' the boundary of the disc to the circle in $A$.
Using higher inductive types, this process of attaching a disc works as follows:
\begin{enumerate}
\item First we add all the points of $A$ to $A'$, i.e.~$A'$ comes equipped with a map
\begin{equation*}
i : A \to A'
\end{equation*}
\item Next, we add a new point, which is to be thought of as the center of the disc that we're attaching. In other words, $A'$ comes equipped with
\begin{equation*}
\mathrm{pt} : A'
\end{equation*}
\item Finally, for each point $x$ on the circle we add a path from the center of the disc to $i(f(x))$. In other words, $A'$ comes equipped with a path constructor
\begin{equation*}
r:\prd{x:\sphere{1}} \mathrm{pt}=i(f(x)).
\end{equation*}
\end{enumerate}
Moreover, since we're only attaching a disc to $A$ along $f$, we suppose that $A'$ satisfies an induction principle with respect to the constructors $i$, $\mathrm{pt}$, and $r$. 

The process of attaching a disc to a type $A$ along a map $f:\sphere{1}\to A$ can be generalized, so that we will also be able to attach cells of different shapes to a type. This generalization is called homotopy pushouts. Homotopy pushouts are dual to homotopy pullbacks. However, unlike pullbacks we will \emph{assume} that pushouts exist by postulating rules for higher inductive types. For the purpose of this course, the only higher inductive types that we add to our type theory are the pushouts. Some of the more exotic higher inductive types, including the Cauchy real numbers, are described in \cite{hottbook}.

\section{Pushouts as higher inductive types}

The idea of pushouts is to glue two types $A$ and $B$ together using a mediating type $S$ and maps $f:S\to A$ and $g:S\to B$. In other words, we start with a diagram of the form
\begin{equation*}
\begin{tikzcd}
A & S \arrow[l,swap,"f"] \arrow[r,"g"] & B.
\end{tikzcd}
\end{equation*}
We call such a triple $\mathcal{S}\jdeq (S,f,g)$ a \define{span}\index{span} from $A$ to $B$.
A span from $A$ to $B$ can be thought of as a relation\index{relation} from $A$ to $B$, relating $f(x)$ to $g(x)$ for any $x:S$.
Indeed, an equivalence between the type of all spans and the type of relations from $A$ to $B$ is established in \cref{ex:span_rel}.

Given a span $\mathcal{S}$ from $A$ to $B$, we form the higher inductive type $A \sqcup^{\mathcal{S}} B$. It comes equipped with the following constructors\index{inl@{$\inl$}!for pushouts}\index{inr@{$\inr$}!for pushouts}\index{glue@{$\glue$}}
\begin{align*}
\inl & : A \to A \sqcup^{\mathcal{S}} B \\
\inr & : B \to A \sqcup^{\mathcal{S}} B \\
\glue & : \prd{x:S} \inl(f(x))=\inr(g(x))
\end{align*}
and we require that it satisfies an induction principle and computation rules.

To see what the induction principle has to be, consider first a dependent function $s:\prd{x:A\sqcup^{\mathcal{S}}B}P(x)$. When we evaluate this function at the constructors, we obtain
\begin{align*}
s\circ \inl & : \prd{a:A} P(\inl(a)) \\
s\circ \inr & : \prd{b:B} P(\inr(b)) \\
\apdfunc{s}\circ \glue & : \prd{x:S} \mathsf{tr}_P(\glue(x),s(f(x)))=s(g(x)).
\end{align*}

\begin{defn}
Consider a span $\mathcal{S}\jdeq (S,f,g)$ from $A$ to $B$, and let $P$ be a family over $A\sqcup^{\mathcal{S}} B$. The \define{dependent action on generators}\index{dependent action on generators!for pushouts|textbf} is defined to be the map\index{dgen_S@{$\mathsf{dgen}_{\mathcal{S}}$}|textbf}
\begin{align*}
\mathsf{dgen}_{\mathcal{S}}^P & : \Big(\prd{x:A\sqcup^{\mathcal{S}} B} P(x)\Big) \to \Big(\sm{f': \prd{a:A}P(\inl(a))}{g':\prd{b:B}P(\inr(b))}\Big.\\
& \qquad\qquad\qquad\qquad\qquad\qquad \Big.\prd{x:S} \mathsf{tr}_P(\glue(x),f'(f(x)))=g'(g(x))\Big).
\end{align*}
given by $s\mapsto (s\circ\inl,s\circ\inr,\apdfunc{s}\circ\glue)$.
\end{defn}

We can now fully specify homotopy pushouts.

\begin{defn}
Given a span $\mathcal{S}\jdeq (S,f,g)$, the \define{(homotopy) pushout}\index{pushout|textbf} $A\sqcup^{\mathcal{S}} B$ of $\mathcal{S}$ is defined to be the higher inductive\index{higher inductive types} type equipped with\index{inl@{$\inl$}!for pushouts|textbf}\index{inr@{$\inr$}!for pushouts|textbf}\index{glue@{$\glue$}|textbf}
\begin{align*}
\inl & : A \to A \sqcup^{\mathcal{S}} B \\
\inr & : B \to A \sqcup^{\mathcal{S}} B \\
\glue & : \prd{x:S} \inl(f(x))=\inr(g(x)),
\end{align*}
satisfying the \define{induction principle} for pushouts\index{induction principle!for pushouts|textbf}, which asserts that for each type family $P$ over $A\sqcup^{\mathcal{S}} B$ the map $\mathsf{dgen}_{\mathcal{S}}^P$ has a section.
\end{defn}

\begin{rmk}
The induction principle of the pushout $A\sqcup^{\mathcal{S}} B$ provides us with a dependent function
\begin{equation*}
\ind{\mathcal{S}}(f',g',G) : \prd{x:A\sqcup^{\mathcal{S}} B} P(x),
\end{equation*}
for every
\begin{align*}
f' & : \prd{a:A}P(\inl(a)) \\
g' & : \prd{b:B}P(\inr(b)) \\
G & : \prd{x:S} \mathsf{tr}_P(\glue(x),f'(f(x)))=g'(g(x))
\end{align*}
Moreover, the function $\ind{\mathcal{S}}(f',g',G)$ comes equipped with an identification
\begin{equation*}
\mathsf{dgen}_{\mathcal{S}}(\ind{\mathcal{S}}(f',g',G))=(f',g',G).
\end{equation*}
Writing $s\defeq \ind{\mathcal{S}}(f',g',G)$, we see that such an identification between triples is equivalently described by a triple $(H,K,L)$ consisting of
\begin{align*}
H : s\circ \inl \htpy f' \\
K : s\circ\inr \htpy g' 
\end{align*}
and a homotopy $L$ witnessing that the square
\begin{equation*}
\begin{tikzcd}[column sep=8em]
\mathsf{tr}_{P}(\glue(x),s(\inl(f(x)))) \arrow[r,equals,"\ap{\mathsf{tr}_P(\glue(x))}{H(x)}"] \arrow[d,equals,swap,"\apd{s}{\glue(x)}"] & \mathsf{tr}_{P}(\glue(x),f'(f(x))) \arrow[d,equals,"G(x)"] \\
s(\inr(g(x))) \arrow[r,equals,swap,"K(x)"] & g'(g(x))
\end{tikzcd}
\end{equation*}
commutes, for every $x:S$. These are the \define{computation rules} for pushouts\index{computation rules!for pushouts}.
\end{rmk}

\begin{comment}
The \define{formation rule} for pushouts simply states that for any span $\mathcal{S}\defeq (S,f,g)$ from $A$ to $B$, a type $A\sqcup^{\mathcal{S}} B$ can be formed. We call $A\sqcup^{\mathcal{S}} B$ the \define{canonical pushout} of $\mathcal{S}$. 

\begin{prooftree}
\AxiomC{$\Gamma\vdash f:S\to A$}
\AxiomC{$\Gamma\vdash g:S\to B$}
\BinaryInfC{$\Gamma\vdash A\sqcup^{\mathcal{S}} B~\mathrm{type}$}
\end{prooftree}

The \define{introduction rules} for pushouts provide ways to construct terms of the type $A\sqcup^{\mathcal{S}} B$, and ways to identify some of those.
\begin{prooftree}
\AxiomC{$\Gamma\vdash f:S\to A$}
\AxiomC{$\Gamma\vdash g:S\to B$}
\BinaryInfC{$\Gamma\vdash \inl : A \to A\sqcup^{\mathcal{S}} B$}
\end{prooftree}

\begin{prooftree}
\AxiomC{$\Gamma\vdash f:S\to A$}
\AxiomC{$\Gamma\vdash g:S\to B$}
\BinaryInfC{$\Gamma\vdash \inr : B \to A\sqcup^{\mathcal{S}} B$}
\end{prooftree}

\begin{prooftree}
\AxiomC{$\Gamma\vdash f:S\to A$}
\AxiomC{$\Gamma\vdash g:S\to B$}
\BinaryInfC{$\Gamma\vdash \glue : \inl\circ f \htpy \inr\circ g$}
\end{prooftree}
We assume that $A\sqcup^{\mathcal{S}} B$ is span inductive in the sense of \autoref{thm:pushout_up}. Moreover, if $A$, $B$, and $S$ are types in $\UU$, then we assume that also $A\sqcup^{\mathcal{S}} B$ is in $\UU$. In other words, we assume that the universe is \emph{closed under pushouts}.
\end{comment}

\section{Examples of pushouts}
Many interesting types can be defined as homotopy pushouts. 

\begin{defn}
Let $X$ be a type. We define the \define{suspension}\index{suspension|textbf} $\susp X$\index{SX@{$\susp X$}|textbf} of $X$ to be the pushout of the span
\begin{equation*}
\begin{tikzcd}
X \arrow[r] \arrow[d] & \unit \arrow[d,"\inr"] \\
\unit \arrow[r,swap,"\inl"] & \susp X 
\end{tikzcd}
\end{equation*}
\end{defn}

\begin{defn}
We define the \define{$n$-sphere}\index{n-sphere@{$n$-sphere}|textbf} $\sphere{n}$\index{Sn@{$\sphere{n}$}|textbf} for any $n:\N$ by induction on $n$, by taking
\begin{align*}
\sphere{0} & \defeq \bool \\
\sphere{n+1} & \defeq \susp{\sphere{n}}.
\end{align*}
\end{defn}

\begin{defn}
Given a map $f:A\to B$, we define the \define{cofiber}\index{cofiber|textbf} $\mathsf{cofib}_f$\index{cofib_f@{$\mathsf{cofib}_f$}|textbf} of $f$ as the pushout
\begin{equation*}
\begin{tikzcd}
A \arrow[r,"f"] \arrow[d] & B \arrow[d,"\inr"] \\
\unit \arrow[r,swap,"\inl"] & \mathsf{cofib}_f. 
\end{tikzcd}
\end{equation*}
The cofiber of a map is sometimes also called the \define{mapping cone}\index{mapping cone|textbf}.
\end{defn}

\begin{eg}
The suspension $\susp X$ of $X$ is the cofiber of the map $X\to \unit$.\index{suspension!as cofiber} 
\end{eg}

\begin{defn}
We define the \define{join}\index{join} $\join{X}{Y}$\index{join X Y@{$\join{X}{Y}$}|textbf} of $X$ and $Y$ to be the pushout 
\begin{equation*}
\begin{tikzcd}
X\times Y \arrow[r,"\proj 2"] \arrow[d,swap,"\proj 1"] & Y \arrow[d,"\inr"] \\
X \arrow[r,swap,"\inl"] & X \ast Y. 
\end{tikzcd}
\end{equation*}
\end{defn}

\begin{defn}
Suppose $A$ and $B$ are pointed types, with base points $a_0$ and $b_0$, respectively. The \define{(binary) wedge}\index{wedge@(binary) wedge|textbf} $A\vee B$ of $A$ and $B$ is defined as the pushout
\begin{equation*}
\begin{tikzcd}
\bool \arrow[r] \arrow[d] & A+B \arrow[d] \\
\unit \arrow[r] & A\vee B.
\end{tikzcd}
\end{equation*}
\end{defn}

\begin{defn}
Given a type $I$, and a family of pointed types $A$ over $i$, with base points $a_0(i)$. We define the \define{(indexed) wedge}\index{wedge@{(indexed) wedge}|textbf} $\bigvee_{(i:I)}A_i$ as the pushout
\begin{equation*}
\begin{tikzcd}[column sep=huge]
I \arrow[d] \arrow[r,"{\lam{i}(i,a_0(i))}"] & \sm{i:I}A_i \arrow[d] \\
\unit \arrow[r] & \bigvee_{(i:I)} A_i.
\end{tikzcd}
\end{equation*}
\end{defn}

\begin{comment}
\begin{defn}
Let $X$ and $Y$ be types with base points $x_0$ and $y_0$, respectively.
We define the \define{wedge} $X\lor Y$ of $X$ and $Y$ to be the pushout
\begin{equation*}
\begin{tikzcd}[column sep=8em]
\bool \arrow[r,"{\ind{\bool}(\inl(x_0),\inr(y_0))}"] \arrow[d,swap,"\mathsf{const}_\ttt"] & X+Y \arrow[d,"\inr"] \\
\unit \arrow[r,swap,"\inl"] & X\lor Y
\end{tikzcd}
\end{equation*}
\end{defn}

\begin{defn}
Let $X$ and $Y$ be types with base points $x_0$ and $y_0$, respectively.
We define a map
\begin{equation*}
\mathsf{wedge\usc{}incl} : X \lor Y \to X\times Y.
\end{equation*}
as the unique map obtained from the commutative square
\begin{equation*}
\begin{tikzcd}[column sep=8em]
\bool \arrow[r,"{\ind{\bool}(\inl(x_0),\inr(y_0))}"] \arrow[d,swap,"\mathsf{const}_\ttt"] & X+Y \arrow[d,"{\ind{X+Y}(\lam{x}\pairr{x,y_0},\lam{y}\pairr{x_0,y})}"] \\
\unit \arrow[r,swap,"\lam{t}\pairr{x_0,y_0}"] & X\times Y.
\end{tikzcd}
\end{equation*}
\end{defn}

\begin{defn}
We define the \define{smash product} $X\wedge Y$ of $X$ and $Y$ to be the pushout
\begin{equation*}
\begin{tikzcd}[column sep=huge]
X\lor Y \arrow[r,"\mathsf{wedge\usc{}incl}"] \arrow[d,swap,"\mathsf{const}_\ttt"] & X\times Y \arrow[d,"\inr"] \\
\unit \arrow[r,swap,"\inl"] & X\wedge Y.
\end{tikzcd}
\end{equation*}
\end{defn}
\end{comment}

\section{The universal property of pushouts}

\begin{defn}
Consider a span $\mathcal{S}\jdeq (S,f,g)$ from $A$ to $B$, and let $X$ be a type.
A \define{cocone}\index{cocone|textbf} with vertex $X$ on $\mathcal{S}$ is a triple $(i,j,H)$ consisting of maps $i:A\to X$ and $j:B\to X$, and a homotopy $H:i\circ f\htpy j\circ g$ witnessing that the square
\begin{equation*}
\begin{tikzcd}
S \arrow[r,"g"] \arrow[d,swap,"f"] & B \arrow[d,"j"] \\
A \arrow[r,swap,"i"] & X
\end{tikzcd}
\end{equation*}
commutes.
We write $\mathsf{cocone}_{\mathcal{S}}(X)$\index{cocone_S(X)@{$\mathsf{cocone}_{\mathcal{S}}(X)$}|textbf} for the type of cocones on $\mathcal{S}$ with vertex $X$.
\end{defn}

\begin{defn}
Consider a cocone $(i,j,H)$ with vertex $X$ on the span $\mathcal{S}\jdeq (S,f,g)$, as indicated in the following commuting square
\begin{equation*}
\begin{tikzcd}
S \arrow[r,"g"] \arrow[d,swap,"f"] & B \arrow[d,"j"] \\
A \arrow[r,swap,"i"] & X.
\end{tikzcd}
\end{equation*}
For every type $Y$, we define the map\index{cocone_map@{$\mathsf{cocone\usc{}map}$}|textbf}
\begin{equation*}
\mathsf{cocone\usc{}map}(i,j,H):(X\to Y)\to \mathsf{cocone}(Y)
\end{equation*}
by $f\mapsto (f\circ i,f\circ j,f\cdot H)$.
\end{defn}

\begin{defn}
A commuting square
\begin{equation*}
\begin{tikzcd}
S \arrow[r,"g"] \arrow[d,swap,"f"] & B \arrow[d,"j"] \\
A \arrow[r,swap,"i"] & X.
\end{tikzcd}
\end{equation*}
with $H:i\circ f \htpy j\circ g$ is said to be a \define{(homotopy) pushout square}\index{pushout square} if the cocone $(i,j,H)$ with vertex $X$ on the span $\mathcal{S}\jdeq (S,f,g)$
satisfies the \define{universal property of pushouts}\index{universal property!of pushouts|textbf}, which asserts that the map
\begin{equation*}
\mathsf{cocone\usc{}map}(i,j,H):(X\to Y)\to \mathsf{cocone}(Y)
\end{equation*}
is an equivalence for any type $Y$. Sometimes pushout squares are also called \define{cocartesian squares}\index{cocartesian square|textbf}.
\end{defn}

\begin{comment}
\begin{rmk}
Given a pushout square
\begin{equation*}
\begin{tikzcd}
S \arrow[r,"g"] \arrow[d,swap,"f"] & B \arrow[d,"j"] \\
A \arrow[r,swap,"i"] & X,
\end{tikzcd}
\end{equation*}
we can view the cocone $(i,j,H)$ as \emph{structure} on $X$, in the sense that $X$ comes equipped with
\begin{align*}
i & : A\to X \\
j & : B\to X \\
H & : \prd{s:S} i(f(s))=j(g(s)).
\end{align*}
As we will see in \cref{thm:pushout_up}, the type $X$ is a pushout precisely when it satisfies an \emph{induction principle} formulated in terms of $(i,j,H)$. However, the homotopy $H$ provides \emph{path constructors} of $X$. 

The induction principle of pushouts is formulated with respect to families $P$ over $X$, and provides a way to construct sections of $P$. Note that from any section $s:\prd{x:X}P(x)$ we obtain
\begin{align*}
s\circ i & : \prd{a:A}P(i(a)) \\
s\circ j & : \prd{b:B}P(j(b)) \\
s\cdot H & : \prd{x:S}\mathsf{tr}_P(H(x),s(i(x)))=s(j(x)).
\end{align*}
It will be useful to write
\begin{equation*}
i' \htpy_H j' \defeq \prd{s:S} \mathsf{tr}_P(H(s),i'(f(s)))=j'(g(s))
\end{equation*}
for the type of $s\cdot H$. Thus we see that there is a map
\begin{equation*}
\Big(\prd{x:X}P(x)\Big)\to \sm{i':\prd{a:A}P(i(a))}{j':\prd{b:B}P(j(b))} i'\htpy_H j'
\end{equation*}
given by $s\mapsto (s\circ i,s\circ j,s\cdot H)$.
\end{rmk}
\end{comment}

\begin{lem}
For any span $\mathcal{S}\jdeq (S,f,g)$ from $A$ to $B$, and any type $X$ the square\index{cocone_S(X)@{$\mathsf{cocone}_{\mathcal{S}}(X)$}!as a pullback|textit}
\begin{equation*}
\begin{tikzcd}
\mathsf{cocone}_{\mathcal{S}}(X) \arrow[r,"\pi_2"] \arrow[d,swap,"\pi_1"] & X^B \arrow[d,"\blank\circ g"] \\
X^A \arrow[r,swap,"\blank\circ f"] & X^S,
\end{tikzcd}
\end{equation*}
which commutes by the homotopy $\pi_3' \defeq\lam{(i,j,H)} \mathsf{eq\usc{}htpy}(H)$, is a pullback square.
\end{lem}

\begin{proof}
The function extensionality principle induces an equivalence on total spaces
\begin{align*}
X^A\times_{X^S} X^B & \jdeq \sm{i:A\to X}{j:B\to X} i\circ f= j\circ g\\
& \eqvsym \sm{i:A\to X}{j:B\to X} i\circ f\htpy j\circ g \\
& \jdeq \mathsf{cocone}_{\mathcal{S}}(X).
\end{align*}
Concretely, the equivalence $X^A\times_{X^S} X^B \to \mathsf{cocone}_{\mathcal{S}}(X)$ is defined as
\begin{equation*}
e\defeq \lam{(i,j,p)}(i,j,\mathsf{htpy\usc{}eq}(p)),
\end{equation*}
and its inverse is the function $\lam{(i,j,H)}(i,j,\mathsf{eq\usc{}htpy}(H))$.
To complete the proof using \cref{thm:pb_3for2} that the square is a pullback it remains to construct homotopies
\begin{samepage}
\begin{align*}
K & : \pi_1\circ e \htpy \pi_1 \\
L & : \pi_2\circ e \htpy \pi_2 \\
M & : \ct{(\pi_3'\cdot e)}{((\blank\circ g)\cdot L)} \htpy \ct{((\blank\circ f)\cdot K)}{\pi_3}
\end{align*}
\end{samepage}%
For $K$ and $L$ we simply take
\begin{align*}
K & \defeq \lam{(i,j,p)} \refl{i} \\
L & \defeq \lam{(i,j,p)} \refl{j}
\end{align*}
Then we have homotopies 
\begin{align*}
(\blank\circ f)\cdot K & \htpy \lam{(i,j,p)}\refl{i\circ f} \\
(\blank\circ g)\cdot L & \htpy \lam{(i,j,p)}\refl{j\circ g}.
\end{align*}
There is also a homotopies 
\begin{align*}
\pi_3'\cdot e & \htpy \lam{(i,j,p)}\mathsf{eq\usc{}htpy}(\mathsf{htpy\usc{}eq}(p)) \\
& \htpy \lam{(i,j,p)}p
\end{align*}
by the definition of $e$. In order to construct the homotopy $M$ it is therefore equivalent to construct an identification
\begin{equation*}
\ct{p}{\refl{i\circ f}}=\ct{\refl{j\circ g}}{p},
\end{equation*}
for every $(i,j,p):X^A\times_{X^S} X^B$. This can be obtained by the unit laws.
\end{proof}

In the following theorem we establish an alternative characterization of the universal property of pushouts.
\begin{thm}\label{thm:pushout_up}
Consider a commuting square\index{universal property!of pushouts|textit}
\begin{equation*}
\begin{tikzcd}
S \arrow[r,"g"] \arrow[d,swap,"f"] & B \arrow[d,"j"] \\
A \arrow[r,swap,"i"] & X,
\end{tikzcd}
\end{equation*}
with $H:i\circ f\htpy j\circ g$. The following are equivalent:
\begin{enumerate}
\item The square is a pushout square.
\item The square
\begin{equation*}
\begin{tikzcd}
T^X \arrow[r,"\blank\circ j"] \arrow[d,swap,"\blank\circ i"] & T^B \arrow[d,"\blank\circ g"] \\
T^A \arrow[r,swap,"\blank\circ f"] & T^S
\end{tikzcd}
\end{equation*}
which commutes by the homotopy
\begin{equation*}
\lam{h} \mathsf{eq\usc{}htpy}(h\cdot H)
\end{equation*}
is a pullback square, for every type $T$.
%\item The type $X$ satisfies \define{span induction} for the span $A\leftarrow S \rightarrow B$, in the sense that for any type family $P$ over $X$, the map
%\begin{equation*}
%\Big(\prd{x:X}P(x)\Big)\to \Big(\sm{i':\prd{a:A}P(i(a))}{j':\prd{b:B}P(j(b))} i'\htpy_H j'\Big)
%\end{equation*}
%given by $s\mapsto (s\circ i,s\circ j,s\cdot H)$ has a section.
\end{enumerate}
\end{thm}

\begin{proof}
It is straightforward to verify that the diagram
\begin{equation*}
\begin{tikzcd}
T^X \arrow[dr,densely dotted,"{\mathsf{cocone\usc{}map}(i,j,H)}" {description,xshift=1em}] \arrow[ddr,bend right=15] \arrow[drr,bend left=15] &[2em] \\
& \mathsf{cocone}(T) \arrow[r,"\pi_2"] \arrow[d,swap,"\pi_1"] & T^B \arrow[d,"\blank\circ g"] \\
& T^A \arrow[r,swap,"\blank\circ f"] & T^S
\end{tikzcd}
\end{equation*}
commutes, i.e.~that there are homotopies
\begin{align*}
K & : \pi_1\circ \mathsf{cocone\usc{}map}(i,j,H) \htpy \lam{h}h\circ i \\
L & : \pi_2\circ \mathsf{cocone\usc{}map}(i,j,H) \htpy \lam{h}h\circ j \\
M & : \ct{(\pi_3'\cdot e)}{((\blank\circ g)\cdot L)} \htpy \ct{((\blank\circ f)\cdot K)}{(\lam{h} \mathsf{eq\usc{}htpy}(h\cdot H))}.
\end{align*}
Therefore it follows by \cref{thm:pb_3for2} that the cocone $(i,j,H)$ is colimiting precisely when for every type $T$ the asserted square is a pullback.
\end{proof}

\begin{eg}\label{eg:circle_pushout}
By \autoref{ex:circle_up_pushout} and the second characterization of pushouts in \autoref{thm:pushout_up} it follows that the circle is a pushout\index{circle!S1 equiv susp 2@{$\eqv{\sphere{1}}{\susp\bool}$}|textit}
\begin{equation*}
\begin{tikzcd}
\bool \arrow[r] \arrow[d] & \unit \arrow[d] \\
\unit \arrow[r] & \sphere{1}.
\end{tikzcd}
\end{equation*}
In other words, $\eqv{\sphere{1}}{\susp{\bool}}$. 
\end{eg}

\begin{thm}\label{thm:pushout}
Consider a span $\mathcal{S}\jdeq (S,f,g)$ from $A$ to $B$. Then the square
\begin{equation*}
\begin{tikzcd}
S \arrow[r,"g"] \arrow[d,swap,"f"] & B \arrow[d,"\inr"] \\
A \arrow[r,swap,"\inl"] & A \sqcup^{\mathcal{S}} B
\end{tikzcd}
\end{equation*}
is a pushout square.\index{pushout!universal property|textit}
\end{thm}

\begin{proof}
Let $X$ be a type. Our goal is to show that the map
\begin{equation*}
\mathsf{cocone\usc{}map}(\inl,\inr,\glue):(A\sqcup^{\mathcal{S}} B \to X)\to \mathsf{cocone}_{\mathcal{S}}(X)
\end{equation*}
is an equivalence. For notational breveity we will just write $\mathsf{gen}_{\mathcal{S}}$\index{gen_S@{$\mathsf{gen}_{\mathcal{S}}$}} for $\mathsf{cocone\usc{}map}_{\mathcal{S}}(\inl,\inr,\glue)$, because $\mathsf{cocone\usc{}map}_{\mathcal{S}}(\inl,\inr,\glue)$ is just the action on generators.

We first note that by \cref{ex:trans_triv} there is a commuting triangle
\begin{equation*}
\begin{tikzcd}[column sep=0]
& X^{A\sqcup^{\mathcal{S}} B} \arrow[dl,swap,"\mathsf{gen}_{\mathcal{S}}"] \arrow[dr,"\mathsf{dgen}_{\mathcal{S}}"] \\
\mathsf{cocone}_{\mathcal{S}}(X) \arrow[rr,"\eqvsym"] & & \mathsf{cocone}'_{\mathcal{S}}(X)
\end{tikzcd}
\end{equation*}
where we write
\begin{align*}
\mathsf{cocone}'_{\mathcal{S}}(X) & : \Big(\sm{f': A\to X}{g':A\to X}\Big.\\
& \qquad\qquad\Big.\prd{x:S} \mathsf{tr}_{W_{A\sqcup^{\mathcal{S}}B}(X)}(\glue(x),f'(f(x)))=g'(g(x))\Big).
\end{align*}
By the induction principle for $A\sqcup^{\mathcal{S}} B$ we have a section $\ind{\mathcal{S}}$ of $\mathsf{dgen}_{\mathcal{S}}$. Thus we obtain a section $\rec{\mathcal{S}}$ of $\mathsf{gen}_{\mathcal{S}}$. Our goal is now to show that $\rec{\mathcal{S}}$ is also a retraction of $\mathsf{gen}_{\mathcal{S}}$. We establish in \cref{lem:pushout_up_htpy} that
\begin{equation*}
(\mathsf{gen}_{\mathcal{S}}(\rec{\mathcal{S}}(\mathsf{gen}_{\mathcal{S}}(h)))= \mathsf{gen}_{\mathcal{S}}(h))
\to (\rec{\mathcal{S}}(\mathsf{gen}_{\mathcal{S}}(h))= h)
\end{equation*}
Then we obtain that $\rec{\mathcal{S}}$ is a retraction of $\mathsf{gen}_{\mathcal{S}}$ by using this implication and the fact that $\rec{\mathcal{S}}$ is a section of $\mathsf{gen}_{\mathsf{S}}$.
\end{proof}

\begin{lem}\label{lem:pushout_up_htpy}
Let $h,h':A\sqcup^{\mathcal{S}}B\to X$ be two functions. Then we have
\begin{equation*}
(\mathsf{gen}_{\mathcal{S}}(h)=\mathsf{gen}_{\mathcal{S}}(h'))\to (h=h').
\end{equation*}
\end{lem}

\begin{proof}
Suppose we have $\mathsf{gen}_{\mathcal{S}}(h)=\mathsf{gen}_{\mathcal{S}}(h')$. This type of equalities between triples is equivalent to the type of triples $(K,L,M)$ consisting of
\begin{align*}
K & : h\circ \inl \htpy h'\circ \inl \\
L & : h\circ \inr \htpy h'\circ \inr,
\end{align*}
and a homotopy $M$ witnessing that the square 
\begin{equation*}
\begin{tikzcd}
h\circ \inl\circ f \arrow[r,"{K\cdot f}"] \arrow[d,swap,"{h\cdot\glue}"] & h'\circ\inl\circ f \arrow[d,"{h'\cdot\glue}"] \\
h\circ \inr\circ f \arrow[r,swap,"{L\cdot g}"] & h'\circ\inr\circ g
\end{tikzcd}
\end{equation*}
of homotopies commutes. By function extensionality, our goal is equivalent to constructing a homotopy (i.e.~a dependent function) of type
\begin{equation*}
\prd{t:A\sqcup^{\mathcal{S}} B} f(t)=g(t).
\end{equation*}
We will construct such a function by the induction principle for $A\sqcup^{\mathcal{S}} B$. Therefore it suffices to construct
\begin{align*}
K & : h\circ \inl \htpy h'\circ \inl \\
L & : h\circ \inr \htpy h'\circ \inr \\
M' & : \mathsf{tr}_{E_{h,h'}}(\glue,K)=L
\end{align*}
The type of $M'$ is equivalent to the type of $M$, so we obtain the requested structure from our assumptions.
\end{proof}

\begin{comment}
\begin{cor}
Consider two commuting squares
\begin{equation*}
\begin{tikzcd}
S \arrow[r,"g"] \arrow[d,swap,"f"] & B \arrow[d,"j"] & S \arrow[r,"g"] \arrow[d,swap,"f"] & B \arrow[d,"{j'}"] \\
A \arrow[r,swap,"i"] & X & A \arrow[r,swap,"{i'}"] & {X'}
\end{tikzcd}
\end{equation*}
with homotopies $H:i\circ f\htpy j\circ g$ and $H':i'\circ f\htpy j'\circ g$. Furthermore, consider a map
\begin{equation*}
h:X\to X'
\end{equation*}
equipped with
\begin{align*}
K & : h\circ i\htpy i' \\
L & : h\circ j\htpy j' \\
M & : \ct{(h\cdot H)}{(L\cdot g)} \htpy \ct{(K\cdot f)}{H'}.
\end{align*}
If any two of the following three properties hold, then so does the third:
\begin{enumerate}
\item $X$ is a pushout.
\item $X'$ is a pushout.
\item $h$ is an equivalence.
\end{enumerate}
\end{cor}
\end{comment}

As a basic application we establish the universal property of suspensions.
\begin{cor}
Let $X$ and $Y$ be types. Then the map\index{universal property!of suspensions|textit}
\begin{equation*}
(\susp{X}\to Y)\to \sm{y,y':Y} X\to (y=y')
\end{equation*}
given by $f\mapsto (f(\inl(\ttt)),f(\inr(\ttt)),\ap{f}{\glue(\blank)})$ is an equivalence.
\end{cor}

\begin{proof}
We have equivalences
\begin{align*}
(\susp{X}\to Y) & \eqvsym \sm{y,y':\unit \to Y} X\to (y(\ttt)=y'(\ttt)) \\
& \eqvsym \sm{y,y':Y} X\to (y=y').\qedhere
\end{align*}
\end{proof}

\section{The pasting property for pushouts}
\begin{thm}\label{thm:pushout_pasting}
Consider the following configuration of commuting squares:\index{pushout!pasting property|textit}\index{pasting property!for pushouts|textit}
\begin{equation*}
\begin{tikzcd}
A \arrow[r,"i"] \arrow[d,swap,"f"] & B \arrow[r,"k"] \arrow[d,swap,"g"] & C \arrow[d,"h"] \\
X \arrow[r,swap,"j"] & Y \arrow[r,swap,"l"] & Z
\end{tikzcd}
\end{equation*}
with homotopies $H:j\circ f\htpy g\circ i$ and $K:l\circ g\htpy h\circ k$, and suppose that the square on the left is a pushout square. 
Then the square on the right is a pushout square if and only if the outer rectangle is a pushout square.
\end{thm}

\begin{proof}
Let $T$ be a type. Taking the exponent $T^{(\blank)}$ of the entire diagram of the statement of the theorem, we obtain the following commuting diagram
\begin{equation*}
\begin{tikzcd}
T^Z \arrow[r,"\blank\circ l"] \arrow[d,swap,"\blank\circ h"] & T^Y \arrow[d,swap,"\blank\circ g"] \arrow[r,"\blank\circ j"] & T^X \arrow[d,"\blank\circ f"] \\
T^C \arrow[r,swap,"\blank\circ k"] & T^B \arrow[r,swap,"\blank\circ i"] & T^A.
\end{tikzcd}
\end{equation*}
By the assumption that $Y$ is the pushout of $B\leftarrow A \rightarrow X$, it follows that the square on the right is a pullback square. It follows by \autoref{thm:pb_pasting} that the rectangle on the left is a pullback if and only if the outer rectangle is a pullback. Thus the statement follows by the second characterization in \autoref{thm:pushout_up}.
\end{proof}

\begin{lem}
Consider a map $f:A\to B$. Then the cofiber of the map $\inr:B\to \mathsf{cofib}_f$ is equivalent to the suspension $\susp{A}$ of $A$. 
\end{lem}

\begin{exercises}
\item \label{ex:span_rel}Use \cref{thm:choice,thm:fam_proj,cor:times_up_out} to show that the type 
\begin{equation*}
\mathsf{span}(A,B)\defeq \sm{S:\UU} (S\to A)\times (S\to B)
\end{equation*}
of small spans from $A$ to $B$ is equivalent to the type $A\to (B\to\UU)$ of small relations from $A$ to $B$.
\item Use \cref{thm:pushout_up,cor:pb_equiv,ex:equiv_precomp} to show that for any commuting square
\begin{equation*}
\begin{tikzcd}
S \arrow[r,"g"] \arrow[d,swap,"f","{\eqvsym}"'] & B \arrow[d,"j"] \\
A \arrow[r,swap,"i"] & C
\end{tikzcd}
\end{equation*} 
where $f$ is an equivalence, the square is a pushout square if and only if $j:B\to C$ is an equivalence.
Use this observation to conclude the following:
\begin{enumerate}
\item If $X$ is contractible, then $\susp X$ is contractible.
\item The cofiber of any equivalence is contractible.
\item The cofiber of a point in $B$ (i.e.~of a map of the type $\unit\to B$) is equivalent to $B$.
\item There is an equivalence $\eqv{X}{\join{\emptyt}{X}}$.
\item If $X$ is contractible, then $\join{X}{Y}$ is contractible. 
\item If $A$ is contractible, then there is an equivalence $\eqv{A\vee B}{B}$ for any pointed type $B$.
\end{enumerate}
\item Let $P$ and $Q$ be propositions.
\begin{subexenum}
\item Show that $\join{P}{Q}$ satisfies the \emph{universal property of disjunction}, i.e.~that for any proposition $R$, the map
\begin{equation*}
(\join{P}{Q}\to R)\to (P\to R)\times (Q\to R)
\end{equation*}
given by $f\mapsto (f\circ \inl,f\circ \inr)$, is an equivalence.
\item Use the proposition $R\defeq\iscontr(\join{P}{Q})$ to show that $\join{P}{Q}$ is again a proposition.
\end{subexenum}
\item Let $Q$ be a proposition, and let $A$ be a type. Show that $\inr:A\to \join{Q}{A}$ is an equivalence if and only if $Q\to\iscontr(A)$.
\item Let $P$ be a proposition. Show that $\susp P$ is a set, with an equivalence
\begin{equation*}
\eqv{\Big(\inl(\ttt)=\inr(\ttt)\Big)}{P}.
\end{equation*}
\item Show that $\eqv{A\sqcup^{\mathcal{S}} B}{B\sqcup^{\mathcal{S}^{\mathsf{op}}} A}$, where $\mathcal{S^{\mathsf{op}}}\defeq (S,g,f)$ is the \define{opposite span} of $\mathcal{S}$. 
\item Show that if
\begin{equation*}
\begin{tikzcd}
S \arrow[r] \arrow[d] & Y \arrow[d] \\
X \arrow[r] & Z
\end{tikzcd}
\end{equation*}
is a pushout square, then so is
\begin{equation*}
\begin{tikzcd}
A\times S \arrow[r] \arrow[d] & A\times Y \arrow[d] \\
A\times X \arrow[r] & A\times Z
\end{tikzcd}
\end{equation*}
for any type $A$.
\item Show that if
\begin{equation*}
\begin{tikzcd}
S_1 \arrow[r] \arrow[d] & Y_1 \arrow[d] & S_2 \arrow[r] \arrow[d] & Y_2 \arrow[d] \\
X_1 \arrow[r] & Z_1 & X_2 \arrow[r] & Z_2
\end{tikzcd}
\end{equation*}
are pushout squares, then so is
\begin{equation*}
\begin{tikzcd}
S_1+S_2 \arrow[r] \arrow[d] & Y_1+ Y_2 \arrow[d] \\
X_1 +X_2 \arrow[r] & Z_1+Z_2. 
\end{tikzcd}
\end{equation*}
\item 
\begin{subexenum}
\item Consider a span $(S,f,g)$ from $A$ to $B$. Show that the square
\begin{equation*}
\begin{tikzcd}[column sep=large]
S+S \arrow[d,swap,"{f+g}"] \arrow[r,"{[\idfunc,\idfunc]}"] & S \arrow[d,"{\inr\circ g}"] \\
A+B \arrow[r,swap,"{[\inl,\inr]}"] & A\sqcup^\mathcal{S} B
\end{tikzcd}
\end{equation*}
is again a pushout square.
\item Show that $\eqv{\susp X}{\join{\bool}{X}}$.
\end{subexenum}
\item Consider a commuting triangle
\begin{equation*}
\begin{tikzcd}[column sep=tiny]
A \arrow[rr,"h"] \arrow[dr,swap,"f"] & & B \arrow[dl,"g"] \\
& X
\end{tikzcd}
\end{equation*}
with $H:f\htpy g\circ h$. 
\begin{subexenum}
\item Construct a map $\mathsf{cofib}_{(h,H)}: \mathsf{cofib}_{g}\to \mathsf{cofib}_f$.
\item Show that $\eqv{\mathsf{cofib}_{\mathsf{cofib}(h,H)}}{\mathsf{cofib}_h}$.
\end{subexenum}
\item Use \cref{ex:circle_connected} to show that for $n\geq 0$, $X$ is an $n$-type if and only if the map
\begin{equation*}
\lam{x}\mathsf{const}_x : X \to (\sphere{n+1}\to X)
\end{equation*}
is an equivalence.
\item 
\begin{subexenum}
\item Construct a function
\begin{equation*}
(X\to Y)\to (\susp X\to \susp Y),
\end{equation*}
for any two types $X$ and $Y$. Thus we obtain a function $\susp f:\susp X \to \susp Y$, for any $f:X\to Y$.
\item Show that if $f\htpy g$, then $\susp f \htpy \susp g$. 
\item Show that $\susp \idfunc[X]\htpy\idfunc[\susp X]$
\item Show that
\begin{equation*}
\susp(g\circ f)\htpy (\susp g)\circ (\susp f).
\end{equation*}
for any $f:X\to Y$ and $g:Y\to Z$.
\end{subexenum}
\item 
\begin{subexenum}
\item Let $I$ be a type, and let $A$ be a family over $I$. Construct an equivalence
\begin{equation*}
\eqv{\Big(\bigvee\nolimits_{(i:I)}\susp A_i\Big)}{\susp\Big(\bigvee\nolimits_{(i:I)}A_i\Big)}.
\end{equation*}
\item Show that for any type $X$ there is an equivalence
\begin{equation*}
\eqv{\Big(\bigvee\nolimits_{(x:X)}\bool\Big)}{X+1}.
\end{equation*}
\item Construct an equivalence
\begin{equation*}
\eqv{\susp(\mathsf{Fin}(n+1))}{\bigvee\nolimits_{(i:\mathsf{Fin}(n))}\sphere{1}}.
\end{equation*}
\end{subexenum}
\end{exercises}

\chapter{Cubical diagrams}

\section{Commuting cubes}
\begin{defn}\label{defn:cube}
A \define{commuting cube}\index{commuting cube|textbf}
\begin{equation*}
\begin{tikzcd}
& A_{111} \arrow[dl] \arrow[dr] \arrow[d] \\
A_{110} \arrow[d] & A_{101} \arrow[dl] \arrow[dr] & A_{011} \arrow[dl,crossing over] \arrow[d] \\
A_{100} \arrow[dr] & A_{010} \arrow[d] \arrow[from=ul,crossing over] & A_{001} \arrow[dl] \\
& A_{000},
\end{tikzcd}
\end{equation*}
consists of 
\begin{enumerate}
\item types
\begin{equation*}
A_{111},A_{110},A_{101},A_{011},A_{100},A_{010},A_{001},A_{000},
\end{equation*}
\item \begin{samepage}%
maps
\begin{align*}
f_{11\check{1}} & : A_{111}\to A_{110} & f_{\check{1}01} & : A_{101}\to A_{001} \\
f_{1\check{1}1} & : A_{111}\to A_{101} & f_{01\check{1}} & : A_{011}\to A_{010} \\
f_{\check{1}11} & : A_{111}\to A_{011} & f_{0\check{1}1} & : A_{011}\to A_{001} \\
f_{1\check{1}0} & : A_{110}\to A_{100} & f_{\check{1}00} & : A_{100}\to A_{000} \\
f_{\check{1}10} & : A_{110}\to A_{010} & f_{0\check{1}0} & : A_{010}\to A_{000} \\
f_{10\check{1}} & : A_{101}\to A_{100} & f_{00\check{1}} & : A_{001}\to A_{000},
\end{align*}
\end{samepage}%
\item homotopies
\begin{align*}
H_{1\check{1}\check{1}} & : f_{1\check{1}0}\circ f_{11\check{1}} \htpy f_{10\check{1}}\circ f_{1\check{1}1} & H_{0\check{1}\check{1}} & : f_{0\check{1}0}\circ f_{01\check{1}} \htpy f_{00\check{1}}\circ f_{0\check{1}1} \\
H_{\check{1}1\check{1}} & : f_{\check{1}10}\circ f_{11\check{1}} \htpy f_{01\check{1}}\circ f_{\check{1}11} & H_{\check{1}0\check{1}} & : f_{\check{1}00}\circ f_{10\check{1}} \htpy f_{00\check{1}}\circ f_{\check{1}01} \\
H_{\check{1}\check{1}1} & : f_{\check{1}01}\circ f_{1\check{1}1} \htpy f_{0\check{1}1}\circ f_{\check{1}11} & H_{\check{1}\check{1}0} & : f_{\check{1}00}\circ f_{1\check{1}0} \htpy f_{0\check{1}0}\circ f_{\check{1}10},
\end{align*}
\item and a homotopy 
\begin{align*}
C & : \ct{(f_{\check{1}00}\cdot H_{1\check{1}\check{1}})}{(\ct{(H_{\check{1}0\check{1}}\cdot f_{1\check{1}1})}{(f_{00\check{1}}\cdot H_{\check{1}\check{1}1})})} \\
& \qquad \htpy \ct{(H_{\check{1}\check{1}0}\cdot f_{11\check{1}})}{(\ct{(f_{0\check{1}0}\cdot H_{\check{1}1\check{1}})}{(H_{0\check{1}\check{1}}\cdot f_{\check{1}11})})}
\end{align*}
filling the cube.
\end{enumerate}
\end{defn}

\begin{lem}
Given a commuting cube as in \cref{defn:cube} we obtain a commuting square
\begin{equation*}
\begin{tikzcd}
\fib{f_{1\check{1}1}}{x} \arrow[r] \arrow[d] & \fib{f_{0\check{1}1}}{f_{\check{1}01}(x)} \arrow[d] \\
\fib{f_{1\check{1}0}}{f_{10\check{1}}(x)} \arrow[r] & \fib{f_{0\check{1}0}}{f_{00\check{1}}(x)}
\end{tikzcd}
\end{equation*}
for any $x:A_{101}$. 
\end{lem}

\begin{lem}
Consider a commuting cube
\begin{equation*}
\begin{tikzcd}
& A_{111} \arrow[dl] \arrow[dr] \arrow[d] \\
A_{110} \arrow[d] & A_{101} \arrow[dl] \arrow[dr] & A_{011} \arrow[dl,crossing over] \arrow[d] \\
A_{100} \arrow[dr] & A_{010} \arrow[d] \arrow[from=ul,crossing over] & A_{001} \arrow[dl] \\
& A_{000}.
\end{tikzcd}
\end{equation*}
If the bottom and front right squares are pullback squares, then the back left square is a pullback if and only if the top square is.
\end{lem}

\begin{rmk}
By rotating the cube we also obtain:
\begin{enumerate}
\item If the bottom and front left squares are pullback squares, then the back right square is a pullback if and only if the top square is.
\item If the front left and front right squares are pullback, then the back left square is a pullback if and only if the back right square is.
\end{enumerate}
By combining these statements it also follows that if the front left, front right, and bottom squares are pullback squares, then if any of the remaining three squares are pullback squares, all of them are. Cubes that consist entirely of pullback squares are sometimes called \define{strongly cartesian}\index{strongly cartesian cube}.
\end{rmk}

\section{Fiberwise pullbacks}

\begin{lem}
Consider a pullback square\index{pullback!Sigma-type of pullbacks@{$\Sigma$-type of pullbacks}}
  \begin{equation*}
    \begin{tikzcd}
      C \arrow[r,"q"] \arrow[d,swap,"p"] & B \arrow[d,"g"] \\
      A \arrow[r,swap,"f"] & X
    \end{tikzcd}
  \end{equation*}
  with $H : f \circ p \htpy g \circ h$. Furthermore, consider type families $P_X$, $P_A$, $P_B$, and $P_C$ over $X$, $A$, $B$, and $C$ respectively, equipped with families of maps
  \begin{align*}
    f' & : \prd{a:A} P_A(a) \to P_X(f(a)) \\
    g' & : \prd{b:B} P_B(b) \to P_X(g(b)) \\
    p' & : \prd{c:C} P_C(c) \to P_A(p(c)) \\
    q' & : \prd{c:C} P_C(c) \to P_B(q(c)),
  \end{align*}
  and for each $c:C$ a homotopy $H'_c$ witnessing that the square
  \begin{equation}\label{eq:family-squares-pullback}
    \begin{tikzcd}
      P_C(c) \arrow[rr,"{q'_c}"] \arrow[d,swap,"{p'_c}"] & &[3em] P_B(q(c)) \arrow[d,"{g'_{q(c)}}"] \\
      P_A(p(c)) \arrow[r,swap,"{f'_{p(c)}}"] & P_X(f(p(c))) \arrow[r,swap,"{\tr_{P_X}(H(c))}"] & P_X(g(q(c)))
    \end{tikzcd}
  \end{equation}
  commutes. Then the following are equivalent:
  \begin{enumerate}
  \item For each $c:C$ the square in \cref{eq:family-squares-pullback} is a pullback square.
  \item The square
    \begin{equation}\label{eq:total-square-pullback}
      \begin{tikzcd}[column sep=huge]
        \sm{c:C}P_C(c)
        \arrow[r,"{\total[q]{q'}}"] \arrow[d,swap,"{\total[p]{p'}}"] &
        \sm{b:B}P_B(b) \arrow[d,"{\total[g]{g'}}"] \\
        \sm{a:A}P_A(a) \arrow[r,swap,"{\total[f]{f'}}"] & \sm{x:X}P_X(x)
      \end{tikzcd}
    \end{equation}
    is a pullback square.
  \end{enumerate}
\end{lem}


\begin{cor}
Consider a pullback square
\begin{equation*}
\begin{tikzcd}
C \arrow[r,"q"] \arrow[d,swap,"p"] & B \arrow[d,"g"] \\
A \arrow[r,swap,"f"] & X,
\end{tikzcd}
\end{equation*}
with $H:f\circ p\htpy g\circ q$, and let $c_1,c_2:C$. Then the square
\begin{equation*}
\begin{tikzcd}[column sep=8em]
(c_1=c_2) \arrow[r,"\apfunc{q}"] \arrow[d,swap,"\apfunc{p}"] & (q(c_1)=q(c_2)) \arrow[d,"\lam{\beta}\ct{H(c_1)}{\ap{g}{\beta}}"] \\
(p(c_1)=p(c_2)) \arrow[r,swap,"\lam{\alpha}\ct{\ap{f}{\alpha}}{H(c_2)}"] & f(p(c_1))=g(q(c_2)),
\end{tikzcd}
\end{equation*}
commutes and is a pullback square.
\end{cor}


\begin{thm}
  Consider a commuting cube
  \begin{equation*}
    \begin{tikzcd}
      & C' \arrow[dl] \arrow[dr] \arrow[d] \\
      A' \arrow[d] & C \arrow[dl] \arrow[dr] & B' \arrow[crossing over,dl] \arrow[d] \\
      A \arrow[dr] & X' \arrow[d] \arrow[from=ul,crossing over] & B \arrow[dl] \\
      & X
    \end{tikzcd}
  \end{equation*}
  in which the bottom square is a pullback square. Then the following are equivalent:
  \begin{enumerate}
  \item The top square is a pullback square.
  \item The square
    \begin{equation*}
      \begin{tikzcd}
        \fib{\gamma}{c} \arrow[d] \arrow[r] & \fib{\beta}{q(c)} \arrow[d] \\
        \fib{\alpha}{p(c)} \arrow[r] & \fib{\varphi}{f(p(c))}
      \end{tikzcd}
    \end{equation*}
    is a pullback square for each $c:C$.
  \end{enumerate}
\end{thm}


\section{The 3-by-3-properties for pullbacks and pushouts}

\begin{thm}
  Consider a commuting diagram of the form
  \begin{equation*}
    \begin{tikzcd}
      A_0 \arrow[r] \arrow[d] & B_0 \arrow[d] & C_0 \arrow[l] \arrow[d] \\
      A_1 \arrow[r] & B_1 & C_1 \arrow[l] \\
      A_2 \arrow[u] \arrow[r] & B_2 \arrow[u] & C_2 \arrow[u] \arrow[l]
    \end{tikzcd}
  \end{equation*}
  with homotopies filling the (small) squares. Furthermore, consider
  pullback squares
  \begin{equation*}
    \begin{tikzcd}
      D_0 \arrow[r] \arrow[d] & C_0 \arrow[d] & D_1 \arrow[r] \arrow[d] & C_1 \arrow[d] & D_2 \arrow[r] \arrow[d] & C_2 \arrow[d] \\
      A_0 \arrow[r] & B_0 & A_1 \arrow[r] & B_1 & A_2 \arrow[r] & B_2
    \end{tikzcd}
  \end{equation*}
  \begin{equation*}
    \begin{tikzcd}
      A_3 \arrow[r] \arrow[d] & A_2 \arrow[d] & B_3 \arrow[r] \arrow[d] & B_2 \arrow[d] & C_3 \arrow[r] \arrow[d] & C_2 \arrow[d] \\
      A_0 \arrow[r] & A_1 & B_0 \arrow[r] & B_1 & C_0 \arrow[r] & C_1.
    \end{tikzcd}
  \end{equation*}
  Finally, consider a commuting square
  \begin{equation*}
    \begin{tikzcd}
      D_3 \arrow[r] \arrow[d] & D_2 \arrow[d] \\
      D_0 \arrow[r] & D_1.
    \end{tikzcd}
  \end{equation*}
  Then the following are equivalent:
  \begin{enumerate}
  \item This square is a pullback square.
  \item The induced square
    \begin{equation*}
      \begin{tikzcd}
        D_3 \arrow[r] \arrow[d] & C_3 \arrow[d] \\
        A_3 \arrow[r] & B_3
      \end{tikzcd}
    \end{equation*}
    is a pullback square.
  \end{enumerate}
\end{thm}


\begin{exercises}
\item Some exercises.
\end{exercises}

\chapter{Universality and descent for pushouts}\label{chap:descent}

We begin this lecture with the idea that pushouts can be presented as higher inductive types. The general idea behind higher inductive types is that we can introduce new inductive types not only with constructors at the level of points, but also with constructors at the level of identifications. Pushouts form a basic class of examples that can be obtained as higher inductive types, because they come equipped with the structure of a cocone. The cocone $(i,j,H)$ in the commuting square
\begin{equation*}
  \begin{tikzcd}
    S \arrow[r,"g"] \arrow[d,swap,"f"] & B \arrow[d,"j"] \\
    A \arrow[r,swap,"i"] & C
  \end{tikzcd}
\end{equation*}
equips the type $C$ with two \emph{point constructors}
\begin{align*}
  i & : A \to C \\
  j & : B \to C \\
  \intertext{and a \emph{path constructor}}
  H & : \prd{s:S}i(f(s)) = j(g(s))
\end{align*}
that provides an identifcation $H(s):i(f(s))=j(g(s))$ for every $s:S$. The induction principle then specifies how to construct sections of families over $C$. Naturally, it takes not only the point constructors $i$ and $j$, but also the path constructor $H$ into account. 

The induction principle is one of several equivalent characterizations of pushouts. We will prove a theorem providing five equivalent characterizations of homotopy pushouts. Two of those we have already seen in \cref{thm:pushout-up}: the universal property and the pullback property. The other three are
\begin{enumerate}
\item the \emph{dependent pullback property},
\item the \emph{dependent universal property},
\item the \emph{induction principle}.
\end{enumerate}

An implication that is particularly useful among our five characterizations of pushouts, is the fact that the pullback property implies the dependent pullback property. We use the dependent pullback property to derive the \emph{universality of pushouts} (not to be confused with the universal property of pushouts), showing that for any commuting cube
\begin{equation*}
  \begin{tikzcd}
    & S' \arrow[dl] \arrow[d] \arrow[dr] \\
    A' \arrow[d] & S \arrow[dl] \arrow[dr] & B' \arrow[d] \arrow[dl,crossing over] \\
    A \arrow[dr] & D' \arrow[from=ul,crossing over] \arrow[d] & B \arrow[dl] \\
    & D
  \end{tikzcd}
\end{equation*}
in which the back left and right squares are pullback squares, if the front left and right squares are also pullback squares, then so is the induced square
\begin{equation*}
  \begin{tikzcd}
    A'\sqcup^{\mathcal{S}'}B' \arrow[r,densely dotted] \arrow[d,densely dotted] & D' \arrow[d] \\
    A\sqcup^S B \arrow[r,densely dotted] & D
  \end{tikzcd}
\end{equation*}

We then observe that the univalence axiom can be used together with the universal property of pushouts to obtain such families over pushouts in the first place. We prove the descent theorem, which asserts that for any diagram of the form
\begin{equation*}
  \begin{tikzcd}
    & S' \arrow[dl] \arrow[d] \arrow[dr] \\
    A' \arrow[d] & S \arrow[dl] \arrow[dr] & B' \arrow[d] \\
    A \arrow[dr] & & B \arrow[dl] \\
    & C
  \end{tikzcd}
\end{equation*}
in which the bottom square is a pushout square and the back left and right squares are pullback squares, there is a unique way of extending this to a commuting cube
\begin{equation*}
  \begin{tikzcd}
    & S' \arrow[dl] \arrow[d] \arrow[dr] \\
    A' \arrow[d] & S \arrow[dl] \arrow[dr] & B' \arrow[d] \\
    A \arrow[dr] & C' \arrow[from=ul,crossing over,densely dotted] \arrow[from=ur,crossing over,densely dotted] \arrow[d,densely dotted] & B \arrow[dl] \\
    & C
  \end{tikzcd}
\end{equation*}
in which also the front left and right squares are pullback squares. Thus the converse of the universality theorem for pushouts also follows. The descent property used to show that pullbacks distribute over pushouts, and to compute the fibers of maps out of pushouts (the source of many exercises).

We note that the computation rules in our treatment for the induction principle of homotopy pushouts are weak. In other words, they are identifications. In this course we have no need for judgmental computation rules. Our focus is instead on universal properties. We refer the reader who is interested in the more `traditional' higher inductive types with judgmental computation rules to \cite{hottbook}.

\section{The induction principle of pushouts}

The pushout $A\sqcup^{\mathcal{S}} B$ of a span $\mathcal{S}$ from $A$ to $B$ comes equipped with the structure of a cocone\index{inl@{$\inl$}!for pushouts}\index{inr@{$\inr$}!for pushouts}\index{glue@{$\glue$}}
\begin{align*}
\inl & : A \to A \sqcup^{\mathcal{S}} B \\
\inr & : B \to A \sqcup^{\mathcal{S}} B \\
\glue & : \prd{x:S} \inl(f(x))=\inr(g(x))
\end{align*}


To see what the induction principle has to be, consider first a dependent function $s:\prd{x:A\sqcup^{\mathcal{S}}B}P(x)$. When we evaluate this function at the constructors, we obtain
\begin{align*}
s\circ \inl & : \prd{a:A} P(\inl(a)) \\
s\circ \inr & : \prd{b:B} P(\inr(b)) \\
\apdfunc{s}\circ \glue & : \prd{x:S} \mathsf{tr}_P(\glue(x),s(f(x)))=s(g(x)).
\end{align*}

\begin{defn}
Consider a span $\mathcal{S}\jdeq (S,f,g)$ from $A$ to $B$, and let $P$ be a family over $A\sqcup^{\mathcal{S}} B$. The \define{dependent action on generators}\index{dependent action on generators!for pushouts|textbf} is defined to be the map\index{dgen_S@{$\mathsf{dgen}_{\mathcal{S}}$}|textbf}
\begin{align*}
\mathsf{dgen}_{\mathcal{S}}^P & : \Big(\prd{x:A\sqcup^{\mathcal{S}} B} P(x)\Big) \to \Big(\sm{f': \prd{a:A}P(\inl(a))}{g':\prd{b:B}P(\inr(b))}\Big.\\
& \qquad\qquad\qquad\qquad\qquad\qquad \Big.\prd{x:S} \mathsf{tr}_P(\glue(x),f'(f(x)))=g'(g(x))\Big).
\end{align*}
given by $s\mapsto (s\circ\inl,s\circ\inr,\apdfunc{s}\circ\glue)$.
\end{defn}

We can now fully specify homotopy pushouts.

\begin{defn}
Given a span $\mathcal{S}\jdeq (S,f,g)$, the \define{(homotopy) pushout}\index{pushout|textbf} $A\sqcup^{\mathcal{S}} B$ of $\mathcal{S}$ is defined to be the higher inductive\index{higher inductive types} type equipped with\index{inl@{$\inl$}!for pushouts|textbf}\index{inr@{$\inr$}!for pushouts|textbf}\index{glue@{$\glue$}|textbf}
\begin{align*}
\inl & : A \to A \sqcup^{\mathcal{S}} B \\
\inr & : B \to A \sqcup^{\mathcal{S}} B \\
\glue & : \prd{x:S} \inl(f(x))=\inr(g(x)),
\end{align*}
satisfying the \define{induction principle} for pushouts\index{induction principle!for pushouts|textbf}, which asserts that for each type family $P$ over $A\sqcup^{\mathcal{S}} B$ the map $\mathsf{dgen}_{\mathcal{S}}^P$ has a section.
\end{defn}

\begin{rmk}
The induction principle of the pushout $A\sqcup^{\mathcal{S}} B$ provides us with a dependent function
\begin{equation*}
\ind{\mathcal{S}}(f',g',G) : \prd{x:A\sqcup^{\mathcal{S}} B} P(x),
\end{equation*}
for every
\begin{align*}
f' & : \prd{a:A}P(\inl(a)) \\
g' & : \prd{b:B}P(\inr(b)) \\
G & : \prd{x:S} \mathsf{tr}_P(\glue(x),f'(f(x)))=g'(g(x))
\end{align*}
Moreover, the function $\ind{\mathcal{S}}(f',g',G)$ comes equipped with an identification
\begin{equation*}
\mathsf{dgen}_{\mathcal{S}}(\ind{\mathcal{S}}(f',g',G))=(f',g',G).
\end{equation*}
Writing $s\defeq \ind{\mathcal{S}}(f',g',G)$, we see that such an identification between triples is equivalently described by a triple $(H,K,L)$ consisting of
\begin{align*}
H : s\circ \inl \htpy f' \\
K : s\circ\inr \htpy g' 
\end{align*}
and a homotopy $L$ witnessing that the square
\begin{equation*}
\begin{tikzcd}[column sep=8em]
\mathsf{tr}_{P}(\glue(x),s(\inl(f(x)))) \arrow[r,equals,"\ap{\mathsf{tr}_P(\glue(x))}{H(x)}"] \arrow[d,equals,swap,"\apd{s}{\glue(x)}"] & \mathsf{tr}_{P}(\glue(x),f'(f(x))) \arrow[d,equals,"G(x)"] \\
s(\inr(g(x))) \arrow[r,equals,swap,"K(x)"] & g'(g(x))
\end{tikzcd}
\end{equation*}
commutes, for every $x:S$. These are the \define{computation rules} for pushouts\index{computation rules!for pushouts}.
\end{rmk}

\begin{thm}\label{defn:dependent-pullback-property-pushout}
  Consider a commuting square
  \begin{equation}\label{eq:dppp1}
    \begin{tikzcd}
      S \arrow[d,swap,"f"] \arrow[r,"g"] & B \arrow[d,"j"] \\
      A \arrow[r,swap,"i"] & C
    \end{tikzcd}
  \end{equation}
  with $H:(i\circ f) \htpy (j \circ g)$. Then the following are equivalent:
  \begin{enumerate}
  \item The square in \cref{eq:dppp1} is a pushout square.
  \item The square in \cref{eq:dppp1} satisfies the pullback property of pushouts.
  \item The square satisfies the \define{dependent pullback property} of pushouts: For every family $P$ over $C$, the square
    \begin{equation}\label{eq:dppp2}
      \begin{tikzcd}[column sep=10em]
        \prd{z:C}P(z) \arrow[r,"h\mapsto h\circ j"] \arrow[d,swap,"h\mapsto h\circ i"] & \prd{y:B}P(j(y)) \arrow[d,"h\mapsto h\circ g"] \\
        \prd{x:A}P(i(x)) \arrow[r,swap,"{h\mapsto \lam{s}\mathsf{tr}_P(H(s),h(f(s)))}"] & \prd{s:S}P(j(g(s))),
      \end{tikzcd}
    \end{equation}
    which commutes by the homotopy
    \begin{equation*}
      \lam{h}\mathsf{eq\usc{}htpy}(\lam{s}\apd{h}{H(s)}),
    \end{equation*}
    is a pullback square.
  \item The type $C$ satisfies the \define{dependent universal property} of pushouts: For any family $P$ over $C$, the map
    \begin{equation*}
      \mathsf{gen}_P : \Big(\prd{c:C}P(c)\Big)\to \sm{i':\prd{a:A}P(i(a))}{j':\prd{b:B}P(j(b))}\prd{s:S}\mathsf{tr}_P(H(s),j'(f(s)))=j'(g(s))
    \end{equation*}
    is an equivalence.
  \item The type $C$ satisfies the \define{induction principle} of pushouts: For any family $P$ over $C$ the map $\mathsf{gen}_P$ has a section.
  \end{enumerate}
\end{thm}

\begin{proof}
  We have already seen in \cref{thm:pushout_up} that (i) $\Leftrightarrow$ (ii).
  Next, we show that (ii) $\Rightarrow$ (iii). To see this, note that we have a commuting cube
  \begin{equation*}
    \begin{tikzcd}[column sep=tiny]
      &  \sm{h:C\to C}\prd{c:C}P(h(c)) \arrow[dl] \arrow[d] \arrow[dr] \\
      \sm{h:A\to C}\prd{a:A}P(h(a)) \arrow[d] & \big(\sm{c:C}P(c)\big)^C \arrow[dl] \arrow[dr] & \sm{h:B\to C}\prd{b:B}P(h(b)) \arrow[d] \arrow[dl,crossing over] \\
      \big(\sm{c:C}P(c)\big)^A \arrow[dr] & \sm{h:S\to C}\prd{s:S}P(h(s)) \arrow[d] \arrow[from=ul,crossing over] & \big(\sm{c:C}P(c)\big)^B \arrow[dl] \\
      & \big(\sm{c:C}P(c)\big)^S
    \end{tikzcd}
  \end{equation*}
  in which the vertical maps are equivalences. Moreover, the bottom square is a pullback square by the pullback property of pushouts, so we conclude that the top square is a pullback square. Since this is a square of total spaces over a pullback square, we invoke \cref{lem:fiberwise-pullback} to conclude that for each $h:C \to C$, the square
  \begin{equation*}
    \begin{tikzcd}
      \prd{c:C}P(h(c)) \arrow[r] \arrow[d] & \prd{b:B} P(h(j(b))) \arrow[d] \\
      \prd{a:A}P(h(i(a))) \arrow[r] & \prd{s:S}P(h(j(g(s))))
    \end{tikzcd}
  \end{equation*}
  is a pullback square. We use the case $h\jdeq\idfunc:C\to C$ to conclude that
  the dependent pullback property holds.

  To see that (iii) implies (ii) we recall that transport with respect to a trivial family is homotopic to the identity function. Thus we obtain the pullback property from the dependent pullback property using the trivial family $\lam{c}T$ over $C$.

  To see that (iii) $\Leftrightarrow$ (iv) we note that 
\end{proof}

\section{Type families over pushouts}

Given a pushout square
\begin{equation*}
\begin{tikzcd}
S \arrow[r,"g"] \arrow[d,swap,"f"] & B \arrow[d,"j"] \\
A \arrow[r,swap,"i"] & X.
\end{tikzcd}
\end{equation*}
with $H:i\circ f\htpy j\circ g$, and a family $P:X\to\UU$, we obtain
\begin{align*}
P\circ i & : A \to \UU \\
P\circ j & : B \to \UU \\
\lam{x}\mathsf{tr}_P(H(x)) & : \prd{x:S} \eqv{P(i(f(x)))}{P(j(g(x)))}.
\end{align*}
Our goal in the current section is to show that the triple $(P_A,P_B,P_S)$ consisting of $P_A\defeq P\circ i$, $P_B\defeq P\circ j$, and $P_S\defeq \lam{x}\mathsf{tr}_P(H(x))$ characterizes the family $P$ over $X$.

\begin{defn}
Consider a commuting square
\begin{equation*}
\begin{tikzcd}
S \arrow[r,"g"] \arrow[d,swap,"f"] & B \arrow[d,"j"] \\
A \arrow[r,swap,"i"] & X.
\end{tikzcd}
\end{equation*}
with $H:i\circ f\htpy j\circ g$, where all types involved are in $\UU$. The type $\mathsf{Desc}(\mathcal{S})$\index{Desc@{$\mathsf{Desc}(\mathcal{S})$}|textbf} of \define{descent data}\index{descent data|textbf} for $X$, is defined defined to be the type of triples $(P_A,P_B,P_S)$ consisting of
\begin{align*}
P_A & : A \to \UU \\
P_B & : B \to \UU \\
P_S & : \prd{x:S} \eqv{P_A(f(x))}{P_B(g(x))}.
\end{align*}
\end{defn}

\begin{defn}
Given a commuting square
\begin{equation*}
\begin{tikzcd}
S \arrow[r,"g"] \arrow[d,swap,"f"] & B \arrow[d,"j"] \\
A \arrow[r,swap,"i"] & X.
\end{tikzcd}
\end{equation*}
with $H:i\circ f\htpy j\circ g$, we define the map\index{desc_fam@{$\mathsf{desc\usc{}fam}_{\mathcal{S}}$}|textbf}
\begin{equation*}
\mathsf{desc\usc{}fam}_{\mathcal{S}}(i,j,H) : (X\to \UU)\to \mathsf{Desc}(\mathcal{S})
\end{equation*}
by $P\mapsto (P\circ i,P\circ j,\lam{x}\mathsf{tr}_P(H(x)))$.
\end{defn}

\begin{thm}\label{thm:desc_fam}
Consider a pushout square
\begin{equation*}
\begin{tikzcd}
S \arrow[r,"g"] \arrow[d,swap,"f"] & B \arrow[d,"j"] \\
A \arrow[r,swap,"i"] & X.
\end{tikzcd}
\end{equation*}
with $H:i\circ f\htpy j\circ g$, where all types involved are in $\UU$, and suppose we have
\begin{align*}
P_A & : A \to \UU \\
P_B & : B \to \UU \\
P_S & : \prd{x:S} \eqv{P_A(f(x))}{P_B(g(x))}.
\end{align*}
Then the function\index{desc_fam@{$\mathsf{desc\usc{}fam}_{\mathcal{S}}$}!is an equivalence|textit}
\begin{equation*}
\mathsf{desc\usc{}fam}_{\mathcal{S}}(i,j,H) : (X\to \UU)\to \mathsf{Desc}(\mathcal{S})
\end{equation*}
is an equivalence.
\end{thm}

\begin{proof}
By the 3-for-2 property of equivalences it suffices to construct an equivalence $\varphi:\mathsf{cocone}_{\mathcal{S}}(\UU)\to\mathsf{Desc}(\mathcal{S})$ such that the triangle
\begin{equation*}
\begin{tikzcd}[column sep=tiny]
& \UU^X \arrow[dl,swap,"{\mathsf{cocone\usc{}map}_{\mathcal{S}}(i,j,H)}"] \arrow[dr,"{\mathsf{desc\usc{}fam}_{\mathcal{S}}(i,j,H)}"] & \phantom{\mathsf{cocone}_{\mathcal{S}}(\UU)} \\
\mathsf{cocone}_{\mathcal{S}}(\UU) \arrow[rr,densely dotted,"\eqvsym","\varphi"'] & & \mathsf{Desc}(\mathcal{S})
\end{tikzcd}
\end{equation*}
commutes.

Since we have equivalences
\begin{equation*}
\mathsf{equiv\usc{}eq}:\eqv{\Big(P_A(f(x))=P_B(g(x))\Big)}{\Big(\eqv{P_A(f(x))}{P_B(g(x))}\Big)}
\end{equation*}
for all $x:S$, we obtain by \cref{ex:equiv_pi} an equivalence on the dependent products
\begin{equation*}
{\Big(\prd{x:S}P_A(f(x))=P_B(g(x))\Big)}\to{\Big(\prd{x:S}\eqv{P_A(f(x))}{P_B(g(x))}\Big)}.
\end{equation*}
We define $\varphi$ to be the induced map on total spaces. Explicitly, we have
\begin{equation*}
\varphi\defeq \lam{(P_A,P_B,K)}(P_A,P_B,\lam{x}\mathsf{equiv\usc{}eq}(K(x))).
\end{equation*}
Then $\varphi$ is an equivalence by \cref{thm:fib_equiv}, and the triangle commutes by \cref{ex:tr_ap}.
\end{proof}

\begin{cor}\label{cor:desc_fam}
Consider descent data $(P_A,P_B,P_S)$ for a pushout square as in \cref{thm:desc_fam}.
Then the type of quadruples $(P,e_A,e_B,e_S)$ consisting of a family $P:X\to\UU$ equipped with fiberwise equivalences
\begin{samepage}
\begin{align*}
e_A & : \prd{a:A}\eqv{P_A(a)}{P(i(a))} \\
e_B & : \prd{b:B}\eqv{P_B(a)}{P(j(b))}
\end{align*}
\end{samepage}%
and a homotopy $e_S$ witnessing that the square
\begin{equation*}
\begin{tikzcd}[column sep=huge]
P_A(f(x)) \arrow[r,"e_A(f(x))"] \arrow[d,swap,"P_S(x)"] & P(i(f(x))) \arrow[d,"\mathsf{tr}_P(H(x))"] \\
P_B(g(x)) \arrow[r,swap,"e_B(g(x))"] & P(j(g(x)))
\end{tikzcd}
\end{equation*}
commutes, is contractible.
\end{cor}

\begin{proof}
The fiber of this map at $(P_A,P_B,P_S)$ is equivalent to the type of quadruples $(P,e_A,e_B,e_S)$ as described in the theorem, which are contractible by \cref{thm:contr_equiv}.
\end{proof}

\section{The flattening lemma for pushouts}

In this section we consider a pushout square
\begin{equation*}
\begin{tikzcd}
S \arrow[r,"g"] \arrow[d,swap,"f"] & B \arrow[d,"j"] \\
A \arrow[r,swap,"i"] & X.
\end{tikzcd}
\end{equation*}
with $H:i\circ f\htpy j\circ g$, descent data
\begin{align*}
P_A & : A \to \UU \\
P_B & : B \to \UU \\
P_S & : \prd{x:S} \eqv{P_A(f(x))}{P_B(g(x))},
\end{align*}
and a family $P:X\to\UU$ equipped with 
\begin{align*}
e_A & : \prd{a:A}\eqv{P_A(a)}{P(i(a))} \\
e_B & : \prd{b:B}\eqv{P_B(a)}{P(j(b))}
\end{align*}
and a homotopy $e_S$ witnessing that the square
\begin{equation*}
\begin{tikzcd}[column sep=huge]
P_A(f(x)) \arrow[r,"e_A(f(x))"] \arrow[d,swap,"P_S(x)"] & P(i(f(x))) \arrow[d,"\mathsf{tr}_P(H(x))"] \\
P_B(g(x)) \arrow[r,swap,"e_B(g(x))"] & P(j(g(x)))
\end{tikzcd}
\end{equation*}
commutes.

\begin{defn}
We define a commuting square
\begin{equation*}
\begin{tikzcd}
\sm{x:S}P_A(f(x)) \arrow[d,swap,"{f'}"] \arrow[r,"{g'}"] & \sm{b:B}P_B(b) \arrow[d,"{j'}"] \\
\sm{a:A}P_A(a) \arrow[r,swap,"{i'}"] & \sm{x:X}P(x)
\end{tikzcd}
\end{equation*}
with a homotopy $H':i'\circ f'\htpy j'\circ g'$.
\end{defn}

\begin{constr}
We define
\begin{align*}
f' & \defeq \total[f]{\lam{x}\idfunc[P_A(f(x))]} \\
g' & \defeq \total[g]{e_S} \\
i' & \defeq \total[i]{e_A} \\
j' & \defeq \total[j]{e_B}.
\end{align*}
The remaining goal is to construct a homotopy $H':i'\circ f'\htpy j'\circ g'$. Thus, we have to show that
\begin{equation*}
(i(f(x)),e_A(y))=(j(g(x)),e_B(e_S(y)))
\end{equation*}
for any $x:S$ and $y:P_A(f(x))$. We have he identification
\begin{equation*}
\mathsf{eq\usc{}pair}(H(x),e_S(x,y)^{-1})
\end{equation*}
of this type.
\end{constr}

\begin{defn}
We will write $\mathcal{S'}$ for the span
\begin{equation*}
\begin{tikzcd}
\sm{a:A}P_A(a) & \sm{x:S}P_A(f(x)) \arrow[l,swap,"{f'}"] \arrow[r,"{g'}"] & \sm{b:B}P_B(b).
\end{tikzcd}
\end{equation*}
\end{defn}

\begin{lem}[The flattening lemma]\label{lem:flattening}
The commuting square\index{flattening lemma!for pushouts|textit}
\begin{equation*}
\begin{tikzcd}
\sm{x:S}P_A(f(x)) \arrow[d,swap,"{f'}"] \arrow[r,"{g'}"] & \sm{b:B}P_B(b) \arrow[d,"{j'}"] \\
\sm{a:A}P_A(a) \arrow[r,swap,"{i'}"] & \sm{x:X}P(x)
\end{tikzcd}
\end{equation*}
is a pushout square.
\end{lem}

\begin{proof}
  To show that the square of total spaces satisfies the pullback property of pullbacks, note that we have a commuting cube
  \begin{equation*}
    \begin{tikzcd}
      & T^{\sm{x:X}P(x)} \arrow[dl] \arrow[d] \arrow[dr] \\
      T^{\sm{a:A}P_A(a)} \arrow[d] & \prd{x:X}P(x) \arrow[dl] \arrow[dr] & T^{\sm{b:B}P_B(b)} \arrow[dl,crossing over] \arrow[d] \\
      \prd{a:A}T^{P_A(a)} \arrow[dr] & T^{\sm{x:S}P_A(f(x))} \arrow[from=ul,crossing over] \arrow[d] & \prd{b:B}T^{P_B(b)} \arrow[dl] \\
      & \prd{x:S}T^{P_A(f(x))}
    \end{tikzcd}
  \end{equation*}
  for any type $T$. In this cube, the vertical maps are all equivalences, and the bottom square is a pullback square by the dependent pullback property of pushouts. Therefore it follows that the top square is a pullback square.
\end{proof}

\section{The universality theorem}
\begin{thm}
  Consider two pushout squares
  \begin{equation*}
    \begin{tikzcd}
      S' \arrow[r] \arrow[d] & B' \arrow[d] & S \arrow[r] \arrow[d] & B \arrow[d] \\
      A' \arrow[r] & C' & A \arrow[r] & C
    \end{tikzcd}
  \end{equation*}
  and a commuting cube
  \begin{equation*}
    \begin{tikzcd}
      & S' \arrow[dl] \arrow[d] \arrow[dr] \\
      A' \arrow[d] & S \arrow[dl] \arrow[dr] & B' \arrow[d] \arrow[dl,crossing over] \\
      A \arrow[dr] & D' \arrow[from=ul,crossing over] \arrow[d] & B \arrow[dl] \\
      & D
    \end{tikzcd}
  \end{equation*}
  in which the back left and right squares are pullback squares. The following are equivalent:
  \begin{enumerate}
  \item The front left and right squares are pullback squares.
  \item The induced commuting square
    \begin{equation*}
      \begin{tikzcd}
        C' \arrow[r,densely dotted] \arrow[d,densely dotted] & D' \arrow[d] \\
        C \arrow[r,densely dotted] & D
      \end{tikzcd}
    \end{equation*}
    is a pullback square.
  \end{enumerate}
\end{thm}


\section{The descent property for pushouts}

In the previous section there was a significant role for fiberwise equivalences, and we know by \cref{thm:pb_fibequiv,cor:pb_fibequiv}: fiberwise equivalences indicate the presence of pullbacks. In this section we reformulate the results of the previous section using pullbacks where we used fiberwise equivalences before, to obtain new and useful results. We begin by considering the type of descent data from the perspective of pullback squares.

\begin{defn}
Consider a span $\mathcal{S}$ from $A$ to $B$, and a span $\mathcal{S}'$ from $A'$ to $B'$. A \define{cartesian transformation} of spans\index{cartesian transformation!of spans|textbf} from $\mathcal{S}'$ to $\mathcal{S}$ is a diagram of the form
\begin{equation*}
\begin{tikzcd}
A' \arrow[d,swap,"h_A"]  & S' \arrow[l,swap,"{f'}"] \arrow[r,"{g'}"] \arrow[d,swap,"h_S"] & B' \arrow[d,"h_B"] \\
A & S \arrow[l,"f"] \arrow[r,swap,"g"] & B
\end{tikzcd}
\end{equation*}
with $F:f\circ h_S\htpy h_A\circ f'$ and $G:g\circ h_S\htpy h_B\circ g'$, where both squares are pullback squares. 

The type $\mathsf{cart}(\mathcal{S}',\mathcal{S})$\index{cart(S,S')@{$\mathsf{cart}(\mathcal{S},\mathcal{S}')$}|textbf} of cartesian transformation is the type of tuples
\begin{equation*}
(h_A,h_S,h_B,F,G,p_f,p_g)
\end{equation*}
where $p_f:\mathsf{is\usc{}pullback}(h_S,h_A,F)$ and $p_g:\mathsf{is\usc{}pullback}(h_S,h_B,G)$, and we write
\begin{equation*}
\mathsf{Cart}(\mathcal{S}) \defeq \sm{A',B':\UU}{\mathcal{S}':\mathsf{span}(A',B')}\mathsf{cart}(\mathcal{S}',\mathcal{S}).
\end{equation*}
\end{defn}

\begin{lem}\label{lem:cart_desc}
There is an equivalence\index{cart_desc@{$\mathsf{cart\usc{}desc}_{\mathcal{S}}$}|textit}
\begin{equation*}
\mathsf{cart\usc{}desc}_{\mathcal{S}}:\mathsf{Desc}(\mathcal{S})\to \mathsf{Cart}(\mathcal{S}).
\end{equation*}
\end{lem}

\begin{proof}
Note that by \cref{thm:pb_fibequiv_complete} it follows that the types of triples $(f',F,p_f)$ and $(g',G,p_g)$ are equivalent to the types of fiberwise equivalences
\begin{align*}
& \prd{x:S}\eqv{\fib{h_S}{x}}{\fib{h_A}{f(x)}} \\
& \prd{x:S}\eqv{\fib{h_S}{x}}{\fib{h_B}{g(x)}}
\end{align*} 
respectively. Furthermore, by \cref{thm:fam_proj} the types of pairs $(S',h_S)$, $(A',h_A)$, and $(B',h_B)$ are equivalent to the types $S\to \UU$, $A\to \UU$, and $B\to \UU$, respectively. Therefore it follows that the type $\mathsf{Cart}(\mathcal{S})$ is equivalent to the type of tuples $(Q,P_A,\varphi,P_B,P_S)$ consisting of
\begin{align*}
Q & : S\to \UU \\
P_A & : A \to \UU \\
P_B & : B \to \UU \\
\varphi & : \prd{x:S}\eqv{Q(x)}{P_A(f(x))} \\
P_S & : \prd{x:S}\eqv{Q(x)}{P_B(g(x))}.
\end{align*}
However, the type of $\varphi$ is equivalent to the type $P_A\circ f=Q$. Thus we see that the type of pairs $(Q,\varphi)$ is contractible, so our claim follows.
\end{proof}

\begin{defn}
We define an operation\index{cart map!{$\mathsf{cart\usc{}map}_{\mathcal{S}}$}|textbf}
\begin{equation*}
\mathsf{cart\usc{}map}_{\mathcal{S}}:{\Big(\sm{X':\UU}X'\to X\Big)}\to \mathsf{Cart}(\mathcal{S}).
\end{equation*}
\end{defn}

\begin{constr}
Let $X':\UU$ and $h_X:X'\to X$. Then we define the types
\begin{align*}
A' & \defeq A\times_X X' \\
B' & \defeq B\times_X X'.
\end{align*}
Next, we define a span $\mathcal{S'}\defeq(S',f',g')$ from $A'$ to $B'$. We take
\begin{align*}
S' & \defeq S\times_A A' \\
f' & \defeq \pi_2.
\end{align*}
To define $g'$, let $s:S$, let $(a,x',p):A\times_X X'$, and let $q:f(s)=a$. Our goal is to construct a term of type $B\times_X X'$. We have $g(s):B$ and $x':X'$, so it remains to show that $j(g(s))=h_X(x')$. We construct such an identification as a concatenation
\begin{equation*}
\begin{tikzcd}
j(g(s)) \arrow[r,equals,"H(s)^{-1}"] &[1ex] i(f(s)) \arrow[r,equals,"\ap{i}{q}"] &[1ex] i(a) \arrow[r,equals,"p"] & h_X(x').
\end{tikzcd}
\end{equation*}
To summaze, the map $g'$ is defined as
\begin{equation*}
g' \defeq \lam{(s,(a,x',p),q)}(g(s),x',\ct{H(s)^{-1}}{(\ct{\ap{i}{q}}{p})}).
\end{equation*}
Then we have commuting squares
\begin{equation*}
\begin{tikzcd}
A\times_X X' \arrow[d] & S\times_A A' \arrow[d] \arrow[l] \arrow[r] & B\times_X X' \arrow[d] \\
A & S \arrow[l] \arrow[r] & B.
\end{tikzcd}
\end{equation*}
Moreover, these squares are pullback squares by \cref{thm:pb_pasting}.
\end{constr}

The following theorem is analogous to \cref{thm:desc_fam}.

\begin{thm}[The descent theorem for pushouts]\label{thm:cart_map}\index{descent theorem!for pushouts|textit}
The operation $\mathsf{cart\usc{}map}_{\mathcal{S}}$\index{cart map!{$\mathsf{cart\usc{}map}_{\mathcal{S}}$}!is an equivalence|textit} is an equivalence
\begin{equation*}
\eqv{\Big(\sm{X':\UU}X'\to X\Big)}{\mathsf{Cart}(\mathcal{S})}
\end{equation*}
\end{thm}

\begin{proof}
It suffices to show that the square
\begin{equation*}
\begin{tikzcd}[column sep=huge]
X\to \UU \arrow[r,"{\mathsf{desc\usc{}fam}_{\mathcal{S}}(i,j,H)}"] \arrow[d,swap,"\mathsf{map\usc{}fam}_X"] & \mathsf{Desc}(\mathcal{S}) \arrow[d,"\mathsf{cart\usc{}desc}_{\mathcal{S}}"] \\
\sm{X':\UU}X'\to X \arrow[r,swap,"\mathsf{cart\usc{}map}_{\mathcal{S}}"] & \mathsf{Cart}(\mathcal{S})
\end{tikzcd}
\end{equation*}
commutes. To see that this suffices, note that the operation $\mathsf{map\usc{}fam}_X$ is an equivalence by \cref{thm:fam_proj}, the operation $\mathsf{desc\usc{}fam}_{\mathcal{S}}(i,j,H)$ is an equivalence by \cref{thm:desc_fam}, and the operation $\mathsf{cart\usc{}desc}_{\mathcal{S}}$ is an equivalence by \cref{lem:cart_desc}.

To see that the square commutes, note that the composite
\begin{equation*}
\mathsf{cart\usc{}map}_{\mathcal{S}}\circ \mathsf{map\usc{}fam}_X
\end{equation*}
takes a family $P:X\to \UU$ to the cartesian transformation of spans
\begin{equation*}
\begin{tikzcd}
A\times_X\tilde{P} \arrow[d,swap,"\pi_1"] & S\times_A\Big(A\times_X\tilde{P}\Big) \arrow[l] \arrow[r] \arrow[d,swap,"\pi_1"] & B\times_X\tilde{P} \arrow[d,"\pi_1"] \\
A & S \arrow[l] \arrow[r] & B,
\end{tikzcd}
\end{equation*}
where $\tilde{P}\defeq\sm{x:X}P(x)$.

The composite 
\begin{equation*}
\mathsf{cart\usc{}desc}_{\mathcal{S}}\circ \mathsf{desc\usc{}fam}_X
\end{equation*}
takes a family $P:X\to \UU$ to the cartesian transformation of spans
\begin{equation*}
\begin{tikzcd}
\sm{a:A}P(i(a)) \arrow[d] & \sm{s:S}P(i(f(s))) \arrow[l] \arrow[r] \arrow[d] & \sm{b:B}P(j(b)) \arrow[d] \\
A & S \arrow[l] \arrow[r] & B
\end{tikzcd}
\end{equation*}
These cartesian natural transformations are equal by \cref{lem:pb_subst}
\end{proof}

Since $\mathsf{cart\usc{}map}_{\mathcal{S}}$ is an equivalence it follows that its fibers are contractible. This is essentially the content of the following corollary.

\begin{cor}
Consider a diagram of the form 
\begin{equation*}
\begin{tikzcd}
& S' \arrow[d,swap,"h_S"] \arrow[dl,swap,"{f'}"] \arrow[dr,"{g'}"] \\
A' \arrow[d,swap,"h_A"] & S \arrow[dl,swap,"f"] \arrow[dr,"g"] & B' \arrow[d,"{h_B}"] \\
A \arrow[dr,swap,"i"] & & B \arrow[dl,"j"] \\
& X
\end{tikzcd}
\end{equation*}
with homotopies
\begin{align*}
F & : f\circ h_S \htpy h_A\circ f' \\
G & : g\circ h_S \htpy h_B\circ g' \\
H & : i\circ f \htpy j\circ g,
\end{align*}
and suppose that the bottom square is a pushout square, and the top squares are pullback squares.
Then the type of tuples $((X',h_X),(i',I,p),(j',J,q),(H',C))$ consisting of
\begin{enumerate}
\item A type $X':\UU$ together with a morphism
\begin{equation*}
h_X : X'\to X,
\end{equation*}
\item A map $i':A'\to X'$, a homotopy $I:i\circ h_A\htpy h_X\circ i'$, and a term $p$ witnessing that the square
\begin{equation*}
\begin{tikzcd}
A' \arrow[d,swap,"h_A"] \arrow[r,"{i'}"] & X' \arrow[d,"h_X"] \\
A \arrow[r,swap,"i"] & X
\end{tikzcd}
\end{equation*}
is a pullback square.
\item A map $j':B'\to X'$, a homotopy $J:j\circ h_B\htpy h_X\circ j'$, and a term $q$ witnessing that the square
\begin{equation*}
\begin{tikzcd}
B' \arrow[d,swap,"h_B"] \arrow[r,"{j'}"] & X' \arrow[d,"h_X"] \\
B \arrow[r,swap,"j"] & X
\end{tikzcd}
\end{equation*}
is a pullback square,
\item A homotopy $H':i'\circ f'\htpy j'\circ g'$, and a homotopy
\begin{equation*}
C : \ct{(i\cdot F)}{(\ct{(I\cdot f')}{(h_X\cdot H')})} \htpy \ct{(H\cdot h_S)}{(\ct{(j\cdot G)}{(J\cdot g')})}
\end{equation*}
witnessing that the cube
\begin{equation*}
\begin{tikzcd}
& S' \arrow[dl] \arrow[dr] \arrow[d] \\
A' \arrow[d] & S \arrow[dl] \arrow[dr] & B' \arrow[dl,crossing over] \arrow[d] \\
A \arrow[dr] & X' \arrow[d] \arrow[from=ul,crossing over] & B \arrow[dl] \\
& X,
\end{tikzcd}
\end{equation*}
commutes,
\end{enumerate}
is contractible.
\end{cor}

The following theorem should be compared to the flattening lemma, \cref{lem:flattening}.\index{flattening lemma!for pushouts}

\begin{thm}
Consider a commuting cube
\begin{equation*}
\begin{tikzcd}
& S' \arrow[dl,swap,"{f'}"] \arrow[dr,"{g'}"] \arrow[d,"h_S"] \\
A' \arrow[d,swap,"h_A"] & S \arrow[dl,swap,"f" near start] \arrow[dr,"g" near start] & B' \arrow[dl,crossing over,"{j'}" near end] \arrow[d,"h_B"] \\
A \arrow[dr,swap,"i"] & X' \arrow[d,"h_X" near start] \arrow[from=ul,crossing over,"{i'}"' near end] & B \arrow[dl,"j"] \\
& X.
\end{tikzcd}
\end{equation*}
If each of the vertical squares is a pullback, and the bottom square  is a pushout, then the top square is a pushout.
\end{thm}

\begin{proof}
By \cref{cor:pb_fibequiv} we have fiberwise equivalences
\begin{align*}
F & : \prd{x:S}\eqv{\fib{h_S}{x}}{\fib{h_A}{f(x)}} \\
G & : \prd{x:S}\eqv{\fib{h_S}{x}}{\fib{h_B}{g(x)}} \\
I & : \prd{a:A}\eqv{\fib{h_A}{a}}{\fib{h_X}{i(a)}} \\
J & : \prd{b:B}\eqv{\fib{h_B}{b}}{\fib{h_X}{j(b)}}. 
\end{align*}
Moreover, since the cube commutes we obtain a fiberwise homotopy
\begin{equation*}
K : \prd{x:S} I(f(x))\circ F(x) \htpy J(g(x))\circ G(x).
\end{equation*}
We define the descent data $(P_A,P_B,P_S)$ consisting of $P_A:A\to\UU$, $P_B:B\to\UU$, and $P_S:\prd{x:S}\eqv{P_A(f(x))}{P_B(g(x))}$ by
\begin{align*}
P_A(a) & \defeq \fib{h_A}{a} \\
P_B(b) & \defeq \fib{h_B}{b} \\
P_S(x) & \defeq G(x)\circ F(x)^{-1}.
\end{align*}
We have
\begin{align*}
P & \defeq \fibf{h_X} \\
e_A & \defeq I \\
e_B & \defeq J \\
e_S & \defeq K.
\end{align*}
Now consider the diagram
\begin{equation*}
\begin{tikzcd}
\sm{s:S}\fib{h_S}{s} \arrow[r] \arrow[d] & \sm{s:S}\fib{h_A}{f(s)} \arrow[r] \arrow[d] & \sm{b:B}\fib{h_B}{b} \arrow[d] \\
\sm{a:A}\fib{h_A}{a} \arrow[r] & \sm{a:A}\fib{h_A}{a} \arrow[r] & \sm{x:X}\fib{h_X}{x}
\end{tikzcd}
\end{equation*}
Since the top and bottom map in the left square are equivalences, we obtain from \cref{ex:pushout_equiv} that the left square is a pushout square. Moreover, the right square is a pushout by \cref{lem:flattening}. Therefore it follows by \cref{thm:pushout_pasting} that the outer rectangle is a pushout square.

Now consider the commuting cube
\begin{equation*}
\begin{tikzcd}
& \sm{s:S}\fib{h_S}{s} \arrow[dl] \arrow[dr] \arrow[d] \\
\sm{a:A}\fib{h_A}{a} \arrow[d] & S' \arrow[dl] \arrow[dr] & \sm{b:B}\fib{h_B}{b} \arrow[dl,crossing over] \arrow[d] \\
A' \arrow[dr,swap] & \sm{x:X}\fib{h_X}{x} \arrow[d] \arrow[from=ul,crossing over] & B' \arrow[dl] \\
& X'.
\end{tikzcd}
\end{equation*}
We have seen that the top square is a pushout. The vertical maps are all equivalences, so the vertical squares are all pushout squares. Thus it follows from one more application of \cref{thm:pushout_pasting} that the bottom square is a pushout.
\end{proof}

%\begin{cor}
%For any map $f:A\sqcup^S B\to X$, and any $x:X$, the square
%\begin{equation*}
%\begin{tikzcd}
%\fib{f_S}{x} \arrow[r] \arrow[d] & \fib{f_B}{x} \arrow[d] \\
%\fib{f_A}{x} \arrow[r] & \fib{f}{x}
%\end{tikzcd}
%\end{equation*}
%is a pushout square.
%\end{cor}

\begin{thm}
Consider a commuting cube of types 
\begin{equation*}\label{eq:cube}
\begin{tikzcd}
& S' \arrow[dl] \arrow[dr] \arrow[d] \\
A' \arrow[d] & S \arrow[dl] \arrow[dr] & B' \arrow[dl,crossing over] \arrow[d] \\
A \arrow[dr] & X' \arrow[d] \arrow[from=ul,crossing over] & B \arrow[dl] \\
& X,
\end{tikzcd}
\end{equation*}
and suppose the vertical squares are pullback squares. Then the commuting square
\begin{equation*}
\begin{tikzcd}
A' \sqcup^{S'} B' \arrow[r] \arrow[d] & X' \arrow[d] \\
A\sqcup^{S} B \arrow[r] & X
\end{tikzcd}
\end{equation*}
is a pullback square.
\end{thm}

\begin{proof}
It suffices to show that the pullback 
\begin{equation*}
(A\sqcup^{S} B)\times_{X}X'
\end{equation*}
has the universal property of the pushout. This follows by the descent theorem, since the vertical squares in the cube
\begin{equation*}
\begin{tikzcd}
& S' \arrow[dl] \arrow[dr] \arrow[d] \\
A' \arrow[d] & S \arrow[dl] \arrow[dr] & B' \arrow[dl,crossing over] \arrow[d] \\
A \arrow[dr] & (A\sqcup^{S} B)\times_{X}X' \arrow[d] \arrow[from=ul,crossing over] & B \arrow[dl] \\
& A\sqcup^{S} B
\end{tikzcd}
\end{equation*}
are pullback squares by \cref{thm:pb_pasting}.
\end{proof}

\section{Applications of the descent theorem}

\begin{thm}
  Consider a commuting cube
  \begin{equation*}
    \begin{tikzcd}
      & S' \arrow[dl] \arrow[dr] \arrow[d] \\
      A' \arrow[d] & S \arrow[dl] \arrow[dr] & B' \arrow[dl,crossing over] \arrow[d] \\
      A \arrow[dr] & C' \arrow[from=ul,crossing over] \arrow[d] & B \arrow[dl] \\
      & C
    \end{tikzcd}
  \end{equation*}
  in which the bottom square is a pushout square. If the vertical sides are pullback squares, then for each $c:C$ the square of fibers
  \begin{equation*}
    \begin{tikzcd}
      \fib{i\circ f\circ h_S}{c} \arrow[d] \arrow[r] & \fib{j\circ g\circ h_S}{c} \arrow[r] & \fib{j\circ h_B}{c} \arrow[d] \\
      \fib{i\circ h_A}{c} \arrow[rr] & & \fib{h_C}{c}
    \end{tikzcd}
  \end{equation*}
  is a pushout square.
\end{thm}

\begin{exercises}
\item Use the characterization of the circle\index{circle} as a pushout given in \cref{eg:circle_pushout} to show that the square
\begin{equation*}
\begin{tikzcd}[column sep=large]
\sphere{1}+\sphere{1} \arrow[r,"{[\idfunc,\idfunc]}"] \arrow[d,swap,"{[\idfunc,\idfunc]}"] & \sphere{1} \arrow[d,"{\lam{t}(t,\base)}"] \\
\sphere{1} \arrow[r,swap,"{\lam{t}(t,\base)}"] & \sphere{1}\times\sphere{1}
\end{tikzcd}
\end{equation*}
is a pushout square.
\item Let $f:A\to B$ be a map. The \define{codiagonal}\index{codiagonal}\index{nabla@{$\nabla_f$}} $\nabla_f$ of $f$ is the map obtained from the universal property of the pushout, as indicated in the diagram
\begin{equation*}
\begin{tikzcd}
A \arrow[d,swap,"f"] \arrow[r,"f"] \arrow[dr, phantom, "\ulcorner", very near end] & B \arrow[d,"\inr"] \arrow[ddr,bend left=15,"{\idfunc[B]}"] \\
A \arrow[r,"\inl"] \arrow[drr,bend right=15,swap,"{\idfunc[B]}"] & B\sqcup^{A} B \arrow[dr,densely dotted,near start,swap,"\nabla_f"] \\
& & B
\end{tikzcd}
\end{equation*}
Show that $\fib{\nabla_f}{b}\eqvsym \susp(\fib{f}{b})$ for any $b:B$.
\item \label{ex:fib_join}Consider two maps $f:A\to X$ and $g:B\to X$. The \define{fiberwise join}\index{fiberwise join} $\join{f}{g}$ is defined by the universal property of the pushout as the unique map rendering the diagram
\begin{equation*}
\begin{tikzcd}
A\times_X B \arrow[d,"\pi_1"] \arrow[r,"\pi_2"] \arrow[dr, phantom, "\ulcorner", very near end] & B \arrow[d,"\inr"] \arrow[ddr,bend left=15,"g"] \\
A \arrow[r,"\inl"] \arrow[drr,bend right=15,swap,"f"] & \join[X]{A}{B} \arrow[dr,densely dotted,near start,swap,"\join{f}{g}"] \\
& & X
\end{tikzcd}
\end{equation*}
commutative, where $\join[X]{A}{B}$ is defined as a pushout, as indicated.
Construct an equivalence
\begin{equation*}
\eqv{\fib{\join{f}{g}}{x}}{\join{\fib{f}{x}}{\fib{g}{x}}}
\end{equation*}
for any $x:X$. 
\item Consider two maps $f:A\to B$ and $g:C\to D$.
The \define{pushout-product}\index{pushout-product}
\begin{equation*}
f\square g : (A\times D)\sqcup^{A\times C} (B\times C)\to B\times D
\end{equation*}
of $f$ and $g$ is defined by the universal property of the pushout as the unique map rendering the diagram
\begin{equation*}
\begin{tikzcd}
A\times C \arrow[r,"{f\times \idfunc[C]}"] \arrow[d,swap,"{\idfunc[A]\times g}"] & B\times C \arrow[d,"\inr"] \arrow[ddr,bend left=15,"{\idfunc[B]\times g}"] \\
A\times D \arrow[r,"\inl"] \arrow[drr,bend right=15,swap,"{f\times\idfunc[D]}"] & (A\times D)\sqcup^{A\times C} (B\times C) \arrow[dr,densely dotted,swap,near start,"f\square g"] \\
& & B\times D
\end{tikzcd}
\end{equation*}
commutative. Construct an equivalence
\begin{equation*}
\eqv{\fib{f\square g}{b,d}}{\join{\fib{f}{b}}{\fib{g}{d}}}
\end{equation*}
for all $b:B$ and $d:D$.
\item Let $A$ and $B$ be pointed types with base points $a_0:A$ and $b_0:B$. The \define{wedge inclusion}\index{wedge inclusion} is defined as follows by the universal property of the wedge:
\begin{equation*}
\begin{tikzcd}[column sep=huge]
\unit \arrow[r] \arrow[d] & B \arrow[d,"\inr"] \arrow[ddr,bend left=15,"{\lam{b}(a_0,b)}"] \\
A \arrow[r,"\inl"] \arrow[drr,bend right=15,swap,"{\lam{a}(a,b_0)}"] & A\vee B \arrow[dr,densely dotted,swap,"{\mathsf{wedge\usc{}in}_{A,B}}"{near start,xshift=1ex}] \\
& & A\times B
\end{tikzcd}
\end{equation*}
Show that the fiber of the wedge inclusion $A\vee B\to A\times B$ is equivalent to $\join{\loopspace{B}}{\loopspace{A}}$.
\item Let $f:X\vee X\to X$ be the map defined by the universal property of the wedge as indicated in the diagram
\begin{equation*}
\begin{tikzcd}
\unit \arrow[d,swap,"x_0"] \arrow[r,"x_0"] \arrow[dr, phantom, "\ulcorner", very near end] & X \arrow[d,"\inr"] \arrow[ddr,bend left=15,"{\idfunc[X]}"] \\
X \arrow[r,"\inl"] \arrow[drr,bend right=15,swap,"{\idfunc[X]}"] & X\vee X \arrow[dr,densely dotted,near start,swap,"f"] \\
& & X.
\end{tikzcd}
\end{equation*}
\begin{subexenum}
\item Show that $\eqv{\fib{f}{x_0}}{\susp\loopspace{X}}$. 
\item Show that $\eqv{\mathsf{cof}_f}{\susp X}$.
\end{subexenum}
\item Consider a pushout square
\begin{equation*}
\begin{tikzcd}
S \arrow[r,"g"] \arrow[d,swap,"f"] & B \arrow[d,"j"] \\
A \arrow[r,swap,"i"] & X,
\end{tikzcd}
\end{equation*}
and suppose that $f$ is an embedding. Show that $j$ is an embedding, and that the square is also a pullback square.
\end{exercises}

\section{The identity types of pushouts}

\subsection{Characterizing families of maps over pushouts}
  
\begin{defn}
    Consider a span $\mathcal{S}$
  \begin{equation*}
    \begin{tikzcd}
      A & S \arrow[l,swap,"f"] \arrow[r,"g"] & B,
    \end{tikzcd}
  \end{equation*}
  and consider $P,Q:\mathsf{Fam\usc{}pushout}(\mathcal{S})$.
  A morphism of descent data from $P$ to $Q$ over $\mathcal{S}$ is defined to be a triple $(h_A,h_B,h_S)$ consisting of
  \begin{align*}
    h_A & : \prd{x:A} P_A(x)\to Q_A(x) \\
    h_B & : \prd{y:B} P_B(y)\to Q_B(y)
  \end{align*}
  equipped with a homotopy $h_S$ witnessing that the square
  \begin{equation*}
    \begin{tikzcd}[column sep=huge]
      P_A(f(s)) \arrow[r,"h_A(f(s))"] \arrow[d,swap,"P_S(s)"] & Q_A(f(s)) \arrow[d,"Q_S(s)"] \\
      P_B(g(s)) \arrow[r,swap,"h_B(g(s))"] & Q_B(g(s))
    \end{tikzcd}
  \end{equation*}
  commutes for every $s:S$. We write $\mathsf{hom}_{\mathcal{S}}(P,Q)$ for the type of morphisms of descent data over $\mathcal{S}$.

  An equivalence of descent data from $P$ to $Q$ is a morphism $h$ such that $h_A$ and $h_B$ are families of equivalences.
\end{defn}

\begin{rmk}\label{rmk:id-hom-Fam-pushout}
  The identity type $h=h'$ of $\mathsf{hom}_{\mathcal{S}}(P,Q)$ is characterized as the type of triples $(H_A,H_B,H_S)$ consisting of
  \begin{align*}
    H_A & : \prd{a:A} h_A(a)\htpy h'_A(a) \\
    H_B & : \prd{b:B} h_B(b)\htpy h'_B(b)
  \end{align*}
  and a homotopy $K_S(s)$ witnessing that the square
  \begin{equation*}
    \begin{tikzcd}
      h_B(g(s))\circ P_S(s) \arrow[d] \arrow[r] & Q_S(s)\circ h_A(f(s)) \arrow[d] \\
      h'_B(g(s))\circ P_S(s) \arrow[r] & Q_S(s)\circ h'_A(g(s))
    \end{tikzcd}
  \end{equation*}
  of homotopies commutes for every $s:S$.
\end{rmk}

\begin{defn}\label{defn:hom-Fam-pushout-map}
  Consider a commuting square
  \begin{equation*}
    \begin{tikzcd}
      S \arrow[r,"g"] \arrow[d,swap,"f"] & B \arrow[d,"j"] \\
      A \arrow[r,swap,"i"] & X
    \end{tikzcd}
  \end{equation*}
  with $H:i\circ f \htpy j \circ f$, and let $P$ and $Q$ be type families over $X$. We define a map
  \begin{equation*}
    \Big(\prd{x:X}P(x)\to Q(x)\Big) \to \mathsf{hom}_{\mathcal{S}}(\mathsf{desc\usc{}fam}(P),\mathsf{desc\usc{}fam}(Q)).
  \end{equation*}
\end{defn}

\begin{constr}
  Let $h:\prd{x:X}P(x)\to Q(x)$. Then we define
  \begin{align*}
    h_A & : \prd{a:A}P(i(a))\to Q(i(a)) \\
    h_B & : \prd{b:B}P(j(b))\to Q(j(b))
  \end{align*}
  by $h_A(a,p)\defeq h(i(a),p)$ and $h_B(b,q)\defeq h(j(b),q)$. Then it remains to define for every $s:S$ a homotopy $h_S(s)$ witnessing that the square
  \begin{equation*}
    \begin{tikzcd}[column sep=huge]
      P(i(f(s))) \arrow[r,"h_A(f(s))"] \arrow[d,swap,"\mathsf{tr}_P(H(s))"] & Q(i(f(s))) \arrow[d,"\mathsf{tr}_Q(H(s))"] \\
      P(j(g(s))) \arrow[r,swap,"h_B(g(s))"] & Q(j(g(s)))
    \end{tikzcd}
  \end{equation*}
  commutes. Note that every family of maps $h:\prd{x:X}P(x)\to Q(x)$ is natural in the sense that for any path $p:x=x'$ in $X$, there is a homotopy $\psi(p,h)$ witnessing that the square
  \begin{equation*}
    \begin{tikzcd}
      P(x) \arrow[r,"h(x)"] \arrow[d,swap,"\mathsf{tr}_P(p)"] & Q(x) \arrow[d,"\mathsf{tr}_Q(p)"] \\
      P(x') \arrow[r,swap,"{h(x')}"] & Q(x')
    \end{tikzcd}
  \end{equation*}
  commutes. Therefore we define $h_S(s)\defeq\psi(H(s),h)$.
\end{constr}

\begin{thm}\label{thm:hom-Fam-pushout}
  The map defined in \cref{defn:hom-Fam-pushout-map} is an equivalence.
\end{thm}

\begin{proof}
  We will first construct a commuting triangle
  \begin{equation*}
    \begin{tikzcd}[column sep=-3em]
      \phantom{\mathsf{hom}_{\mathcal{S}}(\mathsf{desc\usc{}fam}(P),\mathsf{desc\usc{}fam}(Q))} & \prd{x:X}P(x)\to Q(x) \arrow[dl] \arrow[dr] & \phantom{\mathsf{dep\usc{}cocone}_{(i,j,H)}(x\mapsto P(x)\to Q(x))} \\
      \mathsf{dep\usc{}cocone}_{(i,j,H)}(x\mapsto P(x)\to Q(x)) \arrow[rr,densely dotted] & &
      \mathsf{hom}_{\mathcal{S}}(\mathsf{desc\usc{}fam}(P),\mathsf{desc\usc{}fam}(Q))
    \end{tikzcd}
  \end{equation*}
  Recall from \cref{thm:dependent-pullback-property-pushout} that $X$ satisfies the dependent universal property, so the map on the left is an equivalence. Therefore we will prove the claim by showing that the bottom map is an equivalence.

  In order to construct the bottom map, we first note that for any two maps $\alpha:P(x)\to Q(x)$ and $\alpha':P(x')\to Q(x')$ and any path $p:x=x'$, there is an equivalence
  \begin{equation*}
    \varphi(p,f,f'):\Big(\mathsf{tr}_{x\mapsto P(x)\to Q(x)}(p,f)=f'\Big) \simeq \Big(\prd{y:B(x)} f'(\mathsf{tr}_B(p,y))=\mathsf{tr}_C(p,f(y))\Big).
  \end{equation*}
  The equivalence $\varphi$ is defined by path induction on $p$, where we take
  \begin{equation*}
    \varphi(\refl{},f,f')\defeq \mathsf{htpy\usc{}eq}\circ\mathsf{inv}.
  \end{equation*}
  Now we define the bottom map in the asserted triangle to be the map
  \begin{equation*}
    (h_A,h_B,h_S)\mapsto (h_A,h_B,\lam{s}\varphi(H(s),h_A(f(s)),h_B(g(s)),h_S(s))).
  \end{equation*}
  Note that this map is an equivalence, since it is the induced map on total spaces of an equivalence.

  It remains to show that the triangle commutes. By \cref{rmk:id-hom-Fam-pushout} it suffices to construct families of homotopies
  \begin{align*}
    K_A : \prd{a:A} h_{i(a)}\htpy h_{i(a)} \\
    K_B : \prd{b:B} h_{j(b)}\htpy h_{j(b)}
  \end{align*}
  and for each $s:S$ a homotopy $K_S(s)$ witnessing that the square
  \begin{equation*}
    \begin{tikzcd}[column sep=13em]
      h_{j(g(s))} \circ \mathsf{tr}_P(H(s)) \arrow[d] \arrow[r,"{\psi(H(s),h)}"] & \mathsf{tr}_Q(H(s)) \circ h_{i(f(s))} \arrow[d] \\
      h_{j(g(s))} \circ \mathsf{tr}_P(H(s)) \arrow[r,swap,"{\varphi(H(s),h_{i(f(s))},h_{j(g(s))},\apd{h}{H(s)})}"] & \mathsf{tr}_Q(H(s))\circ h_{i(f(s))}
    \end{tikzcd}
  \end{equation*}
  commutes. Of course, we take $K_A(a)\defeq\mathsf{htpy\usc{}refl}$ and $K_B(b)\defeq\mathsf{htpy\usc{}refl}$, so it suffices to show that
  \begin{equation*}
    \psi(H(s),h)\htpy \varphi(H(s),h_{i(f(s))},h_{j(g(s))},\apd{h}{H(s)}).
  \end{equation*}
  Now we would like to proceed by homotopy induction on $H:i\circ f \htpy j\circ g$. However, we can only do so after we generalize the problem sufficiently to a situation where $H$ has free endpoints. It is indeed possible by homotopy induction to construct for every $f,g:S\to X$ equipped with a homotopy $H:f\htpy g$, every family of maps $h:\prd{x:X}P(x)\to Q(x)$ and every $s:S$, a homotopy
  \begin{equation*}
    \psi(H(s),h)\htpy \varphi(H(s),h_{f(s)},h_{g(s)},\apd{h}{H(s)}).\qedhere
  \end{equation*}
\end{proof}

\subsection{Characterizing the identity types of pushouts}

\begin{defn}
  Consider a span $\mathcal{S}$ equipped with $a:A$, and consider
  $P:\mathsf{Fam\usc{}pushout}(\mathcal{S})$ equipped with $p:P_A(a)$. We say that $P$ is \define{universal} if for every $Q:\mathsf{Fam\usc{}pushout}(\mathcal{S})$ the evaluation map
  \begin{equation*}
    \mathsf{hom}_{\mathcal{S}}(P,Q)\to Q_A(a)
  \end{equation*}
  given by $h\mapsto h_A(a,p)$ is an equivalence.
\end{defn}

\begin{lem}
  Consider a pushout square
  \begin{equation*}
    \begin{tikzcd}
      S \arrow[r,"g"] \arrow[d,swap,"f"] & B \arrow[d,"j"] \\
      A \arrow[r,swap,"i"] & X
    \end{tikzcd}
  \end{equation*}
  with $H:i\circ f \htpy j\circ g$, and let $a:A$. Furthermore, let $P$ be the descent data for the type family $x\mapsto i(a)=x$ over $X$. Then $P$ is universal.
\end{lem}

\begin{proof}
  Since $\mathsf{desc\usc{}fam}$ is an equivalence, it suffices to show that for every type family $Q$ over $X$, the map
  \begin{equation*}
    \mathsf{hom}_{\mathcal{S}}(\mathsf{desc\usc{}fam}(\mathsf{Id}(i(a))),\mathsf{desc\usc{}fam}(Q))\to Q(i(a))
  \end{equation*}
  given by $h\mapsto h_A(a,\refl{i(a)})$ is an equivalence. 
  Note that we have a commuting triangle
  \begin{equation*}
    \begin{tikzcd}
      \prd{x:X}(i(a)=x)\to Q(x) \arrow[r] \arrow[dr,swap,"\mathsf{ev\usc{}refl}"] &
      \mathsf{hom}_{\mathcal{S}}(\mathsf{desc\usc{}fam}(\mathsf{Id}(i(a))),\mathsf{desc\usc{}fam}(Q)) \arrow[d,"{h\mapsto h_A(\refl{i(a)})}"] \\
      & Q(i(a))
    \end{tikzcd}
  \end{equation*}
  The map $\mathsf{ev\usc{}refl}$ is an equivalence by \cref{thm:yoneda}, and the top map is an equivalence by \cref{thm:hom-Fam-pushout}. Therefore it follows that the remaining map is an equivalence.
\end{proof}

\begin{thm}
  Consider a pushout square
  \begin{equation*}
    \begin{tikzcd}
      S \arrow[r,"g"] \arrow[d,swap,"f"] & B \arrow[d,"j"] \\
      A \arrow[r,swap,"i"] & X
    \end{tikzcd}
  \end{equation*}
  with $H:i\circ f \htpy j\circ g$, and let $a:A$. Furthermore consider a pair $(P,p_0)$ consisting of $P:\mathsf{Fam\usc{}pushout}(\mathcal{S})$ and $p:P_A(a)$. If $P$ is universal, then we have two families of equivalences
  \begin{align*}
    e_A & : \prd{x:A}P_A(x)\simeq (i(a)=i(x)) \\
    e_B & : \prd{y:B} P_B(y)\simeq (i(a)=j(b)) 
  \end{align*}
  equipped with a homotopy $e_S$ witnessing that the square
  \begin{equation*}
    \begin{tikzcd}[column sep=huge]
      P_A(f(s)) \arrow[r,"e(s)"] \arrow[d,swap,"e_A(f(s))"] & P_B(g(s)) \arrow[d,"e_B(g(s))"] \\
      (i(a)=i(f(s))) \arrow[r,swap,"\lam{p}\ct{p}{H(s)}"] & (i(a)=g(s))
    \end{tikzcd}
  \end{equation*}
  commutes for each $s:S$, and an identification $e_A(a,r)=\refl{i(a)}$
\end{thm}

\begin{thm}
  Let $X$ be a pointed type with base point $x_0:X$. Then the loop space of $\susp{X}$ is the initial type $Y$ equipped with a base point $y_0:Y$, and a pointed map
  \begin{equation*}
    X \to_\ast (Y\simeq Y).
  \end{equation*}
\end{thm}

\begin{proof}
  The type of pairs $(Y,\mu)$ consisting of a pointed type $Y$ and a pointed map $\mu:X\to_\ast (Y \simeq Y)$ is equivalent to the type of triples $(Y,Z,\mu)$ consisting of a pointed type $Y$, a type $Z$, and a map $\mu:X\to (Y\simeq Z)$.  
\end{proof}

\begin{cor}
  The loop space of $\sphere{2}$ is the initial type $X$ equipped with a point $x_0:X$ and a homotopy $H:\idfunc\htpy\idfunc$.
\end{cor}

\begin{exercises}
\item Consider the suspension
  \begin{equation*}
    \begin{tikzcd}
      P \arrow[r] \arrow[d] & \unit \arrow[d,"\south"] \\
      \unit \arrow[r,swap,"\north"] & \susp{P}
    \end{tikzcd}
  \end{equation*}
  of a proposition $P$. Show that $(\north=\south)\simeq P$. 
\item Show that if $X$ has decidable equality, then $\susp{X}$ is a $1$-type.
\item Consider a pushout square
  \begin{equation*}
    \begin{tikzcd}
      A \arrow[r] \arrow[d,swap,"f"] & \unit \arrow[d,"j"] \\
      B \arrow[r,swap,"i"] & X
    \end{tikzcd}
  \end{equation*}
  where $f:A\to B$ is an embedding.
  \begin{subexenum}
  \item Show that there are equivalences
  \begin{align*}
    (i(b)=i(y)) & \simeq (b=y)\ast \fib{f}{b} \\
    (i(b)=j(\ttt)) & \simeq \fib{f}{b}
  \end{align*}
  for any $b,y:B$.
  \item Use \cref{ex:trunc-join-with-prop} to show that if $B$ is a $k$-type, then so is $X$, for any $k\geq 0$.
  \end{subexenum}
\item Consider the join
  \begin{equation*}
    \begin{tikzcd}
      P \times X \arrow[r,"\proj 2"] \arrow[d,swap,"\proj 1"] & X \arrow[d,"\inr"] \\
      P \arrow[r,swap,"\inl"] & \join{P}{X}
    \end{tikzcd}
  \end{equation*}
  of a proposition $P$ and an arbitrary type $X$.
  \begin{subexenum}
  \item Show that for any $x,y:X$ there is an equivalence
    $e:(\inr(x)=\inr(y))\simeq \join{P}{(x=y)}$ for which the triangle
  \begin{equation*}
    \begin{tikzcd}[column sep=tiny]
      \phantom{\join{P}{(x=y)}} & (x=y) \arrow[dl,swap,"\apfunc{\inr}"] \arrow[dr,"\inr"] & \phantom{(\inr(x)=\inr(y))} \\
      (\inr(x)=\inr(y)) \arrow[rr,swap,"e"] & & \join{P}{(x=y)}
    \end{tikzcd}
  \end{equation*}
  commutes.
\item \label{ex:trunc-join-with-prop}Show that if $X$ is a $k$-type, then so is $\join{P}{X}$.
  \end{subexenum}
\end{exercises}


\chapter{The homotopy image of a map}
\section{Sequential colimits}

\emph{Note: This chapter currently contains only the statements of the definitions and theorems, but no proofs. I hope to make a complete version available soon.}

\subsection{The universal property of sequential colimits}

Type sequences are diagrams of the following form.
\begin{equation*}
\begin{tikzcd}
A_0 \arrow[r,"f_0"] & A_1 \arrow[r,"f_1"] & A_2 \arrow[r,"f_2"] & \cdots.
\end{tikzcd}
\end{equation*}
Their formal specification is as follows.

\begin{defn}
An \define{(increasing) type sequence} $\mathcal{A}$ consists of
\begin{align*}
A & : \N\to\UU \\
f & : \prd{n:\N} A_n\to A_{n+1}. 
\end{align*}
\end{defn}

In this section we will introduce the sequential colimit of a type sequence.
The sequential colimit includes each of the types $A_n$, but we also identify each $x:A_n$ with its value $f_n(x):A_{n+1}$. 
Imagine that the type sequence $A_0\to A_1\to A_2\to\cdots$ defines a big telescope, with $A_0$ sliding into $A_1$, which slides into $A_2$, and so forth.

As usual, the sequential colimit is characterized by its universal property.

\begin{defn}
\begin{enumerate}
\item A \define{(sequential) cocone} on a type sequence $\mathcal{A}$ with vertex $B$ consists of
\begin{align*}
h & : \prd{n:\N} A_n\to B \\
H & : \prd{n:\N} f_n\htpy f_{n+1}\circ H_n.
\end{align*}
We write $\mathsf{cocone}(B)$ for the type of cones with vertex $X$.
\item Given a cone $(h,H)$ with vertex $B$ on a type sequence $\mathcal{A}$ we define the map
\begin{equation*}
\mathsf{cocone\usc{}map}(h,H) : (B\to C)\to \mathsf{cocone}(B)
\end{equation*}
given by $f\mapsto (f\circ h,\lam{n}{x}\mathsf{ap}_f(H_n(x)))$. 
\item We say that a cone $(h,H)$ with vertex $B$ is \define{colimiting} if $\mathsf{cocone\usc{}map}(h,H)$ is an equivalence for any type $C$. 
\end{enumerate}
\end{defn}

\begin{thm}\label{thm:sequential_up}
Consider a cocone $(h,H)$ with vertex $B$ for a type sequence $\mathcal{A}$. The following are equivalent:
\begin{enumerate}
\item The cocone $(h,H)$ is colimiting.
\item The cocone $(h,H)$ is inductive in the sense that for every type family $P:B\to \UU$, the map
\begin{align*}
\Big(\prd{b:B}P(b)\Big)\to {}& \sm{h:\prd{n:\N}{x:A_n}P(h_n(x))}\\ 
& \qquad \prd{n:\N}{x:A_n} \mathsf{tr}_P(H_n(x),h_n(x))={h_{n+1}(f_n(x))}
\end{align*}
given by
\begin{equation*}
s\mapsto (\lam{n}s\circ h_n,\lam{n}{x} \mathsf{apd}_{s}(H_n(x)))
\end{equation*}
has a section.
\item The map in (ii) is an equivalence.
\end{enumerate}
\end{thm}

\subsection{The construction of sequential colimits}

We construct sequential colimits using pushouts.

\begin{defn}
Let $\mathcal{A}\jdeq (A,f)$ be a type sequence. We define the type $A_\infty$ as a pushout
\begin{equation*}
\begin{tikzcd}[column sep=large]
\tilde{A}+\tilde{A} \arrow[r,"{[\idfunc,\sigma_{\mathcal{A}}]}"] \arrow[d,swap,"{[\idfunc,\idfunc]}"] & \tilde{A} \arrow[d,"\inr"] \\
\tilde{A} \arrow[r,swap,"\inl"] & A_\infty.
\end{tikzcd}
\end{equation*}
\end{defn}

\begin{defn}
The type $A_\infty$ comes equipped with a cocone structure consisting of
\begin{align*}
\mathsf{seq\usc{}in} & : \prd{n:\N} A_n\to A_\infty \\
\mathsf{seq\usc{}glue} & : \prd{n:\N}{x:A_n} \mathsf{in}_n(x)=\mathsf{in}_{n+1}(f_n(x)).
\end{align*}
\end{defn}

\begin{constr}
We define
\begin{align*}
\mathsf{seq\usc{}in}(n,x)\defeq \inr(n,x) \\
\mathsf{seq\usc{}glue}(n,x)\defeq \ct{\glue(\inl(n,x))^{-1}}{\glue(\inr(n,x))}.
\end{align*}
\end{constr}

\begin{thm}
Consider a type sequence $\mathcal{A}$, and write $\tilde{A}\defeq\sm{n:\N}A_n$. Moreover, consider the map
\begin{equation*}
\sigma_{\mathcal{A}}:\tilde{A}\to\tilde{A}
\end{equation*}
defined by $\sigma_{\mathcal{A}}(n,a)\defeq (n+1,f_n(a))$. Furthermore, consider a cocone $(h,H)$ with vertex $B$.
The following are equivalent:
\begin{enumerate}
\item The cocone $(h,H)$ with vertex $B$ is colimiting.
\item The defining square
\begin{equation*}
\begin{tikzcd}[column sep=large]
\tilde{A}+\tilde{A} \arrow[r,"{[\idfunc,\sigma_{\mathcal{A}}]}"] \arrow[d,swap,"{[\idfunc,\idfunc]}"] & \tilde{A} \arrow[d,"{\lam{(n,x)}h_n(x)}"] \\
\tilde{A} \arrow[r,swap,"{\lam{(n,x)}h_n(x)}"] & A_\infty,
\end{tikzcd}
\end{equation*}
of $A_\infty$ is a pushout square.
\end{enumerate}
\end{thm}

\subsection{Descent for sequential colimits}

\begin{defn}
The type of \define{descent data} on a type sequence $\mathcal{A}\jdeq (A,f)$ is defined to be
\begin{equation*}
\mathsf{Desc}(\mathcal{A}) \defeq \sm{B:\prd{n:\N}A_n\to\UU}\prd{n:\N}{x:A_n}\eqv{B_n(x)}{B_{n+1}(f_n(x))}.
\end{equation*}
\end{defn}

\begin{defn}
We define a map
\begin{equation*}
\mathsf{desc\usc{}fam} : (A_\infty\to\UU)\to\mathsf{Desc}(\mathcal{A})
\end{equation*}
by $B\mapsto (\lam{n}{x}B(\mathsf{seq\usc{}in}(n,x)),\lam{n}{x}\mathsf{tr}_B(\mathsf{seq\usc{}glue}(n,x)))$.
\end{defn}

\begin{thm}
The map 
\begin{equation*}
\mathsf{desc\usc{}fam} : (A_\infty\to\UU)\to\mathsf{Desc}(\mathcal{A})
\end{equation*}
is an equivalence.
\end{thm}

\begin{defn}
A \define{cartesian transformation} of type sequences from $\mathcal{A}$ to $\mathcal{B}$ is a pair $(h,H)$ consisting of
\begin{align*}
h & : \prd{n:\N} A_n\to B_n \\
H & : \prd{n:\N} g_n\circ h_n \htpy h_{n+1}\circ f_n,
\end{align*}
such that each of the squares in the diagram
\begin{equation*}
\begin{tikzcd}
A_0 \arrow[d,swap,"h_0"] \arrow[r,"f_0"] & A_1 \arrow[d,swap,"h_1"] \arrow[r,"f_1"] & A_2 \arrow[d,swap,"h_2"] \arrow[r,"f_2"] & \cdots \\
B_0 \arrow[r,swap,"g_0"] & B_1 \arrow[r,swap,"g_1"] & B_2 \arrow[r,swap,"g_2"] & \cdots
\end{tikzcd}
\end{equation*}
is a pullback square. We define
\begin{align*}
\mathsf{cart}(\mathcal{A},\mathcal{B}) & \defeq\sm{h:\prd{n:\N}A_n\to B_n} \\
& \qquad\qquad \sm{H:\prd{n:\N}g_n\circ h_n\htpy h_{n+1}\circ f_n}\prd{n:\N}\mathsf{is\usc{}pullback}(h_n,f_n,H_n),
\end{align*}
and we write
\begin{equation*}
\mathsf{Cart}(\mathcal{B}) \defeq \sm{\mathcal{A}:\mathsf{Seq}}\mathsf{cart}(\mathcal{A},\mathcal{B}).
\end{equation*}
\end{defn}

\begin{defn}
We define a map
\begin{equation*}
\mathsf{cart\usc{}map}(\mathcal{B}) : \Big(\sm{X':\UU}X'\to X\Big)\to\mathsf{Cart}(\mathcal{B}).
\end{equation*}
which associates to any morphism $h:X'\to X$ a cartesian transformation of type sequences into $\mathcal{B}$.
\end{defn}

\begin{thm}
The operation $\mathsf{cart\usc{}map}(\mathcal{B})$ is an equivalence.
\end{thm}

\subsection{The flattening lemma for sequential colimits}

The flattening lemma for sequential colimits essentially states that sequential colimits commute with $\Sigma$. 

\begin{lem}
Consider
\begin{align*}
B & : \prd{n:\N}A_n\to\UU \\
g & : \prd{n:\N}{x:A_n}\eqv{B_n(x)}{B_{n+1}(f_n(x))}.
\end{align*}
and suppose $P:A_\infty\to\UU$ is the unique family equipped with
\begin{align*}
e & : \prd{n:\N}\eqv{B_n(x)}{P(\mathsf{seq\usc{}in}(n,x))}
\end{align*}
and homotopies $H_n(x)$ witnessing that the square
\begin{equation*}
\begin{tikzcd}[column sep=7em]
B_n(x) \arrow[r,"g_n(x)"] \arrow[d,swap,"e_n(x)"] & B_{n+1}(f_n(x)) \arrow[d,"e_{n+1}(f_n(x))"] \\
P(\mathsf{seq\usc{}in}(n,x)) \arrow[r,swap,"{\mathsf{tr}_P(\mathsf{seq\usc{}glue}(n,x))}"] & P(\mathsf{seq\usc{}in}(n+1,f_n(x)))
\end{tikzcd}
\end{equation*}
commutes. Then $\sm{t:A_\infty}P(t)$ satisfies the universal property of the sequential colimit of the type sequence
\begin{equation*}
\begin{tikzcd}
\sm{x:A_0}B_0(x) \arrow[r,"{\tot[f_0]{g_0}}"] & \sm{x:A_1}B_1(x) \arrow[r,"{\tot[f_1]{g_1}}"] & \sm{x:A_2}B_2(x) \arrow[r,"{\tot[f_2]{g_2}}"] & \cdots.
\end{tikzcd}
\end{equation*}
\end{lem}

In the following theorem we rephrase the flattening lemma in using cartesian transformations of type sequences.

\begin{thm}
Consider a commuting diagram of the form
\begin{equation*}
\begin{tikzcd}[column sep=small,row sep=small]
A_0 \arrow[rr] \arrow[dd] & & A_1 \arrow[rr] \arrow[dr] \arrow[dd] &[-.9em] &[-.9em] A_2 \arrow[dl] \arrow[dd] & & \cdots \\
& & & X \arrow[from=ulll,crossing over] \arrow[from=urrr,crossing over] \arrow[from=ur,to=urrr] \\
B_0 \arrow[rr] \arrow[drrr] & & B_1 \arrow[rr] \arrow[dr] & & B_2 \arrow[rr] \arrow[dl] & & \cdots \arrow[dlll] \\
& & & Y \arrow[from=uu,crossing over] 
\end{tikzcd}
\end{equation*}
If each of the vertical squares is a pullback square, and $Y$ is the sequential colimit of the type sequence $B_n$, then $X$ is the sequential colimit of the type sequence $A_n$. 
\end{thm}

\begin{cor}
Consider a commuting diagram of the form
\begin{equation*}
\begin{tikzcd}[column sep=small,row sep=small]
A_0 \arrow[rr] \arrow[dd] & & A_1 \arrow[rr] \arrow[dr] \arrow[dd] &[-.9em] &[-.9em] A_2 \arrow[dl] \arrow[dd] & & \cdots \\
& & & X \arrow[from=ulll,crossing over] \arrow[from=urrr,crossing over] \arrow[from=ur,to=urrr] \\
B_0 \arrow[rr] \arrow[drrr] & & B_1 \arrow[rr] \arrow[dr] & & B_2 \arrow[rr] \arrow[dl] & & \cdots \arrow[dlll] \\
& & & Y \arrow[from=uu,crossing over] 
\end{tikzcd}
\end{equation*}
If each of the vertical squares is a pullback square, then the square
\begin{equation*}
\begin{tikzcd}
A_\infty \arrow[r] \arrow[d] & X \arrow[d] \\
B_\infty \arrow[r] & Y
\end{tikzcd}
\end{equation*} 
is a pullback square.
\end{cor}

\begin{exercises}
\exercise \label{ex:seqcolim_shift}
Show that the sequential colimit of a type sequence
\begin{equation*}
\begin{tikzcd}
A_0 \arrow[r,"f_0"] & A_1 \arrow[r,"f_1"] & A_2 \arrow[r,"f_2"] & \cdots
\end{tikzcd}
\end{equation*}
is equivalent to the sequential colimit of its shifted type sequence
\begin{equation*}
\begin{tikzcd}
A_1 \arrow[r,"f_1"] & A_2 \arrow[r,"f_2"] & A_3 \arrow[r,"f_3"] & \cdots.
\end{tikzcd}
\end{equation*}
\exercise \label{ex:seqcolim_contr}Consider a type sequence
\begin{equation*}
\begin{tikzcd}
A_0 \arrow[r,"f_0"] & A_1 \arrow[r,"f_1"] & A_2 \arrow[r,"f_2"] & \cdots
\end{tikzcd}
\end{equation*}
and suppose that $f_n\htpy \mathsf{const}_{a_{n+1}}$ for some $a_n:\prd{n:\N}A_n$. Show that the sequential colimit is contractible.
\exercise Define the $\infty$-sphere $\sphere{\infty}$ as the sequential colimit of
\begin{equation*}
\begin{tikzcd}
\sphere{0} \arrow[r,"f_0"] & \sphere{1} \arrow[r,"f_1"] & \sphere{2} \arrow[r,"f_2"] & \cdots
\end{tikzcd}
\end{equation*}
where $f_0:\sphere{0}\to\sphere{1}$ is defined by $f_0(\bfalse)\jdeq \inl(\ttt)$ and $f_0(\btrue)\jdeq \inr(\ttt)$, and $f_{n+1}:\sphere{n+1}\to\sphere{n+2}$ is defined as $\susp(f_n)$. Use \cref{ex:seqcolim_contr} to show that $\sphere{\infty}$ is contractible.
\exercise Consider a type sequence
\begin{equation*}
\begin{tikzcd}
A_0 \arrow[r,"f_0"] & A_1 \arrow[r,"f_1"] & A_2 \arrow[r,"f_2"] & \cdots
\end{tikzcd}
\end{equation*}
in which $f_n:A_n\to A_{n+1}$ is weakly constant in the sense that
\begin{equation*}
\prd{x,y:A_n} f_n(x)=f_n(y)
\end{equation*}
Show that $A_\infty$ is a mere proposition.
\exercise Show that $\N$ is the sequential colimit of
\begin{equation*}
  \begin{tikzcd}
    \Fin(0) \arrow[r,"\inl"] & \Fin(1) \arrow[r,"\inl"] & \Fin(2) \arrow[r,"\inl"] & \cdots.
  \end{tikzcd}
\end{equation*}
\end{exercises}

\section{The image of a map and the replacement axiom}\label{chap:image}

In this section we will study two closely related concepts: the propositional truncation operation and the \emph{homotopy image} of a map $f:A\to X$.

The propositional truncation operation is a universal way of turning type a type $A$ into a proposition $\brck{A}$. Informally, the proposition $\brck{A}$ is the proposition that $A$ is inhabited. More precisely, the propositional truncation of $A$ comes equipped with a map $A\to\brck{A}$ and it is characterized by its universal property, which asserts that any map $A\to P$ into a proposition $P$ extends uniquely to a map $\brck{A}\to P$, as indicated in the diagram
\begin{equation*}
  \begin{tikzcd}
    A \arrow[dr] \arrow[d] \\
    \brck{A} \arrow[r,densely dotted] & P.
  \end{tikzcd}
\end{equation*}
Using the propositional truncation operation we can define all the logical connectives and quantifiers in type theory.

The idea of the image of a map $f:A\to X$ is that it is, in a way, the least subtype of $X$ that contains all the values of $f$. More precisely, the image of $f$ is an embedding $i:\im(f)\hookrightarrow X$ that fits in a commuting triangle
\begin{equation*}
  \begin{tikzcd}[column sep=tiny]
    A \arrow[rr,"q"] \arrow[dr,swap,"f"] & & \im(f) \arrow[dl,hook,"i"] \\
    \phantom{\im(f)} & X
  \end{tikzcd}
\end{equation*}
and satisfies the \emph{universal property} of the image of $f$. The universal property of the image of $f$ asserts that if a subtype $B\hookrightarrow X$ contains all the values of $f$, then it contains the image of $f$.
%In other words, for  asserts that there is a unique map $h:\im(f)\to B$ for which the tetrahedron
%\begin{equation*}
%  \begin{tikzcd}[column sep=large]
%    A \arrow[rr] \arrow[dr,"q"] \arrow[dddr,swap,"f"] & & B \arrow[dddl,"m"] \\
%    & \im(f) \arrow[ur,densely dotted,"h"] \arrow[dd,"i"] \\ \\
%    & X
%  \end{tikzcd}
%\end{equation*}
%commutes.
The image of a map can be constructed using the propositional truncation operation. In fact, we can also go the other way around: The propositional truncation of a type $A$ is the image of the map $A\to\unit$.

The final topic of this section is the type theoretic replacement axiom. A specific instance of the replacement axiom asserts that the image of any map $f:A\to\UU$ is equivalent to a type in $\UU$, provided that $A$ is equivant to a type in $\UU$. This property will be used to construct quotients in type theory, much in the same way as quotients are constructed in set theory.

We should note that the existence of the propositional truncation operation and the replacement axiom will be assumed for now. However, once we assume that universes are closed under pushouts, we will be able to construct the propositional truncations and we will be able to prove the replacement axiom. These constructions will be given in \cref{sec:join-construction}.

\subsection{Propositional truncations}\label{sec:propositional-truncation-up}

\begin{defn}
Let $A$ be a type, and let $f:A\to P$ be a map into a proposition $P$. We say that $f$ satisfies the \define{universal property of propositional truncation of $A$}\index{universal property!of propositional truncation} if for every proposition $Q$, the precomposition map
\begin{equation*}
\blank\circ f:(P\to Q)\to (A\to Q)
\end{equation*}
is an equivalence.
\end{defn}

\begin{rmk}
  Note that if $Q$ is a proposition, then the type $X\to Q$ is a proposition for any type $X$. Furthermore, recall from \cref{ex:equiv-bi-implication} that the map $(P\to Q)\to (A\to Q)$ is an equivalence as soon as there is a map in the converse direction. Therefore, in order to prove the universal property of the propositional truncation it suffices to show that
  \begin{equation*}
    (A\to Q)\to (P\to Q).
  \end{equation*}
  We also note that the universal property of the propositional truncation of a type is formulated with respect to all propositions, regardless of the universe they live in. 
\end{rmk}

\begin{eg}
  Suppose $A$ is a type that comes equipped with a point $a:A$, such as the booleans or the type of natural numbers. Then the constant map
  \begin{equation*}
    \const_\ttt: A\to\unit
  \end{equation*}
  satisfies the universal property of the propositional truncation. To see this, let $Q$ be an arbitrary proposition. It suffices to show that
  \begin{equation*}
    (A\to Q)\to (\unit\to Q).
  \end{equation*}
  To see this, let $f:A\to Q$. Then we have $f(a):Q$, so we define $\const_{f(a)}:\unit\to Q$. Thus we see that we have
  \begin{equation*}
    \lam{f}\const_{f(a)}:(A\to Q)\to (\unit\to Q).
  \end{equation*}
  This proves that $\const_\ttt:A\to \unit$ satisfies the universal property of the propositional truncation of $A$. 
\end{eg}

\begin{eg}
  If the type $A$ is already a proposition, then the identity function
  \begin{equation*}
    \idfunc:A\to A
  \end{equation*}
  satisfies the universal property of the propositional truncation. To see this, simply note that the precomposittion function with the identity function
  \begin{equation*}
    \blank\circ\idfunc : (A\to Q)\to (A\to Q)
  \end{equation*}
  is itself just the identity function. In particular, it is an equivalence.
\end{eg}

The universal property of the propositional truncation determines the propositional truncation up to equivalence. Such proofs of uniqueness from a universal property always follow the same pattern.

\begin{prp}\label{prp:propositional-truncation-3-for-2}
  Let $A$ be a type, and consider a commuting triangle
  \begin{equation*}
    \begin{tikzcd}[column sep=tiny]
      \phantom{P'} & A \arrow[dl,swap,"f"] \arrow[dr,"{f'}"] \\
      P \arrow[rr,swap,"h"] & & P'
    \end{tikzcd}
  \end{equation*}
  where $P$ and $P'$ are propositions. If any two of the following three assertions hold, so does the third:
  \begin{enumerate}
  \item The map $f$ satisfies the universal property of the propositional truncation of $A$.
  \item The map $f'$ satisfies the universal propertyof the propositional truncation of $A$.
  \item The map $h$ is an equivalence.
  \end{enumerate}
\end{prp}

\begin{proof}
  Note that the map $h:P\to P'$ is an equivalence if and only if for every proposition $Q$, the precomposition map
  \begin{equation*}
    \blank\circ h:(P'\to Q)\to (P\to Q)
  \end{equation*}
  is an equivalence. Thus, the claim follows by observing that for every proposition $Q$ we have the triangle
  \begin{equation*}
    \begin{tikzcd}[column sep=-1em]
      (P'\to Q) \arrow[rr,"\blank\circ h"] \arrow[dr,swap,"\blank\circ {f'}"] & & (P\to Q) \arrow[dl,"\blank\circ f"] \\
      & (A\to Q). & \phantom{(P'\to Q)}
    \end{tikzcd}
  \end{equation*}
\end{proof}

\begin{cor}\label{cor:uniquely-unique-brck}
  Consider two maps $f:A\to P$ and $f':A\to P'$ into propositions $P$ and $P'$, both satisfying the universal property of the propositional truncation of $A$. Then the type of equivalences $e:P \simeq P'$ for which the triangle
  \begin{equation*}
    \begin{tikzcd}[column sep=tiny]
      \phantom{P'} & A \arrow[dl,swap,"f"] \arrow[dr,"{f'}"] \\
      P \arrow[rr,swap,"e"] & & P'
    \end{tikzcd}
  \end{equation*}
  commutes, is contractible.
\end{cor}

\begin{rmk}
  Note that the triangles in \cref{prp:propositional-truncation-3-for-2,cor:uniquely-unique-brck} always commutes, since $P$ and $P'$ are assumed to be propositions.
\end{rmk}

Now that we have shown that propositional truncations are determined uniquely, we will assume that any universe is closed under propositional truncations.

\begin{axiom}
  Any universe $\UU$ is closed under propositional truncation: for any type $A:\UU$ there is a proposition $\brck{A}:\UU$ equipped with a map $\eta:A\to\brck{A}$ that satisfies the universal property of the propositional truncation.
\end{axiom}

Note that, given a family of propositions $P$ over a type $A$, the type $\sm{a:A}P(a)$ isn't necessarily a proposition. Instead, we think of $\sm{a:A}P(a)$ of the \emph{subtype} of $A$ containing the terms that satisfies $P$. Using the propositional truncation we can assert that there \emph{exists} a term in $A$ that satisfies $P$ without requiring one to construct it. 

\begin{defn}
Let $P:A\to \prop$ be a family of propositions over a type $A$. Then we define
\begin{equation*}
\exists_{(a:A)}P(a)\defeq \brck{\sm{a:A}P(a)}.
\end{equation*}
\end{defn}

Similarly, we can define the disjunction of two propositions $P$ and $Q$ to be the \emph{proposition} $\brck{P+Q}$, which clearly satisfies the universal property of disjunction\footnote{Alternatively, we have shown in \cref{ex:join_propositions} that the join $\join{P}{Q}$ also is a proposition that satisfies the universal property of disjunction.}. In the following table we give an overview of the logical connectives on propositions.

\begin{center}
\begin{tabular}{lll}
\toprule
\emph{Logical connective} & \emph{Interpretation in HoTT} & \emph{Proof of being a proposition} \\
\midrule
$\top$ & $\unit$ & \cref{eg:prop_contr} \\
$\bot$ & $\emptyt$ & \cref{eg:prop_contr}\\
$P\land Q$ & $P\times Q$ & \cref{ex:istrunc_sigma} \\
$P\lor Q$ & $\brck{P+Q}$ & By definition \\
$P\to Q$ & $P\to Q$ & \cref{cor:funtype_trunc} \\
$P\leftrightarrow Q$ & $\eqv{P}{Q}$ \\
$\neg P$ & $P\to\emptyt$ & \cref{cor:funtype_trunc} \\
$\forall x.P(x)$ & $\prd{x:A}P(x)$ & \cref{thm:trunc_pi} \\
$\exists x.P(x)$ & $\brck{\sm{x:A}P(x)}$ & By definition \\
$\exists! x.P(x)$ & $\iscontr(\sm{x:A}P(x))$ & \cref{ex:isprop_istrunc} \\
\bottomrule
\end{tabular}
\end{center}

\subsection{The image of a map}\label{sec:image-construction}
 Note that there is quite a lot of information in this diagram: not only are there the three small commuting triangles; there is also the large commuting triange in the back, and there is a three-dimensional solid filling the space between the four triangles. We make the following definition, in order to express the universal property of the image efficiently.

\begin{defn}
  Let $f:A\to X$ and $g:B\to X$ be maps. A \define{morphism} from $f$ to $g$ over $X$ consists of a map $h:A\to B$ equipped with a homotopy $H:f\htpy g\circ h$ witnessing that the triangle
\begin{equation*}
\begin{tikzcd}[column sep=tiny]
A \arrow[rr,"h"] \arrow[dr,swap,"f"] & & B \arrow[dl,"g"] \\
& X
\end{tikzcd}
\end{equation*}
commutes. Thus, we define the type
\begin{equation*}
\mathrm{hom}_X(f,g)\defeq\sm{h:A\to B}f\htpy g\circ h.
\end{equation*}
Composition of morphisms over $X$ is defined by
\begin{equation*}
  (k,K)\circ (h,H) \defeq (k\circ h,\ct{H}{(K\cdot h)}).
\end{equation*}
\end{defn}

\begin{defn}
Consider a commuting triangle
\begin{equation*}
\begin{tikzcd}[column sep=small]
A \arrow[rr,"q"] \arrow[dr,swap,"f"] & & I \arrow[dl,"i"] \\
& X
\end{tikzcd}
\end{equation*}
with $H:f\htpy i\circ q$, where $i$ is an embedding\index{embedding}.
We say that $i$ has the \define{universal property of the image of $f$}\index{universal property!of the image} if the map
\begin{equation*}
\blank\circ(q,H) : \mathrm{hom}_X(i,m)\to\mathrm{hom}_X(f,m)
\end{equation*}
is an equivalence for every embedding $m:B\to X$. 
\end{defn}

\begin{rmk}
  Consider a commuting triangle
\begin{equation*}
\begin{tikzcd}[column sep=small]
A \arrow[rr,"q"] \arrow[dr,swap,"f"] & & I \arrow[dl,"i"] \\
& X
\end{tikzcd}
\end{equation*}
with $H:f\htpy i\circ q$, where $i$ is an embedding. Then it is not hard to see that the embedding $i$ satisfies the universal property of the image inclusion if and only if for every commuting triangle
\begin{equation*}
  \begin{tikzcd}[column sep=small]
    A \arrow[dr,swap,"f"] \arrow[rr,"g"] & & B \arrow[dl,"m"] \\
    & X
  \end{tikzcd}
\end{equation*}
with $G:f\htpy m\circ g$, where $m$ is an embedding, the type of quadruples $(h,K,L,M)$ consisting of
\begin{enumerate}
\item a map $h:I\to B$,
\item a homotopy $K:i\htpy m\circ h$ witnessing that the triangle
  \begin{equation*}
    \begin{tikzcd}[column sep=small]
      I \arrow[rr,"h"] \arrow[dr,swap,"i"] & & B \arrow[dl,"m"] \\
      & X
    \end{tikzcd}
  \end{equation*}
  commutes,
\item a homotopy $L:g\htpy h\circ q$ witnessing that the triangle
  \begin{equation*}
    \begin{tikzcd}[column sep=small]
      A \arrow[rr,"q"] \arrow[dr,swap,"g"] & & I \arrow[dl,"h"] \\
      & B
    \end{tikzcd}
  \end{equation*}
  commutes,
\item a homotopy $M:\ct{H}{(K\cdot q)}\htpy\ct{G}{(m\cdot L)}$ witnessing that the square
  \begin{equation*}
    \begin{tikzcd}
      f \arrow[d,swap,"H"] \arrow[r,"G"] & m\circ g \arrow[d,"m\cdot L"] \\
      i\circ q \arrow[r,swap,"K\cdot q"] & m\circ h\circ g
    \end{tikzcd}
  \end{equation*}
  commutes,
\end{enumerate}
is contractible. However, the situation is in fact much simpler, because the type $\mathrm{hom}_X(f,m)$ is a proposition whenever $m$ is an embedding.
\end{rmk}

\begin{lem}
For any $f:A\to X$ and any embedding\index{embedding} $m:B\to X$, the type $\mathrm{hom}_X(f,m)$ is a proposition.
\end{lem}

\begin{proof}
  Recall from \cref{ex:triangle_fib} that the type $\mathrm{hom}_X(f,m)$ is equivalent to the type
  \begin{equation*}
    \prd{x:X}\fib{f}{x}\to\fib{m}{x}.
  \end{equation*}
  Therefore it suffices to show that this type is a proposition. Recall from \cref{cor:prop_emb} that a map is an embedding if and only if its fibers are propositions.
  Thus we see that the type $\prd{x:X}\fib{f}{x}\to\fib{m}{x}$ is a product of propositions, hence it is a proposition by \cref{thm:trunc_pi}.
\end{proof}

\begin{prp}\label{prp:simplifly-universal-property-image}
  Consider a commuting triangle
  \begin{equation*}
    \begin{tikzcd}[column sep=small]
      A \arrow[rr,"q"] \arrow[dr,swap,"f"] & & I \arrow[dl,"i"] \\
      & X
\end{tikzcd}
  \end{equation*}
  with $H:f\htpy i\circ q$, where $i$ is an embedding. Then the following are equivalent:
  \begin{enumerate}
  \item The embedding $i$ satisfies the universal property of the image inclusion of $f$.
  \item For every embedding $m:B\to X$ there is a map
    \begin{equation*}
      \mathrm{hom}_X(f,m)\to\mathrm{hom}_X(i,m).
    \end{equation*}
  \end{enumerate}
\end{prp}

\begin{proof}
Since $\mathrm{hom}_X(f,m)$ is a proposition for every every embedding $m:B\to X$, the claim follows immediately by \cref{ex:prop_equiv}.
\end{proof}

Just as in the cases for pullbacks and pushouts, the universal property of the image implies that the image is determined uniquely. We will show here that the type of image factorizations of any map is a proposition. In \cref{sec:image-construction} we will construct the image, after constructing the propositional truncation.

\begin{prp}
  Let $f$ be a map, and consider two commuting triangles
  \begin{equation*}
    \begin{tikzcd}[column sep=tiny]
      A \arrow[dr,swap,"f"] \arrow[rr,"q"] & & B \arrow[dl,"i"] &[2em] A \arrow[dr,swap,"f"] \arrow[rr,"{q'}"] & & B' \arrow[dl,"{i'}"] \\
      & X & & \phantom{B'} & X
    \end{tikzcd}
  \end{equation*}
  with $I:f\htpy i\circ q$ and $I':f\htpy i'\circ q'$, in which $i$ and $i'$ are assumed to be embeddings. Moreover, consider
  \begin{equation*}
    (h,H):\mathrm{hom}_X(i,i')
  \end{equation*}
  equipped with an identification $(h,H)\circ(q,I)=(q',I')$ in $\mathrm{hom}_X(f,i')$. Then, if any two of the following properties hold, so does the third:
  \begin{enumerate}
  \item The embedding $i$ satisfies the universal property of the image inclusion of $f$.
  \item The embedding $i'$ satisfies the universal property of the image inclusion of $f$.
  \item The map $h$ is an equivalence.
  \end{enumerate}
\end{prp}

\begin{proof}
  Consider an embedding $m:C\to X$. Then we have a commuting triangle
  \begin{equation*}
    \begin{tikzcd}[column sep=-1em]
      \mathrm{hom}_X(i',m) \arrow[rr,"{\blank\circ(h,H)}"] \arrow[dr,swap,"{\blank\circ(q',I')}"] & & \mathrm{hom}_X(i,m) \arrow[dl,"{\blank\circ(q,I)}"] \\
      & \mathrm{hom}_X(f,m), & \phantom{\mathrm{hom}_X(i',m)}
    \end{tikzcd}
  \end{equation*}
  so it follows that if any two of these maps are equivalences, then so is the third. The claim now follows by the observation that $\blank\circ(h,H)$ is an equivalence for every embedding $m:C\to X$ if and only if $h$ is an equivalence.
\end{proof}

\begin{cor}\label{cor:uniqueness-image}
  Consider two image factorizations
  \begin{equation*}
    \begin{tikzcd}[column sep=tiny]
      A \arrow[dr,swap,"f"] \arrow[rr,"q"] & & B \arrow[dl,"i"] &[2em] A \arrow[dr,swap,"f"] \arrow[rr,"{q'}"] & & B' \arrow[dl,"{i'}"] \\
      & X & & \phantom{B'} & X
    \end{tikzcd}
  \end{equation*}
  of a map $f$, with $I:f\htpy i\circ q$ and $I':f\htpy i'\circ q'$. Then the type of $(e,H):\mathrm{hom}_X(i,i')$ in which $e$ is an equivalence, equipped with an identification
  \begin{equation*}
    (e,H)\circ(q,I)=(q',I')
  \end{equation*}
  in $\mathrm{hom}_X(f,i')$, is contractible.
\end{cor}

The image of a map $f:A\to X$ can now be defined using the propositional truncation:

\begin{defn}
For any map $f:A\to X$ we define the \define{image}\index{image} of $f$ to be the type
\begin{equation*}
\im(f) \defeq \sm{x:X}\brck{\fib{f}{x}}.
\end{equation*}
Furthermore, we define:
\begin{enumerate}
\item The \define{image inclusion}
  \begin{equation*}
    i_f:\im(f)\to X
  \end{equation*}
  to be the projection $\proj 1$.
\item The map
  \begin{equation*}
    q_f:A\to\im(f)
  \end{equation*}
  to be the map given by $q_f(x)\defeq(f(x),\eta(x,\refl{f(x)}))$.
\item The homotopy $I_f:f\htpy i_f\circ q_f$ witnessing that the triangle
  \begin{equation*}
    \begin{tikzcd}[column sep=tiny]
      A \arrow[rr,"q_f"] \arrow[dr,swap,"f"] & & \im(f) \arrow[dl,"i_f"] \\
      \phantom{\im(f)} & X
    \end{tikzcd}
  \end{equation*}
  commutes, to be given by $I_f(x)\defeq\refl{f(x)}$.
\end{enumerate}
\end{defn}

\begin{prp}
  The image inclusion $i_f:\im(f)\to X$ of any map $f:A\to X$ is an embedding.
\end{prp}

\begin{proof}
  The fiber of $i_f$ at $x:X$ is equivalent to the type $\brck{\fib{f}{x}}$. In particular we see that the fibers are propositions, so $i_f$ is an embedding.
\end{proof}

\begin{thm}
  The image inclusion $i_f:\im(f)\to X$ of any map $f:A\to X$ satisfies the universal property of the image inclusion of $f$.
\end{thm}

\begin{proof}
  Consider an embedding $m:B\to X$. Note that we have a commuting square
  \begin{equation*}
    \begin{tikzcd}[column sep=6em]
      \mathrm{hom}_X(i_f,m) \arrow[d] \arrow[r] & \mathrm{hom}_X(f,m) \arrow[d] \\
      \Big(\prd{x:X}\fib{i_f}{x}\to\fib{m}{x}\Big) \arrow[r,swap,"h\mapsto{\lam{x}h_x\circ\varphi_x}"] & \Big(\prd{x:X}\fib{f}{x}\to\fib{m}{x}\Big)
    \end{tikzcd}
  \end{equation*}
  The vertical maps are of the form
  \begin{equation*}
    (h,H) \mapsto \lam{x}{(y,p)}(h(y),\ct{H(y)^{-1}}{p}),
  \end{equation*}
  and they are both equivalences. The map
  \begin{equation*}
    \varphi_x:\fib{f}{x}\to\fib{i_f}{x}
  \end{equation*}
  given by $\varphi_x(a,p)\defeq((h(a),\eta(a,p)),p)$ is a propositional truncation for every $x:X$. Therefore it follows that the map
  \begin{equation*}
    (\fib{i_f}{x}\to\fib{m}{x})\to(\fib{f}{x}\to\fib{m}{x})
  \end{equation*}
  is an equivalence, for every $x:X$. Thus we conclude that the bottom map in the above square is an equivalence, which implies that the top map is an equivalence. 
\end{proof}

\begin{eg}
  An important special case of the homotopy image of a map is the image of the terminal projection
\begin{equation*}
  \const_\ttt : A \to \unit,
\end{equation*}
which results in an embedding $I\hookrightarrow \unit$. Embeddings into the unit type are in fact just propositions. To see this, note that
\begin{align*}
\sm{A:\UU}{f:A\to\unit}\isemb(f)
& \eqvsym \sm{A:\UU}\isemb(\const_\ttt) \\
& \eqvsym \sm{A:\UU}\prd{x:\unit}\isprop(\fib{\const_\ttt}{x}) \\
& \eqvsym \sm{A:\UU}\isprop(\fib{\const_\ttt}{\ttt}) \\
& \eqvsym \sm{A:\UU}\isprop(A).
\end{align*}
Therefore, the universal property of the image of the map $A\to\unit$ is equivalently described as a proposition $P$ satisfying the universal property of the propositional truncation.
\end{eg}

\subsection{Surjective maps}

Another application of the propositional truncation is the notion of surjective map.

\begin{defn}
A map $f:A\to B$ is said to be \define{surjective} if there is a term of type
\begin{equation*}
\issurj(f)\defeq \prd{y:B}\brck{\fib{f}{b}}.
\end{equation*}
\end{defn}

\begin{eg}
Any equivalence is a surjective map, and so is any map that has a section (those are sometimes called \define{split epimorphisms}). Other examples include the base point inclusion $\unit\to\sphere{n}$ for any $n\geq 1$. 
\end{eg}

\begin{prp}\label{prp:surjective}
  Consider a map $f:A\to B$. Then the following are equivalent:
  \begin{enumerate}
  \item The map $f:A\to B$ is surjective.
  \item For any family $P$ of propositions over $B$, the precomposition map
    \begin{equation*}
      \blank\circ f : \Big(\prd{y:B}P(y)\Big)\to\Big(\prd{x:A}P(f(x))\Big)
    \end{equation*}
    is an equivalence.
  \end{enumerate}
\end{prp}

\begin{proof}
  Suppose first that $f$ is surjective, and consider the commuting square
  \begin{equation*}
    \begin{tikzcd}[column sep=6em]
      \Big(\prd{y:B}P(y)\Big) \arrow[r,"\blank\circ f"] \arrow[d,swap,"h\mapsto\lam{y}\const_{h(y)}"] & \Big(\prd{x:A}P(f(x))\Big)  \\
      \Big(\prd{y:B}\brck{\fib{f}{y}}\to P(y)\Big) \arrow[r,swap,"h\mapsto\lam{y}h(y)\circ\eta"] & \Big(\prd{y:B}\fib{f}{y}\to P(y)\Big) \arrow[u,swap,"{h\mapsto\lam{x}h(f(x),(x,\refl{f(x)}))}"]
    \end{tikzcd}
  \end{equation*}
  In this square, the bottom map is an equivalence by the universal property of the propositional truncation of $\fib{f}{y}$. The map on the right is also easily seen to be an equivalence. Furthermore, the map on the left is an equivalence by the assumption that $f$ is surjective, from which it follows that the types $\brck{\fib{f}{y}}$ are contractible. Therefore it follows that the top map is an equivalence, which completes the proof that (i) implies (ii).

  For the converse, it follows immediately from the assumption (ii) that
  \begin{equation*}
    \blank\circ f : \Big(\prd{y:B}\brck{\fib{f}{y}}\Big)\to\Big(\prd{x:A}\brck{\fib{f}{f(x)}}\Big)
  \end{equation*}
  is an equivalence. Hence it suffices to construct a term of type $\brck{\fib{f}{f(x)}}$ for each $x:A$. This is easy, because we have
  \begin{equation*}
    \eta(x,\refl{f(x)}):\brck{\fib{f}{f(x)}}.\qedhere.
  \end{equation*}
\end{proof}

\begin{thm}\label{thm:surjective}
Consider a commuting triangle
\begin{equation*}
\begin{tikzcd}[column sep=tiny]
A \arrow[rr,"q"] \arrow[dr,swap,"f"] & & B \arrow[dl,"m"] \\
& X
\end{tikzcd}
\end{equation*}
in which $m$ is an embedding. Then the following are equivalent:
\begin{enumerate}
\item The embedding $m$ satisfies the universal property of the image inclusion of $f$.
\item The map $q$ is surjective.
\end{enumerate}
\end{thm}

\begin{proof}
  First assume that $m$ satisfies the universal property of the image inclusion of $f$, and consider the composite function
  \begin{equation*}
    \begin{tikzcd}
      \Big(\sm{y:B}\brck{\fib{q}{y}}\Big) \arrow[r,"\proj 1"] & B \arrow[r,"m"] & X.
    \end{tikzcd}
  \end{equation*}
  Note that $m\circ\proj 1$ is a composition of embeddings, so it is an embedding. By the universal property of $m$ there is a unique map $h$ for which the triangle
  \begin{equation*}
    \begin{tikzcd}[column sep=0]
      B \arrow[dr,swap,"m"] \arrow[rr,densely dotted,"h"] & & \sm{y:B}\brck{\fib{q}{y}} \arrow[dl,"m\circ\proj 1"] \\
      \phantom{\sm{y:B}\brck{\fib{q}{y}}} & X
    \end{tikzcd}
  \end{equation*}
  commutes. Now note that $\proj 1\circ h$ is a map such that $m\circ (\proj 1\circ h)\htpy m$. The identity function is another map for which we have $m\circ\idfunc\htpy m$, so it follows by uniqueness that $\proj 1\circ h\htpy \idfunc$. In other words, the map $h$ is a section of the projection map. Therefore we obtain by \cref{ex:pi_sec} a dependent function
  \begin{equation*}
    \prd{b:B}\brck{\fib{q}{b}},
  \end{equation*}
  showing that $q$ is surjective.

  For the converse, suppose that $q$ is surjective. To prove that $m$ satisfies the universal property of the image factorization of $f$, it suffices to construct an equivalence
  \begin{equation*}
    \mathrm{hom}_X(f,m')\to\mathrm{hom}_X(m,m'),
  \end{equation*}
  for any embedding $m':B'\to X$. To see that there is such an equivalence, we make the following calculation
  \begin{align*}
    \mathrm{hom}_X(m,m') & \simeq \prd{x:X}\fib{m}{x}\to\fib{m'}{x} \\
                         & \simeq \prd{b:B}\fib{m'}{m(b)} \\
                         & \simeq \prd{a:A}\fib{m'}{m(q(a))} \\
                         & \simeq \prd{a:A}\fib{m'}{f(a)} \\
                         & \simeq \prd{x:X}\fib{f}{x}\to\fib{m'}{x} \\
                         & \simeq \mathrm{hom}_X(f,m').
  \end{align*}
  In this calculation, the first and last equivalence hold by \cref{ex:triangle_fib}. The second and second to last equivalences hold by \cref{ex:pi-fib}. The third equivalence holds by \cref{prp:surjective}, since $q$ is assumed to be surjective, and the fourth equivalence holds since we have a homotopy $f\htpy m\circ f$.
\end{proof}

\begin{cor}
  Every map factors uniquely as a surjective map followed by an embedding.
\end{cor}

\begin{proof}
  Consider a map $f:A\to X$, and two factorizations
  \begin{equation*}
    \begin{tikzcd}[column sep=tiny]
      A \arrow[rr,"q"] \arrow[dr,swap,"f"] & & B \arrow[dl,"i"] &[3em] A \arrow[rr,"{q'}"] \arrow[dr,swap,"f"] & & B' \arrow[dl,"{i'}"] \\
      & X & & & X
    \end{tikzcd}
  \end{equation*}
  of $f$ where $m$ and $m'$ are embeddings, and $q$ and $q'$ are surjective. Then both $m$ and $m'$ satisfy the universal property of the image factorization of $f$ by \cref{thm:surjective}. Now it follows by \cref{cor:uniqueness-image} that the type of $(e,H):\mathrm{hom}_X(i,i')$ in which $e$ is an equivalence, equipped with an identification
  \begin{equation*}
    (e,H)\circ(q,I)=(q',I')
  \end{equation*}
  in $\mathrm{hom}_X(f,i')$, is contractible.
\end{proof}

\subsection{Type theoretic replacement}

\begin{comment}
We have constructed the set quotient $A/R$ as the image of the equivalence relation
\begin{equation*}
  R:A\to \UU^A.
\end{equation*}
However, the type $\UU^A$ is itself in the next universe $\UU^+$. Hence the quotient is also in the universe $\UU^+$. We prove in this section that $A/R$ is nevertheless equivalent to a type in $\UU$. In other words, we show that $A/R$ is \emph{essentially} small.
\end{comment}

\begin{defn}\label{defn:ess_small}
\begin{enumerate}
\item A type $A$ is said to be \define{essentially small}\index{essentially small!type} if there is a type $X:\UU$ and an equivalence $\eqv{A}{X}$. We write\index{ess_small(A)@{$\mathsf{ess\usc{}small}(A)$}}
\begin{equation*}
\mathsf{ess\usc{}small}(A)\defeq\sm{X:\UU}\eqv{A}{X}.
\end{equation*}
\item A map $f:A\to B$ is said to be \define{essentially small}\index{essentially small!map} if for each $b:B$ the fiber $\fib{f}{b}$ is essentially small.
We write\index{ess_small(f)@{$\mathsf{ess\usc{}small}(f)$}}
\begin{equation*}
\mathsf{ess\usc{}small}(f)\defeq\prd{b:B}\mathsf{ess\usc{}small}(\fib{f}{b}).
\end{equation*}
\item A type $A$ is said to be \define{locally small}\index{locally small!type} if for every $x,y:A$ the identity type $x=y$ is essentially small.
We write\index{loc_small(A)@{$\mathsf{loc\usc{}small}(A)$}}
\begin{equation*}
\mathsf{loc\usc{}small}(A)\defeq \prd{x,y:A}\mathsf{ess\usc{}small}(x=y).
\end{equation*}
\end{enumerate}
\end{defn}

\begin{eg}
  \begin{enumerate}
  \item Any essentially $\UU$-small type is also locally $\UU$-small.
  \item Any univalent universe $\UU$ is locally $\UU$-small, because by the univalence axiom we have equivalences
    \begin{equation*}
      (A=B)\simeq (A\simeq B)
    \end{equation*}
    for each $A,B:\UU$, and the type $A\simeq B$ is in $\UU$.
  \item Any proposition is locally small with respect to any universe $\UU$.
  \item For any family $P$ of locally $\UU$-small types over a essentially $\UU$-small type $A$, the dependent product $\prd{x:A}P(x)$ is locally $\UU$-small. In particular, any type $A\to B$ of functions from an essentially small type into a locally small type is again locally small.
  \end{enumerate}
\end{eg}

\begin{lem}\label{lem:isprop_ess_small}
The type $\mathsf{ess\usc{}small}(A)$ is a proposition for any type $A$.\index{essentially small!is a proposition}
\end{lem}

\begin{proof}
Let $A$ be a type, not necessarily in $\UU$. In order to show that $\mathsf{ess\usc{}small}(A)$ is a proposition, we will use \cref{lem:isprop_eq} and show that for any $X:\UU$ and any equivalence $e:A\simeq X$, the type
\begin{equation*}
\sm{Y:\UU}\eqv{A}{Y}
\end{equation*}
is contractible. Note that we have an equivalence
\begin{equation*}
\eqv{\Big(\sm{Y:\UU}\eqv{X}{Y}\Big)}{\Big(\sm{Y:\UU}\eqv{A}{Y}\Big)}
\end{equation*}
because precomposing with the equivalence $e:A \simeq X$ is an equivalence. However, the type $\sm{Y:\UU}\eqv{X}{Y}$ is contractible by \cref{thm:univalence}. This shows that $\mathsf{ess\usc{}small}(A)$ is equivalent to a contractible type, assuming that $A$ is essentially small.
\end{proof}

\begin{cor}
For each function $f:A\to B$, the type $\mathsf{ess\usc{}small}(f)$ is a proposition, and for each type $X$ the type $\mathsf{loc\usc{}small}(X)$ is a proposition.
\end{cor}

\begin{proof}
This follows from the fact that propositions are closed under dependent products, established in \cref{thm:trunc_pi}.
\end{proof}

Recall that in set theory, the replacement axiom asserts that for any family of sets $\{X_i\}_{i\in I}$ indexed by a set $I$, there is a set $X[I]$ consisting of precisely those sets $x$ for which there exists an $i\in I$ such that $x\in X_i$. In other words: the image of a set-indexed family of sets is again a set. Without the replacement axiom, $X[I]$ would be a class. In the following corollary we establish a type-theoretic analogue of the replacement axiom: the image of a family of small types indexed by a small type is again (essentially) small.

\begin{axiom}\label{axiom:replacement}
  For any map $f:A\to B$ from an essentially small type $A$ into a locally small type $B$, the image of $f$ is again essentially small.
\end{axiom}

\begin{cor}
  Consider a $\UU$-small type $A$, and an equivalence relation $R$ over $A$ valued in the $\UU$-small propositions. Then the set quotient $A/R$ is essentially small.
\end{cor}

\begin{exercises}
  \exercise Show that if $f:A\to X$ is an embedding, then $f$ itself satisfies the universal property of the image inclusion of $f$.
  \exercise Show that
  \begin{equation*}
    \eqv{\brck{A}}{\prd{P:\prop}(A\to P)\to P}
  \end{equation*}
  for any type $A:\UU$. This is called the \define{impredicative encoding} of the propositional truncation.
  \exercise For any $B:A\to\UU$, construct an equivalence
  \begin{equation*}
    \eqv{\Big(\exists_{(a:A)}\brck{B(a)}\Big)}{\brck{\sm{a:A}B(a)}}
  \end{equation*}
  %\exercise \label{also}(Mart\'in Escard\'o) For any two propositions $P$ and $Q$, define
  %\begin{equation*}
  %P\boxplus Q \defeq ((P\to Q)\to Q)\times ((Q\to P)\to P).
  %\end{equation*}
  %\begin{subexenum}
  %\item Show that $P\lor Q\to P\boxplus Q$ and $P\boxplus Q\to\neg(\neg P\land \neg Q)$.
  %\end{subexenum}
  %\item \label{ex:brck_comp} Formulate the computation rule corresponding to the path constructor $\mu$. That is, compute the type of $\apd{\rec{\brck{\blank}}(f,g)}{\mu(x,y)}$, and find a canonical element in it.
  \exercise Let
  \begin{tikzcd}
    P_0 \arrow[r] & P_1 \arrow[r] & P_2 \arrow[r] & \cdots
  \end{tikzcd}
  be a sequence of propositions. Show that
  \begin{equation*}
    \eqv{\colim_n(P_n)}{\exists_{(n:\N)} P_n}.
  \end{equation*}
  \exercise Show that the relation $x,y\mapsto\brck{x=y}$ is an equivalence relation, on any type.
  %\exercise Let $f:A\to X$ be a map. Construct an equivalence
  %\begin{equation*}
  %\eqv{\Big(\sm{y:\mathsf{join\usc{}power}_X(n,A)}f(x)=f^{\ast n}(y)\Big)}{\Big(\sm{y:A}f(x)=f(y)\Big)^{\ast n}}
  %\end{equation*}
  %for any $x:A$.
  \exercise Let $f:A\to B$ be a map. Show that the following are equivalent:
  \begin{enumerate}
  \item The commuting square
    \begin{equation*}
      \begin{tikzcd}
        A \arrow[d,swap,"f"] \arrow[r] & \brck{A} \arrow[d,"\brck{f}"] \\
        B \arrow[r] & \brck{B}.
      \end{tikzcd}
    \end{equation*}
    is a pullback square.
  \item There is a term of type $A\to\isequiv(f)$.
  \item The commuting square
    \begin{equation*}
      \begin{tikzcd}
        A\times A \arrow[r,"f\times f"] \arrow[d,swap,"\proj 1"] & B \times B \arrow[d,"\proj 1"] \\
        A \arrow[r,swap,"f"] & B
      \end{tikzcd}
    \end{equation*}
    is a pullback square. 
  \end{enumerate}
  \exercise Consider a pullback square
  \begin{equation*}
    \begin{tikzcd}
      A' \arrow[d,swap,"{f'}"] \arrow[r,"p"] & A \arrow[d,"f"] \\
      B' \arrow[r,swap,"q"] & B,
    \end{tikzcd}
  \end{equation*}
  in which $q:B'\to B$ is surjective. Show that if $f':A'\to B'$ is an embedding, then so is $f:A\to B$.
  \exercise Show that a type $A$ is a proposition if and only if the map $\inl:A\to \join{A}{A}$ is an equivalence.
  \exercise Let $A$ be a type, and let $P$ be a proposition.
  \begin{subexenum}
  \item Show that $\inl:P\to \join{P}{A}$ is an embedding.
  \item Show that $\inl:P\to \join{P}{A}$ is an equivalence if and only if $\brck{A}\to P$ holds.
  \end{subexenum}
  \exercise Consider a family of diagrams of the form
  \begin{equation*}
    \begin{tikzcd}
      A_i \arrow[r] \arrow[d,swap,"{f_i}"] &
      C \arrow[r] \arrow[d,"g"] & X \arrow[d,"h"] \\
      B_i \arrow[r] & D \arrow[r] & Y 
    \end{tikzcd}
  \end{equation*}
  indexed by $i:I$, in which the left squares are pullback squares,
  and assume that the induced map
  \begin{equation*}
    \Big(\sm{i:I}B_i\Big)\to D
  \end{equation*}
  is surjective. Show that the following are equivalent:
  \begin{enumerate}
  \item For each $i:I$ the outer rectangle is a pullback square.
  \item The right square is a pullback square.
  \end{enumerate}
  Hint: By \cref{thm:descent-Sigma} it suffices to prove this equivalence for a single diagram of the form
  \begin{equation*}
    \begin{tikzcd}
      A \arrow[r] \arrow[d,swap,"{f}"] &
      C \arrow[r] \arrow[d,swap,"g"] & X \arrow[d,"h"] \\
      B \arrow[r] & D \arrow[r] & Y 
    \end{tikzcd}
  \end{equation*}
  where the map $B \to D$ is assumed to be surjective.
  \exercise
  \begin{subexenum}
  \item \label{ex:surjective-precomp}Consider a map $f:A\to B$. Show that the following are equivalent:
    \begin{enumerate}
      \item The map $f$ is surjective.
      \item For every set $C$, the precomposition function
        \begin{equation*}
          \blank\circ f:(B\to C)\to (A\to C)
        \end{equation*}
        is an embedding.
    \end{enumerate}
    Hint: To show that (ii) implies (i), use the assumption with the set $C\jdeq\prop$.
  \item Give an example of a surjective map $f:A\to B$ and a type $C$ which is not a set, such that the precomposition function
    \begin{equation*}
      \blank\circ f:(B\to C)\to (A\to C)
    \end{equation*}
    is not an embedding.
  \end{subexenum}
  \exercise Consider a pushout square
  \begin{equation*}
    \begin{tikzcd}
      S \arrow[d,swap,"f"] \arrow[r,"g"] & B \arrow[d,"j"] \\
      A \arrow[r,swap,"i"] & X.
    \end{tikzcd}
  \end{equation*}
  \begin{subexenum}
  \item Show that if $f$ is surjective, then so is $j$.
  \item Show that the two small squares in the diagram 
    \begin{equation*}
      \begin{tikzcd}
        S \arrow[r,"g"] \arrow[d,swap,"q_f"] & B \arrow[d,"q_j"] \\
        \im(f) \arrow[r,densely dotted] \arrow[d,swap,"i_f"] & \im(j) \arrow[d,"i_j"] \\
        A \arrow[r,swap,"i"] & X
      \end{tikzcd}
    \end{equation*}
    are both pushout squares, and that the bottom square is also a pullback square.
  \end{subexenum}
  \exercise Consider a pullback square
  \begin{equation*}
    \begin{tikzcd}
      E' \arrow[d,swap,"{p'}"] \arrow[r,"g"] & E \arrow[d,"p"] \\
      B' \arrow[r,swap,"f"] & B
    \end{tikzcd}
  \end{equation*}
  in which $p$ is assumed to be surjective. Show that $p'$ is also surjective, and show that the following are equivalent:
  \begin{enumerate}
  \item The map $f$ is an equivalence.
  \item The map $g$ is an equivalence.
  \end{enumerate}
  \exercise
  Consider a map $f:A\to B$. Show that the following are equivalent:
  \begin{enumerate}
  \item $f$ is an equivalence.
  \item $f$ is both surjective and an embedding.
  \end{enumerate}
\end{exercises}

\endinput

\begin{thm}
Consider a commuting triangle
\begin{equation*}
\begin{tikzcd}[column sep=small]
A \arrow[rr,"i"] \arrow[dr,swap,"f"] & & B \arrow[dl,"m"] \\
& X
\end{tikzcd}
\end{equation*}
with $I:f\htpy m\circ i$, where $m$ is an embedding. The following are equivalent:
\begin{enumerate}
\item $m$ satisfies the universal property of the image of $f$.
\item for each $x:X$, the proposition $\fib{m}{x}$ satisfies the universal property of the propositional truncation of $\fib{f}{x}$.
\end{enumerate}
\end{thm}



\section{Set quotients}

In this section we construct the quotient of a type by an equivalence relation. By an equivalence relation we understand a binary relation $R:A\to(A\to\prop)$ which is reflexive, symmetric, and transitive. In particular, we note that equivalence relations take values in $\prop$. The quotient $A/R$ is constructed as the type of equivalence classes, which is just the image of the map $R:A\to (A\to\prop)$. Thus, our construction of the quotient by an equivalence relation is very much like the classical construction of a quotient set.

However, there is a subtle issue. What is the universe level of the quotient $A/R$? Let us suppose that $\UU$ is a universe that contains $A$ and each $R(x,y)$. Then $\prop$, the type of propositions in $\UU$, is a type in the universe $\UU^+$, constructed in \cref{rmk:universe-constructions}. Therefore the type $\prop^A$ as well as the quotient $A/R$ are also types in $\UU^+$. That seems unfortunate, because in Zermelo-Fraenkel set theory the quotient of a set by an equivalence relation is an ordinary set, and not a more general class.

In Zermelo-Fraenkel set theory quotients are are sets because of the axiom schema of replacement. The replacement axioms assert that the image of any function is again a set. This leads us to wonder about a type theoretical variant of the replacement axioms. Indeed, there is such a variant. The type theoretic replacement property asserts that for any map $f:A\to B$ from a type $A$ in $\UU$ to a type $B$ of which the \emph{identity types} are equivalent to types in $\UU$, the image of $f$ is also equivalent to a type in $\UU$. Moreover, the type theoretic replacement property is a theorem! We prove it in \cref{thm:replacement}, using the univalence axiom and a new construction of the image of a map.

\subsection{Equivalence relations}

\begin{defn}\label{defn:eq_rel}
Let $R:A\to (A\to\prop)$ be a binary relation valued in the propositions. We say that $R$ is an \define{equivalence relation}\index{equivalence relation} if $R$ comes equipped with
\begin{align*}
\rho & : \prd{x:A}R(x,x) \\
\sigma & : \prd{x,y:A} R(x,y)\to R(y,x) \\
\tau & : \prd{x,y,z:A} R(x,y)\to (R(y,z)\to R(x,z)),
\end{align*}
witnessing that $R$ is reflexive, symmetric, and transitive.
\end{defn}

\begin{defn}
  Let $R:A\to (A\to\prop)$ be an equivalence relation. The \define{equivalence class} of $x:A$ is defined to be
  \begin{equation*}
    [x]_R\defeq R(x).
  \end{equation*}
  More generally, a subtype $P:A\to \prop$ is said to be an \define{equivalence class} if it satisfies
  \begin{equation*}
    \mathsf{is\usc{}equivalence\usc{}class}(P)\defeq\exists_{(x:A)}P=R(x).
  \end{equation*}
  Furthermore, we define $A/R$ to be the type of equivalence classes, i.e., we define
  \begin{equation*}
    A/R\defeq \sm{P:A\to\prop}\mathsf{is\usc{}equivalence\usc{}class}(P).
  \end{equation*}
\end{defn}

In other words, $A/R$ is the image of the map $[{-}]_R:A\to (A\to\prop)$. In the following proposition we characterize the identity type of $A/R$. As a corollary, we obtain equivalences
\begin{equation*}
  ([x]_R=[y]_R)\simeq R(x,y),
\end{equation*}
justifying that the quotient $A/R$ is defined to be the type of equivalence classes. Note that in our characterization of the identity type of $A/R$ we make use of the univalence axiom.

\begin{prp}\label{prp:eq-quotient}
  Let $R:A\to (A\to\prop)$ be an equivalence relation. Furthermore, consider $x:A$ and an equivalence class $P$. Then the canonical map
  \begin{equation*}
    ([x]_R=P)\to P(x)
  \end{equation*}
  is an equivalence.
\end{prp}

\begin{proof}
  By \cref{thm:id_fundamental} it suffices to show that the total space
  \begin{equation*}
    \sm{P:A/R}P(x)
  \end{equation*}
  is contractible. The center of contraction is of course $[x]_R$, which satisfies $[x]_R(x)$ by reflexivity of $R$. It remains to construct a contraction. Since $\sm{P:A/R}P(x)$ is a subtype of $A/R$, we construct a contraction by showing that
  \begin{equation*}
    [x]_R=P
  \end{equation*}
  whenever $P(x)$ holds. Recall that $P$ is an equivalence relation, i.e., that there exists a $y:A$ such that $P=[y]_R$. Note that our goal is a proposition, so we may assume that we have such a $y$. Then we obtain that $R(x,y)$ holds from the assumption that $P(x)$ holds. Thus, we have to show that
  \begin{equation*}
    [x]_R=[y]_R
  \end{equation*}
  given that $R(x,y)$ holds. By function extensionality and the univalence axiom, it is equivalent to show that
  \begin{equation*}
    \prd{z:A}R(x,z)\simeq R(y,z)
  \end{equation*}
  We get a function $R(x,z)\to R(y,z)$ by transitivity, since $R(y,x)$ holds by symmetry. Conversely, we get a function $R(y,z)\to R(x,z)$ directly by transitivity. Thus, we obtain that
  \begin{equation*}
    R(x,z)\leftrightarrow R(y,z)
  \end{equation*}
  for any $z:A$, which is sufficient to prove that they are equivalent because $R$ is valued in $\prop$.
\end{proof}

\begin{cor}
  Consider an equivalence relation $R$ on a type $A$, and let $x,y:A$. Then there is an equivalence
  \begin{equation*}
    ([x]_R=[y]_R)\simeq R(x,y).
  \end{equation*}
\end{cor}

\begin{proof}
  By \cref{prp:eq-quotient} we have an equivalence
  \begin{equation*}
    ([x]_R=[y]_R)\simeq R(y,x).
  \end{equation*}
  Moreover, $R(y,x)$ is equivalent to $R(x,y)$ by symmetry of $R$.
\end{proof}

\begin{comment}
The notion of $0$-equivalence relation which we defined in \cref{defn:eq_rel} fits in a hierarchy of `$n$-equivalence relations'\index{n-equivalence relation@{$n$-equivalence relation}}, the study of which is a research topic on its own. However, we already know an example of a relation that should count as an `$\infty$-equivalence relation'\index{infinity-equivalence relation@{$\infty$-equivalence relation}}: the identity type. Analogous to \cref{thm:equivalence_classes}, the following theorem shows that the canonical map
\begin{equation*}
(x=y)\to (\idtypevar{A}(x)=\idtypevar{A}(y))
\end{equation*}
is an equivalence, for any $x,y:A$. In other words, $\idtypevar{A}(x)$ can be thought of as the equivalence class of $x$ with respect to the relation $\idtypevar{A}$.

\begin{thm}
Assuming the univalence axiom on $\UU$, the map
\begin{equation*}
\idtypevar{A}:A\to (A\to\UU)
\end{equation*}
is an embedding, for any type $A:\UU$.\index{identity type!is an embedding}
\end{thm}

\begin{proof}
Let $a:A$. By function extensionality it suffices to show that the canonical map
\begin{equation*}
(a=b)\to \idtypevar{A}(a)\htpy\idtypevar{A}(b)
\end{equation*}
that sends $\refl{a}$ to $\lam{x}\refl{(a=x)}$ is an equivalence for every $b:A$, and by univalence it therefore suffices to show that the canonical map
\begin{equation*}
(a=b)\to \prd{x:A}\eqv{(a=x)}{(b=x)}
\end{equation*}
that sends $\refl{a}$ to $\lam{x}\idfunc[(a=x)]$ is an equivalence for every $b:B$. To do this we employ the type theoretic Yoneda lemma, \cref{thm:yoneda}.

By the type theoretic Yoneda lemma\index{Yoneda lemma} we have an equivalence
\begin{equation*}
\Big(\prd{x:A} (b=x)\to (a=x)\Big)\to (a=b)
\end{equation*}
given by $\lam{f} f(b,\refl{b})$, for every $b:A$. Note that any family of maps $\prd{x:A}(b=x)\to (a=x)$ induces an equivalence of total spaces by \cref{ex:contr_equiv}, since their total spaces are are both contractible by \cref{cor:contr_path}. It follows that we have an equivalence
\begin{equation*}
\varphi_b:\Big(\prd{x:A} \eqv{(b=x)}{(a=x)}\Big)\to (a=b)
\end{equation*}
given by $\lam{f} f(b,\refl{b})$, for every $b:A$. 

Note that $\varphi_a(\lam{x}\idfunc[(a=x)])\jdeq\refl{a}$. Therefore it follows by another application of \cref{thm:yoneda} that the unique family of maps 
\begin{equation*}
\alpha_b:(a=b)\to \Big(\prd{x:A} \eqv{(b=x)}{(a=x)}\Big)
\end{equation*}
that satisfies $\alpha_a(\refl{a})=\lam{x}\idfunc[(a=x)]$ is a family of sections of $\varphi$. 
It follows that $\alpha$ is a family of equivalences. Now the proof is completed by reverting the direction of the family of equivalences in the codomain.
\end{proof}
\end{comment}

\subsection{The universal property of set quotients}

The quotient $A/R$ is constructed as the image of $R$, so we obtain a commuting triangle
\begin{equation*}
  \begin{tikzcd}[column sep=-1em]
    A \arrow[rr,"q_R"] \arrow[dr,swap,"R"] & & A/R \arrow[dl,hook,"i_R"] \\
    \phantom{A/R} & \prop^A,
  \end{tikzcd}
\end{equation*}
and the embedding $i_R:A/R\to\prop^A$ satisfies the universal property of the image of $R$. This universal property is, however, not the usual universal property of the quotient.

\begin{defn}
  Consider a map $q:A\to B$ into a set $B$ satisfying the property that $f(x)=f(y)$ whenever $R(x,y)$ holds. We say that $q$ satisfies the \define{universal property of the set quotient by $R$} if for every map $f:A\to X$ into a set $X$ such that $f(x)=f(y)$ whenever $R(x,y)$ holds, there is a unique extension
  \begin{equation*}
    \begin{tikzcd}
      A \arrow[d,swap,"q"] \arrow[dr,"f"] \\
      B \arrow[r,densely dotted] & X.
    \end{tikzcd}
  \end{equation*}
\end{defn}

\begin{rmk}
  Formally, we express the universal property of the quotient by $R$ as follows. Consider a map $q:A\to B$ that satisfies the property that
  \begin{equation*}
    H:\prd{x,y:A}R(x,y)\to (f(x)=f(y)).
  \end{equation*}
  Then there is for any set $X$ a map
  \begin{equation*}
    q^\ast:(B\to X) \to \Big(\sm{f:A\to X}\prd{x,y:A}R(x,y)\to (f(x)=f(y))\Big).
  \end{equation*}
  This map takes a function $h:B\to X$ to the pair
  \begin{equation*}
    q^\ast(h)\defeq(h\circ q,\lam{x}{y}{r}\ap{h}{H_{x,y}(r)}).
  \end{equation*}
  The universal property of the set quotient of $R$ asserts that the map $q^\ast$ is an equivalence for every set $X$. It is important to note that the universal property of set quotients is formulated with respect to sets.
\end{rmk}

\begin{lem}
Let $R:A\to (A\to \prop)$ be an equivalence relation\index{equivalence relation}, for $A:\UU$, and consider a commuting triangle
\begin{equation*}
\begin{tikzcd}[column sep=tiny]
A \arrow[rr,"q"] \arrow[dr,swap,"R"] & & U \arrow[dl,"m"] \\
& \prop^A
\end{tikzcd}
\end{equation*}
with $H:R\htpy m\circ q$, where $m$ is an embedding. Then we have
\begin{equation*}
\prd{x,y:A}R(x,y)\to (q(x)=q(y)).
\end{equation*}
\end{lem}

\begin{thm}\label{thm:quotient_up}
Let $R:A\to (A\to \prop)$ be an equivalence relation\index{equivalence relation}, for $A:\UU$, and consider a commuting triangle
\begin{equation*}
\begin{tikzcd}[column sep=tiny]
A \arrow[rr,"q"] \arrow[dr,swap,"R"] & & U \arrow[dl,"m"] \\
& \prop^A
\end{tikzcd}
\end{equation*}
with $H:R\htpy m\circ q$, where $m$ is an embedding. Then the following are equivalent:
\begin{enumerate}
\item The embedding $m:U\to \prop^A$ satisfies the universal property of the image of $R$.
\item The map $q:A\to U$ satisfies the universal property of the set quotient $A/R$.
\end{enumerate}
\end{thm}

\begin{proof}
Suppose $m:U\to \prop^A$ satisfies the universal property of the image of $R$. Then it follows by \cref{thm:surjective} that the map $q:A\to U$ is surjective. Our goal is to prove that $U$ satisfies the universal property of the set quotient $A/R$. 
\end{proof}

\begin{rmk}
\cref{thm:quotient_up} suggests that we can define the quotient of an equivalence relation $R$ on a type $A$ as the image of a map. However, the type $\prop^A$ of which the quotient is a subtype is not a small type, even if $A$ is a small type.
Therefore it is not clear that the quotient $A/R$ is essentially small\index{essentially small}, as it should be. Luckily, our construction of the image of a map allows us to show that the image is indeed essentially small, using the fact that $\prop^A$ is locally small\index{locally small}.
\end{rmk}

\subsection{Type theoretic replacement}
\subsubsection{Essentially small types and maps}
It is a trivial observation, but nevertheless of fundamental importance, that by the univalence axiom the identity types of $\UU$ are equivalent to types in $\UU$, because it provides an equivalence $\eqv{(A=B)}{(\eqv{A}{B})}$, and the type $\eqv{A}{B}$ is in $\UU$ for any $A,B:\UU$. Since the identity types of $\UU$ are equivalent to types in $\UU$, we also say that the universe is \emph{locally small}.

\begin{defn}\label{defn:ess_small}
\begin{enumerate}
\item A type $A$ is said to be \define{essentially small}\index{essentially small!type} if there is a type $X:\UU$ and an equivalence $\eqv{A}{X}$. We write\index{ess_small(A)@{$\mathsf{ess\usc{}small}(A)$}}
\begin{equation*}
\mathsf{ess\usc{}small}(A)\defeq\sm{X:\UU}\eqv{A}{X}.
\end{equation*}
\item A map $f:A\to B$ is said to be \define{essentially small}\index{essentially small!map} if for each $b:B$ the fiber $\fib{f}{b}$ is essentially small.
We write\index{ess_small(f)@{$\mathsf{ess\usc{}small}(f)$}}
\begin{equation*}
\mathsf{ess\usc{}small}(f)\defeq\prd{b:B}\mathsf{ess\usc{}small}(\fib{f}{b}).
\end{equation*}
\item A type $A$ is said to be \define{locally small}\index{locally small!type} if for every $x,y:A$ the identity type $x=y$ is essentially small.
We write\index{loc_small(A)@{$\mathsf{loc\usc{}small}(A)$}}
\begin{equation*}
\mathsf{loc\usc{}small}(A)\defeq \prd{x,y:A}\mathsf{ess\usc{}small}(x=y).
\end{equation*}
\end{enumerate}
\end{defn}

\begin{lem}\label{lem:isprop_ess_small}
The type $\mathsf{ess\usc{}small}(A)$ is a proposition for any type $A$.\index{essentially small!is a proposition}
\end{lem}

\begin{proof}
Let $X$ be a type. Our goal is to show that the type
\begin{equation*}
\sm{Y:\UU}\eqv{X}{Y}
\end{equation*}
is a proposition. Suppose there is a type $X':\UU$ and an equivalence $e:\eqv{X}{X'}$, then the map
\begin{equation*}
(\eqv{X}{Y})\to (\eqv{X'}{Y})
\end{equation*}
given by precomposing with $e^{-1}$ is an equivalence. This induces an equivalence on total spaces
\begin{equation*}
\eqv{\Big(\sm{Y:\UU}\eqv{X}{Y}\Big)}{\Big(\sm{Y:\UU}\eqv{X'}{Y}\Big)}
\end{equation*}
However, the codomain of this equivalence is contractible by \cref{thm:univalence}. Thus it follows by \cref{cor:contr_prop} that the asserted type is a proposition.
\end{proof}

\begin{cor}
For each function $f:A\to B$, the type $\mathsf{ess\usc{}small}(f)$ is a proposition, and for each type $X$ the type $\mathsf{loc\usc{}small}(X)$ is a proposition.
\end{cor}

\begin{proof}
This follows from the fact that propositions are closed under dependent products, established in \cref{thm:trunc_pi}.
\end{proof}

\begin{thm}\label{thm:fam_proj}
For any small type $A:\UU$ there is an equivalence
\begin{equation*}
\mathsf{map\usc{}fam}_A:\eqv{(A\to \UU)}{\Big(\sm{X:\UU} X\to A\Big)}.
\end{equation*}
\end{thm}

\begin{proof}
Note that we have the function
\begin{equation*}
\varphi :\lam{B} \Big(\sm{x:A}B(x),\proj 1\Big) : (A\to \UU)\to \Big(\sm{X:\UU}X\to A\Big).
\end{equation*}
The fiber of this map at $(X,f)$ is by univalence and function extensionality equivalent to the type
\begin{equation*}
\sm{B:A\to \UU}{e:\eqv{(\sm{x:A}B(x))}{X}} \proj 1\htpy f\circ e.
\end{equation*}
By \cref{ex:triangle_fib} this type is equivalent to the type
\begin{equation*}
\sm{B:A\to \UU}\prd{a:A} \eqv{B(a)}{\fib{f}{a}},
\end{equation*}
and by `type theoretic choice', which was established in \cref{thm:choice}, this type is equivalent to
\begin{equation*}
\prd{a:A}\sm{X:\UU}\eqv{X}{\fib{f}{a}}.
\end{equation*}
We conclude that the fiber of $\varphi$ at $(X,f)$ is equivalent to the type $\mathsf{ess\usc{}small}(f)$. However, since $f:X\to A$ is a map between small types it is essentially small. Moreover, since being essentially small is a proposition by \cref{lem:isprop_ess_small}, it follows that $\fib{\varphi}{(X,f)}$ is contractible for every $f:X\to A$. In other words, $\varphi$ is a contractible map, and therefore it is an equivalence.
\end{proof}

\begin{rmk}
The inverse of the map
\begin{equation*}
\varphi : (A\to \UU)\to \Big(\sm{X:\UU}X\to A\Big).
\end{equation*}
constructed in \cref{thm:fam_proj} is the map $(X,f)\mapsto \fibf{f}$.
\end{rmk}

\begin{thm}\label{thm:classifier}
Let $f:A\to B$ be a map. Then there is an equivalence
\begin{equation*}
\eqv{\mathsf{ess\usc{}small}(f)}{\mathsf{is\usc{}classified}(f)},
\end{equation*}
where $\mathsf{is\usc{}classified}(f)$\index{is_classified(f)@{$\mathsf{is\usc{}classified}(f)$}} is the type of quadruples $(F,\tilde{F},H,p)$ consisting of maps
$F:B\to \UU$ and $\tilde{F}:A\to \sm{X:\UU}X$, a homotopy $H:F\circ f\htpy \proj 1\circ \tilde{F}$,  such that the commuting square
\begin{equation*}
\begin{tikzcd}
A \arrow[r,"\tilde{F}"] \arrow[d,swap,"f"] & \sm{X:\UU}X \arrow[d,"\proj 1"] \\
B \arrow[r,swap,"F"] & \UU
\end{tikzcd}
\end{equation*}
is a pullback square, as witnessed by $p$\footnote{The universal property of the pullback is not expressible by a type. However, we may take the type of $p:\isequiv(h)$, where $h:A\to B\times_\UU\big(\sm{X:\UU}X\big)$ is the map obtained by the universal property of the canonical pullback.}. If $f$ comes equipped with a term of type $\mathsf{is\usc{}classified}(f)$, we also say that $f$ is \define{classified}\index{classified by the universal family} by the universal family. 
\end{thm}

\begin{proof}
From \cref{ex:sq_fib} we obtain that the type of pairs $(\tilde{F},H)$ is equivalent to the type of fiberwise transformations
\begin{equation*}
\prd{b:B}\fib{f}{b}\to F(b).
\end{equation*}
By \cref{cor:pb_fibequiv} the square is a pullback square if and only if the induced map
\begin{equation*}
\prd{b:B}\fib{f}{b}\to F(b)
\end{equation*}
is a fiberwise equivalence. Thus the data $(F,\tilde{F},H,pb)$ is equivalent to the type of pairs $(F,e)$ where $e$ is a fiberwise equivalence from $\fibf{f}$ to $F$. By \cref{thm:choice} the type of pairs $(F,e)$ is equivalent to the type $\mathsf{ess\usc{}small}(f)$. 
\end{proof}

\begin{rmk}
For any type $A$ (not necessarily small), and any $B:A\to \UU$, the square\index{Sigma-type@{$\Sigma$-type}!as pullback of universal family}
\begin{equation*}
\begin{tikzcd}[column sep=6em]
\sm{x:A}B(x) \arrow[d,swap,"\proj 1"] \arrow[r,"{\lam{(x,y)}(B(x),y)}"] & \sm{X:\UU}X \arrow[d,"\proj 1"] \\
A \arrow[r,swap,"B"] & \UU
\end{tikzcd}
\end{equation*}
is a pullback square. Therefore it follows that for any family $B:A\to\UU$ of small types, the projection map $\proj 1:\sm{x:A}B(x)\to A$ is an essentially small map.
To see that the claim is a direct consequence of \cref{lem:pb_subst} we write the asserted square in its rudimentary form:
\begin{equation*}
%\begin{gathered}[b]
\begin{tikzcd}[column sep=6em]
\sm{x:A}\mathrm{El}(B(x)) \arrow[d,swap,"\proj 1"] \arrow[r,"{\lam{(x,y)}(B(x),y)}"] & \sm{X:\UU}\mathrm{El}(X) \arrow[d,"\proj 1"] \\
A \arrow[r,swap,"B"] & \UU.
\end{tikzcd}%\\[-\dp\strutbox]\end{gathered}\qedhere
\end{equation*}
\end{rmk}

In the following theorem we show that a type is small if and only if its diagonal is classified by $\UU$.

\begin{thm}
Let $A$ be a type. The following are equivalent:
\begin{enumerate}
\item $A$ is locally small.\index{locally small}
\item There are maps $I:A\times A\to\UU$ and $\tilde{I}:A\to\sm{X:\UU}X$, and a homotopy $H:I\circ \delta_A\htpy \proj 1\circ\tilde{I}$
such that the commuting square
\begin{equation*}
\begin{tikzcd}
A \arrow[r,"\tilde{I}"] \arrow[d,swap,"\delta_A"] & \sm{X:\UU}X \arrow[d,"\proj 1"] \\
A\times A \arrow[r,swap,"{I}"] & \UU
\end{tikzcd}
\end{equation*}
is a pullback square.\index{diagonal!of a type}
\end{enumerate}
\end{thm}

\begin{proof}
In \cref{ex:diagonal} we have established that the identity type $x=y$ is the fiber of $\delta_A$ at $(x,y):A\times A$. Therefore it follows that $A$ is locally small if and only if the diagonal $\delta_A$ is essentially small.
Now the result follows from \cref{thm:classifier}.
\end{proof}

\subsubsection{Univalent universes are object classifiers}

\begin{defn}
  Consider a map $p:E\to B$ and a map $f:X\to Y$. The type $\cart(f,p)$ of \define{cartesian morphisms} from $f$ to $p$ is the type of quadruples $(g,h,H,t)$ consisting of maps
  \begin{align*}
    g & : Y\to B \\
    h & : X\to E,
  \end{align*}
  a homotopy $H:g\circ f\htpy p\circ h$, and a term $t$ witnessing that the commuting square
  \begin{equation*}
    \begin{tikzcd}
      X \arrow[r,"h"] \arrow[d,swap,"f"] & E \arrow[d,"p"] \\
      Y \arrow[r,swap,"g"] & B
    \end{tikzcd}
  \end{equation*}
  is a pullback square.
\end{defn}

\begin{defn}
  A map $p:E\to B$ is called an \define{object classifier} if for every map $f:X\to Y$, the type $\cart(f,p)$ of cartesian morphisms
  \begin{equation*}
    \begin{tikzcd}
      X \arrow[r] \arrow[d,swap,"f"] & E \arrow[d,"p"] \\
      Y \arrow[r] & B
    \end{tikzcd}
  \end{equation*}
  from $f$ to $p$ is a proposition. 
\end{defn}

\begin{prp}
  Consider a map $p:E\to B$. The following are equivalent:
  \begin{enumerate}
  \item The map $p$ is an object classifier.
  \item The function
    \begin{equation*}
      \tr_{\fibf{p}}:(x=y)\to (\fib{p}{x}\simeq\fib{p}{y})
    \end{equation*}
    is an equivalence.
  \end{enumerate}
\end{prp}

\begin{cor}
  A universe is an object classifier if and only if it is univalent.
\end{cor}

\subsubsection{Smallness of images}
However, the construction of the fiberwise join in \cref{ex:fib_join} suggests that we can also define the image of $f$ as the infinite join power $f^{\ast\infty}$, where we repeatedly take the fiberwise join of $f$ with itself. The reasons for defining the image in this way are twofold: we will be able to use this construction to show that the set-quotients of a small type are small, and second, we many interesting types appear in this construction.

\begin{lem}
Consider a map $f:A\to X$, an embedding $m:U\to X$, and $h:\mathrm{hom}_X(f,m)$. Then the map
\begin{equation*}
\mathrm{hom}_X(\join{f}{g},m)\to \mathrm{hom}_X(g,m)
\end{equation*}
is an equivalence for any $g:B\to X$.
\end{lem}

\begin{proof}
Note that both types are propositions, so any equivalence can be used to prove the claim. Thus, we simply calculate
\begin{align*}
\mathrm{hom}_X(\join{f}{g},m) & \eqvsym \prd{x:X}\fib{\join{f}{g}}{x}\to \fib{m}{x} \\
& \eqvsym \prd{x:X}\join{\fib{f}{x}}{\fib{g}{x}}\to\fib{m}{x} \\
& \eqvsym \prd{x:X}\fib{g}{x}\to\fib{m}{x} \\
& \eqvsym \mathrm{hom}_X(g,m).
\end{align*}
The first equivalence holds by \cref{ex:triangle_fib}; the second equivalence holds by \cref{ex:fib_join}, also using \cref{ex:equiv_precomp,lem:postcomp_equiv} where we established that that pre- and postcomposing by an equivalence is an equivalence; the third equivalence holds by \cref{lem:extend_join_prop,lem:postcomp_equiv}; the last equivalence again holds by \cref{ex:triangle_fib}.
\end{proof}

For the construction of the image of $f:A\to X$ we observe that if we are given an embedding $m:U\to X$ and a map $(i,I):\mathrm{hom}_X(f,m)$, then $(i,I)$ extends uniquely along $\inr:A\to \join[X]{A}{A}$ to a map $\mathrm{hom}_X(\join{f}{f},m)$. This extension again extends uniquely along $\inr:\join[X]{A}{A}\to \join[X]{A}{(\join[X]{A}{A})}$ to a map $\mathrm{hom}_X(\join{f}{(\join{f}{f})},m)$ and so on, resulting in a diagram of the form
\begin{equation*}
\begin{tikzcd}
A \arrow[dr] \arrow[r,"\inr"] & \join[X]{A}{A} \arrow[d,densely dotted] \arrow[r,"\inr"] & \join[X]{A}{(\join[X]{A}{A})} \arrow[dl,densely dotted] \arrow[r,"\inr"] & \cdots \arrow[dll,densely dotted,bend left=10] \\
& U
\end{tikzcd}
\end{equation*}

\begin{defn}
Suppose $f:A\to X$ is a map. Then we define the \define{fiberwise join powers} 
\begin{equation*}
f^{\ast n}:A_X^{\ast n} X.
\end{equation*}
\end{defn}

\begin{constr}
Note that the operation $(B,g)\mapsto (\join[X]{A}{B},\join{f}{g})$ defines an endomorphism on the type
\begin{equation*}
\sm{B:\UU}B\to X.
\end{equation*}
We also have $(\emptyt,\ind{\emptyt})$ and $(A,f)$ of this type. For $n\geq 1$ we define
\begin{align*}
A_X^{\ast (n+1)} & \defeq \join[X]{A}{A_X^{\ast n}} \\
f^{\ast (n+1)} & \defeq \join{f}{f^{\ast n}}.\qedhere
\end{align*}
\end{constr}

\begin{defn}
We define $A_X^{\ast\infty}$ to be the sequential colimit of the type sequence
\begin{equation*}
\begin{tikzcd}
A_X^{\ast 0} \arrow[r] & A_X^{\ast 1} \arrow[r,"\inr"] & A_X^{\ast 2} \arrow[r,"\inr"] & \cdots.
\end{tikzcd}
\end{equation*}
Since we have a cocone
\begin{equation*}
\begin{tikzcd}
A_X^{\ast 0} \arrow[r] \arrow[dr,swap,"f^{\ast 0}" near start] & A_X^{\ast 1} \arrow[r,"\inr"] \arrow[d,swap,"f^{\ast 1}" near start] & A_X^{\ast 2} \arrow[r,"\inr"] \arrow[dl,swap,"f^{\ast 2}" xshift=1ex] & \cdots \arrow[dll,bend left=10] \\
& X
\end{tikzcd}
\end{equation*}
we also obtain a map $f^{\ast\infty}:A_X^{\ast\infty}\to X$ by the universal property of $A_X^{\ast\infty}$. 
\end{defn}

\begin{lem}\label{lem:finfjp_up}
Let $f:A\to X$ be a map, and let $m:U\to X$ be an embedding. Then the function
\begin{equation*}
\blank\circ \seqin_0: \mathrm{hom}_X(f^{\ast\infty},m)\to \mathrm{hom}_X(f,m)
\end{equation*}
is an equivalence. 
\end{lem}

\begin{thm}\label{lem:isprop_infjp}
For any map $f:A\to X$, the map $f^{\ast\infty}:A_X^{\ast\infty}\to X$ is an embedding that satisfies the universal property of the image inclusion of $f$.
\end{thm}

\begin{lem}
Consider a commuting square
\begin{equation*}
\begin{tikzcd}
A \arrow[r] \arrow[d] & B \arrow[d] \\
C \arrow[r] & D.
\end{tikzcd}
\end{equation*}
\begin{enumerate}
\item If the square is cartesian, $B$ and $C$ are essentially small, and $D$ is locally small, then $A$ is essentially small.
\item If the square is cocartesian, and $A$, $B$, and $C$ are essentially small, then $D$ is essentially small. 
\end{enumerate}
\end{lem}

\begin{cor}
Suppose $f:A\to X$ and $g:B\to X$ are maps from essentially small types $A$ and $B$, respectively, to a locally small type $X$. Then $A\times_X B$ is again essentially small. 
\end{cor}

\begin{lem}
Consider a type sequence
\begin{equation*}
\begin{tikzcd}
A_0 \arrow[r,"f_0"] & A_1 \arrow[r,"f_1"] & A_2 \arrow[r,"f_2"] & \cdots
\end{tikzcd}
\end{equation*}
where each $A_n$ is essentially small. Then its sequential colimit is again essentially small. 
\end{lem}

\begin{thm}
For any map $f:A\to X$ from a small type $A$ into a locally small type $X$, the image $\im(f)$ is an essentially small type.
\end{thm}

Recall that in set theory, the replacement axiom asserts that for any family of sets $\{X_i\}_{i\in I}$ indexed by a set $I$, there is a set $X[I]$ consisting of precisely those sets $x$ for which there exists an $i\in I$ such that $x\in X_i$. In other words: the image of a set-indexed family of sets is again a set. Without the replacement axiom, $X[I]$ would be a class. In the following corollary we establish a type-theoretic analogue of the replacement axiom: the image of a family of small types indexed by a small type is again (essentially) small.

\begin{thm}\label{thm:replacement}
  For any map $f:A\to B$ from an essentially small type $A$ into a locally small type $B$, the image of $f$ is again essentially small.
\end{thm}

\subsection{Connected components of types}

\subsection{Set truncation}

\begin{lem}
For each type $A$, the relation $I_{(-1)}:A\to (A\to\prop)$ given by
\begin{equation*}
I_{(-1)}(x,y)\defeq\brck{x=y}
\end{equation*}
is an equivalence relation.
\end{lem}

\begin{proof}
For every $x:A$ we have $\bproj{\refl{x}}:\brck{x=x}$, so the relation is reflexive. To see that the relation is symmetric note that by the universal property of propositional truncation there is a unique map $\brck{\invfunc}:\brck{x=y}\to\brck{y=x}$ for which the square
\begin{equation*}
\begin{tikzcd}
(x=y) \arrow[r,"\invfunc"] \arrow[d,swap,"\bproj{\blank}"] & (y=x) \arrow[d,"\bproj{\blank}"] \\
\brck{x=y} \arrow[r,densely dotted,swap,"\brck{\invfunc}"] & \brck{y=x}
\end{tikzcd}
\end{equation*}
commutes. This shows that the relation is symmetric. Similarly, we show by the universal property of propositional truncation that the relation is transitive.
\end{proof}

\begin{defn}
For each type $A$ we define the \define{set truncation}
\begin{equation*}
\trunc{0}{A}\defeq A/I_{(-1)},
\end{equation*}
and the unit of the set truncation is defined to be the quotient map.
\end{defn}

\begin{thm}
For each type $A$, the set truncation satisfies the universal property of the set truncation.
\end{thm}

\begin{exercises}
\exercise
\begin{subexenum}
\item Show that any proposition is locally small.\index{proposition!is locally small}
\item Show that any essentially small type is locally small.\index{essentially small!type!is locally small}
\item Show that the function type $A\to X$ is locally small whenever $A$ is essentially small and $X$ is locally small.
\end{subexenum}
\exercise Let $f:A\to B$ be a map. Show that the following are equivalent:
\begin{enumerate}
\item The map $f$ is \define{locally small}\index{locally small!map} in the sense that for every $x,y:A$, the action on paths of $f$
\begin{equation*}
\apfunc{f}:(x=y)\to (f(x)=f(y))
\end{equation*}
is an essentially small map.
\item The diagonal $\delta_f$ of $f$ as defined in \cref{ex:trunc_diagonal_map} is classified by the universal fibration.
\end{enumerate}
\exercise \label{ex:span_rel}Use \cref{thm:choice,thm:fam_proj,cor:times_up_out} to show that the type 
\begin{equation*}
\mathsf{span}(A,B)\defeq \sm{S:\UU} (S\to A)\times (S\to B)
\end{equation*}
of small spans from $A$ to $B$ is equivalent to the type $A\to (B\to\UU)$ of small relations from $A$ to $B$.
\end{exercises}

\section{The classifying type of a group}
\chapter{Set quotients}

\section{The universal property of set quotients}

\begin{defn}
Let $R:A\to (A\to \prop)$ be an equivalence relation\index{equivalence relation|textit}, for $A:\UU$, and consider a map $q:A\to B$ where the type $B$ is a set, for which we have
\begin{equation*}
\prd{x,y:A}R(x,y)\to q(x)=q(y).
\end{equation*}
We will define a map
\begin{equation*}
\quotientrestr:(B\to X) \to \Big(\sm{f:A\to X}\prd{x,y:A}R(x,y)\to (f(x)=f(y))\Big).
\end{equation*}
\end{defn}

\begin{constr}
Let $h:B\to X$. Then we have $h\circ q : A\to X$, so it remains to show that
\begin{equation*}
\prd{x,y:A}R(x,y)\to (h(q(x))=h(q(y)))
\end{equation*}
Consider $x,y:A$ which are related by $R$. Then we have an identification $p:q(x)=q(y)$, so it follows that $\ap{h}{p}:h(q(x))=h(q(y))$.  
\end{constr}

\begin{defn}
Let $R:A\to (A\to \prop)$ be an equivalence relation\index{equivalence relation|textit}, for $A:\UU$, and consider a map $q:A\to B$ satisfying
\begin{equation*}
\prd{x,y:A}R(x,y)\to q(x)=q(y),
\end{equation*}
where the type $B$ is a set. We say that the map $q:A\to B$ satisfies the universal property of the \define{set quotient}\index{set quotient}\index{universal property!of set quotients|textit} $A/R$ if for any set $X$ the map
\begin{equation*}
\quotientrestr : (B\to X) \to \Big(\sm{f:A\to X}\prd{x,y:A}R(x,y)\to (f(x)=f(y))\Big)
\end{equation*}
is an equivalence.
\end{defn}

\begin{lem}
Let $R:A\to (A\to \prop)$ be an equivalence relation\index{equivalence relation|textit}, for $A:\UU$, and consider a commuting triangle
\begin{equation*}
\begin{tikzcd}[column sep=tiny]
A \arrow[rr,"q"] \arrow[dr,swap,"R"] & & U \arrow[dl,"m"] \\
& \prop^A
\end{tikzcd}
\end{equation*}
with $H:R\htpy m\circ q$, where $m$ is an embedding. Then we have
\begin{equation*}
\prd{x,y:A}R(x,y)\to (q(x)=q(y)).
\end{equation*}
\end{lem}

\begin{thm}\label{thm:quotient_up}
Let $R:A\to (A\to \prop)$ be an equivalence relation\index{equivalence relation|textit}, for $A:\UU$, and consider a commuting triangle
\begin{equation*}
\begin{tikzcd}[column sep=tiny]
A \arrow[rr,"q"] \arrow[dr,swap,"R"] & & U \arrow[dl,"m"] \\
& \prop^A
\end{tikzcd}
\end{equation*}
with $H:R\htpy m\circ q$, where $m$ is an embedding. Then the following are equivalent:
\begin{enumerate}
\item The embedding $m:U\to \prop^A$ satisfies the universal property of the image of $R$.
\item The map $q:A\to U$ satisfies the universal property of the set quotient $A/R$.
\end{enumerate}
\end{thm}

\begin{proof}
Suppose $m:U\to \prop^A$ satisfies the universal property of the image of $R$. Then it follows by \cref{thm:surjective} that the map $q:A\to U$ is surjective. Our goal is to prove that $U$ satisfies the universal property of the set quotient $A/R$. 
\end{proof}

\begin{rmk}
\cref{thm:quotient_up} suggests that we can define the quotient of an equivalence relation $R$ on a type $A$ as the image of a map. However, the type $\prop^A$ of which the quotient is a subtype is not a small type, even if $A$ is a small type.
Therefore it is not clear that the quotient $A/R$ is essentially small\index{essentially small}, as it should be. Luckily, our construction of the image of a map allows us to show that the image is indeed essentially small, using the fact that $\prop^A$ is locally small\index{locally small}.
\end{rmk}

\section{The construction of set quotients}
\begin{lem}
Consider a commuting square
\begin{equation*}
\begin{tikzcd}
A \arrow[r] \arrow[d] & B \arrow[d] \\
C \arrow[r] & D.
\end{tikzcd}
\end{equation*}
\begin{enumerate}
\item If the square is cartesian, $B$ and $C$ are essentially small, and $D$ is locally small, then $A$ is essentially small.
\item If the square is cocartesian, and $A$, $B$, and $C$ are essentially small, then $D$ is essentially small. 
\end{enumerate}
\end{lem}

\begin{cor}
Suppose $f:A\to X$ and $g:B\to X$ are maps from essentially small types $A$ and $B$, respectively, to a locally small type $X$. Then $A\times_X B$ is again essentially small. 
\end{cor}

\begin{lem}
Consider a type sequence
\begin{equation*}
\begin{tikzcd}
A_0 \arrow[r,"f_0"] & A_1 \arrow[r,"f_1"] & A_2 \arrow[r,"f_2"] & \cdots
\end{tikzcd}
\end{equation*}
where each $A_n$ is essentially small. Then its sequential colimit is again essentially small. 
\end{lem}

\begin{thm}
For any map $f:A\to X$ from a small type $A$ into a locally small type $X$, the image $\im(f)$ is an essentially small type.
\end{thm}

Recall that in set theory, the replacement axiom asserts that for any family of sets $\{X_i\}_{i\in I}$ indexed by a set $I$, there is a set $X[I]$ consisting of precisely those sets $x$ for which there exists an $i\in I$ such that $x\in X_i$. In other words: the image of a set-indexed family of sets is again a set. Without the replacement axiom, $X[I]$ would be a class. In the following corollary we establish a type-theoretic analogue of the replacement axiom: the image of a family of small types indexed by a small type is again (essentially) small.

\begin{cor}\label{cor:im_small}
For any small type family $B:A\to\UU$, where $A$ is small, the image $\im(B)$ is essentially small. We call $\im(B)$ the \define{univalent completion} of $B$. 
\end{cor}

\section{Connected components of types}

\section{Set truncation}

\begin{lem}
For each type $A$, the relation $I_{(-1)}:A\to (A\to\prop)$ given by
\begin{equation*}
I_{(-1)}(x,y)\defeq\brck{x=y}
\end{equation*}
is an equivalence relation.
\end{lem}

\begin{proof}
For every $x:A$ we have $\bproj{\refl{x}}:\brck{x=x}$, so the relation is reflexive. To see that the relation is symmetric note that by the universal property of propositional truncation there is a unique map $\brck{\invfunc}:\brck{x=y}\to\brck{y=x}$ for which the square
\begin{equation*}
\begin{tikzcd}
(x=y) \arrow[r,"\invfunc"] \arrow[d,swap,"\bproj{\blank}"] & (y=x) \arrow[d,"\bproj{\blank}"] \\
\brck{x=y} \arrow[r,densely dotted,swap,"\brck{\invfunc}"] & \brck{y=x}
\end{tikzcd}
\end{equation*}
commutes. This shows that the relation is symmetric. Similarly, we show by the universal property of propositional truncation that the relation is transitive.
\end{proof}

\begin{defn}
For each type $A$ we define the \define{set truncation}
\begin{equation*}
\trunc{0}{A}\defeq A/I_{(-1)},
\end{equation*}
and the unit of the set truncation is defined to be the quotient map.
\end{defn}

\begin{thm}
For each type $A$, the set truncation satisfies the universal property of the set truncation.
\end{thm}

\begin{comment}
\section{The universal property of the truncations}

\begin{defn}\label{defn:is_truncation}
Let $X$ be a type. A map $f:X\to Y$ into an $k$-type $Y$ is said to satisfy the \define{universal property of $k$-truncation} if the precomposition map
\begin{equation*}
\blank\circ f: (Y\to Z)\to (X\to Z)
\end{equation*}
is an equivalence for every $k$-type $Z$.
\end{defn}

\begin{rmk}
A map $f:X\to Y$ into an $k$-type $Y$ satisfies the universal property of $k$-truncation if of for every $g:X\to Z$ the type of extensions
\begin{equation*}
\begin{tikzcd}
X \arrow[dr,"g"] \arrow[d,swap,"f"] \\
Y \arrow[r,densely dotted] & Z
\end{tikzcd}
\end{equation*}
is contractible. Indeed, the type of such extensions is the type
\begin{equation*}
\sm{h:Y\to Z} h\circ f\htpy g,
\end{equation*}
which is equivalent to the fiber of the precomposition map $\blank\circ f$ at $g$. 
\end{rmk}

\begin{thm}\label{thm:trunc_dup}
Suppose the map $f:X\to Y$ into an $k$-type $Y$. The following are equivalent:
\begin{enumerate}
\item The map $f$ satisfies the universal property of $k$-truncation.
\item For any type family $P$ of $k$-types over $Y$, the precomposition map
\begin{equation*}
\blank\circ f : \Big(\prd{y:Y}P(y)\Big)\to \Big(\prd{x:X}P(f(x))\Big)
\end{equation*}
is an equivalence. This property is also called the \define{dependent universal property} of the $k$-truncation.
\end{enumerate}
\end{thm}

\begin{proof}
The direction from (ii) to (i) is immediate, so we only have to show that (i) implies (ii).

Suppose $P$ is a family of $k$-truncated types over $Y$.  
Then we have a commuting square
\begin{equation*}
\begin{tikzcd}
\Big(Y\to\sm{y:Y}P(y)\Big) \arrow[r,"\blank\circ f"] \arrow[d,swap,"\proj 1 \circ\blank"] & \Big(X\to \sm{y:Y}P(y)\Big) \arrow[d,"\proj 1\circ \blank"] \\
\Big(Y\to Y\Big) \arrow[r,swap,"\blank\circ f"] & \Big(X\to Y)
\end{tikzcd}
\end{equation*}
Since the total space $\sm{y:Y}P(y)$ is again $k$-truncated by \cref{ex:istrunc_sigma}, it follows by the universal property of the $k$-truncation that the top map is an equivalence, and by the universal property the bottom map is an equivalence too. It follows from \cref{cor:pb_equiv} that this square is a pullback square, so it induces equivalences on the fibers by \cref{cor:pb_fibequiv}. In particular we have a commuting square
\begin{equation*}
\begin{tikzcd}
\Big(\prd{y:Y}P(y)\Big) \arrow[r] \arrow[d] & \Big(\prd{x:X}P(f(x))\Big) \arrow[d] \\
\fib{(\proj 1\circ \blank)}{\idfunc[Y]} \arrow[r] & \fib{(\proj 1 \circ \blank)}{f}
\end{tikzcd}
\end{equation*}
in which the left and right maps are equivalences by \cref{ex:pi_sec}, and the bottom map is an equivalence as we have just established. Therefore the top map is an equivalence, so we conclude that $f$ satisfies the dependent universal property.
\end{proof}

%\begin{thm}
%Suppose the map $f:X\to Y$ into an $k$-type $Y$ is an $k$-truncation, and let $Q$ be a type family of $(k+l)$-types over $Y$.
%Then the precomposition map
%\begin{equation*}
%\blank\circ f : \Big(\prd{y:Y}Q(y)\Big)\to \Big(\prd{x:X}Q(f(x))\Big)
%\end{equation*}
%is $(l-2)$-truncated, for $l:\N$. 
%\end{thm}

\begin{thm}\label{thm:trunc_id}
For any $x,y:X$, there is an equivalence
\begin{equation*}
\eqv{(\tproj{k+1}{x}=\tproj{k+1}{y})}{\trunc{k}{x=y}}.
\end{equation*}
\end{thm}

\begin{proof}
Let $x:X$. Then we define a family $E_x : \trunc{k+1}{X}\to \UU^{\leq k}$ as the unique extension
\begin{equation*}
\begin{tikzcd}[column sep=huge]
X \arrow[r] \arrow[d,swap,"\tproj{k}{\blank}"] \arrow[r,"{y\mapsto \trunc{k}{x=y}}"] & \UU^{\leq k} \\
\trunc{k}{X} \arrow[ur,densely dotted,swap,"E_x"] 
\end{tikzcd}
\end{equation*}
This unique extension exists by the universal property of $(k+1)$-truncation, because the universe $\UU^{\leq k}$ is itself a $(k+1)$-truncated type by \cref{ex:istrunc_UUtrunc}. 

To see that there is an equivalence 
\begin{equation*}
\eqv{(\tproj{k}{x}=y)}{E_x(y)}
\end{equation*}
for each $y:\trunc{k+1}{X}$, it suffices by \cref{thm:id_fundamental} to show that the total space 
\begin{equation*}
\sm{y:\trunc{k+1}{X}}E_x(y)
\end{equation*}
is contractible. At the center of contraction we have $(\tproj{k+1}{x},\tproj{k}{\refl{x}})$. It remains to construct the contraction
\begin{equation*}
\prd{y:\trunc{k+1}{X}}{p:E_x(y)} (\tproj{k+1}{x},\tproj{k}{\refl{x}})=(y,p).
\end{equation*}
We note that the type $(\tproj{k+1}{x},\tproj{k}{\refl{x}})=(y,p)$ is $k$-truncated, since it is an identity type in the total space
\begin{equation*}
\sm{y:\trunc{k+1}{X}}E_x(y),
\end{equation*}
which is $(k+1)$-truncated by \cref{ex:istrunc_sigma,thm:istrunc_next}. Therefore it suffices by \cref{thm:trunc_dup}, applied twice, to construct a term of type
\begin{equation*}
\prd{y:X}{p:x=y} (\tproj{k+1}{x},\tproj{k}{\refl{x}})=(\tproj{k+1}{y},\tproj{k}{p}).
\end{equation*}
We get such an identification for each $p:x=y$ by path induction on $p$.
\end{proof}
\end{comment}

\begin{comment}
\section{Connected maps}

\section{The join extension and connectivity theorems}

\begin{defn}\label{defn:local}
For a given type $M$, a type $A$ is said to be \define{$M$-null} if the map
\begin{equation*}
\lam{a}{m}a : A \to (M \to A)  
\end{equation*}
is an equivalence.
\end{defn}

In other words, the type $A$ is $M$-null if each $f:M\to A$ has a unique extension along the
map $M\to\unit$, as indicated in the diagram
\begin{equation*}
\begin{tikzcd}
M \arrow[r,"f"] \arrow[d] & A \\
\unit. \arrow[ur,densely dotted]
\end{tikzcd}
\end{equation*}
Note that being $M$-null in the above sense is a proposition, so that the
type of all $M$-null types in $\UU$ is a subuniverse of $\UU$. 

\begin{eg}
By \cref{ex:sphere_null}, a type is $\sphere{n+1}$-null precisely when it is $n$-truncated,
for each $n\geq -2$ (taking the $(-1)$-sphere to be the empty type).
\end{eg}

The notion of $M$-connected type is in a sense dual to the notion of $M$-null types.

\begin{defn}
A type $A$ is said to be \define{$M$-connected} if every $M$-null
type is $A$-null. That is, if for every $M$-null type $B$, the map
\begin{equation*}
\lam{b}{a}{b} : B \to (A \to B)
\end{equation*}
is an equivalence. A map is said to be \define{$M$-connected} if its fibers are $M$-connected.
\end{defn}

Thus in particular, $M$ itself is $M$-connected, and the unit type $\unit$ is $M$-connected for every $M$. 

\begin{defn}
Let $M$ be a type. We say that a type $X$ has the \define{$M$-extension property}
with respect to a map $F:A\to B$, if the map
\begin{equation*}
\lam{g}{a} g(F(a)) : (B\to X)\to (A\to X)
\end{equation*}
is $M$-null. In the case $M\jdeq\unit$, we say that $X$ is \define{$F$-local}.
\end{defn}

\begin{lem}\label{lem:equivalent-extension-problems}
For any three types $A$, $A'$ and $B$, the type $B$ is $(\join{A}{A'})$-null
if and only if for any any $f:A\to B$, the type
\begin{equation*}
\sm{b:B}\prd{a:A}f(a)=b
\end{equation*}
is $A'$-null.
\end{lem}

\begin{proof}
To give $f:A\to B$ and $(f',H):A'\to\sm{b:B}\prd{a:A}f(a)=b$ is equivalent to giving a map $g:\join{A}{A'}\to B$. Concretely, the equivalence is given by substituting in $g:\join{A}{A'}\to B$ the constructors of the join, to obtain $\pairr{g\circ\inl,g\circ\inr,\apfunc{g}\circ\glue}$. 

Now observe that the fiber of precomposing with the unique map $!_{\join{A}{A'}} : \join{A}{A'}\to\unit$ at $g : \join{A}{A'}\to B$, is equivalent to
\begin{equation*}
\sm{b:B}\prd{t:\join{A}{A'}}g(t)=b.
\end{equation*}
Similarly, the fiber of precomposing with the unique map $!_{A'} : A'\to\unit$ at $\pairr{g\circ\inr,\apfunc{g}\circ\glue} : A'\to\sm{b:B}\prd{a:A}f(a)=b$ is equivalent to
\begin{equation*}
\sm{b:B}{h:\prd{a:A}g(\inl(a))=b}\prd{a':A'}\pairr{g(\inr(a')),\apfunc{g}(\glue(a,a'))}=\pairr{b,h}.
\end{equation*}
By the universal property of the join, these types are equivalent.
\end{proof}

\begin{lem}\label{lem:join-null}
Suppose $A$ is an $M$-connected type, and that $B$ is an $(\join{M}{N})$-null type. Then $B$ is $(\join{A}{N})$-null.
\end{lem}

\begin{proof}
Let $B$ be a $(\join{M}{N})$-null type. Our goal of showing that $B$ is
$(\join{A}{N})$-null is equivalent to showing that for any $f:N\to B$, 
the type 
\begin{equation*}
\sm{b:B}\prd{a:A}f(a)=b
\end{equation*}
is $A$-null. 
Since $B$ is assumed to be $(\join{M}{N})$-null, we know that this type is 
$M$-null. Since $A$ is $M$-connected, this type is also $A$-null.
\end{proof}

\begin{lem}\label{lem:N-extension-simple}
Let $A$ be $M$-connected and let $B$ be $(\join{M}{N})$-null. Then the map
\begin{equation*}
\lam{b}{a}b:B\to B^A
\end{equation*}
is $N$-null. 
\end{lem}

\begin{proof}
The fiber of $\lam{b}{a}b$ at a function $f:A\to B$ is equivalent to the type $\sm{b:B}\prd{a:A}f(a)=b$. Therefore, it suffices to show that this type is $N$-null. By \autoref{lem:equivalent-extension-problems}, it is equivalent to show that $B$ is $(\join{A}{N})$-null. This is solved in \autoref{lem:join-null}.
\end{proof}

\begin{thm}[Join extension theorem]\label{thm:join-extension}
Suppose $f:X\to Y$ is $M$-connected, and let $P:Y\to\UU$ be a family of
$(\join{M}{N})$-null types for some type $N$. Then precomposition by $f$, i.e.
\begin{equation*}
\lam{s}s\circ f : \Big(\prd{y:Y}P(y)\Big)\to\Big(\prd{x:X}P(f(x))\Big),
\end{equation*}
is an $N$-null map.
\end{thm}

\begin{proof}
Let $g:\prd{x:X}P(f(x))$. Then we have the equivalences
\begin{align*}
\hfib{(\blank\circ f)}{g} 
& \eqvsym \sm{s:\prd{y:Y}P(y)}\prd{x:X}s(f(x))=g(x) \\
& \eqvsym \sm{s:\prd{y:Y}P(y)}\prd{y:Y}{(x,p):\hfib{f}{y}} s(y)= \trans{p}{g(x)} \\
& \eqvsym \prd{y:Y}\sm{z:P(y)}\prd{(x,p):\hfib{f}{y}} \trans{p}{g(x)}=z \\
& \eqvsym \prd{y:Y}\hfib{\lam{z}{(x,p)}z}{\lam{(x,p)}\trans{p}{g(x)}}.
\end{align*}
Therefore, it suffices to show for every $y:Y$, that $P(y)$ has the $N$-extension property with respect to the unique map of type $\hfib{f}{y}\to\unit$. This is a special case of \autoref{lem:N-extension-simple}.
\end{proof}

\begin{thm}\label{thm:simple-join}
Suppose $X$ is an $M$-connected type and $Y$ is an $N$-connected type. Then $\join{X}{Y}$ is an $(\join{M}{N})$-connected type.
\end{thm}

\begin{proof}
It suffices to show that any $(\join{M}{N})$-null type is $(\join{X}{Y})$-null.
Let $Z$ be an $(\join{M}{N})$-null type.
Since $Z$ is assumed to be $(\join{M}{N})$-null, it follows by \autoref{lem:join-null} that $Z$ is $(\join{X}{N})$-null. By symmetry of the join, it also follows that $Z$ is $(\join{X}{Y})$-null.
\end{proof}

\begin{thm}[Join connectivity theorem]\label{thm:join-connectivity}
Consider an $M$-connected map $f:A\to X$ and an $N$-connected map $g:B\to X$. Then $\join{f}{g}$ is $(\join{M}{N})$-connected.
\end{thm}

\begin{proof}
This follows from \autoref{thm:simple-join} and \autoref{defn:join-fiber}.
\end{proof}

\begin{thm}\label{thm:joinconstruction-connectivity}
Consider the factorization
\begin{equation*}
\begin{tikzcd}
A_n \arrow[dr,swap,"f^{\ast n}"] \arrow[r,"q_n"] & \im(f) \arrow[d] \\
& X
\end{tikzcd}
\end{equation*}
of $f^{\ast n}$ through the image $\im(f)$. 
Then the map $q_n$ is $(n-2)$-connected, for each $n:\N$.
\end{thm}

\begin{proof}
We first show the assertion that, given a commuting diagram of the form
\begin{equation*}
\begin{tikzcd}
A \arrow[r,"q"] \arrow[dr,swap,"f"] & Y \arrow[d,"m"] & A' \arrow[l,swap,"{q'}"] \arrow[dl,"{f'}"] \\
& X
\end{tikzcd}
\end{equation*}
in which $m$ is an embedding, then $\join{f}{f'}=\join{(m\circ q)}{(m\circ q')}=m\circ (\join{q}{q'})$.
In other words, postcomposition with embeddings distributes over 
the join operation.

Note that, since $m$ is assumed to be an embedding, we have an equivalence of
type $\eqv{f(a)=f'(a)}{q(a)=q'(a)}$, for every $a:A$. Hence the pullback of
$f$ and $f'$ is equivalent to the pullback of $q$ along $q'$. Consequently, the
two pushouts
\begin{equation*}
\begin{tikzcd}
A\times_X A' \arrow[r,"\pi_2"] \arrow[d,swap,"\pi_1"] & A' \arrow[d] \\
A \arrow[r] & \join[X]{A}{A'}
\end{tikzcd}
\qquad\text{and}\qquad
\begin{tikzcd}
A\times_Y A' \arrow[r,"\pi_2"] \arrow[d,swap,"\pi_1"] & A' \arrow[d] \\
A \arrow[r] & \join[Y]{A}{A'}
\end{tikzcd}
\end{equation*}
are equivalent. Hence the claim follows.

As a corollary, we get that $q_n=q_f^{\ast n}$. Note that $q_f$ is surjective,
in the sense that $q_f$ is $\bool$-connected, where $\bool$ is the type of booleans%
\footnote{Recall that the $\bool$-null types are precisely the mere propositions.}.
Hence it follows that $q_n$ is $\bool^{\ast n}$-connected. 

Now recall that the $n$-th join power of $\bool$ is the $(n-1)$-sphere $\Sn^{n-1}$, and that
a type is $(\Sn^{n-1})$-connected if and only if it is $(n-2)$-connected.
\end{proof}
\end{comment}

\begin{comment}
\section{The construction of the $n$-truncation}\label{sec:truncation}

Our goal in this section is to prove the following theorem. Its proof will take up the entire section.

\begin{thm}\label{thm:truncation}
For every $k\geq -2$, there is a $k$-truncation operation
\begin{equation*}
\trunc{k}{\blank} : \UU\to\UU
\end{equation*}
equipped with a fiberwise transformation
\begin{equation*}
\tproj{k}{\blank}:\prd{X:\UU}X\to\trunc{k}{X},
\end{equation*}
such that for each $X:\UU$ the type $\trunc{k}{X}$ is a $k$-type satisfying the (dependent) universal property of $k$-truncation.
\end{thm}

We will define the $k$-truncation operation by induction on $k\geq-2$,
with the trivial operation as the base case. For $k\geq -2$, suppose we have
a $k$-truncation operation as described in the statement of the theorem.

\cref{thm:trunc_id} suggests that we can think of the type $\trunc{k+1}{X}$ is as the quotient of $X$ modulo the
`$(k+1)$-equivalence relation' given by $\trunc{k}{a=b}$. 

\begin{defn}
We define the reflexive relation $I_k(A) : A \to (A \to \UU)$ by
\begin{equation*}
I_k(A)(a,b) \defeq \trunc{k}{a=b},
\end{equation*}
and then we define
\begin{align*}
\trunc{k+1}{A} & \defeq \im(I_k(A)) \\
\tproj{k+1}{\blank} & \defeq q_{I_k(A)},
\end{align*}
where $q_{I_k(A)}:A\to \im(I_k(A))$ is the map with which the image comes equipped.
\end{defn}

Note that the codomain $(A\to\UU)$ of $I_k(A)$ is locally small since it is the exponent of
the locally small type $\UU$ by a small type $A$. 
Therefore the image of $I_k(A)$ is essentially small by \cref{cor:im_small}.
Since we want the $(k+1)$-truncation to be an operation $\UU\to\UU$, it would be more precise to define $\trunc{k+1}{A}$ as the (unique) type in $\UU$ that is equivalent to $\im(I_k(A))$. Of course, this makes no substantial difference.

\begin{lem}\label{lem:modal_contr}
For every $a,b:A$, we have an equivalence
\begin{equation*}
\eqv{(I_k(A)(a)=I_k(A)(b))}{\trunc{k}{a=b}}.
\end{equation*}
\end{lem}

\begin{proof}
Since $\im(I_k(A))$ is a subtype of $\UU^A$, there is for any $b:A$ a `tautological' family $E_b$ of types over $\im(I_k(A))$, given by
\begin{equation*}
E_b(P) \defeq P(b).
\end{equation*}
Note that $E_b(I_k(A)(a))\jdeq \trunc{k}{a=b}$. Therefore we can prove the claim by showing that the canonical map
\begin{equation*}
\prd{P:\im(I_k(A))} (I_k(A)(b)=P)\to P(b)
\end{equation*}
is a fiberwise equivalence. By \cref{thm:id_fundamental} it suffices to show that for each $b:A$, the total space
\begin{equation*}
\sm{P:\im(I_k(A))}P(b)
\end{equation*}
is contractible. 

For the center of contraction we take the pair
$\pairr{I_k(A)(b),\tproj{k}{\refl{b}}}$.
For the contraction we construct a term of type
\begin{equation*}
\prd{P:\im(I_k(A))}{y:P(b)} \pairr{I_k(A)(b),\tproj{k}{\refl{b}}}=\pairr{P,y}.
\end{equation*}
Since $I_k(A)(b,a)\jdeq\trunc{k}{b=a}$, it is equivalent to construct a term of type
\begin{equation*}
\prd{P:\im(I_k(A))}{y:P(b)}\sm{\alpha:\prd{a:A} \eqv{\trunc{k}{b=a}}{P(a)}} \alpha_b(\tproj{k}{\refl{b}})=y.
\end{equation*}
Let $P:\im(I_k(A))$ and $y:P(b)$. Then $P(a)$ is $n$-truncated for any $a:A$. Therefore, to construct a map
$\alpha(P,y)_a:\trunc{k}{b=a}\to P(a)$, it suffices to construct a map of type $(b=a)\to P(a)$. This may be done by
path induction, using $y:P(b)$. Since it follows that $\alpha(P,y)_b(\tproj{k}{\refl{b}})=y$, it only remains to show that each $\alpha(P,y)_a$ is an equivalence.  

Note that the type of those $P:\im(I_k(A))$ such that for all $y:P(b)$ and all $a:A$ the map $\alpha(P,y)_a$ is an equivalence, is a subtype of $\im(I_k(A))$, we may use the universal property of the image of $I_k(A)$: it suffices to lift
\begin{equation*}
\begin{tikzcd}
& \sm{P:\im(I_k(A))}\prd{y:P(b)}{a:A}\isequiv(\alpha(P,y)_a) \arrow[d] \\
A \arrow[ur,densely dotted] \arrow[r,swap,"I_k(A)"] & \im(I_k(A)).
\end{tikzcd}
\end{equation*}
In other words, it suffices to show that 
\begin{equation*}
\prd{x:A}{y:I_k(A)(x,b)}{a:A}\isequiv(\alpha(I_k(A)(x),y)_a).
\end{equation*}
Thus, we want to show that for any $y:\trunc{k}{x=b}$, the map $\trunc{k}{a=b}\to\trunc{k}{x=b}$ constructed above is an equivalence.
Since the fibers of this map are $n$-truncated, and $\iscontr(X)$ of an $n$-truncated type $X$ is always $n$-truncated, we may assume that $y$ is of the form $\tproj{k}{p}$ for $p:x=b$. 
Now it is easy to see that our map of type $\trunc{k}{b=a}\to\trunc{k}{x=a}$ is the unique map which
extends the path concatenation $\ct{p}{\blank}$, as indicated in the diagram
\begin{equation*}
\begin{tikzcd}[column sep=8em]
(b=a) \arrow[r,"\ct{p}{\blank}"] \arrow[d] & (x=a) \arrow[d] \\
\trunc{k}{b=a} \arrow[r,densely dotted,swap,"{\alpha(I_k(A)(x),y)_a}"] & \trunc{k}{x=a}.
\end{tikzcd}
\end{equation*}
Since the top map is an equivalence, it follows that the map $\alpha(I_k(A)(x),y)_a$ is an equivalence.
\end{proof}

\begin{cor}\label{cor:truncated}
The image $\im(I_k(A))$ is an $(n+1)$-truncated type. 
\end{cor}


\begin{proof}[Construction]
We will show that $\trunc{n+1}{A}$ is indeed $(n+1)$-truncated in \autoref{cor:truncated} of \autoref{lem:modal_contr} below. Once this fact is established, it remains to verify the dependent universal property of $(n+1)$-truncation.
By the join extension theorem \autoref{thm:join-extension} (using $N\defeq \emptyt$), it suffices to show that the map $\tproj{n+1}{\blank}:A\to\trunc{n+1}{A}$ is $\sphere{n+2}$-connected. Note that $\tproj{n+1}{\blank}$ is surjective, so the claim that $\tproj{n+1}{\blank}$ is $\sphere{n+2}$-connected follows from \autoref{lem:ap_connectivity}, where we show that for any surjective map $f:A\to X$, if the action on paths is $M$-connected for any two points in $A$, then $f$ is $\susp(M)$-connected. To apply this lemma, we also need to know that $\tproj{k}{\blank}:A\to\trunc{k}{A}$ is $\sphere{n+1}$-connected. This is shown in Corollary 7.5.8 of \cite{hottbook}.
\end{proof}


Before we are able to show that for any surjective map $f:A\to X$, if the action on paths is $M$-connected for any two points in $A$, then $f$ is $\susp(M)$-connected, we show that a type is $\susp(M)$-connected precisely when its identity types are $M$-connected.

\begin{lem}\label{lem:local_id}
Let $M$ be a type. Then a type $X$ is $(\join{\bool}{M})$-null
if and only if all of its identity types are $M$-null. 
\end{lem}

\begin{proof}
The map
\begin{equation*}
\lam{p}{m}p : (x=y)\to (M\to (x=y))
\end{equation*}
is an equivalence if and only if the induced map on total spaces
\begin{equation*}
\lam{\pairr{x,y,p}}\pairr{x,y,\lam{m}p} : \Big(\sm{x,y:X}x=y\Big)\to\Big(\sm{x,y:X}M\to (x=y)\Big)
\end{equation*}
is an equivalence. 
Since the map $\lam{x}\pairr{x,x,\refl{x}}:X\to\sm{x,y:X}x=y$ is an equivalence,
the above map is an equivalence if and only if the map
\begin{equation*}
\lam{x}\pairr{x,x,\lam{m}\refl{x}} : X\to\Big(\sm{x,y:X}M\to (x=y)\Big)
\end{equation*}
is an equivalence. For every $x:X$, the triple $\pairr{x,x,\lam{m}\refl{x}}$
induces a map $\susp(M)\to X$. By uniqueness of the universal property,
it follows that this map is the constant map $\lam{m}x$.
Thus we see that $\lam{x}\pairr{x,x,\lam{m}\refl{x}}$ is an equivalence if
and only if the map
\begin{equation*}
\lam{x}{m}x : X \to (\susp(M)\to X)
\end{equation*}
is an equivalence. 
\end{proof}

\begin{lem}\label{lem:ap_connectivity}
Suppose $f:A\to X$ is a surjective map, with the property that for every
$a,b:A$, the map
\begin{equation*}
\mapfunc{f}(a,b):(a=b)\to (f(a)=f(b))
\end{equation*}
is $M$-connected. Then $f$ is $\susp(M)$-connected. 
\end{lem}

\begin{proof}
We have to show that $\fib{f}{x}$ is $\susp(M)$-connected for each $x:X$. 
Since this is a mere proposition, and we assume that $f$ is surjective, it
is equivalent to show that $\fib{f}{f(a)}$ is $\susp(M)$-connected for each $a:A$. 
Let $Y$ be a $\susp(M)$-null type. 
For every $g:\fib{f}{f(a)}\to Y$ be a map we have the point $\theta(g)\defeq g(a,\refl{f(a)})$ in $Y$,
so we obtain a map
\begin{equation*}
\theta : (\fib{f}{f(a)}\to Y)\to Y
\end{equation*}
It is clear that $\theta(\lam{\pairr{b,p}}y)=y$, so it remains to show that
for every $g:\fib{f}{f(a)}\to Y$ we have $\lam{\pairr{b,p}}\theta(g)=g$.
That is, we must show that
\begin{equation*}
\prd{b:A}{p:f(a)=f(b)} g(a,\refl{f(a)})=g(b,p).
\end{equation*}
Using the assumption that $Y$ is $\susp(M)$-connected, it follows from
\autoref{lem:local_id} that the type $g(a,\refl{f(a)})=g(b,p)$ is $M$-connected,
for every $b:A$ and $p:f(a)=f(b)$.
Therefore it follows, since the map $\mapfunc{f}(a,b):(a=b)\to(f(a)=f(b))$ is connected, that our goal is equivalent to
\begin{equation*}
\prd{b:A}{p:a=b} g(a,\refl{f(a)})=g(b,\mapfunc{f}(a,b,p)).
\end{equation*}
This follows by path induction. 
\end{proof}
\end{comment}

\begin{exercises}
\item Consider an equivalence relation $R:A\to (A\to\prop)$. Show that the map $\tproj{0}{\blank}\circ \inl:A\to \trunc{0}{A\sqcup^{R} A}$ satisfies the universal property of the quotient $A/R$, where $A\sqcup^{R} A$ is the canonical pushout
\begin{equation*}
\begin{tikzcd}
\sm{x,y:A}R(x,y) \arrow[r,"\pi_2"] \arrow[d,swap,"\pi_1"] & A \arrow[d,"\inr"] \\
A \arrow[r,swap,"\inl"] & A\sqcup^{R} A.
\end{tikzcd}
\end{equation*}
\item Consider the trivial relation $\unit\defeq\lam{x}{y}\unit:A\to (A\to\prop)$. Show that the set quotient $A/\unit$ is a proposition satisfying the universal property of the propositional truncation.
\item Show that the type of pointed $2$-element sets
\begin{equation*}
\sm{X:\UU_{\bool}}X
\end{equation*}
is contractible.
\item Define the type $\mathbb{F}$ of finite sets by
\begin{equation*}
\mathbb{F}\defeq \im(\fin),
\end{equation*}
where $\fin:\N\to\UU$ is defined in \cref{defn:fin}. 
\begin{subexenum}
\item Show that $\eqv{\mathbb{F}}{\sm{n:\N}\UU_{\fin(n)}}$. 
\item Show that $\mathbb{F}$ is closed under $\Sigma$ and $\Pi$. 
\end{subexenum}
\end{exercises}

\section{Truncations}

\chapter{Synthetic homotopy theory}
\chapter{Homotopy groups of types}

\section{Pointed types}

\begin{defn}
\begin{enumerate}
\item A pointed type consists of a type $X$ equipped with a base point $x:X$. We will write $\UU_\ast$ for the type $\sm{X:\UU}X$ of all pointed types.
\item Let $(X,\ast_X)$ be a pointed type. A \define{pointed family} over $(X,\ast_X)$ consists of a type family $P:X\to \UU_\ast$ equipped with a base point $\ast_P:P(\ast_X)$. 
\item Let $(P,\ast_P)$ be a pointed family over $(X,\ast_X)$. A \define{pointed section} of $(P,\ast_P)$ consists of a dependent function $f:\prd{x:X}P(x)$ and an identification $p:f(\ast_X)=\ast_P$. We define the \define{pointed $\Pi$-type} to be the type of pointed sections:
\begin{equation*}
\Pi^\ast_{(x:X)}P(x) \defeq \sm{f:\prd{x:X}P(x)}f(\ast_X)=\ast_P
\end{equation*}
In the case of two pointed types $X$ and $Y$, we may also view $Y$ as a pointed family over $X$. In this case we write $X\to_\ast Y$ for the type of pointed functions.
\item Given any two pointed sections $f$ and $g$ of a pointed family $P$ over $X$, we define the type of pointed homotopies
\begin{equation*}
f\htpy_\ast g \defeq \Pi^\ast_{(x:X)} f(x)=g(x),
\end{equation*}
where the family $x\mapsto f(x)=g(x)$ is equipped with the base point $\ct{p}{q^{-1}}$. 
\end{enumerate}
\end{defn}

\begin{eg}
The circle $\sphere{1}$ is a pointed type with base point $\base:\sphere{1}$.
\end{eg}

\begin{eg}
If $X$ is a pointed type, then in the suspension of $X$ we have the canonical identification $\merid(\ast_X):\north=\south$. Therefore we do not have to worry about whether to choose $\north$ or $\south$ as the base point of $\susp{X}$. 
\end{eg} 

\begin{rmk}
Since pointed homotopies are defined as certain pointed sections, we can use the same definition of pointed homotopies again to consider pointed homotopies between pointed homotopies, and so on.
\end{rmk}

\begin{defn}
\begin{enumerate}
\item For any pointed type $X$, we define the \define{pointed identity function} $\mathsf{id}^\ast_X\defeq (\idfunc[X],\refl{\ast})$. 
\item For any two pointed maps $f:X\to_\ast Y$ and $g:Y\to_\ast Z$, we define the \define{pointed composite}
\begin{equation*}
g\mathbin{\circ_\ast} f \defeq (g\circ f,\ct{\ap{g}{p_f}}{p_g}).
\end{equation*}
\end{enumerate}
\end{defn}

\section{Loop spaces}
\begin{defn}
Let $X$ be a pointed type with base point $x$. We define the \define{loop space} $\loopspace{X,x}$ of $X$ at $x$ to be the pointed type $x=x$ with base point $\refl{x}$. 
\end{defn}

\begin{defn}
The loop space operation $\loopspacesym$ is \emph{functorial} in the sense that
\begin{enumerate}
\item For every pointed map $f:X\to_\ast Y$ there is a pointed map
\begin{equation*}
\loopspace{f}:\loopspace{X}\to_\ast \loopspace{Y},
\end{equation*}
defined by $\loopspace{f}(\omega)\defeq \ct{p_f}{\ap{f}{\omega}}{p_f^{-1}}$, which is base point preserving by $\mathsf{right\usc{}inv}(p_f)$. 
\item For every pointed type $X$ there is a pointed homotopy
\begin{equation*}
\loopspace{\mathsf{id}_X^\ast}\htpy_\ast \mathsf{id}^\ast_{\loopspace{X}}.
\end{equation*}
\item For any two pointed maps $f:X\to_\ast Y$ and $g:Y\to_\ast X$, there is a pointed homotopy witnessing that the triangle
\begin{equation*}
\begin{tikzcd}
& \loopspace{Y} \arrow[dr,"\loopspace{g}"] \\
\loopspace{X} \arrow[rr,swap,"\loopspace{g\circ_\ast f}"] \arrow[ur,"\loopspace{f}"] & & \loopspace{Z}
\end{tikzcd}
\end{equation*}
of pointed types commutes.
\end{enumerate}
\end{defn}

\section{Homotopy groups}
In homotopy type theory we use $0$-types to define groups.
\begin{defn}
A \define{group} $\mathcal{G}$ consists of a set $G$ with a unit $e:G$, a multiplication $x,y\mapsto x\cdot y$, and an inverse operation $x\mapsto x^{-1}$ satisfying the \define{group laws}:
\begin{align*}
(x\cdot y)\cdot z & =x\cdot(y\cdot z) & x^{-1}\cdot x & = e \\
e\cdot x & = x & x\cdot x^{-1} & = e. \\
x\cdot e & =x
\end{align*}
\end{defn}

\begin{defn}
For $n\geq 1$, the \define{$n$-th homotopy group} of a type $X$ at a base point $x:X$ consists of the type
\begin{equation*}
|\pi_n(X,x)| \defeq \trunc{0}{\loopspace[n]{X,x}}
\end{equation*}
equipped with the group operations inherited from the path operations on $\loopspace[n]{X,x}$. 
Often we will simply write $\pi_n(X)$ when it is clear from the context what the base point of $X$ is.

For $n\jdeq 0$ we define $\pi_0(X,x)\defeq \trunc{0}{X}$. 
\end{defn}

\begin{eg}
In \autoref{circle_loopspace} we established that $\loopspace{\sphere{1}}=\Z$. It follows that
\begin{equation*}
\pi_1(\sphere{1})=\Z \qquad\text{ and }\qquad\pi_n(\sphere{1})=0\qquad\text{for $n\geq 2$.}
\end{equation*}
Furthermore, we have seen in \autoref{circle_conn} that $\trunc{0}{\sphere{1}}$ is contractible. 
Therefore we also have $\pi_0(\sphere{1})=0$.
\end{eg}

\begin{thm}[The Eckmann-Hilton argument]
For $n\geq 2$, the $n$-th homotopy group is abelian.
\end{thm}

\begin{exercises}
\item Show that the type of pointed families over a pointed type $(X,x)$ is equivalent to the type
\begin{equation*}
\sm{Y:\UU_\ast} Y\to_\ast X.
\end{equation*}
\item Given two pointed types $A$ and $X$, we say that $A$ is a (pointed) retract of $X$ if we have $i:A\to_\ast X$, a retraction $r:X\to_\ast A$, and a pointed homotopy $H:r\circ_\ast i\htpy_\ast \idfunc^\ast$. 
\begin{subexenum}
\item Show that if $A$ is a pointed retract of $X$, then $\loopspace{A}$ is a pointed retract of $\loopspace{X}$. 
\item Show that if $A$ is a pointed retract of $X$ and $\pi_n(X)$ is a trivial group, then $\pi_n(A)$ is a trivial group.
\end{subexenum}
\item Construct by path induction a family of maps
\begin{equation*}
\prd{A,B:\UU}{a:A}{b:B} (\id{\pairr{A,a}}{\pairr{B,b}})\to \sm{e:\eqv{A}{B}}e(a)=b,
\end{equation*}
and show that this map is an equivalence. In other words, an \emph{identification of pointed types} is a base point preserving equivalence.
\item Let $\pairr{A,a}$ and $\pairr{B,b}$ be two pointed types. Construct by path induction a family of maps
\begin{equation*}
\prd{f,g:A\to B}{p:f(a)=b}{q:g(a)=b} (\id{\pairr{f,p}}{\pairr{g,q}})\to \sm{H:f\htpy g} p = \ct{H(a)}{q},
\end{equation*}
and show that this map is an equivalence. In other words, an \emph{identification of pointed maps} is a base point preserving homotopy.
\item Show that if $A\leftarrow S\rightarrow B$ is a span of pointed types, then for any pointed type $X$ the square
\begin{equation*}
\begin{tikzcd}
(A\sqcup^S B \to_\ast X) \arrow[r] \arrow[d] & (B \to_\ast X) \arrow[d] \\
(A\to_\ast X) \arrow[r] & (S\to_\ast X)
\end{tikzcd}
\end{equation*}
is a pullback square.
\item \label{ex:yoneda_ptd_types}Let $f:A\to_\ast B$ be a pointed map. Show that the following are equivalent:
\begin{enumerate}
\item $f$ is an equivalence.
\item For any pointed type $X$, the precomposition map
\begin{equation*}
\blank\mathbin{\circ_\ast}f:(B\to_\ast X)\to_\ast (A\to_\ast X)
\end{equation*}
is an equivalence. 
\end{enumerate}
\item In this exercise we prove the suspension-loopspace adjunction.
\begin{subexenum}
\item Construct a pointed equivalence
\begin{equation*}
\tau_{X,Y}:(\susp(X)\to_\ast Y) \eqvsym_\ast (X\to \loopspace{Y})
\end{equation*}
for any two pointed spaces $X$ and $Y$.
\item Show that for any $f:X\to_\ast X'$ and $g:Y'\to_\ast Y$, there is a pointed homotopy witnessing that the square
\begin{equation*}
\begin{tikzcd}[column sep=large]
(\susp(X')\to_\ast Y') \arrow[r,"\tau_{X',Y'}"] \arrow[d,swap,"h\mapsto g\circ h\circ \susp(f)"] & (X'\to_\ast \loopspace{Y'}) \arrow[d,"h\mapsto\loopspace{g}\circ h\circ f"] \\
(\susp(X)\to_\ast Y) \arrow[r,swap,"\tau_{X,Y}"] & (X\to_\ast \loopspace{Y})
\end{tikzcd}
\end{equation*}
\end{subexenum}
\item Show that if
\begin{equation*}
\begin{tikzcd}
C \arrow[r] \arrow[d] & B \arrow[d] \\
A \arrow[r] & X
\end{tikzcd}
\end{equation*}
is a pullback square of pointed types, then so is
\begin{equation*}
\begin{tikzcd}
\loopspace{C} \arrow[r] \arrow[d] & \loopspace{B} \arrow[d] \\
\loopspace{A} \arrow[r] & \loopspace{X}.
\end{tikzcd}
\end{equation*}
\end{exercises}

\section{The long exact sequence of homotopy groups}
\sectionmark{The long exact sequence}

\subsection{The long exact sequence}

\begin{defn}
A fiber sequence $F\hookrightarrow E \twoheadrightarrow B$ consists of:
\begin{enumerate}
\item Pointed types $F$, $E$, and $B$, with base points $x_0$, $y_0$, and $b_0$ respectively, 
\item Base point preserving maps $i:F\to_\ast E$ and $p:E\to_\ast B$, with $\alpha:i(x_0)=y_0$ and $\beta:p(y_0)=b_0$,
\item A pointed homotopy $H:\mathsf{const}_{b_0}\htpy_\ast p\circ_\ast i$ witnessing that the square
\begin{equation*}
\begin{tikzcd}
F \arrow[r,"i"] \arrow[d] & E \arrow[d,"p"] \\
\unit \arrow[r,swap,"\mathsf{const}_{b_0}"] & B,
\end{tikzcd}
\end{equation*}
commutes and is a pullback square.
\end{enumerate}
\end{defn}

\begin{lem}
Any fiber sequence $F\hookrightarrow E\twoheadrightarrow B$ induces a sequence of pointed maps
\begin{equation*}
\begin{tikzcd}
\loopspace{F} \arrow[r,"\loopspace{i}"] & \loopspace{E} \arrow[r,"\loopspace{p}"] & \loopspace{B} \arrow[r,"\partial"] & F \arrow[r,"i"] & E \arrow[r,"p"] & B,
\end{tikzcd}
\end{equation*}
in which every two consecutive maps form a fiber sequence.
\end{lem}

\begin{proof}
By taking pullback squares repeatedly, we obtain the diagram
\begin{equation*}
\begin{gathered}[b]
\begin{tikzcd}[column sep=large]
\loopspace{F} \arrow[d,swap,"\loopspace{i}"] \arrow[r] & \unit \arrow[d,"\mathsf{const}_{\refl{b_0}}"] \\
\loopspace{E} \arrow[r,"\loopspace{p}"] \arrow[d] & \loopspace{B} \arrow[r] \arrow[d,swap,"\partial"] & \unit \arrow[d,"\mathsf{const}_{y_0}"] \\
\unit \arrow[r,swap,"\mathsf{const}_{x_0}"] & F \arrow[r,"i"] \arrow[d] & E \arrow[d,"p"] \\
& \unit \arrow[r,swap,"\mathsf{const}_{b_0}"] & B.
\end{tikzcd}\\[-\dp\strutbox]
\end{gathered}\qedhere
\end{equation*}
\end{proof}

\begin{defn}
We say that a consecutive pair of pointed maps between pointed sets
\begin{equation*}
\begin{tikzcd}
A \arrow[r,"f"] & B \arrow[r,"g"] & C
\end{tikzcd}
\end{equation*}
is \define{exact} at $B$ if we have
\begin{equation*}
\Big(\exis{a:A}f(a)=b\Big)\leftrightarrow (g(b)=c)
\end{equation*}
for any $b:B$. 
\end{defn}

\begin{rmk}
If a pair of consecutive pointed maps between pointed sets
\begin{equation*}
\begin{tikzcd}
A \arrow[r,"f"] & B \arrow[r,"g"] & C
\end{tikzcd}
\end{equation*}
is exact at $B$, it directly that $\im(f)=\fib{g}{c}$. Indeed, such a pair of pointed maps is exact at $B$ if and only if there is an equivalence $e:\im(f)\eqvsym \fib{g}{c}$ such that the triangle
\begin{equation*}
\begin{tikzcd}[column sep=tiny]
\im(f) \arrow[dr] \arrow[rr,"e"] & & \fib{g}{c} \arrow[dl] \\
& B
\end{tikzcd}
\end{equation*}
commutes. In other words, $\im(f)$ and $\fib{g}{c}$ are equal \emph{as subsets of $B$}.
\end{rmk}

\begin{lem}
Suppose $F\hookrightarrow E \twoheadrightarrow B$ is a fiber sequence. Then the sequence
\begin{equation*}
\begin{tikzcd}
\trunc{0}{F} \arrow[r,"\trunc{0}{i}"] & \trunc{0}{E} \arrow[r,"\trunc{0}{p}"] & \trunc{0}{B}
\end{tikzcd}
\end{equation*}
is exact at $\trunc{0}{E}$. 
\end{lem}

\begin{proof}
To show that the image $\im\trunc{0}{i}$ is the fiber $\fib{\trunc{0}{p}}{\tproj{0}{b_0}}$, it suffices to construct a fiberwise equivalence
\begin{equation*}
\prd{x:\trunc{0}{E}} \trunc{-1}{\fib{\trunc{0}{i}}{x}} \eqvsym \trunc{0}{p}(x)=\tproj{0}{b_0}.
\end{equation*}
By the universal property of $0$-truncation it suffices to show that
\begin{equation*}
\prd{x:E} \trunc{-1}{\fib{\trunc{0}{i}}{\tproj{0}{x}}} \eqvsym \trunc{0}{p}(\tproj{0}{x})=\tproj{0}{b_0}.
\end{equation*}
First we note that 
\begin{align*}
\trunc{0}{p}(\tproj{0}{x})=\tproj{0}{b_0} & \eqvsym \tproj{0}{p(x)} = \tproj{0}{b_0} \\
& \eqvsym \trunc{-1}{p(x)=b_0}.
\end{align*}
Next, we note that
\begin{align*}
\fib{\trunc{0}{i}}{\tproj{0}{x}} & \eqvsym \sm{y:\trunc{0}{F}}\trunc{0}{i}(y)=\tproj{0}{x} \\
& \eqvsym \trunc{0}{\sm{y:F}\trunc{0}{i}(\tproj{0}{y})=\tproj{0}{x}} \\
& \eqvsym \trunc{0}{\sm{y:F}\tproj{0}{i(y)}=\tproj{0}{x}} \\
& \eqvsym \trunc{0}{\sm{y:F}\trunc{-1}{i(y)=x}}.
\end{align*}
Therefore it follows that
\begin{align*}
\trunc{-1}{\fib{\trunc{0}{i}}{\tproj{0}{x}}} & \eqvsym \trunc{-1}{\sm{y:F}\trunc{-1}{i(y)=x}} \\
& \eqvsym \trunc{-1}{\sm{y:F}i(y)=x} \\
\end{align*}
Now it suffices to show that $\eqv{\big(\sm{y:F}i(y)=x\big)}{p(x)=b_0}$. This follows by the pasting lemma of pullbacks
\begin{equation*}
\begin{tikzcd}
(p(x)=b_0) \arrow[r] \arrow[d] & \unit \arrow[d] \\
F \arrow[r] \arrow[d] & E \arrow[d] \\
\unit \arrow[r] & B
\end{tikzcd}
\end{equation*}
\end{proof}

\begin{thm}
Any fiber sequence $F\hookrightarrow E\twoheadrightarrow B$ induces a long exact sequence on homotopy groups
\begin{equation*}
\begin{tikzcd}
& & \cdots \arrow[out=355,in=175,dll] \\
\pi_n(F) \arrow[r,"\pi_n(i)"] & \pi_n(E) \arrow[r,"\pi_n(p)"] & \pi_n(B) \arrow[out=355,in=175,dll,densely dotted] \\
\pi_1(F) \arrow[r,"\pi_1(i)"] & \pi_1(E) \arrow[r,"\pi_1(p)"] & \pi_1(B) \arrow[out=355,in=175,dll] \\
\pi_0(F) \arrow[r,"\pi_0(i)"] & \pi_0(E) \arrow[r,"\pi_0(p)"] & \pi_0(B)
\end{tikzcd}
\end{equation*}
\end{thm}

\subsection{The Hopf fibration}
Our goal in this section is to construct the Hopf fibration, i.e.~a fiber sequence
\begin{equation*}
\sphere{1}\hookrightarrow\sphere{3}\twoheadrightarrow\sphere{2}.
\end{equation*}
This fiber sequence involves the complex multiplication of the unit sphere in the complex number, which is a circle. Viewing the circle as a subspace of the complex numbers, we write $1$ for the base point of the circle.

\begin{defn}
We define the \define{complex multiplication} operation
\begin{equation*}
\mu_{\mathbb{C}}:\sphere{1}\to(\sphere{1}\to\sphere{1}).
\end{equation*}
\end{defn}

\begin{constr}
By the universal property of the circle, it is equivalent to define
\begin{align*}
\mu_{\mathbb{C}}(1) & : \sphere{1}\to\sphere{1} \\
\ap{\mu_{\mathbb{C}}}{\lloop} & : \mu_{\mathbb{C}}(1)=\mu_{\mathbb{C}}(1). 
\end{align*}
The function $\mu_{\mathbb{C}}(1)$ is multiplication by $1$, which is the identity function. The type of $\ap{\mu_{\mathbb{C}}}{\lloop}$ is equivalent to the type of homotopies
\begin{equation*}
\idfunc[\sphere{1}] \htpy \idfunc[\sphere{1}]. 
\end{equation*}
We construct this homotopy by induction on $\sphere{1}$. Therefore it suffices to construct
\begin{align*}
p & : 1=1 \\
q & : \mathsf{tr}_{L}(\lloop,p)=p
\end{align*}
\end{constr}

\begin{lem}
The complex multiplication operation $\mu_{\mathbb{C}}$ on the circle satisfies the unit laws
\begin{align*}
\mathsf{left\usc{}unit}_{\mathbb{C}}(x) & : \mu_{\mathbb{C}}(1,x) = x \\
\mathsf{right\usc{}unit}_{\mathbb{C}}(x) & : \mu_{\mathbb{C}}(x,1) = x \\
\mathsf{coh\usc{}unit}_{\mathbb{C}} & : \mathsf{left\usc{}unit}_{\mathbb{C}}(1)=\mathsf{right\usc{}unit}_{\mathbb{C}}(1),
\end{align*}
and the functions $\mu_{\mathbb{C}}(x,\blank)$ and $\mu_{\mathbb{C}}(\blank,y)$ are equivalences for each $x:\sphere{1}$ and $y:\sphere{1}$, respectively.
\end{lem}

\begin{lem}
Both commuting squares in the diagram
\begin{equation*}
\begin{tikzcd}
\sphere{1} \arrow[d] & \sphere{1}\times\sphere{1} \arrow[d,swap,"\mu_{\mathbb{C}}"] \arrow[l,swap,"\proj 1"] \arrow[r,"\proj 2"] & \sphere{1} \arrow[d] \\
\unit & \sphere{1} \arrow[l] \arrow[r] & \unit
\end{tikzcd}
\end{equation*}
are pullback squares.
\end{lem}

\begin{cor}
There is a fiber sequence
\begin{equation*}
\sphere{1}\hookrightarrow \join{\sphere{1}}{\sphere{1}} \twoheadrightarrow \sphere{2}.
\end{equation*}
\end{cor}

\begin{lem}
The join operation is associative
\end{lem}

\begin{proof}
\begin{equation*}
\begin{tikzcd}
A & A\times C \arrow[l] \arrow[r] & A\times C \\
A\times B \arrow[u] \arrow[d] & A\times B \times C \arrow[r] \arrow[d] \arrow[l] \arrow[u] & A\times C \arrow[u] \arrow[d] \\
B & B\times C \arrow[l] \arrow[r] & C
\end{tikzcd}
\end{equation*}
\end{proof}

\begin{cor}
There is an equivalence $\eqv{\join{\sphere{1}}{\sphere{1}}}{\sphere{3}}$.
\end{cor}

\begin{thm}
There is a fiber sequence $\sphere{1}\hookrightarrow\sphere{3}\twoheadrightarrow\sphere{2}$. 
\end{thm}

\begin{lem}
Suppose $f:G\to H$ is a group homomorphism, such that the sequence
\begin{equation*}
\begin{tikzcd}
0 \arrow[r] & G \arrow[r,"f"] & H \arrow[r] & 0
\end{tikzcd}
\end{equation*}
is exact at $G$ and $H$, where we write $0$ for the trivial group consisting of just the unit element. Then $f$ is a group isomorphism.
\end{lem}

\begin{cor}
We have $\pi_2(\sphere{2})=\Z$, and for $k>2$ we have $\pi_k(\sphere{2})=\pi_k(\sphere{3})$.
\end{cor}


\begin{exercises}
\exercise Give the $0$-sphere $\sphere{0}$ the structure of an H-space.
\exercise For any pointed type $A$, give $\loopspace{A}$ the structure of an H-space.
\exercise Show that the type of (small) fiber sequences is equivalent to the type of quadruples $(B,P,b_0,x_0)$, consisting of
\begin{align*}
B & : \UU \\
P & : B \to \UU \\
b_0 & : B \\
x_0 & : P(b_0).
\end{align*}
\exercise Show that there is a fiber sequence
\begin{equation*}
  \sphere{3}\hookrightarrow\sphere{2}\twoheadrightarrow\trunc{2}{\sphere{2}},
\end{equation*}
where the map $\sphere{2}\to\trunc{2}{\sphere{2}}$ is the unit of the $2$-truncation.
\end{exercises}

% !TEX root = hott_intro.tex

\section{Connected types and maps}

\begin{defn}
A type $A$ is said to be $n$-connected if $\trunc{n}{A}$ is contractible.
If $A$ is $0$-connected we also say that $A$ is \define{(path) connected},
and if $A$ is $1$-connected we also say that $A$ is \define{simply connected}.
\end{defn}

\begin{thm}
Let $f:A\to B$ be a map. The following are equivalent:
\begin{enumerate}
\item The map $f$ is \define{$n$-connected}, i.e. the fibers of $f$ are $n$-connected types.
\item The map $f$ is \define{left orthogonal} with respect to every $n$-truncated map $g:X\to Y$, i.e. the square
\begin{equation*}
\begin{tikzcd}
X^B \arrow[r,"\blank\circ f"] \arrow[d,swap,"g\circ \blank"] & X^A \arrow[d,"g\circ\blank"] \\
Y^B \arrow[r,swap,"\blank\circ f"] & Y^A
\end{tikzcd}
\end{equation*}
is a pullback square.
\item For each $i\leq n$, the map $f$ induces an isomorphism
\begin{equation*}
\pi_i(f):\pi_i(A)\to\pi_i(B)
\end{equation*}
of homotopy groups, and 
\begin{equation*}
\pi_{n+1}(f):\pi_{n+1}(A)\to\pi_{n+1}(B)
\end{equation*}
is surjective.
\end{enumerate}
\end{thm}

\begin{thm}
Consider a commuting square
\begin{equation*}
\begin{tikzcd}
A \arrow[d,swap,"f"] \arrow[r] & B \arrow[d,"g"] \\
X \arrow[r] & Y
\end{tikzcd}
\end{equation*}
The following are equivalent:
\begin{enumerate}
\item The map $A\to X\times_Y B$ is $n$-connected. In this case the square is called \define{$n$-cartesian}.
\item For each $x:X$ the map
\begin{equation*}
\fib{f}{x}\to \fib{g}{f(x)}
\end{equation*}
is $n$-connected.
\end{enumerate}
\end{thm}

\begin{thm}
If $X$ is $m$-connected and $Y$ is $n$-connected, then $\join{X}{Y}$ is $(m+n+2)$-connected. 
\end{thm}

\begin{thm}
Consider a pullback square
\begin{equation*}
\begin{tikzcd}
C \arrow[r] \arrow[d] & B \arrow[d] \\
A \arrow[r] & X.
\end{tikzcd}
\end{equation*}
If the maps $A\to X$ and $B\to X$ are $m$- and $n$-connected, respectively, then the map $A\sqcup^C B\to X$ is $(m+n+2)$-connected.
\end{thm}

\begin{thm}
  The connected maps contain the equivalences, are closed under coproducts, pushouts, retracts, and transfinite compositions.
\end{thm}

\begin{exercises}
\exercise Show that a pointed $X$ is $n$-connected precisely when $\pi_k(X)=0$ for each $k\leq n$.
\exercise Show that retracts of $n$-connected types are again $n$-connected.
\exercise Let $f:A\to_\ast B$ be a pointed map between pointed $n$-connected types, for $n\geq -1$. Show that the following are equivalent:
\begin{enumerate}
\item $f$ is an equivalence.
\item $\loopspace[n+1]{f}$ is an equivalence. 
\end{enumerate}
\exercise Let $f:A\to B$ be a surjective map, and let $g:A\to C$ be any map. Show that if there is a unique extension
\begin{equation*}
\begin{tikzcd}
\fib{f}{f(a)} \arrow[r,"g\circ\pi_1"] \arrow[d] & C \\
\unit \arrow[ur,densely dotted]
\end{tikzcd}
\end{equation*}
for any $a:A$, then $g$ extends uniquely along $f$.
\exercise Consider a span $A \leftarrow S \rightarrow B$, in which the map $S\to A$ is $n$-connected. Show that the map $\inr : B\to A\sqcup^S B$ is again $n$-connected.
\exercise Show that if
\begin{equation*}
\begin{tikzcd}
A \arrow[r] \arrow[d,swap,"f"] & B \arrow[d,"g"] \\
X \arrow[r] & Y
\end{tikzcd}
\end{equation*}
is a $k$-cocartesian, then the map $\mathsf{cofib}(f)\to \mathsf{cofib}(g)$ is $k$-connected.
\exercise Consider a commuting square
\begin{equation*}
\begin{tikzcd}
A \arrow[d,swap,"f"] \arrow[r] & B \arrow[d,"g"] \\
X \arrow[r] & Y
\end{tikzcd}
\end{equation*}
\begin{subexenum}
\item Show that if the square is $n$-cartesian and $g$ is $n$-connected, then so is $f$.
\item Show that if $f$ is $n$-connected and $g$ is $(n+1)$-connected, then the square is $n$-cartesian. 
\end{subexenum}
\exercise Show that any sequential colimit of $n$-connected types is again $n$-connected.
\exercise Consider a pushout square
\begin{equation*}
\begin{tikzcd}
A \arrow[d,swap,"f"] \arrow[r] & B \arrow[d,"g"] \\
X \arrow[r] & Y
\end{tikzcd}
\end{equation*}
Show that if $f$ is $n$-connected, then $g$ is $n$-connected.
\end{exercises}

\section{The wedge and smash product of pointed types}

\begin{defn}
Let $A$ and $B$ be pointed types.
\begin{enumerate}
\item We define the \define{wedge} $A\vee B$ of $A$ and $B$ to be the pushout
\begin{equation*}
\begin{tikzcd}
\unit \arrow[r] \arrow[d] & B \arrow[d] \\
A \arrow[r] & A\vee B
\end{tikzcd}
\end{equation*}
\item We define the \define{smash product} $A\wedge B$ of $A$ and $B$ to be the cofiber of the cogap map of the square
\begin{equation*}
\begin{tikzcd}
\unit \arrow[r] \arrow[d] & B \arrow[d] \\
A \arrow[r] & A\times B
\end{tikzcd}
\end{equation*}
That is, the smash product is defined as the cofiber of the canonical map $A\vee B\to A\times B$. 
\end{enumerate}
\end{defn}

For any two pointed types $A$ and $B$, there is a pointed map
\begin{equation*}
\mathsf{pair}_\ast : A \to_\ast (B\to_\ast A\wedge B).
\end{equation*}

\begin{thm}\label{thm:smash_adj}
Let $A$, $B$, and $X$ be pointed types. Then the pointed map
\begin{equation*}
(A\wedge B \to_\ast X)\to_\ast (A \to_\ast (B\to_\ast X))
\end{equation*}
given by $f\mapsto f\mathbin{\circ_\ast}\mathsf{pair}_\ast$ is an equivalence.
Moreover, these equivalences are natural in $A$, $B$, and $X$ in the sense that...
\end{thm}

\begin{cor}
For any $m,n:\N$ we have an equivalence
\begin{equation*}
\eqv{{\sphere{m}}\wedge{\sphere{n}}}{\sphere{m+n}}.
\end{equation*}
\end{cor}

\begin{proof}
We have
\begin{align*}
({\sphere{m}}\wedge{\sphere{n}}\to_\ast X) & \eqvsym (\sphere{m}\to_\ast (\sphere{n}\to_\ast X)) \\
& \eqvsym \loopspace[m]{\loopspace[n]{X}} \\
& \eqvsym \loopspace[m+n]{X} \\
& \eqvsym (\sphere{m+n}\to_\ast X)
\end{align*}
By the naturality of the equivalences in \cref{thm:smash_adj} it follows that the composite equivalence is given by precomposition by the pointed map 
\begin{equation*}
\sphere{m}\wedge\sphere{n}\to_\ast \sphere{m+n}
\end{equation*}
that corresponds to the identity map $\sphere{m+n}\to_\ast \sphere{m+n}$. Thus it follows by \cref{ex:yoneda_ptd_types} that this pointed map is an equivalence.
\end{proof}

\begin{thm}
Given two pointed spaces, there is an equivalence
\begin{equation*}
\eqv{\join{X}{Y}}{\susp(X\wedge Y)}.
\end{equation*}
\end{thm}

\begin{exercises}
\exercise 
\begin{subexenum}
\item Show that $Y$ is equivalent to the mapping cone of $X\to X\vee Y$.
\item Show that the pushout of $X \leftarrow X\vee Y \rightarrow Y$ is contractible.
\end{subexenum}
\exercise Let $A$ and $B$ be pointed types. Show that the square
\begin{equation*}
\begin{tikzcd}
A+B \arrow[r] \arrow[d] & A\times B \arrow[d] \\
1+1 \arrow[r] & A\wedge B
\end{tikzcd}
\end{equation*}
is cocartesian.
\exercise Show that if
\begin{equation*}
\begin{tikzcd}
S_1 \arrow[r] \arrow[d] & Y_1 \arrow[d] & S_2 \arrow[r] \arrow[d] & Y_2 \arrow[d] \\
X_1 \arrow[r] & Z_1 & X_2 \arrow[r] & Z_2
\end{tikzcd}
\end{equation*}
are pushout squares, where all types, maps and homotopies are pointed, then so is
\begin{equation*}
\begin{tikzcd}
S_1\vee S_2 \arrow[r] \arrow[d] & Y_1\vee Y_2 \arrow[d] \\
X_1 \vee X_2 \arrow[r] & Z_1\vee Z_2. 
\end{tikzcd}
\end{equation*}
\exercise Show that if
\begin{equation*}
\begin{tikzcd}
S \arrow[r] \arrow[d] & Y \arrow[d] \\
X \arrow[r] & Z
\end{tikzcd}
\end{equation*}
is a cocartesian square of pointed spaces, then the cofiber of $X\vee Y\to Z$ is equivalent to $\susp(S)$.
\exercise Show that there is an equivalence
\begin{equation*}
\eqv{\susp(X\times Y)}{\susp(X\vee Y)\vee \susp(X\wedge Y)}
\end{equation*}
\exercise Show that $\susp(X\vee Y)$ is a retract of $\susp(X\times Y)$. 
\exercise Show that if $f:A\to X$ is a constant of pointed spaces, then $\eqv{M_f}{X\vee \susp(A)}$. 
\exercise Show that the cofiber of the diagonal $\delta:\sphere{1}\to \sphere{1}\times\sphere{1}$ is equivalent to $\sphere{2}\vee\sphere{2}$.
\exercise Show that $\eqv{\mathsf{Fin}(n+1)\wedge \mathsf{Fin}(m+1)}{\mathsf{Fin}(n\cdot m)+\unit}$.
\end{exercises}

% !TEX root = hott_intro.tex

\section{The Blakers-Massey theorem}
The Blakers-Massey theorem is a connectivity theorem which can be used to prove the Freudenthal suspension theorem, giving rise to the field of \emph{stable homotopy theory}. It was proven in the setting of homotopy type theory by Lumsdaine et al, and their proof was the first that was given entirely in an elementary way, using only constructions that are invariant under homotopy equivalence. 

Consider a span $A \leftarrow S \rightarrow B$, consisting of an $m$-connected map $f:S\to A$ and an $n$-connected map $g:S\to B$. We take the pushout of this span, and subsequently the pullback of the resulting cospan, as indicated in the diagram
\begin{equation}\label{eq:BM}
\begin{tikzcd}
S \arrow[drr,bend left=15,"g"] \arrow[ddr,bend right=15,swap,"f"] \arrow[dr,densely dotted,"u" near end] \\
& A \times_{(A \sqcup^S B)} B \arrow[r,"\pi_2"] \arrow[d,"\pi_1"] & B \arrow[d,"\inr"] \\
& A \arrow[r,swap,"\inl"] & A \sqcup^S B.
\end{tikzcd}
\end{equation}
The universal property of the pullback determines a unique map $u:S\to A \times_{(A\sqcup^S B)} B$ as indicated.

\begin{thm}[Blakers-Massey]
The map $u:S\to A \times_{(A\sqcup^S B)} B$ of \cref{eq:BM} is $(n+m)$-connected.
\end{thm}

\begin{exercises}
\item Show that if $X$ is $m$-connected and $f:X\to Y$ is $n$-connected, then the map
\begin{equation*}
X \to \fib{m_f}{\ast}
\end{equation*}
where $m_f:Y\to M_f$ is the inclusion of $Y$ into the cofiber of $f$, is $(m+n)$-connected.
\item Suppose that $X$ is a connected type, and let $f:X\to Y$ be a map.
Show that the following are equivalent:
\begin{enumerate}
\item $f$ is $n$-connected.
\item The mapping cone of $f$ is $(n+1)$-connected.
\end{enumerate}
\item Apply the Blakers-Massey theorem to the defining pushout square of the smash product to show that if $A$ and $B$ are $m$- and $n$-connected respectively, then there is a $(m+n+\min(m,n)+2)$-connected map
\begin{equation*}
\join{\loopspace{A}}{\loopspace{B}}\to \loopspace{A \wedge B}.
\end{equation*}
\item Show that the square
\begin{equation*}
\begin{tikzcd}
\unit \arrow[r] \arrow[d] & \bool \arrow[d] \\
X \arrow[r] & X+\unit
\end{tikzcd}
\end{equation*}
is both a pullback and a pushout. Conclude that the result of the Blakers-Massey theorem is not always sharp.
\end{exercises}


%\chapter{Open problems in homotopy type theory}
We list only the open problems of synthetic homotopy theory, omitting the open problems related to the semantics of homotopy type theory.
\begin{itemize}
\item Compute the homotopy groups of the spheres.
\item Define the Grassmannians.
\item Prove Freyd's generating hypothesis
\item Hurewicz theorem
\item Barratt-Priddy theorem
\item (Semi-)simplicial types.
\item Ring structure on the sphere spectrum
\item Higher Van Kampen theorems
\item Higher Blakers-Massey theorems
\item Find a delooping of $\sphere{3}$, and define the octonionic Hopf fibration.
\end{itemize}


\backmatter

\printbibliography

\printindex

\end{document}
