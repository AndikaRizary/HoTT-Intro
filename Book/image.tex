\section{The homotopy image of a map}\label{chap:image}

\subsection{The universal property of the image of a map}
In this section we will construct the \emph{homotopy image} of an arbitrary map $f:A\to X$. The idea of the image of $f$ is that it is in a way the least subtype of $X$ that contains all the values of $f$. More precisely, the image of $f$ is an embedding $i:\im(f)\hookrightarrow X$ that fits in a commuting triangle
\begin{equation*}
  \begin{tikzcd}[column sep=tiny]
    A \arrow[rr,"q"] \arrow[dr,swap,"f"] & & \im(f) \arrow[dl,hook,"i"] \\
    \phantom{\im(f)} & X
  \end{tikzcd}
\end{equation*}
and satisfies the \emph{universal property} of the image inclusion of $f$. Informally, the universal property of the image asserts that there is a unique map $h:\im(f)\to B$ for which the diagram
\begin{equation*}
  \begin{tikzcd}[column sep=large]
    A \arrow[rr] \arrow[dr,"q"] \arrow[dddr,swap,"f"] & & B \arrow[dddl,"m"] \\
    & \im(f) \arrow[ur,densely dotted,"h"] \arrow[dd,"i"] \\ \\
    & X
  \end{tikzcd}
\end{equation*}
commutes. Note that there is quite a lot of information in this diagram: not only are there the three small commuting triangles; there is also the large commuting triange in the back, and there is a three-dimensional solid filling the space between the four triangles. We make the following definition, in order to express the universal property of the image efficiently.

\begin{defn}
  Let $f:A\to X$ and $g:B\to X$ be maps. A \define{morphism} from $f$ to $g$ over $X$ consists of a map $h:A\to B$ equipped with a homotopy $H:f\htpy g\circ h$ witnessing that the triangle
\begin{equation*}
\begin{tikzcd}[column sep=tiny]
A \arrow[rr,"h"] \arrow[dr,swap,"f"] & & B \arrow[dl,"g"] \\
& X
\end{tikzcd}
\end{equation*}
commutes. Thus, we define the type
\begin{equation*}
\mathrm{hom}_X(f,g)\defeq\sm{h:A\to B}f\htpy g\circ h.
\end{equation*}
Composition of morphisms over $X$ is defined by
\begin{equation*}
  (k,K)\circ (h,H) \defeq (k\circ h,\ct{H}{(K\cdot h)}).
\end{equation*}
\end{defn}

\begin{defn}
Consider a commuting triangle
\begin{equation*}
\begin{tikzcd}[column sep=small]
A \arrow[rr,"q"] \arrow[dr,swap,"f"] & & I \arrow[dl,"i"] \\
& X
\end{tikzcd}
\end{equation*}
with $H:f\htpy i\circ q$, where $i$ is an embedding\index{embedding}.
We say that $i$ has the \define{universal property of the image of $f$}\index{universal property!of the image} if the map
\begin{equation*}
\blank\circ(q,H) : \mathrm{hom}_X(i,m)\to\mathrm{hom}_X(f,m)
\end{equation*}
is an equivalence for every embedding $m:B\to X$. 
\end{defn}

\begin{rmk}
  Consider a commuting triangle
\begin{equation*}
\begin{tikzcd}[column sep=small]
A \arrow[rr,"q"] \arrow[dr,swap,"f"] & & I \arrow[dl,"i"] \\
& X
\end{tikzcd}
\end{equation*}
with $H:f\htpy i\circ q$, where $i$ is an embedding. Then it is not hard to see that the embedding $i$ satisfies the universal property of the image inclusion if and only if for every commuting triangle
\begin{equation*}
  \begin{tikzcd}[column sep=small]
    A \arrow[dr,swap,"f"] \arrow[rr,"g"] & & B \arrow[dl,"m"] \\
    & X
  \end{tikzcd}
\end{equation*}
with $G:f\htpy m\circ g$, where $m$ is an embedding, the type of quadruples $(h,K,L,M)$ consisting of
\begin{enumerate}
\item a map $h:I\to B$,
\item a homotopy $K:i\htpy m\circ h$ witnessing that the triangle
  \begin{equation*}
    \begin{tikzcd}[column sep=small]
      I \arrow[rr,"h"] \arrow[dr,swap,"i"] & & B \arrow[dl,"m"] \\
      & X
    \end{tikzcd}
  \end{equation*}
  commutes,
\item a homotopy $L:g\htpy h\circ q$ witnessing that the triangle
  \begin{equation*}
    \begin{tikzcd}[column sep=small]
      A \arrow[rr,"q"] \arrow[dr,swap,"g"] & & I \arrow[dl,"h"] \\
      & B
    \end{tikzcd}
  \end{equation*}
  commutes,
\item a homotopy $M:\ct{H}{(K\cdot q)}\htpy\ct{G}{(m\cdot L)}$ witnessing that the square
  \begin{equation*}
    \begin{tikzcd}
      f \arrow[d,swap,"H"] \arrow[r,"G"] & m\circ g \arrow[d,"m\cdot L"] \\
      i\circ q \arrow[r,swap,"K\cdot q"] & m\circ h\circ g
    \end{tikzcd}
  \end{equation*}
  commutes,
\end{enumerate}
is contractible. However, the situation is in fact much simpler, because the type $\mathrm{hom}_X(f,m)$ is a proposition whenever $m$ is an embedding.
\end{rmk}

\begin{lem}
For any $f:A\to X$ and any embedding\index{embedding} $m:B\to X$, the type $\mathrm{hom}_X(f,m)$ is a proposition.
\end{lem}

\begin{proof}
  Recall from \cref{ex:triangle_fib} that the type $\mathrm{hom}_X(f,m)$ is equivalent to the type
  \begin{equation*}
    \prd{x:X}\fib{f}{x}\to\fib{m}{x}.
  \end{equation*}
  Therefore it suffices to show that this type is a proposition. Recall from \cref{cor:prop_emb} that a map is an embedding if and only if its fibers are propositions.
  Thus we see that the type $\prd{x:X}\fib{f}{x}\to\fib{m}{x}$ is a product of propositions, hence it is a proposition by \cref{thm:trunc_pi}.
\end{proof}

\begin{prp}\label{prp:simplifly-universal-property-image}
  Consider a commuting triangle
  \begin{equation*}
    \begin{tikzcd}[column sep=small]
      A \arrow[rr,"q"] \arrow[dr,swap,"f"] & & I \arrow[dl,"i"] \\
      & X
\end{tikzcd}
  \end{equation*}
  with $H:f\htpy i\circ q$, where $i$ is an embedding. Then the following are equivalent:
  \begin{enumerate}
  \item The embedding $i$ satisfies the universal property of the image inclusion of $f$.
  \item For every embedding $m:B\to X$ there is a map
    \begin{equation*}
      \mathrm{hom}_X(f,m)\to\mathrm{hom}_X(i,m).
    \end{equation*}
  \end{enumerate}
\end{prp}

\begin{proof}
Since $\mathrm{hom}_X(f,m)$ is a proposition for every every embedding $m:B\to X$, the claim follows immediately by \cref{ex:prop_equiv}.
\end{proof}

\subsection{The universal property of propositional truncation}

An important special case of the homotopy image of a map is the image of the terminal projection
\begin{equation*}
  \const_\ttt : A \to \unit,
\end{equation*}
which results in an embedding $I\hookrightarrow \unit$. Embeddings into the unit type are in fact just propositions. To see this, note that
\begin{align*}
\sm{A:\UU}{f:A\to\unit}\isemb(f)
& \eqvsym \sm{A:\UU}\isemb(\const_\ttt) \\
& \eqvsym \sm{A:\UU}\prd{x:\unit}\isprop(\fib{\const_\ttt}{x}) \\
& \eqvsym \sm{A:\UU}\isprop(\fib{\const_\ttt}{\ttt}) \\
& \eqvsym \sm{A:\UU}\isprop(A).
\end{align*}
Therefore, the universal property of the image of the map $A\to\unit$ is equivalently described as a proposition $P$ satisfying the universal property of the propositional truncation:

\begin{defn}
Let $A$ be a type, and let $P$ be a proposition that comes equipped with a map $f:A\to P$. We say that $f:A\to P$ satisfies the \define{universal property of propositional truncation}\index{universal property!of propositional truncation} of $A$ if for every proposition $Q$, the precomposition map
\begin{equation*}
\blank\circ f:(P\to Q)\to (A\to Q)
\end{equation*}
is an equivalence.
\end{defn}

\begin{prp}
  Let $A$ be a type, and consider a commuting triangle
  \begin{equation*}
    \begin{tikzcd}[column sep=tiny]
      \phantom{P'} & A \arrow[dl,swap,"f"] \arrow[dr,"{f'}"] \\
      P \arrow[rr,swap,"h"] & & P'
    \end{tikzcd}
  \end{equation*}
  where $P$ and $P'$ are propositions. If any two of the following three assertions hold, so does the third:
  \begin{enumerate}
  \item The map $f$ satisfies the universal property of the propositional truncation of $A$.
  \item The map $f'$ satisfies the universal propertyof the propositional truncation of $A$.
  \item The map $h$ is an equivalence.
  \end{enumerate}
\end{prp}

\begin{proof}
  Note that the map $h:P\to P'$ is an equivalence if and only if for every proposition $Q$, the precomposition map
  \begin{equation*}
    \blank\circ h:(P'\to Q)\to (P\to Q)
  \end{equation*}
  is an equivalence. Thus, the claim follows by observing that for every proposition $Q$ we have the triangle
  \begin{equation*}
    \begin{tikzcd}[column sep=-1em]
      (P'\to Q) \arrow[rr,"\blank\circ h"] \arrow[dr,swap,"\blank\circ {f'}"] & & (P\to Q) \arrow[dl,"\blank\circ f"] \\
      & (A\to Q). & \phantom{(P'\to Q)}
    \end{tikzcd}
  \end{equation*}
\end{proof}

\subsection{Constructing the propositional truncation}
The propositional truncation can be used to construct the image of a map, so we construct that first. We construct the propositional truncation of $A$ via a construction called the \define{join construction}, as the colimit of the sequence of join-powers of $A$
\begin{equation*}
  \begin{tikzcd}
    A \arrow[r] & \join{A}{A} \arrow[r] & \join{A}{(\join{A}{A})} \arrow[r] & \cdots
  \end{tikzcd}
\end{equation*}
The join-powers of $A$ are defined recursively on $n$, by taking\footnote{In this definition, the case $A^{\ast1}\defeq A$ is slightly redundant because we have an equivalence
\begin{equation*}
  \join{A}{\emptyt}\simeq A.
\end{equation*}
Nevertheless, it is nice to have that $A^{\ast 1}\jdeq A$.}
\begin{align*}
  A^{\ast0} & \defeq \emptyt \\
  A^{\ast 1} & \defeq A \\
  A^{\ast(n+2)} & \defeq \join{A}{A^{\ast (n+1)}}.
\end{align*}
We will write $A^{\ast\infty}$ for the colimit of the sequence
\begin{equation*}
  \begin{tikzcd}
    A \arrow[r,"\inr"] & \join{A}{A} \arrow[r,"\inr"] & \join{A}{(\join{A}{A})} \arrow[r,"\inr"] & \cdots.
  \end{tikzcd}
\end{equation*}
The sequential colimit $A^{\ast\infty}$ comes equipped with maps $\inseq_n:A^{\ast (n+1)}\to A^{\ast\infty}$, and we will write
\begin{equation*}
  \eta\defeq\inseq_0:A\to A^{\ast\infty}.
\end{equation*}
Our goal is to show $A^{\ast\infty}$ is a proposition, and that $\eta:A\to A^{\ast\infty}$ satisfies the universal property of the propositional truncation of $A$. Before showing that $A^{\ast\infty}$ is indeed a proposition, let us show in two steps that for any proposition $P$, the map
\begin{equation*}
  (A^{\ast\infty}\to P)\to (A\to P)
\end{equation*}
is indeed an equivalence. 

\begin{lem}\label{lem:extend_join_prop}
Suppose $f:A\to P$, where $A$ is any type and $P$ is a proposition.
Then the precomposition function
\begin{equation*}
\blank\circ\inr:(\join{A}{B}\to P)\to (B\to P)
\end{equation*}
is an equivalence, for any type $B$.
\end{lem}

\begin{proof}
  Since the precomposition function
  \begin{equation*}
    \blank\circ\inr:(\join{A}{B}\to P)\to (B\to P)
  \end{equation*}
  is a map between propositions, it suffices to construct a map
  \begin{equation*}
    (B\to P)\to (\join{A}{B}\to P).
  \end{equation*}
  Let $g:B\to P$. Then the square
  \begin{equation*}
    \begin{tikzcd}
      A\times B \arrow[r,"\proj 2"] \arrow[d,swap,"\proj 1"] & B \arrow[d,"g"] \\
      A \arrow[r,swap,"f"] & P
    \end{tikzcd}
  \end{equation*}
  commutes since $P$ is a proposition. Therefore we obtain a map $\join{A}{B}\to P$ by the universal property of the join.
\end{proof}

\begin{prp}\label{prp:universal-property-brck}
Let $A$ be a type, and let $P$ be a proposition. Then the function
\begin{equation*}
\blank\circ \eta : (A^{\ast\infty}\to P)\to (A\to P)
\end{equation*}
is an equivalence. 
\end{prp}

\begin{proof}
  Since the map
  \begin{equation*}
    \blank\circ \eta : (A^{\ast\infty}\to P)\to (A\to P)
  \end{equation*}
  is a map between propositions, it suffices to construct a map in the converse direction.

  Let $f:A\to P$. First, we show by recursion that there are maps
  \begin{equation*}
    f_n:A^{\ast(n+1)}\to P.
  \end{equation*}
  The map $f_0$ is of course just defined to be $f$. Given $f_n:A^{\ast(n+1)}$ we obtain $f_{n+1}:\join{A}{A^{\ast(n+1)}}\to P$ by \cref{lem:extend_join_prop}. Because $P$ is assumed to be a proposition it is immediate that the maps $f_n$ form a cocone with vertex $P$ on the sequence
  \begin{equation*}
    \begin{tikzcd}
      A \arrow[r,"\inr"] & \join{A}{A} \arrow[r,"\inr"] & \join{A}{(\join{A}{A})} \arrow[r,"\inr"] & \cdots.
    \end{tikzcd}
  \end{equation*}
  From this cocone we obtain the desired map $(A^{\ast\infty}\to P)$.
\end{proof}

\begin{prp}\label{prp:isprop-infjp}
The type $A^{\ast\infty}$ is a proposition for any type $A$.
\end{prp}

\begin{proof}
  By \cref{lem:isprop_eq} it suffices to show that
  \begin{equation*}
    A^{\ast\infty}\to \iscontr(A^{\ast\infty}).
  \end{equation*}
  Since the type $\iscontr(A^{\ast\infty})$ is already known to be a proposition by \cref{ex:isprop_istrunc}, it follows from \cref{prp:universal-property-brck} that it suffices to show that
\begin{equation*}
A\to \iscontr(A^{\ast\infty}).
\end{equation*}

Let $x:A$. To see that $A^{\ast\infty}$ is contractible it suffices by \cref{ex:seqcolim_contr} to show that $\inr:A^{\ast n}\to A^{\ast(n+1)}$ is homotopic to the constant function $\const_{\inl(x)}$. However, we get a homotopy $\const_{\inl(x)}\htpy \inr$ immediately from the path constructor $\glue$.  
\end{proof}

All the definitions are now in place to define the propositional truncation of a type.

\begin{defn}
  For any type $A$ we define the type
  \begin{equation*}
    \trunc{-1}{A}\defeq A^{\ast\infty},
  \end{equation*}
  and we define $\eta:A\to\trunc{-1}{A}$ to be the constructor $\seqin_0$ of the sequential colimit $A^{\ast\infty}$. Often we simply write $\brck{A}$ for $\trunc{-1}{A}$.
\end{defn}

The type $\trunc{-1}{A}$ is a proposition by \cref{prp:isprop-infjp}, and
\begin{equation*}
  \eta:A\to\trunc{-1}{A}
\end{equation*}
satisfies the universal property of propositional truncation by \cref{prp:universal-property-brck}.

\subsection{The image of a map}
The image of a map $f:A\to X$ can now be defined using the propositional truncation:

\begin{defn}
For any map $f:A\to X$ we define the \define{image}\index{image} of $f$ to be the type
\begin{equation*}
\im(f) \defeq \sm{x:X}\brck{\fib{f}{x}}.
\end{equation*}
Furthermore, we define
\begin{enumerate}
\item The \define{image inclusion}
  \begin{equation*}
    i_f:\im(f)\to X
  \end{equation*}
  to be the projection $\proj 1$.
\item The map
  \begin{equation*}
    q_f:A\to\im(f)
  \end{equation*}
  to be the map given by $q_f(x)\defeq(f(x),\eta(x,\refl{f(x)}))$.
\item The homotopy $I_f:f\htpy i_f\circ q_f$ witnessing that the triangle
  \begin{equation*}
    \begin{tikzcd}[column sep=tiny]
      A \arrow[rr,"q_f"] \arrow[dr,swap,"f"] & & \im(f) \arrow[dl,"i_f"] \\
      \phantom{\im(f)} & X
    \end{tikzcd}
  \end{equation*}
  commutes, to be given by $I_f(x)\defeq\refl{f(x)}$.
\end{enumerate}
\end{defn}

\begin{prp}
  The image inclusion $i_f:\im(f)\to X$ of any map $f:A\to X$ is an embedding.
\end{prp}

\begin{proof}
  The fiber of $i_f$ at $x:X$ is equivalent to the type $\brck{\fib{f}{x}}$. In particular we see that the fibers are propositions, so $i_f$ is an embedding.
\end{proof}

\begin{thm}
  The image inclusion $i_f:\im(f)\to X$ of any map $f:A\to X$ satisfies the universal property of the image inclusion of $f$.
\end{thm}

\begin{proof}
  Consider an embedding $m:B\to X$. Note that we have a commuting square
  \begin{equation*}
    \begin{tikzcd}[column sep=6em]
      \mathrm{hom}_X(i_f,m) \arrow[d] \arrow[r] & \mathrm{hom}_X(f,m) \arrow[d] \\
      \Big(\prd{x:X}\fib{i_f}{x}\to\fib{m}{x}\Big) \arrow[r,swap,"h\mapsto{\lam{x}h_x\circ\varphi_x}"] & \Big(\prd{x:X}\fib{f}{x}\to\fib{m}{x}\Big)
    \end{tikzcd}
  \end{equation*}
  The vertical maps are of the form
  \begin{equation*}
    (h,H) \mapsto \lam{x}{(y,p)}(h(y),\ct{H(y)^{-1}}{p}),
  \end{equation*}
  and they are both equivalences. The map
  \begin{equation*}
    \varphi_x:\fib{f}{x}\to\fib{i_f}{x}
  \end{equation*}
  given by $\varphi_x(a,p)\defeq((h(a),\eta(a,p)),p)$ is a propositional truncation for every $x:X$. Therefore it follows that the map
  \begin{equation*}
    (\fib{i_f}{x}\to\fib{m}{x})\to(\fib{f}{x}\to\fib{m}{x})
  \end{equation*}
  is an equivalence, for every $x:X$. Thus we conclude that the bottom map in the above square is an equivalence, which implies that the top map is an equivalence. 
\end{proof}



\subsection{Surjective maps}

Another application of the propositional truncation is the notion of surjective map.

\begin{defn}
A map $f:A\to B$ is said to be \define{surjective} if there is a term of type
\begin{equation*}
\issurj(f)\defeq \prd{y:B}\brck{\fib{f}{b}}.
\end{equation*}
\end{defn}

\begin{eg}
Any equivalence is a surjective map, and so is any map that has a section (those are sometimes called \define{split epimorphisms}). Other examples include the base point inclusion $\unit\to\sphere{n}$ for any $n\geq 1$. 
\end{eg}

\begin{prp}
  Consider a map $f:A\to B$. Then the following are equivalent:
  \begin{enumerate}
  \item The map $f:A\to B$ is surjective.
  \item For any family $P$ of propositions over $B$, the precomposition map
    \begin{equation*}
      \blank\circ f : \Big(\prd{y:B}P(y)\Big)\to\Big(\prd{x:A}P(f(x))\Big)
    \end{equation*}
    is an equivalence.
  \end{enumerate}
\end{prp}

\begin{proof}
  Suppose first that $f$ is surjective, and consider the commuting square
  \begin{equation*}
    \begin{tikzcd}[column sep=6em]
      \Big(\prd{y:B}P(y)\Big) \arrow[r,"\blank\circ f"] \arrow[d,swap,"h\mapsto\lam{y}\const_{h(y)}"] & \Big(\prd{x:A}P(f(x))\Big)  \\
      \Big(\prd{y:B}\brck{\fib{f}{y}}\to P(y)\Big) \arrow[r,swap,"h\mapsto\lam{y}h(y)\circ\eta"] & \Big(\prd{y:B}\fib{f}{y}\to P(y)\Big) \arrow[u,swap,"{h\mapsto\lam{x}h(f(x),(x,\refl{f(x)}))}"]
    \end{tikzcd}
  \end{equation*}
  In this square, the bottom map is an equivalence by the universal property of the propositional truncation of $\fib{f}{y}$. The map on the right is also easily seen to be an equivalence. Furthermore, the map on the left is an equivalence by the assumption that $f$ is surjective, from which it follows that the types $\brck{\fib{f}{y}}$ are contractible. Therefore it follows that the top map is an equivalence, which completes the proof that (i) implies (ii).

  For the converse, it follows immediately from the assumption (ii) that
  \begin{equation*}
    \blank\circ f : \Big(\prd{y:B}\brck{\fib{f}{y}}\Big)\to\Big(\prd{x:A}\brck{\fib{f}{f(x)}}\Big)
  \end{equation*}
  is an equivalence. Hence it suffices to construct a term of type $\brck{\fib{f}{f(x)}}$ for each $x:A$. This is easy, because we have
  \begin{equation*}
    \eta(x,\refl{f(x)}):\brck{\fib{f}{f(x)}}.\qedhere.
  \end{equation*}
\end{proof}

\begin{thm}\label{thm:surjective}
Consider a commuting triangle
\begin{equation*}
\begin{tikzcd}[column sep=tiny]
A \arrow[rr,"q"] \arrow[dr,swap,"f"] & & B \arrow[dl,"m"] \\
& X
\end{tikzcd}
\end{equation*}
in which $m$ is an embedding. Then $m$ satisfies the universal property of the image inclusion of $f$ if and only if $q:A\to B$ is surjective.
\end{thm}

\begin{proof}
  First assume that $m$ satisfies the universal property of the image inclusion of $f$, and consider the composite function
  \begin{equation*}
    \begin{tikzcd}
      \Big(\sm{y:B}\brck{\fib{q}{y}}\Big) \arrow[r,"\proj 1"] & B \arrow[r,"m"] & X.
    \end{tikzcd}
  \end{equation*}
  Note that $m\circ\proj 1$ is a composition of embeddings, so it is an embedding. By the universal property of $m$ we have a commuting triangle
  \begin{equation*}
    \begin{tikzcd}[column sep=0]
      B \arrow[dr,swap,"m"] \arrow[rr,densely dotted,"h"] & & \sm{y:B}\brck{\fib{q}{y}} \arrow[dl,"m\circ\proj 1"] \\
      \phantom{\sm{y:B}\brck{\fib{q}{y}}} & X.
    \end{tikzcd}
  \end{equation*}
  Now note that $\proj 1\circ h$ is a map such that $m\circ (\proj 1\circ h)\htpy m$. 
\end{proof}

\begin{thm}
Let $f:A\to B$ be a map. The following are equivalent:
\begin{enumerate}
\item $f$ is an equivalence.
\item $f$ is both surjective and an embedding.
\end{enumerate}
\end{thm}

\begin{thm}
  Every map factors uniquely as a surjective map followed by an embedding.
\end{thm}

\subsection{Logic in type theory}
Note that, given a family of propositions $P$ over a type $A$, the type $\sm{a:A}P(a)$ isn't necessarily a proposition. Instead, we think of $\sm{a:A}P(a)$ of the \emph{subtype} of $A$ containing the terms that satisfies $P$. Using the propositional truncation we can assert that there \emph{exists} a term in $A$ that satisfies $P$ without requiring one to construct it. 

\begin{defn}
Let $P:A\to \prop$ be a family of propositions over a type $A$. Then we define
\begin{equation*}
\exists_{(a:A)}P(a)\defeq \brck{\sm{a:A}P(a)}.
\end{equation*}
\end{defn}

Similarly, we can define the disjunction of two propositions $P$ and $Q$ to be the \emph{proposition} $\brck{P+Q}$, which clearly satisfies the universal property of disjunction\footnote{Alternatively, we have shown in \cref{ex:join_propositions} that the join $\join{P}{Q}$ also is a proposition that satisfies the universal property of disjunction.}. In \cref{table:logic} we give an overview of the logical connectives on propositions.

\begin{table}
\caption{\label{table:logic}Logic in type theory}
\begin{center}
\begin{tabular}{ll}
\toprule
\emph{Logical connective} & \emph{Interpretation in HoTT} \\
\midrule
$\top$ & $\unit$ \\
$\bot$ & $\emptyt$ \\
$P\land Q$ & $P\times Q$ \\
$P\lor Q$ & $\brck{P+Q}$ \\
$P\to Q$ & $P\to Q$ \\
$P\leftrightarrow Q$ & $\eqv{P}{Q}$ \\
$\neg P$ & $P\to\emptyt$ \\
$\forall x.P(x)$ & $\prd{x:A}P(x)$ \\
$\exists x.P(x)$ & $\brck{\sm{x:A}P(x)}$ \\
$\exists! x.P(x)$ & $\iscontr(\sm{x:A}P(x))$ \\
\bottomrule
\end{tabular}
\end{center}
\end{table}

\begin{exercises}
  \exercise Show that if $f:A\to X$ is an embedding, then $f$ itself satisfies the universal property of the image inclusion of $f$.
  \exercise Show that
  \begin{equation*}
    \eqv{\brck{A}}{\prd{P:\prop}(A\to P)\to P}
  \end{equation*}
  for any type $A:\UU$. This is called the \define{impredicative encoding} of the propositional truncation.
  \exercise For any $B:A\to\UU$, construct an equivalence
  \begin{equation*}
    \eqv{\Big(\exists_{(a:A)}\brck{B(a)}\Big)}{\brck{\sm{a:A}B(a)}}
  \end{equation*}
  %\exercise \label{also}(Mart\'in Escard\'o) For any two propositions $P$ and $Q$, define
  %\begin{equation*}
  %P\boxplus Q \defeq ((P\to Q)\to Q)\times ((Q\to P)\to P).
  %\end{equation*}
  %\begin{subexenum}
  %\item Show that $P\lor Q\to P\boxplus Q$ and $P\boxplus Q\to\neg(\neg P\land \neg Q)$.
  %\end{subexenum}
  %\item \label{ex:brck_comp} Formulate the computation rule corresponding to the path constructor $\mu$. That is, compute the type of $\apd{\rec{\brck{\blank}}(f,g)}{\mu(x,y)}$, and find a canonical element in it.
  \exercise Let
  \begin{tikzcd}
    P_0 \arrow[r] & P_1 \arrow[r] & P_2 \arrow[r] & \cdots
  \end{tikzcd}
  be a sequence of propositions. Show that
  \begin{equation*}
    \eqv{\colim_n(P_n)}{\exists_{(n:\N)} P_n}.
  \end{equation*}
  \exercise Show that the relation $x,y\mapsto\brck{x=y}$ is an equivalence relation, on any type.
  %\exercise Let $f:A\to X$ be a map. Construct an equivalence
  %\begin{equation*}
  %\eqv{\Big(\sm{y:\mathsf{join\usc{}power}_X(n,A)}f(x)=f^{\ast n}(y)\Big)}{\Big(\sm{y:A}f(x)=f(y)\Big)^{\ast n}}
  %\end{equation*}
  %for any $x:A$.
  \exercise Let $f:A\to B$ be a map. Show that the following are equivalent:
  \begin{enumerate}
  \item The commuting square
    \begin{equation*}
      \begin{tikzcd}
        A \arrow[d,swap,"f"] \arrow[r] & \brck{A} \arrow[d,"\brck{f}"] \\
        B \arrow[r] & \brck{B}.
      \end{tikzcd}
    \end{equation*}
    is a pullback square.
  \item There is a term of type $A\to\isequiv(f)$.
  \item The commuting square
    \begin{equation*}
      \begin{tikzcd}
        A\times A \arrow[r,"f\times f"] \arrow[d,swap,"\proj 1"] & B \times B \arrow[d,"\proj 1"] \\
        A \arrow[r,swap,"f"] & B
      \end{tikzcd}
    \end{equation*}
    is a pullback square. 
  \end{enumerate}
  \exercise Consider a pullback square
  \begin{equation*}
    \begin{tikzcd}
      A' \arrow[d,swap,"{f'}"] \arrow[r,"p"] & A \arrow[d,"f"] \\
      B' \arrow[r,swap,"q"] & B,
    \end{tikzcd}
  \end{equation*}
  in which $q:B'\to B$ is surjective. Show that if $f':A'\to B'$ is an embedding, then so is $f:A\to B$.
  \exercise Show that a type $A$ is a proposition if and only if the map $\inl:A\to \join{A}{A}$ is an equivalence.
  \exercise Show that $\inl:\brck{A}\to \join{\brck{A}}{A}$ is an equivalence for any type $A$.
  \exercise Consider a family of diagrams of the form
  \begin{equation*}
    \begin{tikzcd}
      A_i \arrow[r] \arrow[d,swap,"{f_i}"] &
      C \arrow[r] \arrow[d,"g"] & X \arrow[d,"h"] \\
      B_i \arrow[r] & D \arrow[r] & Y 
    \end{tikzcd}
  \end{equation*}
  indexed by $i:I$, in which the left squares are pullback squares,
  and assume that the induced map
  \begin{equation*}
    \Big(\sm{i:I}B_i\Big)\to D
  \end{equation*}
  is surjective. Show that the following are equivalent:
  \begin{enumerate}
  \item For each $i:I$ the outer rectangle is a pullback square.
  \item The right square is a pullback square.
  \end{enumerate}
  Hint: By \cref{thm:descent-Sigma} it suffices to prove this equivalence for a single diagram of the form
  \begin{equation*}
    \begin{tikzcd}
      A \arrow[r] \arrow[d,swap,"{f}"] &
      C \arrow[r] \arrow[d,swap,"g"] & X \arrow[d,"h"] \\
      B \arrow[r] & D \arrow[r] & Y 
    \end{tikzcd}
  \end{equation*}
  where the map $B \to D$ is assumed to be surjective.
  \exercise
    \begin{subexenum}
    \item Consider a map $f:A\to B$. Show that the following are equivalent:
      \begin{enumerate}
      \item The map $f$ is surjective.
      \item For every set $C$, the precomposition function
        \begin{equation*}
          \blank\circ f:(B\to C)\to (A\to C)
        \end{equation*}
        is an embedding.
      \end{enumerate}
    Hint: To show that (ii) implies (i), use the assumption with the set $C\jdeq\prop$.
    \item Give an example of a surjective map $f:A\to B$ and a type $C$ which is not a set, such that the precomposition function
              \begin{equation*}
          \blank\circ f:(B\to C)\to (A\to C)
        \end{equation*}
        is not an embedding.
    \end{subexenum}
\end{exercises}

\endinput

\begin{thm}
Consider a commuting triangle
\begin{equation*}
\begin{tikzcd}[column sep=small]
A \arrow[rr,"i"] \arrow[dr,swap,"f"] & & B \arrow[dl,"m"] \\
& X
\end{tikzcd}
\end{equation*}
with $I:f\htpy m\circ i$, where $m$ is an embedding. The following are equivalent:
\begin{enumerate}
\item $m$ satisfies the universal property of the image of $f$.
\item for each $x:X$, the proposition $\fib{m}{x}$ satisfies the universal property of the propositional truncation of $\fib{f}{x}$.
\end{enumerate}
\end{thm}
