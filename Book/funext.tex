\section{Function extensionality}
\label{chap:funext}

\subsection{Equivalent forms of function extensionality}
\begin{defn}
  The \define{axiom of function extensionality}\index{function extensionality}\index{identity type!of a Pi-type@{of a $\Pi$-type}}\index{extensionality principle!for functions} asserts that for any type family $B$ over $A$, and any two dependent functions $f,g:\prd{x:A}B(x)$, the canonical map\index{htpy-eq@{$\htpyeq$}}\index{htpy-eq@{$\htpyeq$}!is an equivalence}\index{is an equivalence!htpy-eq@{$\htpyeq$}}
  \begin{equation*}
    \mathsf{htpy\usc{}eq}:(f=g)\to (f\htpy g)
  \end{equation*}
  that sends $\refl{f}$ to $\mathsf{htpy\usc{}refl}_{f}$ is an equivalence. We will write $\mathsf{eq\usc{}htpy}$\index{eq-htpy@{$\mathsf{eq\usc{}htpy}$}} for its inverse, if it is assumed to exist.
\end{defn}

In other words, the axiom of function extensionality asserts that for any two dependent functions $f,g:\prd{x:A}B(x)$, the type of identifications $f=g$ is equivalent to the type of homotopies $f\htpy g$ from $f$ to $g$. By the fundamental theorem of identity types (\cref{thm:id_fundamental})\index{fundamental theorem of identity types} there are three equivalent ways of asserting function extensionality. In the following theorem we state one further equivalent condition.

\begin{thm}\label{thm:funext_wkfunext}
The following are equivalent:\index{function extensionality}
\begin{enumerate}
\item The axion of function extensionality.
\item For any type family $B$ over $A$ and any dependent function $f:\prd{x:A}B(x)$, the total space
  \begin{equation*}
    \sm{g:\prd{x:A}B(x)}f\htpy g
  \end{equation*}
  is contractible.
\item
  The principle of \define{homotopy induction}\index{homotopy induction}\index{induction principle!for homotopies}:
  for any type family $B$ over $A$, any dependent function $f:\prd{x:A}B(x)$, and any family of types $P(g,H)$ indexed by $g:\prd{x:A}B(x)$ and $H:f\htpy g$, the evaluation function
\begin{equation*}
\Big(\prd{g:\prd{x:A}B(x)}{H:f\htpy g}P(g,H)\Big)\to P(f,\mathsf{htpy\usc{}refl}_f)
\end{equation*}
given by $s\mapsto s(f,\mathsf{htpy\usc{}refl}_f)$ has a section.
\item The \define{weak function extensionality principle}\index{weak function extensionality} holds: For every type family $B$ over $A$ one has\index{contractible!weak function extensionality}
\begin{equation*}
\Big(\prd{x:A}\iscontr(B(x))\Big)\to\iscontr\Big(\prd{x:A}B(x)\Big).
\end{equation*}
\end{enumerate}
\end{thm}

\begin{proof}
The fact that function extensionality is equivalent to (ii) and (iii) follows directly from \cref{thm:id_fundamental}.

To show that function extensionality implies weak function extensionality, suppose that each $B(a)$ is contractible with center of contraction $c(a)$ and contraction $C_a:\prd{y:B(a)}c(a)=y$. Then we take $c\defeq \lam{a}c(a)$ to be the center of contraction of $\prd{x:A}B(x)$. To construct the contraction we have to define a term of type
\begin{equation*}
\prd{f:\prd{x:A}B(x)}c=f.
\end{equation*}
Let $f:\prd{x:A}B(x)$. By function extensionality we have a map $(c\htpy f)\to (c=f)$, so it suffices to construct a term of type $c\htpy f$. Here we take $\lam{a}C_a(f(a))$. This completes the proof that function extensionality implies weak function extensionality.

In the remaining part of the proof, we will show that weak function extensionality implies that the type
\begin{equation*}
\sm{g:\prd{x:A}B(x)}f\htpy g
\end{equation*}
is contractible for any $f:\prd{x:A}B(x)$. In order to do this, we first note that we have a section-retraction pair
\begin{equation*}
  \begin{tikzcd}[column sep=small]
    \Big(\sm{g:\prd{x:A}B(x)}f\htpy g\Big) \arrow[r,"i"] & \Big(\prd{x:A}\sm{b:B(x)}f(x)=b\Big) \arrow[r,"r"] & \Big(\sm{g:\prd{x:A}B(x)}f\htpy g\Big).
  \end{tikzcd}
\end{equation*}
Here we have the functions
\begin{align*}
i & \defeq \lam{(g,H)}{x}(g(x),H(x)) \\
r & \defeq \lam{p}\pairr{\lam{x}\proj 1(p(x)),\lam{x}\proj 2(p(x))}.
\end{align*}
Their composite is homotopic to the identity function by the computation rule for $\Sigma$-types and the $\eta$-rule for $\Pi$-types:
\begin{align*}
r(i(g,H)) & \jdeq r(\lam{x}\pairr{g(x),H(x)}) \\
& \jdeq \pairr{\lam{x}g(x),\lam{x}H(x)} \\
& \jdeq \pairr{g,H}.
\end{align*}
Now we observe that the type $\prd{x:A}\sm{b:B(x)}f(x)=b$ is a product of contractible types, so it is contractible by our assumption of the weak function extensionality principle. The claim therefore follows, since retracts of contractible types are contractible by \cref{ex:contr_retr}\index{contractible!retract of}.
\end{proof}

For the remainder of this chapter we will assume that the function extensionality axiom holds\index{axiom!function extensionality}. In \cref{thm:funext-univalence} we will derive function extensionality from the univalence axiom. 

As a first application of the function extensionality axiom we generalize the weak function extensionality axiom to $k$-types.

\begin{thm}\label{thm:trunc_pi}\index{k-type@{$k$-type}}
Assume function extensionality. Then for any type family $B$ over $A$ one has\index{truncated type!closed under Pi@{closed under $\Pi$}}
\begin{equation*}
\Big(\prd{x:A}\istrunc{k}(B(x))\Big)\to \istrunc{k}\Big(\prd{x:A}B(x)\Big).
\end{equation*}
\end{thm}

\begin{proof}
The theorem is proven by induction on $k\geq -2$. The base case is just the weak function extensionality principle\index{weak function extensionality}, which was shown to follow from function extensionality in \cref{thm:funext_wkfunext}.

For the inductive hypothesis, assume that the $k$-types are closed under dependent function types. Assume that $B$ is a family of $(k+1)$-types. By function extensionality, the type $f=g$ is equivalent to $f\htpy g$ for any two dependent functions $f,g:\prd{x:A}B(x)$. Now observe that $f\htpy g$ is a dependent product of $k$-types, and therefore it is an $k$-type by our inductive hypotheses. Therefore, it follows by \cref{thm:ktype_eqv} that $f=g$ is an $k$-type, and hence that $\prd{x:A}B(x)$ is an $(k+1)$-type.
\end{proof}

\begin{cor}\label{cor:funtype_trunc}\index{truncated type!closed under exponentials}
Suppose $B$ is a $k$-type. Then $A\to B$ is also a $k$-type, for any type $A$.
\end{cor}

\subsection{The type theoretic principle of choice}

The type theoretic principle of choice\index{type theoretic choice} asserts that $\Pi$ distributes over $\Sigma$\index{distributivity!of Pi over Sigma@{of $\Pi$ over $\Sigma$}}. More precisely, it asserts that the canonical map\index{choice@{$\mathsf{choice}$}}
\begin{equation*}
  \mathsf{choice}:\Big(\prd{x:A}\sm{y:B(x)}C(x,y)\Big)\to \Big(\sm{f:\prd{x:A}B(x)}\prd{x:A}C(x,f(x))\Big)
\end{equation*}
given by $\lam{h}(\proj 1(h(x)),\proj 2(h(x)))$, is an equivalence. In order to see this as a principle of choice, one can view the left hand side as the type of functions $h$ that pick for every $x:A$ a term $y:B(x)$ equipped with a term of type $C(x,y)$. The function $\mathsf{choice}$ then constructs a dependent function $f:\prd{x:A}B(x)$ equipped with a term of type $\prd{x:A}C(x,f(x))$. In this section we show that the map $\mathsf{choice}$ is an equivalence, and we use this to characterize the identity of any dependent function type $\prd{x:A}B(x)$ in terms of any characterization of the identity types of the individual types $B(x)$.

\begin{thm}\label{thm:choice}
Consider a family of types $C(x,y)$ indexed by $x:A$ and $y:B(x)$. Then the map
\begin{equation*}
  \mathsf{choice}:\Big(\prd{x:A}\sm{y:B(x)}C(x,y)\Big)\to \Big(\sm{f:\prd{x:A}B(x)}\prd{x:A}C(x,f(x))\Big)
\end{equation*}
given by $\lam{h}(\proj 1(h(x)),\proj 2(h(x)))$ is an equivalence.
\end{thm}

\begin{proof}
  We define the map\index{choice-1@{$\mathsf{choice}^{-1}$}}
  \begin{equation*}
    \mathsf{choice}^{-1}:\Big(\sm{f:\prd{x:A}B(x)}\prd{x:A}C(x,f(x))\Big)\to\Big(\prd{x:A}\sm{y:B(x)}C(x,y)\Big)
  \end{equation*}
  by $\lam{(f,g)}{x}(f(x),g(x))$. Then we have to construct homotopies
  \begin{equation*}
    \mathsf{choice}\circ\mathsf{choice}^{-1}\htpy\idfunc,\qquad\text{and}\qquad
    \mathsf{choice}^{-1}\circ\mathsf{choice}\htpy\idfunc.
  \end{equation*}
  For the first homotopy it suffices to construct an identification
  \begin{equation*}
    \mathsf{choice}(\mathsf{choice}^{-1}(f,g))=(f,g)
  \end{equation*}
  for any $f:\prd{x:A}B(x)$ and any $g:\prd{x:A}C(x,f(x))$. We compute the left-hand side as follows:
  \begin{align*}
    \mathsf{choice}(\mathsf{choice}^{-1}(f,g))
    & \jdeq \mathsf{choice}(\lam{x}(f(x),g(x))) \\
    & \jdeq (\lam{x}f(x),\lam{x}g(x)).
  \end{align*}
  By the $\eta$-rule it follows that $f\jdeq \lam{x}f(x)$ and $g\jdeq\lam{x}g(x)$. Therefore we have the identification
  \begin{equation*}
    \refl{(f,g)}:\mathsf{choice}(\mathsf{choice}^{-1}(f,g))=(f,g).
  \end{equation*}
  This completes the construction of the first homotopy.

  For the second homotopy we have to construct an identification
  \begin{equation*}
    \mathsf{choice}^{-1}(\mathsf{choice}(h))=h
  \end{equation*}
  for any $h:\prd{x:A}\sm{y:B(x)}C(x,y)$. We compute the left-hand side as follows:
  \begin{align*}
    \mathsf{choice}^{-1}(\mathsf{choice}(h))
    & \jdeq \mathsf{choice}^{-1}(\lam{x}\proj 1(h(x)),(\lam{x}\proj 2(h(x)))) \\
    & \jdeq \lam{x}(\proj 1(h(x)),\proj 2(h(x)))
  \end{align*}
  However, it is \emph{not} the case that $(\proj 1(h(x)),\proj 2(h(x)))\jdeq h(x)$ for any $h:\prd{x:A}\sm{y:B(x)}C(x,y)$. Nevertheless, we have the identification
  \begin{equation*}
    \mathsf{eq\usc{}pair}(\refl{},\refl{}):(\proj 1(h(x)),\proj 2(h(x)))= h(x).
  \end{equation*}
  Therefore we obtain the required homotopy by function extensionality:
  \begin{equation*}
    \lam{h}\mathsf{eq\usc{}htpy}(\lam{x}\mathsf{eq\usc{}pair}(\refl{\proj 1(h(x))},\refl{\proj 2(h(x))})):\mathsf{choice}^{-1}\circ\mathsf{choice}\htpy\idfunc.\qedhere
  \end{equation*}
\end{proof}

\begin{cor}
For type $A$ and any type family $C$ over $B$, the map
\begin{equation*}
\Big(\sm{f:A\to B} \prd{x:A}C(f(x))\Big)\to\Big(A\to\sm{y:B}C(x)\Big)
\end{equation*}
given by $\lam{(f,g)}{x}(f(x),g(x))$ is an equivalence.
\end{cor}

\begin{rmk}
  The type theoretic choice principle can be used to derive the binomial theorem\index{binomial theorem}. We give an informal argument of how this goes. Recall that the bimomial theorem asserts that
  \begin{equation*}
    (n+m)^k=\sum_{l=0}^k\binom{k}{l}n^l m^{k-l}
  \end{equation*}
  for any three natural numbers $k,m,n$.

  Consider the types $A\defeq\mathsf{Fin}(k)$\index{Fin@{$\mathsf{Fin}$}}, $B\defeq\mathsf{Fin}(n)$ and $C\defeq\mathsf{Fin}(m)$. Then we can define the type family $P:\bool\to\UU$ given by
  \begin{align*}
    P(\btrue) & \defeq B \\
    P(\bfalse) & \defeq C.
  \end{align*}
  Now, the type theoretic principle of choice gives us an equivalence
  \begin{equation*}
    \Big(\prd{x:A}\sm{t:\bool}P(t)\Big)\simeq \Big(\sm{f:A\to\bool}\prd{x:A}P(f(x))\Big).
  \end{equation*}
  Now we note that the type $(f(x)=1)+(f(x)=0)$ is contractible for any $f:A\to\bool$ and $x:A$. Therefore we have equivalences
  \begin{align*}
    \sm{f:A\to\bool}\prd{x:A}P(f(x) & \simeq
    \sm{f:A\to\bool}\prd{x:A}{t:(f(x)=1)+(f(x)=0)}P(f(x)) \\
    & \simeq \sm{f:A\to\bool}(\fib{f}{1}\to B)\times (\fib{f}{0}\to C)
  \end{align*}
  Now we note that, because there are $\binom{k}{l}$\index{binomial coefficient} ways to choose a subset of $l$ elements of $A$, there are
  \begin{equation*}
    \sum_{l=0}^k\binom{k}{l}n^l m^{k-l}
  \end{equation*}
  elements in the above type.
\end{rmk}

\subsection{Universal properties}
The function extensionality principle allows us to prove \emph{universal properties}. Universal properties are characterizations of all maps out of or into a given type, so they are very important. Among other applications, universal properties characterize a type up to equivalence. In the following theorem we prove the universal property of dependent pair types.

\begin{thm}\index{universal property!Sigma-types@{$\Sigma$-types}}\index{Sigma-type@{$\Sigma$-type}!universal property}
Let $B$ be a type family over $A$, and let $X$ be a type. Then the map\index{ev-pair@{$\mathsf{ev\usc{}pair}$}}
\begin{equation*}
\mathsf{ev\usc{}pair}:\Big(\Big(\sm{x:A}B(x)\Big)\to X\Big)\to \Big(\prd{x:A}(B(x)\to X)\Big)
\end{equation*}
given by $f\mapsto\lam{a}{b}f(a,b)$ is an equivalence.
\end{thm}

\begin{proof}
The map in the converse direction is simply
\begin{equation*}
\ind{\Sigma} : \Big(\prd{x:A}(B(x)\to X)\Big)\to \Big(\Big(\sm{x:A}B(x)\Big)\to X\Big).
\end{equation*}
By the computation rules for $\Sigma$-types we have
\begin{equation*}
\lam{f}\refl{f}:\mathsf{ev\usc{}pair}\circ\ind{\Sigma}\htpy\idfunc
\end{equation*}

To show that $\ind{\Sigma}\circ\mathsf{ev\usc{}pair}\htpy\idfunc$ we will also apply function extensionality. Thus, it suffices to show that $\ind{\Sigma}(\lam{x}{y}f((x,y)))=f$. We apply function extensionality again, so it suffices to show that
\begin{equation*}
\prd{t:\sm{x:A}B(x)}\ind{\Sigma}\big(\lam{x}{y}f((x,y))\big)(t)=f(t).
\end{equation*}
We obtain this homotopy by another application of $\Sigma$-induction. 
\end{proof}

\begin{cor}\label{cor:times_up_out}\index{universal property!cartesian product}\index{cartesian product!universal property}
Let $A$, $B$, and $X$ be types. Then the map\index{ev-pair@{$\mathsf{ev\usc{}pair}$}}
\begin{equation*}
\mathsf{ev\usc{}pair}: (A\times B \to X)\to (A\to (B\to X))
\end{equation*}
given by $f\mapsto\lam{a}{b}f((a,b))$ is an equivalence.
\end{cor}

The universal property of identity types is sometimes called the \emph{type theoretical Yoneda lemma}\index{Yoneda lemma (type theoretical)}: families of maps out of the identity type are uniquely determined by their action on the reflexivity identification.

\begin{thm}\label{thm:yoneda}\index{universal property!identity type}\index{identity type!universal property}
Let $B$ be a type family over $A$, and let $a:A$. Then the map\index{ev-refl@{$\mathsf{ev\usc{}refl}$}}
\begin{equation*}
\mathsf{ev\usc{}refl}:\Big(\prd{x:A} (a=x)\to B(x)\Big)\to B(a)
\end{equation*}
given by $\lam{f} f(a,\refl{a})$ is an equivalence. 
\end{thm}

\begin{proof}
The inverse $\varphi$ is defined by path induction, taking $b:B(a)$ to the function $f$ satisfying $f(a,\refl{a})\jdeq b$. It is immediate that $\mathsf{ev\usc{}refl}\circ\varphi\htpy \idfunc$.

To see that $\varphi\circ \mathsf{ev\usc{}refl}\htpy\idfunc$, let $f:\prd{x:A}(a=x)\to B(x)$. To show that $\varphi(f(a,\refl{a}))=f$ we use function extensionality (twice), so it suffices to show that
\begin{equation*}
\prd{x:A}{p:a=x} \varphi(f(a,\refl{a}),x,p)=f(x,p).
\end{equation*}
This follows by path induction on $p$, since $\varphi(f(a,\refl{a}),a,\refl{a})\jdeq f(a,\refl{a})$.
\end{proof}

\subsection{Composing with equivalences}

We show in this section that a map $f:A\to B$ is an equivalence if and only if for any type $X$ the precomposition map\index{precomposition map}
\begin{equation*}
\blank\circ f: (B\to X)\to (A\to X)
\end{equation*}
is an equivalence. Moreover, we will show in \cref{ex:equiv_precomp} that the `dependent version' of this statement also holds: a map $f:A\to B$ is an equivalence if and only if for any type family $P$ over $B$, the precomposition map
\begin{equation*}
\blank\circ f: \Big(\prd{y:B}P(y)\Big)\to\Big(\prd{x:A}P(f(x))\Big)
\end{equation*}
is an equivalence.

\begin{thm}\label{ex:equiv_precomp}\index{equivalence!precomposition}
For any map $f:A\to B$, the following are equivalent:
\begin{enumerate}
\item $f$ is an equivalence.
\item For any type family $P$ over $B$ the map
\begin{equation*}
\Big(\prd{y:B}P(y)\Big)\to\Big(\prd{x:A}P(f(x))\Big)
\end{equation*}
given by $h\mapsto h\circ f$ is an equivalence.
\item For any type $X$ the map
\begin{equation*}
(B\to X)\to (A\to X)
\end{equation*}
given by $g\mapsto g\circ f$ is an equivalence. 
\end{enumerate}
\end{thm}

\begin{proof}
To show that (i) implies (ii), we first recall from \cref{lem:coherently-invertible} that any equivalence is also coherently invertible\index{coherently invertible}. Therefore $f$ comes equipped with
\begin{align*}
g & : B \to A\\
G & : f\circ g \htpy \idfunc[B] \\
H & : g\circ f \htpy \idfunc[A] \\
K & : G\cdot f \htpy f\cdot H.
\end{align*}
Then we define the inverse of $\blank\circ f$ to be the map
\begin{equation*}
\varphi:\Big(\prd{x:A}P(f(x))\Big)\to\Big(\prd{y:B}P(y)\Big)
\end{equation*}
given by $h\mapsto \lam{y}\mathsf{tr}_P(G(y),h(g(y)))$. 

To see that $\varphi$ is a section of $\blank\circ f$, let $h:\prd{x:A}P(f(x))$. By function extensionality it suffices to construct a homotopy $\varphi(h)\circ f\htpy h$. In other words, we have to show that
\begin{equation*}
\mathsf{tr}_P(G(f(x)),h(g(f(x)))=h(x)
\end{equation*}
for any $x:A$. Now we use the additional homotopy $K$ from our assumption that $f$ is coherently invertible. Since we have $K(x):G(f(x))=\ap{f}{H(x)}$ it suffices to show that
\begin{equation*}
\mathsf{tr}_P(\ap{f}{H(x)},hgf(x))=h(x).
\end{equation*}
A simple path-induction argument yields that
\begin{equation*}
\mathsf{tr}_P(\ap{f}{p})\htpy \mathsf{tr}_{P\circ f}(p)
\end{equation*}
for any path $p:x=y$ in $A$, so it suffices to construct an identification
\begin{equation*}
\mathsf{tr}_{P\circ f}(H(x),hgf(x))=h(x).
\end{equation*}
We have such an identification by $\apd{h}{H(x)}$.

To see that $\varphi$ is a retraction of $\blank\circ f$, let $h:\prd{y:B}P(y)$. By function extensionality it suffices to construct a homotopy $\varphi(h\circ f)\htpy h$. In other words, we have to show that
\begin{equation*}
\mathsf{tr}_P(G(y),hfg(y))=h(y)
\end{equation*}
for any $y:B$. We have such an identification by $\apd{h}{G(y)}$. This completes the proof that (i) implies (ii).

Note that (iii) is an immediate consequence of (ii), since we can just choose $P$ to be the constant family $X$.

It remains to show that (iii) implies (i). Suppose that
\begin{equation*}
\blank\circ f:(B\to X)\to (A\to X)
\end{equation*}
is an equivalence for every type $X$. Then its fibers are contractible by \cref{thm:contr_equiv}. In particular, choosing $X\jdeq A$ we see that the fiber
\begin{equation*}
\fib{\blank\circ f}{\idfunc[A]}\jdeq \sm{h:B\to A}h\circ f=\idfunc[A]
\end{equation*}
is contractible. Thus we obtain a function $h:B\to A$ and a homotopy $H:h\circ f\htpy\idfunc[A]$ showing that $h$ is a retraction of $f$. We will show that $h$ is also a section of $f$. To see this, we use that the fiber
\begin{equation*}
\fib{\blank\circ f}{f}\jdeq \sm{i:B\to B} i\circ f=f
\end{equation*}
is contractible (choosing $X\jdeq B$). 
Of course we have $(\idfunc[B],\refl{f})$ in this fiber. However we claim that there also is an identification $p:(f\circ h)\circ f=f$, showing that $(f\circ h,p)$ is in this fiber, because
\begin{align*}
(f\circ h)\circ f & \jdeq f\circ (h\circ f) \\
& = f\circ \idfunc[A] \\
& \jdeq f
\end{align*}
Now we conclude by the contractibility of the fiber that $(\idfunc[B],\refl{f})=(f\circ h,p)$. In particular we obtain that $\idfunc[B]=f\circ h$, showing that $h$ is a section of $f$.
\end{proof}

\begin{exercises}
\item Show that the functions\index{htpy-inv@{$\htpyinv$}!is an equivalence}\index{htpy-concat@{$\htpyconcat$}!is a family of equivalences}\index{htpy-concat'@{$\htpyconcat'$}!is a family of equivalences}\index{is an equivalence!htpy-inv@{$\htpyinv$}}\index{is an equivalence!htpy-concat(H)@{$\htpyconcat(H)$}}\index{is an equivalence!htpy-concat'(K)@{$\htpyconcat'(K)$}}
\begin{align*}
\htpyinv & : (f \htpy g) \to (g \htpy f) \\
\htpyconcat(H) & : (g \htpy h) \to (f \htpy h) \\
\htpyconcat'(K) & : (f \htpy g) \to (f \htpy h)
\end{align*}
are equivalences for every $f,g,h : \prd{x:A}B(x)$. Here, $\htpyconcat'(K)$ is the function defined by $H\mapsto \ct{H}{K}$.
\item \label{ex:isprop_istrunc}
\begin{subexenum}
\item Show that for any type $A$ the type $\iscontr(A)$ is a proposition\index{is-cont(A)r@{$\iscontr(A)$}!is a proposition}\index{is contractible!is a property}. %There's an easy proof using double singleton induction. This is a nice application of weak funext.
\item Show that for any type $A$ and any $k\geq-2$, the type $\istrunc{k}(A)$ is a proposition.\index{istrunc@{$\istrunc{k}$}!is a proposition}
\end{subexenum}
\item \label{lem:postcomp_equiv}
Let $f:X\to Y$ be a map. Show that the following are equivalent:
\begin{enumerate}
\item $f$ is an equivalence.\index{equivalence!postcomposition}
\item The map $f\circ\blank : X^A\to Y^A$ is an equivalence for every type $A$.
\end{enumerate}
\item \label{ex:isprop_isequiv}Let $f:A\to B$ be a function.
\begin{subexenum}
\item Show that if $f$ is an equivalence, then the type $\sm{g:B\to A}f\circ g\htpy \idfunc$ of sections of $f$ is contractible.
\item Show that if $f$ is an equivalence, then the type $\sm{h:B\to A}h\circ f\htpy \idfunc$ of retractions of $f$ is contractible.
\item Show that $\isequiv(f)$ is a proposition.\index{isequiv@{$\isequiv$}!is a proposition}
\item Use \cref{ex:prop_equiv,ex:isprop_istrunc} to show that $\isequiv(f)\eqvsym \iscontr(f)$.\index{isequiv@{$\isequiv$}!isequiv iscontr@{$\isequiv(f)\eqvsym\iscontr(f)$}}
\end{subexenum}
Conclude that $\eqv{A}{B}$ is a subtype of $A\to B$, and in particular that the map $\proj 1 : (\eqv{A}{B})\to (A\to B)$ is an embedding.
\item \label{ex:prop_equiv}
\begin{subexenum}
\item \label{ex:equiv-bi-implication}Let $P$ and $Q$ be propositions. Show that\index{bi-implication}
\begin{equation*}
\eqv{(P\leftrightarrow Q)}{(\eqv{P}{Q})}.
\end{equation*}
\item Show that $P$ is a proposition if and only if $P\to P$ is contractible.
\end{subexenum}
\item Show that $\mathsf{path\usc{}split}(f)$\index{path-split!is a proposition} and $\mathsf{is\usc{}coh\usc{}invertible}(f)$\index{coherently invertible!is a proposition} are propositions for any map $f:A\to B$. Conclude that we have equivalences\index{isequiv@{$\isequiv$}!isequiv path-split@{$\isequiv(f)\eqvsym\mathsf{path\usc{}split}(f)$}}\index{isequiv@{$\isequiv$}!isequiv iscohinvertible@{$\isequiv(f)\eqvsym\mathsf{is\usc{}coh\usc{}invertible}(f)$}}
  \begin{equation*}
    \isequiv(f) \eqvsym \mathsf{path\usc{}split}(f) \eqvsym \mathsf{is\usc{}coh\usc{}invertible}(f).
  \end{equation*}
%\item Let $B$ and $C$ be type families over $A$, suppose that $p:\id{a}{a'}$ in $A$, and consider two functions $f:B(a)\to C(a)$ and $g:B(a')\to C(a')$.
%\begin{subexenum}
%\item Show that the square
%\begin{equation*}
%\begin{tikzcd}
%B(a) \arrow[r,"f"] \arrow[d,swap,"\mathsf{tr}_B(p)"] & C(a) \arrow[d,"\mathsf{tr}_C(p)"] \\
%B(a') \arrow[r,swap,"g"] & C(a')
%\end{tikzcd}
%\end{equation*}
%commutes for every homotopy $H:\mathsf{tr}_{B(x)\to C(x)}(p,f)\htpy g$. In other words, construct a function of type
%\begin{equation*} 
%\Big(\mathsf{tr}_{B(x)\to C(x)}(p,f)\htpy g\Big)\to \Big(\mathsf{tr}_C(p)\circ f\htpy g\circ \mathsf{tr}_B(p)\Big)
%\end{equation*}
%\item Show that this map is an equivalence.
%\end{subexenum}
\item \label{ex:idfunc_autohtpy}Construct for any type $A$ an equivalence\index{isinvertible@{$\mathsf{is\usc{}invertible}$}}
\begin{equation*}
\eqv{\mathsf{is\usc{}invertible}(\idfunc[A])}{\Big(\idfunc[A]\htpy\idfunc[A]\Big)}.
\end{equation*}
Note: We will use this fact in \cref{ex:is_invertible_id_S1} to show that there
are types for which $\mathsf{is\usc{}invertible}(\idfunc[A])\not\eqvsym\isequiv(\idfunc[A])$.
\item 
\begin{subexenum}
\item Show that the type\index{universal property!empty type}\index{empty type!universal property}
\begin{equation*}
\prd{t:\emptyt}P(t)
\end{equation*}
is contractible for any $P:\emptyt\to \UU$.
\item Show that for any type $X$ the following are equivalent:
  \begin{enumerate}
  \item the unique map $\emptyt \to X$ is an equivalence.
  \item The type $Y^X$ is contractible for any type $Y$.
  \end{enumerate}
\end{subexenum}
\item Consider two types $A$ and $B$.\index{universal property!coproduct}\index{coproduct!universal property}
\begin{subexenum}
\item Show that the map\index{ev-inl-inr@{$\mathsf{ev\usc{}inl\usc{}inr}$}}
\begin{equation*}
  \mathsf{ev\usc{}inl\usc{}inr}: \Big(\prd{t:A+B}P(t)\Big) \to
  \Big(\prd{x:A}P(\inl(x))\Big)\times\Big(\prd{y:B}P(\inr(y))\Big)
\end{equation*}
given by $f\mapsto (f\circ\inl,f\circ\inr)$ is an equivalence.
\item Show that the following are equivalent for any type $X$ equipped with maps $i:A\to X$ and $j:B\to X$:
  \begin{enumerate}
  \item The map $\mathsf{ind\usc{}coprod}(i,j) :A+B\to X$ is an equivalence.
  \item For any type $Y$, the map
    \begin{equation*}
      \lam{f}(f\circ i,f\circ j):(X\to Y)\to (A\to Y)\times (B \to Y)
    \end{equation*}
    is an equivalence.
  \end{enumerate}
\end{subexenum}
\item 
\begin{subexenum}
\item Show that the map\index{universal property!unit type}\index{unit type!universal property}\index{ev-pt@{$\mathsf{ev\usc{}pt}$}}
\begin{equation*}
\Big(\prd{t:\unit}P(t)\Big)\to P(\ttt)
\end{equation*}
given by $\lam{f}f(\ttt)$ is an equivalence. 
\item Consider a type $X$ equipped with a point $x:X$. Show that the following are equivalent: 
\begin{enumerate}
\item The map $\mathsf{ind\usc{}unit}(x):\unit\to X$ is an equivalence (i.e., $X$ is contractible).
\item For any type $Y$ the map
\begin{equation*}
\lam{f}f(x) : (X\to Y)\to Y
\end{equation*}
is an equivalence.
\end{enumerate}
\end{subexenum}
\item \label{ex:sec_retr}Consider a commuting triangle 
\begin{equation*}
\begin{tikzcd}[column sep=tiny]
A \arrow[rr,"h"] \arrow[dr,swap,"f"] & & B \arrow[dl,"g"] \\
& X
\end{tikzcd}
\end{equation*}
with $H:f\htpy g\circ h$.
\begin{subexenum}
\item Show that if $h$ has a section, then $\mathsf{sec}(g)$ is a retract of $\mathsf{sec}(f)$.
\item Show that if $g$ has a retraction, then $\mathsf{retr}(h)$ is a retract of $\mathsf{sec}(f)$.
\end{subexenum}
\item \label{ex:equiv_pi}Let $e_i:\eqv{A_i}{B_i}$ be an equivalence for every $i:I$. Show that the map
\begin{equation*}
\lam{f}{i}e_i\circ f:\Big(\prd{i:I}A_i\Big)\to\Big(\prd{i:I}B_i\Big)
\end{equation*}
is an equivalence.
\item \label{ex:triangle_fib}Consider a diagram of the form
\begin{equation*}
\begin{tikzcd}[column sep=tiny]
A \arrow[dr,swap,"f"] & & B \arrow[dl,"g"] \\
& X
\end{tikzcd}
\end{equation*}
\begin{subexenum}
\item Show that the type $\sm{h:A\to B} f\htpy g\circ h$ is equivalent to the type of families of maps
\begin{equation*}
\prd{x:X}\fib{f}{x}\to\fib{g}{x}.
\end{equation*}
\item Show that the type $\sm{h:\eqv{A}{B}} f\htpy g\circ h$ is equivalent to the type of families of equivalences
\begin{equation*}
\prd{x:X}\fib{f}{x}\eqvsym\fib{g}{x}.
\end{equation*}
\end{subexenum}
\item \label{ex:sq_fib}Consider a diagram of the form
\begin{equation*}
\begin{tikzcd}
A \arrow[d,swap,"f"] & B \arrow[d,"g"] \\
X \arrow[r,swap,"h"] & Y.
\end{tikzcd}
\end{equation*}
Show that the type $\sm{i:A\to B}h\circ f\htpy g\circ i$ is equivalent to the type of families of maps
\begin{equation*}
\prd{x:X}\fib{f}{x}\to\fib{g}{h(x)}.
\end{equation*}
Note: In \cref{thm:pb_fibequiv_complete} we will characterize the type of families of equivalences $\prd{x:X}\fib{f}{x}\simeq\fib{g}{x}$.
%\item Show that the type $\sm{i:\eqv{A}{B}}h\circ f\htpy g\circ i$ is equivalent to the type of families of equivalences
%\begin{equation*}
%\prd{x:X}\fib{f}{x}\eqvsym\fib{g}{h(x)}.
%\end{equation*}
\item \label{ex:iso_equiv}Let $A$ and $B$ be sets. Show that type type $\eqv{A}{B}$ of equivalences from $A$ to $B$ is equivalent to the type $A\cong B$ of \define{isomorphisms}\index{isomorphism}\index{set!isomorphism} from $A$ to $B$, i.e., the type of quadruples $(f,g,H,K)$ consisting of
\begin{align*}
f & : A\to B \\
g & : B\to A \\
H & : f\circ g = \idfunc[B] \\
K & : g\circ f = \idfunc[A].
\end{align*}
\item \label{ex:pi_sec}Let $B$ be a type family over $A$, and consider the maps
  \begin{align*}
    \proj 1 & : \sm{x:A} B(x)\to A \\
    \proj1 \circ \blank & : \Big(\sm{x:A} B(x)\Big)^A \to A^A.
  \end{align*}
  Construct equivalences
  \begin{equation*}
    \Big(\prd{x:A}B(x)\Big) \eqvsym \mathsf{sec}(\proj 1) \eqvsym \fib{\proj 1 \circ\blank}{\idfunc[A]}.
  \end{equation*}
\item Suppose that $A:I\to \UU$ is a type family over a set $I$ with decidable equality. Show that
  \begin{equation*}
    \Big(\prd{i:I}\iscontr(A_i)\Big)\leftrightarrow \iscontr\Big(\prd{i:I}A_i\Big).
  \end{equation*}
\item Construct equivalences
  \begin{align*}
    \mathsf{Fin}(n^m) & \simeq (\mathsf{Fin}(m)\to\mathsf{Fin}(n)) \\
    \mathsf{Fin}(n!) & \simeq (\mathsf{Fin}(n)\simeq\mathsf{Fin}(n)).
  \end{align*}
\end{exercises}
