% !TEX root = hott_intro.tex

\section{The circle}

We have seen inductive types, in which we describe a type by its constructors and an induction principle that allows us to construct sections of dependent types. Inductive types are freely generated by their constructors, which describe how we can construct their terms. 

However, many familiar constructions in algebra involve the construction of algebras by generators and relations. 
For example, the free abelian group with two generators is described as the group with generators $x$ and $y$, and the relation $xy=yx$. 

In this chapter we introduce higher inductive types, where we follow a similar idea: to allow in the specification of inductive types not only \emph{point constructors}, but also \emph{path constructors} that give us relations between the point constructors. 
The ideas behind the definition of higher inductive types are introduced by studying the simplest non-trivial example: the \emph{circle}.

\subsection{The induction principle of the circle}
The \emph{circle}\index{circle} is defined as a higher inductive type\index{higher inductive type} $\sphere{1}$\index{S 1@{$\sphere{1}$}} that comes equipped with\index{base@{$\base$}}\index{loop@{$\lloop$}}
\begin{align*}
\base & : \sphere{1} \\
\lloop & : \id{\base}{\base}.
\end{align*}
Just like for ordinary inductive types, the induction principle for higher inductive types provides us with a way of constructing sections of dependent types. However, we need to take the \emph{path constructor}\index{path constructor} $\lloop$ into account in the induction principle. 

By applying a section $f:\prd{t:\sphere{1}}P(t)$ to the base point of the circle, we obtain a term $f(\base):P(\base)$. Moreover, using the dependent action on paths\index{dependent action on paths} of $f$ of \cref{defn:apd} we also obtain for any dependent function $f:\prd{t:\sphere{1}}P(t)$ a path
\begin{align*}
\apd{f}{\lloop} & : \id{\mathsf{tr}_P(\lloop,f(\base))}{f(\base)}
\end{align*}
in the fiber $P(\base)$.

\begin{defn}
Let $P$ be a type family over the circle. The \define{dependent action on generators}\index{dependent action on generators!for the circle} is the map\index{dgen_S1@{$\mathsf{dgen}_{\sphere{1}}$}}
\begin{equation}\label{eq:dgen_circle}
\mathsf{dgen}_{\sphere{1}}:\Big(\prd{t:\sphere{1}}P(t)\Big)\to\Big(\sm{y:P(\base)}\id{\mathsf{tr}_P(\lloop,y)}{y}\Big)
\end{equation}
given by $\mathsf{dgen}_{\sphere{1}}(f)\defeq\pairr{f(\base),\apd{f}{\lloop}}$.
\end{defn}

We now give the full specification of the circle.

\begin{defn}
The \define{circle}\index{circle} is a type $\sphere{1}$\index{S 1@{$\sphere{1}$}} that comes equipped with\index{base@{$\base$}}\index{loop@{$\lloop$}}
\begin{align*}
\base & : \sphere{1} \\
\lloop & : \id{\base}{\base},
\end{align*}
and satisfies the \define{induction principle of the circle}\index{induction principle!of the circle}, which provides for each type family $P$ over $\sphere{1}$ a map
\begin{equation*}
\ind{\sphere{1}}:\Big(\sm{y:P(\base)}\id{\mathsf{tr}_P(\lloop,y)}{y}\Big)\to \Big(\prd{t:\sphere{1}}P(t)\Big),
\end{equation*}
and a homotopy witnessing that $\ind{\sphere{1}}$ is a section of $\mathsf{dgen}_{\sphere{1}}$
\begin{equation*}
\mathsf{dgen}_{\sphere{1}}\circ \ind{\sphere{1}}\htpy \idfunc
\end{equation*}
for the computation rule\index{computation rules!of the circle}.
\end{defn}

\begin{rmk}
The induction principle of the circle provides us with a dependent function $f:\prd{t:\sphere{1}}P(t)$ equipped with an identification
\begin{equation*}
(f(\base),\apd{f}{\lloop})=(x,p),
\end{equation*}
for any $x : P(\base)$ and $p : \mathsf{tr}_P(\lloop,x)=x$. By \cref{thm:eq_sigma} the identification
$(f(\base),\apd{f}{\lloop})=(x,p)$ is equivalently described as a pair of identifications
\begin{samepage}
\begin{align*}
\alpha & : f(\base)= x \\
\beta & : \mathsf{tr}(\alpha,\apd{f}{\lloop}) = p.
\end{align*}\end{samepage}%
Here, the transport is taken with respect to the family $x\mapsto \mathsf{tr}_P(\lloop,x)=x$. 

The identity type $\mathsf{tr}(\alpha,\apd{f}{\lloop}) = p$ is equivalent to the type
\begin{equation*}
\ct{\apd{f}{\lloop}}{\alpha}=\ct{\mathsf{ap}_{\mathsf{tr}_P(\lloop)}(\alpha)}{p}.
\end{equation*}
Indeed, such an equivalence can be constructed by path induction, because types reduce to the type $\apd{f}{\lloop}=p$ when $\alpha\jdeq\refl{f(x)}$. Therefore we obtain from the computation rule of the circle an identification $\alpha:f(\base)=x$, and an identification
\begin{equation*}
\beta':\ct{\apd{f}{\lloop}}{\alpha}=\ct{\mathsf{ap}_{\mathsf{tr}_P(\lloop)}(\alpha)}{p}
\end{equation*}
witnessing that the square
\begin{equation*}
\begin{tikzcd}[column sep=huge]
\mathsf{tr}_P(\lloop,f(\base)) \arrow[d,equals,swap,"\apd{f}{\lloop}"] \arrow[r,equals,"\ap{\mathsf{tr}_P(\lloop)}{\alpha}"] & \mathsf{tr}_P(\lloop,x) \arrow[d,equals,"p"] \\
f(\base) \arrow[r,equals,swap,"\alpha"] & x
\end{tikzcd}
\end{equation*}
commutes.
\end{rmk}

\subsection{The universal property of the circle}

In the following theorem we establish the \define{universal property}\index{universal property!of the circle} of the circle. The proof requires \cref{lem:circle_up_htpy,lem:circle_up_tr_compute}, which we state after we encounter their application.

\begin{thm}\label{thm:circle_up} 
For each type $X$, the \define{action on generators}\index{action on generators!for the circle}\index{gen_S1@{$\mathsf{gen}_{\sphere{1}}$}}
\begin{equation*}
\mathsf{gen}_{\sphere{1}}:(\sphere{1}\to X)\to \sm{x:X}x=x
\end{equation*}
given by $f\mapsto (f(\base),\ap{f}{\lloop})$ is an equivalence.
\end{thm}

\begin{proof}
Let $x:X$ and let $p:x=x$. By \cref{ex:trans_triv} we have an identification 
\begin{equation*}
\mathsf{tr\usc{}triv}(\lloop,x):\mathsf{tr}_{W_{\sphere{1}}X}(\lloop,x)=x,
\end{equation*}
from which we obtain a fiberwise equivalence
\begin{equation*}
\varphi : \prd{x:X} (x=x) \to (\mathsf{tr}_{W_{\sphere{1}}X}(\lloop,x)=x)
\end{equation*}
given by $p\mapsto \ct{\mathsf{tr\usc{}triv}(\lloop,x)}{p}$.
Moreover, for any $f:A\to B$, and any $p:x=y$ there is an identification $\ct{\mathsf{tr\usc{}triv}(p,f(x))}{\mathsf{ap}_f(p)}=\apd{f}{p}$, so it follows that the triangle
\begin{equation*}
\begin{tikzcd}[column sep=0]
& (\sphere{1}\to X) \arrow[dl,swap,"\mathsf{gen}_{\sphere{1}}"] \arrow[dr,swap,"\mathsf{dgen}_{\sphere{1}}" near start] \\
\sm{x:X}x=x \arrow[rr,"\total{\varphi}"',"\eqvsym"] & & \sm{x:X} \mathsf{tr}_{W_{\sphere{1}}X}(\lloop,x)=x \arrow[ul,densely dotted,bend right=15,swap,"\ind{\sphere{1}}"]
\end{tikzcd}
\end{equation*}
commutes, and the map $\total{\varphi}$ is a fiberwise equivalence by \cref{thm:fib_equiv}. Since the triangle commutes and $\ind{\sphere{1}}$ is a section of $\mathsf{dgen}_{\sphere{1}}$, it follows that the composite
\begin{equation*}
\ind{\sphere{1}}\defeq \ind{\sphere{1}}\circ \total{\varphi}
\end{equation*}
is a section of $\mathsf{gen}_{\sphere{1}}$. Therefore it remains to show that $\ind{\sphere{1}}$ is also a retraction of $\mathsf{gen}_{\sphere{1}}$, i.e., we have to show that for every $f:\sphere{1}\to X$ there is an identification
\begin{equation*}
\ind{\sphere{1}}(\mathsf{gen}_{\sphere{1}}(f))=f.
\end{equation*}
In \cref{lem:circle_up_htpy} below we establish that
\begin{equation*}
(\mathsf{gen}_{\sphere{1}}(\ind{\sphere{1}}(\mathsf{gen}_{\sphere{1}}(f)))=\mathsf{gen}_{\sphere{1}}(f))\to (\ind{\sphere{1}}(\mathsf{gen}_{\sphere{1}}(f))=f).
\end{equation*}
We get an identification $\mathsf{gen}_{\sphere{1}}(\ind{\sphere{1}}(\mathsf{gen}_{\sphere{1}}(f)))=\mathsf{gen}_{\sphere{1}}(f)$ from the fact that $\ind{\sphere{1}}$ is a section of $\mathsf{gen}_{\sphere{1}}$.
\end{proof}

\begin{lem}\label{lem:circle_up_htpy}
Let $f,g:\sphere{1}\to X$ be two dependent functions. Then there is a map
\begin{equation*}
(\mathsf{gen}_{\sphere{1}}(f)=\mathsf{gen}_{\sphere{1}}(g))\to (f=g)
\end{equation*}
\end{lem}

\begin{proof}
Let $p:\mathsf{gen}_{\sphere{1}}(f)=\mathsf{gen}_{\sphere{1}}(g)$. By function extensionality, it suffices to show that $f\htpy g$. However, since $f\htpy g$ is just the type $\prd{t:\sphere{1}}f(t)=g(t)$, we can construct such a homotopy by $\sphere{1}$-induction. Thus, it suffices to construct a term of type
\begin{equation*}
\sm{p:f(\base)=g(\base)} \mathsf{tr}_{E_{f,g}}(\lloop,p)=p, 
\end{equation*}
where $E_{f,g}$ is the family over $\sphere{1}$ given by $t\mapsto f(t)=g(t)$.

We claim that it suffices to construct for each $p:f(\base)=g(\base)$ an equivalence
\begin{equation*}
\Big(\mathsf{tr}_{E_{f,g}}(\lloop,p)=p\Big)\eqvsym\Big(\mathsf{tr}_{L}(p,\ap{f}{\lloop})=\ap{g}{\lloop}\Big),
\end{equation*}
where $L$ is the family over $X$ given by $x\mapsto x=x$. 
To see that this suffices, we note that such a fiberwise equivalence induces an equivalence on total spaces, and the total space
\begin{align*}
\sm{p:f(\base)=g(\base)} \mathsf{tr}_{L}(p,\ap{f}{\lloop})=\ap{g}{\lloop},
\end{align*}
and is equivalent to $\mathsf{gen}(f)=\mathsf{gen}(g)$, of which we have assumed a term.

The asserted fiberwise equivalence that we need for this proof to go through requires a sufficient generalization so that it can be constructed by path induction, so it is established separately in \cref{lem:circle_up_tr_compute} below.
\end{proof}

\begin{comment}
Consider $f,g:\sphere{1}\to X$ with a homotopy $H:f\htpy g$. Then we have $H(\base):f(\base)=g(\base)$, and the square
\begin{equation*}
\begin{tikzcd}[column sep=large]
f(\base) \arrow[r,equals,"H(\base)"] \arrow[d,swap,equals,"\ap{f}{\lloop}"] & g(\base) \arrow[d,equals,"\ap{g}{\lloop}"] \\
f(\base) \arrow[r,equals,swap,"H(\base)"] & g(\base)
\end{tikzcd}
\end{equation*}
commutes by the naturality of homotopies, established in \cref{defn:htpy_nat}\index{naturality!of homotopies}. In the following lemma we will relate such squares in two ways to a transport, by generalizing the above situation sufficiently so that path induction becomes applicable. We will use these computations of transports to establish the universal property of the circle. 
\end{comment}

With the following lemma we complete the proof of the universal property of the circle. 

\begin{samepage}%
\begin{lem}\label{lem:circle_up_tr_compute} ~
\begin{enumerate}
\item Let $f,g:A \to B$, and let $E_{f,g}$ be the family over $A$ given by 
\begin{equation*}
E_{f,g}(x)\defeq f(x)=g(x).
\end{equation*}
Then for any $p:x=x'$ in $A$ there is an equivalence
\begin{equation*}
\eqv{(\mathsf{tr}_{E_{f,g}}(p,q)=q')}{(\ct{\ap{f}{p}}{q'}=\ct{q}{\ap{g}{p}})}.
\end{equation*}
for any $q:f(x)=g(x)$ and $q':f(x')=g(x')$. In other words, there is an identification $\mathsf{tr}_{E_{f,g}}(p,q)=q'$ if and only if the square
\begin{equation*}
\begin{tikzcd}
f(x) \arrow[r,equals,"q"] \arrow[d,equals,swap,"\ap{f}{p}"] & g(x) \arrow[d,equals,"\ap{g}{p}"] \\
f(x') \arrow[r,equals,swap,"{q'}"] & g(x') 
\end{tikzcd}
\end{equation*}
commutes.
\item Let $L$ be the family over $B$ given by $L(y)\defeq y=y$, and let $q:y=y'$ be an identification in $B$. Then there is an equivalence
\begin{equation*}
\eqv{(\mathsf{tr}_L(q,p)=p')}{(\ct{q}{p'}=\ct{p}{q})}. 
\end{equation*}
for any $p:y=y$ and $p':y'=y'$. In other words, there is an identification $\mathsf{tr}_L(q,p)=p'$ if and only if the square
\begin{equation*}
\begin{tikzcd}
y \arrow[r,equals,"p"] \arrow[d,swap,equals,"q"] & y \arrow[d,equals,"q"] \\
y' \arrow[r,equals,swap,"{p'}"] & y'
\end{tikzcd}
\end{equation*}
commutes.
\item Let $f,g:A \to B$, let $p:x=x$ be a loop in $A$, and let $q:f(x)=g(x)$. Then there is an equivalence
\begin{equation*}
\eqv{(\mathsf{tr}_{E_{f,g}}(p,q)=q)}{(\mathsf{tr}_L(q,\ap{f}{p})=\ap{g}{p}).}
\end{equation*}
\end{enumerate}
\end{lem}
\end{samepage}%

\begin{proof}
The first claim follows by path induction on $p$, and the second claim follows by path induction on $q$. The third claim follows by combining the first two, since the types on both sides are equivalent to the type
\begin{equation*}
\ct{\ap{f}{p}}{q}=\ct{q}{\ap{g}{p}}
\end{equation*}
of witnesses that the square
\begin{equation*}
\begin{tikzcd}[column sep=large]
f(x) \arrow[r,equals,"q"] \arrow[d,swap,equals,"\ap{f}{p}"] & g(x) \arrow[d,equals,"\ap{g}{p}"] \\
f(x) \arrow[r,equals,swap,"q"] & g(x)
\end{tikzcd}
\end{equation*}
commutes.
\end{proof}

\begin{exercises}
\item \label{ex:circle_up_pushout}Show that
\begin{equation*}
\begin{tikzcd}[column sep=huge]
X^{\sphere{1}} \arrow[r,"\blank\circ\mathsf{const}_{\base}"] \arrow[d,swap,"\blank\circ\mathsf{const}_{\base}"] & X^\unit \arrow[d,"\blank\circ\mathsf{const}_{\ttt}"] \\
X^\unit \arrow[r,swap,"\blank\circ\mathsf{const}_{\ttt}"] & X^\bool
\end{tikzcd}
\end{equation*}
is a pullback square for each type $X$.
\item \label{ex:circle_dup}In this exercise we establish the \emph{dependent universal property} of the circle, analogous to the proof of \cref{thm:circle_up}.
\begin{subexenum}
\item Let $f,g:\prd{x:A}B(x)$, and let $E_{f,g}$ be the family over $A$ given by 
\begin{equation*}
E_{f,g}(x)\defeq f(x)=g(x).
\end{equation*}
Construct for any $p:x=x'$ in $A$ an equivalence
\begin{equation*}
\eqv{(\mathsf{tr}_{E_{f,g}}(p,q)=q')}{(\ct{\apd{f}{p}}{q'}=\ct{\ap{\mathsf{tr}_B(p)}{q}}{\apd{g}{p}})}.
\end{equation*}
for any $q:f(x)=g(x)$ and $q':f(x')=g(x')$.
\item Let $B$ be a family over $A$, and for $l:x=_A x$ let $L_x$ be the family over $B(x)$ given by 
\begin{equation*}
L_x(y)\defeq \mathsf{tr}_B(l,y)=y.
\end{equation*}
Furthermore, let $q:y=y'$ be an identification in $B(x)$. 
Construct an equivalence
\begin{equation*}
\eqv{(\mathsf{tr}_{L_x}(q,p)=p')}{(\ct{\ap{\mathsf{tr}_B(l)}{q}}{p'}=\ct{p}{q})}. 
\end{equation*}
for any $p:\mathsf{tr}_B(l,y)=y$ and $p':\mathsf{tr}_B(l,y')=y'$.
\item Let $f,g:\prd{x:A}B(x)$, let $p:x=x$ be a loop in $A$, and let $q:f(x)=g(x)$. 
Construct an equivalence
\begin{equation*}
\eqv{(\mathsf{tr}_{E_{f,g}}(p,q)=q)}{(\mathsf{tr}_{L_x}(q,\apd{f}{p})=\apd{g}{p}).}
\end{equation*}
\item Show that for any $f,g:\prd{t:\sphere{1}}P(t)$ there is a function
\begin{equation*}
\Big(\mathsf{dgen}_{\sphere{1}}(f)=\mathsf{dgen}_{\sphere{1}}(g)\Big)\to (f=g).
\end{equation*}
\item Show that for any type family $P$ over $\sphere{1}$, the \emph{dependent action on generators}
\begin{equation*}
\Big(\prd{t:\sphere{1}}P(t)\Big)\to \sm{u:P(\base)}\mathsf{tr}_P(\lloop,u)=u
\end{equation*}
is an equivalence.
\end{subexenum}
\item \label{ex:circle-connected}Let $P:\sphere{1}\to\prop$ be a family of propositions over the circle. Show that
\begin{equation*}
P(\base)\to\prd{t:\sphere{1}}P(t).
\end{equation*}
In this sense the circle is \emph{connected}.
\item Show that
\begin{equation*}
\prd{x,y:\sphere{1}}\neg\neg(x=y).
\end{equation*}
\item \label{ex:circle_constant}
Show that for any type $X$ and any $x:X$, the map
\begin{equation*}
\ind{\sphere{1}}(x,\refl{x}):\sphere{1}\to X
\end{equation*}
is homotopic to the constant map $\mathsf{const}_x$.
\item \label{ex:circle_connected}
\begin{subexenum}
\item Show that a type $X$ is a set if and only if the map
\begin{equation*}
\lam{x}{t} x : X \to (\sphere{1}\to X)
\end{equation*}
is an equivalence.
\item Show that a type $X$ is a set if and only if the map
\begin{equation*}
\lam{f}f(\base) : (\sphere{1}\to X)\to X
\end{equation*}
is an equivalence.
\end{subexenum}
\end{exercises}
