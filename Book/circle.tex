% !TEX root = hott_intro.tex

\section{The circle}
\index{circle|(}
\index{inductive type!circle|(}

We have seen inductive types, in which we describe a type by its constructors and an induction principle that allows us to construct sections of dependent types. Inductive types are freely generated by their constructors, which describe how we can construct their terms. 

However, many familiar constructions in algebra involve the construction of algebras by generators and relations. 
For example, the free abelian group with two generators is described as the group with generators $x$ and $y$, and the relation $xy=yx$. 

In this chapter we introduce higher inductive types\index{higher inductive type!circle}, where we follow a similar idea: to allow in the specification of inductive types not only \emph{point constructors}, but also \emph{path constructors} that give us relations between the point constructors. 
The ideas behind the definition of higher inductive types are introduced by studying the simplest non-trivial example: the \emph{circle}.

\subsection{The induction principle of the circle}
The \emph{circle} is defined as a higher inductive type $\sphere{1}$\index{S 1@{$\sphere{1}$}|see {circle}} that comes equipped with\index{base@{$\base$}}\index{loop@{$\lloop$}}\index{circle!base@{$\base$}}\index{circle!loop@{$\lloop$}}
\begin{align*}
\base & : \sphere{1} \\
\lloop & : \id{\base}{\base}.
\end{align*}
Just like for ordinary inductive types, the induction principle for higher inductive types provides us with a way of constructing sections of dependent types. However, we need to take the \emph{path constructor}\index{path constructor} $\lloop$ into account in the induction principle. 

By applying a section $f:\prd{x:\sphere{1}}P(x)$ to the base point of the circle, we obtain a term $f(\base):P(\base)$. Moreover, using the dependent action on paths\index{dependent action on paths} of $f$ of \cref{defn:apd} we also obtain for any dependent function $f:\prd{x:\sphere{1}}P(x)$ a path
\begin{align*}
\apd{f}{\lloop} & : \id{\tr_P(\lloop,f(\base))}{f(\base)}
\end{align*}
in the fiber $P(\base)$.

\begin{defn}
Let $P$ be a type family over the circle. The \define{dependent action on generators}\index{dependent action on generators!for the circle} is the map\index{dgen_S1@{$\dgen_{\sphere{1}}$}}
\begin{equation}\label{eq:dgen_circle}
\dgen_{\sphere{1}}:\Big(\prd{x:\sphere{1}}P(x)\Big)\to\Big(\sm{y:P(\base)}\id{\tr_P(\lloop,y)}{y}\Big)
\end{equation}
given by $\dgen_{\sphere{1}}(f)\defeq\pairr{f(\base),\apd{f}{\lloop}}$.
\end{defn}

We now give the full specification of the circle.

\begin{defn}
The \define{circle}\index{circle} is a type $\sphere{1}$\index{S 1@{$\sphere{1}$}} that comes equipped with\index{base@{$\base$}}\index{loop@{$\lloop$}}
\begin{align*}
\base & : \sphere{1} \\
\lloop & : \id{\base}{\base},
\end{align*}
and satisfies the \define{induction principle of the circle}\index{induction principle!of the circle}, which provides for each type family $P$ over $\sphere{1}$ a map
\begin{equation*}
\ind{\sphere{1}}:\Big(\sm{y:P(\base)}\id{\tr_P(\lloop,y)}{y}\Big)\to \Big(\prd{x:\sphere{1}}P(x)\Big),
\end{equation*}
and a homotopy witnessing that $\ind{\sphere{1}}$ is a section of $\dgen_{\sphere{1}}$
\begin{equation*}
\comphtpy{\sphere{1}}:\dgen_{\sphere{1}}\circ \ind{\sphere{1}}\htpy \idfunc
\end{equation*}
for the computation rule\index{computation rules!of the circle}.
\end{defn}

\begin{rmk}\label{rmk:circle-induction}
  The type of identifications $(y,p)=(y',p')$ in the type
  \begin{equation*}
    \sm{y:P(\base)}\tr_P(\lloop,y)=y
  \end{equation*}
  is equivalent to the type of pairs $(\alpha,\beta)$ consisting of an identification $\alpha:y=y'$, and an identification $\beta$ witnessing that the square
  \begin{equation*}
    \begin{tikzcd}[column sep=6em]
      \tr_P(\lloop,y) \arrow[d,equals,swap,"p"] \arrow[r,equals,"\ap{\tr_P(\lloop)}{\alpha}"] & \tr_P(\lloop,y') \arrow[d,equals,"{p'}"] \\
      y \arrow[r,equals,swap,"\alpha"] & y'
    \end{tikzcd}
  \end{equation*}
  commutes. Therefore it follows from the induction principle of the circle that for any $(y,p):\sm{y:P(\base)}\tr_P(\lloop,y)=y$, there is a dependent function $f:\prd{x:\sphere{1}}P(x)$ equipped with an identification
  \begin{equation*}
    \alpha : f(\base)=y,
  \end{equation*}
  and an identification $\beta$ witnessing that the square
  \begin{equation*}
    \begin{tikzcd}[column sep=6em]
      \tr_P(\lloop,f(\base)) \arrow[d,equals,swap,"{\apd{f}{\lloop}}"] \arrow[r,equals,"\ap{\tr_P(\lloop)}{\alpha}"] & \tr_P(\lloop,y) \arrow[d,equals,"{p}"] \\
      f(\base) \arrow[r,equals,swap,"\alpha"] & y
    \end{tikzcd}
  \end{equation*}
  commutes.  
\end{rmk}

\subsection{The (dependent) universal property of the circle}
\subsectionmark{The universal property of the circle}

Our goal is now to use the induction principle of the circle to derive the \define{universal property}\index{universal property!of the circle} of the circle. This universal property states that, for any type $X$ the canonical map
\begin{equation*}
  \Big(\sphere{1}\to X\Big)\to\Big(\sm{x:X}x=x\Big)
\end{equation*}
given by $f\mapsto(f(\base),\ap{f}{\lloop})$ is an equivalence. It turns out that it is easier to prove the \define{dependent universal property}\index{dependent universal property!of the circle} first. The dependent universal property states that for any type family $P$ over the circle, the canonical map
\begin{equation*}
  \Big(\prd{x:\sphere{1}}P(x)\Big)\to\Big(\sm{y:P(\base)}\tr_P(\lloop,y)=y\Big)
\end{equation*}
given by $f\mapsto(f(\base),\apd{f}{\lloop})$ is an equivalence.

\begin{thm}\label{thm:circle-dependent-universal-property}
  For any type family $P$ over the circle, the map
  \begin{equation*}
    \Big(\prd{x:\sphere{1}}P(x)\Big)
    \to
    \Big(\sm{y:P(\base)}\tr_P(\lloop,y)=y\Big)
  \end{equation*}
  given by $f\mapsto(f(\base),\apd{f}{\lloop})$ is an equivalence.
\end{thm}

\begin{proof}
  By the induction principle of the circle we know that the map has a section, i.e., we have
  \begin{align*}
    \ind{\sphere{1}} & : \Big(\sm{y:P(\base)}\tr_P(\lloop,y)=y\Big) \to \Big(\prd{x:\sphere{1}}P(x)\Big) \\
    \comphtpy{\sphere{1}} & : \dgen_{\sphere{1}}\circ\ind{\sphere{1}}\htpy\idfunc
  \end{align*}
  Therefore it remains to construct a homotopy
  \begin{equation*}
    \ind{\sphere{1}}\circ\dgen_{\sphere{1}}\htpy\idfunc.
  \end{equation*}
  Thus, for any $f:\prd{x:\sphere{1}}P(x)$ our task is to construct an identification
  \begin{equation*}
    \ind{\sphere{1}}(\dgen_{\sphere{1}}(f))=f.
  \end{equation*}
  By function extensionality it suffices to construct a homotopy
  \begin{equation*}
    \prd{x:\sphere{1}} \ind{\sphere{1}}(\dgen_{\sphere{1}}(f))(x)= f(x).
  \end{equation*}
  We proceed by the induction principle of the circle using the family of types $E_{g,f}(x)\defeq g(x)=f(x)$ indexed by $x:\sphere{1}$, where $g$ is the function
  \begin{equation*}
    g\defeq\ind{\sphere{1}}(\dgen_{\sphere{1}}(f)).
  \end{equation*}
  Thus, it suffices to construct
  \begin{align*}
    \alpha & : g(\base)=f(\base)\\
    \beta  & : \tr_{E_{g,f}}(\lloop,\alpha)=\alpha. 
  \end{align*}
  An argument by path induction on $p$ yields that
  \begin{equation*}
    \Big(\ct{\apd{g}{p}}{r}=\ct{\ap{\tr_P(p)}{q}}{\apd{f}{p}}\Big)\to\Big(\tr_{E_{g,f}}(p,q)=r\Big),
  \end{equation*}
  for any $f,g:\prd{x:X}P(x)$ and any $p:x=x'$, $q:g(x)=f(x)$ and $r:g(x')=f(x')$.
  Therefore it suffices to construct an identification $\alpha:g(\base)=f(\base)$ equipped with an identification $\beta$ witnessing that the square
  \begin{equation*}
    \begin{tikzcd}[column sep=6em]
      \tr_P(\lloop,g(\base)) \arrow[d,equals,swap,"\apd{g}{\lloop}"] \arrow[r,equals,"\ap{\tr_P(\lloop)}{\alpha}"] & \tr_P(\lloop,f(\base)) \arrow[d,equals,"\apd{f}{\lloop}"] \\
      g(\base) \arrow[r,equals,swap,"\alpha"] & f(\base)"
    \end{tikzcd}
  \end{equation*}
  commutes. Notice that we get exactly such a pair $(\alpha,\beta)$ from the computation rule of the circle, by \cref{rmk:circle-induction}.
\end{proof}

As a corollary we obtain the following uniqueness principle for dependent functions defined by the induction principle of the circle.

\begin{cor}
  Consider a type family $P$ over the circle, and let
  \begin{align*}
    y & : P(\base) \\
    p & : \tr_{P}(\lloop,y)=y.
  \end{align*}
  Then the type of functions $f:\prd{x:\sphere{1}}P(x)$ equipped with an identification
  \begin{equation*}
    \alpha: f(\base)=y
  \end{equation*}
  and an identification $\beta$ witnessing that the square
  \begin{equation*}
    \begin{tikzcd}[column sep=6em]
      \tr_P(\lloop,f(\base)) \arrow[d,equals,swap,"{\apd{f}{\lloop}}"] \arrow[r,equals,"\ap{\tr_P(\lloop)}{\alpha}"] & \tr_P(\lloop,y) \arrow[d,equals,"{p}"] \\
      f(\base) \arrow[r,equals,swap,"\alpha"] & y
    \end{tikzcd}
  \end{equation*}
  commutes, is contractible.
\end{cor}

Now we use the dependent universal property to derive the ordinary universal property of the circle. It would be tempting to say that it is a direct corollary, but we need to address the transport that occurs in the dependent universal property.

\begin{thm}\label{thm:circle_up} 
For each type $X$, the \define{action on generators}\index{action on generators!for the circle}\index{gen_S1@{$\mathsf{gen}_{\sphere{1}}$}}
\begin{equation*}
\mathsf{gen}_{\sphere{1}}:(\sphere{1}\to X)\to \sm{x:X}x=x
\end{equation*}
given by $f\mapsto (f(\base),\ap{f}{\lloop})$ is an equivalence.
\end{thm}

\begin{proof}
  We prove the claim by constructing a commuting triangle
  \begin{equation*}
    \begin{tikzcd}[column sep=-2em]
      \phantom{\Big(\sm{x:X}\tr_{\const_X}(\lloop,x)=x\Big)} & (\sphere{1}\to X) \arrow[dl,swap,"\gen_{\sphere{1}}"] \arrow[dr,"\dgen_{\sphere{1}}"] \\
      \Big(\sm{x:X}x=x\Big) \arrow[rr,swap,"\simeq"] & & \Big(\sm{x:X}\tr_{\const_X}(\lloop,x)=x\Big)
    \end{tikzcd}
  \end{equation*}
  in which the bottom map is an equivalence. Indeed, once we have such a triangle, we use the fact from \cref{thm:circle-dependent-universal-property} that $\dgen_{\sphere{1}}$ is an equivalence to conclude that $\gen_{\sphere{1}}$ is an equivalence.

  To construct the bottom map, we first observe that for any constant type family $\const_B$ over a type $A$, any $p:a=a'$ in $A$, and any $b:B$, there is an identification
  \begin{equation*}
    \mathsf{tr\usc{}const}_B(p,b)=b.
  \end{equation*}
  This identification is easily constructed by path induction on $p$. Now we construct the bottom map as the induced map on total spaces of the family of maps
  \begin{equation*}
    l\mapsto \ct{\mathsf{tr\usc{}const}_X(\lloop,x)}{l},
  \end{equation*}
  indexed by $x:X$. Since concatenating by a path is an equivalence, it follows by \cref{thm:fib_equiv} that the induced map on total spaces is indeed an equivalence.

  To show that the triangle commutes, it suffices to construct for any $f:\sphere{1}\to X$ an identification witnessing that the triangle
  \begin{equation*}
    \begin{tikzcd}[column sep=1em]
      \tr_{\const_X}(\lloop,f(\base)) \arrow[dr,equals,swap,"\apd{f}{\lloop}"] \arrow[rr,equals,"{\mathsf{tr\usc{}const}_X(\lloop,f(\base))}"] & & f(\base) \arrow[dl,equals,"\ap{f}{\lloop}"] \\
      & f(\base) & \phantom{\tr_{\const_X}(\lloop,f(\base))}
    \end{tikzcd}
  \end{equation*}
  commutes. This again follows from general considerations: for any $f:A\to B$ and any $p:a=a'$ in $A$, the triangle
  \begin{equation*}
    \begin{tikzcd}[column sep=1em]
      \tr_{\const_B}(p,f(a)) \arrow[dr,equals,swap,"\apd{f}{p}"] \arrow[rr,equals,"{\mathsf{tr\usc{}const}_B(p,f(a))}"] & & f(a) \arrow[dl,equals,"\ap{f}{p}"] \\
      & f(a') & \phantom{\tr_{\const_B}(p,f(a))}
    \end{tikzcd}
  \end{equation*}
  commutes by path induction on $p$.
\end{proof}

\begin{cor}
  For any loop $l:x=x$ in a type $X$, the type of maps $f:\sphere{1}\to X$ equipped with an identification
  \begin{equation*}
    \alpha : f(\base)=x 
  \end{equation*}
  and an identification $\beta$ witnessing that the square
  \begin{equation*}
    \begin{tikzcd}
      f(\base) \arrow[r,equals,"\alpha"] \arrow[d,equals,swap,"\ap{f}{\lloop}"] & x \arrow[d,equals,"l"] \\
      f(\base) \arrow[r,equals,swap,"\alpha"] & x
    \end{tikzcd}
  \end{equation*}
  commutes, is contractible.
\end{cor}

\subsection{Multiplication on the circle}
\label{sec:mulcircle}

One way the circle arises classically, is as the set of complex numbers at distance $1$ from the origin. It is an elementary fact that $|xy|=|x||y|$ for any two complex numbers $x,y\in\mathbb{C}$, so it follows that when we multiply two complex numbers that both lie on the unit circle, then the result lies again on the unit circle. Thus, using complex multiplication we see that there is a multiplication operation on the circle. And there is a shadow of this operation in type theory, even though our circle arises in a very different way!

\begin{defn}\label{defn:mul-circle}
  We define a binary operation
\begin{equation*}
  \mulcircle : \sphere{1}\to(\sphere{1}\to\sphere{1}).
\end{equation*}
\end{defn}

\begin{proof}[Construction]
  Using the universal property of the circle, we define $\mulcircle$ as the unique map $\sphere{1}\to(\sphere{1}\to\sphere{1})$ equipped with an identification
  \begin{equation*}
    \basemulcircle :\mulcircle(\base)=\idfunc
  \end{equation*}
  and an identification $\loopmulcircle$ witnessing that the square
  \begin{equation*}
    \begin{tikzcd}[column sep=huge]
      \mulcircle(\base) \arrow[r,equals,"\basemulcircle"] \arrow[d,equals,swap,"\ap{\mulcircle}{\lloop}"] & \idfunc \arrow[d,equals,"\eqhtpy(\htpyidcircle)"] \\
      \mulcircle(\base) \arrow[r,equals,swap,"\basemulcircle"] & \idfunc
  \end{tikzcd}
  \end{equation*}
  commutes. Note that in this square we have a homotopy $\htpyidcircle:\idfunc\htpy\idfunc$, which is not yet defined. We  use the dependent universal property of the circle with respect to the family $E_{\idfunc,\idfunc}$ given by
  \begin{equation*}
    E_{\idfunc,\idfunc}(x) \defeq (x=x),
  \end{equation*}
  to define $\htpyidcircle$ as the unique homotopy equipped with an identification
  \begin{equation*}
    \basehtpyidcircle : \htpyidcircle(\base)=\lloop
  \end{equation*}
  and an identification $\loophtpyidcircle$ witnessing that the square
  \begin{equation*}
    \begin{tikzcd}[column sep=8em]
      \tr_{E_{\idfunc,\idfunc}}(\lloop,\htpyidcircle(\base)) \arrow[r,equals,"\ap{\tr_{E_{\idfunc,\idfunc}}(\lloop)}{\basehtpyidcircle}"] \arrow[d,equals,swap,"\apd{\htpyidcircle}{\lloop}"] & \tr_{E_{\idfunc,\idfunc}}(\lloop,\lloop) \arrow[d,equals,"\gamma"] \\
      \htpyidcircle(\base) \arrow[r,equals,swap,"\basehtpyidcircle"] & \lloop
    \end{tikzcd}
  \end{equation*}
  commutes. Now it remains to define the path $\gamma:\tr_{E_{\idfunc,\idfunc}}(\lloop,\lloop)=\lloop$ in the above square. To proceed, we first observe that a simple path induction argument yields a function
  \begin{equation*}
    \Big(\ct{p}{r}=\ct{q}{p}\Big)\to\Big(\tr_{E_{\idfunc,\idfunc}}(p,q)=r\Big),
  \end{equation*}
  for any $p:\base=x$, $q:\base=\base$ and $r:x=x$. In particular, we have a function
  \begin{equation*}
    \Big(\ct{\lloop}{\lloop}=\ct{\lloop}{\lloop}\Big)\to\Big(\tr_{E_{\idfunc,\idfunc}}(\lloop,\lloop)=\lloop\Big).
  \end{equation*}
  Now we apply this function to $\refl{\ct{\lloop}{\lloop}}$ to obtain the desired identification
  \begin{equation*}
    \gamma:\tr_{E_{\idfunc,\idfunc}}(\lloop,\lloop)=\lloop.\qedhere
  \end{equation*}
\end{proof}

\begin{rmk}
  In the definition of $H:\idfunc\htpy\idfunc$ above, it is important that we didn't choose $H$ to be $\htpyrefl$. If we had done so, the resulting operation would be homotopic to $x,y\mapsto y$, which is clearly not what we had in mind with the multiplication operation on the circle. See also \cref{ex:circle-constant}.
\end{rmk}


The left unit law $\mulcircle(\base,x)=x$ holds by the computation rule of the universal property. More precisely, we define
\begin{equation*}
  \leftunit_{\sphere{1}}\defeq \htpyeq(\basemulcircle).
\end{equation*}
For the right unit law, however, we need to give a separate argument that is surprisingly involved, because all the aspects of the definition of $\mulcircle$ will come out and play their part.

\begin{thm}
  The multiplication operation on the circle satisfies the right unit law, i.e., we have
  \begin{equation*}
    \mulcircle(x,\base)=x
  \end{equation*}
  for any $x:\sphere{1}$.
\end{thm}

\begin{proof}
  The proof is by induction on the circle. In the base case we use the left unit law
  \begin{equation*}
    \leftunit_{\sphere{1}}(\base):\mulcircle(\base,\base)=\base.
  \end{equation*}
  Thus, it remains to show that
  \begin{equation*}
    \tr_P(\lloop,\leftunit_{\sphere{1}}(\base))=\leftunit_{\sphere{1}}(\base),
  \end{equation*}
  where $P$ is the family over the circle given by
  \begin{equation*}
    P(x) \defeq \mulcircle(x,\base)=x.
  \end{equation*}
  Now we observe that there is a function
  \begin{equation*}
    \Big(\ct{\htpyeq(\ap{\mulcircle}{p})(\base)}{r}=\ct{q}{p}\Big)\to\Big(\tr_{P}(p,q)=r\Big),
  \end{equation*}
  for any
  \begin{align*}
    p & : \base=x \\
    q & : \mulcircle(\base,\base)=\base \\
    r & : \mulcircle(x,\base)=x.
  \end{align*}
  Thus we see that, in order to construct an identification
  \begin{equation*}
    \tr_{P}(\lloop,\leftunit_{\sphere{1}})=\leftunit_{\sphere{1}},
  \end{equation*}
  it suffices to show that the square
  \begin{equation*}
    \begin{tikzcd}[column sep=8em]
      \mulcircle(\base,\base) \arrow[d,equals,swap,"\htpyeq(\ap{\mulcircle}{\lloop})(\base)"] \arrow[r,equals,"\leftunit_{\sphere{1}}(\base)"] & \base \arrow[d,equals,"\lloop"] \\
      \mulcircle(\base,\base) \arrow[r,equals,swap,"\leftunit_{\sphere{1}}(\base)"] & \base
    \end{tikzcd}
  \end{equation*}
  commutes. Now we note that we have an identification $H(\base)=\lloop$. It is indeed at this point, where it is important that $H$ is not the trivial homotopy, because now we can proceed by observing that the above square commutes if and only if the square
  \begin{equation*}
    \begin{tikzcd}[column sep=12em]
      \mulcircle(\base,\base) \arrow[d,equals,swap,"\htpyeq(\ap{\mulcircle}{\lloop})(\base)"] \arrow[r,equals,"\htpyeq(\basemulcircle)(\base)"] & \base \arrow[d,equals,"H(\base)"] \\
      \mulcircle(\base,\base) \arrow[r,equals,swap,"\htpyeq(\basemulcircle)(\base)"] & \base
    \end{tikzcd}
  \end{equation*}
  commutes. The commutativity of this square easily follows from the identification $\loopmulcircle$ constructed in \cref{defn:mul-circle}.
\end{proof}

\begin{exercises}
  \exercise \label{ex:circle-connected}Let $P:\sphere{1}\to\prop$ be a family of propositions over the circle. Show that
  \begin{equation*}
    P(\base)\to\prd{x:\sphere{1}}P(x).
  \end{equation*}
  In this sense the circle is \emph{connected}.
  \exercise Show that
  \begin{equation*}
    \prd{x,y:\sphere{1}}\neg\neg(x=y).
  \end{equation*}
  \exercise \label{ex:circle-constant}
  Show that for any type $X$ and any $x:X$, the map
  \begin{equation*}
    \ind{\sphere{1}}(x,\refl{x}):\sphere{1}\to X
  \end{equation*}
  is homotopic to the constant map $\mathsf{const}_x$.
  \exercise \label{ex:mulcircle-is-equiv}
  \begin{subexenum}
  \item Show that for any $x:\sphere{1}$, both functions
    \begin{equation*}
      \mulcircle(x,\blank)\qquad\text{and}\qquad\mulcircle(\blank,x)
    \end{equation*}
    are equivalences.
  \item Show that the function
    \begin{equation*}
      \mulcircle : \sphere{1}\to(\sphere{1}\to\sphere{1})
    \end{equation*}
    is an embedding. Compare this fact with \cref{ex:groupop-embedding}.
  \item Show that multiplication on the circle is associative and commutative.
  \end{subexenum}
  \exercise \label{ex:circle_connected}
  \begin{subexenum}
  \item Show that a type $X$ is a set if and only if the map
    \begin{equation*}
      \lam{x}{t} x : X \to (\sphere{1}\to X)
    \end{equation*}
is an equivalence.
\item Show that a type $X$ is a set if and only if the map
  \begin{equation*}
    \lam{f}f(\base) : (\sphere{1}\to X)\to X
  \end{equation*}
  is an equivalence.
  \end{subexenum}
  \exercise Show that the multiplicative operation on the circle is commutative, i.e.~construct an identification
  \begin{equation*}
    \mulcircle(x,y)=\mulcircle(y,x).
  \end{equation*}
  for every $x,y:\sphere{1}$.
  \exercise Show that the circle, equipped with the multiplicative operation $\mulcircle$ is an abelian group, i.e.~construct an inverse operation
  \begin{equation*}
    \invcircle : \sphere{1}\to\sphere{1}
  \end{equation*}
  and construct identifications
  \begin{align*}
    \leftinv_{\sphere{1}} & : \mulcircle(\invcircle(x),x) = \base \\
    \rightinv_{\sphere{1}} & : \mulcircle(x,\invcircle(x)) = \base.
  \end{align*}
  Moreover, show that the square
  \begin{equation*}
    \begin{tikzcd}
      \invcircle(\base) \arrow[d,equals] \arrow[r,equals] & \mulcircle(\base,\invcircle(\base)) \arrow[d,equals] \\
      \mulcircle(\invcircle(\base),\base) \arrow[r,equals] & \base
    \end{tikzcd}
  \end{equation*}
  commutes.
\end{exercises}
