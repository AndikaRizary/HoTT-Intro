\section{Truncations}

\subsection{The universal property of the truncations}

\begin{defn}\label{defn:is_truncation}
Let $X$ be a type. A map $f:X\to Y$ into an $k$-type $Y$ is said to satisfy the \define{universal property of $k$-truncation} if the precomposition map
\begin{equation*}
\blank\circ f: (Y\to Z)\to (X\to Z)
\end{equation*}
is an equivalence for every $k$-type $Z$.
\end{defn}

\begin{rmk}
A map $f:X\to Y$ into an $k$-type $Y$ satisfies the universal property of $k$-truncation if of for every $g:X\to Z$ the type of extensions
\begin{equation*}
\begin{tikzcd}
X \arrow[dr,"g"] \arrow[d,swap,"f"] \\
Y \arrow[r,densely dotted] & Z
\end{tikzcd}
\end{equation*}
is contractible. Indeed, the type of such extensions is the type
\begin{equation*}
\sm{h:Y\to Z} h\circ f\htpy g,
\end{equation*}
which is equivalent to the fiber of the precomposition map $\blank\circ f$ at $g$. 
\end{rmk}

\begin{thm}\label{thm:trunc_dup}
Suppose the map $f:X\to Y$ into an $k$-type $Y$. The following are equivalent:
\begin{enumerate}
\item The map $f$ satisfies the universal property of $k$-truncation.
\item For any type family $P$ of $k$-types over $Y$, the precomposition map
\begin{equation*}
\blank\circ f : \Big(\prd{y:Y}P(y)\Big)\to \Big(\prd{x:X}P(f(x))\Big)
\end{equation*}
is an equivalence. This property is also called the \define{dependent universal property} of the $k$-truncation.
\end{enumerate}
\end{thm}

\begin{proof}
The direction from (ii) to (i) is immediate, so we only have to show that (i) implies (ii).

Suppose $P$ is a family of $k$-truncated types over $Y$.  
Then we have a commuting square
\begin{equation*}
\begin{tikzcd}
\Big(Y\to\sm{y:Y}P(y)\Big) \arrow[r,"\blank\circ f"] \arrow[d,swap,"\proj 1 \circ\blank"] & \Big(X\to \sm{y:Y}P(y)\Big) \arrow[d,"\proj 1\circ \blank"] \\
\Big(Y\to Y\Big) \arrow[r,swap,"\blank\circ f"] & \Big(X\to Y)
\end{tikzcd}
\end{equation*}
Since the total space $\sm{y:Y}P(y)$ is again $k$-truncated by \cref{ex:istrunc_sigma}, it follows by the universal property of the $k$-truncation that the top map is an equivalence, and by the universal property the bottom map is an equivalence too. It follows from \cref{cor:pb_equiv} that this square is a pullback square, so it induces equivalences on the fibers by \cref{cor:pb_fibequiv}. In particular we have a commuting square
\begin{equation*}
\begin{tikzcd}
\Big(\prd{y:Y}P(y)\Big) \arrow[r] \arrow[d] & \Big(\prd{x:X}P(f(x))\Big) \arrow[d] \\
\fib{(\proj 1\circ \blank)}{\idfunc[Y]} \arrow[r] & \fib{(\proj 1 \circ \blank)}{f}
\end{tikzcd}
\end{equation*}
in which the left and right maps are equivalences by \cref{ex:pi_sec}, and the bottom map is an equivalence as we have just established. Therefore the top map is an equivalence, so we conclude that $f$ satisfies the dependent universal property.
\end{proof}

%\begin{thm}
%Suppose the map $f:X\to Y$ into an $k$-type $Y$ is an $k$-truncation, and let $Q$ be a type family of $(k+l)$-types over $Y$.
%Then the precomposition map
%\begin{equation*}
%\blank\circ f : \Big(\prd{y:Y}Q(y)\Big)\to \Big(\prd{x:X}Q(f(x))\Big)
%\end{equation*}
%is $(l-2)$-truncated, for $l:\N$. 
%\end{thm}

\begin{thm}\label{thm:trunc_id}
For any $x,y:X$, there is an equivalence
\begin{equation*}
\eqv{(\tproj{k+1}{x}=\tproj{k+1}{y})}{\trunc{k}{x=y}}.
\end{equation*}
\end{thm}

\begin{proof}
Let $x:X$. Then we define a family $E_x : \trunc{k+1}{X}\to \UU^{\leq k}$ as the unique extension
\begin{equation*}
\begin{tikzcd}[column sep=huge]
X \arrow[r] \arrow[d,swap,"\tproj{k}{\blank}"] \arrow[r,"{y\mapsto \trunc{k}{x=y}}"] & \UU^{\leq k} \\
\trunc{k}{X} \arrow[ur,densely dotted,swap,"E_x"] 
\end{tikzcd}
\end{equation*}
This unique extension exists by the universal property of $(k+1)$-truncation, because the universe $\UU^{\leq k}$ is itself a $(k+1)$-truncated type by \cref{ex:istrunc_UUtrunc}. 

To see that there is an equivalence 
\begin{equation*}
\eqv{(\tproj{k}{x}=y)}{E_x(y)}
\end{equation*}
for each $y:\trunc{k+1}{X}$, it suffices by \cref{thm:id_fundamental} to show that the total space 
\begin{equation*}
\sm{y:\trunc{k+1}{X}}E_x(y)
\end{equation*}
is contractible. At the center of contraction we have $(\tproj{k+1}{x},\tproj{k}{\refl{x}})$. It remains to construct the contraction
\begin{equation*}
\prd{y:\trunc{k+1}{X}}{p:E_x(y)} (\tproj{k+1}{x},\tproj{k}{\refl{x}})=(y,p).
\end{equation*}
We note that the type $(\tproj{k+1}{x},\tproj{k}{\refl{x}})=(y,p)$ is $k$-truncated, since it is an identity type in the total space
\begin{equation*}
\sm{y:\trunc{k+1}{X}}E_x(y),
\end{equation*}
which is $(k+1)$-truncated by \cref{ex:istrunc_sigma,thm:istrunc_next}. Therefore it suffices by \cref{thm:trunc_dup}, applied twice, to construct a term of type
\begin{equation*}
\prd{y:X}{p:x=y} (\tproj{k+1}{x},\tproj{k}{\refl{x}})=(\tproj{k+1}{y},\tproj{k}{p}).
\end{equation*}
We get such an identification for each $p:x=y$ by path induction on $p$.
\end{proof}

\begin{comment}
\subsection{Connected maps}

\subsection{The join extension and connectivity theorems}

\begin{defn}\label{defn:local}
For a given type $M$, a type $A$ is said to be \define{$M$-null} if the map
\begin{equation*}
\lam{a}{m}a : A \to (M \to A)  
\end{equation*}
is an equivalence.
\end{defn}

In other words, the type $A$ is $M$-null if each $f:M\to A$ has a unique extension along the
map $M\to\unit$, as indicated in the diagram
\begin{equation*}
\begin{tikzcd}
M \arrow[r,"f"] \arrow[d] & A \\
\unit. \arrow[ur,densely dotted]
\end{tikzcd}
\end{equation*}
Note that being $M$-null in the above sense is a proposition, so that the
type of all $M$-null types in $\UU$ is a subuniverse of $\UU$. 

\begin{eg}
By \cref{ex:sphere_null}, a type is $\sphere{n+1}$-null precisely when it is $n$-truncated,
for each $n\geq -2$ (taking the $(-1)$-sphere to be the empty type).
\end{eg}

The notion of $M$-connected type is in a sense dual to the notion of $M$-null types.

\begin{defn}
A type $A$ is said to be \define{$M$-connected} if every $M$-null
type is $A$-null. That is, if for every $M$-null type $B$, the map
\begin{equation*}
\lam{b}{a}{b} : B \to (A \to B)
\end{equation*}
is an equivalence. A map is said to be \define{$M$-connected} if its fibers are $M$-connected.
\end{defn}

Thus in particular, $M$ itself is $M$-connected, and the unit type $\unit$ is $M$-connected for every $M$. 

\begin{defn}
Let $M$ be a type. We say that a type $X$ has the \define{$M$-extension property}
with respect to a map $F:A\to B$, if the map
\begin{equation*}
\lam{g}{a} g(F(a)) : (B\to X)\to (A\to X)
\end{equation*}
is $M$-null. In the case $M\jdeq\unit$, we say that $X$ is \define{$F$-local}.
\end{defn}

\begin{lem}\label{lem:equivalent-extension-problems}
For any three types $A$, $A'$ and $B$, the type $B$ is $(\join{A}{A'})$-null
if and only if for any any $f:A\to B$, the type
\begin{equation*}
\sm{b:B}\prd{a:A}f(a)=b
\end{equation*}
is $A'$-null.
\end{lem}

\begin{proof}
To give $f:A\to B$ and $(f',H):A'\to\sm{b:B}\prd{a:A}f(a)=b$ is equivalent to giving a map $g:\join{A}{A'}\to B$. Concretely, the equivalence is given by substituting in $g:\join{A}{A'}\to B$ the constructors of the join, to obtain $\pairr{g\circ\inl,g\circ\inr,\apfunc{g}\circ\glue}$. 

Now observe that the fiber of precomposing with the unique map $!_{\join{A}{A'}} : \join{A}{A'}\to\unit$ at $g : \join{A}{A'}\to B$, is equivalent to
\begin{equation*}
\sm{b:B}\prd{t:\join{A}{A'}}g(t)=b.
\end{equation*}
Similarly, the fiber of precomposing with the unique map $!_{A'} : A'\to\unit$ at $\pairr{g\circ\inr,\apfunc{g}\circ\glue} : A'\to\sm{b:B}\prd{a:A}f(a)=b$ is equivalent to
\begin{equation*}
\sm{b:B}{h:\prd{a:A}g(\inl(a))=b}\prd{a':A'}\pairr{g(\inr(a')),\apfunc{g}(\glue(a,a'))}=\pairr{b,h}.
\end{equation*}
By the universal property of the join, these types are equivalent.
\end{proof}

\begin{lem}\label{lem:join-null}
Suppose $A$ is an $M$-connected type, and that $B$ is an $(\join{M}{N})$-null type. Then $B$ is $(\join{A}{N})$-null.
\end{lem}

\begin{proof}
Let $B$ be a $(\join{M}{N})$-null type. Our goal of showing that $B$ is
$(\join{A}{N})$-null is equivalent to showing that for any $f:N\to B$, 
the type 
\begin{equation*}
\sm{b:B}\prd{a:A}f(a)=b
\end{equation*}
is $A$-null. 
Since $B$ is assumed to be $(\join{M}{N})$-null, we know that this type is 
$M$-null. Since $A$ is $M$-connected, this type is also $A$-null.
\end{proof}

\begin{lem}\label{lem:N-extension-simple}
Let $A$ be $M$-connected and let $B$ be $(\join{M}{N})$-null. Then the map
\begin{equation*}
\lam{b}{a}b:B\to B^A
\end{equation*}
is $N$-null. 
\end{lem}

\begin{proof}
The fiber of $\lam{b}{a}b$ at a function $f:A\to B$ is equivalent to the type $\sm{b:B}\prd{a:A}f(a)=b$. Therefore, it suffices to show that this type is $N$-null. By \cref{lem:equivalent-extension-problems}, it is equivalent to show that $B$ is $(\join{A}{N})$-null. This is solved in \cref{lem:join-null}.
\end{proof}

\begin{thm}[Join extension theorem]\label{thm:join-extension}
Suppose $f:X\to Y$ is $M$-connected, and let $P:Y\to\UU$ be a family of
$(\join{M}{N})$-null types for some type $N$. Then precomposition by $f$, i.e.
\begin{equation*}
\lam{s}s\circ f : \Big(\prd{y:Y}P(y)\Big)\to\Big(\prd{x:X}P(f(x))\Big),
\end{equation*}
is an $N$-null map.
\end{thm}

\begin{proof}
Let $g:\prd{x:X}P(f(x))$. Then we have the equivalences
\begin{align*}
\hfib{(\blank\circ f)}{g} 
& \eqvsym \sm{s:\prd{y:Y}P(y)}\prd{x:X}s(f(x))=g(x) \\
& \eqvsym \sm{s:\prd{y:Y}P(y)}\prd{y:Y}{(x,p):\hfib{f}{y}} s(y)= \trans{p}{g(x)} \\
& \eqvsym \prd{y:Y}\sm{z:P(y)}\prd{(x,p):\hfib{f}{y}} \trans{p}{g(x)}=z \\
& \eqvsym \prd{y:Y}\hfib{\lam{z}{(x,p)}z}{\lam{(x,p)}\trans{p}{g(x)}}.
\end{align*}
Therefore, it suffices to show for every $y:Y$, that $P(y)$ has the $N$-extension property with respect to the unique map of type $\hfib{f}{y}\to\unit$. This is a special case of \cref{lem:N-extension-simple}.
\end{proof}

\begin{thm}\label{thm:simple-join}
Suppose $X$ is an $M$-connected type and $Y$ is an $N$-connected type. Then $\join{X}{Y}$ is an $(\join{M}{N})$-connected type.
\end{thm}

\begin{proof}
It suffices to show that any $(\join{M}{N})$-null type is $(\join{X}{Y})$-null.
Let $Z$ be an $(\join{M}{N})$-null type.
Since $Z$ is assumed to be $(\join{M}{N})$-null, it follows by \cref{lem:join-null} that $Z$ is $(\join{X}{N})$-null. By symmetry of the join, it also follows that $Z$ is $(\join{X}{Y})$-null.
\end{proof}

\begin{thm}[Join connectivity theorem]\label{thm:join-connectivity}
Consider an $M$-connected map $f:A\to X$ and an $N$-connected map $g:B\to X$. Then $\join{f}{g}$ is $(\join{M}{N})$-connected.
\end{thm}

\begin{proof}
This follows from \cref{thm:simple-join} and \cref{defn:join-fiber}.
\end{proof}

\begin{thm}\label{thm:joinconstruction-connectivity}
Consider the factorization
\begin{equation*}
\begin{tikzcd}
A_n \arrow[dr,swap,"f^{\ast n}"] \arrow[r,"q_n"] & \im(f) \arrow[d] \\
& X
\end{tikzcd}
\end{equation*}
of $f^{\ast n}$ through the image $\im(f)$. 
Then the map $q_n$ is $(n-2)$-connected, for each $n:\N$.
\end{thm}

\begin{proof}
We first show the assertion that, given a commuting diagram of the form
\begin{equation*}
\begin{tikzcd}
A \arrow[r,"q"] \arrow[dr,swap,"f"] & Y \arrow[d,"m"] & A' \arrow[l,swap,"{q'}"] \arrow[dl,"{f'}"] \\
& X
\end{tikzcd}
\end{equation*}
in which $m$ is an embedding, then $\join{f}{f'}=\join{(m\circ q)}{(m\circ q')}=m\circ (\join{q}{q'})$.
In other words, postcomposition with embeddings distributes over 
the join operation.

Note that, since $m$ is assumed to be an embedding, we have an equivalence of
type $\eqv{f(a)=f'(a)}{q(a)=q'(a)}$, for every $a:A$. Hence the pullback of
$f$ and $f'$ is equivalent to the pullback of $q$ along $q'$. Consequently, the
two pushouts
\begin{equation*}
\begin{tikzcd}
A\times_X A' \arrow[r,"\pi_2"] \arrow[d,swap,"\pi_1"] & A' \arrow[d] \\
A \arrow[r] & \join[X]{A}{A'}
\end{tikzcd}
\qquad\text{and}\qquad
\begin{tikzcd}
A\times_Y A' \arrow[r,"\pi_2"] \arrow[d,swap,"\pi_1"] & A' \arrow[d] \\
A \arrow[r] & \join[Y]{A}{A'}
\end{tikzcd}
\end{equation*}
are equivalent. Hence the claim follows.

As a corollary, we get that $q_n=q_f^{\ast n}$. Note that $q_f$ is surjective,
in the sense that $q_f$ is $\bool$-connected, where $\bool$ is the type of booleans%
\footnote{Recall that the $\bool$-null types are precisely the mere propositions.}.
Hence it follows that $q_n$ is $\bool^{\ast n}$-connected. 

Now recall that the $n$-th join power of $\bool$ is the $(n-1)$-sphere $\Sn^{n-1}$, and that
a type is $(\Sn^{n-1})$-connected if and only if it is $(n-2)$-connected.
\end{proof}
\end{comment}

\begin{comment}
\subsection{The construction of the $n$-truncation}\label{sec:truncation}

Our goal in this section is to prove the following theorem. Its proof will take up the entire section.

\begin{thm}\label{thm:truncation}
For every $k\geq -2$, there is a $k$-truncation operation
\begin{equation*}
\trunc{k}{\blank} : \UU\to\UU
\end{equation*}
equipped with a fiberwise transformation
\begin{equation*}
\tproj{k}{\blank}:\prd{X:\UU}X\to\trunc{k}{X},
\end{equation*}
such that for each $X:\UU$ the type $\trunc{k}{X}$ is a $k$-type satisfying the (dependent) universal property of $k$-truncation.
\end{thm}

We will define the $k$-truncation operation by induction on $k\geq-2$,
with the trivial operation as the base case. For $k\geq -2$, suppose we have
a $k$-truncation operation as described in the statement of the theorem.

\cref{thm:trunc_id} suggests that we can think of the type $\trunc{k+1}{X}$ is as the quotient of $X$ modulo the
`$(k+1)$-equivalence relation' given by $\trunc{k}{a=b}$. 

\begin{defn}
We define the reflexive relation $I_k(A) : A \to (A \to \UU)$ by
\begin{equation*}
I_k(A)(a,b) \defeq \trunc{k}{a=b},
\end{equation*}
and then we define
\begin{align*}
\trunc{k+1}{A} & \defeq \im(I_k(A)) \\
\tproj{k+1}{\blank} & \defeq q_{I_k(A)},
\end{align*}
where $q_{I_k(A)}:A\to \im(I_k(A))$ is the map with which the image comes equipped.
\end{defn}

Note that the codomain $(A\to\UU)$ of $I_k(A)$ is locally small since it is the exponent of
the locally small type $\UU$ by a small type $A$. 
Therefore the image of $I_k(A)$ is essentially small by \cref{cor:im_small}.
Since we want the $(k+1)$-truncation to be an operation $\UU\to\UU$, it would be more precise to define $\trunc{k+1}{A}$ as the (unique) type in $\UU$ that is equivalent to $\im(I_k(A))$. Of course, this makes no substantial difference.

\begin{lem}\label{lem:modal_contr}
For every $a,b:A$, we have an equivalence
\begin{equation*}
\eqv{(I_k(A)(a)=I_k(A)(b))}{\trunc{k}{a=b}}.
\end{equation*}
\end{lem}

\begin{proof}
Since $\im(I_k(A))$ is a subtype of $\UU^A$, there is for any $b:A$ a `tautological' family $E_b$ of types over $\im(I_k(A))$, given by
\begin{equation*}
E_b(P) \defeq P(b).
\end{equation*}
Note that $E_b(I_k(A)(a))\jdeq \trunc{k}{a=b}$. Therefore we can prove the claim by showing that the canonical map
\begin{equation*}
\prd{P:\im(I_k(A))} (I_k(A)(b)=P)\to P(b)
\end{equation*}
is a fiberwise equivalence. By \cref{thm:id_fundamental} it suffices to show that for each $b:A$, the total space
\begin{equation*}
\sm{P:\im(I_k(A))}P(b)
\end{equation*}
is contractible. 

For the center of contraction we take the pair
$\pairr{I_k(A)(b),\tproj{k}{\refl{b}}}$.
For the contraction we construct a term of type
\begin{equation*}
\prd{P:\im(I_k(A))}{y:P(b)} \pairr{I_k(A)(b),\tproj{k}{\refl{b}}}=\pairr{P,y}.
\end{equation*}
Since $I_k(A)(b,a)\jdeq\trunc{k}{b=a}$, it is equivalent to construct a term of type
\begin{equation*}
\prd{P:\im(I_k(A))}{y:P(b)}\sm{\alpha:\prd{a:A} \eqv{\trunc{k}{b=a}}{P(a)}} \alpha_b(\tproj{k}{\refl{b}})=y.
\end{equation*}
Let $P:\im(I_k(A))$ and $y:P(b)$. Then $P(a)$ is $n$-truncated for any $a:A$. Therefore, to construct a map
$\alpha(P,y)_a:\trunc{k}{b=a}\to P(a)$, it suffices to construct a map of type $(b=a)\to P(a)$. This may be done by
path induction, using $y:P(b)$. Since it follows that $\alpha(P,y)_b(\tproj{k}{\refl{b}})=y$, it only remains to show that each $\alpha(P,y)_a$ is an equivalence.  

Note that the type of those $P:\im(I_k(A))$ such that for all $y:P(b)$ and all $a:A$ the map $\alpha(P,y)_a$ is an equivalence, is a subtype of $\im(I_k(A))$, we may use the universal property of the image of $I_k(A)$: it suffices to lift
\begin{equation*}
\begin{tikzcd}
& \sm{P:\im(I_k(A))}\prd{y:P(b)}{a:A}\isequiv(\alpha(P,y)_a) \arrow[d] \\
A \arrow[ur,densely dotted] \arrow[r,swap,"I_k(A)"] & \im(I_k(A)).
\end{tikzcd}
\end{equation*}
In other words, it suffices to show that 
\begin{equation*}
\prd{x:A}{y:I_k(A)(x,b)}{a:A}\isequiv(\alpha(I_k(A)(x),y)_a).
\end{equation*}
Thus, we want to show that for any $y:\trunc{k}{x=b}$, the map $\trunc{k}{a=b}\to\trunc{k}{x=b}$ constructed above is an equivalence.
Since the fibers of this map are $n$-truncated, and $\iscontr(X)$ of an $n$-truncated type $X$ is always $n$-truncated, we may assume that $y$ is of the form $\tproj{k}{p}$ for $p:x=b$. 
Now it is easy to see that our map of type $\trunc{k}{b=a}\to\trunc{k}{x=a}$ is the unique map which
extends the path concatenation $\ct{p}{\blank}$, as indicated in the diagram
\begin{equation*}
\begin{tikzcd}[column sep=8em]
(b=a) \arrow[r,"\ct{p}{\blank}"] \arrow[d] & (x=a) \arrow[d] \\
\trunc{k}{b=a} \arrow[r,densely dotted,swap,"{\alpha(I_k(A)(x),y)_a}"] & \trunc{k}{x=a}.
\end{tikzcd}
\end{equation*}
Since the top map is an equivalence, it follows that the map $\alpha(I_k(A)(x),y)_a$ is an equivalence.
\end{proof}

\begin{cor}\label{cor:truncated}
The image $\im(I_k(A))$ is an $(n+1)$-truncated type. 
\end{cor}


\begin{proof}[Construction]
We will show that $\trunc{n+1}{A}$ is indeed $(n+1)$-truncated in \cref{cor:truncated} of \cref{lem:modal_contr} below. Once this fact is established, it remains to verify the dependent universal property of $(n+1)$-truncation.
By the join extension theorem \cref{thm:join-extension} (using $N\defeq \emptyt$), it suffices to show that the map $\tproj{n+1}{\blank}:A\to\trunc{n+1}{A}$ is $\sphere{n+2}$-connected. Note that $\tproj{n+1}{\blank}$ is surjective, so the claim that $\tproj{n+1}{\blank}$ is $\sphere{n+2}$-connected follows from \cref{lem:ap_connectivity}, where we show that for any surjective map $f:A\to X$, if the action on paths is $M$-connected for any two points in $A$, then $f$ is $\susp(M)$-connected. To apply this lemma, we also need to know that $\tproj{k}{\blank}:A\to\trunc{k}{A}$ is $\sphere{n+1}$-connected. This is shown in Corollary 7.5.8 of \cite{hottbook}.
\end{proof}


Before we are able to show that for any surjective map $f:A\to X$, if the action on paths is $M$-connected for any two points in $A$, then $f$ is $\susp(M)$-connected, we show that a type is $\susp(M)$-connected precisely when its identity types are $M$-connected.

\begin{lem}\label{lem:local_id}
Let $M$ be a type. Then a type $X$ is $(\join{\bool}{M})$-null
if and only if all of its identity types are $M$-null. 
\end{lem}

\begin{proof}
The map
\begin{equation*}
\lam{p}{m}p : (x=y)\to (M\to (x=y))
\end{equation*}
is an equivalence if and only if the induced map on total spaces
\begin{equation*}
\lam{\pairr{x,y,p}}\pairr{x,y,\lam{m}p} : \Big(\sm{x,y:X}x=y\Big)\to\Big(\sm{x,y:X}M\to (x=y)\Big)
\end{equation*}
is an equivalence. 
Since the map $\lam{x}\pairr{x,x,\refl{x}}:X\to\sm{x,y:X}x=y$ is an equivalence,
the above map is an equivalence if and only if the map
\begin{equation*}
\lam{x}\pairr{x,x,\lam{m}\refl{x}} : X\to\Big(\sm{x,y:X}M\to (x=y)\Big)
\end{equation*}
is an equivalence. For every $x:X$, the triple $\pairr{x,x,\lam{m}\refl{x}}$
induces a map $\susp(M)\to X$. By uniqueness of the universal property,
it follows that this map is the constant map $\lam{m}x$.
Thus we see that $\lam{x}\pairr{x,x,\lam{m}\refl{x}}$ is an equivalence if
and only if the map
\begin{equation*}
\lam{x}{m}x : X \to (\susp(M)\to X)
\end{equation*}
is an equivalence. 
\end{proof}

\begin{lem}\label{lem:ap_connectivity}
Suppose $f:A\to X$ is a surjective map, with the property that for every
$a,b:A$, the map
\begin{equation*}
\mapfunc{f}(a,b):(a=b)\to (f(a)=f(b))
\end{equation*}
is $M$-connected. Then $f$ is $\susp(M)$-connected. 
\end{lem}

\begin{proof}
We have to show that $\fib{f}{x}$ is $\susp(M)$-connected for each $x:X$. 
Since this is a mere proposition, and we assume that $f$ is surjective, it
is equivalent to show that $\fib{f}{f(a)}$ is $\susp(M)$-connected for each $a:A$. 
Let $Y$ be a $\susp(M)$-null type. 
For every $g:\fib{f}{f(a)}\to Y$ be a map we have the point $\theta(g)\defeq g(a,\refl{f(a)})$ in $Y$,
so we obtain a map
\begin{equation*}
\theta : (\fib{f}{f(a)}\to Y)\to Y
\end{equation*}
It is clear that $\theta(\lam{\pairr{b,p}}y)=y$, so it remains to show that
for every $g:\fib{f}{f(a)}\to Y$ we have $\lam{\pairr{b,p}}\theta(g)=g$.
That is, we must show that
\begin{equation*}
\prd{b:A}{p:f(a)=f(b)} g(a,\refl{f(a)})=g(b,p).
\end{equation*}
Using the assumption that $Y$ is $\susp(M)$-connected, it follows from
\cref{lem:local_id} that the type $g(a,\refl{f(a)})=g(b,p)$ is $M$-connected,
for every $b:A$ and $p:f(a)=f(b)$.
Therefore it follows, since the map $\mapfunc{f}(a,b):(a=b)\to(f(a)=f(b))$ is connected, that our goal is equivalent to
\begin{equation*}
\prd{b:A}{p:a=b} g(a,\refl{f(a)})=g(b,\mapfunc{f}(a,b,p)).
\end{equation*}
This follows by path induction. 
\end{proof}
\end{comment}

\subsection{Theorems not to forget}

\begin{thm}
  Consider a type $X$ and a family $P$ of $(k+n)$-truncated types over $\trunc{k}{X}$. Then the precomposition map
  \begin{equation*}
    \blank\circ\eta : \Big(\prd{y:\trunc{k}{X}}P(y)\Big)\to\Big(\prd{x:X}P(\eta(x))\Big)
  \end{equation*}
  is $(n-2)$-truncated.
\end{thm}

\begin{exercises}
\exercise Consider an equivalence relation $R:A\to (A\to\prop)$. Show that the map $\tproj{0}{\blank}\circ \inl:A\to \trunc{0}{A\sqcup^{R} A}$ satisfies the universal property of the quotient $A/R$, where $A\sqcup^{R} A$ is the canonical pushout
\begin{equation*}
\begin{tikzcd}
\sm{x,y:A}R(x,y) \arrow[r,"\pi_2"] \arrow[d,swap,"\pi_1"] & A \arrow[d,"\inr"] \\
A \arrow[r,swap,"\inl"] & A\sqcup^{R} A.
\end{tikzcd}
\end{equation*}
\exercise Consider the trivial relation $\unit\defeq\lam{x}{y}\unit:A\to (A\to\prop)$. Show that the set quotient $A/\unit$ is a proposition satisfying the universal property of the propositional truncation.
\exercise Show that the type of pointed $2$-element sets
\begin{equation*}
\sm{X:\UU_{\bool}}X
\end{equation*}
is contractible.
\exercise Define the type $\mathbb{F}$ of finite sets by
\begin{equation*}
\mathbb{F}\defeq \im(\fin),
\end{equation*}
where $\fin:\N\to\UU$ is defined in \cref{defn:fin}. 
\begin{subexenum}
\item Show that $\eqv{\mathbb{F}}{\sm{n:\N}\UU_{\fin(n)}}$. 
\item Show that $\mathbb{F}$ is closed under $\Sigma$ and $\Pi$. 
\end{subexenum}
\exercise
\begin{subexenum}
\item A type $Y$ is called \define{$k$-separated} if for every type $X$ the map
  \begin{equation*}
    (\trunc{k}{X}\to Y)\to(X\to Y)
  \end{equation*}
  is an embedding. Show that $Y$ is $k$-separated  if and only if it is $(k+1)$-truncated.
\item A type $Y$ is called \define{$n$-fold $k$-separated} if for every type $X$ the map
  \begin{equation*}
    (\trunc{k}{X}\to Y)\to (X\to Y)
  \end{equation*}
  is $(n-2)$-truncated. Show that $Y$ is $n$-fold $k$-separated if and only if it is $(k+n)$-truncated.
\end{subexenum}
\end{exercises}
