\section{Elementary number theory}

One of the things type theory is great for, is for the formalization of mathematics in a computer proof assistant. Those are programs that can compile any type theoretical construction to check that this construction indeed has the type it was claimed it has.

At this point in our development of type theory there are two areas of mathematics that would be natural to try to do in type theory: discrete mathematics and elementary number theory. Indeed, how does one define in type theory the greatest common divisor of two natural numbers, or how does one show that there are infinitely many primes? How does one even formalize that every non-empty subset of the natural numbers has a least element?

To answer these questions we will run into questions of decidability. How do we write a term that decides wheter a number is prime or not? Or indeed, is it even true that every non-empty subset of the natural numbers has a least element? What about the subset of $\N$ that contains $1$, and it contains $0$ if and only if Goldbach's conjecture holds? Finding the least element of this subset is equivalent to settling the conjecture!

Therefore, we will prove the well-foundedness of the natural numbers for decidable subsets of $\N$. In fact, we will show it for decidable families, because sometimes we don't know in advance whether a family of types is in fact a subtype. A consequence of involving decidability in the well-foundedness of the natural numbers is that for many properties one has to prove that they are decidable. Luckily this is the case: many of the familiar properties that one encounters in number theory are indeed decidable.

\subsection{Decidability}

A common way of reasoning in mathematics is via a proof by contradiction: ``in order to show that $P$ holds we show that it cannot be the case that $P$ doesn't hold". There are no inference rules in type theory that allow us to obtain a term of type $P$ from a term of type $\neg\neg P$. However, for some propositions $P$ one can construct a function $\neg\neg P \to P$. The \emph{decidable propositions} from a class of such propositions $P$ for which we can show $\neg\neg P \to P$.

\begin{defn}
  A type $A$ is said to be decidable if it comes equipped with a term of type
  \begin{equation*}
    \mathsf{is\usc{}decidable}(A)\defeq A+\neg A.
  \end{equation*}
  Decidable propositions are called \define{classical}. We will write
  \begin{equation*}
    \mathsf{classical\usc{}Prop} \defeq \sm{P:\mathsf{Prop}}\mathsf{is\usc{}decidable}(P)
  \end{equation*}
  for the type of all classical propositions (with respect to a universe $\UU$).
\end{defn}

\begin{eg}
  The types $\unit$ and $\emptyt$ are decidable. Indeed, we have
  \begin{align*}
    \inl(\ttt) & :\mathsf{is\usc{}decidable}(\unit) \\
    \inr(\idfunc) & : \mathsf{is\usc{}decidable}(\emptyt).\qedhere
  \end{align*}
  Any type $A$ equipped with a point $a:A$ is decidable.
\end{eg}

\begin{lem}
  For each $m,n:\N$, the types $\EqN(m,n)$, $m\leq n$ and $m<n$ are decidable.
\end{lem}

\begin{proof}
  The proofs in each of the three cases is similar, so we only show that $\EqN(m,n)$ is decidable for each $m,n:\N$. This is done by induction on $m$ and $n$. Note that the types
  \begin{align*}
    \EqN(\zeroN,\zeroN) & \jdeq \unit \\
    \EqN(\zeroN,\succN(n)) & \jdeq \emptyt \\
    \EqN(\succN(m),\zeroN) & \jdeq \emptyt 
  \end{align*}
  are all decidable. Moreover, the type $\EqN(\succN(m),\succN(n))\jdeq \EqN(m,n)$ is decidable by the inductive hypothesis.
\end{proof}

Typically we are mostly interested in decidability of propositions. However, we have defined the notion of decidability for general types, because the condition on an arbitrary type that its identity types are decidable is of some interest. We now study such types.

\begin{defn}
  We say that a type $A$ has decidable equality if the identity type $x=y$ is decidable for every $x,y:A$. 
\end{defn}

\begin{cor}
  Equality on the natural numbers is decidable.
\end{cor}

\begin{proof}
  This follows immediately from the equivalences $(m=n)\simeq \EqN(m,n)$, and the fact that $\EqN(m,n)$ is decidable.
\end{proof}

\begin{lem}
  Suppose that $A$ and $B$ are types with decidable equality. Then the coproduct $A+B$ also has decidable equality.
\end{lem}

\begin{proof}
  Our goal is to construct a dependent function
  \begin{equation*}
    d_{A+B} : \prd{z,z':A+B}\mathsf{is\usc{}decidable}(z=z').
  \end{equation*}
  This function is constructed by coproduct induction on both $z$ and $z'$, so we have four cases to consider. Recall from \cref{thm:id-coprod-compute} that we have equivalences
  \begin{align*}
    (\inl(x)=\inl(x')) & \simeq (x=x') \\
    (\inl(x)=\inr(y')) & \simeq \emptyt \\
    (\inr(y)=\inl(x')) & \simeq \emptyt \\
    (\inr(y)=\inr(y')) & \simeq (y=y').
  \end{align*}
  Therefore the type $z=z'$ is equivalent to a decidable type in each of the four cases.
\end{proof}

\begin{cor}
  The type $\Z$ has decidable equality.
\end{cor}

\begin{cor}
  For any $n:\N$ the type $\mathsf{Fin}(n)$ has decidable equality. 
\end{cor}

We have already shown in \cref{thm:eq_nat} that the type of natural numbers is a set. In fact, any type with decidable equality is a set. This fact is known as Hedberg's theorem. Our proof of Hedberg's theorem might appear to be slightly complicated compared to the proof in \cite{hottbook}. This is due to the fact that we haven't introduced function extensionality yet, which is used there to observe that $\neg\neg(x=y)$ is a proposition for any $x,y:A$.

\begin{thm}[Hedberg]
  Any type with decidable equality is a set.
\end{thm}

\begin{proof}
  Let $A$ be a type, and let
  \begin{equation*}
    d:\prd{x,y:A}(x=y)+\neg(x=y).
  \end{equation*}
  Our proof is an application of \cref{lem:prop_to_id}. In order to construct a binary relation $R$, we first consider the type family $D(x,y):((\id{x}{y})+\neg(\id{x}{y}))\to\UU$ given by
  \begin{align*}
    D(x,y,\inl(p)) & \defeq \unit \\
    D(x,y,\inr(p)) & \defeq \emptyt.
  \end{align*}
  Now we define the binary relation $R:A\to (A\to\UU)$ by
  \begin{equation*}
    R(x,y)\defeq D(x,y,d(x,y)).
  \end{equation*}
  It follows that $R(x,y)$ is a proposition for every $x,y:A$, since $D(x,y,z)$ is a proposition for every $z:(x=y)+\neg(x=y)$.

  To see that $R$ is reflexive it suffices to construct a term of type
  \begin{equation*}
    \rho':\prd{x:A}{p:(x=x)+\neg(x=x)}D(x,x,p)
  \end{equation*}
  For any $x:A$ we define $\rho'(x)$ by case analysis on $p:(x=x)+\neg(x=x)$. Given $p:x=x$, we take $\rho'(x,\inl(p))\defeq\ttt$. Given $p:\neg(x=x)$, we obtain $p(\refl{x}):\emptyt$, so there is nothing to define in this case. This completes the definition of $\rho'$.
  
  It remains to show that $R(x,y)\to (x=y)$ for any $x,y:A$. Note that it suffices to construct a function
  \begin{equation*}
    \alpha:D(x,y,p)\to (x=y)
  \end{equation*}
  for any $x,y:A$ and $p:(x=y)+\neg(x=y)$. This function is constructed by case analisys on $p$. Given $p:x=y$, we simply define $\alpha(x,y,\inl(p))\defeq p$. Given $p:\neg(x=y)$ the type $D(x,y,p)$ is empty, so there is nothing to define.
\end{proof}

\subsection{The well-ordering principle for decidable families over \texorpdfstring{$\N$}{ℕ}}

\begin{defn}
  A family $P$ over a type $A$ is said to be decidable if $P(x)$ is decidable for every $x:A$. A \define{decidable subset} of a type $A$ is a map
  \begin{equation*}
    P:A\to\mathsf{classical\usc{}Prop}.
  \end{equation*}
\end{defn}

\begin{defn}
  Let $P$ be a decidable family over $\N$, and let $n:\N$ be a natural number equipped with $p:P(n)$. We say that $n$ is a \define{minimal $P$-element} if it comes equipped with a term of type
  \begin{equation*}
    \mathsf{is\usc{}minimal}_P(n,p)\defeq \Big(\prd{m:\N}P(m)\to (n\leq m)\Big)
  \end{equation*}
\end{defn}

Note that the type $\mathsf{is\usc{}minimal}_P(n,p)$ doesn't depend on $p$. However, it doesn't make much sense that $n$ is a minimal element of $P$ unless we already know that $n$ is in $P$. Indeed, if we would omit the hypothesis that $n$ is in $P$, it would be more accurate to say that $n$ is a \emph{lower bound} of $P$. The following theorem is the well-ordering principle of $\N$. 

\begin{thm}
  Let $P$ be a decidable family over $\N$. Then there is a function
  \begin{equation*}
    \Big(\sm{n:\N}P(n)\Big)\to\Big(\sm{m:\N}{p:P(m)}\mathsf{is\usc{}minimal}_P(m,p)\Big).
  \end{equation*}
\end{thm}

\begin{proof}
  Consider a universe $\UU$ that contains $P$. We show by induction on $n:\N$ that there is a function
  \begin{equation*}
    Q(n)\to \Big(\sm{m:\N}{p:Q(m)}\mathsf{is\usc{}minimal}_Q(m,p)\Big) 
  \end{equation*}
  for every decidable family $Q:\N\to\UU$. Note that we performed a swap in the order of quantification, using the universe that contains $P$. This slightly strengthens the inductive hypothesis, which we will be able to exploit.

  The base case is trivial, since $\zeroN$ is the least natural number. For the inductive step, suppose that $Q(\succN(n))$ holds. Note that $Q(\zeroN)$ is assumed to be decidable, so we proceed by case analysis on $Q(\zeroN)+\neg Q(\zeroN)$. Given $q:Q(\zeroN)$, it follows immediately that $\zeroN$ must be minimal. In the case where $\neg Q(\zeroN)$, we consider the decidable subset $Q'$ of $\N$ given by
  \begin{equation*}
    Q'(n)\defeq Q(\succN(n)).
  \end{equation*}
  Since we have $q:Q'(n)$, we obtain a minimal element in $Q'$ by the inductive hypothesis. Of course, by the assumption that $Q(\zeroN)$ doesn't hold, the minimal element of $Q'$ is also the minimal element of $Q$.
\end{proof}

\subsection{The pigeonhole principle}

The pigeonhole principle states that if we place more than $n$ balls in $n$ bags, then at least one bag will contain more than one ball. In this section we will give a type theoretical proof of the pigeonhole principle.

First we give a definition of a function that counts the number of elements in a decidable subset of $\mathsf{Fin}(n)$.

\begin{defn}
  Let $P$ be a decidable subset of $\mathsf{Fin}(n)$. We define the number $|P|:\N$ of elements in $P$.
\end{defn}

\begin{proof}[Construction]
  We give the proof by induction on $n:\N$. In the base case we note that $\mathsf{Fin}(\zeroN)$ has no elements, so we define $|P|\defeq\zeroN$.

  For the inductive step, we define $|P|$ by case analysis on $P(\inr(\ttt))+\neg P(\inr(\ttt))$. Let $P'$ be the family over $\mathsf{Fin}(n)$ given by $P'(i)\defeq P(\inl(i))$. In the case where $P(\inr(\ttt))$ holds, then we define $|P|\defeq \succN |P'|$. In the case where $P(\inr(\ttt))$ doesn't hold we define $|P|\defeq |P'|$.
\end{proof}

\begin{thm}\label{thm:pigeonhole}
  For any $m,n:\N$ and any function $f:\mathsf{Fin}(m)\to\mathsf{Fin}(n)$, if $m>n$, then there is an $i:\mathsf{Fin}(n)$ which is in the image of more than one point in $\mathsf{Fin}(m)$.
\end{thm}

\begin{proof}
  The pigeonhole principle is proven by induction on $m,n:\N$. In the base case for $m$ we immediately obtain a contradiction from the assumption that $m>n$. For the inductive step on $m$ and the base case for $n$, we note that $\mathsf{Fin}(\succN(m))\jdeq \mathsf{Fin}(m)+\unit$ and $\mathsf{Fin}(\zeroN)\jdeq \empty$. Therefore $f:\mathsf{Fin}(\succN(m))\to\mathsf{Fin}(\zeroN)$ is a function from a pointed type to the empty type, which gives us a contradiction.

  It remains to give the inductive step for $n$. Let $i\defeq f(\inr(\ttt)):\mathsf{Fin}(\succN(n))$. Since the ordering relation $<$ on $\N$ is decidable, we can decide whether $i$ is in the image of more than one point in $\mathsf{Fin}(m)$. If this is the case, this completes the proof. If this is not the case, note that we have a commuting square
  \begin{equation*}
    \begin{tikzcd}
      \mathsf{Fin}(m) \arrow[r,"{f'}"] \arrow[d,swap,"\inl"] & \mathsf{Fin}(n) \arrow[d,"{\hat{i}}"] \\
      \mathsf{Fin}(\succN(m)) \arrow[r,swap,"f"] & \mathsf{Fin}(\succN(n))
    \end{tikzcd}
  \end{equation*}
  where $\hat{i}$ is the inclusion that omits the value $i$. The function $f':\mathsf{Fin}(m)\to\mathsf{Fin}(n)$ is defined 

  Now note that both the left and right maps in this square are embeddings, and that by the induction hypothesis the pigeonhole principle applies to the function $f':\mathsf{Fin}(m)\to\mathsf{Fin}(n)$. 
\end{proof}

\begin{cor}\label{cor:pigeonhole}
  Given $m>n$, no function $\mathsf{Fin}(m)\to\mathsf{Fin}(n)$ is an embedding.
\end{cor}

It is straightforward to see that the statements of \cref{thm:pigeonhole,cor:pigeonhole} are equivalent, and one might argue that the statement of \cref{cor:pigeonhole} is the more `type theoretical way' of phrasing the pigeonhole principle. However, the relation to counting the number of points that get mapped to 

\begin{thm}\label{thm:generalized-pigeonhole}
  For any $m,n:\N$ and any function $f:\mathsf{Fin}(m)\to\mathsf{Fin}(n)$, if $m>kn$ for some $k:\N$, then there is an $i:\mathsf{Fin}(n)$ which is in the image of more than $k$ points in $\mathsf{Fin}(m)$. 
\end{thm}

\subsection{Defining the greatest common divisor}

\begin{lem}
  For any $d,n:\N$, the type $d\mid n$ is decidable.
\end{lem}

\begin{proof}
  We give the proof by case analysis on $(d=\zeroN)+(d\neq\zeroN)$. If $d=\zeroN$, then $d\mid n$ holds if and only if $\zeroN=n$, which is decidable.

  If $d\neq\zeroN$, then it follows that $n\leq nd$. Therefore we obtain by the well-ordering principle of the natural numbers a minimal $m:\N$ that satisfies the decidable property $n\leq md$. Now we observe that $d\mid n$ holds if and only if $n=md$, which is decidable.
\end{proof}

\begin{lem}
  For any decidable subtype of $\mathsf{Fin}(n)$, if it contains a number $i:\mathsf{Fin}(n)$, then it contains both a minimal and maximal element.
\end{lem}

\begin{defn}
  For any two natural numbers $m,n$ we define the \define{greatest common divisor} $\gcd(m,n)$, which satisfies the following two properties:
  \begin{enumerate}
  \item We have both $\gcd(m,n)\mid m$ and $\gcd(m,n)\mid n$.
  \item For any $d:\N$ we have $d\mid \gcd(m,n)$ if and only if both $d\mid m$ and $d\mid n$ hold.
  \end{enumerate}
\end{defn}

\subsection{The trial division primality test}

\begin{thm}
  For any $n:\N$, the proposition $\mathsf{is\usc{}prime}(n)$ is decidable.
\end{thm}

It is important to note that, even when we prove that a type such as $\mathsf{is\usc{}prime}(n)$ is decidable, it is only after we \emph{evaluate} the proof term that we know whether the type under consideration has a term or not. In other words, for any given $n$ we don't know right away whether it is prime or not. Evaluating whether $n$ is prime can be computationally costly, so it may be desirable in any specific situation to give a separate mathematical \emph{argument} that decides whether or not the number is prime.

\subsection{The infinitude of primes}

\begin{thm}
  There are infinitely many primes.
\end{thm}

Some further ideas to include in this chapter:
\begin{enumerate}
\item If $2^n-1$ is prime, then $n$ is prime.
\item Fermat's little theorem.
\end{enumerate}

\begin{exercises}
\item Show that for any $f:\mathsf{Fin}(m)\to\mathsf{Fin}(n)$ and any $i:\mathsf{Fin}(n)$, the type $\fib{f}{i}$ is decidable.
\item Consider a decidable type $P(i)$ indexed by $i:\mathsf{Fin}(n)$.
  \begin{subexenum}
  \item Show that the type
    \begin{equation*}
      \prd{i:\mathsf{Fin}(n)}P(i)
    \end{equation*}
    is decidable.
  \item Show that the type
    \begin{equation*}
      \sm{i:\mathsf{Fin}(n)}P(i)
    \end{equation*}
    is decidable.
  \end{subexenum}
\item
  \begin{subexenum}
  \item Show that $\nat$ and $\bool$ have decidable equality. Hint: to show that $\mathbb{N}$ has decidable equality, show first that the successor function is injective.
  \item Show that if $A$ and $B$ have decidable equality, then so do $A+B$ and $A\times B$. Conclude that $\Z$ has decidable equality.
  \item Show that if $A$ is a retract of a type $B$ with decidable equality, then $A$ also has decidable equality.
  \end{subexenum}
\item Define the prime-counting function $\pi:\N\to\N$.
\item (The Cantor-Schr\"oder-Bernstein theorem) Let $X$ and $Y$ be two sets with decidable equality, and consider two maps $f:X\to Y$ and $g:Y\to X$, both of which we assume to be injective. Construct an equivalence $X\simeq Y$.
\item For any $k:\Z$, define a function $i\mapsto i+k \mod n$ of type $\mathsf{Fin}(n)\to\mathsf{Fin}(n)$. Show that this function is an equivalence.
\item For any $k:\Z$, define a function $i\mapsto i\cdot k \mod n$ of type $\mathsf{Fin}(n)\to\mathsf{Fin}(n)$. Show that this function is an equivalence if and only if $\gcd(n,k)=1$.
\item Show that
  \begin{equation*}
    \sum_{i=0}^n \binom{n-i}{i}=F_{n+1}
  \end{equation*}
\end{exercises}
