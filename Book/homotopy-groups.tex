\section{Homotopy groups of types}

\subsection{The suspension-loop space adjunction}

We get an even better version of the universal property of $\susp{X}$ if we know in advance that the type $X$ is a pointed type: on pointed types, the suspension functor is left adjoint to the loop space functor. This property manifests itself in the setting of pointed types, so we first give some definitions regarding pointed types.

\begin{defn}
\begin{enumerate}
\item A pointed type consists of a type $X$ equipped with a base point $x:X$. We will write $\UU_\ast$ for the type $\sm{X:\UU}X$ of all pointed types.
\item Let $(X,\ast_X)$ be a pointed type. A \define{pointed family} over $(X,\ast_X)$ consists of a type family $P:X\to \UU$ equipped with a base point $\ast_P:P(\ast_X)$. 
\item Let $(P,\ast_P)$ be a pointed family over $(X,\ast_X)$. A \define{pointed section} of $(P,\ast_P)$ consists of a dependent function $f:\prd{x:X}P(x)$ and an identification $p:f(\ast_X)=\ast_P$. We define the \define{pointed $\Pi$-type} to be the type of pointed sections:
\begin{equation*}
\prdsym^\ast_{(x:X)}P(x) \defeq \sm{f:\prd{x:X}P(x)}f(\ast_X)=\ast_P
\end{equation*}
In the case of two pointed types $X$ and $Y$, we may also view $Y$ as a pointed family over $X$. In this case we write $X\to_\ast Y$ for the type of pointed functions.
\item Given any two pointed sections $f$ and $g$ of a pointed family $P$ over $X$, we define the type of pointed homotopies
\begin{equation*}
f\htpy_\ast g \defeq \Pi^\ast_{(x:X)} f(x)=g(x),
\end{equation*}
where the family $x\mapsto f(x)=g(x)$ is equipped with the base point $\ct{p}{q^{-1}}$. 
\end{enumerate}
\end{defn}

\begin{rmk}
  Since pointed homotopies are defined as certain pointed sections, we can use the same definition of pointed homotopies again to consider pointed homotopies between pointed homotopies, and so on.
\end{rmk}

\begin{comment}
\begin{rmk}
  Note that the type of pointed sections $\Pi^\ast_{(x:X)}P(x)$ is the fiber of the evaluation function at $\ast_X$. Thus, we have a pullback square
  \begin{equation*}
    \begin{tikzcd}
      \Pi^\ast_{(x:X)}P(x) \arrow[r] \arrow[d] & \unit \arrow[d,"\ast_P"] \\
      \prd{x:X}P(x) \arrow[r,swap,"\mathsf{ev}_{\ast_X}"] & P(\ast_X).
    \end{tikzcd}
  \end{equation*}
\end{rmk}
\end{comment}

\begin{eg}
  For any type $X$, the suspension $\susp{X}$ is a pointed type where the base point is taken to be the north pole $\N$.
\end{eg}

\begin{defn}
Let $X$ be a pointed type with base point $x$. We define the \define{loop space} $\loopspace{X,x}$ of $X$ at $x$ to be the pointed type $x=x$ with base point $\refl{x}$. 
\end{defn}

\begin{defn}
The loop space operation $\loopspacesym$ is \emph{functorial} in the sense that
\begin{enumerate}
\item For every pointed map $f:X\to_\ast Y$ there is a pointed map
\begin{equation*}
\loopspace{f}:\loopspace{X}\to_\ast \loopspace{Y},
\end{equation*}
defined by $\loopspace{f}(\omega)\defeq \ct{p_f}{\ap{f}{\omega}}{p_f^{-1}}$, which is base point preserving by $\mathsf{right\usc{}inv}(p_f)$. 
\item For every pointed type $X$ there is a pointed homotopy
\begin{equation*}
\loopspace{\mathsf{id}_X^\ast}\htpy_\ast \mathsf{id}^\ast_{\loopspace{X}}.
\end{equation*}
\item For any two pointed maps $f:X\to_\ast Y$ and $g:Y\to_\ast X$, there is a pointed homotopy witnessing that the triangle
\begin{equation*}
\begin{tikzcd}
& \loopspace{Y} \arrow[dr,"\loopspace{g}"] \\
\loopspace{X} \arrow[rr,swap,"\loopspace{g\circ_\ast f}"] \arrow[ur,"\loopspace{f}"] & & \loopspace{Z}
\end{tikzcd}
\end{equation*}
of pointed types commutes.
\end{enumerate}
\end{defn}

In order to introduce the suspension-loop space adjunction, we also need to construct the functorial action of suspension.

\begin{defn}
  \begin{enumerate}
  \item Given a pointed map $f:X\to_\ast Y$, we define a map
    \begin{equation*}
      \susp(f):\susp{X}\to_\ast\susp{Y}
    \end{equation*}
  \end{enumerate}
\end{defn}

\begin{defn}
  We define a pointed map
  \begin{equation*}
    \varepsilon_{X} : X \to_\ast \loopspace{\susp{X}}
  \end{equation*}
  for any pointed type $X$. This map is called the \define{counit} of the suspension-loop space adjunction. Moreover, $\varepsilon$ is natural in $X$ in the sense that for any pointed map $f:X\to_\ast Y$ we have a commuting square
  \begin{equation*}
    \begin{tikzcd}
      X \arrow[r,"\varepsilon_X"] \arrow[d,swap,"f"] & \loopspace{\susp{X}} \arrow[d,"\loopspace{\susp{f}}"] \\
      Y \arrow[r,swap,"\varepsilon_X"] & \loopspace{\susp{Y}}
    \end{tikzcd}
  \end{equation*}
\end{defn}

\begin{proof}[Construction]
  The underlying map of $\varepsilon_X$ takes $x:X$ to the concatenation
  \begin{equation*}
    \begin{tikzcd}
      \north \arrow[r,equals,"\merid(x)"] & \south \arrow[r,equals,"\merid(\ast_X)^{-1}"] & \north.
    \end{tikzcd}
  \end{equation*}
  This map preserves the base point, since $\ct{\merid(\ast_X)}{\merid(\ast_X)^{-1}}=\refl{\north}$. 
\end{proof}

\begin{defn}
\begin{enumerate}
\item For any pointed type $X$, we define the \define{pointed identity function} $\mathsf{id}^\ast_X\defeq (\idfunc[X],\refl{\ast})$. 
\item For any two pointed maps $f:X\to_\ast Y$ and $g:Y\to_\ast Z$, we define the \define{pointed composite}
\begin{equation*}
g\mathbin{\circ_\ast} f \defeq (g\circ f,\ct{\ap{g}{p_f}}{p_g}).
\end{equation*}
\end{enumerate}
\end{defn}

\begin{defn}
  Given two pointed types $X$ and $Y$, a pointed map from $X$ to $Y$ is a pair $(f,p)$ consisting of a map $f:X\to Y$ and a path $p:f(x_0)=y_0$ witnessing that $f$ preserves the base point. We write
  \begin{equation*}
    X\to_\ast Y
  \end{equation*}
  for the type of \define{pointed maps} from $X$ to $Y$. The type $X\to_\ast Y$ is itself a pointed type, with base point $(\mathsf{const}_{y_0},\refl{y_0})$. 
\end{defn}

Now suppose that we have a pointed map $f:\susp X\to_\ast Y$ with $p:f(x_0)=y_0$. Then the composite
\begin{equation*}
  \begin{tikzcd}
    X \arrow[r,"\varepsilon_X"] & \loopspace{\susp{X}} \arrow[r,"\loopspace{f}"] & \loopspace{Y}
  \end{tikzcd}
\end{equation*}
yields a pointed map $\tilde{f}:X\to\loopspace{Y}$. Therefore we obtain a map
\begin{equation*}
  \tau_{X,Y} :(\susp X \to_\ast Y) \to (X \to_\ast \loopspace{Y}).
\end{equation*}
It is not hard to see that also $\tau_{X,Y}$ is pointed. We leave this to the reader. The following theorem is also called the adjointness of the suspension and loop space functors. This is an extremely important relation that pops up in many calculations of homotopy groups.

\begin{thm}
  Let $X$ and $Y$ be pointed types. Then the pointed map
  \begin{equation*}
    \tau_{X,Y} : (\susp X \to_\ast Y) \to_\ast (X\to_\ast \loopspace Y)
  \end{equation*}
  is an equivalence. Moreover, $\tau$ is pointedly natural in $X$ and $Y$. 
\end{thm}

\subsection{Homotopy groups}

\begin{defn}
For $n\geq 1$, the \define{$n$-th homotopy group} of a type $X$ at a base point $x:X$ consists of the type
\begin{equation*}
|\pi_n(X,x)| \defeq \trunc{0}{\loopspace[n]{X,x}}
\end{equation*}
equipped with the group operations inherited from the path operations on $\loopspace[n]{X,x}$. 
Often we will simply write $\pi_n(X)$ when it is clear from the context what the base point of $X$ is.

For $n\jdeq 0$ we define $\pi_0(X,x)\defeq \trunc{0}{X}$. 
\end{defn}

\begin{eg}
In \cref{cor:circle_loopspace} we established that $\eqv{\loopspace{\sphere{1}}}{\Z}$. It follows that
\begin{equation*}
\pi_1(\sphere{1})=\Z \qquad\text{ and }\qquad\pi_n(\sphere{1})=0\qquad\text{for $n\geq 2$.}
\end{equation*}
Furthermore, we have seen in \cref{circle_conn} that $\trunc{0}{\sphere{1}}$ is contractible. 
Therefore we also have $\pi_0(\sphere{1})=0$.
\end{eg}

\subsection{The Eckmann-Hilton argument}

Given a diagram of identifications
\begin{equation*}
\begin{tikzcd}[column sep=7em]
x \arrow[r,equals,bend left=60,"p",""{name=A,below}] \arrow[r,equals,""{name=B},""{name=E,below},"{p'}"{near end}] \arrow[r,equals,bend right=60,"{p''}"{below},""{name=F,above}] \arrow[from=A,to=B,phantom,"r\Downarrow"] \arrow[from=E,to=F,phantom,"{r'\Downarrow}"] 
& y
\end{tikzcd}
\end{equation*}
in a type $A$, where $r:p=p'$ and $r':p'=p''$,
we obtain by concatenation an identification $\ct{r}{r'}:p=p''$. This operation on identifications of identifications is sometimes called the \define{vertical concatenation}, because there is also a \emph{horizontal} concatenation operation.

\begin{defn}
Consider identifications of identifications $r:p=p'$ and $s:q=q'$, where $p,p':x=y$, and $q,q':y=z$ are identifications in a type $A$, as indicated in the diagram
\begin{equation*}
\begin{tikzcd}[column sep=huge]
x \arrow[r,equals,bend left=30,"p",""{name=A,below}] \arrow[r,equals,bend right=30,""{name=B,above},"{p'}"{below}] \arrow[from=A,to=B,phantom,"r\Downarrow"] & y \arrow[r,equals,bend left=30,"q",""{name=C,below}] \arrow[r,equals,bend right=30,""{name=D,above},"{q'}"{below}] \arrow[from=C,to=D,phantom,"s\Downarrow"] & z.
\end{tikzcd}
\end{equation*}
We define the \define{horizontal concatenation} $\ct[h]{r}{s}:\ct{p}{q}=\ct{p'}{q'}$ of $r$ and $s$.
\end{defn}

\begin{proof}
First we induct on $r$, so it suffices to define $\ct[h]{\refl{p}}{s}:\ct{p}{q}=\ct{p}{q'}$. 
Next, we induct on $p$, so it suffices to define $\ct[h]{\refl{\refl{y}}}{s}:\ct{\refl{y}}{q}=\ct{\refl{y}}{q'}$. 
Since $\ct{\refl{y}}{q}\jdeq q$ and $\ct{\refl{y}}{q'}\jdeq q'$, we take $\ct[h]{\refl{\refl{y}}}{s}\defeq s$. 
\end{proof}

\begin{lem}
Horizontal concatenation satisfies the left and right unit laws.
\end{lem}

In the following lemma we establish the \define{interchange law} for horizontal and vertical concatenation.

\begin{lem}
Consider a diagram of the form
\begin{equation*}
\begin{tikzcd}[column sep=7em]
x \arrow[r,equals,bend left=60,"p",""{name=A,below}] \arrow[r,equals,""{name=B},""{name=E,below}] \arrow[r,equals,bend right=60,"{p''}"{below},""{name=F,above}] \arrow[from=A,to=B,phantom,"r\Downarrow"] \arrow[from=E,to=F,phantom,"{r'\Downarrow}"] 
& y \arrow[r,equals,bend left=60,"q",""{name=C,below}] \arrow[r,equals,""{name=G,above},""{name=H,below}] \arrow[r,equals,bend right=60,""{name=D,above},"{q''}"{below}] \arrow[from=C,to=G,phantom,"s\Downarrow"] \arrow[from=H,to=D,phantom,"{s'\Downarrow}"] & z.
\end{tikzcd}
\end{equation*}
Then there is an identification
\begin{equation*}
\ct[h]{(\ct{r}{r'})}{(\ct{s}{s'})}=\ct{(\ct[h]{r}{s})}{(\ct[h]{r'}{s'})}.
\end{equation*}
\end{lem}

\begin{proof}
We use path induction on both $r$ and $r'$, followed by path induction on $p$. Then it suffices to show that
\begin{equation*}
\ct[h]{(\ct{\refl{\refl{y}}}{\refl{\refl{y}}})}{(\ct{s}{s'})}=\ct{(\ct[h]{\refl{\refl{y}}}{s})}{(\ct[h]{\refl{\refl{y}}}{s'})}.
\end{equation*}
Using the computation rules, we see that this reduces to
\begin{equation*}
\ct{s}{s'}=\ct{s}{s'},
\end{equation*}
which we have by reflexivity.
\end{proof}

\begin{thm}
For $n\geq 2$, the $n$-th homotopy group is abelian.
\end{thm}

\begin{proof}
Our goal is to show that 
\begin{equation*}
\prd{r,s:\pi_2(X)} r\cdot s=s\cdot r.
\end{equation*}
Since we are constructing an identification in a set, we can use the universal property of $0$-truncation on both $r$ and $s$. Therefore it suffices to show that
\begin{equation*}
\prd{r,s:\refl{x_0}=\refl{x_0}} \tproj{0}r\cdot \tproj{0}s=\tproj{0}s\cdot \tproj{0}r.
\end{equation*}
Now we use that $\tproj{0}{r}\cdot\tproj{0}{s}\jdeq \tproj{0}{\ct{r}{s}}$ and $\tproj{0}{s}\cdot\tproj{0}{r}\jdeq \tproj{0}{\ct{s}{r}}$, to see that it suffices to show that $\ct{r}{s}=\ct{s}{r}$, for every $r,s:\refl{x}=\refl{x}$. Using the unit laws and the interchange law, this is a simple computation:
\begin{align*}
\ct{r}{s} & = \ct{(\ct[h]{r}{\refl{x}})}{(\ct[h]{\refl{x}}{s})} \\
& = \ct[h]{(\ct{r}{\refl{x}})}{(\ct{\refl{x}}{s})} \\
& = \ct[h]{(\ct{\refl{x}}{r})}{(\ct{s}{\refl{x}})} \\
& = \ct{(\ct[h]{\refl{x}}{s})}{(\ct[h]{r}{\refl{x}})} \\
& = \ct{s}{r}.\qedhere
\end{align*}
\end{proof}

\begin{exercises}
\exercise Show that the type of pointed families over a pointed type $(X,x)$ is equivalent to the type
\begin{equation*}
\sm{Y:\UU_\ast} Y\to_\ast X.
\end{equation*}
\exercise Given two pointed types $A$ and $X$, we say that $A$ is a (pointed) retract of $X$ if we have $i:A\to_\ast X$, a retraction $r:X\to_\ast A$, and a pointed homotopy $H:r\circ_\ast i\htpy_\ast \idfunc^\ast$. 
\begin{subexenum}
\item Show that if $A$ is a pointed retract of $X$, then $\loopspace{A}$ is a pointed retract of $\loopspace{X}$. 
\item Show that if $A$ is a pointed retract of $X$ and $\pi_n(X)$ is a trivial group, then $\pi_n(A)$ is a trivial group.
\end{subexenum}
\exercise Construct by path induction a family of maps
\begin{equation*}
\prd{A,B:\UU}{a:A}{b:B} (\id{\pairr{A,a}}{\pairr{B,b}})\to \sm{e:\eqv{A}{B}}e(a)=b,
\end{equation*}
and show that this map is an equivalence. In other words, an \emph{identification of pointed types} is a base point preserving equivalence.
\exercise Let $\pairr{A,a}$ and $\pairr{B,b}$ be two pointed types. Construct by path induction a family of maps
\begin{equation*}
\prd{f,g:A\to B}{p:f(a)=b}{q:g(a)=b} (\id{\pairr{f,p}}{\pairr{g,q}})\to \sm{H:f\htpy g} p = \ct{H(a)}{q},
\end{equation*}
and show that this map is an equivalence. In other words, an \emph{identification of pointed maps} is a base point preserving homotopy.
\exercise Show that if $A\leftarrow S\rightarrow B$ is a span of pointed types, then for any pointed type $X$ the square
\begin{equation*}
\begin{tikzcd}
(A\sqcup^S B \to_\ast X) \arrow[r] \arrow[d] & (B \to_\ast X) \arrow[d] \\
(A\to_\ast X) \arrow[r] & (S\to_\ast X)
\end{tikzcd}
\end{equation*}
is a pullback square.
\exercise \label{ex:yoneda_ptd_types}Let $f:A\to_\ast B$ be a pointed map. Show that the following are equivalent:
\begin{enumerate}
\item $f$ is an equivalence.
\item For any pointed type $X$, the precomposition map
\begin{equation*}
\blank\mathbin{\circ_\ast}f:(B\to_\ast X)\to_\ast (A\to_\ast X)
\end{equation*}
is an equivalence. 
\end{enumerate}
\exercise In this exercise we prove the suspension-loopspace adjunction.
\begin{subexenum}
\item Construct a pointed equivalence
\begin{equation*}
\tau_{X,Y}:(\susp(X)\to_\ast Y) \eqvsym_\ast (X\to \loopspace{Y})
\end{equation*}
for any two pointed spaces $X$ and $Y$.
\item Show that for any $f:X\to_\ast X'$ and $g:Y'\to_\ast Y$, there is a pointed homotopy witnessing that the square
\begin{equation*}
\begin{tikzcd}[column sep=large]
(\susp(X')\to_\ast Y') \arrow[r,"\tau_{X',Y'}"] \arrow[d,swap,"h\mapsto g\circ h\circ \susp(f)"] & (X'\to_\ast \loopspace{Y'}) \arrow[d,"h\mapsto\loopspace{g}\circ h\circ f"] \\
(\susp(X)\to_\ast Y) \arrow[r,swap,"\tau_{X,Y}"] & (X\to_\ast \loopspace{Y})
\end{tikzcd}
\end{equation*}
\end{subexenum}
\exercise Show that if
\begin{equation*}
\begin{tikzcd}
C \arrow[r] \arrow[d] & B \arrow[d] \\
A \arrow[r] & X
\end{tikzcd}
\end{equation*}
is a pullback square of pointed types, then so is
\begin{equation*}
\begin{tikzcd}
\loopspace{C} \arrow[r] \arrow[d] & \loopspace{B} \arrow[d] \\
\loopspace{A} \arrow[r] & \loopspace{X}.
\end{tikzcd}
\end{equation*}
\exercise 
\begin{subexenum}
\item Show that if $X$ is $k$-truncated, then its $n$-th homotopy group $\pi_n(X)$ is trivial for each choice of base point, and each $n> k$.
\item Show that if $X$ is $(k+l)$-truncated, and for each $0< i\leq l$ the $(k+i)$-th homotopy groups $\pi_{k+i}(X)$ are trivial for each choice of base point, then $X$ is $k$-truncated.
\end{subexenum}
It is consistent to assume that there are types for which all homotopy groups are trivial, but which aren't contractible nonetheless. Such types are called \define{$\infty$-connected}.
\exercise
  Consider a cospan
  \begin{equation*}
    \begin{tikzcd}
      A \arrow[r,"f"] & X & B \arrow[l,swap,"g"]
    \end{tikzcd}
  \end{equation*}
  of pointed types and pointed maps between them.
  \begin{subexenum}
  \item Define the type of pointed cones $\mathsf{cone}_\ast(C)$, where the vertex $C$ is a pointed type. Also characterize its identity type.
  \item Define for any pointed cone $(p,q,H)$ with vertex $C$ the map
    \begin{equation*}
      \mathsf{cone\usc{}map}_\ast(p,q,H) : (C' \to_\ast C) \to \mathsf{cone}_\ast(C').
    \end{equation*}
    Now we can say that the cone $(p,q,H)$ satisfies the universal property of the pointed pullback of the cospan $A\rightarrow X \leftarrow A$ if this map is an equivalence for each pointed type $C'$.
  \item Now consider a commuting square
    \begin{equation*}
      \begin{tikzcd}
        C \arrow[r,"q"] \arrow[d,swap,"p"] & B \arrow[d,"g"] \\
        A \arrow[r,swap,"f"] & X,
      \end{tikzcd}
    \end{equation*}
    where $f$ and $g$ are assumed to be pointed maps between pointed types (they come equipped with $\alpha : f(a_0)=x_0$ and $\beta : g(b_0)=x_0$, respectively). Show that if $C$ is a pullback (in the usual unpointed sense), then $C$ can be given the structure of a pointed pullback in a unique way, i.e., show that the type of
    \begin{align*}
      c_0 & : C \\
      \gamma & : p(c_0)=a_0 \\
      \delta & : q(c_0)=b_0 \\
      \varepsilon & : \ct{\ap{f}{\gamma}}{\alpha}=\ct{H(c_0)}{(\ct{\ap{g}{\delta}}{\beta})}
    \end{align*}
    for which $C$ satisfies the universal property of a pointed pullback, is contractible.
  \item Conclude that a commuting square of pointed types is a pointed pullback square if and only if the underlying square of unpointed types is an ordinary pullback square.
  \end{subexenum}
\end{exercises}
