\section{Cubical diagrams}

In order to proceed with the development of pullbacks and pushouts, it is useful to study commuting diagrams of the form
\begin{equation*}
  \begin{tikzcd}
    & C' \arrow[dl] \arrow[d] \arrow[dr] \\
    A' \arrow[d] & C \arrow[dl] \arrow[dr] & B' \arrow[dl,crossing over] \arrow[d] \\
    A \arrow[dr] & X' \arrow[from=ul,crossing over] \arrow[d] & B \arrow[dl] \\
    & X.
  \end{tikzcd}
\end{equation*}
In these diagrams there are six homotopies witnessing that the faces of the cube commute, as well as a homotopy of homotopies witnessing that the cube as a whole commutes.

Once the basic definitions of cubes are established, we focus on pullbacks and pushouts that appear in different configurations in these cubical diagrams. For example, if all the vertical maps in a commuting cube are equivalences, then the top square is a pullback square if and only if the bottom square is a pullback square. In \cref{chap:descent} we will use cubical diagrams in our formulation of the universality and descent theorems for pushouts.

In the first main theorem of this section we show that given a commuting cube in which the bottom square is a pullback square, the top square is a pullback square if and only if the induced square of fibers of the vertical maps is a pullback square. This theorem should be compared to \cref{cor:pb_fibequiv}, where we showed that a square is a pullback square if and only if it induces equivalences on the fibers of the vertical maps.

In our second main theorem we use the previous result to derive the 3-by-3 properties for pullbacks and pushouts.

\subsection{Commuting cubes}
\begin{defn}\label{defn:cube}
A \define{commuting cube}\index{commuting cube}
\begin{equation*}
\begin{tikzcd}[column sep=large,row sep=large]
& C' \arrow[dl,swap,"{p'}"] \arrow[dr,"{q'}"] \arrow[d,swap,"{h_C}" near end] \\
A' \arrow[d,swap,"{h_A}"] & C \arrow[dl,swap,"{p}" very near start] \arrow[dr,"{q}" very near start] & B' \arrow[dl,crossing over,"{g'}" near end] \arrow[d,"{h_B}"] \\
A \arrow[dr,swap,"f"] & X' \arrow[d,swap,"{h_X}" near start] \arrow[from=ul,crossing over,swap,"{f'}" near end] & B \arrow[dl,"{g}"] \\
& X
\end{tikzcd}
\end{equation*}
consists of types and maps as indicated in the diagram, equipped with
\begin{enumerate}
\item homotopies
  \begin{align*}
    \mathsf{top} & : f' \circ p' \htpy g' \circ q' \\
    \mathsf{back\usc{}left} & : p \circ h_C \htpy h_A \circ p' \\
    \mathsf{back\usc{}right} & : q \circ h_C \htpy h_B \circ q' \\
    \mathsf{front\usc{}left} & : f \circ h_A \htpy h_X \circ f' \\
    \mathsf{front\usc{}right} & : g \circ h_B \htpy h_X \circ g' \\
    \mathsf{bottom} & : f \circ p \htpy g \circ q
  \end{align*}
  witnessing that the 6 faces of the cube commute,
\item and a homotopy 
  \begin{align*}
    % ((((h ·l back-left) ∙h (front-left ·r f')) ∙h (hD ·l top))) ~
    % ((bottom ·r hA) ∙h ((k ·l back-right) ∙h (front-right ·r g')))
\mathsf{coh\usc{}cube} & : \ct{(\ct{(f \cdot \mathsf{back\usc{}left})}{(\mathsf{front\usc{}left}\cdot p')})}{(h_X \cdot \mathsf{top})} \\
& \qquad \htpy \ct{(\mathsf{bottom}\cdot h_C)}{(\ct{(g \cdot \mathsf{back\usc{}right})}{(\mathsf{front\usc{}right}\cdot q')})}
\end{align*}
filling the cube.
\end{enumerate}
\end{defn}

In the following lemma we show that if a cube commutes, then so do its rotations and mirror symmetries (that preserve the directions of the arrows).\footnote{The group acting on commuting cubes of maps is the \emph{dihedral group} $D_3$ which has order $6$.} This fact is obviously true, but there is some `path algebra' involved that we wish to demonstrate at least once.

\begin{lem}
  Consider a commuting cube
  \begin{equation*}
    \begin{tikzcd}
      & C' \arrow[dl] \arrow[d] \arrow[dr] \\
      A' \arrow[d] & C \arrow[dl] \arrow[dr] & B' \arrow[dl,crossing over] \arrow[d] \\
      A \arrow[dr] & X' \arrow[from=ul,crossing over] \arrow[d] & B \arrow[dl] \\
      & X.
    \end{tikzcd}
  \end{equation*}
  Then the cubes

  \begin{center}
  \begin{minipage}{.3\textwidth}
  \begin{equation*}
    \begin{tikzcd}
      & C' \arrow[dl] \arrow[d] \arrow[dr] \\
      C \arrow[d] & B' \arrow[dl] \arrow[dr] & A' \arrow[dl,crossing over] \arrow[d] \\
      B \arrow[dr] & A \arrow[from=ul,crossing over] \arrow[d] & X' \arrow[dl] \\
      & X
    \end{tikzcd}
  \end{equation*}
  \end{minipage}
  \begin{minipage}{.3\textwidth}
  \begin{equation*}
    \begin{tikzcd}
      & C' \arrow[dl] \arrow[d] \arrow[dr] \\
      B' \arrow[d] & A' \arrow[dl] \arrow[dr] & C \arrow[dl,crossing over] \arrow[d] \\
      X' \arrow[dr] & B \arrow[from=ul,crossing over] \arrow[d] & A \arrow[dl] \\
      & X
    \end{tikzcd}
  \end{equation*}
  \end{minipage}

  \begin{minipage}{.3\textwidth}
  \begin{equation*}
    \begin{tikzcd}
      & C' \arrow[dl] \arrow[d] \arrow[dr] \\
      C \arrow[d] & A' \arrow[dl] \arrow[dr] & B' \arrow[dl,crossing over] \arrow[d] \\
      A \arrow[dr] & B \arrow[from=ul,crossing over] \arrow[d] & X' \arrow[dl] \\
      & X.
    \end{tikzcd}
  \end{equation*}
  \end{minipage}
  \begin{minipage}{.3\textwidth}
  \begin{equation*}
    \begin{tikzcd}
      & C' \arrow[dl] \arrow[d] \arrow[dr] \\
      A' \arrow[d] & B' \arrow[dl] \arrow[dr] & C \arrow[dl,crossing over] \arrow[d] \\
      X' \arrow[dr] & A \arrow[from=ul,crossing over] \arrow[d] & B \arrow[dl] \\
      & X.
    \end{tikzcd}
  \end{equation*}
  \end{minipage}
  \begin{minipage}{.3\textwidth}
  \begin{equation*}
    \begin{tikzcd}
      & C' \arrow[dl] \arrow[d] \arrow[dr] \\
      B' \arrow[d] & C \arrow[dl] \arrow[dr] & A' \arrow[dl,crossing over] \arrow[d] \\
      B \arrow[dr] & X' \arrow[from=ul,crossing over] \arrow[d] & A \arrow[dl] \\
      & X.
    \end{tikzcd}
  \end{equation*}
  \end{minipage}
  \end{center}
  also commute.
\end{lem}

\begin{proof}
  We only show that the first cube commutes, which is obtained by a counter-clockwise rotation of the original cube around the axis through $C'$ and $X$. The other cases are similar, and they are formalized in the accompagnying Agda library.

  First we list the homotopies witnessing that the faces of the cube commute:
  \begin{align*}
    \mathsf{top}' & \defeq \mathsf{back\usc{}left} \\
    \mathsf{back\usc{}left}' & \defeq \mathsf{back\usc{}right}^{-1} \\
    \mathsf{back\usc{}right}' & \defeq \mathsf{top}^{-1} \\
    \mathsf{front\usc{}left}' & \defeq \mathsf{bottom}^{-1} \\
    \mathsf{front\usc{}right}' & \defeq \mathsf{front\usc{}left}^{-1} \\
    \mathsf{bottom}' & \defeq \mathsf{front\usc{}right}. 
  \end{align*}
  Thus, to show that the cube commutes, we have to show that there is a homotopy of type
  \begin{align*}
    & \ct{\Big(\ct{(g \cdot \mathsf{back\usc{}right}^{-1})}{(\mathsf{bottom}^{-1}\cdot h_C)}\Big)}{(f \cdot \mathsf{back\usc{}left})} \\
    & \qquad\qquad \htpy \ct{(\mathsf{front\usc{}right}\cdot q')}{\Big(\ct{(h_X \cdot \mathsf{top}^{-1})}{(\mathsf{front\usc{}left}^{-1}\cdot p')}\Big)}.
  \end{align*}
  Recall that $h\cdot H^{-1}\htpy (h\cdot H)^{-1}$ and $H^{-1}\cdot h\htpy (H\cdot h)^{-1}$, so it suffices to construct a homotopy
  \begin{align*}
    & \ct{\Big(\ct{(g \cdot \mathsf{back\usc{}right})^{-1}}{(\mathsf{bottom}\cdot h_C)^{-1}}\Big)}{(f \cdot \mathsf{back\usc{}left})} \\
    & \qquad\qquad \htpy \ct{(\mathsf{front\usc{}right}\cdot q')}{\Big(\ct{(h_X \cdot \mathsf{top})^{-1}}{(\mathsf{front\usc{}left}\cdot p')^{-1}}\Big)}.
  \end{align*}
  Now we note that pointwise, our goal is of the form
  \begin{equation*}
    \ct{(\ct{\varepsilon^{-1}}{\delta^{-1}})}{\alpha}=\ct{\zeta}{(\ct{\gamma^{-1}}{\beta^{-1}})}, %%% check greek alphabet
  \end{equation*}
  whereas the assumption that the original cube commutes yields an identification of the form
  \begin{equation*}
    \ct{(\ct{\alpha}{\beta})}{\gamma}=\ct{\delta}{(\ct{\varepsilon}{\zeta})}
  \end{equation*}
  Indeed, in the case that $\alpha$, $\beta$, $\gamma$, $\delta$, $\varepsilon$, and $\zeta$ are general identifications, we can conclude our goal using path induction on all of them.
\end{proof}

\begin{lem}
Given a commuting cube as in \cref{defn:cube} we obtain a commuting square
\begin{equation*}
\begin{tikzcd}
\fib{f_{1\check{1}1}}{x} \arrow[r] \arrow[d] & \fib{f_{0\check{1}1}}{f_{\check{1}01}(x)} \arrow[d] \\
\fib{f_{1\check{1}0}}{f_{10\check{1}}(x)} \arrow[r] & \fib{f_{0\check{1}0}}{f_{00\check{1}}(x)}
\end{tikzcd}
\end{equation*}
for any $x:A_{101}$. 
\end{lem}

\begin{lem}
Consider a commuting cube
\begin{equation*}
\begin{tikzcd}[column sep=large,row sep=large]
& C' \arrow[dl] \arrow[dr] \arrow[d] \\
A' \arrow[d] & C \arrow[dl] \arrow[dr] & B' \arrow[dl,crossing over] \arrow[d] \\
A \arrow[dr] & X' \arrow[d] \arrow[from=ul,crossing over] & B \arrow[dl] \\
& X,
\end{tikzcd}
\end{equation*}
If the bottom and front right squares are pullback squares, then the back left square is a pullback if and only if the top square is.
\end{lem}

\begin{rmk}\label{rmk:strongly-cartesian}
By rotating the cube we also obtain:
\begin{enumerate}
\item If the bottom and front left squares are pullback squares, then the back right square is a pullback if and only if the top square is.
\item If the front left and front right squares are pullback, then the back left square is a pullback if and only if the back right square is.
\end{enumerate}
By combining these statements it also follows that if the front left, front right, and bottom squares are pullback squares, then if any of the remaining three squares are pullback squares, all of them are. Cubes that consist entirely of pullback squares are sometimes called \define{strongly cartesian}\index{strongly cartesian cube}.
\end{rmk}

\subsection{Families of pullbacks}

\begin{lem}\label{lem:fiberwise-pullback}
Consider a pullback square\index{pullback!Sigma-type of pullbacks@{$\Sigma$-type of pullbacks}}
  \begin{equation*}
    \begin{tikzcd}
      C \arrow[r,"q"] \arrow[d,swap,"p"] & B \arrow[d,"g"] \\
      A \arrow[r,swap,"f"] & X
    \end{tikzcd}
  \end{equation*}
  with $H : f \circ p \htpy g \circ h$. Furthermore, consider type families $P_X$, $P_A$, $P_B$, and $P_C$ over $X$, $A$, $B$, and $C$ respectively, equipped with families of maps
  \begin{align*}
    f' & : \prd{a:A} P_A(a) \to P_X(f(a)) \\
    g' & : \prd{b:B} P_B(b) \to P_X(g(b)) \\
    p' & : \prd{c:C} P_C(c) \to P_A(p(c)) \\
    q' & : \prd{c:C} P_C(c) \to P_B(q(c)),
  \end{align*}
  and for each $c:C$ a homotopy $H'_c$ witnessing that the square
  \begin{equation}\label{eq:family-squares-pullback}
    \begin{tikzcd}
      P_C(c) \arrow[rr,"{q'_c}"] \arrow[d,swap,"{p'_c}"] & &[3em] P_B(q(c)) \arrow[d,"{g'_{q(c)}}"] \\
      P_A(p(c)) \arrow[r,swap,"{f'_{p(c)}}"] & P_X(f(p(c))) \arrow[r,swap,"{\tr_{P_X}(H(c))}"] & P_X(g(q(c)))
    \end{tikzcd}
  \end{equation}
  commutes. Then the following are equivalent:
  \begin{enumerate}
  \item For each $c:C$ the square in \cref{eq:family-squares-pullback} is a pullback square.
  \item The square
    \begin{equation}\label{eq:total-square-pullback}
      \begin{tikzcd}[column sep=huge]
        \sm{c:C}P_C(c)
        \arrow[r,"{\tot[q]{q'}}"] \arrow[d,swap,"{\tot[p]{p'}}"] &
        \sm{b:B}P_B(b) \arrow[d,"{\tot[g]{g'}}"] \\
        \sm{a:A}P_A(a) \arrow[r,swap,"{\tot[f]{f'}}"] & \sm{x:X}P_X(x)
      \end{tikzcd}
    \end{equation}
    is a pullback square.
  \end{enumerate}
\end{lem}


\begin{cor}
Consider a pullback square
\begin{equation*}
\begin{tikzcd}
C \arrow[r,"q"] \arrow[d,swap,"p"] & B \arrow[d,"g"] \\
A \arrow[r,swap,"f"] & X,
\end{tikzcd}
\end{equation*}
with $H:f\circ p\htpy g\circ q$, and let $c_1,c_2:C$. Then the square
\begin{equation*}
\begin{tikzcd}[column sep=8em]
(c_1=c_2) \arrow[r,"\apfunc{q}"] \arrow[d,swap,"\apfunc{p}"] & (q(c_1)=q(c_2)) \arrow[d,"\lam{\beta}\ct{H(c_1)}{\ap{g}{\beta}}"] \\
(p(c_1)=p(c_2)) \arrow[r,swap,"\lam{\alpha}\ct{\ap{f}{\alpha}}{H(c_2)}"] & f(p(c_1))=g(q(c_2)),
\end{tikzcd}
\end{equation*}
commutes and is a pullback square.
\end{cor}


\begin{thm}
  Consider a commuting cube
  \begin{equation*}
    \begin{tikzcd}
      & C' \arrow[dl] \arrow[dr] \arrow[d] \\
      A' \arrow[d] & C \arrow[dl] \arrow[dr] & B' \arrow[crossing over,dl] \arrow[d] \\
      A \arrow[dr] & X' \arrow[d] \arrow[from=ul,crossing over] & B \arrow[dl] \\
      & X
    \end{tikzcd}
  \end{equation*}
  in which the bottom square is a pullback square. Then the following are equivalent:
  \begin{enumerate}
  \item The top square is a pullback square.
  \item The square
    \begin{equation*}
      \begin{tikzcd}
        \fib{\gamma}{c} \arrow[d] \arrow[r] & \fib{\beta}{q(c)} \arrow[d] \\
        \fib{\alpha}{p(c)} \arrow[r] & \fib{\varphi}{f(p(c))}
      \end{tikzcd}
    \end{equation*}
    is a pullback square for each $c:C$.
  \end{enumerate}
\end{thm}


\subsection{The 3-by-3-properties for pullbacks and pushouts}

\begin{thm}
  Consider a commuting diagram of the form
  \begin{equation*}
    \begin{tikzcd}[column sep=large,row sep=large]
      AA \arrow[r,"Af"] \arrow[d,swap,"fA"] \arrow[dr,phantom,"\Rightarrow" description] & AX \arrow[d,swap,"fX"] & AB \arrow[l,swap,"Ag"] \arrow[d,"gB"] \arrow[dl,phantom,"\Leftarrow" description] \\
      XA \arrow[r,"Xf"] & XX & XB \arrow[l,swap,"Xg"] \\
      BA \arrow[u,"gA"] \arrow[r,swap,"Bf"] & BX \arrow[u,"gX"] & BB \arrow[u,swap,"gB"] \arrow[l,"Bg"]
    \end{tikzcd}
  \end{equation*}
  with homotopies
  \begin{align*}
    ff & : Xf \circ fA \htpy Af \circ fX \\
    fg & : Xg \circ gB \htpy Ag \circ fX \\
    gf & : 
  \end{align*}
  filling the (small) squares. Furthermore, consider
  pullback squares
  \begin{equation*}
    \begin{tikzcd}
      AC \arrow[r] \arrow[d] & AB \arrow[d] & XC \arrow[r] \arrow[d] & XB \arrow[d] & BC \arrow[r] \arrow[d] & BB \arrow[d] \\
      AA \arrow[r] & AX & XA \arrow[r] & XX & BA \arrow[r] & BX
    \end{tikzcd}
  \end{equation*}
  \begin{equation*}
    \begin{tikzcd}
      CA \arrow[r] \arrow[d] & BA \arrow[d] & CX \arrow[r] \arrow[d] & BX \arrow[d] & CB \arrow[r] \arrow[d] & BB \arrow[d] \\
      AA \arrow[r] & XA & AX \arrow[r] & XX & AB \arrow[r] & XB.
    \end{tikzcd}
  \end{equation*}
  Finally, consider a commuting square
  \begin{equation*}
    \begin{tikzcd}
      D_3 \arrow[r] \arrow[d] & D_2 \arrow[d] \\
      D_0 \arrow[r] & D_1.
    \end{tikzcd}
  \end{equation*}
  Then the following are equivalent:
  \begin{enumerate}
  \item This square is a pullback square.
  \item The induced square
    \begin{equation*}
      \begin{tikzcd}
        D_3 \arrow[r] \arrow[d] & C_3 \arrow[d] \\
        A_3 \arrow[r] & B_3
      \end{tikzcd}
    \end{equation*}
    is a pullback square.
  \end{enumerate}
\end{thm}

\begin{proof}
  First we construct an equivalence
  \begin{equation*}
    (A_0\times_{B_0}C_0)\times_{(A_1\times_{B_1}C_1)}(A_2\times_{B_2} C_2) \eqvsym (A_0\times_{A_1}A_2)\times_{(B_0\times_{B_1} B_2)} (C_0\times_{C_1}C_2).
  \end{equation*}
  Now it follows that we have an equivalence
  \begin{equation*}
    \mathsf{cone}(f_0,g_0)
  \end{equation*}
\end{proof}


\begin{exercises}
\item Some exercises.
\end{exercises}
