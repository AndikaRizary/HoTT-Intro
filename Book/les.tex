\section{The long exact sequence of homotopy groups}
\sectionmark{The long exact sequence}

\subsection{The long exact sequence}

\begin{defn}
A fiber sequence $F\hookrightarrow E \twoheadrightarrow B$ consists of:
\begin{enumerate}
\item Pointed types $F$, $E$, and $B$, with base points $x_0$, $y_0$, and $b_0$ respectively, 
\item Base point preserving maps $i:F\to_\ast E$ and $p:E\to_\ast B$, with $\alpha:i(x_0)=y_0$ and $\beta:p(y_0)=b_0$,
\item A pointed homotopy $H:\mathsf{const}_{b_0}\htpy_\ast p\circ_\ast i$ witnessing that the square
\begin{equation*}
\begin{tikzcd}
F \arrow[r,"i"] \arrow[d] & E \arrow[d,"p"] \\
\unit \arrow[r,swap,"\mathsf{const}_{b_0}"] & B,
\end{tikzcd}
\end{equation*}
commutes and is a pullback square.
\end{enumerate}
\end{defn}

\begin{lem}
Any fiber sequence $F\hookrightarrow E\twoheadrightarrow B$ induces a sequence of pointed maps
\begin{equation*}
\begin{tikzcd}
\loopspace{F} \arrow[r,"\loopspace{i}"] & \loopspace{E} \arrow[r,"\loopspace{p}"] & \loopspace{B} \arrow[r,"\partial"] & F \arrow[r,"i"] & E \arrow[r,"p"] & B,
\end{tikzcd}
\end{equation*}
in which every two consecutive maps form a fiber sequence.
\end{lem}

\begin{proof}
By taking pullback squares repeatedly, we obtain the diagram
\begin{equation*}
\begin{gathered}[b]
\begin{tikzcd}[column sep=large]
\loopspace{F} \arrow[d,swap,"\loopspace{i}"] \arrow[r] & \unit \arrow[d,"\mathsf{const}_{\refl{b_0}}"] \\
\loopspace{E} \arrow[r,"\loopspace{p}"] \arrow[d] & \loopspace{B} \arrow[r] \arrow[d,swap,"\partial"] & \unit \arrow[d,"\mathsf{const}_{y_0}"] \\
\unit \arrow[r,swap,"\mathsf{const}_{x_0}"] & F \arrow[r,"i"] \arrow[d] & E \arrow[d,"p"] \\
& \unit \arrow[r,swap,"\mathsf{const}_{b_0}"] & B.
\end{tikzcd}\\[-\dp\strutbox]
\end{gathered}\qedhere
\end{equation*}
\end{proof}

\begin{defn}
We say that a consecutive pair of pointed maps between pointed sets
\begin{equation*}
\begin{tikzcd}
A \arrow[r,"f"] & B \arrow[r,"g"] & C
\end{tikzcd}
\end{equation*}
is \define{exact} at $B$ if we have
\begin{equation*}
\Big(\exis{a:A}f(a)=b\Big)\leftrightarrow (g(b)=c)
\end{equation*}
for any $b:B$. 
\end{defn}

\begin{rmk}
If a pair of consecutive pointed maps between pointed sets
\begin{equation*}
\begin{tikzcd}
A \arrow[r,"f"] & B \arrow[r,"g"] & C
\end{tikzcd}
\end{equation*}
is exact at $B$, it directly that $\im(f)=\fib{g}{c}$. Indeed, such a pair of pointed maps is exact at $B$ if and only if there is an equivalence $e:\im(f)\eqvsym \fib{g}{c}$ such that the triangle
\begin{equation*}
\begin{tikzcd}[column sep=tiny]
\im(f) \arrow[dr] \arrow[rr,"e"] & & \fib{g}{c} \arrow[dl] \\
& B
\end{tikzcd}
\end{equation*}
commutes. In other words, $\im(f)$ and $\fib{g}{c}$ are equal \emph{as subsets of $B$}.
\end{rmk}

\begin{lem}
Suppose $F\hookrightarrow E \twoheadrightarrow B$ is a fiber sequence. Then the sequence
\begin{equation*}
\begin{tikzcd}
\trunc{0}{F} \arrow[r,"\trunc{0}{i}"] & \trunc{0}{E} \arrow[r,"\trunc{0}{p}"] & \trunc{0}{B}
\end{tikzcd}
\end{equation*}
is exact at $\trunc{0}{E}$. 
\end{lem}

\begin{proof}
To show that the image $\im\trunc{0}{i}$ is the fiber $\fib{\trunc{0}{p}}{\tproj{0}{b_0}}$, it suffices to construct a fiberwise equivalence
\begin{equation*}
\prd{x:\trunc{0}{E}} \trunc{-1}{\fib{\trunc{0}{i}}{x}} \eqvsym \trunc{0}{p}(x)=\tproj{0}{b_0}.
\end{equation*}
By the universal property of $0$-truncation it suffices to show that
\begin{equation*}
\prd{x:E} \trunc{-1}{\fib{\trunc{0}{i}}{\tproj{0}{x}}} \eqvsym \trunc{0}{p}(\tproj{0}{x})=\tproj{0}{b_0}.
\end{equation*}
First we note that 
\begin{align*}
\trunc{0}{p}(\tproj{0}{x})=\tproj{0}{b_0} & \eqvsym \tproj{0}{p(x)} = \tproj{0}{b_0} \\
& \eqvsym \trunc{-1}{p(x)=b_0}.
\end{align*}
Next, we note that
\begin{align*}
\fib{\trunc{0}{i}}{\tproj{0}{x}} & \eqvsym \sm{y:\trunc{0}{F}}\trunc{0}{i}(y)=\tproj{0}{x} \\
& \eqvsym \trunc{0}{\sm{y:F}\trunc{0}{i}(\tproj{0}{y})=\tproj{0}{x}} \\
& \eqvsym \trunc{0}{\sm{y:F}\tproj{0}{i(y)}=\tproj{0}{x}} \\
& \eqvsym \trunc{0}{\sm{y:F}\trunc{-1}{i(y)=x}}.
\end{align*}
Therefore it follows that
\begin{align*}
\trunc{-1}{\fib{\trunc{0}{i}}{\tproj{0}{x}}} & \eqvsym \trunc{-1}{\sm{y:F}\trunc{-1}{i(y)=x}} \\
& \eqvsym \trunc{-1}{\sm{y:F}i(y)=x} \\
\end{align*}
Now it suffices to show that $\eqv{\big(\sm{y:F}i(y)=x\big)}{p(x)=b_0}$. This follows by the pasting lemma of pullbacks
\begin{equation*}
\begin{tikzcd}
(p(x)=b_0) \arrow[r] \arrow[d] & \unit \arrow[d] \\
F \arrow[r] \arrow[d] & E \arrow[d] \\
\unit \arrow[r] & B
\end{tikzcd}
\end{equation*}
\end{proof}

\begin{thm}
Any fiber sequence $F\hookrightarrow E\twoheadrightarrow B$ induces a long exact sequence on homotopy groups
\begin{equation*}
\begin{tikzcd}
& & \cdots \arrow[out=355,in=175,dll] \\
\pi_n(F) \arrow[r,"\pi_n(i)"] & \pi_n(E) \arrow[r,"\pi_n(p)"] & \pi_n(B) \arrow[out=355,in=175,dll,densely dotted] \\
\pi_1(F) \arrow[r,"\pi_1(i)"] & \pi_1(E) \arrow[r,"\pi_1(p)"] & \pi_1(B) \arrow[out=355,in=175,dll] \\
\pi_0(F) \arrow[r,"\pi_0(i)"] & \pi_0(E) \arrow[r,"\pi_0(p)"] & \pi_0(B)
\end{tikzcd}
\end{equation*}
\end{thm}

\subsection{The Hopf fibration}
Our goal in this section is to construct the Hopf fibration, i.e.~a fiber sequence
\begin{equation*}
\sphere{1}\hookrightarrow\sphere{3}\twoheadrightarrow\sphere{2}.
\end{equation*}
This fiber sequence involves the complex multiplication of the unit sphere in the complex number, which is a circle. Viewing the circle as a subspace of the complex numbers, we write $1$ for the base point of the circle.

\begin{defn}
We define the \define{complex multiplication} operation
\begin{equation*}
\mu_{\mathbb{C}}:\sphere{1}\to(\sphere{1}\to\sphere{1}).
\end{equation*}
\end{defn}

\begin{constr}
By the universal property of the circle, it is equivalent to define
\begin{align*}
\mu_{\mathbb{C}}(1) & : \sphere{1}\to\sphere{1} \\
\ap{\mu_{\mathbb{C}}}{\lloop} & : \mu_{\mathbb{C}}(1)=\mu_{\mathbb{C}}(1). 
\end{align*}
The function $\mu_{\mathbb{C}}(1)$ is multiplication by $1$, which is the identity function. The type of $\ap{\mu_{\mathbb{C}}}{\lloop}$ is equivalent to the type of homotopies
\begin{equation*}
\idfunc[\sphere{1}] \htpy \idfunc[\sphere{1}]. 
\end{equation*}
We construct this homotopy by induction on $\sphere{1}$. Therefore it suffices to construct
\begin{align*}
p & : 1=1 \\
q & : \mathsf{tr}_{L}(\lloop,p)=p
\end{align*}
\end{constr}

\begin{lem}
The complex multiplication operation $\mu_{\mathbb{C}}$ on the circle satisfies the unit laws
\begin{align*}
\mathsf{left\usc{}unit}_{\mathbb{C}}(x) & : \mu_{\mathbb{C}}(1,x) = x \\
\mathsf{right\usc{}unit}_{\mathbb{C}}(x) & : \mu_{\mathbb{C}}(x,1) = x \\
\mathsf{coh\usc{}unit}_{\mathbb{C}} & : \mathsf{left\usc{}unit}_{\mathbb{C}}(1)=\mathsf{right\usc{}unit}_{\mathbb{C}}(1),
\end{align*}
and the functions $\mu_{\mathbb{C}}(x,\blank)$ and $\mu_{\mathbb{C}}(\blank,y)$ are equivalences for each $x:\sphere{1}$ and $y:\sphere{1}$, respectively.
\end{lem}

\begin{lem}
Both commuting squares in the diagram
\begin{equation*}
\begin{tikzcd}
\sphere{1} \arrow[d] & \sphere{1}\times\sphere{1} \arrow[d,swap,"\mu_{\mathbb{C}}"] \arrow[l,swap,"\proj 1"] \arrow[r,"\proj 2"] & \sphere{1} \arrow[d] \\
\unit & \sphere{1} \arrow[l] \arrow[r] & \unit
\end{tikzcd}
\end{equation*}
are pullback squares.
\end{lem}

\begin{cor}
There is a fiber sequence
\begin{equation*}
\sphere{1}\hookrightarrow \join{\sphere{1}}{\sphere{1}} \twoheadrightarrow \sphere{2}.
\end{equation*}
\end{cor}

\begin{lem}
The join operation is associative
\end{lem}

\begin{proof}
\begin{equation*}
\begin{tikzcd}
A & A\times C \arrow[l] \arrow[r] & A\times C \\
A\times B \arrow[u] \arrow[d] & A\times B \times C \arrow[r] \arrow[d] \arrow[l] \arrow[u] & A\times C \arrow[u] \arrow[d] \\
B & B\times C \arrow[l] \arrow[r] & C
\end{tikzcd}
\end{equation*}
\end{proof}

\begin{cor}
There is an equivalence $\eqv{\join{\sphere{1}}{\sphere{1}}}{\sphere{3}}$.
\end{cor}

\begin{thm}
There is a fiber sequence $\sphere{1}\hookrightarrow\sphere{3}\twoheadrightarrow\sphere{2}$. 
\end{thm}

\begin{lem}
Suppose $f:G\to H$ is a group homomorphism, such that the sequence
\begin{equation*}
\begin{tikzcd}
0 \arrow[r] & G \arrow[r,"f"] & H \arrow[r] & 0
\end{tikzcd}
\end{equation*}
is exact at $G$ and $H$, where we write $0$ for the trivial group consisting of just the unit element. Then $f$ is a group isomorphism.
\end{lem}

\begin{cor}
We have $\pi_2(\sphere{2})=\Z$, and for $k>2$ we have $\pi_k(\sphere{2})=\pi_k(\sphere{3})$.
\end{cor}


\begin{exercises}
\item Give the $0$-sphere $\sphere{0}$ the structure of an H-space.
\item For any pointed type $A$, give $\loopspace{A}$ the structure of an H-space.
\item Show that the type of (small) fiber sequences is equivalent to the type of quadruples $(B,P,b_0,x_0)$, consisting of
\begin{align*}
B & : \UU \\
P & : B \to \UU \\
b_0 & : B \\
x_0 & : P(b_0).
\end{align*}
\end{exercises}
