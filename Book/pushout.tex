\section{Homotopy pushouts}

A common way in topology to construct new spaces is by attaching cells\index{attaching cells} to a given space. A $0$-cell is just a point, an $1$-cell is an interval, a $2$-cell is a disc, a $3$-cell is the solid ball, and so forth. Many spaces can be obtained by attaching cells. For example, the circle\index{circle} is obtained by attaching a $1$-cell to a $0$-cell, so that both end-points of the interval are mapped to the point. More generally, an $n$-sphere is obtained by attaching an $n$-disc to the point, so that its entire boundary gets mapped to the point.

In type theory we can also consider a notion of $n$-cells. Just as in topology, a $0$-cell is just a point (i.e., a term). A $1$-cell, however, is in type theory an identification, i.e., a term of the identity type. A $1$-cell is then an identification of identifications, and so forth. Then we can attach cells to a type by taking a pushout, which is a process dual to taking a pullback. 

The idea of pushouts is to glue two types $A$ and $B$ together using a mediating type $S$ and maps $f:S\to A$ and $g:S\to B$. In other words, we start with a diagram of the form
\begin{equation*}
\begin{tikzcd}
A & S \arrow[l,swap,"f"] \arrow[r,"g"] & B.
\end{tikzcd}
\end{equation*}
We call such a triple $\mathcal{S}\jdeq (S,f,g)$ a \define{span}\index{span} from $A$ to $B$.
A span from $A$ to $B$ can be thought of as a relation\index{relation} from $A$ to $B$, relating $f(s)$ to $g(s)$ for any $s:S$. The pushout of the span $\mathcal{S}$ is then a type $X$ that comes equipped with inclusion maps $i:A\to X$ and $j:B\to X$ and a homotopy $H$ witnessing that the square
\begin{equation*}
  \begin{tikzcd}
    S \arrow[d,swap,"f"] \arrow[r,"g"] & B \arrow[d,"j"] \\
    A \arrow[r,swap,"i"] & X
  \end{tikzcd}
\end{equation*}
Note that this homotopy makes sure that there is a path $H(s):i(f(s))=j(g(s))$ for every $s:S$. In other words, any $x:A$ and $y:B$ that are related by in $\mathcal{S}$ become identified in the pushout. The last requirement of the pushout is that it satisfies a universal property that is dual to the universal property of pullbacks.

There are several equivalent characterizations of pushouts. Two such characterizations are studied in this section, establishing the duality between pullbacks and pushouts. Other characterizations, including the induction principle of pushouts, and the \emph{dependent universal property} of pushouts, are studied in \cref{chap:descent}.

Unlike pullbacks, however, it is not automatically the case that pushouts always exist. We will therefore postulate as an axiom that pushouts always exist. Moreover, we will assume that universes are closed under pushouts.

\subsection{The universal property of pushouts}

\begin{defn}
Consider a span $\mathcal{S}\jdeq (S,f,g)$ from $A$ to $B$, and let $X$ be a type.
A \define{cocone}\index{cocone} with vertex $X$ on $\mathcal{S}$ is a triple $(i,j,H)$ consisting of maps $i:A\to X$ and $j:B\to X$, and a homotopy $H:i\circ f\htpy j\circ g$ witnessing that the square
\begin{equation*}
\begin{tikzcd}
S \arrow[r,"g"] \arrow[d,swap,"f"] & B \arrow[d,"j"] \\
A \arrow[r,swap,"i"] & X
\end{tikzcd}
\end{equation*}
commutes.
We write $\mathsf{cocone}_{\mathcal{S}}(X)$\index{cocone_S(X)@{$\mathsf{cocone}_{\mathcal{S}}(X)$}} for the type of cocones on $\mathcal{S}$ with vertex $X$.
\end{defn}

\begin{rmk}\label{rmk:htpy-cocone}
  Given two cocones $(i,j,H)$ and $(i',j',H')$ with vertex $X$, the type of identifications $(i,j,H)=(i',j',H')$ in $\mathsf{cocone}_{\mathcal{S}}(X)$ is equivalent to the type of triples $(K,L,M)$ consisting of
  \begin{align*}
    K : i\htpy i' \\
    L : j\htpy j',
  \end{align*}
  and a homotopy $M$ witnessing that the square
  \begin{equation*}
    \begin{tikzcd}
      %  ((pr2 (pr2 c)) ∙h (L ·r g)) ~ ((K ·r f) ∙h (pr2 (pr2 c')))
      i\circ f \arrow[d,swap,"H"] \arrow[r,"K\cdot f"] & i' \circ f \arrow[d,"{H'}"] \\
      j\circ g \arrow[r,swap,"L\cdot g"] & j'\circ g
    \end{tikzcd}
  \end{equation*}
  of homotopies commutes.
\end{rmk}

\begin{defn}
Consider a cocone $(i,j,H)$ with vertex $X$ on the span $\mathcal{S}\jdeq (S,f,g)$, as indicated in the following commuting square
\begin{equation*}
\begin{tikzcd}
S \arrow[r,"g"] \arrow[d,swap,"f"] & B \arrow[d,"j"] \\
A \arrow[r,swap,"i"] & X.
\end{tikzcd}
\end{equation*}
For every type $Y$, we define the map\index{cocone_map@{$\mathsf{cocone\usc{}map}$}}
\begin{equation*}
\mathsf{cocone\usc{}map}(i,j,H):(X\to Y)\to \mathsf{cocone}(Y)
\end{equation*}
by $h\mapsto (h\circ i,h\circ j,h\cdot H)$.
\end{defn}

\begin{defn}
  A commuting square
  \begin{equation*}
    \begin{tikzcd}
      S \arrow[r,"g"] \arrow[d,swap,"f"] & B \arrow[d,"j"] \\
      A \arrow[r,swap,"i"] & X.
    \end{tikzcd}
  \end{equation*}
  with $H:i\circ f \htpy j\circ g$ is said to be a \define{(homotopy) pushout square}\index{pushout square} if the cocone $(i,j,H)$ with vertex $X$ on the span $\mathcal{S}\jdeq (S,f,g)$
  satisfies the \define{universal property of pushouts}\index{universal property!of pushouts}, which asserts that the map
  \begin{equation*}
    \mathsf{cocone\usc{}map}(i,j,H):(X\to Y)\to \mathsf{cocone}(Y)
  \end{equation*}
  is an equivalence for any type $Y$. Sometimes pushout squares are also called \define{cocartesian squares}\index{cocartesian square}.
\end{defn}

\begin{lem}\label{lem:unique-mapping-property-pushout}
  Consider a pushout square
  \begin{equation*}
    \begin{tikzcd}
      S \arrow[r,"g"] \arrow[d,swap,"f"] & B \arrow[d,"j"] \\
      A \arrow[r,swap,"i"] & X.
    \end{tikzcd}
  \end{equation*}
  with $H:i\circ f \htpy j\circ g$, and consider a commuting square
  \begin{equation*}
    \begin{tikzcd}
      S \arrow[r,"g"] \arrow[d,swap,"f"] & B \arrow[d,"{j'}"] \\
      A \arrow[r,swap,"{i'}"] & X'.
    \end{tikzcd}
  \end{equation*}
  with $H':i'\circ f \htpy j'\circ g$. Then the type of maps $h:X\to X'$ equipped with homotopies
  \begin{align*}
    K & : h\circ i \htpy i' \\
    L & : h\circ j \htpy j'
  \end{align*}
  and a homotopy $M$ witnessing that the square
  \begin{equation*}
    \begin{tikzcd}
      h\circ i\circ f \arrow[r,"K\cdot f"] \arrow[d,swap,"h\cdot H"] & i' \circ f \arrow[d,"{H'}"] \\
      h\circ j\circ g \arrow[r,swap,"L\cdot g"] & j'\circ g
    \end{tikzcd}
  \end{equation*}
  commutes, is contractible.
\end{lem}

\begin{proof}
  For any map $h:X\to X$', the type of triples $(K,L,M)$ as in the statement of the lemma is equivalent to the type of identifications
  \begin{equation*}
    \mathsf{cocone\usc{}map}((i,j,H),h)=(i',j',H'),
  \end{equation*}
  by \cref{rmk:htpy-cocone}. Therefore it follows that the type of quadruples $(h,K,L,M)$ is equivalent to the fiber of $\mathsf{cocone\usc{}map}(i,j,H)$ at $(i',j',H')$. Since we have assumed that the cocone $(i,j,H)$ satisfies the universal property of the pushout of $\mathcal{S}$, the map $\mathsf{cocone\usc{}map}(i,j,H)$ is an equivalence, and therefore it has contractible fibers by \cref{thm:contr_equiv}.
\end{proof}

\begin{thm}\label{thm:3-for-2-pushout}
  Consider two cocones
  \begin{equation*}
    \begin{tikzcd}
      S \arrow[r,"g"] \arrow[d,swap,"f"] & B \arrow[d,"j"]
      & & S \arrow[r,"g"] \arrow[d,swap,"f"] & B \arrow[d,"{j'}"] \\
      A \arrow[r,swap,"i"] & X
      & & A \arrow[r,swap,"{i'}"] & X'
    \end{tikzcd}
  \end{equation*}
  on a span $\mathcal{S}\jdeq(S,f,g)$, and let $h:X\to X'$ be a map equipped with homotopies
  \begin{align*}
    K & : h\circ i \htpy i' \\
    L & : h\circ j \htpy j'
  \end{align*}
  and a homotopy $M$ witnessing that the square
  \begin{equation*}
    \begin{tikzcd}
      h\circ i\circ f \arrow[r,"K\cdot f"] \arrow[d,swap,"h\cdot H"] & i' \circ f \arrow[d,"{H'}"] \\
      h\circ j\circ g \arrow[r,swap,"L\cdot g"] & j'\circ g
    \end{tikzcd}
  \end{equation*}
  commutes. Then if any two of the following three statements hold, so does the third:
  \begin{enumerate}
  \item The cocone $(i,j,H)$ satisfies the universal property of the pushout of $\mathcal{S}$.
  \item The cocone $(i',j',H')$ satisfies the universal property of the pushotu of $\mathcal{S}$.
  \item The map $h$ is an equivalence.
  \end{enumerate}
\end{thm}

\begin{proof}
  First we observe that we have a commuting triangle
  \begin{equation*}
    \begin{tikzcd}[column sep=0]
      (X'\to Y) \arrow[rr,"\blank\circ h"]
      \arrow[dr,swap,"{\mathsf{cocone\usc{}map}(i',j',H')}"]
      & & (X\to Y) \arrow[dl,"{\mathsf{cocone\usc{}map}(i,j,H)}"] \\
      & \mathsf{cocone}_{\mathcal{S}}(Y) & \phantom{(X'\to Y)}
    \end{tikzcd}
  \end{equation*}
  for any type $Y$. Therefore it follows from the 3-for-2 property of equivalences that if any two of the maps in this triangle is an equivalence, so is the third. Now the claim follows from the observation in \cref{ex:equiv_precomp} that $h$ is an equivalence if and only if the map $\blank\circ h:(X'\to Y)\to (X\to Y)$ is an equivalence for any type $Y$.
\end{proof}

In the following corollary we establish the fact that pushouts are \emph{uniquely unique}.

\begin{cor}
  Consider two pushouts
  \begin{equation*}
    \begin{tikzcd}
      S \arrow[r,"g"] \arrow[d,swap,"f"] & B \arrow[d,"j"]
      & & S \arrow[r,"g"] \arrow[d,swap,"f"] & B \arrow[d,"{j'}"] \\
      A \arrow[r,swap,"i"] & X
      & & A \arrow[r,swap,"{i'}"] & X'
    \end{tikzcd}
  \end{equation*}
  of a given span $\mathcal{S}\jdeq (S,f,g)$. Then the type of equivalences $e:\eqv{X}{X'}$ equipped with homotopies
  \begin{align*}
    K & : h\circ i \htpy i' \\
    L & : h\circ j \htpy j'
  \end{align*}
  and a homotopy $M$ witnessing that the square
  \begin{equation*}
    \begin{tikzcd}
      h\circ i\circ f \arrow[r,"K\cdot f"] \arrow[d,swap,"h\cdot H"] & i' \circ f \arrow[d,"{H'}"] \\
      h\circ j\circ g \arrow[r,swap,"L\cdot g"] & j'\circ g
    \end{tikzcd}
  \end{equation*}
  commutes, is contractible.
\end{cor}

\begin{proof}
  This follows from combining \cref{lem:unique-mapping-property-pushout,thm:3-for-2-pushout}.
\end{proof}

\begin{cor}\label{cor:uniquely-unique-pushout}
  Consider a span
  \begin{equation*}
    \begin{tikzcd}
      A & S \arrow[l,swap,"f"] \arrow[r,"g"] & B
    \end{tikzcd}
  \end{equation*}
  in a universe $\UU$. Then the type
  \begin{equation*}
    \sm{X:\UU}{c:\mathsf{cocone}(X)}\prd{Y:\UU}\mathsf{is\usc{}equiv}(\mathsf{cocone\usc{}map}_Y(c))
  \end{equation*}
  of is a proposition.
\end{cor}

\begin{proof}
  It is routine to verify that the type of quadruples $(e,K,L,M)$ as in \cref{cor:uniquely-unique-pushout} is equivalent to the identity type of the type of pushouts of the span $\mathcal{S}\jdeq (S,f,g)$. The claim then follows, since \cref{cor:uniquely-unique-pushout} asserts that this type of quadruples is contractible. 
\end{proof}

\subsection{Suspensions}
A particularly important class of examples of pushouts are suspensions.

\begin{defn}
  Let $X$ be a type. A \define{suspension}\index{suspension} of $X$ is a type $\susp X$ equipped with a \define{north pole} $\mathsf{N}:\susp X$, a \define{south pole} $\mathsf{S}:\susp X$, and a \define{meridian}
  \begin{equation*}
    \mathsf{merid} : X \to (\mathsf{N}=\mathsf{S}),
  \end{equation*}
  such that the commuting square
  \begin{equation*}
    \begin{tikzcd}
      X \arrow[r,"\mathsf{const}_\ttt"] \arrow[d,swap,"\mathsf{const}_\ttt"] & \unit \arrow[d,"{\mathsf{const}_\south}"] \\
      \unit \arrow[r,swap,"\mathsf{const}_\north"] & \susp X
    \end{tikzcd}
  \end{equation*}
  is a pushout square.
\end{defn}

We can use suspensions to present the spheres in type theory. The $2$-sphere is a space which, like the surface of the earth, has a north pole and a south pole. Moreover, for each point of the equator there is a meridian that connects the north pole to the south pool. Of course, the equator is a circle, so we see that the $2$-sphere is just the suspension of the circle.

Similarly we can see that the $(n+1)$-sphere must be the suspension of the $n$-sphere. The $(n+1)$-sphere is the unit sphere in the vector space $\mathbb{R}^{n+2}$. This vector space has an orthogonal basis $e_1,\ldots,e_{n+2}$. Then the north and the south pole are given by $e_{n+2}$ and $-e_{n+2}$, respectively, and for each unit vector in $\mathbb{R}^{n+1}\subseteq\mathbb{R}^{n+2}$ we have a meridian connecting the north pole with the south pole. The unit sphere in $\mathbb{R}^{n+1}$ is of course the $n$-sphere, so we see that the $(n+1)$-sphere must be a suspension of the $n$-sphere.

These observations suggest that we can define the spheres by recursion on $n$. Note that the spheres in type theory are defined entirely synthetically, i.e., without reference to the ambient topological space $\mathbb{R}^{n+1}$. Indeed, from a homotopical point of view each space $\mathbb{R}^{n}$ is contractible, so in type theory it is just presented as the unit type\footnote{It is an entirely different matter to define the \emph{set} $\mathbb{R}$ rather than the homotopy type of $\mathbb{R}$. See Chapter 11 of \cite{hottbook} for definitions of the Dedekind reals and the Cauchy reals.}.

\begin{defn}
We define the \define{$n$-sphere}\index{n-sphere@{$n$-sphere}} $\sphere{n}$\index{Sn@{$\sphere{n}$}} for any $n:\N$ by induction on $n$, by taking
\begin{align*}
\sphere{0} & \defeq \bool \\
\sphere{n+1} & \defeq \susp{\sphere{n}}.
\end{align*}
\end{defn}

\begin{rmk}
  Note that this recursive definition of the spheres only goes through in type theory if we have (or assume) a universe that is closed under suspensions.
\end{rmk}

In the following lemma we give a slight simplification of the universal property of suspensions, making it just a little easier to work with them.

\begin{lem}
Let $X$ and $Y$ be types, and let $\susp X$ be a suspension of $X$. Then the map\index{universal property!of suspensions}
\begin{equation*}
(\susp{X}\to Y)\to \sm{y,y':Y} X\to (y=y')
\end{equation*}
given by $f\mapsto (f(\north),f(\south),f\cdot\merid)$ is an equivalence.
\end{lem}

\begin{proof}
  Note that we have a commuting triangle
  \begin{equation*}
    \begin{tikzcd}[column sep=-1em]
      \phantom{\sm{y,y':Y}X\to (y=y')} & (\susp X \to Y) \arrow[dl,swap,"\mathsf{cocone\usc{}map}"] \arrow[dr,"{f\mapsto (f(\north),f(\south),f\cdot\merid)}"] &
      \phantom{\mathsf{cocone}_{\mathcal{S}}(Y)} \\
      \mathsf{cocone}_{\mathcal{S}}(Y) \arrow[rr] & & \sm{y,y':Y}X\to (y=y')
    \end{tikzcd}
  \end{equation*}
  where $\mathcal{S}$ is the span $\unit \leftarrow X \rightarrow \unit$. The bottom map is given by $(i,j,H)\mapsto (i(\ttt),j(\ttt),H)$. This map is an equivalence, and the map on the left is an equivalence by the assumption that $\susp X$ is a suspension of $X$. Therefore the claim follows by the 3-for-2 property of equivalences.
\end{proof}

\subsection{The duality of pullbacks and pushouts}
\begin{lem}\label{lem:cocone_pb}
For any span $\mathcal{S}\jdeq (S,f,g)$ from $A$ to $B$, and any type $X$ the square\index{cocone_S(X)@{$\mathsf{cocone}_{\mathcal{S}}(X)$}!as a pullback}
\begin{equation*}
\begin{tikzcd}
\mathsf{cocone}_{\mathcal{S}}(X) \arrow[r,"\pi_2"] \arrow[d,swap,"\pi_1"] & X^B \arrow[d,"\blank\circ g"] \\
X^A \arrow[r,swap,"\blank\circ f"] & X^S,
\end{tikzcd}
\end{equation*}
which commutes by the homotopy $\pi_3' \defeq\lam{(i,j,H)} \mathsf{eq\usc{}htpy}(H)$, is a pullback square.
\end{lem}

\begin{proof}
The gap map $\mathsf{cocone}_{\mathcal{S}}(X)\to X^A\times_{X^S} X^B$ is the function 
\begin{equation*}
\lam{(i,j,H)}(i,j,\mathsf{eq\usc{}htpy}(H)).
\end{equation*}
This is an equivalence by \cref{thm:fib_equiv}, since it is the induced map on total spaces of the family of equivalences $\mathsf{eq\usc{}htpy}$. Therefore, the square is a pullback square by \cref{thm:is_pullback}.
\end{proof}

In the following theorem we establish the duality between pullbacks and pushouts.

\begin{thm}\label{thm:pushout_up}
Consider a commuting square\index{universal property!of pushouts}
\begin{equation*}
\begin{tikzcd}
S \arrow[r,"g"] \arrow[d,swap,"f"] & B \arrow[d,"j"] \\
A \arrow[r,swap,"i"] & X,
\end{tikzcd}
\end{equation*}
with $H:i\circ f\htpy j\circ g$. The following are equivalent:
\begin{enumerate}
\item The square is a pushout square.
\item The square
\begin{equation*}
\begin{tikzcd}
T^X \arrow[r,"\blank\circ j"] \arrow[d,swap,"\blank\circ i"] & T^B \arrow[d,"\blank\circ g"] \\
T^A \arrow[r,swap,"\blank\circ f"] & T^S
\end{tikzcd}
\end{equation*}
which commutes by the homotopy
\begin{equation*}
\lam{h} \mathsf{eq\usc{}htpy}(h\cdot H)
\end{equation*}
is a pullback square, for every type $T$.
%\item The type $X$ satisfies \define{span induction} for the span $A\leftarrow S \rightarrow B$, in the sense that for any type family $P$ over $X$, the map
%\begin{equation*}
%\Big(\prd{x:X}P(x)\Big)\to \Big(\sm{i':\prd{a:A}P(i(a))}{j':\prd{b:B}P(j(b))} i'\htpy_H j'\Big)
%\end{equation*}
%given by $s\mapsto (s\circ i,s\circ j,s\cdot H)$ has a section.
\end{enumerate}
\end{thm}

\begin{proof}
It is straightforward to verify that the triangle
\begin{equation*}
\begin{tikzcd}[column sep=3em]
& T^X \arrow[dl,swap,"{\mathsf{cocone\usc{}map}(i,j,H)}"] \arrow[dr,"{\mathsf{gap}(\blank\circ i,\blank\circ j, \mathsf{eq\usc{}htpy}(\blank\cdot H))}"] \\
\mathsf{cocone}(T) \arrow[rr,swap,"{\mathsf{gap}(i,j,\mathsf{eq\usc{}htpy}(H))}"] & & T^A \times_{T^S} T^B
\end{tikzcd}
\end{equation*}
commutes. Since the bottom map is an equivalence by \cref{lem:cocone_pb}, it follows that if either one of the remaining maps is an equivalence, so is the other. The claim now follows by \cref{thm:is_pullback}.
\end{proof}

\begin{eg}\label{eg:circle_pushout}
  The square
  \begin{equation*}
    \begin{tikzcd}[column sep=huge]
      X^{\sphere{1}} \arrow[r,"\blank\circ\mathsf{const}_{\base}"] \arrow[d,swap,"\blank\circ\mathsf{const}_{\base}"] & X^\unit \arrow[d,"\blank\circ\mathsf{const}_{\ttt}"] \\
      X^\unit \arrow[r,swap,"\blank\circ\mathsf{const}_{\ttt}"] & X^\bool
    \end{tikzcd}
  \end{equation*}
  is a pullback square for each type $X$. Therefore it follows by the second characterization of pushouts in \cref{thm:pushout_up} that the circle is a pushout\index{circle!S1 equiv susp 2@{$\eqv{\sphere{1}}{\susp\bool}$}}
  \begin{equation*}
    \begin{tikzcd}
      \bool \arrow[r] \arrow[d] & \unit \arrow[d] \\
      \unit \arrow[r] & \sphere{1}.
    \end{tikzcd}
  \end{equation*}
  In other words, $\eqv{\sphere{1}}{\susp{\bool}}$. 
\end{eg}

\begin{thm}\label{thm:pushout_pasting}
Consider the following configuration of commuting squares:\index{pushout!pasting property}\index{pasting property!for pushouts}
\begin{equation*}
\begin{tikzcd}
A \arrow[r,"i"] \arrow[d,swap,"f"] & B \arrow[r,"k"] \arrow[d,swap,"g"] & C \arrow[d,"h"] \\
X \arrow[r,swap,"j"] & Y \arrow[r,swap,"l"] & Z
\end{tikzcd}
\end{equation*}
with homotopies $H:j\circ f\htpy g\circ i$ and $K:l\circ g\htpy h\circ k$, and suppose that the square on the left is a pushout square. 
Then the square on the right is a pushout square if and only if the outer rectangle is a pushout square.
\end{thm}

\begin{proof}
Let $T$ be a type. Taking the exponent $T^{(\blank)}$ of the entire diagram of the statement of the theorem, we obtain the following commuting diagram
\begin{equation*}
\begin{tikzcd}
T^Z \arrow[r,"\blank\circ l"] \arrow[d,swap,"\blank\circ h"] & T^Y \arrow[d,swap,"\blank\circ g"] \arrow[r,"\blank\circ j"] & T^X \arrow[d,"\blank\circ f"] \\
T^C \arrow[r,swap,"\blank\circ k"] & T^B \arrow[r,swap,"\blank\circ i"] & T^A.
\end{tikzcd}
\end{equation*}
By the assumption that $Y$ is the pushout of $B\leftarrow A \rightarrow X$, it follows that the square on the right is a pullback square. It follows by \cref{thm:pb_pasting} that the rectangle on the left is a pullback if and only if the outer rectangle is a pullback. Thus the statement follows by the second characterization in \cref{thm:pushout_up}.
\end{proof}

\begin{lem}
Consider a map $f:A\to B$. Then the cofiber of the map $\inr:B\to \mathsf{cofib}_f$ is equivalent to the suspension $\susp{A}$ of $A$. 
\end{lem}

\subsection{Fiber sequences and cofiber sequences}

\begin{defn}
Given a map $f:A\to B$, we define the \define{cofiber}\index{cofiber} $\mathsf{cofib}_f$\index{cofib_f@{$\mathsf{cofib}_f$}} of $f$ as the pushout
\begin{equation*}
\begin{tikzcd}
A \arrow[r,"f"] \arrow[d] & B \arrow[d,"\inr"] \\
\unit \arrow[r,swap,"\inl"] & \mathsf{cofib}_f. 
\end{tikzcd}
\end{equation*}
The cofiber of a map is sometimes also called the \define{mapping cone}\index{mapping cone}.
\end{defn}

\begin{eg}
The suspension $\susp X$ of $X$ is the cofiber of the map $X\to \unit$.\index{suspension!as cofiber} 
\end{eg}

\subsection{Further examples of pushouts}

\begin{defn}
We define the \define{join}\index{join} $\join{X}{Y}$\index{join X Y@{$\join{X}{Y}$}} of $X$ and $Y$ to be the pushout 
\begin{equation*}
\begin{tikzcd}
X\times Y \arrow[r,"\proj 2"] \arrow[d,swap,"\proj 1"] & Y \arrow[d,"\inr"] \\
X \arrow[r,swap,"\inl"] & X \ast Y. 
\end{tikzcd}
\end{equation*}
\end{defn}

\begin{defn}
Suppose $A$ and $B$ are pointed types, with base points $a_0$ and $b_0$, respectively. The \define{(binary) wedge}\index{wedge@(binary) wedge} $A\vee B$ of $A$ and $B$ is defined as the pushout
\begin{equation*}
\begin{tikzcd}
\bool \arrow[r] \arrow[d] & A+B \arrow[d] \\
\unit \arrow[r] & A\vee B.
\end{tikzcd}
\end{equation*}
\end{defn}

\begin{defn}
Given a type $I$, and a family of pointed types $A$ over $i$, with base points $a_0(i)$. We define the \define{(indexed) wedge}\index{wedge@{(indexed) wedge}} $\bigvee_{(i:I)}A_i$ as the pushout
\begin{equation*}
\begin{tikzcd}[column sep=huge]
I \arrow[d] \arrow[r,"{\lam{i}(i,a_0(i))}"] & \sm{i:I}A_i \arrow[d] \\
\unit \arrow[r] & \bigvee_{(i:I)} A_i.
\end{tikzcd}
\end{equation*}
\end{defn}

\begin{defn}
Let $X$ and $Y$ be types with base points $x_0$ and $y_0$, respectively.
We define the \define{wedge} $X\lor Y$ of $X$ and $Y$ to be the pushout
\begin{equation*}
\begin{tikzcd}[column sep=8em]
\bool \arrow[r,"{\ind{\bool}(\inl(x_0),\inr(y_0))}"] \arrow[d,swap,"\mathsf{const}_\ttt"] & X+Y \arrow[d,"\inr"] \\
\unit \arrow[r,swap,"\inl"] & X\lor Y
\end{tikzcd}
\end{equation*}
\end{defn}

\begin{defn}
Let $X$ and $Y$ be types with base points $x_0$ and $y_0$, respectively.
We define a map
\begin{equation*}
\mathsf{wedge\usc{}incl} : X \lor Y \to X\times Y.
\end{equation*}
as the unique map obtained from the commutative square
\begin{equation*}
\begin{tikzcd}[column sep=8em]
\bool \arrow[r,"{\ind{\bool}(\inl(x_0),\inr(y_0))}"] \arrow[d,swap,"\mathsf{const}_\ttt"] & X+Y \arrow[d,"{\ind{X+Y}(\lam{x}\pairr{x,y_0},\lam{y}\pairr{x_0,y})}"] \\
\unit \arrow[r,swap,"\lam{t}\pairr{x_0,y_0}"] & X\times Y.
\end{tikzcd}
\end{equation*}
\end{defn}

\begin{defn}
We define the \define{smash product} $X\wedge Y$ of $X$ and $Y$ to be the pushout
\begin{equation*}
\begin{tikzcd}[column sep=huge]
X\lor Y \arrow[r,"\mathsf{wedge\usc{}incl}"] \arrow[d,swap,"\mathsf{const}_\ttt"] & X\times Y \arrow[d,"\inr"] \\
\unit \arrow[r,swap,"\inl"] & X\wedge Y.
\end{tikzcd}
\end{equation*}
\end{defn}

\begin{exercises}
\exercise \label{ex:pushout_equiv}Use \cref{thm:pushout_up,cor:pb_equiv,ex:equiv_precomp} to show that for any commuting square
\begin{equation*}
\begin{tikzcd}
S \arrow[r,"g"] \arrow[d,swap,"f","{\eqvsym}"'] & B \arrow[d,"j"] \\
A \arrow[r,swap,"i"] & C
\end{tikzcd}
\end{equation*} 
where $f$ is an equivalence, the square is a pushout square if and only if $j:B\to C$ is an equivalence.
Use this observation to conclude the following:
\begin{enumerate}
\item If $X$ is contractible, then $\susp X$ is contractible.
\item The cofiber of any equivalence is contractible.
\item The cofiber of a point in $B$ (i.e., of a map of the type $\unit\to B$) is equivalent to $B$.
\item There is an equivalence $\eqv{X}{\join{\emptyt}{X}}$.
\item If $X$ is contractible, then $\join{X}{Y}$ is contractible. 
\item If $A$ is contractible, then there is an equivalence $\eqv{A\vee B}{B}$ for any pointed type $B$.
\end{enumerate}
\exercise \label{ex:join_propositions}Let $P$ and $Q$ be propositions.
\begin{subexenum}
\item Show that $\join{P}{Q}$ satisfies the \emph{universal property of disjunction}, i.e., that for any proposition $R$, the map
\begin{equation*}
(\join{P}{Q}\to R)\to (P\to R)\times (Q\to R)
\end{equation*}
given by $f\mapsto (f\circ \inl,f\circ \inr)$, is an equivalence.
\item Use the proposition $R\defeq\iscontr(\join{P}{Q})$ to show that $\join{P}{Q}$ is again a proposition.
\end{subexenum}
\exercise Let $Q$ be a proposition, and let $A$ be a type. Show that the following are equivalent:
\begin{enumerate}
\item The map $(Q\to A)\to(\emptyt\to A)$ is an equivalence.
\item The type $A^Q$ is contractible.
\item There is a term of type $Q\to\iscontr(A)$.
\item The map $\inr:A\to \join{Q}{A}$ is an equivalence.
\end{enumerate}
\exercise Let $P$ be a proposition. Show that $\susp P$ is a set, with an equivalence
\begin{equation*}
\eqv{\Big(\inl(\ttt)=\inr(\ttt)\Big)}{P}.
\end{equation*}
\exercise Show that \({A\sqcup^{\mathcal{S}} B} \simeq {B\sqcup^{\mathcal{S}^{\mathsf{op}}} A}\), where $\mathcal{S^{\mathsf{op}}}\defeq (S,g,f)$ is the \define{opposite span} of $\mathcal{S}$. 
\exercise Use \cref{ex:pb_pi} to show that if
\begin{equation*}
\begin{tikzcd}
S \arrow[r] \arrow[d] & Y \arrow[d] \\
X \arrow[r] & Z
\end{tikzcd}
\end{equation*}
is a pushout square, then so is
\begin{equation*}
\begin{tikzcd}
A\times S \arrow[r] \arrow[d] & A\times Y \arrow[d] \\
A\times X \arrow[r] & A\times Z
\end{tikzcd}
\end{equation*}
for any type $A$.
\exercise Use \cref{ex:pb_prod} to show that if
\begin{equation*}
\begin{tikzcd}
S_1 \arrow[r] \arrow[d] & Y_1 \arrow[d] & S_2 \arrow[r] \arrow[d] & Y_2 \arrow[d] \\
X_1 \arrow[r] & Z_1 & X_2 \arrow[r] & Z_2
\end{tikzcd}
\end{equation*}
are pushout squares, then so is
\begin{equation*}
\begin{tikzcd}
S_1+S_2 \arrow[r] \arrow[d] & Y_1+ Y_2 \arrow[d] \\
X_1 +X_2 \arrow[r] & Z_1+Z_2. 
\end{tikzcd}
\end{equation*}
\exercise 
\begin{subexenum}
\item Consider a span $(S,f,g)$ from $A$ to $B$. Use \cref{ex:pb_diagonal} to show that the square
\begin{equation*}
\begin{tikzcd}[column sep=large]
S+S \arrow[d,swap,"{f+g}"] \arrow[r,"{[\idfunc,\idfunc]}"] & S \arrow[d,"{\inr\circ g}"] \\
A+B \arrow[r,swap,"{[\inl,\inr]}"] & A\sqcup^\mathcal{S} B
\end{tikzcd}
\end{equation*}
is again a pushout square.
\item Show that $\eqv{\susp X}{\join{\bool}{X}}$.
\end{subexenum}
\exercise Consider a commuting triangle
\begin{equation*}
\begin{tikzcd}[column sep=tiny]
A \arrow[rr,"h"] \arrow[dr,swap,"f"] & & B \arrow[dl,"g"] \\
& X
\end{tikzcd}
\end{equation*}
with $H:f\htpy g\circ h$. 
\begin{subexenum}
\item Construct a map $\mathsf{cofib}_{(h,H)}: \mathsf{cofib}_{g}\to \mathsf{cofib}_f$.
\item Use \cref{ex:pb_fib} to show that $\eqv{\mathsf{cofib}_{\mathsf{cofib}(h,H)}}{\mathsf{cofib}_h}$.
\end{subexenum}
\exercise \label{ex:sphere_null}Use \cref{ex:circle_connected} to show that for $n\geq 0$, $X$ is an $n$-type if and only if the map
\begin{equation*}
\lam{x}\mathsf{const}_x : X \to (\sphere{n+1}\to X)
\end{equation*}
is an equivalence.
\exercise 
\begin{subexenum}
\item Construct for every $f:X\to Y$ a function
\begin{equation*}
\susp f : \susp X\to \susp Y.
\end{equation*}
\item Show that if $f\htpy g$, then $\susp f \htpy \susp g$. 
\item Show that $\susp \idfunc[X]\htpy\idfunc[\susp X]$
\item Show that
\begin{equation*}
\susp(g\circ f)\htpy (\susp g)\circ (\susp f).
\end{equation*}
for any $f:X\to Y$ and $g:Y\to Z$.
\end{subexenum}
\exercise 
\begin{subexenum}
\item Let $I$ be a type, and let $A$ be a family over $I$. Construct an equivalence
\begin{equation*}
\eqv{\Big(\bigvee\nolimits_{(i:I)}\susp A_i\Big)}{\susp\Big(\bigvee\nolimits_{(i:I)}A_i\Big)}.
\end{equation*}
\item Show that for any type $X$ there is an equivalence
\begin{equation*}
\eqv{\Big(\bigvee\nolimits_{(x:X)}\bool\Big)}{X+\unit}.
\end{equation*}
\item Construct an equivalence
\begin{equation*}
\eqv{\susp(\mathsf{Fin}(n+1))}{\bigvee\nolimits_{(i:\mathsf{Fin}(n))}\sphere{1}}.
\end{equation*}
\end{subexenum}
\exercise Show that $\eqv{\join{\mathsf{Fin}(n+1)}{\mathsf{Fin}(m+1)}}{\bigvee\nolimits_{(i:\mathsf{Fin}(n\cdot m))}\sphere{1}}$, for any $n,m:\N$.
\exercise For any pointed set $X$, show that the squares
  \begin{equation*}
    \begin{tikzcd}
      \sphere{1} \arrow[r] \arrow[d] & \unit \arrow[d] \\
      \bigvee_{(x:X)}\sphere{1} \arrow[r] & \susp{X}
    \end{tikzcd}
    \qquad\text{and}\qquad
    \begin{tikzcd}
      X\times \sphere{1} \arrow[d] \arrow[r] & \unit \arrow[d] \\
      \bigvee_{(x:X)}\sphere{1} \arrow[r] & \susp{X}
    \end{tikzcd}
  \end{equation*}
  are pushout squares.
\exercise Show that the square
  \begin{equation*}
    \begin{tikzcd}
      \sphere{1} \arrow[r] \arrow[d] & \unit \arrow[d] \\
      \sphere{1}\times\sphere{1} \arrow[r] & \sphere{2}\vee\sphere{1}
    \end{tikzcd}
  \end{equation*}
  is a pushout square.
\exercise For any type $X$, show that the mapping cone of the fold map $X+X\to X$ is the suspension of $X+\unit$, i.e.~show that the following square
  \begin{equation*}
    \begin{tikzcd}
      X+X \arrow[d] \arrow[r] & \unit \arrow[d] \\
      X \arrow[r] & \susp{X+\unit}
    \end{tikzcd}
  \end{equation*}
  is a pushout square.
  \exercise Consider a map $f:A\to B$. Show that $f$ is a $k$-truncated map if and only if the square
  \begin{equation*}
    \begin{tikzcd}
      A \arrow[r,"\delta"] \arrow[d,swap,"f"] & A^{\sphere{k+1}} \arrow[d,"f^{\sphere{k+1}}"] \\
      B \arrow[r,swap,"\delta"] & B^{\sphere{k+1}}
    \end{tikzcd}
  \end{equation*}
  is a pullback square.
\end{exercises}
