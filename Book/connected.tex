% !TEX root = hott_intro.tex

\section{Connected types and maps}

In this section we introduce the concept of $k$-connected types and maps. We define $k$-connected types to be types with contractible $k$-truncation, and a $k$-connected map is just a map of which the fibers are $k$-connected. The idea is that a type is $k$-connected if and only if its homotopy groups $\pi_i(X)$ are trivial for all $i\leq k$.

One of the main theorems in this section is a characterization of $k$-connected maps in terms of their action on homotopy groups: A map $f:X\to Y$ is $k$-connected if and only if it induces isomorphisms
\begin{equation*}
  \pi_i(f,x):\pi_i(X,x)\to\pi_i(Y,f(x))
\end{equation*}
of homotopy groups, for each $i\leq k$ and each $x:X$, and a \emph{surjective} group homomorphism
\begin{equation*}
  \pi_{k+1}(f,x):\pi_{k+1}(X,x)\to\pi_{k+1}(Y,f(x))
\end{equation*}
on the $(k+1)$-st homotopy group, for each $x:X$. If one drops the condition that $f$ induces a surjective group homomorphism on the $(k+1)$-st homotopy group, then the map is only a $k$-equivalence, i.e., a map of which $\trunc{k}{f}$ is an equivalence. We see from the above characterization that any $k$-connected map is a $k$-equivalence, and also that any $(k+1)$-equivalence is a $k$-connected map. Nevertheless, the difference between the classes of $k$-equivalences and $k$-connected maps is somewhat subtle.

We will study $k$-equivalences and $k$-connected maps synchronously, because understanding the subtle differences between the results about either of them will increase the understanding of both classes of maps. For instance, we will show that the $k$-connected maps enjoy a dependent elimination property, while the $k$-equivalences only satisfy a non-dependent elimination property. We will see that the $k$-equivalences satisfy the 3-for-2 property, while one of the cases of the 3-for-2 property fails for $k$-connected maps.

The $k$-connected maps can be characterized as the class of maps that is left orthogonal to the class of $k$-truncated maps, where a map $f:A\to B$ is said to be left orthogonal to a map $g:X\to Y$ if the type of diagonal fillers of any commuting square of the form
\begin{equation*}
  \begin{tikzcd}
    A \arrow[d,swap,"f"] \arrow[r] & X \arrow[d,"g"] \\
    B \arrow[r] \arrow[ur,densely dotted] & Y
  \end{tikzcd}
\end{equation*}
is contractible. Similarly, the class of $k$-equivalences is the class of maps that is left orthogonal to any map between $k$-truncated types. However, this result is not entirely sharp, because there are more maps that the $k$-equivalences are left orthogonal to. It turns out that a map is a $k$-equivalence if and only if it is left orthogonal to any map $g:X\to Y$ for which the naturality square
\begin{equation*}
  \begin{tikzcd}[column sep=large]
    X \arrow[r,"g"] \arrow[d,swap,"\eta"] & Y \arrow[d] \\
    \trunc{k}{X} \arrow[r,swap,"\trunc{k}{g}"] & \trunc{k}{Y}
  \end{tikzcd}
\end{equation*}
is a pullback square. Such maps are called $k$-\'etale, and they induce isomorphisms
\begin{equation*}
  \pi_i(g,x):\pi_i(X,x)\to\pi_i(Y,g(x))
\end{equation*}
on homotopy groups for $i>k$.

In the final part of this section we will use the results about $k$-equivalences to show that the $n$-sphere is $(n-1)$-connected, for each $n:\N$, and that the join $\join{A}{B}$ is $(k+l+2)$-connected if $A$ is $k$-connected and $B$ is $l$-connected.

\subsection{Connected types}

\begin{defn}
  A type $X$ is said to be \define{$k$-connected} if its $k$-truncation $\trunc{k}{X}$ is contractible. We define
  \begin{equation*}
    \isconn_k(X)\defeq\iscontr\trunc{k}{X}.
  \end{equation*}
\end{defn}

\begin{rmk}
  Since the $(-2)$-truncation of any type is just $\unit$, it follows that every type is $(-2)$-connected. Furthermore, since any proposition is contractible as soon as it comes equipped with a term, it follows that any type is $(-1)$-connected as soon as it is inhabited.

    In \cref{thm:conn-succ} below, we will see that a type $X$ is $0$-connected if and only if it is inhabited and every two points are connected by an unspecified path. In this sense $0$-connected types are also called \define{path connected}, or just \define{connected}. Thus, it is immediate that the circle is an example of a connected type.

  Similarly, in the case where $k\jdeq 0$ the theorem states that a type $X$ is $1$-conneced if and only if it is inhabited and for every $x,y:X$ the identity type $x=y$ is path connected. In other words, a type is \define{simply connected} if it is $1$-connected! The $2$-sphere is an example of a simply connected type. This fact is shown in \cref{cor:conn-sphere} below, where we will show more generally that the $n$-sphere is $(n-1)$-connected, for each $n:\N$.
\end{rmk}

\begin{lem}
  If a type is $(k+1)$-connected, then it is also $k$-connected.
\end{lem}

\begin{proof}
  This follows from the fact that $\trunc{k}{\trunc{k+1}{X}}\simeq\trunc{k}{X}$. Indeed, if $\trunc{k+1}{X}$ is contractible, then its $k$-truncation is also contractible, so it follows that $\trunc{k}{X}$ is contractible.
\end{proof}

For the following theorem, recall that a type $X$ is said to be inhabited if it comes equipped with a term $\trunc{-1}{X}$.

\begin{thm}\label{thm:conn-succ}
  Consider a type $X$. Then the following are equivalent:
  \begin{enumerate}
  \item The type $X$ is $(k+1)$-connected.
  \item The type $X$ is inhabited, and the type $x=y$ is $k$-connected for each $x,y:X$.
  \end{enumerate}
\end{thm}

\begin{proof}
  Suppose first that $X$ is $(k+1)$-connected. It is immediate that $X$ is inhabited in this case. Moreover, since we have equivalences
  \begin{equation*}
    (\eta(x)=\eta(y))\simeq \trunc{k}{x=y}
  \end{equation*}
  for each $x,y:X$, it follows from the assumption that $\trunc{k+1}{X}$ is contractible that the type $\trunc{k}{x=y}$ is equivalent to a contractible type. This proves that (i) implies (ii).

  To see that (ii) implies (i), suppose that $X$ is inhabited and that its identity types are $k$-connected. Our goal is to construct a term of type
  \begin{equation*}
    \iscontr\trunc{k+1}{X},
  \end{equation*}
  which is a proposition, so we may eliminate the assumption that $X$ is inhabited and assume to have $x:X$. Now we simply take $\eta(x)$ for the center of contraction of $\trunc{k+1}{X}$. To construct the contraction, note that by the dependent universal property of $(k+1)$-truncation we have an equivalence
  \begin{equation*}
    \Big(\prd{y:\trunc{k+1}{X}}\eta(x)=y\Big)\simeq\Big(\prd{y:X}\eta(x)=\eta(y)\Big).
  \end{equation*}
  Therefore it suffices to construct an identification $\eta(x)=\eta(y)$ for every $y:X$. However, this type is contractible, since it is equivalent to the contractible type $\trunc{k}{x=y}$. This completes the proof of (ii) implies (i).
\end{proof}

In the case where $k\geq -1$ we can improve \cref{thm:conn-succ} and characterize a high degree of connectedness entirely in terms of the triviality of homotopy groups. This is what connectedness is all about.

\begin{thm}\label{thm:conn-htpy-groups}
  Consider a type $X$, and suppose that $k\geq 0$. Then the following are equivalent:
  \begin{enumerate}
  \item The type $X$ is $k$-connected.
  \item The type $X$ is connected, and for every $x:X$ the loop space
    \begin{equation*}
      \loopspace{X,x}
    \end{equation*}
    is $(k-1)$-connected.
  \item For each $i\leq k$ and each $x:X$, the $i$-th homotopy group $\pi_i(X,x)$ is trivial.
  \end{enumerate}
\end{thm}

\begin{proof}
  If $X$ is $k$-connected for $k\geq 0$, then it is certainly connected, and $\loopspace{X,x}$ is $(k-1)$-connected by \cref{thm:conn-succ}. Thus, the fact that (i) implies (ii) is immediate.

  To see that (ii) implies (i), note that if $X$ is connected and its loop spaces are $(k-1)$-connected, then all its identity types are $(k-1)$-connected, since we have
  \begin{align*}
    \prd{x,y:X}\iscontr(\trunc{k-1}{x=y}) & \simeq \prd{x,y:X}\trunc{-1}{x=y}\to\iscontr(\trunc{k-1}{x=y}) \\
    & \simeq \prd{x,y:X}(x=y)\to\iscontr(\trunc{k-1}{x=y}) \\
    & \simeq \prd{x:X}\iscontr(\trunc{k-1}{x=x}).
  \end{align*}
  In the first step of this calculation we use that $X$ is connected, so $\trunc{-1}{x=y}$ is contractible; then we use that $\iscontr(\trunc{k-1}{x=y})$ is a proposition; and finally we use the universal property of identity types to arrive at our assumption that the loop spaces of $X$ are $(k-1)$-connected. Since we have shown that the identity types are $(k-1)$-connected, it follows by \cref{thm:conn-succ} that $X$ is $k$-connected, which concludes the proof that (ii) implies (i).

  It is easy to see by induction on $k\geq 0$ that (ii) holds if and only if (iii) holds, since we have
  \begin{equation*}
    \pi_{i+1}(X,x)=\pi_i(\loopspace{X,x}).\qedhere
  \end{equation*}
\end{proof}

\begin{rmk}
  If $X$ is assumed to be a pointed type in \cref{thm:conn-htpy-groups}, then conditions (ii) and (iii) only have to be checked at the base point.
\end{rmk}

\subsection{\texorpdfstring{$k$}{k}-Equivalences and \texorpdfstring{$k$}{k}-connected maps}

We now study two classes of maps that differ only slightly: the $k$-equivalences and the $k$-connected maps. 

\begin{defn}
  ~
  \begin{enumerate}
  \item A map $f:X\to Y$ is said to be \define{$k$-connected} if its fibers are $k$-connected. We will write
  \begin{equation*}
    \isconn_k(f)\defeq\prd{y:Y}\isconn_k(\fib{f}{y}).
  \end{equation*}
  \item A map $f:X\to Y$ is said to be a \define{$k$-equivalence} if
    \begin{equation*}
      \trunc{k}{f}:\trunc{k}{X}\to\trunc{k}{Y}
    \end{equation*}
    is an equivalence. We will write
    \begin{equation*}
      \isequiv_k(f)\defeq\isequiv(\trunc{k}{f}).
    \end{equation*}
  \end{enumerate}
\end{defn}

\begin{eg}
  Any equivalence is a $k$-connected map, as well as a $k$-equivalence. Moreover, for any $k$-connected type $X$ the map $\const_\ttt:X\to\unit$ is $k$-connected. It is also immediate that \emph{any} map between $k$-connected types is a $k$-equivalence.
\end{eg}

\begin{eg}
  A $(-1)$-connected map is a map $f:X\to Y$ for which the propositionally truncated fibers
  \begin{equation*}
    \trunc{-1}{\fib{f}{y}}
  \end{equation*}
  are contractible. Since propositions are contractible as soon as they are inhabited, we see that a map is $(-1)$-connected if and only if it is surjective.

  A $(-1)$-equivalence, on the other hand, is just a map $f:X\to Y$ that induces an equivalence $\trunc{-1}{X}\simeq\trunc{-1}{Y}$. The map $\const_\btrue : \unit\to\bool$ is an example of such a map, showing that $(-1)$-equivalences don't need to be surjective.

  However, it is the case that every surjective map $f:X\to Y$ is in fact $(-1)$-equivalence. To see this, we need to show that
  \begin{equation*}
    \trunc{-1}{Y}\to\trunc{-1}{X}.
  \end{equation*}
  Such a map is constructed by the universal property of $(-1)$-truncation. Thus, it suffices to construct a function $Y\to\trunc{-1}{X}$. Since we have assumed that $f$ is surjective, we have for every $y:Y$ a term
  \begin{equation*}
    s(y):\trunc{-1}{\fib{f}{y}}.
  \end{equation*}
  Thus, we define a function $Y\to\trunc{-1}{X}$ by
  \begin{equation*}
    y\mapsto\trunc{-1}{\proj 1}(s(y)).
  \end{equation*}
  This concludes the proof that $f$ is a $(-1)$-equivalence, since we have shown that $\trunc{-1}{X}\leftrightarrow\trunc{-1}{Y}$. 
\end{eg}

\begin{rmk}
  An immediate difference between the classes of $k$-equivalences and $k$-connected maps is that the $k$-connected maps are stable under base change, while the $k$-equivalences are not. By this, we mean that for any pullback square
  \begin{equation*}
    \begin{tikzcd}
      E' \arrow[d,swap,"{p'}"] \arrow[r,"g"] & E \arrow[d,"p"] \\
      B' \arrow[r,swap,"f"] & B,
    \end{tikzcd}
  \end{equation*}
  if the map $p$ is $k$-connected, then the map $p'$ is also $k$-connected. In such a pullback diagram, the map $p'$ is sometimes called the \define{base change} of $p$ along $f$. By \cref{cor:pb_fibequiv} we have an equivalence
  \begin{equation*}
    \fib{p'}{b'}\simeq\fib{p}{f(b')}
  \end{equation*}
  for any $b':B'$, so it is indeed the case that if the fibers of $p$ are $k$-connected, then so are the fibers of $p'$.

  An example showing that the $k$-equivalences are not stable under base change is given by the pullback square
  \begin{equation*}
    \begin{tikzcd}
      \loopspace{\sphere{k+1}} \arrow[r] \arrow[d] & \unit \arrow[d] \\
      \unit \arrow[r] & \sphere{k+1}
    \end{tikzcd}
  \end{equation*}
  We will show in \cref{cor:conn-sphere} that the $(k+1)$-sphere is $k$-connected, so the map $\unit\to\sphere{k+1}$ is a $k$-equivalence. However, its loop space is only $(k-1)$-connected, and indeed we will show in \cref{far-future} that
  $\pi_{k+1}(\sphere{k+1})=\Z$ for $k\geq 0$, showing that $\loopspace{\sphere{k+1}}$ is \emph{not} $k$-connected. Thus, the map $\loopspace{\sphere{k+1}}\to\unit$ is not a $k$-equivalence.
\end{rmk}

\subsubsection{Elimination properties}
We will show that a map $f:X\to Y$ is a $k$-equivalence if and only if the precomposition function
\begin{equation*}
  \blank\circ f : (Y\to Z)\to (X\to Z)
\end{equation*}
is an equivalence for every $k$-type $Z$. On the other hand, we will show that $f$ is $k$-connected if and only if the precomposition function
\begin{equation*}
  \blank\circ f : \Big(\prd{y:Y}P(y)\Big)\to\Big(\prd{x:X}P(f(x))\Big)
\end{equation*}
is an equivalence for every family $P$ of $k$-types over $Y$. In other words, the $k$-connected maps satisfy a \emph{dependent} unique elimination property, while the $k$-equivalences only satisfy a \emph{non-dependent} unique elimination property.

\begin{thm}\label{thm:k-equiv-precomp}
  Consider a function $f:X\to Y$. Then the following are equivalent
  \begin{enumerate}
  \item The map $f$ is a $k$-equivalence.
  \item For every $k$-type $Z$, the precomposition function
    \begin{equation*}
      \blank\circ f:(Y\to Z)\to(X\to Z)
    \end{equation*}
    is an equivalence.
  \end{enumerate}
\end{thm}

\begin{thm}\label{thm:conn-dup}
  Let $f:X\to Y$ be a map. The following are equivalent:
  \begin{enumerate}
  \item The map $f$ is $k$-connected.
  \item For every family $P$ of $k$-truncated types over $Y$, the precomposition map
    \begin{equation*}
      \blank\circ f : \Big(\prd{y:Y}P(y)\Big)\to\Big(\prd{x:X}P(f(x))\Big)
    \end{equation*}
    is an equivalence.
  \end{enumerate}
\end{thm}

\begin{proof}
  Suppose $f$ is $k$-connected and let $P$ be a family of $k$-types over $Y$. Now we may consider the following commuting diagram
  \begin{equation*}
    \begin{tikzcd}[column sep=-10em]
      \phantom{\prd{x:X}{y:Y}{p:f(x)=y}P(y)} & \prd{y:Y}P(y) \arrow[r,"\blank\circ f"] \arrow[dl] &[10em] \prd{x:X}P(f(x)) \\
      \prd{y:Y}\trunc{k}{\fib{f}{y}}\to P(y) \arrow[dr] & \phantom{\prd{y:Y}{x:X}{p:f(x)=y}P(y)} & & \prd{x:X}{y:Y}{p:f(x)=y}P(y) \arrow[ul] \\
      & \prd{y:Y}\fib{f}{y}\to P(y) \arrow[r] & \prd{y:Y}{x:X}{p:f(x)=y}P(y) \arrow[ur]
    \end{tikzcd}
  \end{equation*}
  which commutes by $\reflhtpy$. In this diagram, the five maps going around counter clockwise are all equivalences for obvious reasons, so it follows that the top map is an equivalence.

  Now suppose that $f$ satisfies the dependent elimination property stated in (ii). In order to construct a center of contraction of $\trunc{k}{\fib{f}{y}}$ for every $y:Y$, we use the dependent elimination property with respect to the family $P$ given by $P(y)\defeq\trunc{k}{\fib{f}{y}}$. 
\end{proof}

\begin{cor}
  For any type $X$, the unit $\eta:X\to\trunc{k}{X}$ of the $k$-truncation is a $k$-connected map. 
\end{cor}

\subsubsection{The inclusions}

We will prove the following implications
\begin{equation*}
  \begin{tikzcd}[column sep=8em]
    \isequiv_{k+1}(f) \arrow[r,"\text{\cref{prp:is-k-equiv-is-k-conn}}"] & \isconn_k(f) \arrow[r,"\text{\cref{prp:is-k-conn-is-succk-equiv}}"] & \isequiv_k(f)
  \end{tikzcd}
\end{equation*}
showing that the class of $k$-connected maps is contained in the class of $k$-equivalences, and that the class of $(k+1)$-equivalences is contained in the class of $k$-connected maps. Neither of these implications reverses.

\begin{prp}\label{prp:is-k-equiv-is-k-conn}
  Any $k$-connected map is a $k$-equivalence.
\end{prp}

\begin{prp}\label{prp:is-k-conn-is-succk-equiv}
  Any $(k+1)$-equivalence is $k$-connected.
\end{prp}

\begin{proof}
  Consider a $(k+1)$-equivalence $f:X\to Y$. Recall that the map $\trunc{k+1}{f}$ comes equipped with a homotopy $H:\trunc{k+1}{f}\circ\eta\htpy\eta\circ f$ witnessing that the square
  \begin{equation*}
    \begin{tikzcd}[column sep=large]
      X \arrow[r,"f"] \arrow[d,swap,"\eta"] & Y \arrow[d,"\eta"] \\
      \trunc{k+1}{X} \arrow[r,swap,"\trunc{k+1}{f}"] & \trunc{k+1}{Y}
    \end{tikzcd}
  \end{equation*}
  commutes. We be using this homotopy, and we will use \cref{thm:conn-dup} to show that $f$ is $k$-connected. Thus, our goal is to show that
  \begin{equation*}
    \blank\circ f:\Big(\prd{y:Y}P(y)\Big)\to\Big(\prd{x:X}P(f(x))\Big)
  \end{equation*}
  is an equivalence for any family $P$ of $k$-types over $Y$.

  Note that any family $P$ of $k$-types over $Y$ extends to a family $\tilde{P}$ of $k$-types over $\trunc{k+1}{Y}$, since any univalent universe of $k$-types that contains $P$ is itself a $(k+1)$-type by \cref{ex:istrunc_UUtrunc}. The extended family $\tilde{P}$ of $k$-types over $\trunc{k+1}{Y}$ comes equipped with a family of equivalences
  \begin{equation*}
    e:\prd{y:Y}\tilde{P}(\eta(y))\simeq P(y).
  \end{equation*}
  Now consider the commuting diagram
  \begin{equation*}
    \begin{tikzcd}[column sep=-9em]
      \phantom{\prd{x:X}\tilde{P}(\trunc{k+1}{f}(\eta(x)))} & \phantom{\prd{x:X}\tilde{P}(\trunc{k+1}{f}(\eta(x)))} & \prd{y:\trunc{k+1}{Y}}\tilde{P}(y) \arrow[r,"\blank\circ\trunc{k+1}{f}"] \arrow[ddll,swap,"\blank\circ\eta"] &[11em] \prd{x:\trunc{k+1}{X}}\tilde{P}(\trunc{k+1}{f}(x)) \arrow[dr,"\blank\circ\eta" near end] & & \phantom{\prd{x:X}\tilde{P}(\trunc{k+1}{f}(\eta(x)))} \\
      & & \phantom{\prd{x:\trunc{k+1}{X}}\tilde{P}(\trunc{k+1}{f}(x))} & & \prd{x:X}\tilde{P}(\trunc{k+1}{f}(\eta(x))) \arrow[dr,"{h\mapsto\lam{x}\tr_{\tilde{P}}(H(x),h(x))}" near end] \\
      \prd{y:Y}\tilde{P}(\eta(y)) \arrow[drr,swap,"{h\mapsto\lam{y}e_y(h(y))}" near start] & & & & & \prd{x:X}\tilde{P}(\eta(f(x))) \arrow[dll,"{h\mapsto\lam{x}e_{f(x)}(h(x))}" near start] \\
      & & \prd{y:Y}P(y) \arrow[r,swap,"\blank\circ f"] & \prd{x:X}P(f(x)).
    \end{tikzcd}
  \end{equation*}
  This diagram commutes by the homotopy
  \begin{equation*}
    \lam{h}\eqhtpy(\lam{x}\ap{e(f(x))}{\apd{h}{H(x)}}^{-1}).
  \end{equation*}
  In this diagrams all the maps pointing downwards are equivalences for obvious reasons: the two maps $\blank\circ\eta$ are equivalences since $\tilde{P}$ is a family of $k$-types, and the remaining three maps pointing downwards are all postcomposing with an equivalence. The top map is an equivalence since $\trunc{k+1}{f}$ is assumed to be an equivalence. Thus we conclude that the bottom map $\blank\circ f$ is an equivalence.
\end{proof}

\subsubsection{The 3-for-2 property}
An important distinction between the class of $k$-equivalences and the class of $k$-connected maps is that the $k$-equivalences satisfy the 3-for-2 property, while the $k$-connected maps do not.

\begin{rmk}\label{rmk:conn-3-for-2}
  It is not hard to see that the $k$-connected maps don't satisfy the 3-for-2 property. For example, consider the following commuting triangle
  \begin{equation*}
    \begin{tikzcd}[column sep=tiny]
      \sphere{1} \arrow[rr,"d_2"] \arrow[dr] & & \sphere{1} \arrow[dl] \\
      & \unit,
    \end{tikzcd}
  \end{equation*}
  where $d_2:\sphere{1}\to\sphere{1}$ is the degree $2$ map. Since the circle is a $0$-connected type, it follows that the maps $\sphere{1}\to\unit$ are $0$-connected. However, the fiber of $d_2$ at the base point is equivalent to the booleans, which is a non-contractible set so it is certainly not $0$-connected.
  \end{rmk}

\begin{lem}
  The $k$-equivalences satisfy the 3-for-2 property, i.e., for any commuting triangle
  \begin{equation*}
    \begin{tikzcd}[column sep=tiny]
      A \arrow[rr,"h"] \arrow[dr,swap,"f"] & & B \arrow[dl,"g"] \\
      & X,
    \end{tikzcd}   
  \end{equation*}
  if any two of the three maps are $k$-equivalences, then so is the third.
\end{lem}

\begin{proof}
  This follows immediately from the fact that equivalences satisfy the 3-for-2 property.
\end{proof}

\begin{prp}
  Consider a commuting triangle
  \begin{equation*}
    \begin{tikzcd}[column sep=tiny]
      A \arrow[rr,"h"] \arrow[dr,swap,"f"] & & B \arrow[dl,"g"] \\
      & X
    \end{tikzcd}
  \end{equation*}
  with $H:f\htpy g\circ h$. The following three statements hold:
  \begin{enumerate}
  \item If $f$ and $h$ are $k$-connected, then $g$ is $k$-connected.
  \item If $g$ and $h$ are $k$-connected, then $f$ is $k$-connected.
  \item If $f$ and $g$ are $k$-connected, then $h$ is a $k$-equivalence.
  \end{enumerate}
\end{prp}

\begin{proof}
  The first two statements combined assert that if $h$ is $k$-connected, then $f$ is $k$-connected if and only if $g$ is $k$-connected. To see that this equivalence holds, consider for any family $P$ of $k$-truncated types over $X$ the commuting square
  \begin{equation*}
    \begin{tikzcd}[column sep=10em]
      \prd{x:X}P(x) \arrow[r,"\blank\circ g"] \arrow[d,swap,"\blank\circ f"] & \prd{b:B}P(g(b)) \arrow[d,"\blank\circ h"] \\
      \prd{a:A}P(f(a)) \arrow[r,swap,"{\lam{s}{a}\tr_P(H(a),s(a))}"] & \prd{a:A}P(g(h(a)))
    \end{tikzcd}
  \end{equation*}
  In this square, the bottom map is given by postcomposing with the family of equivalences $\tr_P(H(a))$ indexed by $a:A$, so it is an equivalence. The map on the right is an equivalence by \cref{thm:conn-dup}, using the assumption that $h$ is a $k$-connected map. The square commutes by the homotopy
  \begin{equation*}
    \lam{s}\eqhtpy\big(\lam{a}\apd{s}{H(a)}\big).
  \end{equation*}
  Therefore it follows that the precomposition map $\blank\circ f$ is an equivalence if and only if the precomposition map $\blank\circ g$ is. By \cref{thm:conn-dup} we conclude that $f$ is connected if and only if $g$ is. This proves statements (i) and (ii).

  Statement (iii) follows from the facts that any $k$-connected map is a $k$-equivalence by \cref{cor:k-equiv-k-conn} and that the $k$-equivalences satisfy the 3-for-2 property \cref{lem:3-for-2-k-equiv}.
\end{proof}

\subsubsection{The action on homotopy groups}

\begin{thm}
  Consider a map $f:X\to Y$, and suppose that $k\geq -1$. The following are equivalent:
  \begin{enumerate}
  \item The map $f$ is a $k$-equivalence.
  \item The map $f$ is a $(-1)$-equivalence, and for every $0\leq i\leq k$ and every $x:X$, the induced group homomorphism
    \begin{equation*}
      \pi_i(f,x):\pi_i(X,x)\to\pi_i(Y,f(x))
    \end{equation*}
    is an isomorphism.
  \end{enumerate}
\end{thm}

\begin{defn}
  A map $f:X\to Y$ is said to be a \define{weak equivalence} if it is a $0$-equivalence, and it induces an isomorphism
  \begin{equation*}
    \pi_i(f,x):\pi_i(X,x)\cong\pi_i(Y,f(x))
  \end{equation*}
  on homotopy groups, for every $x:X$ and every $i\geq 1$. 
\end{defn}

The following corollary is an instance of Whitehead's principle, which asserts that a map between any two spaces is a homotopy equivalence if and only if it is a weak equivalence. Thus, by the following corollary, Whitehead's principle holds for $k$-types.

\begin{cor}
  Consider two $k$-types $X$ and $Y$, and consider a map $f:X\to Y$ between them. Then the following are equivalent:
  \begin{enumerate}
  \item The map $f$ is an equivalence.
  \item The map $f$ is a weak equivalence.
  \end{enumerate}
\end{cor}

\begin{thm}
  Consider a map $f:X\to Y$. The following are equivalent:
  \begin{enumerate}
  \item The map $f$ is $(k+1)$-connected.
  \item The map $f$ is surjective, and for each $x,x':X$ the action on paths
    \begin{equation*}
      \apfunc{f} : (x=x')\to (f(x)=f(x'))
    \end{equation*}
    is $k$-connected.
  \end{enumerate}
\end{thm}

\begin{thm}
  Consider a surjective map $f:X\to Y$. The following are equivalent:
  \begin{enumerate}
  \item The map $f$ is $k$-connected.
  \item The induced maps on loop spaces
    \begin{equation*}
      \loopspace{f,x}:\loopspace{X,x}\to\loopspace{Y,f(x)}
    \end{equation*}
    is $(k-1)$-connected for every $x:X$.
  \item The induced maps on homotopy groups
    \begin{equation*}
      \pi_i(f,x):\pi_i(X,x)\to\pi_i(Y,f(x))
    \end{equation*}
    are isomorphisms for $0\leq i\leq k$, and it is surjective for $i=k+1$. 
  \end{enumerate}
\end{thm}

\begin{rmk}
  If $f:X\to Y$ is a pointed map between connected types, then conditions (ii) and (iii) in \cref{thm:htpy-groups-conn-map} only have to be checked at the base point.
\end{rmk}

\subsection{Orthogonality}

The idea of orthogonality is that a map $f:A\to B$ is left orthogonal to a map $g:X\to Y$ if for every commuting square of the form
\begin{equation*}
  \begin{tikzcd}
    A \arrow[d,swap,"f"] \arrow[r,"h"] & X \arrow[d,"g"] \\
    B \arrow[r,swap,"i"] & Y,
  \end{tikzcd}
\end{equation*}
with $H:(i\circ f)\htpy (g\circ h)$, the type of diagonal fillers is contractible. The type of diagonal fillers is the type of maps $j:B\to X$ equipped with homotopies
\begin{align*}
  K & : j\circ f\htpy h \\
  L & : g\circ j\htpy i
\end{align*}
and a homotopy $M$ witnessing that the triangle
\begin{equation*}
  \begin{tikzcd}[column sep=small]
    g\circ j\circ f \arrow[rr,"g\cdot K"] \arrow[dr,swap,"L\cdot f"] & & h\circ g \\
    & i\circ f \arrow[ur,swap,"H"]
  \end{tikzcd}
\end{equation*}
commutes. A slicker way to express this condition is to assert that the map
\begin{equation*}
   (B\to X)\to \sm{h:A\to X}{i:B\to Y} i\circ f\htpy g\circ h
\end{equation*}
given by $j\mapsto(j\circ f,g\circ j,\reflhtpy)$ is an equivalence. Indeed, the type of triples $(h,i,H)$ in the codomain is the type of commuting squares with respect to which we stated the orthogonality condition. Now we may even recognize the above map as a gap map of a commuting square, and we arrive at our actual definition of orthogonality.

\begin{defn}
  A map $f:A\to B$ is said to be \define{left orthogonal} to a map $g:X\to Y$, or equivalently the map $g$ is said to be \define{right orthogonal} to $f$, if the commuting square
  \begin{equation*}
    \begin{tikzcd}[column sep=large]
      X^B \arrow[r,"\blank\circ f"] \arrow[d,swap,"g\circ\blank"] & X^A \arrow[d,"g\circ\blank"] \\
      Y^B \arrow[r,swap,"\blank\circ f"] & Y^A
    \end{tikzcd}
  \end{equation*}
  is a pullback square.
\end{defn}

\begin{thm}
Let $f:A\to B$ be a map. The following are equivalent:
\begin{enumerate}
\item The map $f$ is $k$-connected.
\item The map $f$ is left orthogonal to every $k$-truncated map.
is a pullback square.
\end{enumerate}
\end{thm}

\begin{thm}
  Let $f:A\to B$ be a map. The following are equivalent:
  \begin{enumerate}
  \item The map $f$ is a $k$-equivalence.
  \item The map $f$ is left orthogonal to every map between $k$-truncated types.
  \item The map $f$ is left orthogonal to every map $g:X\to Y$ for which the naturality square
    \begin{equation*}
      \begin{tikzcd}
        X \arrow[r,"g"] \arrow[d,swap,"\eta"] & Y \arrow[d,"\eta"] \\
        \trunc{k}{X} \arrow[r,swap,"\trunc{k}{g}"] & \trunc{k}{Y}
      \end{tikzcd}
    \end{equation*}
    is a pullback square. Such maps are called \define{$k$-\'etale}.
  \end{enumerate}
\end{thm}

\subsection{The connectedness of suspensions}

We will use connected maps to prove the connectedness of suspensions.

\begin{prp}\label{prp:conn-pushout}
  Consider a pushout square
  \begin{equation*}
    \begin{tikzcd}
      S \arrow[d,swap,"f"] \arrow[r,"g"] & B \arrow[d,"j"] \\
      A \arrow[r,swap,"i"] & X.
    \end{tikzcd}
  \end{equation*}
  If the map $f:S\to A$ is $k$-connected, then so is the map $j:B\to X$.
\end{prp}

\begin{proof}
  We claim that the map $j:B\to X$ is left orthogonal to any $k$-truncated map $p:Y\to Z$, which is equivalent to the property that $j$ is $k$-connected. To see that $j$ is left orthogonal to $p$, consider the commuting cube
  \begin{equation*}
    \begin{tikzcd}
      & Y^X \arrow[dl] \arrow[d] \arrow[dr] & \\
      Y^A \arrow[d] & Y^B \arrow[dl] \arrow[dr] & Z^X \arrow[dl,crossing over] \arrow[d] \\
      Y^S \arrow[dr] & Z^A \arrow[d] \arrow[from=ul,crossing over] & Z^B \arrow[dl] \\
      & Z^S.
    \end{tikzcd}
  \end{equation*}
  In this cube, the front left square is a pullback square because the map $f:S\to A$ is assumed to be $k$-connected, and therefore it is left orthogonal to the $k$-truncated map $p$. The back left and front right squares are pullback squares by the pullback property of pushouts. Therefore it follows that the back right square is a pullback square. This shows that $j$ is left orthogonal to $p$.
\end{proof}

\begin{lem}\label{lem:conn-mismatch}
  A pointed type $X$ is $(k+1)$-connected if and only if the point inclusion
  \begin{equation*}
    \unit\to X
  \end{equation*}
  is a $k$-connected map.
\end{lem}

\begin{proof}
  Since $X$ is assumed to have a base point $x_0:X$, it follows that $X$ is $(k+1)$-connected if and only if its identity types $(x=y)$ are $k$-connected. Now the claim follows from the fact that there is an equivalence
  \begin{equation*}
    \fib{\const_{x_0}}{y}\simeq (x_0=y).\qedhere
  \end{equation*}
\end{proof}

\begin{thm}\label{thm:conn-suspension}
  If $X$ is an $k$-connected type, then its suspension $\susp X$ is $(k+1)$-connected.
\end{thm}

\begin{proof}
  The type $X$ is $k$-connected if and only if the map $\const_\ttt:X\to\unit$ is a $k$-connected map. Recall that the suspension of $X$ is a pushout
  \begin{equation*}
    \begin{tikzcd}
      X \arrow[d,swap,"\const_\ttt"] \arrow[r,"\const_\ttt"] & \unit \arrow[d,"\south"] \\
      \unit \arrow[r,swap,"\north"] & \susp X.
    \end{tikzcd}
  \end{equation*}
  Therefore we see by \cref{prp:conn-pushout} that the point inclusions $\north,\south:\unit\to\susp X$ are both $k$-connected maps. By \cref{lem:conn-mismatch} it follows that $\susp X$ is a $(k+1)$-connected type.
\end{proof}

\begin{cor}\label{cor:conn-sphere}
  The $n$-sphere is $(n-1)$-connected.
\end{cor}

\begin{proof}
  The $0$-sphere is $(-1)$-connected, since it contains a point. Thus the claim follows by induction on $n:\N$, using \cref{thm:conn-suspension}.
\end{proof}

\subsection{The join connectivity theorem}

\begin{thm}
  If $X$ is $k$-connected and $Y$ is $l$-connected, then their join $\join{X}{Y}$ is $(k+l+2)$-connected.
\end{thm}

\begin{thm}
Consider a pullback square
\begin{equation*}
\begin{tikzcd}
C \arrow[r] \arrow[d] & B \arrow[d] \\
A \arrow[r] & X.
\end{tikzcd}
\end{equation*}
If the maps $A\to X$ and $B\to X$ are $k$- and $l$-connected, respectively, then the map $A\sqcup^C B\to X$ is $(k+l+2)$-connected.
\end{thm}

\begin{thm}
  The connected maps contain the equivalences, are closed under coproducts, pushouts, retracts, and transfinite compositions.
\end{thm}

\begin{exercises}
  \exercise Show that every type is equivalent to a disjoint union of connected components, i.e., show that for every type $X$ there is a family of connected types $B_i$ by a set $I$, with an equivalence
  \begin{equation*}
    X \eqvsym \sm{i:I}B_i.
  \end{equation*}
\exercise Let $f:A\to_\ast B$ be a pointed map between pointed $n$-connected types, for $n\geq -1$. Show that the following are equivalent:
\begin{enumerate}
\item $f$ is an equivalence.
\item $\loopspace[n+1]{f}$ is an equivalence. 
\end{enumerate}
\exercise Show that if
\begin{equation*}
\begin{tikzcd}
A \arrow[r] \arrow[d,swap,"f"] & B \arrow[d,"g"] \\
X \arrow[r] & Y
\end{tikzcd}
\end{equation*}
is \define{$k$-cocartesian} in the sense that the cogap map is $k$-connected, then the map $\mathsf{cofib}(f)\to \mathsf{cofib}(g)$ is $k$-connected.
\exercise Show that if $f:X\to Y$ is a $k$-connected map, then so is
\begin{equation*}
  \begin{tikzcd}
    \trunc{l}{f}:\trunc{l}{X}\to\trunc{l}{Y}
  \end{tikzcd}
\end{equation*}
for any $l\geq-2$.
\exercise Consider a commuting square
\begin{equation*}
\begin{tikzcd}
A \arrow[d,swap,"f"] \arrow[r] & B \arrow[d,"g"] \\
X \arrow[r] & Y
\end{tikzcd}
\end{equation*}
\begin{subexenum}
\item Show that if the square is $k$-cartesian and $g$ is $k$-connected, then so is $f$.
\item Show that if $f$ is $k$-connected and $g$ is $(k+1)$-connected, then the square is $k$-cartesian. 
\end{subexenum}
\exercise
\begin{subexenum}
\item Show that any sequential colimit of $k$-connected types is again $k$-connected.
\item Show that if every map in a type sequence
  \begin{equation*}
    \begin{tikzcd}
      A_0\arrow[r] & A_1 \arrow[r] & A_2 \arrow[r] & \cdots
    \end{tikzcd}
  \end{equation*}
  is $k$-connected, then so is the transfinite composition $A_0\to A_\infty$.
\end{subexenum}
\exercise Recall that a commuting square is called $k$-cartesian, if its gap map is $k$-connected. Show that $(k+1)$-truncation preserves $l$-cartesian squares for any $l\leq k$, i.e., show that for any $l\leq k$, if a square
\begin{equation*}
  \begin{tikzcd}
    C \arrow[r,"q"] \arrow[d,swap,"p"] & B \arrow[d,"g"] \\
    A \arrow[r,swap,"f"] & X.
  \end{tikzcd}
\end{equation*}
is $l$-cartesian, then the square
\begin{equation*}
  \begin{tikzcd}[column sep=large]
    \trunc{k+1}{C} \arrow[r,"\trunc{k+1}{q}"] \arrow[d,swap,"\trunc{k+1}{p}"] & \trunc{k+1}{B} \arrow[d,"\trunc{k+1}{g}"] \\
    \trunc{k+1}{A} \arrow[r,swap,"\trunc{k+1}{f}"] & \trunc{k+1}{X}
  \end{tikzcd}
\end{equation*}
is $l$-cartesian.
\exercise Generalize \cref{rmk:conn-3-for-2} to show that for every $k\geq-1$, the $k$-connected maps do not satisfy the 3-for-2 property.
\exercise Consider a commuting square
\begin{equation*}
\begin{tikzcd}
A \arrow[d,swap,"f"] \arrow[r] & B \arrow[d,"g"] \\
X \arrow[r] & Y
\end{tikzcd}
\end{equation*}
Show that the following are equivalent:
\begin{enumerate}
\item The map $A\to X\times_Y B$ is $n$-connected. In this case the square is called \define{$n$-cartesian}.
\item For each $x:X$ the map
\begin{equation*}
\fib{f}{x}\to \fib{g}{f(x)}
\end{equation*}
is $n$-connected.
\end{enumerate}
\exercise Consider a map $f:A\to B$. Show that the following are equivalent:
  \begin{enumerate}
  \item The map $f$ is a weak equivalence.
  \item The map $f$ is $\infty$-connected, in the sense that $f$ is $k$-connected for each $k$.
  \item The map $f$ is left orthogonal to any map between truncated types of any truncation level.
  \item The map $f$ is left orthogonal to any truncated map, for any truncation level.
  \end{enumerate}
  Thus we see that, while the classes of $k$-connected maps and $k$-equivalences differ for finite $k\geq-1$, they come to agree at $\infty$.
  \exercise Consider a pointed $(k+1)$-connected type $X$. Show that every $k$-truncated map $f:A\to X$ trivializes, in the sense that there is a $k$-type $B$ and an equivalence $e:\eqv{A}{X\times B}$ for which the triangle
  \begin{equation*}
    \begin{tikzcd}[column sep=0]
      A \arrow[rr,"e"] \arrow[dr,swap,"f"] & & X\times B \arrow[dl,"\proj 1"] \\
      \phantom{X\times B} & X
    \end{tikzcd}
  \end{equation*}
  commutes.
  \exercise Consider a $k$-equivalence $f:B'\to B$. Show that the base-change functor induces an equivalence
  \begin{equation*}
    \Big(\sm{E:\UU}{p:E\to B}\isetale_k(p)\Big)\simeq\Big(\sm{E':\UU}{p':E'\to B'}\isetale_k(p')\Big).
  \end{equation*}
  In other words, for every $k$-\'etale map $p':E'\to B'$ there is a unique $k$-\'etale map $p:E\to B$ equipped with a map $q:E'\to E$ such that the square
  \begin{equation*}
    \begin{tikzcd}
      E' \arrow[d,swap,"{p'}"] \arrow[r,densely dotted,"q"] & E \arrow[d,densely dotted,"p"] \\
      B' \arrow[r,swap,"f"] & B
    \end{tikzcd}
  \end{equation*}
  commutes and is a pullback square. In this sense $k$-\'etale maps descend along $k$-equivalences.
\end{exercises}
