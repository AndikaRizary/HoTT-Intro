% !TEX root = hott_intro.tex

\section{Connected types and maps}

\begin{defn}
A type $A$ is said to be $n$-connected if $\trunc{n}{A}$ is contractible.
If $A$ is $0$-connected we also say that $A$ is \define{(path) connected},
and if $A$ is $1$-connected we also say that $A$ is \define{simply connected}.
\end{defn}

\begin{thm}
Let $f:A\to B$ be a map. The following are equivalent:
\begin{enumerate}
\item The map $f$ is \define{$n$-connected}, i.e. the fibers of $f$ are $n$-connected types.
\item The map $f$ is \define{left orthogonal} with respect to every $n$-truncated map $g:X\to Y$, i.e. the square
\begin{equation*}
\begin{tikzcd}
X^B \arrow[r,"\blank\circ f"] \arrow[d,swap,"g\circ \blank"] & X^A \arrow[d,"g\circ\blank"] \\
Y^B \arrow[r,swap,"\blank\circ f"] & Y^A
\end{tikzcd}
\end{equation*}
is a pullback square.
\item For each $i\leq n$, the map $f$ induces an isomorphism
\begin{equation*}
\pi_i(f):\pi_i(A)\to\pi_i(B)
\end{equation*}
of homotopy groups, and 
\begin{equation*}
\pi_{n+1}(f):\pi_{n+1}(A)\to\pi_{n+1}(B)
\end{equation*}
is surjective.
\end{enumerate}
\end{thm}

\begin{thm}
Consider a commuting square
\begin{equation*}
\begin{tikzcd}
A \arrow[d,swap,"f"] \arrow[r] & B \arrow[d,"g"] \\
X \arrow[r] & Y
\end{tikzcd}
\end{equation*}
The following are equivalent:
\begin{enumerate}
\item The map $A\to X\times_Y B$ is $n$-connected. In this case the square is called \define{$n$-cartesian}.
\item For each $x:X$ the map
\begin{equation*}
\fib{f}{x}\to \fib{g}{f(x)}
\end{equation*}
is $n$-connected.
\end{enumerate}
\end{thm}

\begin{thm}
If $X$ is $m$-connected and $Y$ is $n$-connected, then $\join{X}{Y}$ is $(m+n+2)$-connected. 
\end{thm}

\begin{thm}
Consider a pullback square
\begin{equation*}
\begin{tikzcd}
C \arrow[r] \arrow[d] & B \arrow[d] \\
A \arrow[r] & X.
\end{tikzcd}
\end{equation*}
If the maps $A\to X$ and $B\to X$ are $m$- and $n$-connected, respectively, then the map $A\sqcup^C B\to X$ is $(m+n+2)$-connected.
\end{thm}

\begin{thm}
  The connected maps contain the equivalences, are closed under coproducts, pushouts, retracts, and transfinite compositions.
\end{thm}

\begin{exercises}
\exercise Show that a pointed $X$ is $n$-connected precisely when $\pi_k(X)=0$ for each $k\leq n$.
\exercise Show that retracts of $n$-connected types are again $n$-connected.
\exercise Let $f:A\to_\ast B$ be a pointed map between pointed $n$-connected types, for $n\geq -1$. Show that the following are equivalent:
\begin{enumerate}
\item $f$ is an equivalence.
\item $\loopspace[n+1]{f}$ is an equivalence. 
\end{enumerate}
\exercise Let $f:A\to B$ be a surjective map, and let $g:A\to C$ be any map. Show that if there is a unique extension
\begin{equation*}
\begin{tikzcd}
\fib{f}{f(a)} \arrow[r,"g\circ\pi_1"] \arrow[d] & C \\
\unit \arrow[ur,densely dotted]
\end{tikzcd}
\end{equation*}
for any $a:A$, then $g$ extends uniquely along $f$.
\exercise Consider a span $A \leftarrow S \rightarrow B$, in which the map $S\to A$ is $n$-connected. Show that the map $\inr : B\to A\sqcup^S B$ is again $n$-connected.
\exercise Show that if
\begin{equation*}
\begin{tikzcd}
A \arrow[r] \arrow[d,swap,"f"] & B \arrow[d,"g"] \\
X \arrow[r] & Y
\end{tikzcd}
\end{equation*}
is a $k$-cocartesian, then the map $\mathsf{cofib}(f)\to \mathsf{cofib}(g)$ is $k$-connected.
\exercise Consider a commuting square
\begin{equation*}
\begin{tikzcd}
A \arrow[d,swap,"f"] \arrow[r] & B \arrow[d,"g"] \\
X \arrow[r] & Y
\end{tikzcd}
\end{equation*}
\begin{subexenum}
\item Show that if the square is $n$-cartesian and $g$ is $n$-connected, then so is $f$.
\item Show that if $f$ is $n$-connected and $g$ is $(n+1)$-connected, then the square is $n$-cartesian. 
\end{subexenum}
\exercise Show that any sequential colimit of $n$-connected types is again $n$-connected.
\exercise Consider a pushout square
\begin{equation*}
\begin{tikzcd}
A \arrow[d,swap,"f"] \arrow[r] & B \arrow[d,"g"] \\
X \arrow[r] & Y
\end{tikzcd}
\end{equation*}
Show that if $f$ is $n$-connected, then $g$ is $n$-connected.
\end{exercises}
