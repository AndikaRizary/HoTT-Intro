\section{The univalence axiom}

\subsection{Equivalent forms of the univalence axiom}

The univalence axiom characterizes the identity type of the universe. Roughly speaking, it asserts that equivalent types are equal. It is considered to be an \emph{extensionality principle}\index{extensionality principle!for types} for types.

\begin{axiom}[Univalence]\label{axiom:univalence}
  The \define{univalence axiom}\index{univalence axiom} on a universe $\UU$ is the statement that for any $A:\UU$ the family of maps\index{equiv-eq@{$\equiveq$}}
\begin{equation*}
\equiveq : \prd{B:\UU} (\id{A}{B})\to(\eqv{A}{B}).
\end{equation*}
that sends $\refl{A}$ to the identity equivalence $\idfunc:\eqv{A}{A}$ is a family of equivalences.\index{identity type!of a universe} A universe satisfying the univalence axiom is referred to as a \define{univalent universe}\index{univalent universe}. If $\UU$ is a univalent universe we will write
$\eqequiv$\index{eq-equiv@{$\eqequiv$}}
for the inverse of $\equiveq$.
\end{axiom}

The following theorem is a special case of the fundamental theorem of identity types (\cref{thm:id_fundamental})\index{fundamental theorem of identity types}. Subsequently we will assume that any type is contained in a univalent universe.\index{axiom!univalence}

\begin{thm}\label{thm:univalence}
The following are equivalent:
\begin{enumerate}
\item The univalence axiom holds.
\item The type
\begin{equation*}
\sm{B:\UU}\eqv{A}{B}
\end{equation*}
is contractible for each $A:\UU$.
\item The principle of \define{equivalence induction}\index{equivalence induction}\index{induction principle!for equivalences} holds: for every $A:\UU$ and for every type family
\begin{equation*}
P:\prd{B:\UU} (\eqv{A}{B})\to \UU,
\end{equation*}
the map
\begin{equation*}
\Big(\prd{B:\UU}{e:\eqv{A}{B}}P(B,e)\Big)\to P(A,\idfunc[A])
\end{equation*}
given by $f\mapsto f(A,\idfunc[A])$ has a section.
\end{enumerate}
\end{thm}

\subsection{Univalence implies function extensionality}
One of the first applications of the univalence axiom was Voevodsky's theorem that the univalence axiom on a universe $\UU$ implies function extensionality for types in $\UU$. The proof uses the fact that weak function extensionality implies function extensionality.

We will also make use of the following lemma. Note that this statement was also part of \cref{lem:postcomp_equiv}. That exercise is solved using function extensionality. Since our present goal is to derive function extensionality from the univalence axiom, we cannot make use of that exercise.

\begin{lem}\label{lem:postcomp-equiv}
  For any equivalence $e:\eqv{X}{Y}$ in a univalent universe $\UU$, and any type $A$, the post-composition map
  \begin{equation*}
    e\circ\blank : (A \to X) \to (A\to Y)
  \end{equation*}
  is an equivalence.
\end{lem}

\begin{proof}
  The statement is obvious for the identity equivalence $\idfunc : \eqv{X}{X}$. Therefore the claim follows by equivalence induction, which is by \cref{thm:univalence} one of the equivalent forms of the univalence axiom.
\end{proof}

\begin{thm}\label{thm:funext-univalence}\index{univalence axiom!implies function extensionality}
  For any universe $\UU$, the univalence axiom on $\UU$ implies function extensionality on $\UU$.
\end{thm}

\begin{proof}
  Note that by \cref{thm:funext_wkfunext}\index{weak function extensionality} it suffices to show that univalence implies weak function extensionality, where we note that \cref{thm:funext_wkfunext} also holds when it is restricted to small types.
  
Suppose that $B:A\to \UU$ is a family of contractible types. Our goal is to show that the product $\prd{x:A}B(x)$ is contractible.
Since each $B(x)$ is contractible, the projection map $\proj 1:\big(\sm{x:A}B(x)\big)\to A$ is an equivalence by \cref{ex:proj_fiber}.

Now it follows by \cref{lem:postcomp-equiv} that $\proj1\circ\blank$ is an equivalence. Consequently, it follows from \cref{thm:contr_equiv} that the fibers of
\begin{equation*}
\proj 1\circ\blank : \Big(A\to \sm{x:A}B(x)\Big)\to (A\to A)
\end{equation*}
are contractible. In particular, the fiber at $\idfunc[A]$ is contractible. Therefore it suffices to show that $\prd{x:A}B(x)$ is a retract of $\sm{f:A\to\sm{x:A}B(x)}\proj 1\circ f=\idfunc[A]$. In other words, we will construct
\begin{equation*}
\begin{tikzcd}
\Big(\prd{x:A}B(x)\Big) \arrow[r,"i"] & \Big(\sm{f:A\to\sm{x:A}B(x)}\proj 1\circ f=\idfunc[A]\Big) \arrow[r,"r"] & \Big(\prd{x:A}B(x)\Big),
\end{tikzcd}
\end{equation*}
and a homotopy $r\circ i\htpy \idfunc$.

We define the function $i$ by
\begin{equation*}
  i(f) \defeq (\lam{x}(x,f(x)),\refl{\idfunc}).
\end{equation*}
To see that this definition is correct, we need to know that
\begin{equation*}
  \lam{x}\proj 1(x,f(x))\jdeq \idfunc.
\end{equation*}
This is indeed the case, by the $\eta$-rule\index{eta-rule@{$\eta$-rule}} for $\Pi$-types.

Next, we define the function $r$. Let $h:A\to \sm{x:A}B(x)$, and let $p:\proj 1 \circ h = \idfunc$. Then we have the homotopy $H\defeq\htpyeq(p):\proj 1 \circ h \htpy \idfunc$. Then we have $\proj 2(h(x)):B(\proj 1(h(x)))$ and we have the identification $H(x):\proj 1(h(x))=x$. Therefore we define $r$ by
\begin{equation*}
  r((h,p),x)\defeq \tr_B(H(x),\proj 2(h(x))).
\end{equation*}

We note that if $p\jdeq \refl{\idfunc}$, then $H(x)\jdeq\refl{x}$. In this case we have the judgmental equality $r((h,\refl{}),x)\jdeq\proj 2 (h(x))$. Thus we see that $r\circ i\jdeq \idfunc$ by another application of the $\eta$-rule for $\Pi$-types.
\end{proof}

\subsection{Propositional extensionality and posets}

\begin{thm}\label{thm:propositional-extensionality}
  Propositions satisfy \define{propositional extensionality}\index{propositional extensionality}\index{extensionality principle!for propositions}:
  for any two propositions $P$ and $Q$, the canonical map\index{bi-implication}\index{iff-eq@{$\iffeq$}}
  \begin{equation*}
    \iffeq:(P=Q)\to (P\leftrightarrow Q)
  \end{equation*}
  that sends $\refl{P}$ to $(\idfunc,\idfunc)$ is an equivalence. It follows that the type $\prop$ of propositions in $\UU$ is a set.\index{Prop@{$\prop$}!is a set}
\end{thm}

Note that for any $P:\prop$, we usually also write $P$ for the underlying type of the proposition $P$. If we would be more formal about it we would have to write $\proj 1(P)$ for the underlying type, since $\prop$ is the $\Sigma$-type $\sm{X:\UU}\isprop(X)$. In the following proof it is clearer if we use the more formal notation $\proj 1(P)$ for the underlying type of a proposition $P$.

\begin{proof}
  We note that the identity type $P=Q$ is an identity type in $\prop$. However, since $\isprop(X)$ is a proposition for any type $X$, it follows that the map
  \begin{equation*}
    \apfunc{\proj 1} : (P = Q) \to (\proj 1(P) = \proj 1(Q))
  \end{equation*}
  is an equivalence. Now we observe that we have a commuting square
  \begin{equation*}
    \begin{tikzcd}[column sep=huge]
      (P=Q) \arrow[d,swap,"\apfunc{\proj 1}"] \arrow[r] & (P\leftrightarrow Q) \\
      (\proj 1(P)=\proj 1(Q)) \arrow[r,swap,"\equiveq"] & (\proj 1(P)\simeq\proj 1(Q)) \arrow[u,swap,"\simeq"]
    \end{tikzcd}
  \end{equation*}
  Since the left, bottom, and right map are equivalences, it follows that the top map is an equivalence.
\end{proof}

\begin{defn}
  A \define{partially ordered set (poset)}\index{partially ordered set|see poset}\index{poset} is a set $P$ equipped with a relation
  \begin{equation*}
    \blank\leq\blank : P \to (P \to \prop)
  \end{equation*}
  that is \define{reflexive}\index{reflexive!poset} (for every $x:P$ we have $x\leq x$), \define{transitive}\index{transitive!poset} (for every $x,y,z:P$ such that $x\leq y$ and $y\leq z$ we have $x\leq z$), and \define{anti-symmetric}\index{anti-symmetric!poset} (for every $x,y:P$ such that $x\leq y$ and $y\leq x$ we have $x=y$).
\end{defn}

\begin{rmk}
  The condition that $X$ is a set can be omitted from the definition of a poset. Indeed, if $X$ is any type that comes equipped with a $\prop$-valued ordering relation $\leq$ that is reflexive and anti-symmetric, then $X$ is a set by \cref{lem:prop_to_id}.
\end{rmk}

\begin{eg}
  The type $\prop$ is a poset\index{Prop@{$\prop$}!is a poset}, where the ordering relation is given by implication: $P$ is less than $Q$ if $P\to Q$. The fact that $P\to Q$ is a proposition is a special case of \cref{cor:funtype_trunc}. The relation $P\to Q$ is reflexive by the identity function, and transitive by function composition. Moreover, the relation $P\to Q$ is anti-symmetric by \cref{thm:propositional-extensionality}\index{propositional extensionality}.
\end{eg}

\begin{eg}
  The type of natural numbers\index{natural numbers!is a poset with leq@{is a poset with $\leq$}}\index{natural numbers!is a poset with divisibility}\index{poset!N with leq@{$\N$ with $\leq$}}\index{poset!N with divisibility@{$\N$ with divisibility}} comes equipped with at least two important poset structures. The first is given by the usual ordering relation $\leq$, and the second is given by the relation $d\mid n$ that $d$ divides $n$.
\end{eg}

\begin{thm}
  For any poset $P$ and any type $X$, the set $P^X$ is a poset.\index{poset!closed under exponentials} In particular the type of subtypes of any type is a poset.\index{poset!type of subtypes}\index{subtype!poset}
\end{thm}

\begin{proof}
  Let $P$ be a poset with ordering $\leq$, and let $X$ be a type. Then $P^X$ is a set by \cref{cor:funtype_trunc}. For any $f,g:X\to P$ we define
  \begin{equation*}
    (f\leq g) \defeq \prd{x:X}f(x)\leq g(x).
  \end{equation*}
  Reflexivity and transitivity follow immediately from reflexivity and transitivity of the original relation. Moreover, by the anti-symmetry of the original relation it follows that
  \begin{equation*}
    (f\leq g)\times (g\leq f) \to (f\htpy g). 
  \end{equation*}
  Therefore we obtain an identification $f=g$ by function extensionality. The last claim follows immediately from the fact that a subtype of $X$ is a map $X\to\prop$, and the fact that $\prop$ is a poset.
\end{proof}

\begin{exercises}
\exercise \label{ex:istrunc_UUtrunc}
\begin{subexenum}
\item Use the univalence axiom to show that the type $\sm{A:\UU}\iscontr(A)$ of all contractible types in $\UU$ is contractible.\index{universe!of contractible types}
\item Use \cref{cor:emb_into_ktype,cor:funtype_trunc,ex:isprop_isequiv} to show that if $A$ and $B$ are $(k+1)$-types, then the type $\eqv{A}{B}$ is also a $(k+1)$-type.\index{A simeq B@{$\eqv{A}{B}$}!truncatedness}
\item Use univalence to show that the universe of $k$-types\index{universe!of k-types@{of $k$-types}}\index{U leq k@{$\UU^{\leq k}$}}\index{k-type@{$k$-type}!universe of k-types@{universe of $k$-types}}\index{truncated type!universe of k-types@{universe of $k$-types}}
\begin{equation*}
\UU^{\leq k}\defeq \sm{X:\UU}\istrunc{k}(X)
\end{equation*}
is a $(k+1)$-type, for any $k\geq -2$.
\item Show that $\UU^{\leq-1}$ is not a proposition.\index{universe!of propositions}
\item Show that $\eqv{(\eqv{\bool}{\bool})}{\bool}$, and conclude by the univalence axiom that the universe of sets\index{universe!of sets} $\UU^{\leq 0}$ is not a set. 
\end{subexenum}
\exercise Use the univalence axiom to show that the type $\sm{P:\prop}P$ is contractible.
\exercise Let $A$ and $B$ be small types. 
\begin{subexenum}
\item Construct an equivalence
\begin{equation*}
\eqv{(A\to (B\to\UU))}{\Big(\sm{S:\UU} (S\to A)\times (S\to B)\Big)}
\end{equation*}
\item We say that a relation $R:A\to (B\to\UU)$ is \define{functional}\index{relation!functional} if it comes equipped with a term of type\index{is-function(R)@{$\isfunction(R)$}}
\begin{equation*}
\isfunction(R) \defeq \prd{x:A}\iscontr\Big(\sm{y:B}R(x,y)\Big)
\end{equation*}
For any function $f:A\to B$, show that the \define{graph}\index{graph!of a function} of $f$ 
\begin{equation*}
\graph_f:A\to (B\to \UU)
\end{equation*}
given by $\graph_f(a,b)\defeq (f(a)=b)$ is a functional relation from $A$ to $B$.
\item Construct an equivalence
\begin{equation*}
\eqv{\Big(\sm{R:A\to (B\to\UU)}\isfunction(R)\Big)}{(A\to B)}
\end{equation*}
\item Given a relation $R:A\to (B\to \UU)$ we define the \define{opposite relation}\index{relation!opposite relation}\index{opposite relation}\index{op R@{$\opp{R}$}}
\begin{equation*}
\opp{R} : B\to (A\to\UU)
\end{equation*}
by $\opp{R}(y,x)\defeq R(x,y)$. Construct an equivalence\index{A simeq B@{$\eqv{A}{B}$}!as relation}
\begin{equation*}
\eqv{\Big(\sm{R:A\to (B\to \UU)}\isfunction(R)\times\isfunction(\opp{R})\Big)}{(\eqv{A}{B})}.
\end{equation*}
\end{subexenum}
\exercise
  \begin{subexenum}
  \item Show that $\isdecidable(P)$ is a proposition\index{is-decidable@{$\isdecidable$}!is a proposition}, for any proposition $P$.
  \item Show that $\classicalprop$%
    \index{classical-Prop@{$\classicalprop$}!classical-Prop bool@{$\classicalprop\eqvsym\bool$}} is equivalent to $\bool$.
  \end{subexenum}
\exercise Recall that $\UU_\ast$ is the universe of pointed types\index{UU*@{$\UU_\ast$}}.
  \begin{subexenum}
  \item For any $(A,a)$ and $(B,b)$ in $\UU_\ast$, write $(A,a)\simeq_\ast(B,b)$ for the type of \define{pointed equivalences}\index{pointed equivalence}\index{equivalence!pointed equivalence} from $A$ to $B$, i.e.,
    \begin{equation*}
      (A,a)\simeq_\ast (B,b)\defeq \sm{e:A\simeq B}e(a)=b.
    \end{equation*}
    Show that the canonical map\index{UU*@{$\UU_\ast$}!identity type}\index{identity type!of UU*@{of $\UU_\ast$}}
    \begin{equation*}
      \big((A,a)=(B,b)\big)\to \Big((A,a)\simeq (B,b)\Big)
    \end{equation*}
    sending $\refl{(A,a)}$ to the pair $(\idfunc,\refl{a})$, is an equivalence.
  \item Construct for any pointed type $(X,x_0)$ an equivalence
    \begin{equation*}
      \Big(\sm{P:X\to \UU}P(x_0)\Big)\simeq \sm{(A,a_0):\UU_\ast}(A,a_0)\to_\ast(X,x_0).
    \end{equation*}
  \end{subexenum}
\exercise Show that any subuniverse\index{subuniverse!closed under equivalences} is closed under equivalences, i.e., show that there is a map
  \begin{equation*}
    (X\simeq Y) \to (P(X)\to P(Y))
  \end{equation*}
  for any subuniverse $P:\UU\to\prop$, and any $X,Y:\UU$.
  \exercise Show that the universe inclusions
  \begin{equation*}
    \UU \to \UU^+\qquad\text{and}\qquad \UU\to \UU\sqcup\VV
  \end{equation*}
  defined in \cref{rmk:universe-constructions}, are embeddings.
\end{exercises}
