% !TEX root = hott_intro.tex

\section{The Blakers-Massey theorem}
The Blakers-Massey theorem is a connectivity theorem which can be used to prove the Freudenthal suspension theorem, giving rise to the field of \emph{stable homotopy theory}. It was proven in the setting of homotopy type theory by Lumsdaine et al, and their proof was the first that was given entirely in an elementary way, using only constructions that are invariant under homotopy equivalence. 

\subsection{The Blakers-Massey theorem}
Consider a span $A \leftarrow S \rightarrow B$, consisting of an $m$-connected map $f:S\to A$ and an $n$-connected map $g:S\to B$. We take the pushout of this span, and subsequently the pullback of the resulting cospan, as indicated in the diagram
\begin{equation}\label{eq:BM}
\begin{tikzcd}
S \arrow[drr,bend left=15,"g"] \arrow[ddr,bend right=15,swap,"f"] \arrow[dr,densely dotted,"u" near end] \\
& A \times_{(A \sqcup^S B)} B \arrow[r,"\pi_2"] \arrow[d,"\pi_1"] & B \arrow[d,"\inr"] \\
& A \arrow[r,swap,"\inl"] & A \sqcup^S B.
\end{tikzcd}
\end{equation}
The universal property of the pullback determines a unique map $u:S\to A \times_{(A\sqcup^S B)} B$ as indicated.

\begin{thm}[Blakers-Massey]
The map $u:S\to A \times_{(A\sqcup^S B)} B$ of \cref{eq:BM} is $(n+m)$-connected.
\end{thm}

\subsection{The Freudenthal suspension theorem}

\begin{thm}
  If $X$ is a $k$-connected pointed type, then the canonical map
  \begin{equation*}
    X \to \loopspace{\susp{X}}
  \end{equation*}
  is $2k$-connected.
\end{thm}

\begin{thm}
  $\pi_n(\sphere{n})=\Z$ for $n\geq 1$.
\end{thm}

\subsection{Higher groups}
\label{sec:higher-groups}

Recall that types in HoTT may be viewed as $\infty$-groupoids:
elements are objects, paths are morphisms, higher paths are higher
morphisms, etc.

It follows that \emph{pointed connected} types $B$ may be viewed as higher
groups, with \define{carrier} $\loopspacesym B$.
The neutral element is the identity path,
the group operation is given by path composition,
and higher paths witness the unit and associativity laws.
Of course, these higher paths are themselves subject to further laws,
etc., but the beauty of the type-theoretic definition is
that we don't have to worry about that:
all the (higher) laws follow from the rules of the identity types.
Writing $G$ for the carrier $\loopspacesym B$, it is common to write $BG$ for the pointed
connected type $B$, which comes equipped with an identification $G = \loopspacesym BG$.
We call $BG$ the \define{delooping} of $G$.

The type of pointed types is
$\UU_\pt \defeq  \sm{A:\UU} A$. The type of $n$-truncated types is
$\UU^{\le n} \defeq  \sm{A:\UU}\istrunc{n}{A}$ and for $n$-connected types it is
$\UU^{>n} \defeq  \sm{A:\UU}\mathsf{is\usc{}conn}_n(A)$. We will combine these notations as needed.

\begin{defn}
We define the type of \define{higher groups}, or \define{$\infty$-groups}, to be
\begin{equation*}
\infty\mathsf{Grp}\defeq \sm{G:\UU}{BG:\UU_\pt^{>0}} \eqv{G}{\loopspacesym BG}.
\end{equation*}
When $G$ is an $\infty$-group, we also write $G$ for its first projection, called the \define{carrier} of $G$.
\end{defn}

\begin{rmk}
Note that we have equivalences
\begin{align*}
  \infty\mathsf{Grp}
  &\jdeq   \sm{G:\UU}{BG:\UU_\pt^{>0}} \eqv{G}{\loopspacesym BG} \\
  &\eqvsym \sm{G:\UU_\pt}{BG:\UU_\pt^{>0}} G \eqvsym_\pt \loopspacesym BG \\
  &\eqvsym \UU_\pt^{>0}
\end{align*}
for the type of higher groups. 
\end{rmk}

Automorphism groups form a major class of examples of $\infty$-groups.
Given \emph{any} type $A$ and any object $a : A$, the automorphism group at $a$ is defined as
\define{automorphism group} $\Aut a\defeq (a=a)$. 
This is indeed an $\infty$-group, because it is the loop space of the connected component of $A$ at $a$, i.e. we define~$\BAut a \defeq  \im(a : 1 \to A) = (x : A) \times \trunc{-1}{a=x}$.
From this definition it is immediate that $\Aut a=\loopspacesym\BAut a$, so we see that $\Aut a$ is indeed an example of an $\infty$-group. 

If we take $A = \mathsf{Set}$, we get the usual symmetric groups
$\Sym_n \defeq  \Aut(\Fin(n))$, where $\Fin(n)$ is a set with $n$
elements. (Note that $\BS_n = \BAut(\Fin (n))$ is the type of all
$n$-element sets.)

We recover the ordinary set-level groups by requiring that $G$ is a $0$-type, or equivalently, that $BG$
is a $1$-type. This leads us to introduce:

\begin{defn}
We define the type of \define{groupal $(n-1)$-groupoids}, or \define{$n$-groups}, to be
\begin{equation*}
n\mathsf{Grp} \defeq \sm{G:\UU_\pt^{<n}}{BG :\UU_\pt^{>0}} G \eqvsym_\pt \loopspacesym BG.
\end{equation*}
We write $\mathsf{Grp}$ for the type of $1$-groups.
\end{defn}

The type of $n$-groups is therefore equivalent to the type of pointed connected $(n+1)$-types. Note that if $A$ is an $(n+1)$-type, then $\Aut a$ is an $(n+1)$-group because $\Aut a$ is $n$-truncated.

For example, the integers $\mathbb{Z}$ as an additive group are from this
perspective represented by their delooping $\mathop{\mathrm{B}\mathbb{Z}}=\bS^1$, i.e., the circle.
Indeed, any set-level group $G$ is represented as its delooping $BG\defeq K(G,1)$.

Moving across the homotopy hypothesis, for every pointed type $(X,x)$
we have the \define{fundamental $\infty$-group of $X$},
$\Pi_\infty(X,x)\defeq \Aut x$. Its $(n-1)$-truncation (an instance of
decategorification, see \cref{sec:stabilization}) is the
\define{fundamental $n$-group of $X$}, $\Pi_n(X,x)$,
with corresponding delooping $\mathrm{B}\Pi_n(X,x) = \trunc{n}{\BAut x}$.

Double loop spaces are more well-behaved than mere loop
spaces. For example, they are commutative up to homotopy
by the Eckmann-Hilton argument~\cite[Theorem~2.1.6]{hottbook}.
Triple loop spaces are even better behaved than double loop spaces, and so on.

\begin{defn}
A type $G$ is said to be \define{$k$-tuply groupal} if it comes equipped with a \define{$k$-fold delooping}, i.e.~ a pointed $k$-connected
$B^kG : \UU_\pt^{\ge k}$ and an equivalence $G \eqvsym \loopspacesym^kB^kG$.

Mixing the two directions, we also define
\begin{align*}
  (n,k)\GType
  &\defeq  \sm{G : \UU_\pt^{\le n}}{B^kG : \UU_\pt^{\ge k}}
    G \eqvsym_\pt \loopspacesym^kB^kG \\
  & \phantom{:}\eqvsym \UU_\pt^{\ge k,\le n+k}
\end{align*}
for the type of \define{$k$-tuply groupal $n$-groupoids}\footnote{This
  is called $n\UU_k$ in \cite{BaezDolan1998}, but here we give equal
  billing to $n$ and $k$,
  and we add the ``G'' to indicate group-structure.}.
We allow taking $n=\infty$, in which case the truncation requirement
is simply dropped.
\end{defn}

Note that $n\mathsf{Grp} = (n-1,1)\GType$. This shift in indexing is slightly
annoying, but we keep it to stay consistent with the literature.

Note that for each $k\geq 0$ there is a forgetful map
\begin{equation*}
(n,k+1)\GType \to (n,k)\GType,
\end{equation*}
given by $B^{k+1}G\mapsto \loopspacesym B^{k+1}G$, defining a sequence
\begin{equation*}
\begin{tikzcd}
\cdots \arrow[r] & (n,2)\GType \arrow[r] & (n,1)\GType \arrow[r] & (n,0)\GType.
\end{tikzcd}
\end{equation*}
Thus we define $(n,\infty)\GType$ as the limit of this sequence:
\begin{align*}
(n,\infty)\GType & \defeq  \lim_k{}(n,k)\GType \\
&\phantom{:}\eqvsym \sm{B^{\blank}G : \prd{k : \bN}\UU_\pt^{\ge k,\le n+k}}\prd{k : \bN} B^kG \eqvsym_\pt \loopspacesym B^{k+1}G.
\end{align*}
In \cref{sec:stabilization} we prove the stabilization theorem
(\cref{thm:stabilization}), from which it follows that
$(n,\infty)\GType=(n,k)\GType$ for $k\geq n+2$.

The type $(\infty,\infty)\GType$ is the type of \define{stably groupal $\infty$-groups},
also known as \define{connective spectra}. If we also relax the
connectivity requirement, we get the type of all spectra, and we can
think of a spectrum as a kind of $\infty$-groupoid with $k$-morphisms
for all $k\in\mathbb{Z}$.

The double hierarchy of higher groups is summarized in~\cref{tab:periodic}.
We shall prove the correctness of the $n=0$ column in~\cref{sec:n=0}.
\begin{table}
  \caption{\label{tab:periodic}Periodic table of $k$-tuply groupal $n$-groupoids.}
  \centering
  \begin{tabular}{clllll} \toprule
    $k\setminus n$ & $0$ & $1$ & $2$ & $\cdots$ & $\infty$ \\
    \midrule
    $0$ & pointed set & pointed groupoid & pointed $2$-groupoid & $\cdots$ & pointed $\infty$-groupoid \\
    $1$ & group & $2$-group & $3$-group & $\cdots$ & $\infty$-group \\
    $2$ & abelian group & braided $2$-group & braided $3$-group & $\cdots$ & braided $\infty$-group \\
    $3$ & \ditto & symmetric $2$-group & sylleptic $3$-group & $\cdots$ & sylleptic $\infty$-group \\
    $4$ & \ditto & \ditto & symmetric $3$-group & $\cdots$ & ?? $\infty$-group \\
    $\vdots$ & \mbox{}\quad$\vdots$ & \mbox{}\quad$\vdots$ & \mbox{}\quad$\vdots$ & $\ddots$ & \mbox{}\quad$\vdots$ \\
    $\loopspacesym$ & \ditto & \ditto & \ditto & $\cdots$ & connective spectrum \\
    \bottomrule
  \end{tabular}
\end{table}

A homomorphism between higher groups is any
function that can be suitably delooped.

\begin{defn}
For $G,H : (n,k)\GType$, we define
\begin{align*}
\hom_{(n,k)}(G,H) & \defeq  
\sm{h: G \to_\pt H}{B^k h: B^kG \to_\pt B^kH} \loopspacesym^k(B^k h) \sim_\pt h \\
&\phantom{:}\eqvsym (B^k h: B^kG \to_\pt B^kH).
\end{align*}
For (connective) spectra we need
pointed maps between all the deloopings and pointed homotopies showing
they cohere.
\end{defn}

Note that if $h,k : G \to H$ are homomorphisms between set-level
groups, then $h$ and $k$ are \define{conjugate} if $Bh, Bk : BG \to_\pt BH$ are
\define{freely} homotopic (i.e., equal as maps $BG \to BH$).

Also observe that 
\begin{align*}
\pi_j(B^kG \to_\pt B^kH) & \eqvsym \trunc{0}{B^kG \to_\pt \loopspacesym^jB^kH} \\
& \eqvsym \trunc{0}{\Sigma^jB^kG \to_\pt B^kH} \\
& \eqvsym 0
\end{align*}
for $j>n$, which suggests that $\hom_{(n,k)}(G,H)$ is $n$-truncated. To prove this, we deviate slightly from the approach in \cite{BuchholtzDoornRijke} and use the following intermediate result.

\subsection{The stabilization theorem for higher groups}
\label{sec:stabilization}

\begin{defn}
The \define{decategorification} $\Decat G$ of a $k$-tuply groupal $(n+1)$-group is defined to be the $k$-tuply groupal $n$-group $\trunc{n-1}{G}$, which has delooping $\trunc{n+k-1}{B^kG}$. Thus, decategorification is an operation
\begin{equation*}
\Decat : (n,k)\GType \to (n-1,k)\GType.
\end{equation*}
The functorial action of $\Decat$ is defined in the expected way. We also define the \define{$\infty$-decategorification} $\iDecat G$ of a $k$-tuply groupal $\infty$-group as the $k$-tuply groupal $n$-group $\trunc{n}{G}$, which has delooping $\trunc{n+k}{B^k G}$. 
\end{defn}

\begin{defn}
The \define{discrete categorification} $\Disc G$ of a $k$-tuply-groupal $(n+1)$-group is defined to be the same $\infty$-group $G$, now considered as a $k$-tuply groupal $(n+2)$-group. Thus, the discrete categorification is an operation
\begin{equation*}
\Disc : (n,k)\GType \to (n+1,k)\GType.
\end{equation*}
Similarly, the \define{discrete $\infty$-decategorification} $\iDisc G$ of a $k$-tuply groupal $(n+1)$-group is defined to be the same group, now considered as a $k$-tuply groupal $\infty$-group.
\end{defn}

\begin{rmk}
The decategorification and discrete categorification functors make the $(n+1)$-category $(n,k)\GType$ a reflective sub-$(\infty,1)$-category of $(n+1,k)\GType$. That is, there is an adjunction ${\Decat} \dashv {\Disc}$. These properties are straightforward consequences of the universal property of truncation.
Similarly, we have ${\iDecat} \dashv {\iDisc}$ such that the counit induces an isomorphism ${\iDecat} \circ {\iDisc} = \idfunc$.
\end{rmk}

For the next constructions, we need the following properties.
\begin{defn}
  For $A : \UU_\pt$ we define the \define{$n$-connected cover} of $A$ to be 
  $A{\angled n} \defeq  \fibf{A \to \trunc{n}{A}}$. We have the projection $p_1: A{\angled n} \to_\pt A$.
\end{defn}

\begin{lem} \label{lem:connected-cover-univ}
  The universal property of the $n$-connected cover states the following. For any $n$-connected pointed type $B$, the pointed map
  $$(B \to_\pt A{\angled n}) \to_\pt (B \to_\pt A),$$
  given by postcomposition with $p_1$, is an equivalence.\\
\end{lem}

\begin{proof}
  Given a map $f:B\to_\pt A$, we can form a map $\widetilde f: B \to A{\angled n}$. First note that for $b:B$ the type $\truncunit{fb}_n=_{\trunc{n}{A}}\truncunit{\pt}_n$ is $(n-1)$-truncated and inhabited for $b=\pt$. Since $B$ is $n$-connected, the universal property for connected types shows that we can construct a $qb:\truncunit{fb}_n=\truncunit{\pt}_n$ for all $b$ such that $q_0:qb_0\cdot\mathsf{ap}_{\truncunit{\blank}_n}(f_0)=1$. Then we can define the map $\widetilde f(b)\defeq (fb, qb)$. Now $\widetilde f$ is pointed, because $(f_0,q_0):(fb_0,qb_0)=(a_0,1)$.

  Now we show that this is indeed an inverse to the given map. On the one hand, we need to show that if $f: B \to_\pt A$, then $\proj 1 \circ \widetilde f=f$. The underlying functions are equal because they both send $b$ to $f(b)$. They respect points in the same way, because
  $\mathsf{ap}{p_1}(\widetilde f_0)=f_0$. The proof that the other composite is the identity follows from a computation using fibers and connectivity, which we omit here, but can be found in the formalization.
\end{proof}

The next reflective sub-$(\infty,1)$-category is formed by looping and delooping.
\begin{description}
\item[looping] $\loopspacesym : (n,k)\GType \to (n-1,k+1)\GType$ \\
  $\angled{G,B^kG} \mapsto \angled{\loopspacesym G,B^kG{\angled k}}$
\item[delooping] $\B : (n,k)\GType \to (n+1,k-1)\GType$ \\
  $\angled{G,B^kG} \mapsto \angled{\loopspacesym^{k-1}B^kG,B^kG}$
\end{description}
We have ${\B} \dashv {\loopspacesym}$, which follows from Lemma \ref{lem:connected-cover-univ} %note: autoref writes "Theorem"
and $\loopspacesym\circ{\B} = \idfunc$, which follows from the fact that $A{\angled n}=A$ if $A$ is $n$-connected.

The last adjoint pair of functors is given by stabilization and forgetting. This does not form a reflective sub-$(\infty,1)$-category.
\begin{description}
\item[forgetting] $F : (n,k)\GType \to (n,k-1)\GType$ \\
  $\angled{G,B^kG} \mapsto \angled{G,\loopspacesym B^kG}$
\item[stabilization] $S : (n,k)\GType \to (n,k+1)\GType$ \\
  $\angled{G,B^kG} \mapsto \angled{SG,\trunc{n+k+1}{\susp B^kG}}$,\\
  where $SG = \trunc{n}{\loopspacesym^{k+1}\susp B^kG}$
\end{description}
We have the adjunction ${S} \dashv {F}$ which follows from the suspension-loop adjunction $\Sigma\dashv\loopspacesym$ on pointed types.

The next main goal in this section is the stabilization theorem,
stating that the ditto marks in~\cref{tab:periodic} are justified.

The following corollary is almost \cite[Lemma~8.6.2]{hottbook}, but
proving this in Book HoTT is a bit tricky. See the
formalization for details.
\begin{lem}[Wedge connectivity]
  \label{lem:wedge-connectivity}
  If $A : \UU_\pt$ is $n$-connected and $B: \UU_\pt$ is
  $m$-connected, then the map $A \vee B \to A \times B$ is
  $(n+m)$-connected.
\end{lem}

Let us mention that there is an alternative way to prove the wedge
connectivity lemma: Recall that if $A$ is $n$-connected and $B$ is
$m$-connected, then $A \ast B$ is
$(n+m+2)$-connected~\cite[Theorem~6.8]{joinconstruction}. Hence the
wedge connectivity lemma is also a direct consequence of the following lemma.
\begin{lem}
Let $A$ and $B$ be pointed types.
The fiber of the wedge inclusion $A\vee B\to A\times B$ is equivalent to
$\loopspacesym{A}\ast\loopspacesym{B}$. 
\end{lem}
\begin{proof}
Note that the fiber of $A\to A\times B$ is $\loopspacesym B$, the fiber of $B\to A\times B$ is $\loopspacesym A$, and of course the fiber of $1\to A\times B$ is $\loopspacesym A\times \loopspacesym B$. We get a commuting cube
\begin{equation*}
\begin{tikzcd}
& \loopspacesym A\times \loopspacesym B \arrow[dl] \arrow[d] \arrow[dr] \\
\loopspacesym B \arrow[d] & 1 \arrow[dl] \arrow[dr] & \loopspacesym A \arrow[dl,crossing over] \arrow[d] \\
A \arrow[dr] & 1 \arrow[d] \arrow[from=ul,crossing over] & B \arrow[dl] \\
& A\times B
\end{tikzcd}
\end{equation*}
in which the vertical squares are pullback squares. 

By the descent theorem for pushouts it now follows that $\loopspacesym A\ast \loopspacesym B$ is the fiber of the wedge inclusion.
\end{proof}

The second main tool we need for the stabilization theorem is:
\begin{thm}[Freudenthal]
  If $A : \UU_\pt^{>n}$ with $n\ge 0$, then the map
  $A \to \loopspacesym\susp A$ is $2n$-connected.
\end{thm}
This is \cite[Theorem~8.6.4]{hottbook}.

The final building block we need is:
\begin{lem}
  There is a pullback square
  \[
    \begin{tikzcd}
      \susp\loopspacesym A \ar[d,"\varepsilon_A"']\ar[r] & A \vee A \ar[d] \\
      A \ar[r,"\Delta"'] & A \times A
    \end{tikzcd}
  \]
  for any $A : \UU_\pt$.
\end{lem}

\begin{proof}
Note that the pullback of $\Delta:A\to A\times A$ along either inclusion $A\to A\times A$ is contractible. So we have a cube
\begin{equation*}
\begin{tikzcd}
& \loopspacesym A \arrow[dl] \arrow[d] \arrow[dr] \\
1 \arrow[d] & 1 \arrow[dl] \arrow[dr] & 1 \arrow[dl,crossing over] \arrow[d] \\
A \arrow[dr] & A \arrow[d,"\Delta"] \arrow[from=ul,crossing over] & A \arrow[dl] \\
& A\times A
\end{tikzcd}
\end{equation*}
in which the vertical squares are all pullback squares. Therefore, if we pull back along the wedge inclusion, we obtain by the descent theorem for pushouts that the square in the statement is indeed a pullback square.
\end{proof}

\begin{thm}[Stabilization]
  \label{thm:stabilization}
  If $k\ge n+2$, then $S : (n,k)\GType \to (n,k+1)\GType$ is an
  equivalence, and any $G : (n,k)\GType$ is an infinite loop space.
\end{thm}
\begin{proof}
  We show that $F\circ S=\idfunc=S\circ F : (n,k)\GType \to (n,k)\GType$
  whenever $k\ge n+2$.

  For the first, the unit map of the adjunction factors as
  \[
    B^kG \to \loopspacesym\susp B^kG \to \loopspacesym\trunc{n+k+1}{\susp B^kG}
  \]
  where the first map is $2k-2$-connected by Freudenthal, and the
  second map is $n+k$-connected. Since the domain is $n+k$-truncated,
  the composite is an equivalence whenever $2k-2 \ge n+k$.

  For the second, the counit map of the adjunction factors as
  \[
    \trunc{n+k}{\susp\loopspacesym B^kG} \to \trunc{n+k}{B^kG} \to B^kG,
  \]
  where the second map is an equivalence. By the two lemmas above, the
  first map is $2k-2$-connected.
\end{proof}
For example, for $G : (0,2)\GType$ an abelian group, we have
$B^nG = K(G,n)$, an Eilenberg-MacLane space.

The adjunction ${S} \dashv {F}$ implies that the free group on a
pointed set $X$ is $\loopspacesym\trunc{1}{\susp X}=\pi_1(\susp X)$.  If $X$
has decidable equality, $\susp X$ is already $1$-truncated. It is an
open problem whether this is true in general.

Also, the abelianization of a set-level group $G : 1\mathsf{Grp}$ is
$\pi_2(\susp BG)$. If $G : (n,k)\GType$ is in the stable range ($k \ge
n+2$), then $SFG=G$.

\subsection{Eilenberg-Mac Lane spaces}

\begin{exercises}
\exercise Show that if $X$ is $m$-connected and $f:X\to Y$ is $n$-connected, then the map
\begin{equation*}
X \to \fib{m_f}{\ast}
\end{equation*}
where $m_f:Y\to M_f$ is the inclusion of $Y$ into the cofiber of $f$, is $(m+n)$-connected.
\exercise Suppose that $X$ is a connected type, and let $f:X\to Y$ be a map.
Show that the following are equivalent:
\begin{enumerate}
\item $f$ is $n$-connected.
\item The mapping cone of $f$ is $(n+1)$-connected.
\end{enumerate}
\exercise Apply the Blakers-Massey theorem to the defining pushout square of the smash product to show that if $A$ and $B$ are $m$- and $n$-connected respectively, then there is a $(m+n+\min(m,n)+2)$-connected map
\begin{equation*}
\join{\loopspace{A}}{\loopspace{B}}\to \loopspace{A \wedge B}.
\end{equation*}
\exercise Show that the square
\begin{equation*}
\begin{tikzcd}
\unit \arrow[r] \arrow[d] & \bool \arrow[d] \\
X \arrow[r] & X+\unit
\end{tikzcd}
\end{equation*}
is both a pullback and a pushout. Conclude that the result of the Blakers-Massey theorem is not always sharp.
\item Show that for every pointed type $X$, and any $n:\N$, there is a fiber sequence
  \begin{equation*}
    K(\pi_{n+1}(X),n+1)\hookrightarrow \trunc{n+1}{X}\twoheadrightarrow \trunc{n}{X}.
  \end{equation*}
\end{exercises}
