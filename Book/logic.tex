\chapter{Univalent logic}


\section{Propositional truncation}

Using the same idea of higher inductive types, we can also turn every type into a mere proposition via an operation we call \emph{propositional truncation}. 

\begin{defn}
Let $A:\UU$ be a type. We define the \define{propositional truncation} $\brck{A}:\UU$ as a higher inductive type with constructors
\begin{align*}
\eta & : A\to \brck{A} \\
\mu & : \prd{x,y:\brck{A}} x=y.
\end{align*}
The induction principle of $\brck{A}$ asserts that for any type family $P:\brck{A}\to\type$, if we have
\begin{align*}
f & : \prd{x:A}P(\eta(x)) \\
g & : \prd{x,y:\brck{A}}{p:P(x)}{q:P(y)} \trans{\mu(x,y)}{p}=q,
\end{align*}
then we obtain a section $\rec{\brck{\blank}}(f,g):\prd{x:\brck{A}}P(x)$ satisfying
\begin{align*}
\rec{\brck{\blank}}(f,g,\eta(x)) & = f(x).
\end{align*}
\end{defn}

\begin{rmk}
We will not need a computation rule corresponding to the path constructor $\mu$.
\end{rmk}

\begin{lem}
For any $A:\UU$, the type $\brck{A}$ is a proposition.
\end{lem}

\begin{proof}
The path constructor $\mu:\prd{x,y:\brck{A}}x=y$ directly provides a proof that $\brck{A}$ is a proposition.
\end{proof}

The following theorem asserts that for any map $f:A\to P$ into a proposition $P$, there is a unique map $g:\brck{A}\to P$ such that $g\circ\eta=f$, as indicated in the following diagram
\begin{equation*}
\begin{tikzcd}
A \arrow[dr,"\forall"] \arrow[d,swap,"\eta"] \\
\brck{A} \arrow[r,densely dotted,swap,"\exists!"] & P
\end{tikzcd}
\end{equation*}
We call this property the \define{universal property of propositional truncation}. 

\begin{thm}
Let $A:\UU$ be a type. Then for any proposition $P$, the map
\begin{equation*}
\lam{g} g\circ \eta : (\brck{A}\to P)\to (A\to P)
\end{equation*}
is an equivalence.
\end{thm}

\begin{proof}

\end{proof}

\section{Surjective maps}

\begin{defn}
We say that a type $A$ is \define{inhabited} if there is a term of type $\brck{A}$. 
\end{defn}

\begin{defn}
A function $f:A\to B$ is said to be surjective if its fibers are inhabited. 
More explicitly, we define
\begin{equation*}
\mathsf{is\usc{}surj}(f)\defeq \prd{b:B}\brck{\fib{f}{b}}. 
\end{equation*}
\end{defn}

\section{First order logic in type theory}
\begin{table}
\caption{Logic in type theory}
\begin{center}
\begin{tabular}{ll}
\toprule
\emph{Logical connective} & \emph{Interpretation in HoTT} \\
\midrule
$\top$ & $\unit$ \\
$\bot$ & $\emptyt$ \\
$P\land Q$ & $P\times Q$ \\
$P\lor Q$ & $\brck{P+Q}$ \\
$P\to Q$ & $P\to Q$ \\
$P\leftrightarrow Q$ & $P=Q$ \\
$\neg P$ & $P\to\emptyt$ \\
$\forall x.P(x)$ & $\prd{x:A}P(x)$ \\
$\exists x.P(x)$ & $\brck{\sm{x:A}P(x)}$ \\
$\exists! x.P(x)$ & $\iscontr(\sm{x:A}P(x))$ \\
\bottomrule
\end{tabular}
\end{center}
\end{table}
\begin{enumerate}
\item First order logic in type theory. Stress difference between $\exists$ and $\Sigma$.
\item Propositional extensionality
\end{enumerate}

\begin{exercises}
\item Show that
\begin{equation*}
\eqv{\brck{A}}{\prd{P:\prop}(A\to P)\to P}.
\end{equation*}
\item For any $B:A\to\UU$, construct an equivalence
\begin{equation*}
\eqv{\Big(\exists a.\brck{B(a)}\Big)}{\Big(\brck{\sm{a:A}B(a)}\Big)}
\end{equation*}
\item \label{also}(Mart\'in Escard\'o) For any two propositions $P$ and $Q$, define
\begin{equation*}
P\boxplus Q \defeq ((P\to Q)\to Q)\times ((Q\to P)\to P).
\end{equation*}
\begin{subexenum}
\item Show that $P\lor Q\to P\boxplus Q$ and $P\boxplus Q\to\neg(\neg P\land \neg Q)$.
\end{subexenum}
\item Show that for any mere proposition $Q$, and any type $X$, the following are equivalent:
\begin{enumerate}
\item The map $(Q\to X)\to(\emptyt\to X)$ is an equivalence.
\item The type $X^Q$ is contractible.
\item $Q\to\iscontr(X)$.
\end{enumerate}
\item \label{ex:brck_comp} Formulate the computation rule corresponding to the path constructor $\mu$. That is, compute the type of $\apd{\rec{\brck{\blank}}(f,g)}{\mu(x,y)}$, and find a canonical element in it.
\end{exercises}
