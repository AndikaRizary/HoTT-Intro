\section{Higher group theory}

\subsection{The category of pointed connected \texorpdfstring{$1$}{1}-types}

\begin{prp}\label{prp:istrunc-precomp-isconn}
  Consider a $k$-connected map $f:X\to Y$, and a family $P$ of $(k+n)$-truncated types over $Y$, where $n\geq 0$. Then the precomposition map
  \begin{equation*}
    \blank\circ f : \Big(\prd{y:Y}P(y)\Big)\to\Big(\prd{x:X}P(f(x))\Big)
  \end{equation*}
  is $(n-2)$-truncated. 
\end{prp}

\begin{prp}
Consider a pointed $(k+1)$-connected type $X$, and a family $Y:X\to\UU^{\le n+k}$ of $(n+k)$-truncated types over $X$. Then the map
\begin{equation*}
\evpt : \Big(\prd{x:X}Y(x)\Big) \to Y(\pt)
\end{equation*}
induced by the point inclusion $\unit\to X$, is an $(n-2)$-truncated map.
\end{prp}

\begin{proof}
Note that we have a commuting triangle
\begin{equation*}
\begin{tikzcd}[column sep=-1em]
& \Big(\prd{x:X}Y(x)\Big) \arrow[dl,swap,"\blank\circ \mathsf{const}_\pt"] \arrow[dr,"\evpt"] & \phantom{\Big(\prd{t:\unit} Y(\pt)\Big)} \\
\Big(\prd{t:\unit} Y(\pt)\Big) \arrow[rr,swap,"\evpt","\eqvsym"'] & & Y(\pt),
\end{tikzcd}
\end{equation*}
so the map on the left is an $(n-2)$-truncated map if and only if the map on the right is. For the map on the left, the claim follows immediately from \cref{prp:istrunc-precomp-isconn}, since the point inclusion $\mathsf{const}_\pt:\unit\to X$ is a $k$-connected map by \cref{cor:ptd_connected}.
\end{proof}

\begin{defn}
  If $X : \UU_\pt$ and $Y : X \to \UU_\pt$, then we introduce the
  type of \define{pointed sections},
\begin{equation*}
\textstyle{\prod_{(x:X)}^\ast Y(x)} \defeq \sm{s:\prd{x:X}Y(x)}s(\pt)=\pt
\end{equation*}
  This type is itself pointed by the trivial section $\lam{x}\pt$.
\end{defn}

\begin{cor}
Consider a pointed $k$-connected type $X$, and a family $Y:X\to\UU_\pt^{\le n+k}$ of pointed $(n+k)$-truncated types over $X$. Then the type $\prod_{(x:X)}^\ast Y(x)$ is $(n-1)$-truncated.
\end{cor}

\begin{proof}
Note that we have a pullback square
\begin{equation*}
\begin{tikzcd}
\textstyle{\prod_{(x:X)}^\ast Y(x)} \arrow[r] \arrow[d] & \unit \arrow[d] \\
\prd{x:X}Y(x) \arrow[r,swap,"\evpt"] & Y(\ast),
\end{tikzcd}
\end{equation*}
so the claim follows from the fact that $\evpt$ is an $(n-1)$-truncated map.
\end{proof}

\begin{thm}
  The type $\hom_{(n,k)}(G,H)$ is an $n$-type for any $G,H:(n,k)\GType$.
\end{thm}

\begin{proof}
  If $X$ is $(k-1)$-connected, and $Y$ is $(n+k)$-truncated, then the type of pointed maps $X \to_\pt Y$ is $n$-truncated.
\end{proof}

\begin{cor}
  The type $(n,k)\GType$ is $(n+1)$-truncated.
\end{cor}
\begin{proof}
  This follows immediately from the preceding corollary, as the type
  of equivalences $G \eqvsym H$ is a subtype of the homomorphisms from
  $G$ to $H$.
\end{proof}

If $k\ge n+2$ (so we're in the stable range), then $\hom_{(n,k)}(G,H)$
becomes a stably groupal $n$-groupoid. This generalizes the
fact that the homomorphisms between abelian groups form an abelian
group.

\begin{cor}
The automorphism group $\Aut G$ of a higher group $G:(n,k)\GType$ is a $1$-groupal $(n+1)$-group, equivalent to the automorphism group of the pointed type $B^kG$.
\end{cor}

\begin{prp}
  For any two pointed $n$-connected $(n+k+1)$-truncated types $X$ and $Y$, the type of pointed maps
  \begin{equation*}
    X\to_\ast Y
  \end{equation*}
  is $k$-truncated. 
\end{prp}

\begin{cor}
  For any two pointed $n$-connected $(n+1)$-truncated types $X$ and $Y$, the type of pointed maps
  \begin{equation*}
    X\to_\ast Y
  \end{equation*}
  is a set.
\end{cor}

\begin{thm}
  The pre-category of $n$-connected $(n+1)$-truncated types in a universe $\UU$ is Rezk complete.
\end{thm}

\subsection{Equivalences of categories}

\begin{defn}
  A \define{functor} is...
\end{defn}

\begin{defn}
  A functor $F:\mathcal{C}\to\mathcal{D}$ is an equivalence if ...
\end{defn}

\subsection{The equivalence of groups and pointed connected \texorpdfstring{$1$}{1}-types}

\begin{thm}
  The loop space functor
  \begin{equation*}
    \pcttype{0}{1} \to \Group
  \end{equation*}
  is an equivalence of categories.
\end{thm}
