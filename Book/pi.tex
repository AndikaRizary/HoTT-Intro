\section{Dependent function types}
\label{ch:pi}

A fundamental concept in dependent type theory is that of a dependent function. A dependent function is a function of which the type of the output may depend on the input. They are a generalization of ordinary functions, because an ordinary function $f:A\to B$ is a function of which the output $f(x)$ has type $B$ regardless of the value of $x$.

\subsection{Dependent function types}
Consider a section $b$ of a family $B$ over $A$ in context $\Gamma$, i.e.,
\begin{equation*}
  \Gamma,x:A\vdash b(x):B(x).
\end{equation*}
From one point of view, such a section $b$ is an operation, or a program, that takes as input $x:A$ and produces a term $b(x):B(x)$. From a more mathematical point of view we see $b$ as a choice of an element of each $B(x)$. In other words, we may see $b$ as a function that takes $x:A$ to $b(x):B(x)$. Note that the type $B(x)$ of the output is dependent on $x:A$. In this section we postulate rules for the \emph{type} of all such dependent functions: whenever $B$ is a family over $A$ in context $\Gamma$, there is a type
\begin{equation*}
  \prd{x:A}B(x)
\end{equation*}
in context $\Gamma$, consisting of all the dependent functions of which the output at $x:A$ has type $B(x)$. There are four principal rules for $\Pi$-types:
\begin{enumerate}
\item The formation rule, which tells us how we may form dependent function types.
\item The introduction rule, which tells us how to introduce new terms of dependent function types.
\item The elimination rule, which tells us how to use arbitrary terms of dependent function types.
\item The computation rules, which tell us how the introduction and elimination rules interact. These computation rules guarantee that every term of a dependent function type behaves as expected: as a dependent function.
\end{enumerate}
In the cases of the formation rule, the introduction rule, and the elimination rule, we will also provide conversion rules that ensure that all the constructions respect judgmental equality.

\subsubsection{The $\Pi$-formation rule}
\define{Dependent function types}\index{dependent function type}\index{pi-type@{$\Pi$-type}} are formed by the following \define{$\Pi$-formation rule}\index{rule!pi-formation@{$\Pi$-formation}}:
\begin{prooftree}
\AxiomC{$\Gamma,x:A\vdash B(x)~\textrm{type}$}
\RightLabel{$\Pi$.}
\UnaryInfC{$\Gamma\vdash \prd{x:A}B(x)~\mathrm{type}$}
\end{prooftree}
With the following conversion rule we postulate that formation of dependent function types respects judgmental equality:
\begin{prooftree}
\AxiomC{$\Gamma\vdash A\jdeq A'~\mathrm{type}$}
\AxiomC{$\Gamma,x:A\vdash B(x)\jdeq B'(x)~\textrm{type}$}
\RightLabel{$\Pi$-eq.}
\BinaryInfC{$\Gamma\vdash \prd{x:A}B(x)\jdeq\prd{x:A'}B'(x)~\mathrm{type}$}
\end{prooftree}
Furthermore, when $x'$ is a fresh variable, i.e., which does not occur in the context $\Gamma,x:A$, we also postulate that
\begin{prooftree}
\AxiomC{$\Gamma,x:A\vdash B(x)~\textrm{type}$}
\RightLabel{$\Pi$-$x'/x$.}
\UnaryInfC{$\Gamma\vdash \prd{x:A}B(x)\jdeq \prd{x':A}B(x')~\mathrm{type}$}
\end{prooftree}

\subsubsection{The $\Pi$-introduction rule}
The introduction rule for dependent function types is also called the $\lambda$-abstraction rule. Recall that dependent functions are formed from terms $b(x)$ of type $B(x)$ in context $\Gamma,x:A$. Therefore \define{$\lambda$-abstraction rule}\index{lambda-abstraction@{$\lambda$-abstraction}}\index{rule!lambda-abstraction@{$\lambda$-abstraction}} is as follows:
\begin{prooftree}
  \AxiomC{$\Gamma,x:A \vdash b(x) : B(x)$}
  \RightLabel{$\lambda$}
  \UnaryInfC{$\Gamma\vdash \lam{x}b(x) : \prd{x:A}B(x)$}
\end{prooftree}

Just like ordinary mathematicians, we will sometimes write $x\mapsto f(x)$ for a function $f$. The map $n\mapsto n^2$ is an example. The $\lambda$-abstraction is also required to respect judgmental equality. Therefore we postulate that
\begin{prooftree}
  \AxiomC{$\Gamma,x:A \vdash b(x)\jdeq b'(x) : B(x)$}
  \RightLabel{$\lambda$-eq.}
  \UnaryInfC{$\Gamma\vdash \lam{x}b(x)\jdeq \lam{x}b'(x) : \prd{x:A}B(x)$}
\end{prooftree}

\subsubsection{The $\Pi$-elimination rule}

The elimination rule for dependent function types provides us with a way to \emph{use} dependent functions. The way to use a dependent function is to apply it to an argument of the domain type. The $\Pi$-elimination rule is therefore also called the \define{evaluation rule}\index{evaluation}\index{rule!evaluation}. It asserts that given a dependent function $f:\prd{x:A}B(x)$ in context $\Gamma$ we obtain a term $f(x)$ of type $B(x)$ in context $\Gamma,x:A$. More formally:
\begin{prooftree}
\AxiomC{$\Gamma\vdash f:\prd{x:A}B(x)$}
\RightLabel{$ev$}
\UnaryInfC{$\Gamma,x:A\vdash f(x) : B(x)$}
\end{prooftree}
Again we require that evaluation respects judgmental equality:
\begin{prooftree}
  \AxiomC{$\Gamma\vdash f\jdeq f':\prd{x:A}B(x)$}
  \UnaryInfC{$\Gamma,x:A\vdash f(x)\jdeq f'(x):B(x)$}
\end{prooftree}

\subsubsection{The $\Pi$-computation rules}
The computation rules for dependent function types postulate that $\lambda$-abstraction rule and the evaluation rule are mutual inverses. Thus we have two computation rules.

First we postulate the \define{$\beta$-rule}\index{beta-rule@{$\beta$-rule}}\index{rule!beta-rule@{$\beta$-rule}}
\begin{prooftree}
\AxiomC{$\Gamma,x:A \vdash b(x) : B(x)$}
\RightLabel{$\beta$.}
\UnaryInfC{$\Gamma,x:A \vdash (\lambda y.b(y))(x)\jdeq b(x) : B(x)$}
\end{prooftree}
Second, we postulate the \define{$\eta$-rule}\index{eta-rule@{$\eta$-rule}}\index{rule!eta-rule@{$\eta$-rule}}
\begin{prooftree}
\AxiomC{$\Gamma\vdash f:\prd{x:A}B(x)$}
\RightLabel{$\eta$.}
\UnaryInfC{$\Gamma \vdash \lam{x}f(x) \jdeq f : \prd{x:A}B(x)$}
\end{prooftree}
This completes the specification of dependent function types.

\subsection{Ordinary function types}
In the case where both $A$ and $B$ are types in context $\Gamma$, we may first weaken $B$ by $A$, and then apply the formation rule for the dependent function type:
\begin{prooftree}
\AxiomC{$\Gamma\vdash A~\textrm{type}$}
\AxiomC{$\Gamma\vdash B~\textrm{type}$}
\BinaryInfC{$\Gamma,x:A\vdash B~\textrm{type}$}
\UnaryInfC{$\Gamma\vdash \prd{x:A}B~\textrm{type}$}
\end{prooftree}
The result is the type of functions that take an argument of type $A$, and return a term of type $B$. In other words, terms of the type $\prd{x:A}B$ are \emph{ordinary} functions from $A$ to $B$. We write $A\to B$ for the \define{type of functions}\index{function type} from $A$ to $B$. Sometimes we will also write $B^A$ for the type $A\to B$.

We give a brief summary of the rules specifying ordinary function types, omitting the rules that the asserted operations respect judgmental equality. All of these rules can be derived from the corresponding rules for $\Pi$-types.
\begin{prooftree}
\AxiomC{$\Gamma\vdash A~\textrm{type}$}
\AxiomC{$\Gamma\vdash B~\textrm{type}$}
\RightLabel{$\to$\index{arrow-formation@{$\to$-formation}}\index{rule!arrow-formation@{$\to$-formation}}}
\BinaryInfC{$\Gamma\vdash A\to B~\textrm{type}$}
\end{prooftree}%
\begin{center}
\begin{minipage}{.45\textwidth}
\begin{prooftree}
\AxiomC{$\Gamma\vdash B~\textrm{type}$}
\AxiomC{$\Gamma,x:A\vdash b(x):B$}
\RightLabel{$\lambda$\index{lambda-abstraction@{$\lambda$-abstraction}}\index{rule!lambda-abstraction@{$\lambda$-abstraction}}}
\BinaryInfC{$\Gamma\vdash \lam{x}b(x):A\to B$}
\end{prooftree}%
\end{minipage}
\begin{minipage}{.45\textwidth}
\begin{prooftree}
\AxiomC{$\Gamma\vdash f:A\to B$}
\RightLabel{$ev$\index{rule!evaluation}\index{evaluation}}
\UnaryInfC{$\Gamma,x:A\vdash f(x):B$}
\end{prooftree}%
\end{minipage}
\end{center}
\begin{center}
\begin{minipage}{.45\textwidth}
\begin{prooftree}
\AxiomC{$\Gamma\vdash B~\textrm{type}$}
\AxiomC{$\Gamma,x:A\vdash b(x):B$}
\RightLabel{$\beta$\index{rule!beta-rule@{$\beta$-rule}}\index{beta-rule@{$\beta$-rule}}}
\BinaryInfC{$\Gamma,x:A\vdash(\lam{y}b(y))(x)\jdeq b(x):B$}
\end{prooftree}%
\end{minipage}
\begin{minipage}{.45\textwidth}
\begin{prooftree}
\AxiomC{$\Gamma\vdash f:A\to B$}
\RightLabel{$\eta$\index{rule!eta-rule@{$\eta$-rule}}\index{eta-rule@{$\eta$-rule}}}
\UnaryInfC{$\Gamma\vdash\lam{x} f(x)\jdeq f:A\to B$}
\end{prooftree}
\end{minipage}
\end{center}

\begin{comment}
\begin{rmk}
Similar to \cref{rmk:ev_var}, we can derive
\begin{prooftree}
\AxiomC{$\Gamma\vdash A~\mathrm{type}$}
\AxiomC{$\Gamma\vdash B~\mathrm{type}$}
\BinaryInfC{$\Gamma,f:B^A,x:A\vdash f(x):B$}
\end{prooftree}
\end{rmk}
\end{comment}

\subsection{The identity function, composition, and their laws}
\begin{defn}
For any type $A$ in context $\Gamma$, we define the \define{identity function}\index{identity function|textbf} $\idfunc[A]:A\to A$ using the variable rule\index{variable rule}\index{rule!variable rule}:
\begin{prooftree}
\AxiomC{$\Gamma\vdash A~\textrm{type}$}
\UnaryInfC{$\Gamma,x:A\vdash x:A$}
\UnaryInfC{$\Gamma\vdash \idfunc[A]\defeq \lam{x}x:A\to A$}
\end{prooftree}
\end{defn}

A judgment of the form $\Gamma\vdash a\defeq b:A$ should be read as "$b$ is a well-defined term of type $A$ in context $\Gamma$, and we will refer to it as $a$".

\begin{defn}
For any three types $A$, $B$, and $C$ in context $\Gamma$, there is a \define{composition}\index{composition!of functions|textbf} operation
\begin{equation*}
\mathsf{comp}:(B\to C)\to ((A\to B)\to (A\to C)),
\end{equation*}
i.e., we can derive
\begin{prooftree}
\AxiomC{$\Gamma\vdash A~\textrm{type}$}
\AxiomC{$\Gamma\vdash B~\textrm{type}$}
\AxiomC{$\Gamma\vdash C~\textrm{type}$}
\TrinaryInfC{$\Gamma\vdash\mathsf{comp}:(B\to C)\to ((A\to B)\to (A\to C))$}
\end{prooftree}
We will write $g\circ f$ for $\mathsf{comp}(g,f)$.
\end{defn}

\begin{constr}
  The idea of the definition is to define $\mathsf{comp}(g,f)$ to be the function $\lam{x}g(f(x))$. The derivation we use to construct $\mathsf{comp}$ is as follows:
  \begin{prooftree}
    \AxiomC{$\Gamma\vdash A~\mathrm{type}$}
    \AxiomC{$\Gamma\vdash B~\mathrm{type}$}
    \BinaryInfC{$\Gamma,f:B^A,x:A\vdash f(x):B$}
    \UnaryInfC{$\Gamma,g:C^B,f:B^A,x:A\vdash f(x):B$}
    \AxiomC{$\Gamma\vdash B~\mathrm{type}$}
    \AxiomC{$\Gamma\vdash C~\mathrm{type}$}
    \BinaryInfC{$\Gamma,g:C^B,y:B\vdash g(y):C$}
    \UnaryInfC{$\Gamma,g:C^B,f:B^A,y:B\vdash g(y):C$}
    \UnaryInfC{$\Gamma,g:C^B,f:B^A,x:A,y:B\vdash g(y):C$}
    \BinaryInfC{$\Gamma,g:C^B,f:B^A,x:A\vdash g(f(x)) : C$}
    \UnaryInfC{$\Gamma,g:C^B,f:B^A\vdash \lam{x}g(f(x)):C^A$}
    \UnaryInfC{$\Gamma,g:B\to C\vdash \lam{f}{x}g(f(x)):B^A\to C^A$}
    \UnaryInfC{$\Gamma\vdash\mathsf{comp}\defeq \lam{g}{f}{x}g(f(x)):C^B\to (B^A\to C^A)$}
  \end{prooftree}
\end{constr}

\begin{lem}
Composition of functions is associative\index{associativity!of function composition}\index{composition!of functions!associativity}, i.e., we can derive
\begin{prooftree}
\AxiomC{$\Gamma\vdash f:A\to B$}
\AxiomC{$\Gamma\vdash g:B\to C$}
\AxiomC{$\Gamma\vdash h:C\to D$}
\TrinaryInfC{$\Gamma \vdash (h\circ g)\circ f\jdeq h\circ(g\circ f):A\to D$}
\end{prooftree}
\end{lem}

\begin{proof}
  The main idea of the proof is that both $((h\circ g)\circ f)(x)$ and $(h\circ (g\circ f))(x)$ evaluate to $h(g(f(x))$, and therefore $(h\circ g)\circ f$ and $h\circ(g\circ f)$ must be judgmentally equal. This idea is made formal in the following derivation:
  \begin{prooftree}
    \AxiomC{$\Gamma\vdash f:A\to B$}
    \UnaryInfC{$\Gamma,x:A\vdash f(x):B$}
    \AxiomC{$\Gamma\vdash g:B\to C$}
    \UnaryInfC{$\Gamma,y:B\vdash g(y):C$}
    \UnaryInfC{$\Gamma,x:A,y:B\vdash g(y):C$}
    \BinaryInfC{$\Gamma,x:A\vdash g(f(x)):C$}
    \AxiomC{$\Gamma\vdash h:C\to D$}
    \UnaryInfC{$\Gamma,z:C\vdash h(z):D$}
    \UnaryInfC{$\Gamma,x:A,z:C\vdash h(z):D$}
    \BinaryInfC{$\Gamma,x:A\vdash h(g(f(x))):D$}
    \UnaryInfC{$\Gamma,x:A\vdash h(g(f(x)))\jdeq h(g(f(x))):D$}
    \UnaryInfC{$\Gamma,x:A\vdash (h\circ g)(f(x))\jdeq h((g\circ f)(x)):D$}
    \UnaryInfC{$\Gamma,x:A\vdash ((h\circ g)\circ f)(x)\jdeq (h\circ (g \circ f))(x):D$}
    \UnaryInfC{$\Gamma\vdash (h\circ g)\circ f\jdeq h\circ(g\circ f):A\to D$.}
  \end{prooftree}
\end{proof}

\begin{lem}\label{lem:fun_unit}
Composition of functions satisfies the left and right unit laws\index{left unit law|see {unit laws}}\index{right unit law|see {unit laws}}\index{unit laws!of function composition}, i.e., we can derive
\begin{prooftree}
\AxiomC{$\Gamma\vdash f:A\to B$}
\UnaryInfC{$\Gamma\vdash \idfunc[B]\circ f\jdeq f:A\to B$}
\end{prooftree}
and
\begin{prooftree}
\AxiomC{$\Gamma\vdash f:A\to B$}
\UnaryInfC{$\Gamma\vdash f\circ\idfunc[A]\jdeq f:A\to B$}
\end{prooftree}
\end{lem}

\begin{proof}
The derivation for the left unit law is
\begin{prooftree}
\AxiomC{$\Gamma\vdash f:A\to B$}
\UnaryInfC{$\Gamma,x:A\vdash f(x):B$}
\AxiomC{$\Gamma\vdash B~\mathrm{type}$}
\UnaryInfC{$\Gamma,y:B\vdash \idfunc[B](y)\jdeq y:B$}
\UnaryInfC{$\Gamma,x:A,y:B\vdash \idfunc[B](y)\jdeq y:B$}
\BinaryInfC{$\Gamma,x:A\vdash \idfunc[B](f(x))\jdeq f(x):B$}
\UnaryInfC{$\Gamma,x:A\vdash (\idfunc[B]\circ f)(x)\jdeq f(x):B$}
\UnaryInfC{$\Gamma\vdash \idfunc[B]\circ f\jdeq f:A\to B$}
\end{prooftree}
The right unit law is left as \cref{ex:fun_right_unit}.
\end{proof}

\begin{exercises}
\item \label{ex:fun_right_unit}Give a derivation for the right unit law of \cref{lem:fun_unit}.\index{unit laws!for function composition}
\item Show that the rule
\begin{prooftree}
\AxiomC{$\Gamma,x:A \vdash b(x) : B(x)$}
\RightLabel{$\lambda$-$x'/x$}
\UnaryInfC{$\Gamma\vdash \lam{x}b(x)\jdeq \lam{x'}b(x') : \prd{x:A}B(x)$}
\end{prooftree}
is admissible for any variable $x'$ that does not occur in the context $\Gamma,x:A$.
\item 
  \begin{subexenum}
  \item Construct the \define{constant function}\index{constant function}\index{function!constant function}
    \begin{prooftree}
      \AxiomC{$\Gamma\vdash A~\textrm{type}$}
      \UnaryInfC{$\Gamma,y:B\vdash \mathsf{const}_y:A\to B$}
    \end{prooftree}
  \item Show that
    \begin{prooftree}
      \AxiomC{$\Gamma\vdash f:A\to B$}
      \UnaryInfC{$\Gamma,z:C\vdash \mathsf{const}_z\circ f\jdeq\mathsf{const}_z : A\to C$}
    \end{prooftree}
  \item Show that
    \begin{prooftree}
      \AxiomC{$\Gamma\vdash A~\textrm{type}$}
      \AxiomC{$\Gamma\vdash g:B\to C$}
      \BinaryInfC{$\Gamma,y:B\vdash g\circ\mathsf{const}_y\jdeq \mathsf{const}_{g(y)}:A\to C$}
    \end{prooftree}
  \end{subexenum}
\item In this exercise we generalize the composition operation of non-dependent function types\index{composition!of dependent functions}:
\begin{subexenum}
\item Define a composition operation for dependent function types
\begin{prooftree}
\AxiomC{$\Gamma\vdash f:\prd{x:A}B(x)$}
\AxiomC{$\Gamma\vdash g:\prd{x:A}{y:B(x)} C(x,y)$}
\BinaryInfC{$\Gamma\vdash g\circ' f:\prd{x:A} C(x,f(x))$}
\end{prooftree}
and show that this operation agrees with ordinary composition when it is specialized to non-dependent function types.
\item Show that composition of dependent functions agrees with ordinary composition of functions:
  \begin{prooftree}
    \AxiomC{$\Gamma\vdash f:A\to B$}
    \AxiomC{$\Gamma\vdash g:B\to C$}
    \BinaryInfC{$\Gamma\vdash (\lam{x}g)\circ' f\jdeq g\circ f:A \to C$}
  \end{prooftree}
\item Show that composition of dependent functions is associative.\index{associativity!of dependent function composition}\index{composition!of dependent functions!associativity}
\item Show that composition of dependent functions satisfies the right unit law\index{unit laws!dependent function composition}:
\begin{prooftree}
\AxiomC{$\Gamma\vdash f:\prd{x:A}B(x)$}
\UnaryInfC{$\Gamma\vdash (\lam{x}f)\circ'\idfunc[A]\jdeq f :\prd{x:A}B(x)$}
\end{prooftree}
\item Show that composition of dependent functions satisfies the left unit law\index{unit laws!dependent function composition}:
\begin{prooftree}
\AxiomC{$\Gamma\vdash f:\prd{x:A}B(x)$}
\UnaryInfC{$\Gamma\vdash (\lam{x}\idfunc[B(x)])\circ' f\jdeq f:\prd{x:A}B(x)$}
\end{prooftree}
\end{subexenum}
\item \label{ex:swap}
\begin{subexenum}
\item Given two types $A$ and $B$ in context $\Gamma$, and a type $C$ in context $\Gamma,x:A,y:B$, define the \define{swap function}\index{function!swap}\index{swap function}
\begin{equation*}
\Gamma\vdash \sigma:\Big(\prd{x:A}{y:B}C(x,y)\Big)\to\Big(\prd{y:B}{x:A}C(x,y)\Big)
\end{equation*}
that swaps the order of the arguments.
\item Show that
\begin{equation*}
\Gamma\vdash \sigma\circ\sigma\jdeq\idfunc:\Big(\prd{x:A}{y:B}C(x,y)\Big)\to \Big(\prd{x:A}{y:B}C(x,y)\Big).
\end{equation*}
\end{subexenum}
\end{exercises}
