\section{Sequential colimits}

\emph{Note: This chapter currently contains only the statements of the definitions and theorems, but no proofs. I hope to make a complete version available soon.}

\subsection{The universal property of sequential colimits}

Type sequences are diagrams of the following form.
\begin{equation*}
\begin{tikzcd}
A_0 \arrow[r,"f_0"] & A_1 \arrow[r,"f_1"] & A_2 \arrow[r,"f_2"] & \cdots.
\end{tikzcd}
\end{equation*}
Their formal specification is as follows.

\begin{defn}
An \define{(increasing) type sequence} $\mathcal{A}$ consists of
\begin{align*}
A & : \N\to\UU \\
f & : \prd{n:\N} A_n\to A_{n+1}. 
\end{align*}
\end{defn}

In this section we will introduce the sequential colimit of a type sequence.
The sequential colimit includes each of the types $A_n$, but we also identify each $x:A_n$ with its value $f_n(x):A_{n+1}$. 
Imagine that the type sequence $A_0\to A_1\to A_2\to\cdots$ defines a big telescope, with $A_0$ sliding into $A_1$, which slides into $A_2$, and so forth.

As usual, the sequential colimit is characterized by its universal property.

\begin{defn}
\begin{enumerate}
\item A \define{(sequential) cocone} on a type sequence $\mathcal{A}$ with vertex $B$ consists of
\begin{align*}
h & : \prd{n:\N} A_n\to B \\
H & : \prd{n:\N} h_n\htpy h_{n+1}\circ f_n.
\end{align*}
We write $\mathsf{cocone}(B)$ for the type of cocones with vertex $B$.
\item Given a cocone $(h,H)$ with vertex $B$ on a type sequence $\mathcal{A}$ we define the map
\begin{equation*}
\mathsf{cocone\usc{}map}(h,H) : (B\to C)\to \mathsf{cocone}(C)
\end{equation*}
given by $f\mapsto (\lam{n}f\circ h_n,\lam{n}{x}\mathsf{ap}_f(H_n(x)))$.
\item We say that a cocone $(h,H)$ with vertex $B$ is \define{colimiting} if $\mathsf{cocone\usc{}map}(h,H)$ is an equivalence for any type $C$.
\end{enumerate}
\end{defn}

\begin{thm}\label{thm:sequential_up}
Consider a cocone $(h,H)$ with vertex $B$ for a type sequence $\mathcal{A}$. The following are equivalent:
\begin{enumerate}
\item The cocone $(h,H)$ is colimiting.
\item The cocone $(h,H)$ is inductive in the sense that for every type family $P:B\to \UU$, the map
\begin{align*}
\Big(\prd{b:B}P(b)\Big)\to {}& \sm{h:\prd{n:\N}{x:A_n}P(h_n(x))}\\ 
& \qquad \prd{n:\N}{x:A_n} \mathsf{tr}_P(H_n(x),h_n(x))={h_{n+1}(f_n(x))}
\end{align*}
given by
\begin{equation*}
s\mapsto (\lam{n}s\circ h_n,\lam{n}{x} \mathsf{apd}_{s}(H_n(x)))
\end{equation*}
has a section.
\item The map in (ii) is an equivalence.
\end{enumerate}
\end{thm}

\subsection{The construction of sequential colimits}

We construct sequential colimits using pushouts.

\begin{defn}
Let $\mathcal{A}\jdeq (A,f)$ be a type sequence. We define the type $A_\infty$ as a pushout
\begin{equation*}
\begin{tikzcd}[column sep=large]
\tilde{A}+\tilde{A} \arrow[r,"{[\idfunc,\sigma_{\mathcal{A}}]}"] \arrow[d,swap,"{[\idfunc,\idfunc]}"] & \tilde{A} \arrow[d,"\inr"] \\
\tilde{A} \arrow[r,swap,"\inl"] & A_\infty.
\end{tikzcd}
\end{equation*}
\end{defn}

\begin{defn}
The type $A_\infty$ comes equipped with a cocone structure consisting of
\begin{align*}
\mathsf{seq\usc{}in} & : \prd{n:\N} A_n\to A_\infty \\
\mathsf{seq\usc{}glue} & : \prd{n:\N}{x:A_n} \mathsf{in}_n(x)=\mathsf{in}_{n+1}(f_n(x)).
\end{align*}
\end{defn}

\begin{constr}
We define
\begin{align*}
\mathsf{seq\usc{}in}(n,x)\defeq \inr(n,x) \\
\mathsf{seq\usc{}glue}(n,x)\defeq \ct{\glue(\inl(n,x))^{-1}}{\glue(\inr(n,x))}.
\end{align*}
\end{constr}

\begin{thm}
Consider a type sequence $\mathcal{A}$, and write $\tilde{A}\defeq\sm{n:\N}A_n$. Moreover, consider the map
\begin{equation*}
\sigma_{\mathcal{A}}:\tilde{A}\to\tilde{A}
\end{equation*}
defined by $\sigma_{\mathcal{A}}(n,a)\defeq (n+1,f_n(a))$. Furthermore, consider a cocone $(h,H)$ with vertex $B$.
The following are equivalent:
\begin{enumerate}
\item The cocone $(h,H)$ with vertex $B$ is colimiting.
\item The defining square
\begin{equation*}
\begin{tikzcd}[column sep=large]
\tilde{A}+\tilde{A} \arrow[r,"{[\idfunc,\sigma_{\mathcal{A}}]}"] \arrow[d,swap,"{[\idfunc,\idfunc]}"] & \tilde{A} \arrow[d,"{\lam{(n,x)}h_n(x)}"] \\
\tilde{A} \arrow[r,swap,"{\lam{(n,x)}h_n(x)}"] & A_\infty,
\end{tikzcd}
\end{equation*}
of $A_\infty$ is a pushout square.
\end{enumerate}
\end{thm}

\subsection{Descent for sequential colimits}

\begin{defn}
The type of \define{descent data} on a type sequence $\mathcal{A}\jdeq (A,f)$ is defined to be
\begin{equation*}
\mathsf{Desc}(\mathcal{A}) \defeq \sm{B:\prd{n:\N}A_n\to\UU}\prd{n:\N}{x:A_n}\eqv{B_n(x)}{B_{n+1}(f_n(x))}.
\end{equation*}
\end{defn}

\begin{defn}
We define a map
\begin{equation*}
\mathsf{desc\usc{}fam} : (A_\infty\to\UU)\to\mathsf{Desc}(\mathcal{A})
\end{equation*}
by $B\mapsto (\lam{n}{x}B(\mathsf{seq\usc{}in}(n,x)),\lam{n}{x}\mathsf{tr}_B(\mathsf{seq\usc{}glue}(n,x)))$.
\end{defn}

\begin{thm}
The map 
\begin{equation*}
\mathsf{desc\usc{}fam} : (A_\infty\to\UU)\to\mathsf{Desc}(\mathcal{A})
\end{equation*}
is an equivalence.
\end{thm}

\begin{defn}
A \define{cartesian transformation} of type sequences from $\mathcal{A}$ to $\mathcal{B}$ is a pair $(h,H)$ consisting of
\begin{align*}
h & : \prd{n:\N} A_n\to B_n \\
H & : \prd{n:\N} g_n\circ h_n \htpy h_{n+1}\circ f_n,
\end{align*}
such that each of the squares in the diagram
\begin{equation*}
\begin{tikzcd}
A_0 \arrow[d,swap,"h_0"] \arrow[r,"f_0"] & A_1 \arrow[d,swap,"h_1"] \arrow[r,"f_1"] & A_2 \arrow[d,swap,"h_2"] \arrow[r,"f_2"] & \cdots \\
B_0 \arrow[r,swap,"g_0"] & B_1 \arrow[r,swap,"g_1"] & B_2 \arrow[r,swap,"g_2"] & \cdots
\end{tikzcd}
\end{equation*}
is a pullback square. We define
\begin{align*}
\mathsf{cart}(\mathcal{A},\mathcal{B}) & \defeq\sm{h:\prd{n:\N}A_n\to B_n} \\
& \qquad\qquad \sm{H:\prd{n:\N}g_n\circ h_n\htpy h_{n+1}\circ f_n}\prd{n:\N}\mathsf{is\usc{}pullback}(h_n,f_n,H_n),
\end{align*}
and we write
\begin{equation*}
\mathsf{Cart}(\mathcal{B}) \defeq \sm{\mathcal{A}:\mathsf{Seq}}\mathsf{cart}(\mathcal{A},\mathcal{B}).
\end{equation*}
\end{defn}

\begin{defn}
We define a map
\begin{equation*}
\mathsf{cart\usc{}map}(\mathcal{B}) : \Big(\sm{X':\UU}X'\to X\Big)\to\mathsf{Cart}(\mathcal{B}).
\end{equation*}
which associates to any morphism $h:X'\to X$ a cartesian transformation of type sequences into $\mathcal{B}$.
\end{defn}

\begin{thm}
The operation $\mathsf{cart\usc{}map}(\mathcal{B})$ is an equivalence.
\end{thm}

\subsection{The flattening lemma for sequential colimits}

The flattening lemma for sequential colimits essentially states that sequential colimits commute with $\Sigma$. 

\begin{lem}
Consider
\begin{align*}
B & : \prd{n:\N}A_n\to\UU \\
g & : \prd{n:\N}{x:A_n}\eqv{B_n(x)}{B_{n+1}(f_n(x))}.
\end{align*}
and suppose $P:A_\infty\to\UU$ is the unique family equipped with
\begin{align*}
e & : \prd{n:\N}\eqv{B_n(x)}{P(\mathsf{seq\usc{}in}(n,x))}
\end{align*}
and homotopies $H_n(x)$ witnessing that the square
\begin{equation*}
\begin{tikzcd}[column sep=7em]
B_n(x) \arrow[r,"g_n(x)"] \arrow[d,swap,"e_n(x)"] & B_{n+1}(f_n(x)) \arrow[d,"e_{n+1}(f_n(x))"] \\
P(\mathsf{seq\usc{}in}(n,x)) \arrow[r,swap,"{\mathsf{tr}_P(\mathsf{seq\usc{}glue}(n,x))}"] & P(\mathsf{seq\usc{}in}(n+1,f_n(x)))
\end{tikzcd}
\end{equation*}
commutes. Then $\sm{t:A_\infty}P(t)$ satisfies the universal property of the sequential colimit of the type sequence
\begin{equation*}
\begin{tikzcd}
\sm{x:A_0}B_0(x) \arrow[r,"{\tot[f_0]{g_0}}"] & \sm{x:A_1}B_1(x) \arrow[r,"{\tot[f_1]{g_1}}"] & \sm{x:A_2}B_2(x) \arrow[r,"{\tot[f_2]{g_2}}"] & \cdots.
\end{tikzcd}
\end{equation*}
\end{lem}

In the following theorem we rephrase the flattening lemma in using cartesian transformations of type sequences.

\begin{thm}
Consider a commuting diagram of the form
\begin{equation*}
\begin{tikzcd}[column sep=small,row sep=small]
A_0 \arrow[rr] \arrow[dd] & & A_1 \arrow[rr] \arrow[dr] \arrow[dd] &[-.9em] &[-.9em] A_2 \arrow[dl] \arrow[dd] & & \cdots \\
& & & X \arrow[from=ulll,crossing over] \arrow[from=urrr,crossing over] \arrow[from=ur,to=urrr] \\
B_0 \arrow[rr] \arrow[drrr] & & B_1 \arrow[rr] \arrow[dr] & & B_2 \arrow[rr] \arrow[dl] & & \cdots \arrow[dlll] \\
& & & Y \arrow[from=uu,crossing over] 
\end{tikzcd}
\end{equation*}
If each of the vertical squares is a pullback square, and $Y$ is the sequential colimit of the type sequence $B_n$, then $X$ is the sequential colimit of the type sequence $A_n$. 
\end{thm}

\begin{cor}
Consider a commuting diagram of the form
\begin{equation*}
\begin{tikzcd}[column sep=small,row sep=small]
A_0 \arrow[rr] \arrow[dd] & & A_1 \arrow[rr] \arrow[dr] \arrow[dd] &[-.9em] &[-.9em] A_2 \arrow[dl] \arrow[dd] & & \cdots \\
& & & X \arrow[from=ulll,crossing over] \arrow[from=urrr,crossing over] \arrow[from=ur,to=urrr] \\
B_0 \arrow[rr] \arrow[drrr] & & B_1 \arrow[rr] \arrow[dr] & & B_2 \arrow[rr] \arrow[dl] & & \cdots \arrow[dlll] \\
& & & Y \arrow[from=uu,crossing over] 
\end{tikzcd}
\end{equation*}
If each of the vertical squares is a pullback square, then the square
\begin{equation*}
\begin{tikzcd}
A_\infty \arrow[r] \arrow[d] & X \arrow[d] \\
B_\infty \arrow[r] & Y
\end{tikzcd}
\end{equation*} 
is a pullback square.
\end{cor}

\subsection{Constructing the propositional truncation}\label{sec:propositional-truncation-constr}
The propositional truncation can be used to construct the image of a map, so we construct that first. We construct the propositional truncation of $A$ via a construction called the \define{join construction}, as the colimit of the sequence of join-powers of $A$
\begin{equation*}
  \begin{tikzcd}
    A \arrow[r] & \join{A}{A} \arrow[r] & \join{A}{(\join{A}{A})} \arrow[r] & \cdots
  \end{tikzcd}
\end{equation*}
The join-powers of $A$ are defined recursively on $n$, by taking\footnote{In this definition, the case $A^{\ast1}\defeq A$ is slightly redundant because we have an equivalence
\begin{equation*}
  \join{A}{\emptyt}\simeq A.
\end{equation*}
Nevertheless, it is nice to have that $A^{\ast 1}\jdeq A$.}
\begin{align*}
  A^{\ast0} & \defeq \emptyt \\
  A^{\ast 1} & \defeq A \\
  A^{\ast(n+2)} & \defeq \join{A}{A^{\ast (n+1)}}.
\end{align*}
We will write $A^{\ast\infty}$ for the colimit of the sequence
\begin{equation*}
  \begin{tikzcd}
    A \arrow[r,"\inr"] & \join{A}{A} \arrow[r,"\inr"] & \join{A}{(\join{A}{A})} \arrow[r,"\inr"] & \cdots.
  \end{tikzcd}
\end{equation*}
The sequential colimit $A^{\ast\infty}$ comes equipped with maps $\inseq_n:A^{\ast (n+1)}\to A^{\ast\infty}$, and we will write
\begin{equation*}
  \eta\defeq\inseq_0:A\to A^{\ast\infty}.
\end{equation*}
Our goal is to show $A^{\ast\infty}$ is a proposition, and that $\eta:A\to A^{\ast\infty}$ satisfies the universal property of the propositional truncation of $A$. Before showing that $A^{\ast\infty}$ is indeed a proposition, let us show in two steps that for any proposition $P$, the map
\begin{equation*}
  (A^{\ast\infty}\to P)\to (A\to P)
\end{equation*}
is indeed an equivalence. 

\begin{lem}\label{lem:extend_join_prop}
Suppose $f:A\to P$, where $A$ is any type and $P$ is a proposition.
Then the precomposition function
\begin{equation*}
\blank\circ\inr:(\join{A}{B}\to P)\to (B\to P)
\end{equation*}
is an equivalence, for any type $B$.
\end{lem}

\begin{proof}
  Since the precomposition function
  \begin{equation*}
    \blank\circ\inr:(\join{A}{B}\to P)\to (B\to P)
  \end{equation*}
  is a map between propositions, it suffices to construct a map
  \begin{equation*}
    (B\to P)\to (\join{A}{B}\to P).
  \end{equation*}
  Let $g:B\to P$. Then the square
  \begin{equation*}
    \begin{tikzcd}
      A\times B \arrow[r,"\proj 2"] \arrow[d,swap,"\proj 1"] & B \arrow[d,"g"] \\
      A \arrow[r,swap,"f"] & P
    \end{tikzcd}
  \end{equation*}
  commutes since $P$ is a proposition. Therefore we obtain a map $\join{A}{B}\to P$ by the universal property of the join.
\end{proof}

\begin{prp}\label{prp:universal-property-brck}
Let $A$ be a type, and let $P$ be a proposition. Then the function
\begin{equation*}
\blank\circ \eta : (A^{\ast\infty}\to P)\to (A\to P)
\end{equation*}
is an equivalence. 
\end{prp}

\begin{proof}
  Since the map
  \begin{equation*}
    \blank\circ \eta : (A^{\ast\infty}\to P)\to (A\to P)
  \end{equation*}
  is a map between propositions, it suffices to construct a map in the converse direction.

  Let $f:A\to P$. First, we show by recursion that there are maps
  \begin{equation*}
    f_n:A^{\ast(n+1)}\to P.
  \end{equation*}
  The map $f_0$ is of course just defined to be $f$. Given $f_n:A^{\ast(n+1)}$ we obtain $f_{n+1}:\join{A}{A^{\ast(n+1)}}\to P$ by \cref{lem:extend_join_prop}. Because $P$ is assumed to be a proposition it is immediate that the maps $f_n$ form a cocone with vertex $P$ on the sequence
  \begin{equation*}
    \begin{tikzcd}
      A \arrow[r,"\inr"] & \join{A}{A} \arrow[r,"\inr"] & \join{A}{(\join{A}{A})} \arrow[r,"\inr"] & \cdots.
    \end{tikzcd}
  \end{equation*}
  From this cocone we obtain the desired map $(A^{\ast\infty}\to P)$.
\end{proof}

\begin{prp}\label{prp:isprop-infjp}
The type $A^{\ast\infty}$ is a proposition for any type $A$.
\end{prp}

\begin{proof}
  By \cref{lem:isprop_eq} it suffices to show that
  \begin{equation*}
    A^{\ast\infty}\to \iscontr(A^{\ast\infty}).
  \end{equation*}
  Since the type $\iscontr(A^{\ast\infty})$ is already known to be a proposition by \cref{ex:isprop_istrunc}, it follows from \cref{prp:universal-property-brck} that it suffices to show that
\begin{equation*}
A\to \iscontr(A^{\ast\infty}).
\end{equation*}

Let $x:A$. To see that $A^{\ast\infty}$ is contractible it suffices by \cref{ex:seqcolim_contr} to show that $\inr:A^{\ast n}\to A^{\ast(n+1)}$ is homotopic to the constant function $\const_{\inl(x)}$. However, we get a homotopy $\const_{\inl(x)}\htpy \inr$ immediately from the path constructor $\glue$.  
\end{proof}

All the definitions are now in place to define the propositional truncation of a type.

\begin{defn}
  For any type $A$ we define the type
  \begin{equation*}
    \trunc{-1}{A}\defeq A^{\ast\infty},
  \end{equation*}
  and we define $\eta:A\to\trunc{-1}{A}$ to be the constructor $\seqin_0$ of the sequential colimit $A^{\ast\infty}$. Often we simply write $\brck{A}$ for $\trunc{-1}{A}$.
\end{defn}

The type $\trunc{-1}{A}$ is a proposition by \cref{prp:isprop-infjp}, and
\begin{equation*}
  \eta:A\to\trunc{-1}{A}
\end{equation*}
satisfies the universal property of propositional truncation by \cref{prp:universal-property-brck}.

\begin{prp}
  The propositional truncation operation is functorial in the sense that for any map $f:A\to B$ there is a unique map $\brck{f}:\brck{A}\to\brck{B}$ such that the square
  \begin{equation*}
    \begin{tikzcd}
      A \arrow[r,"f"] \arrow[d,swap,"\eta"] & B \arrow[d,"\eta"] \\
      \brck{A} \arrow[r,swap,"\brck{f}"] & \brck{B}
    \end{tikzcd}
  \end{equation*}
  commutes. Moreover, there are homotopies
  \begin{align*}
    \brck{\idfunc[A]} & \htpy \idfunc[\brck{A}] \\
    \brck{g\circ f} & \htpy \brck{g}\circ\brck{f}.
  \end{align*}
\end{prp}

\begin{proof}
  The functorial action of propositional truncation is immediate by the universal property of propositional truncation. To see that the functorial action preserves the identity, note that the type of maps $\brck{A}\to\brck{A}$ for which the square
  \begin{equation*}
    \begin{tikzcd}
      A \arrow[r,"\idfunc"] \arrow[d,swap,"\eta"] & A \arrow[d,"\eta"] \\
      \brck{A} \arrow[r,densely dotted] & \brck{A}
    \end{tikzcd}
  \end{equation*}
  commutes is contractible. Since this square commutes for both $\brck{\idfunc}$ and for $\idfunc$, it must be that they are homotopic. The proof that the functorial action of propositional truncation preserves composition is similar.
\end{proof}

\subsection{Proving type theoretical replacement}

Our goal is now to show that the image of a map $f:A\to B$ from an essentially small type $A$ into a locally small type $B$ is again essentially small. This property is called the type theoretic replacement property. In order to prove this property, we have to find another construction of the image of a map. In order to make this construction, we define a join operation on maps.

\begin{defn}
  Consider two maps $f:A\to X$ and $g:B\to X$ with a common codomain $X$.
  \begin{enumerate}
  \item We define the type $\join[X]{A}{B}$ as the pushout
    \begin{equation*}
      \begin{tikzcd}
        A\times_X B \arrow[r,"\pi_2"] \arrow[d,swap,"\pi_1"] & B \arrow[d,"\inr"] \\
        A \arrow[r,swap,"\inl"] & \join[X]{A}{B}.
      \end{tikzcd}
    \end{equation*}
  \item We define the \define{join} $\join{f}{g}:\join[X]{A}{B}\to X$ to be the unique map for which the diagram
        \begin{equation*}
      \begin{tikzcd}
        A\times_X B \arrow[r,"\pi_2"] \arrow[d,swap,"\pi_1"] & B \arrow[d,"\inr"] \arrow[ddr,bend left=15,"g"] \\
        A \arrow[r,swap,"\inl"] \arrow[drr,bend right=15,swap,"f"]  & \join[X]{A}{B} \arrow[dr,densely dotted,swap,"\join{f}{g}"] \\
        & & X
      \end{tikzcd}
    \end{equation*}
  \end{enumerate}
\end{defn}

The reason to call the map $\join{f}{g}$ the join of $f$ and $g$ is that the fiber of $\join{f}{g}$ at any $x:X$ is equivalent to the join of the fibers of $f$ and $g$ at $x$.

\begin{lem}
  Consider two maps $f:A\to X$ and $g:B\to X$. Then there is an equivalence
  \begin{equation*}
    \fib{\join{f}{g}}{x}\simeq\join{\fib{f}{x}}{\fib{g}{x}}
  \end{equation*}
  for any $x:X$.
\end{lem}

\begin{proof}
  Consider the commuting cube
  \begin{equation*}
    \begin{tikzcd}
      & \fib{f}{x}\times\fib{g}{x} \arrow[dl] \arrow[dr] \arrow[d] \\
      \fib{f}{x} \arrow[d] & A\times_X B \arrow[dl] \arrow[dr] & \fib{g}{x} \arrow[d] \arrow[dl,crossing over] \\
      A \arrow[dr] & \unit \arrow[from=ul,crossing over] \arrow[d] & B \arrow[dl] \\
      & X
    \end{tikzcd}
  \end{equation*}
  In this cube, the bottom square is a canonical pullback square. The two squares in the front are pullbacks by \cref{lem:fib_pb}, and the top square is a pullback square by \cref{lem:prod_pb}. Therefore it follows by \cref{rmk:strongly-cartesian} that all the faces of this cube are pullback squares, and hence by \cref{thm:effectiveness-pullback} we obtain that the square
  \begin{equation*}
    \begin{tikzcd}
      \join{\fib{f}{x}}{\fib{g}{x}} \arrow[d,densely dotted] \arrow[r] & \unit \arrow[d] \\
      \join[X]{A}{B} \arrow[r,swap,"\join{f}{g}"] & X
    \end{tikzcd}
  \end{equation*}
  is a pullback square. Now the claim follows by the uniqueness of pullbacks, which was shown in \cref{cor:uniquely-unique-pullback}.
\end{proof}

\begin{lem}
Consider a map $f:A\to X$, an embedding $m:U\to X$, and $h:\mathrm{hom}_X(f,m)$. Then the map
\begin{equation*}
\mathrm{hom}_X(\join{f}{g},m)\to \mathrm{hom}_X(g,m)
\end{equation*}
is an equivalence for any $g:B\to X$.
\end{lem}

\begin{proof}
Note that both types are propositions, so any equivalence can be used to prove the claim. Thus, we simply calculate
\begin{align*}
\mathrm{hom}_X(\join{f}{g},m) & \eqvsym \prd{x:X}\fib{\join{f}{g}}{x}\to \fib{m}{x} \\
& \eqvsym \prd{x:X}\join{\fib{f}{x}}{\fib{g}{x}}\to\fib{m}{x} \\
& \eqvsym \prd{x:X}\fib{g}{x}\to\fib{m}{x} \\
& \eqvsym \mathrm{hom}_X(g,m).
\end{align*}
The first equivalence holds by \cref{ex:triangle_fib}; the second equivalence holds by \cref{ex:fib_join}, also using \cref{ex:equiv_precomp,lem:postcomp_equiv} where we established that that pre- and postcomposing by an equivalence is an equivalence; the third equivalence holds by \cref{lem:extend_join_prop,lem:postcomp_equiv}; the last equivalence again holds by \cref{ex:triangle_fib}.
\end{proof}

For the construction of the image of $f:A\to X$ we observe that if we are given an embedding $m:U\to X$ and a map $(i,I):\mathrm{hom}_X(f,m)$, then $(i,I)$ extends uniquely along $\inr:A\to \join[X]{A}{A}$ to a map $\mathrm{hom}_X(\join{f}{f},m)$. This extension again extends uniquely along $\inr:\join[X]{A}{A}\to \join[X]{A}{(\join[X]{A}{A})}$ to a map $\mathrm{hom}_X(\join{f}{(\join{f}{f})},m)$ and so on, resulting in a diagram of the form
\begin{equation*}
\begin{tikzcd}
A \arrow[dr] \arrow[r,"\inr"] & \join[X]{A}{A} \arrow[d,densely dotted] \arrow[r,"\inr"] & \join[X]{A}{(\join[X]{A}{A})} \arrow[dl,densely dotted] \arrow[r,"\inr"] & \cdots \arrow[dll,densely dotted,bend left=10] \\
& U
\end{tikzcd}
\end{equation*}

\begin{defn}
Suppose $f:A\to X$ is a map. Then we define the \define{fiberwise join powers} 
\begin{equation*}
f^{\ast n}:A_X^{\ast n} X.
\end{equation*}
\end{defn}

\begin{constr}
Note that the operation $(B,g)\mapsto (\join[X]{A}{B},\join{f}{g})$ defines an endomorphism on the type
\begin{equation*}
\sm{B:\UU}B\to X.
\end{equation*}
We also have $(\emptyt,\ind{\emptyt})$ and $(A,f)$ of this type. For $n\geq 1$ we define
\begin{align*}
A_X^{\ast (n+1)} & \defeq \join[X]{A}{A_X^{\ast n}} \\
f^{\ast (n+1)} & \defeq \join{f}{f^{\ast n}}.\qedhere
\end{align*}
\end{constr}

\begin{defn}
We define $A_X^{\ast\infty}$ to be the sequential colimit of the type sequence
\begin{equation*}
\begin{tikzcd}
A_X^{\ast 0} \arrow[r] & A_X^{\ast 1} \arrow[r,"\inr"] & A_X^{\ast 2} \arrow[r,"\inr"] & \cdots.
\end{tikzcd}
\end{equation*}
Since we have a cocone
\begin{equation*}
\begin{tikzcd}
A_X^{\ast 0} \arrow[r] \arrow[dr,swap,"f^{\ast 0}" near start] & A_X^{\ast 1} \arrow[r,"\inr"] \arrow[d,swap,"f^{\ast 1}" near start] & A_X^{\ast 2} \arrow[r,"\inr"] \arrow[dl,swap,"f^{\ast 2}" xshift=1ex] & \cdots \arrow[dll,bend left=10] \\
& X
\end{tikzcd}
\end{equation*}
we also obtain a map $f^{\ast\infty}:A_X^{\ast\infty}\to X$ by the universal property of $A_X^{\ast\infty}$. 
\end{defn}

\begin{lem}\label{lem:finfjp_up}
Let $f:A\to X$ be a map, and let $m:U\to X$ be an embedding. Then the function
\begin{equation*}
\blank\circ \seqin_0: \mathrm{hom}_X(f^{\ast\infty},m)\to \mathrm{hom}_X(f,m)
\end{equation*}
is an equivalence. 
\end{lem}

\begin{thm}\label{lem:isprop_infjp}
For any map $f:A\to X$, the map $f^{\ast\infty}:A_X^{\ast\infty}\to X$ is an embedding that satisfies the universal property of the image inclusion of $f$.
\end{thm}

\begin{lem}
Consider a commuting square
\begin{equation*}
\begin{tikzcd}
A \arrow[r] \arrow[d] & B \arrow[d] \\
C \arrow[r] & D.
\end{tikzcd}
\end{equation*}
\begin{enumerate}
\item If the square is cartesian, $B$ and $C$ are essentially small, and $D$ is locally small, then $A$ is essentially small.
\item If the square is cocartesian, and $A$, $B$, and $C$ are essentially small, then $D$ is essentially small. 
\end{enumerate}
\end{lem}

\begin{cor}
Suppose $f:A\to X$ and $g:B\to X$ are maps from essentially small types $A$ and $B$, respectively, to a locally small type $X$. Then $A\times_X B$ is again essentially small. 
\end{cor}

\begin{lem}
Consider a type sequence
\begin{equation*}
\begin{tikzcd}
A_0 \arrow[r,"f_0"] & A_1 \arrow[r,"f_1"] & A_2 \arrow[r,"f_2"] & \cdots
\end{tikzcd}
\end{equation*}
where each $A_n$ is essentially small. Then its sequential colimit is again essentially small. 
\end{lem}

\begin{thm}\label{thm:replacement}
  For any map $f:A\to B$ from an essentially small type $A$ into a locally small type $B$, the image of $f$ is again essentially small.
\end{thm}

\begin{cor}
  Consider a $\UU$-small type $A$, and an equivalence relation $R$ over $A$ valued in the $\UU$-small propositions. Then the set quotient $A/R$ is essentially small.
\end{cor}

\begin{exercises}
\exercise \label{ex:seqcolim_shift}
Show that the sequential colimit of a type sequence
\begin{equation*}
\begin{tikzcd}
A_0 \arrow[r,"f_0"] & A_1 \arrow[r,"f_1"] & A_2 \arrow[r,"f_2"] & \cdots
\end{tikzcd}
\end{equation*}
is equivalent to the sequential colimit of its shifted type sequence
\begin{equation*}
\begin{tikzcd}
A_1 \arrow[r,"f_1"] & A_2 \arrow[r,"f_2"] & A_3 \arrow[r,"f_3"] & \cdots.
\end{tikzcd}
\end{equation*}
  \exercise Let
  \begin{tikzcd}
    P_0 \arrow[r] & P_1 \arrow[r] & P_2 \arrow[r] & \cdots
  \end{tikzcd}
  be a sequence of propositions. Show that
  \begin{equation*}
    \eqv{\colim_n(P_n)}{\exists_{(n:\N)} P_n}.
  \end{equation*}
\exercise \label{ex:seqcolim_contr}Consider a type sequence
\begin{equation*}
\begin{tikzcd}
A_0 \arrow[r,"f_0"] & A_1 \arrow[r,"f_1"] & A_2 \arrow[r,"f_2"] & \cdots
\end{tikzcd}
\end{equation*}
and suppose that $f_n\htpy \mathsf{const}_{a_{n+1}}$ for some $a_n:\prd{n:\N}A_n$. Show that the sequential colimit is contractible.
\exercise Define the $\infty$-sphere $\sphere{\infty}$ as the sequential colimit of
\begin{equation*}
\begin{tikzcd}
\sphere{0} \arrow[r,"f_0"] & \sphere{1} \arrow[r,"f_1"] & \sphere{2} \arrow[r,"f_2"] & \cdots
\end{tikzcd}
\end{equation*}
where $f_0:\sphere{0}\to\sphere{1}$ is defined by $f_0(\bfalse)\jdeq \inl(\ttt)$ and $f_0(\btrue)\jdeq \inr(\ttt)$, and $f_{n+1}:\sphere{n+1}\to\sphere{n+2}$ is defined as $\susp(f_n)$. Use \cref{ex:seqcolim_contr} to show that $\sphere{\infty}$ is contractible.
\exercise Consider a type sequence
\begin{equation*}
\begin{tikzcd}
A_0 \arrow[r,"f_0"] & A_1 \arrow[r,"f_1"] & A_2 \arrow[r,"f_2"] & \cdots
\end{tikzcd}
\end{equation*}
in which $f_n:A_n\to A_{n+1}$ is weakly constant in the sense that
\begin{equation*}
\prd{x,y:A_n} f_n(x)=f_n(y)
\end{equation*}
Show that $A_\infty$ is a mere proposition.
\exercise Show that $\N$ is the sequential colimit of
\begin{equation*}
  \begin{tikzcd}
    \Fin(0) \arrow[r,"\inl"] & \Fin(1) \arrow[r,"\inl"] & \Fin(2) \arrow[r,"\inl"] & \cdots.
  \end{tikzcd}
\end{equation*}
\end{exercises}
