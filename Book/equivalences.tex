% !TEX root = hott_intro.tex

\section{Equivalences}

\index{equivalence|(}
\subsection{Homotopies}
\index{homotopy|(}
In homotopy type theory, a homotopy is just a pointwise equality between two functions $f$ and $g$. We view the type of homotopies as the observational equality for $\Pi$-types.\index{observational equality!of Pi-types@{of $\Pi$-types}}

\begin{defn}
Let $f,g:\prd{x:A}P(x)$ be two dependent functions. The type of \define{homotopies}\index{homotopy} from $f$ to $g$ is defined as\index{f htpy g@{$f\htpy g$}|see {homotopy}}
\begin{equation*}
f\htpy g \defeq \prd{x:A} f(x)=g(x).
\end{equation*}
\end{defn}

Note that the type of homotopies $f\htpy g$ is a special case of a dependent function type. Therefore the definition of homotopies is set up in such a way that we may also consider homotopies \emph{between}\index{homotopy!iterated}\index{iterated homotopies} homotopies, and even further homotopies between those higher homotopies. More concretely, if $H,K:f\htpy g$ are two homotopies, then the type of homotopies $H\htpy K$ between them is just the type
\begin{equation*}
\prd{x:A} H(x)=K(x).
\end{equation*}

\index{groupoid laws!of homotopies|(}
\index{homotopy!groupoid laws|(}
In the following definition we define the groupoidal structure of homotopies. Note that we implement the groupoid laws as \emph{homotopies} rather than as identifications.

\begin{defn}\label{defn:htpy_groupoid}\index{groupoid laws!of homotopies}
For any type family $B$ over $A$ there are operations\index{homotopy!htpy-refl@{$\htpyrefl$}}\index{htpy-refl@{$\htpyrefl$}}\index{homotopy!htpy-inv@{$\htpyinv$}}\index{htpy-inv@{$\htpyinv$}}\index{homotopy!htpy-concat@{$\htpyconcat$}}\index{htpy-concat@{$\htpyconcat$}}
\begin{align*}
& \htpyrefl & & : \prd{f:\prd{x:A}B(x)}f\htpy f \\
& \htpyinv & & : \prd{f,g:\prd{x:A}B(x)} (f\htpy g)\to(g\htpy f) \\
& \htpyconcat & & : \prd{f,g,h:\prd{x:A}B(x)} (f\htpy g)\to ((g\htpy h)\to (f\htpy h)).
\end{align*}
We will write $H^{-1}$ for $\htpyinv(H)$, and $\ct{H}{K}$ for $\htpyconcat(H,K)$. 

Furthermore, we define
\begin{align*}
& \htpyassoc(H,K,L) & & : \ct{(\ct{H}{K})}{L}\htpy\ct{H}{(\ct{K}{L})} \\
& \htpyleftunit(H) & & : \ct{\htpyrefl_f}{H}\htpy H \\
& \htpyrightunit(H) & & : \ct{H}{\htpyrefl_g}\htpy H \\
& \htpyleftinv(H) & & : \ct{H^{-1}}{H} \htpy \htpyrefl_g \\
& \htpyrightinv(H) & & : \ct{H}{H^{-1}} \htpy \htpyrefl_f
\end{align*}
for any $H:f\htpy g$, $K:g\htpy h$ and $L:h\htpy i$, where $f,g,h,i:\prd{x:A}B(x)$.
\end{defn}

\begin{constr}
We define
\begin{align*}
\htpyrefl(f) & \defeq \lam{x} \refl{f(x)} \\
\htpyinv(H) & \defeq \lam{x} H(x)^{-1} \\
\htpyconcat(H,K) & \defeq \lam{x}\ct{H(x)}{K(x)},
\end{align*}
where $H:f\htpy g$ and $K:g\htpy h$ are homotopies. Furthermore, we define
\begin{align*}
\htpyassoc(H,K,L) & \defeq \lam{x}\assoc(H(x),K(x),L(x)) \\
\htpyleftunit(H) & \defeq \lam{x}\leftunit(H(x)) \\
\htpyrightunit(H) & \defeq \lam{x}\rightunit(H(x)) \\
\htpyleftinv(H) & \defeq \lam{x}\leftinv(H(x)) \\
\htpyrightinv(h) & \defeq \lam{x}\rightinv(H(x)).\qedhere
\end{align*}
\end{constr}
\index{groupoid laws!of homotopies|)}
\index{homotopy!groupoid laws|)}


Apart from the groupoid operations and their laws, we will occasionally need \emph{whiskering} operations.

\begin{defn}
We define the following \define{whiskering}\index{homotopy!whiskering operations}\index{whiskering operations!of homotopies} operations on homotopies:
\begin{enumerate}
\item Suppose $H:f\htpy g$ for two functions $f,g:A\to B$, and let $h:B\to C$. We define\index{h . H@{$h\cdot H$}|see {homotopy, whiskering operations}}
\begin{equation*}
h\cdot H\defeq \lam{x}\ap{h}{H(x)}:h\circ f\htpy h\circ g.
\end{equation*}
\item Suppose $f:A\to B$ and $H:g\htpy h$ for two functions $g,h:B\to C$. We define\index{H . f@{$H\cdot f$}|see {homotopy, whiskering operations}}
\begin{equation*}
H\cdot f\defeq\lam{x}H(f(x)):h\circ f\htpy g\circ f.
\end{equation*}
\end{enumerate}
\end{defn}

We also use homotopies to express the commutativity of diagrams. For example, we say that a triangle\index{homotopy!commutative diagram}
\begin{equation*}
  \begin{tikzcd}[column sep=tiny]
    A \arrow[rr,"h"] \arrow[dr,swap,"f"] & & B \arrow[dl,"g"] \\
    & X
  \end{tikzcd}
\end{equation*}
commutes if it comes equipped with a homotopy $H:f\htpy g\circ h$, and we say that a square
\begin{equation*}
  \begin{tikzcd}
    A \arrow[r,"g"] \arrow[d,swap,"f"] & A' \arrow[d,"{f'}"] \\
    B \arrow[r,swap,"h"] & B'
  \end{tikzcd}
\end{equation*}
if it comes equipped with a homotopy $h \circ f~g\circ f'$.
\index{homotopy|)}

\subsection{Bi-invertible maps}

\begin{defn}
Let $f:A\to B$ be a function. We say that $f$ has a \define{section}\index{section!of a map}\index{function!has a section} if there is a term of type\index{sec(f)@{$\sections(f)$}}
\begin{equation*}
\sections(f) \defeq \sm{g:B\to A} f\circ g\htpy \idfunc[B].
\end{equation*}
Dually, we say that $f$ has a \define{retraction}\index{retraction}\index{function!has a retraction} if there is a term of type\index{retr(f)@{$\retractions(f)$}}
\begin{equation*}
\retractions(f) \defeq \sm{h:B\to A} h\circ f\htpy \idfunc[A].
\end{equation*}
If a map $f:A \to B$ has a retraction, we also say that $A$ is a \define{retract}\index{retract!of a type} of $B$.
We say that a function $f:A\to B$ is an \define{equivalence}\index{is an equivalence}\index{function!is an equivalence} if it has both a section and a retraction, i.e., if it comes equipped with a term of type\index{is-equiv(f)@{$\isequiv(f)$}}
\begin{equation*}
\isequiv(f)\defeq\sections(f)\times\retractions(f).
\end{equation*}
We will write $\eqv{A}{B}$\index{A simeq B@{$\eqv{A}{B}$}|see {equivalence}} for the type $\sm{f:A\to B}\isequiv(f)$.
\end{defn}

\begin{rmk}
An equivalence, as we defined it here, can be thought of as a \emph{bi-invertible map}\index{bi-invertible map|see {equivalence}}, since it comes equipped with a separate left and right inverse. Explicitly, if $f$ is an equivalence, then there are
\begin{align*}
g & : B\to A & h & : B\to A \\
G & : f\circ g \htpy \idfunc[B] & H & : h\circ f \htpy \idfunc[A].
\end{align*}
Clearly, if $f$ has an inverse\index{has an inverse}\index{function!has an inverse} in the sense that it comes equipped with a function $g:B\to A$ such that $f\circ g\htpy\idfunc[B]$ and $g\circ f\htpy\idfunc[A]$, then $f$ is an equivalence. We write\index{has-inverse(f)@{$\hasinverse(f)$}}
\begin{equation*}
\hasinverse(f)\defeq\sm{g:B\to A} (f\circ g\htpy \idfunc[B])\times (g\circ f\htpy\idfunc[A]).
\end{equation*}
\end{rmk}

\begin{lem}\label{lem:inv_equiv}
Any equivalence $e:A\simeq B$ can be given the structure of an invertible map.\index{equivalence!has an inverse} We define $e^{-1}$ to be the section $g:B\to A$ of $e$.\index{equivalence!inverse}\index{inverse!of an equivalence}
\end{lem}

\begin{proof}
First we construct for any equivalence $f$ with right inverse $g$ and left inverse $h$ a homotopy $K:g\htpy h$. For any $y:B$, we have 
\begin{equation*}
\begin{tikzcd}[column sep=huge]
g(y) \arrow[r,equals,"H(g(y))^{-1}"] & hfg(y) \arrow[r,equals,"\ap{h}{G(y)}"] & h(y).
\end{tikzcd}
\end{equation*} 
Therefore we define a homotopy $K:g\htpy h$ by $K\defeq \ct{(H\cdot g)^{-1}}{h\cdot G}$.
Using the homotopy $K$ we are able to show that $g$ is also a left inverse of $f$. For $x:A$ we have the identification
\begin{equation*}
\begin{tikzcd}[column sep=large]
gf(x) \arrow[r,equals,"K(f(x))"] & hf(x) \arrow[r,equals,"H(x)"] & x.
\end{tikzcd}\qedhere
\end{equation*}
\end{proof}

\begin{cor}
The inverse of an equivalence is again an equivalence.\index{inverse!of an equivalence!is an equivalence}\index{is an equivalence!inverse of an equivalence}
\end{cor}

\begin{proof}
Let $f:A\to B$ be an equivalence. By \cref{lem:inv_equiv} it follows that the section of $f$ is also a retraction. Therefore it follows that the section is itself an invertible map, with inverse $f$. Hence it is an equivalence.
\end{proof}

\begin{rmk}\label{thm:id_equiv}
For any type $A$, the identity function $\idfunc[A]$ is an equivalence, since it is its own section and its own retraction\index{identity function!is an equivalence}\index{is an equivalence!identity function}
\end{rmk}

\begin{eg}
  For any type $C(x,y)$ indexed by $x:A$ and $y:B$, the swap function\index{function!swap!is an equivalence}\index{is an equivalence!swap function}
\begin{equation*}
\sigma:\Big(\prd{x:A}\prd{y:B}C(x,y)\Big)\to\Big(\prd{y:B}\prd{x:A}C(x,y)\Big)
\end{equation*}
that swaps the order of the arguments $x$ and $y$ is an equivalence by \cref{ex:swap}.\index{swap function!is an equivalence}
\end{eg}

\subsection{The identity type of a \texorpdfstring{$\Sigma$-}{dependent pair }type}

\index{identity type!of a Sigma-type@{of a $\Sigma$-type}|(}
\index{dependent pair type!identity type|(}
\index{characterization of identity type!Sigma-type@{$\Sigma$-type}|(}
In this section we characterize the identity type of a $\Sigma$-type as a $\Sigma$-type of identity types. In this course we will be characterizing the identity types of many types, so we will follow the general outline of how such a characterization goes:
\begin{enumerate}
\item First we define a binary relation $R:A\to A\to \UU$ on the type $A$ that we are interested in. This binary relation is intended to be equivalent to its identity type.
\item Then we will show that this binary relation is reflexive, by constructing a term of type
  \begin{equation*}
    \prd{x:A}R(x,x)
  \end{equation*}
\item Using the reflexivity we will show that there is a canonical map
  \begin{equation*}
    (x=y)\to R(x,y)
  \end{equation*}
  for every $x,y:A$. This map is just constructed by path induction, using the reflexivity of $R$.
\item Finally, it has to be shown that the map
  \begin{equation*}
    (x=y)\to R(x,y)
  \end{equation*}
  is an equivalence for each $x,y:A$. 
\end{enumerate}
The last step is usually the most difficult, and we will refine our methods for this step in \cref{chap:fundamental}, where we establish the fundamental theorem of identity types.\index{fundamental theorem of identity types}

In this section we consider a type family $B$ over $A$. Given two pairs
\begin{equation*}
  (x,y),(x',y'):\sm{x:A}B(x),
\end{equation*}
if we have a path $\alpha:x=x'$ then we can compare $y:B(x)$ to $y':B(x')$ by first transporting $y$ along $\alpha$, i.e., we consider the identity type
\begin{equation*}
  \tr_B(\alpha,y)=y'.
\end{equation*}
Thus it makes sense to think of $(x,y)$ to be identical to $(x',y')$ if there is an identification $\alpha:x=x'$ and an identification $\beta:\tr_B(\alpha,y)=y'$. In the following definition we turn this idea into a binary relation on the $\Sigma$-type.

\begin{defn}
  We will define a relation\index{Eq Sigma@{$\EqSigma$}}\index{dependent pair type!Eq Sigma@{$\EqSigma$}}\index{observational equality!Eq Sigma@{$\EqSigma$}}
  \begin{equation*}
    \EqSigma : \Big(\sm{x:A}B(x)\Big)\to\Big(\sm{x:A}B(x)\Big)\to\UU
  \end{equation*}
  by defining
  \begin{equation*}
    \EqSigma(s,t)\defeq\sm{\alpha:\proj 1(s)=\proj 1(t)}\tr_B(\alpha,\proj 2(s))=\proj 2 (t).
  \end{equation*}
\end{defn}

\begin{lem}
  The relation $\EqSigma$ is reflexive, i.e., there is a term
  \begin{equation*}
    \reflexiveEqSigma:\prd{s:\sm{x:A}B(x)}\EqSigma(s,s).
  \end{equation*}
\end{lem}

\begin{constr}
  This term is constructed by $\Sigma$-induction on $s:\sm{x:A}B(x)$. Thus, it suffices to construct a term of type
  \begin{equation*}
    \prd{x:A}\prd{y:B(x)}\sm{\alpha:x=x}\tr_B(\alpha,y)=y.
  \end{equation*}
  Here we take $\lam{x}\lam{y}(\refl{x},\refl{y})$.
\end{constr}

\begin{defn}
  Consider a type family $B$ over $A$. Then for any $s,t:\sm{x:A}B(x)$ we define a map\index{pair-eq@{$\paireq$}}
  \begin{equation*}
    \paireq: (s=t)\to \EqSigma(s,t)
  \end{equation*}
  by path induction, taking $\paireq(\refl{s})\defeq\reflexiveEqSigma(s)$.
\end{defn}

\begin{thm}\label{thm:eq_sigma}
  Let $B$ be a type family over $A$. Then the map\index{pair-eq@{$\paireq$}!is an equivalence}\index{is an equivalence!pair-eq@{$\paireq$}}
  \begin{equation*}
    \paireq: (s=t)\to \EqSigma(s,t)
  \end{equation*}
  is an equivalence for every $s,t:\sm{x:A}B(x)$.
\end{thm}

\begin{proof}
The maps in the converse direction\index{eq-pair@{$\eqpair$}}
\begin{equation*}
\eqpair : \EqSigma(s,t)\to(\id{s}{t})
\end{equation*}
are defined by repeated $\Sigma$-induction. By $\Sigma$-induction on $s$ and $t$  we see that it suffices to define a map
\begin{equation*}
\eqpair : \Big(\sm{p:x=x'}\id{\tr_B(p,y)}{y'}\Big)\to(\id{(x,y)}{(x',y')}).
\end{equation*}
A map of this type is again defined by $\Sigma$-induction. Thus it suffices to define a dependent function of type
\begin{equation*}
\prd{p:x=x'} (\id{\tr_B(p,y)}{y'}) \to (\id{(x,y)}{(x',y')}).
\end{equation*}
Such a dependent function is defined by double path induction by sending $\pairr{\refl{x},\refl{y}}$ to $\refl{\pairr{x,y}}$. This completes the definition of the function $\eqpair$.

Next, we must show that $\eqpair$ is a section of $\paireq$. In other words, we must construct an identification
\begin{equation*}
\paireq(\eqpair(\alpha,\beta))=\pairr{\alpha,\beta}
\end{equation*}
for each $\pairr{\alpha,\beta}:\sm{\alpha:x=x'}\id{\tr_B(\alpha,y)}{y'}$. We proceed by path induction on $\alpha$, followed by path induction on $\beta$. Then our goal becomes to construct a term of type
\begin{equation*}
\paireq(\eqpair\pairr{\refl{x},\refl{y}})=\pairr{\refl{x},\refl{y}}
\end{equation*}
By the definition of $\eqpair$ we have $\eqpair\pairr{\refl{x},\refl{y}}\jdeq \refl{\pairr{x,y}}$, and by the definition of $\paireq$ we have $\paireq(\refl{\pairr{x,y}})\jdeq\pairr{\refl{x},\refl{y}}$. Thus we may take $\refl{\pairr{\refl{x},\refl{y}}}$ to complete the construction of the homotopy $\paireq\circ\eqpair\htpy\idfunc$.

To complete the proof, we must show that $\eqpair$ is a retraction of $\paireq$. In other words, we must construct an identification
\begin{equation*}
\eqpair(\paireq(p))=p
\end{equation*}
for each $p:s=t$. We proceed by path induction on $p:s=t$, so it suffices to construct an identification 
\begin{equation*}
\eqpair\pairr{\refl{\proj 1(s)},\refl{\proj 2(s)}}=\refl{s}.
\end{equation*}
Now we proceed by $\Sigma$-induction on $s:\sm{x:A}B(x)$, so it suffices to construct an identification
\begin{equation*}
\eqpair\pairr{\refl{x},\refl{y}}=\refl{(x,y)}.
\end{equation*}
Since $\eqpair\pairr{\refl{x},\refl{y}}$ computes to $\refl{(x,y)}$, we may simply take $\refl{\refl{(x,y)}}$.
\end{proof}
\index{identity type!of a Sigma-type@{of a $\Sigma$-type}|)}
\index{dependent pair type!identity type|)}
\index{characterization of identity type!Sigma-type@{$\Sigma$-type}|)}

\begin{exercises}
\exercise \label{ex:equiv_grpd_ops}Show that the functions\index{inv@{$\invfunc$}!is an equivalence}\index{is an equivalence!inv@{$\invfunc$}}\index{concat@{$\concat$}!is a family of equivalences}\index{is an equivalence!concat(p)@{$\concat(p)$}}\index{concat'@{$\concat'$}!is a family of equivalences}\index{is an equivalence!concat'@{$\concat'(q)$}}\index{tr B@{$\tr_B$}!is a family of equivalences}\index{is an equivalence!tr B(p)@{$\tr_B(p)$}}
  \begin{align*}
    \invfunc & :(\id{x}{y})\to(\id{y}{x}) \\
    \concat(p) & : (\id{y}{z})\to(\id{x}{z}) \\
    \concat'(q) & : (\id{x}{y}) \to (\id{x}{z}) \\
    \tr_B(p) & :B(x)\to B(y)
  \end{align*}
  are equivalences, where $\concat'(q,p)\defeq \ct{p}{q}$\index{concat'@{$\concat'$}}. Give their inverses explicitly.
\exercise \index{ex:unit-laws-coprod}Show that the maps\index{unit laws!coproduct}\index{coproduct!unit laws}\index{annihilation laws!cartesian product}\index{cartesian product!annihilation laws}
  \begin{align*}
    \inl & : X \to X+\emptyt &     \proj 1 & : \emptyt \times X \to \emptyt \\
    \inr & : X \to \emptyt+X &    \proj 2 & : X \times \emptyt \to \emptyt
  \end{align*}
  are equivalences.
\exercise
  \begin{subexenum}
  \item \label{ex:htpy_equiv}\index{equivalence!closed under homotopies} Consider two functions $f,g:A\to B$ and a homotopy $H:f\htpy g$. Then
    \begin{equation*}
      \isequiv(f)\leftrightarrow\isequiv(g).
    \end{equation*}
  \item Show that for any two homotopic equivalences $e,e':\eqv{A}{B}$, their inverses are also homotopic.
  \end{subexenum}
\exercise \label{ex:3_for_2}
  Consider a commuting triangle
  \begin{equation*}
    \begin{tikzcd}[column sep=tiny]
      A \arrow[rr,"h"] \arrow[dr,swap,"f"] & & B \arrow[dl,"g"] \\
      & X.
    \end{tikzcd}
  \end{equation*}
  with $H:f\htpy g\circ h$.
  \begin{subexenum}
  \item Suppose that the map $h$ has a section $s:B \to A$. Show that the triangle
    \begin{equation*}
      \begin{tikzcd}[column sep=tiny]
        B \arrow[rr,"s"] \arrow[dr,swap,"g"] & & A \arrow[dl,"f"] \\
        & X.
      \end{tikzcd}
    \end{equation*}
    commutes, and that $f$ has a section if and only if $g$ has a section.
  \item Suppose that the map $g$ has a retraction $r:X\to B$. Show that the triangle
    \begin{equation*}
      \begin{tikzcd}[column sep=tiny]
        A \arrow[rr,"f"] \arrow[dr,swap,"h"] & & X \arrow[dl,"r"] \\
        & B.
      \end{tikzcd}
    \end{equation*}
    commutes, and that $f$ has a retraction if and only if $h$ has a retraction.
  \item (The \define{3-for-2 property} for equivalences.)\index{equivalence!3-for-2 property}\index{3-for-2 property!of equivalences}\index{composition!of equivalences}\index{equivalence!composition} Show that if any two of the functions
    \begin{equation*}
      f,\qquad g,\qquad h
    \end{equation*}
    are equivalences, then so is the third.
  \end{subexenum}
\exercise \label{ex:neg_equiv} 
  \begin{subexenum}
  \item Show that the negation function on the booleans $\negbool:\bool\to\bool$ defined in \cref{eg:boolean-ops} is an equivalence.\index{neg 2@{$\negbool$}!is an equivalence}\index{is an equivalence!neg bool@{$\negbool$}}
  \item Use the observational equality on the booleans\index{observational equality!on booleans!0 2 neq 1 2@{$\bfalse\neq\btrue$}}, defined in \cref{ex:obs_bool}, to show that $\bfalse\neq\btrue$.\index{0 2 neq 1 2@{$\bfalse\neq\btrue$}}
  \item Show that for any $b:\bool$, the constant function $\const_b$ is not an equivalence.\index{booleans!const b is not an equivalence@{$\const_b$ is not an equivalence}}
  \end{subexenum}
\exercise \label{ex:succ_equiv}Show that the successor function on the integers is an equivalence.\index{successor function!on Z@{on $\Z$}!is an equivalence}\index{succ Z@{$\succZ$}!is an equivalence}\index{is an equivalence!succ Z@{$\succZ$}}
\exercise \label{ex:comm_prod}Construct a equivalences $\eqv{A+B}{B+A}$ and $\eqv{A\times B}{B\times A}$.\index{coproduct!is commutative}\index{commutativity!of coproducts}
\exercise \label{ex:retr_id} Consider a section-retraction pair
  \begin{equation*}
    \begin{tikzcd}
      A \arrow[r,"i"] & B \arrow[r,"r"] & A,
    \end{tikzcd}
  \end{equation*}
  with $H:r\circ i\htpy \idfunc$. Show that $\id{x}{y}$ is a retract of $\id{i(x)}{i(y)}$.\index{retract!identity type}\index{identity type!of retract is retract}
\exercise \label{ex:sigma_assoc}\index{Sigma-type@{$\Sigma$-type}!associativity of}Let $B$ be a family of types over $A$, and let $C$ be a family of types indexed by $x:A,y:B(x)$. Construct an equivalence\index{associativity!of Sigma-types@{of $\Sigma$-types}}
  \begin{equation*}
    \assocSigma:\eqv{\Big(\sm{p:\sm{x:A}B(x)}C(\proj 1(p),\proj 2(p))\Big)}{\Big(\sm{x:A}\sm{y:B(x)}C(x,y)\Big)}.
  \end{equation*}
\exercise \label{ex:sigma_swap}Let $A$ and $B$ be types, and let $C$ be a family over $x:A,y:B$. Construct an equivalence
  \begin{equation*}
    \swapSigma:\eqv{\Big(\sm{x:A}\sm{y:B}C(x,y)\Big)}{\Big(\sm{y:B}\sm{x:A}C(x,y)\Big)}.
  \end{equation*}
  %\item \label{ex:sigma_base_equiv}Consider an equivalence $e:A'\eqv A$ and a type family $B$ over $A$. Show that the map
  %\begin{equation*}
  %\lam{(x',y)}(e(x'),y) : \Big(\sm{x':A'}B(e(x'))\Big)\to\Big(\sm{x:A}B(x)\Big)
  %\end{equation*}
  %is an equivalence.
\exercise \label{ex:int_group_laws}\index{integers!group laws} In this exercise we will show that the laws for abelian groups hold for addition on the integers. Note: these are obvious facts, but the proof terms that show \emph{how} the group laws hold are nevertheless fairly involved. This exercise is perfect for a formalization project. 
  \begin{subexenum}
  \item Show that addition satisfies the left and right unit laws, i.e., construct terms\index{unit laws!for addition on Z@{for addition on $\Z$}}
    \begin{align*}
      \leftunitlawaddZ  & : \prd{x:\Z} 0 + x = x \\
      \rightunitlawaddZ  & : \prd{x : \Z} x + 0 = x.
    \end{align*}
  \item Show that addition respects predecessors and successor on both sides, i.e., construct terms
    \begin{align*}
      \leftpredecessorlawaddZ & : \prd{x,y:\Z} \predZ(x)+y = \predZ(x+y) \\
      \rightpredecessorlawaddZ & : \prd{x,y:\Z} x+\predZ(y) = \predZ(x+y) \\
      \leftsuccessorlawaddZ & : \prd{x,y:\Z} \succZ(x)+y = \succZ(x+y) \\
      \rightsuccessorlawaddZ & : \prd{x,y:\Z} x+\succZ(y) = \succZ(x+y).
    \end{align*}
    Hint: to avoid an excessive number of cases, use induction on $x$ but not on $y$. You may need to use the homotopies $\succZ\circ \predZ\htpy \idfunc$ and $\predZ\circ\succZ$ constructed in exercise \cref{ex:succ_equiv}.
  \item Use part (b) to show that addition on the integers is associative and commutative, i.e., construct terms\index{associativity!of addition on Z@{of addition on $\Z$}}\index{commutativity!of addition on Z@{of addition on $\Z$}}
    \begin{align*}
      \assocaddZ & : \prd{x,y,z:\Z} (x+y)+z = x + (y+z) \\
      \commaddZ & : \prd{x,y:\Z} x+y = y+x.
    \end{align*}
    Hint: Especially in the construction of the associator there is a risk of running into an unwieldy amount of cases if you use $\Z$-induction on all arguments. Avoid induction on $y$ and $z$.
  \item Show that addition satisfies the left and right inverse laws:\index{inverse laws!for addition on Z@{for addition on $\Z$}}
    \begin{align*}
      \leftinverselawaddZ & : \prd{x : \Z} (-x)+x=0 \\
      \rightinverselawaddZ & : \prd{x : \Z} x + (-x)=0.
    \end{align*}
    Conclude that the functions $y \mapsto x + y$ and $x\mapsto x + y$ are equivalences for any $x:\Z$ and $y:\Z$, respectively.
  \end{subexenum}
\exercise \label{ex:coproduct_functor}In this exercise we will construct the \define{functorial action}\index{functorial action!of coproducts}\index{coproduct!functorial action} of coproducts.
  \begin{subexenum}
  \item Construct for any two maps $f:A \to A'$ and $g:B \to B'$, a map\index{f + g@{$f+g$}|see {functorial action, of coproducts}}
    \begin{equation*}
      f+g:A+B \to A'+B'.
    \end{equation*}
  \item Show that if $H:f\htpy f'$ and $K:g\htpy g'$, then there is a homotopy
    \begin{equation*}
      H+K:(f+g)\htpy (f'+g').
    \end{equation*}
  \item Show that $\idfunc[A]+\idfunc[B]\htpy \idfunc[A+B]$.
  \item Show that for any
    \begin{equation*}
      \begin{tikzcd}[row sep=small]
        A \arrow[r,"f"] & {A'} \arrow[r,"{f'}"] & {A''} \\
        B \arrow[r,"g"] & {B'} \arrow[r,"{g'}"] & {B''}
      \end{tikzcd}
    \end{equation*}
    there is a homotopy
    \begin{equation*}
      (f'\circ f)+(g'\circ g) \htpy (f'+g')\circ (f\circ g).
    \end{equation*}
  \item \label{ex:coproduct_functor_equivalence}Show that if $f$ and $g$ are equivalences, then so is $f+g$. (The converse of this statement also holds, see \cref{ex:is-equiv-is-equiv-functor-coprod}.)
  \end{subexenum}
\exercise Construct equivalences
  \begin{align*}
    \Fin(m+n) & \simeq \Fin(m)+\Fin(n) \\
    \Fin(mn) & \simeq \Fin(m)\times\Fin(n).
  \end{align*}
\end{exercises}
\index{equivalence|)}
