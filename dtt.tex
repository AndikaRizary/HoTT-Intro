\chapter{Dependent type theory}
\label{ch:dtt}

\section{The primitive judgments of type theory}

The theory of type dependency is formulated as a deductive system in which derivations establish that a given construction is well-formed. In any dependent type theory there are four \define{primitive judgments}\index{primitive judgment}:
\begin{enumerate}
%\item `\emph{$\Gamma$ is a well-formed context.}'
\item `\emph{$A$ is a well-formed \define{type}\index{well-formed type}\index{type} in context $\Gamma$.}'\index{primitive judgment!type in context}
\item `\emph{$A$ and $B$ are \define{judgmentally equal types} in context $\Gamma$.}'\index{primitive judgment!equal types in context}\index{judgmental equality!of types}
\item `\emph{$a$ is a well-formed \define{term}\index{well-formed term}\index{term} of type $A$ in context $\Gamma$.}'\index{primitive judgment!term of a type in context}
\item `\emph{$a$ and $b$ are \define{judgmentally equal terms} of type $A$ in context $\Gamma$.}'\index{primitive judgment!equal terms of a type in context}\index{judgmental equality!of terms}
\end{enumerate}
\begin{samepage}
The symbolic expressions for these four primitive judgments are as follows:
\begin{align*}
%& \vdash \Gamma~\mathrm{ctx} & & \vdash \Gamma\jdeq \Gamma'~\mathrm{ctx} \\
\Gamma & \vdash A~\textrm{type} & \Gamma & \vdash A\jdeq B~\textrm{type}\\
\Gamma & \vdash a:A & \Gamma & \vdash a\jdeq b:A.
\end{align*}
\end{samepage}
A \define{context}\index{context|textbf} is an expression of the form
\begin{equation*}
x_1:A_1,~x_2:A_2(x_1),~\ldots,~x_n:A_n(x_1,\ldots,x_{n-1}),
\end{equation*}
which we often simply write as $x_1:A_1,~x_2:A_2,~\ldots,~x_n:A_n$,
satisfying the condition that for each $1\leq k\leq n$ we have that $A_k$ is a well-formed type in context $x_1:A_1,x_2:A_2,\ldots,x_{k-1}:A_{k-1}$, i.e.
\begin{equation*}
x_1:A_1,x_2:A_2,\ldots,x_{k-1}:A_{k-1} \vdash A_k~\textrm{type}.
\end{equation*}
We say that a context $x_1:A_1,~\ldots,~x_n:A_n$ \define{declares the variables}\index{variable declaration} $x_1,\ldots,x_n$. 
We may use variable names other than $x_1,\ldots,x_n$, as long as \emph{no variable is declared more than once.} 
%For example we used the variable names $A,\mu,u_l,p,u_r,q$ when we displayed the context of \autoref{lem:unit}.

In the special case where $n=0$, the list $x_1:A_1,x_2:A_2,\ldots,x_n:A_n$ is empty, which satisfies the well-formedness condition vacuously. In other words, the \define{empty context}\index{context!empty context|textbf}\index{empty context|textbf} is well-formed. A well-formed type in the empty context is also called a \define{closed type}\index{closed type|textbf}, and a well-formed term of a closed type is called a \define{closed term}\index{closed term|textbf}.

When $B$ is a type in context $\Gamma,x:A$, we also say that $B$ is a \define{family of types}\index{family!of types|textbf} over $A$ (in context $\Gamma$).

\section{Renaming variables}
In some situations one might want to change the name of a variable in a context. This is allowed, provided that the new variable does not occur anywhere else in the context, so that also after renaming no variable is declared more than once. 
The inference rules that rename a variable are sometimes called $\alpha$-conversion\index{alpha-conversion@{$\alpha$-conversion}}\index{conversion rules!alpha-conversion@{$\alpha$-conversion}}\index{rule!alpha conversion@{$\alpha$-conversion}} rules. 

If we are given a type $A$ in context $\Gamma$, then for any type $B$ in context $\Gamma,x:A,\Delta$ we can form the type $B[x'/x]$ in context $\Gamma,x':A,\Delta[x'/x]$, where $B[x'/x]$ is an abbreviation for
\begin{equation*}
B(x_1,\ldots,x_{n-1},x',x_{n+1},\ldots,x_{n+m-1})
\end{equation*}
This definition of \define{renaming}\index{variable renaming|textbf} the variable $x$ by $x'$ is understood to be recursive over the length of $\Delta$. The first variable renaming rule\index{rule!variable renaming} postulates that the renaming of a variable preserves well-formedness of types:
\begin{prooftree}
\AxiomC{$\Gamma,x:A,\Delta\vdash B~\mathrm{type}$}
\RightLabel{$x'/x$}
\UnaryInfC{$\Gamma,x':A,\Delta[x'/x]\vdash B[x'/x]~\mathrm{type}$}
\end{prooftree}

Similarly we obtain for any term $b:B$ in context $\Gamma,x:A,\Delta$ a term $b[x'/x]:B[x'/x]$, and there is a variable renaming rule postulating that the renaming of a variable preserves the well-formedness of terms.
In fact, there is variable renaming rule for each of the primitive judgments. To avoid having to state essentially the same rule four times in a row, we postulate the four variable renaming rules all at once using a \emph{generic judgment}\index{generic judgment} $\mathcal{J}$. 
\begin{prooftree}
\AxiomC{$\Gamma,x:A,\Delta\vdash \mathcal{J}$}
\RightLabel{$x'/x$}
\UnaryInfC{$\Gamma,x':A,\Delta[x'/x]\vdash \mathcal{J}[x'/x]$}
\end{prooftree}
where $\mathcal{J}$ may be a typing judgment, a judgment of equality of types, a term judgment, or a judgment of equality of terms.
We will use generic judgments extensively to postulate the rest of the rules of dependent type theory.

\section{Inference rules governing judgmental equality}

\begin{samepage}
Both on types and on terms, we postulate that judgmental equality is an equivalence relation. That is, we provide inference rules for the reflexivity, symmetry and transitivity of both kinds of judgmental equality\index{judgmental equality!equivalence relation}:
\begin{center}
\begin{small}
\begin{minipage}{.2\textwidth}
\begin{prooftree}
\AxiomC{$\Gamma\vdash A~\textrm{type}$}
\UnaryInfC{$\Gamma\vdash A\jdeq A~\textrm{type}$}
\end{prooftree}
\end{minipage}
\begin{minipage}{.25\textwidth}
\begin{prooftree}
\AxiomC{$\Gamma\vdash A\jdeq A'~\textrm{type}$}
\UnaryInfC{$\Gamma\vdash A'\jdeq A~\textrm{type}$}
\end{prooftree}
\end{minipage}
\begin{minipage}{.5\textwidth}
\begin{prooftree}
\AxiomC{$\Gamma\vdash A\jdeq A'~\textrm{type}$}
\AxiomC{$\Gamma\vdash A'\jdeq A''~\textrm{type}$}
\BinaryInfC{$\Gamma\vdash A\jdeq A''~\textrm{type}$}
\end{prooftree}
\end{minipage}
\\
\bigskip
\begin{minipage}{.2\textwidth}
\begin{prooftree}
\AxiomC{$\Gamma\vdash a:A$}
\UnaryInfC{$\Gamma\vdash a\jdeq a : A$}
\end{prooftree}
\end{minipage}
\begin{minipage}{.25\textwidth}
\begin{prooftree}
\AxiomC{$\Gamma\vdash a\jdeq a':A$}
\UnaryInfC{$\Gamma\vdash a'\jdeq a: A$}
\end{prooftree}
\end{minipage}
\begin{minipage}{.5\textwidth}
\begin{prooftree}
\AxiomC{$\Gamma\vdash a\jdeq a' : A$}
\AxiomC{$\Gamma\vdash a'\jdeq a'': A$}
\BinaryInfC{$\Gamma\vdash a\jdeq a'': A$}
\end{prooftree}
\end{minipage}
\end{small}
\end{center}
\end{samepage}

Apart from the rules postulating that judgmental equality is an equivalence relation, there are also \define{variable conversion rules}\index{judgmental equality!conversion rules}\index{variable conversion rules}\index{conversion rule!variable}\index{rule!variable conversion}.
Informally, these are rules stating that if $A$ and $A'$ are judgmentally equal types in context $\Gamma$, then any valid judgment in context $\Gamma,x:A$ is also a valid judgment in context $\Gamma,x:A'$. In other words: we can convert the type of a variable to a judgmentally equal type. We state this with a generic judgment $\mathcal{J}$
\begin{prooftree}
\AxiomC{$\Gamma\vdash A\jdeq A'~\textrm{type}$}
\AxiomC{$\Gamma,x:A,\Delta\vdash \mathcal{J}$}
\RightLabel{$A'/A$}
\BinaryInfC{$\Gamma,x:A',\Delta\vdash \mathcal{J}$}
\end{prooftree}
An analogous \emph{term conversion rule}\index{term conversion rule}\index{conversion rule!term}\index{rule!term conversion} stated in \cref{ex:term_conversion}, converting the type of a term to a judgmentally equal type, is derivable.


\section{Structural rules of type theory}

We complete the specification of dependent type theory by postulating rules for \emph{weakening} and \emph{substitution}, and the \emph{variable rule}:
\begin{enumerate}
\item If we are given a type $A$ in context $\Gamma$, then any judgment made in a longer context $\Gamma,\Delta$ can also be made in the context $\Gamma,x:A,\Delta$, for a fresh variable $x$. The \define{weakening rule}\index{weakening}\index{rule!weakening} asserts that weakening by a type $A$ in context preserves well-formedness and judgmental equality of types and terms.
\begin{prooftree}
\AxiomC{$\Gamma\vdash A~\textrm{type}$}
\AxiomC{$\Gamma,\Delta\vdash \mathcal{J}$}
\RightLabel{$W_A$}
\BinaryInfC{$\Gamma,x:A,\Delta \vdash \mathcal{J}$}
\end{prooftree}
This process of expanding the context by a fresh variable of type $A$ is called \define{weakening (by $A$)}. The type family $W_A(B)$ over $A$ is also called the \define{constant family}\index{family!constant family}\index{constant family} $B$, or the \define{trivial family}\index{family!trivial family}\index{trivial family} $B$.
\item If we are given a type $A$ in context $\Gamma$, then $x$ is a well-formed term of type $A$ in context $\Gamma,x:A$.
\begin{prooftree}
\AxiomC{$\Gamma\vdash A~\textrm{type}$}
\RightLabel{$\delta_A$}
\UnaryInfC{$\Gamma,x:A\vdash x:A$}
\end{prooftree}
This is called the \define{variable rule}\index{variable rule}\index{rule!variable rule|textbf}. It provides an \emph{identity function}\index{identity function} on the type $A$ in context $\Gamma$.
\item If we are given a term $a:A$ in context $\Gamma$, then for any type $B$ in context $\Gamma,x:A,\Delta$ we can form the type $B[a/x]$ in context $\Gamma,\Delta[a/x]$, where $B[a/x]$ is an abbreviation for
\begin{equation*}
B(x_1,\ldots,x_{n-1},a(x_1,\ldots,x_{n-1}),x_{n+1},\ldots,x_{n+m-1})
\end{equation*}
This definition of substituting $a$ for $x$ is understood to be recursive over the length of $\Delta$. Similarly we obtain for any term $b:B$ in context $\Gamma,x:A,\Delta$ a term $b[a/x]:B[a/x]$. The \define{substitution rule}\index{substitution}\index{rule!substitution} asserts that substitution preserves well-formedness and judgmental equality of types and terms:
\begin{prooftree}
\AxiomC{$\Gamma\vdash a:A$}
\AxiomC{$\Gamma,x:A,\Delta\vdash \mathcal{J}$}
\RightLabel{$S_a$}
\BinaryInfC{$\Gamma,\Delta[a/x]\vdash \mathcal{J}[a/x]$}
\end{prooftree}
Furthermore, we postulate that substitution by judgmentally equal terms results in judgmentally equal types
\begin{prooftree}
\AxiomC{$\Gamma\vdash a\jdeq a':A$}
\AxiomC{$\Gamma,x:A,\Delta\vdash B~\mathrm{type}$}
\BinaryInfC{$\Gamma,\Delta[a/x]\vdash B[a/x]\jdeq B[a'/x]~\mathrm{type}$}
\end{prooftree}
and it also results in judgmentally equal terms
\begin{prooftree}
\AxiomC{$\Gamma\vdash a\jdeq a':A$}
\AxiomC{$\Gamma,x:A,\Delta\vdash b:B$}
\BinaryInfC{$\Gamma,\Delta[a/x]\vdash b[a/x]\jdeq b[a'/x]:B[a/x]$}
\end{prooftree}
When $B$ is a family of types over $A$ and $a:A$, we also say that $B[a/x]$ is the \define{fiber}\index{family!fiber of}\index{fiber!of a family} of $B$ at $a$. Often we write $B(a)$ for $B[a/x]$.
\end{enumerate}




\begin{comment}
\bigskip
\begin{minipage}{.45\textwidth}
\begin{prooftree}
\AxiomC{$\Gamma\vdash A~\textrm{type}$}
\AxiomC{$\Gamma,\Delta\vdash B~\textrm{type}$}
\RightLabel{$W_A$}
\BinaryInfC{$\Gamma,x:A,\Delta \vdash B~\textrm{type}$}
\end{prooftree}
\end{minipage}\hfill
\begin{minipage}{.45\textwidth}
\begin{prooftree}
\AxiomC{$\Gamma\vdash A~\textrm{type}$}
\AxiomC{$\Gamma,\Delta\vdash b:B$}
\RightLabel{$W_A$}
\BinaryInfC{$\Gamma,x:A,\Delta \vdash b:B$}
\end{prooftree}
\end{minipage}

\noindent
\begin{prooftree}
\AxiomC{$\Gamma\vdash A~\textrm{type}$}
\RightLabel{$\delta_A$}
\UnaryInfC{$\Gamma,x:A\vdash x:A$}
\end{prooftree}

\noindent
\begin{minipage}{.5\textwidth}
\begin{prooftree}
\AxiomC{$\Gamma\vdash a:A$}
\AxiomC{$\Gamma,x:A,\Delta\vdash B~\textrm{type}$}
\RightLabel{$S_a$}
\BinaryInfC{$\Gamma,\Delta[a/x]\vdash B[a/x]~\textrm{type}$}
\end{prooftree}
\end{minipage}\hfill
\begin{minipage}{.5\textwidth}
\begin{prooftree}
\AxiomC{$\Gamma\vdash a:A$}
\AxiomC{$\Gamma,x:A,\Delta\vdash b:B$}
\RightLabel{$S_a$}
\BinaryInfC{$\Gamma,\Delta[a/x]\vdash b[a/x] : B[a/x]$}
\end{prooftree}
\end{minipage}

\bigskip
\end{comment}


\begin{eg}
To give an example of how the deductive system works, we give a deduction for the \define{interchange rule}\index{rule!interchange}\index{interchange rule}
\begin{prooftree}
\AxiomC{$\Gamma\vdash B~\textrm{type}$}
\AxiomC{$\Gamma,x:A,y:B,\Delta\vdash \mathcal{J}$}
\BinaryInfC{$\Gamma,y:B,x:A,\Delta\vdash \mathcal{J}$}
\end{prooftree}
In other words, if we have two types $A$ and $B$ in context $\Gamma$, and we make a judgment in context $\Gamma,x:A,y:B$, then we can make that same judgment in context $\Gamma,y:B,x:A$.
The derivation is as follows:
\begin{small}
\begin{prooftree}
\AxiomC{$\Gamma\vdash B~\textrm{type}$}
\RightLabel{$\delta_B$}
\UnaryInfC{$\Gamma,y:B\vdash y:B$}
\RightLabel{$W_{W_B(A)}$}
\UnaryInfC{$\Gamma,y:B,x:A\vdash y:B$}
\AxiomC{$\Gamma,x:A,y:B,\Delta\vdash \mathcal{J}$}
\RightLabel{$y'/y$}
\UnaryInfC{$\Gamma,x:A,y':B,\Delta[y'/y]\vdash \mathcal{J}[y'/y]$}
\RightLabel{$W_B$}
\UnaryInfC{$\Gamma,y:B,x:A,y':B,\Delta[y'/y]\vdash \mathcal{J}[y'/y]$}
\RightLabel{$S_{W_A(y)}$}
\BinaryInfC{$\Gamma,y:B,x:A,\Delta\vdash \mathcal{J}$}
\end{prooftree}
\end{small}
\end{eg}


\begin{comment}
For $A\in T_n$ we define $T_{n+k+1}(A):= \{B\in T_{n+k}\mid \mathrm{ft}^{k+1}(B)=A\}$. Similarly we define $\tilde{T}_{n+k+1}(A):=\{b\in\tilde{T}_{n+k+1}\mid\mathrm{ft}^{k+1}(\partial(b))=A\}$. For any closed type $A$ we maintain the convention that $T_{k}(\mathrm{ft}(A)):= T_k$.
\begin{enumerate}
\item For all $A\in T_n$, and all $k\in\N$, \define{weakening} operations
\begin{align*}
W_A & : T_{n+k}(\mathrm{ft}(A)) \to T_{n+k+1}(A) \\
\tilde{W}_A & : \tilde{T}_{n+k}(\mathrm{ft}(A))\to \tilde{T}_{n+k+1}(A)
\end{align*}
subject to the conditions $\mathrm{ft}(W_A(B))=W_A(\mathrm{ft}(B))$ if $B\in T_{n+k}$ with $k\geq 1$, and $\partial(\tilde{W}_A(b))=W_A(\partial(b))$.
\item For all $A\in T_n$ a term $\delta_A\in \tilde{T}_{n+1}$ satisfying $\partial(\delta_A)=W_A(A)$. 
\item For all $a\in \tilde{T}_n$ satisfying $\partial(a)=A$, and all $k\in\N$, \define{substitution} operations
\begin{align*}
S_a & : \{B\in T_{n+k+1}\mid \mathrm{ft}^{k+1}(B)= A\}\to T_k \\
\tilde{S}_a & : \{b\in \tilde{T}_{n+k+1}\mid \mathrm{ft}^{k+1}(\partial(b))=A\}\to \tilde{T}_{n+k}
\end{align*}
subject to the conditions $\mathrm{ft}(S_a(B))=\mathrm{ft}(A)$ if $B\in T_{n+1}$, $\mathrm{ft}(S_a(B))=S_a(\mathrm{ft}(B))$ if $B\in T_{n+k+1}$ with $k\geq 1$, and $\partial(\tilde{S}_a(b))=S_a(\partial(b))$.
\end{enumerate}
\end{comment}

%\section{Axioms for weakening, substitution, and the diagonal}
\begin{comment}
\begin{prooftree}
\AxiomC{$\Gamma\vdash A~\textrm{type}$}
  \AxiomC{$\Gamma,x:A,\Delta\vdash B~\textrm{type}$}
    \AxiomC{$\Gamma,x:A,\Delta,y:B,E\vdash C~\textrm{type}$}
\TrinaryInfC{$\Gamma,\Delta[a/x],E[b/y][a/x]\vdash C[b/y][a/x]\jdeq C[a/x][b[a/x]/y']~\textrm{type}$}
\end{prooftree}
\begin{prooftree}
\AxiomC{$\Gamma\vdash A~\textrm{type}$}
  \AxiomC{$\Gamma,x:A,\Delta\vdash B~\textrm{type}$}
    \AxiomC{$\Gamma,x:A,\Delta,y:B,E\vdash c:C$}
\TrinaryInfC{$\Gamma,\Delta[a/x],E[b/y][a/x]\vdash c[b/y][a/x]\jdeq c[a/x][b[a/x]/y']:C[b/y][a/x]$}
\end{prooftree}
\begin{prooftree}
\AxiomC{$\Gamma\vdash a:A$}
  \AxiomC{$\Gamma,\Delta\vdash B~\textrm{type}$}
\RightLabel{($x$ not free in $B$)}
\BinaryInfC{$\Gamma,\Delta\vdash B[a/x]\jdeq B~\textrm{type}$}
\end{prooftree}
\end{comment}


\begin{comment}
\section{An algebraic presentation of dependent type theory}

%Let us write $T_n$ for the set of well-formed contexts of length $n$, for $n>1$. Then any well-formed context of length $n+1$ is of the form $\Gamma,x:A$, where $\Gamma$ is a well-formed context of length $n$. Thus we see that there are maps $\eft:T_{n+1}\to T_n$ for $n>1$. Similarly, if we write $\tilde{T}_n$ for the set of all terms of a type in a context of length $n-1$, we see that there is a map $\tilde{T}_n\to T_n$. The following picture emerges:
%\begin{equation*}
%\begin{tikzcd}
%\tilde{T}_3 \arrow[dr,"\partial"] & \vdots \arrow[d,"\mathrm{ft}"] \\
%\tilde{T}_2 \arrow[dr,"\partial"] & T_3 \arrow[d,"\mathrm{ft}"] \\
%\tilde{T}_1 \arrow[dr,"\partial"] & T_2 \arrow[d,"\mathrm{ft}"] \\
%& T_1
%\end{tikzcd}
%\end{equation*}

Observe that given a type $A$ in context $\Gamma$ and a type $B$ in context $\Gamma,\Delta$ we can weaken twice by first weakening by $B$ and then by $A$, as indicated in the following derivation:
\begin{prooftree}
\AxiomC{$\Gamma\vdash A~\textrm{type}$}
\AxiomC{$\Gamma,\Delta\vdash B~\textrm{type}$}
  \AxiomC{$\Gamma,\Delta,\mathrm{E}\vdash \mathcal{J}$}
\BinaryInfC{$\Gamma,\Delta,y:B,\mathrm{E}\vdash \mathcal{J}$}
\BinaryInfC{$\Gamma,x:A,\Delta,y:B,\mathrm{E}\vdash \mathcal{J}$}
\end{prooftree}
However, we can also first weaken by $A$, and then by `$B$ weakened by $A$', as indicated in the following derivation:
\begin{prooftree}
\AxiomC{$\Gamma\vdash A~\textrm{type}$}
  \AxiomC{$\Gamma,\Delta\vdash B~\textrm{type}$}
\BinaryInfC{$\Gamma,x:A,\Delta\vdash B~\textrm{type}$}
  \AxiomC{$\Gamma\vdash A~\textrm{type}$}
    \AxiomC{$\Gamma,\Delta,\mathrm{E}\vdash \mathcal{J}$}
  \BinaryInfC{$\Gamma,x:A,\Delta,\mathrm{E}\vdash \mathcal{J}$}
\BinaryInfC{$\Gamma,x:A,\Delta,y:B,\mathrm{E}\vdash \mathcal{J}$}
\end{prooftree}
For the end result it does not matter in what order the two weakenings are performed. We can express this conveniently as an equation:
\begin{equation*}
W_A(W_B(\mathcal{J}))\jdeq W_{W_A(B)}(W_A(\mathcal{J})).
\end{equation*}
The complete list of rules (in alphabetic order) is
\begin{align*}
S_a(\delta_B) & \jdeq \delta_{S_a(B)} \\
S_a(\delta_A) & \jdeq a \\
S_a(S_b(\mathcal{J})) & \jdeq S_{S_a(b)}(S_a(\mathcal{J})) \\
S_a(W_A(\mathcal{J})) & \jdeq \mathcal{J} \\
S_a(W_B(\mathcal{J})) & \jdeq W_{S_a(B)}(S_a(\mathcal{J})) \\
S_{\delta_A}(W_{W_A}(\mathcal{J})) & \jdeq \mathcal{J} \\
W_A(\delta_B) & \jdeq \delta_{W_A(B)} \\
W_A(S_b(\mathcal{J})) & \jdeq S_{W_A(b)}(W_A(\mathcal{J})) \\
W_A(W_B(\mathcal{J})) & \jdeq W_{W_A(B)}(W_A(\mathcal{J}))
\end{align*}
Here $A$ is a type in context $\Gamma$ and $a$ is a term of type $A$, $B$ is a type in context $\Gamma,x:A,\Delta$ and $b$ is a term of type $B$.
\end{comment}

%\begin{rmk}
%To avoid overly long proof trees maintain as a convention that every derivation with hypotheses $\mathcal{H}_1,\ldots,\mathcal{H}_n$ and conclusion $\mathcal{C}$ can be used as a rule
%\begin{prooftree}
%\AxiomC{$\mathcal{H}_1$}
%\AxiomC{$\cdots$}
%\AxiomC{$\mathcal{H}_n$}
%\TrinaryInfC{$\mathcal{C}$}
%\end{prooftree}
%in later derivations.
%\end{rmk}

\begin{exercises}
\item \label{ex:term_conversion}Give a derivation for the following conversion rule\index{term conversion rule}\index{term conversion rule}\index{rule!term conversion}\index{term conversion rule}\index{conversion rule!term}:
\begin{prooftree}
\AxiomC{$\Gamma\vdash A\jdeq A'~\textrm{type}$}
\AxiomC{$\Gamma\vdash a:A$}
\BinaryInfC{$\Gamma\vdash a:A'$}
\end{prooftree}
\begin{comment}
\item Consider a type $A$ in context $\Gamma$. In this exercise we establish a connection between types in context $\Gamma,x:A$, and uniform choices of types $B_a$, where $a$ ranges over terms of $A$ in a uniform way. A similar connection is made for terms.
\begin{subexenum}
\item We define a \define{uniform family} over $A$ to consist of a type
\begin{equation*}
\Delta,\Gamma\vdash B_a~\mathrm{type}
\end{equation*}
for every context $\Delta$, and every term $\Delta,\Gamma\vdash a:A$, subject to the condition that one can derive
\begin{prooftree}
\AxiomC{$\Delta\vdash d:D$}
\AxiomC{$\Delta,y:D,\Gamma\vdash a:A$}
\BinaryInfC{$\Delta,\Gamma\vdash B_a[d/y]\jdeq B_{a[d/y]}~\mathrm{type}$}
\end{prooftree}
Define a bijection between types in context $\Gamma,x:A$ and uniform families over $A$. 
\item Consider a type $\Gamma,x:A\vdash B$. We define a \define{uniform term} of $B$ over $A$ to consist of a type
\begin{equation*}
\Delta,\Gamma\vdash b_a:B[a/x]~\mathrm{type}
\end{equation*}
for every context $\Delta$, and every term $\Delta,\Gamma\vdash a:A$, subject to the condition that one can derive
\begin{prooftree}
\AxiomC{$\Delta\vdash d:D$}
\AxiomC{$\Delta,y:D,\Gamma\vdash a:A$}
\BinaryInfC{$\Delta,\Gamma\vdash b_a[d/y]\jdeq b_{a[d/y]}:B[a/x][d/y]$}
\end{prooftree}
Define a bijection between terms of $B$ in context $\Gamma,x:A$ and uniform terms of $B$ over $A$. 
\end{subexenum}
\end{comment}
\end{exercises}
