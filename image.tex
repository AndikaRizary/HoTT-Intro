\chapter{The homotopy image of a map}\label{chap:image}
\section{The universal property of the image of a map}

\begin{defn}
Let $f:A\to X$ and $g:B\to X$ be maps. We define
\begin{equation*}
\mathrm{hom}_X(f,g)\defeq\sm{h:A\to B}f\htpy g\circ h.
\end{equation*}
\end{defn}

\begin{rmk}
The type $\mathrm{hom}_X(f,g)$ is equivalent to the type
\begin{equation*}
\prd{x:X}\fib{f}{x}\to\fib{g}{x}.
\end{equation*}
\end{rmk}

\begin{lem}
For any $f:A\to X$ and any embedding $m:B\to X$, the type $\mathrm{hom}_X(f,m)$ is a mere proposition.
\end{lem}

\begin{defn}
Consider a commuting triangle
\begin{equation*}
\begin{tikzcd}[column sep=small]
A \arrow[rr,"i"] \arrow[dr,swap,"f"] & & B \arrow[dl,"m"] \\
& X
\end{tikzcd}
\end{equation*}
with $I:f\htpy m\circ i$, and where $m$ is an embedding.
We say that $m$ has the \define{universal property of the image of $f$} if the map
\begin{equation*}
(i,I)^\ast : \mathrm{hom}_X(m,m')\to\mathrm{hom}_X(f,m')
\end{equation*}
is an equivalence for every embedding $m':B'\to X$. 
\end{defn}

\begin{rmk}
Since $\mathrm{hom}_X(f,m)$ is a mere proposition for every $f:A\to X$ and every embedding $m:B\to X$, it follows that the universal property of the image of $f$ is equivalent to the property that the implication
\begin{equation*}
\mathrm{hom}_X(f,m')\to\mathrm{hom}_X(m,m')
\end{equation*}
holds for every embedding $m':B'\to X$. 
\end{rmk}

\section{The join construction}
\section{Univalent logic}
\begin{table}
\caption{Logic in type theory}
\begin{center}
\begin{tabular}{ll}
\toprule
\emph{Logical connective} & \emph{Interpretation in HoTT} \\
\midrule
$\top$ & $\unit$ \\
$\bot$ & $\emptyt$ \\
$P\land Q$ & $P\times Q$ \\
$P\lor Q$ & $\brck{P+Q}$ \\
$P\to Q$ & $P\to Q$ \\
$P\leftrightarrow Q$ & $P=Q$ \\
$\neg P$ & $P\to\emptyt$ \\
$\forall x.P(x)$ & $\prd{x:A}P(x)$ \\
$\exists x.P(x)$ & $\brck{\sm{x:A}P(x)}$ \\
$\exists! x.P(x)$ & $\iscontr(\sm{x:A}P(x))$ \\
\bottomrule
\end{tabular}
\end{center}
\end{table}

\begin{exercises}
\item Show that
\begin{equation*}
\eqv{\brck{A}}{\prd{P:\prop}(A\to P)\to P}.
\end{equation*}
\item For any $B:A\to\UU$, construct an equivalence
\begin{equation*}
\eqv{\Big(\exists a.\brck{B(a)}\Big)}{\Big(\brck{\sm{a:A}B(a)}\Big)}
\end{equation*}
\item \label{also}(Mart\'in Escard\'o) For any two propositions $P$ and $Q$, define
\begin{equation*}
P\boxplus Q \defeq ((P\to Q)\to Q)\times ((Q\to P)\to P).
\end{equation*}
\begin{subexenum}
\item Show that $P\lor Q\to P\boxplus Q$ and $P\boxplus Q\to\neg(\neg P\land \neg Q)$.
\end{subexenum}
\item Show that for any mere proposition $Q$, and any type $X$, the following are equivalent:
\begin{enumerate}
\item The map $(Q\to X)\to(\emptyt\to X)$ is an equivalence.
\item The type $X^Q$ is contractible.
\item $Q\to\iscontr(X)$.
\end{enumerate}
\item \label{ex:brck_comp} Formulate the computation rule corresponding to the path constructor $\mu$. That is, compute the type of $\apd{\rec{\brck{\blank}}(f,g)}{\mu(x,y)}$, and find a canonical element in it.
\item Let
\begin{tikzcd}
P_0 \arrow[r] & P_1 \arrow[r] & P_2 \arrow[r] & \cdots
\end{tikzcd}
be a sequence of propositions. Show that
\begin{equation*}
\colim_n(P_n)=\exists_n P_n.
\end{equation*}
\end{exercises}
