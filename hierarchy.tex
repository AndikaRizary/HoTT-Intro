\chapter{The hierarchy of homotopical complexity}
\chaptermark{Homotopical complexity}
%Not all types have interesting higher groupoid structure. For example, we will see below that two natural numbers can only be equal in at most one way. Voevodsky articulated a useful notion to detect the homotopical complexity of types, which allows us to distinguish between contractible types (also called \emph{$(-2)$-types}), \emph{propositions} (also called \emph{$(-1)$-types}), \emph{sets} (\emph{$0$-types}), and \emph{$k$-types} for higher $k$.

%We will see [later] that there are types that are not $k$-types for any $k$.

\section{Propositions and subtypes}

\begin{defn}
A type $A$ is said to be a \define{proposition} if there is a term of type
\begin{equation*}
\isprop(A)\defeq\prd{x,y:A}\iscontr(x=y).
\end{equation*}
\end{defn}

\begin{eg}\label{eg:prop_contr}
Any contractible type is a proposition by \cref{ex:prop_contr}. However, propositions do not need to be inhabited: the empty type is also a proposition, since
\begin{equation*}
\prd{x,y:\emptyt}\iscontr(x=y)
\end{equation*}
follows from the induction principle of the empty type.
\end{eg}

In the following lemma we prove that in order to show that a type $A$ is a proposition, it suffices to show that any two terms of $A$ are equal. In other words, propositions are types with \define{proof irrelevance}.

\begin{lem}\label{lem:isprop_eq}
Let $A$ be a type. Then we have
\begin{equation*}
\isprop(A)\leftrightarrow \Big(\prd{x,y:A}x=y\Big).
\end{equation*}
\end{lem}

\begin{proof}
Suppose $A$ is a proposition. By taking the center of contraction of $\id{x}{y}$ for each $x,y:A$ we obtain a term of type $\prd{x,y:A}\id{x}{y}$.

Now suppose that $A$ is a type equipped with $H:\prd{x,y:A}\id{x}{y}$. Then we take $\ct{H(x,x)^{-1}}{H(x,y)}$ as the center of contraction of $\id{x}{y}$. To construct the contraction
\begin{equation*}
\prd{p:\id{x}{y}} \ct{H(x,x)^{-1}}{H(x,y)}=p
\end{equation*}
we proceed by path induction. Our goal is to show that
\begin{equation*}
\ct{H(x,x)^{-1}}{H(x,x)}=\refl{x}.\qedhere
\end{equation*}
\end{proof}

By proof irrelevance it follows that propositions are contractible as soon as they are inhabited.

\begin{cor}\label{cor:contr_prop}
For any proposition $P$ we have $P\to\iscontr(P)$.
\end{cor}

In type theory terms always come equipped with their types, i.e.~they never appear in isolation. This is useful from the perspective that terms are programs with a certain specification, but as a consequence we cannot consider subtypes in the same way as set theorists have subsets. Our definition of subtype is therefore considerably different:

\begin{defn}
A type family $B$ over $A$ is said to be a \define{subtype} of $A$ if for each $x:A$ the type $B(x)$ is a proposition.
\end{defn}

We will show in \cref{thm:subtype} that a type family $B$ over $A$ is a subtype of $A$ if and only if the projection map $\proj 1:\big(\sm{x:A}B(x)\big)\to A$ is an embedding.

\begin{comment}
\begin{samepage}
\begin{thm}\label{thm:subtype}
Let $B$ be a type family over $A$. The following are equivalent:
\begin{enumerate}
\item The family $B$ over $A$ is a \define{subtype} of $A$, in the sense that for each $x:A$ the type $B(x)$ is a proposition.
\item The projection map
\begin{equation*}
\proj 1 : \Big(\sm{x:A}B(x)\Big)\to A
\end{equation*}
is an embedding. 
\end{enumerate}
\end{thm}
\end{samepage}

\begin{proof}
First assume that $B(x)$ is a proposition for each $x:A$. Our goal is to show that
\begin{equation*}
\apfunc{\proj 1} : (\id{s}{t})\to (\id{\proj 1(s)}{\proj 1(t)})
\end{equation*}
is an equivalence for every $s,t:\sm{x:A}B(x)$. By $\Sigma$-induction on $s$ and \autoref{thm:id_fundamental} it suffices to show that the type
\begin{equation*}
\sm{t:\sm{x:A}B(x)} \id{a}{\proj 1(t)}
\end{equation*}
is contractible, for any $a:A$ and $b:B(a)$. 
For the center of contraction we take $\pairr{\pairr{a,b},\refl{a}}$. 
The contraction is constructed by applying $\Sigma$-induction twice, by which it suffices to construct a term of type
\begin{equation*}
\prd{x:A}{y:B(x)}{p:\id{a}{x}} \pairr{\pairr{a,b},\refl{a}}=\pairr{\pairr{x,y},p}.
\end{equation*}
This term is constructed by path induction on $p$, so it suffices to construct a term of type
\begin{equation*}
\prd{y:B(a)} \pairr{\pairr{a,b},\refl{a}}=\pairr{\pairr{a,y},\refl{a}}
\end{equation*}
However, the proposition $B(a)$ is contractible by \cref{cor:contr_prop}, since we have $b:B(a)$. Therefore we may proceed by singleton induction, so it suffices to construct an identification of type
\begin{equation*}
\pairr{\pairr{a,b},\refl{a}}=\pairr{\pairr{a,b},\refl{a}},
\end{equation*}
which we have by reflexivity. This completes the proof that if each $B(x)$ is a proposition, then the projection map $\proj 1 : \big(\sm{x:A}B(x)\big)\to A$ is an embedding.

For the converse, assume that the projection map is an embedding, and let $x:A$. Our goal is to show that $B(x)$ is a proposition. By \cref{lem:isprop_eq} it suffices to show that
\begin{equation*}
\prd{x:A}{y,z:B(x)} \id{y}{z}
\end{equation*}
Let $y,z:B(x)$. By our assumption that the projection map is an embedding we have an equivalence
\begin{equation*}
\eqv{(\id{\pairr{x,y}}{\pairr{x,z}})}{(\id{x}{x})}
\end{equation*}
In particular, we obtain an identification $p:\id{\pairr{x,y}}{\pairr{x,z}}$ which comes equipped with an identification $q:\ap{\proj 1}{p}=\refl{x}$. Now it follows that
\begin{equation*}
\begin{tikzcd}[column sep=huge]
y \arrow[r,equals,"\apfunc{\mathsf{tr}_B(\blank,y)}(q)"] & \mathsf{tr}_B(p,y) \arrow[r,equals,"\apd{\proj 2}{p}"] & z,
\end{tikzcd}
\end{equation*}
where $\apdfunc{\proj 2}$ is the \emph{dependent} action on paths of the dependent function $\proj 2:\prd{t:\sm{x:A}B(x)} B(\proj 1(t))$, constructed in \cref{defn:apd}.
\end{proof}

\begin{cor}
Let $f:A\to B$ be a map. The following are equivalent:
\begin{enumerate}
\item For each $y:B$, the fiber $\fib{f}{y}$ is a proposition. 
\item $f$ is an embedding.
\end{enumerate}
\end{cor}

\begin{proof}
By \cref{ex:fib_replacement} there is a commuting triangle
\begin{equation*}
\begin{tikzcd}[column sep=large]
A \arrow[rr,"\lam{a}\pairr{f(a),\pairr{a,\refl{f(a)}}}"] \arrow[dr,swap,"f"] & & \sm{y:B}\fib{f}{y} \arrow[dl,"\proj 1"] \\
& B
\end{tikzcd}
\end{equation*}
in which the top map is an equivalence. Thus it follows from \autoref{ex:emb_triangle} that $f$ is an embedding if and only if $\proj 1:\big(\sm{y:B}\fib{f}{y}\big)\to B$ is an embedding. Now the claim follows from \cref{thm:subtype}.
\end{proof}
\end{comment}

\section{Sets}

\begin{defn}
A type $A$ is said to be a \define{set} if there is a term of type
\begin{equation*}
\isset(A)\defeq \prd{x,y:A}\isprop(\id{x}{y}).
\end{equation*}
\end{defn}

\begin{lem}
A type $A$ is a set if and only if it satisfies \define{axiom K}, which asserts that
\begin{equation*}
\prd{x:A}{p:\id{x}{x}}\id{\refl{x}}{p}.
\end{equation*}
\end{lem}

\begin{proof}
If $A$ is a set, then $\id{x}{x}$ is a proposition, so any two of its elements are equal. 
This implies axiom $K$. 

For the converse, if $A$ satisfies axiom $K$, then for any $p,q:\id{x}{y}$ we have $\id{\ct{p}{q^{-1}}}{\refl{x}}$, and hence $\id{p}{q}$. This shows that $\id{x}{y}$ is a proposition, and hence that $A$ is a set.
\end{proof}

\begin{lem}\label{lem:prop_to_id}
Let $A$ be a type, and let $R:A\to A\to\UU$ be a binary relation on $A$ satisfying
\begin{enumerate}
\item Each $R(x,y)$ is a proposition,
\item $R$ is reflexive, as witnessed by $\rho:\prd{x:A}R(x,x)$.
\end{enumerate}
Then any fiberwise map
\begin{equation*}
\prd{x,y:A}R(x,y)\to (\id{x}{y})
\end{equation*}
is a fiberwise equivalence. Consequently, if there is such a fiberwise map, then $A$ is a set.
\end{lem}

\begin{proof}
Let $f:\prd{x,y:A}R(x,y)\to(\id{x}{y})$. 
Since $R$ is assumed to be reflexive, we also have a fiberwise transformation
\begin{equation*}
\rec{x=}(\rho(x)):\prd{y:A}(\id{x}{y})\to R(x,y).
\end{equation*}
Since each $R(x,y)$ is assumed to be a proposition, it therefore follows that each $R(x,y)$ is a retract of $\id{x}{y}$. We conclude by \autoref{ex:id_fundamental_retr} that for each $x,y:A$, the map $f(x,y):R(x,y)\to(\id{x}{y})$ must be an equivalence. 
\end{proof}

\begin{thm}\label{thm:eq_nat}
The type of natural numbers is a set.
\end{thm}

\begin{proof}
We will apply \cref{lem:prop_to_id}. Note that the observational equality $\mathrm{Eq}_\N:\N\to(\N\to\UU)$ on $\N$ (\cref{defn:obs_nat}) is a reflexive relation by \autoref{ex:obs_nat_eqrel}, and moreover that $\mathrm{Eq}_\N(n,m)$ is a proposition for every $n,m:\N$ (proof by double induction).
Therefore it suffices to show that
\begin{equation*}
\prd{m,n:\nat}\mathrm{Eq}_\N(m,n)\to (\id{m}{n}).
\end{equation*}
This follows from the fact that observational equality is the \emph{least} reflexive relation, which was shown in \cref{ex:obs_nat_least}.
\end{proof}

\begin{comment}
\begin{thm}[Hedberg]\label{thm:dec_eq}
Any type with decidable equality is a set.
\end{thm}

\begin{proof}
Let $A$ be a type, and let $d:\prd{x,y:A}(\id{x}{y})+\neg(\id{x}{y})$ be the witness that $A$ has decidable equality.
We first construct a reflexive binary relation $E:A\to A\to\type$ such that each $E(x,y)$ is a proposition.
For every $x,y:A$, we first define a type family $E'(x,y):((\id{x}{y})+\neg(\id{x}{y}))\to\type$ by
\begin{align*}
E'(x,y,\inl(p)) & \defeq \unit \\
E'(x,y,\inr(p)) & \defeq \emptyt.
\end{align*}
Note that $E'(x,y,q)$ is a proposition for each $x,y:A$ and $q:(\id{x}{y})+\neg(\id{x}{y})$. 
Now we set $E(x,y)\defeq E'(x,y,d(x,y))$. Then $E$ is clearly reflexive, and a family of propositions.
Therefore it remains to show that $E$ implies identity. 

Since $E$ is defined as an instance of $E'$, it suffices to construct a term of type
\begin{equation*}
\prd{x,y:A}{q:(\id{x}{y})+\neg(\id{x}{y})} E'(q)\to (\id{x}{y}). 
\end{equation*}
By induction of disjoint sums, it suffices to construct terms of types
\begin{align*}
& \prd{x,y:A}{p:\id{x}{y}} \unit\to (\id{x}{y}) \\
& \prd{x,y:A}{p:\neg(\id{x}{y})} \emptyt\to (\id{x}{y}).
\end{align*}
In the first case, we take $\lam{x}{y}{p}{t}p$, and the second case is by induction on the empty type.
\end{proof}
\end{comment}

\section{General truncation levels}
\begin{defn}
We define $\istrunc{} : \Z_{\geq-2}\to\UU\to\UU$ by induction on $k:\Z_{\geq -2}$, taking
\begin{align*}
\istrunc{-2}(A) & \defeq \iscontr(A) \\
\istrunc{k+1}(A) & \defeq \prd{x,y:A}\istrunc{k}(\id{x}{y}).\qedhere
\end{align*}
For any type $A$, we say that $A$ is \define{$k$-truncated}, or a \define{$k$-type}, if there is a term of type $\istrunc{k}(A)$. We say that a map $f:A\to B$ is $k$-truncated if its fibers are $k$-truncated.
\end{defn}

%For the rest of this section, let $k:\Z_{\geq-2}$.

\begin{thm}
If $A$ is a $k$-type, then $A$ is also a $(k+1)$-type.
\end{thm}

\begin{proof}
We have seen in \cref{eg:prop_contr} that contractible types are propositions. This proves the base case.
For the inductive step, note that if any $k$-type is also a $(k+1)$-type, then any $(k+1)$-type is a $(k+2)$-type, since its identity types are $k$-types and therefore $(k+1)$-types.
\end{proof}

\begin{thm}\label{thm:ntype_eqv}
If $e:\eqv{A}{B}$ is an equivalence, and $B$ is a $k$-type, then so is $A$.
\end{thm}

\begin{proof}
We have seen in \autoref{ex:contr_equiv} that if $B$ is contractible and $e:\eqv{A}{B}$ is an equivalence, then $A$ is also contractible. This proves the base case.

For the inductive step, assume that the $k$-types are stable under equivalences, and consider $e:\eqv{A}{B}$ where $B$ is a $(k+1)$-type. In \autoref{cor:emb_equiv} we have seen that
\begin{equation*}
\apfunc{e}:(\id{x}{y})\to(\id{e(x)}{e(y)})
\end{equation*}
is an equivalence for any $x,y$. Note that $\id{e(x)}{e(y)}$ is a $k$-type, so by the induction hypothesis it follows that $\id{x}{y}$ is a $k$-type. This proves that $A$ is a $(k+1)$-type.
\end{proof}

\begin{cor}
If $f:A\to B$ is an embedding, and $B$ is a $(k+1)$-type, then so is $A$.
\end{cor}

\begin{proof}
By the assumption that $f$ is an embedding, the action on paths
\begin{equation*}
\apfunc{f}:(\id{x}{y})\to (\id{f(x)}{f(y)})
\end{equation*}
is an equivalence for every $x,y:A$. Since $B$ is assumed to be a $(k+1)$-type, it follows that $f(x)=f(y)$ is a $k$-type for every $x,y:A$. Therefore we conclude by \cref{thm:ntype_eqv} that $\id{x}{y}$ is a $k$-type for every $x,y:A$. In other words, $A$ is a $(k+1)$-type.
\end{proof}

In the following definition we generalize the notion of contractible map.

\begin{defn}
We say that a map $f:A\to B$ is \define{$k$-truncated} if for each $y:B$ the fiber $\fib{f}{y}$ is $k$-truncated.
\end{defn}

\begin{thm}
Let $B$ be a type family over $A$. Then the following are equivalent:
\begin{enumerate}
\item For each $x:A$ the type $B(x)$ is $k$-truncated.
\item The projection map
\begin{equation*}
\proj 1 : \Big(\sm{x:A}B(x)\Big)\to A
\end{equation*}
is $k$-truncated.
\end{enumerate}
\end{thm}

\begin{proof}
By \cref{ex:fib_replacement,ex:fiber_trans} we obtain equivalences
\begin{equation*}
\eqv{B(x)}{\fib{\proj 1}{x}}
\end{equation*}
for every $x:A$. Therefore the claim follows from \cref{thm:ntype_eqv}.
\end{proof}

\begin{thm}\label{thm:trunc_ap}
Let $f:A\to B$ be a map. The following are equivalent:
\begin{enumerate}
\item The map $f$ is $(k+1)$-truncated.
\item For each $x,y:A$, the map
\begin{equation*}
\apfunc{f} : (x=y)\to (f(x)=f(y))
\end{equation*}
is $k$-truncated. 
\end{enumerate}
\end{thm}

\begin{proof}
First we show that for any $s,t:\fib{f}{b}$ there is an equivalence
\begin{equation*}
\eqv{(s=t)}{\fib{\apfunc{f}}{\ct{\proj 2(s)}{\proj 2(t)^{-1}}}}
\end{equation*}
We do this by $\Sigma$-induction on $s$ and $t$, and then we calculate using \cref{ex:trans_ap} and basic manipulations of identifications that
\begin{align*}
(\pairr{x,p}=\pairr{y,q}) & \eqvsym \sm{r:x=y} \mathsf{tr}_{f(\blank)=b}(r,p)=q \\
& \eqvsym \sm{r:x=y} \ct{\ap{f}{r}^{-1}}{p}=q \\
& \eqvsym \sm{r:x=y} \ap{f}{r}=\ct{p}{q^{-1}} \\
& \jdeq \fib{\apfunc{f}}{\ct{p}{q^{-1}}}.
\end{align*}
By these equivalences, it follows that if $\apfunc{f}$ is $k$-truncated, then for each $s,t:\fib{f}{b}$ the identity type $s=t$ is equivalent to a $k$-truncated type, and therefore we obtain by \cref{thm:ntype_eqv} that $f$ is $(k+1)$-truncated.

For the converse, note that we have equivalences
\begin{align*}
\fib{\apfunc{f}}{p} & \eqvsym ((x,p)=(y,\refl{f(y)})).
\end{align*}
Therefore it follows that if $f$ is $(k+1)$-truncated, then the identity type $(x,p)=(y,\refl{f(y)})$ in $\fib{f}{f(y)}$ is $k$-truncated for any $p:f(x)=f(y)$, and therefore $\fib{\apfunc{f}}{p}$ is $k$-truncated by \cref{thm:ntype_eqv}. 
\end{proof}

\begin{cor}
A map is an embedding if and only if its fibers are propositions.
\end{cor}

\begin{cor}\label{thm:subtype}
A type family $B$ over $A$ is a subtype if and only if the projection map
\begin{equation*}
\proj 1 : \Big(\sm{x:A}B(x)\Big)\to A
\end{equation*}
is an embedding.
\end{cor}

\begin{thm}
Let $f:\prd{x:A}B(x)\to C(x)$ be a fiberwise transformation. Then the following are equivalent:
\begin{enumerate}
\item For each $x:A$ the map $f(x)$ is $k$-truncated.
\item The induced map 
\begin{equation*}
\total{f}:\Big(\sm{x:A}B(x)\Big)\to\Big(\sm{x:A}C(x)\Big)
\end{equation*}
is $k$-truncated.
\end{enumerate}
\end{thm}

\begin{proof}
This follows directly from \cref{lem:fib_total,thm:ntype_eqv}.
\end{proof}

\begin{comment}
\begin{proof}
By \autoref{ex:contr_retr} it follows that if $A$ is a retract of a contractible type, then $A$ is contractible.
For the inductive step, suppose that the $k$-types are closed under retracts, and consider a section-retraction pair
\begin{equation*}
\begin{tikzcd}
A \arrow[r,"i"] & B \arrow[r,"r"] & A,
\end{tikzcd}
\end{equation*}
with $H:r\circ i\htpy \idfunc$, where $B$ is a $(k+1)$-type.
By the induction hypothesis it suffices to show that for any $x,y:A$, the function $\apfunc{i}:(\id{x}{y})\to (\id{i(x)}{i(y)})$ has a retraction.
The retraction $\varphi:(\id{i(x)}{i(y)})\to(\id{x}{y})$ is defined as
\begin{equation*}
\varphi \defeq \lam{q} \ct{H(x)^{-1}}{\ap{r}{q}}{H(y)}
\end{equation*}
To see that $\varphi(\ap{i}{p})=p$, we have to show that the square
\begin{equation*}
\begin{tikzcd}
r(i(x)) \arrow[d,equals,swap,"\ap{r}{q}"] \arrow[r,equals,"H(x)"] & x \arrow[d,equals,"p"] \\
r(i(y)) \arrow[r,equals,swap,"H(y)"] & y
\end{tikzcd}
\end{equation*}
commutes. This square commutes by the naturality of homotopies, proven in \autoref{ex:htpy_nat}.
\end{proof}
\end{comment}

\begin{exercises}
\item \label{ex:diagonal}Let $A$ be a type, and let the \define{diagonal} of $A$ be the map $\delta_A:A\to A\times A$ given by $\lam{x}(x,x)$. 
\begin{subexenum}
\item Show that
\begin{equation*}
{\isequiv(\delta_A)}\leftrightarrow{\isprop(A)}.
\end{equation*}
\item Construct an equivalence $\eqv{\fib{\delta_A}{(x,y)}}{(x=y)}$ for any $x,y:A$.
\item Show that $A$ is $(k+1)$-truncated if and only if $\delta_A:A\to A\times A$ is $k$-truncated.
\end{subexenum}
\item \label{ex:istrunc_sigma}
\begin{subexenum}
\item Let $B$ be a type family over $A$. Show that if $A$ is a $k$-type, and $B(x)$ is a $k$-type for each $x:A$, then so is $\sm{x:A}B(x)$. Hint: for the base case, use \cref{ex:contr_in_sigma,ex:contr_equiv}.
\item Show that if $A$ and $B$ are $k$-types, then so is $A\times B$.
\end{subexenum}
\item \label{ex:eq_bool}Show that $\bool$ is a set by applying \cref{lem:prop_to_id} with the observational equality on $\bool$ defined in \cref{ex:obs_bool}.
\item Show that for any two sets $A$ and $B$, the disjoint sum $A+B$ is again a set.
\item \label{ex:hedberg}(Hedberg's theorem) A type $A$ is said to have \define{decidable equality} if there is a term of type
\begin{equation*}
\prd{x,y:A} (\id{x}{y})+\neg(\id{x}{y}).
\end{equation*}
For any type $A$, and every $x,y:A$, consider the type family $D(x,y):((\id{x}{y})+\neg(\id{x}{y}))\to\UU$ given by
\begin{align*}
D(x,y,\inl(p)) & \defeq \unit \\
D(x,y,\inr(p)) & \defeq \emptyt.
\end{align*}
Use $D$ to show that any type with decidable equality is a set.
\item Show that $\nat$ and $\bool$ have decidable equality, as defined in \autoref{ex:hedberg}.
\item Show that if $A$ and $B$ have decidable equality, then so do $A+B$ and $A\times B$.
\item Use \autoref{ex:contr_retr,ex:retr_id} to show that if $A$ is a retract of a $k$-type $B$, then $A$ is also a $k$-type.
\end{exercises}
