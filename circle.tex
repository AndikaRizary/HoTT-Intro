\chapter{The circle}

We have seen inductive types, in which we describe a type by its constructors and an induction principle that allows us to construct sections of dependent types. Inductive types are freely generated by their constructors, which describe how we can construct their terms. 

However, many familiar constructions in algebra involve the construction of algebras by generators and relations. 
For example, the free abelian group with two generators is described as the group with generators $x$ and $y$, and the relation $xy=yx$. 

In this chapter we introduce higher inductive types, where we follow a similar idea: to allow in the specification of inductive types not only \emph{point constructors}, but also \emph{path constructors} that give us relations between the point constructors. 
We introduce the idea of higher inductive types by studying the simplest non-trivial example: the \emph{circle}.
Moreover, we show that the loop space of the circle is equivalent to $\mathbb{Z}$ by constructing the universal cover of the circle as an application of the univalence axiom. 

\section{The definition and universal property of the circle}
The \define{circle} is defined as a higher inductive type $\sphere{1}$ that comes equipped with
\begin{align*}
\base & : \sphere{1} \\
\lloop & : \id{\base}{\base}.
\end{align*}
Just like for ordinary inductive types, the induction principle for higher inductive types provides us with a way of constructing sections of dependent types. However, we need to take the path constructor $\lloop$ into account in the induction principle. To see how we should do that, let us first start with an arbitrary section $f:\prd{t:\sphere{1}}P(t)$ of a dependent type $P$ over the circle. 

By applying a section $f:\prd{t:\sphere{1}}P(t)$ to the base point of the circle, we obtain a term $f(\base):P(\base)$. Moverover, using the dependent action on paths of $f$ of \autoref{defn:apd} we also obtain
\begin{align*}
\apd{f}{\lloop} & : \id{\trans{\lloop}{f(\base)}}{f(\base)}
\end{align*}
In other words, we have a map
\begin{equation}\label{eq:kappa_circle}
\kappa_{\sphere{1}}:\Big(\prd{t:\sphere{1}}P(t)\Big)\to\Big(\sm{y:P(\base)}\id{\trans{\lloop}{y}}{y}\Big)
\end{equation}
given by $\kappa_{\sphere{1}}(f)\defeq\pairr{f(\base),\apd{f}{\lloop}}$.

The \emph{induction principle of the circle} is a map in the opposite direction:
\begin{equation*}
\ind{\sphere{1}}:\Big(\sm{y:P(\base)}\id{\trans{\lloop}{y}}{y}\Big)\to\Big(\prd{t:\sphere{1}} P(t)\Big).
\end{equation*}
The computation rule asserts that $\kappa_{\sphere{1}}\circ\ind{\sphere{1}}\htpy \idfunc$. In other words, the induction principle together with the computation rule assert that the map $\kappa_{\sphere{1}}$ has a section. 

In the following theorem we prove the \emph{dependent} universal property of the circle.

\begin{thm}
For any dependent type $P:\sphere{1}\to\type$, the map $\kappa_{\sphere{1}}$ defined in \autoref{eq:kappa_circle} is an equivalence.
\end{thm}

\begin{proof}
It suffices to show that $\ind{\sphere{1}}\circ\kappa_{\sphere{1}}\htpy\idfunc$. Let $f:\prd{t:\sphere{1}}P(t)$. By the function extensionality principle it suffices to show that $\ind{\sphere{1}}(\kappa_{\sphere{1}}(f))\htpy f$. In other words, we have to show that
\begin{equation*}
\prd{t:\sphere{1}}\id{\ind{\sphere{1}}(\kappa_{\sphere{1}}(f),t)}{f(t)}.
\end{equation*}
Note that by the induction principle for the circle, it suffices to show that
\begin{equation*}
\sm{p:\id{\ind{\sphere{1}}(\kappa_{\sphere{1}}(f),\base)}{f(\base)}}\id{\trans{\lloop}{p}}{p}
\end{equation*}
\end{proof}

\section{The fundamental cover of the circle}

\begin{defn}
A \define{fiber sequence} 
\begin{equation*}
F \hookrightarrow E \twoheadrightarrow B
\end{equation*}
consists of a \define{base type} $B$ with a base point $b_0$, a type $E$ called the \define{total space}, a type $F$ called the \define{fiber}, a map $p:E\to B$, and an equivalence
\begin{align*}
& \eqv{\fib{p}{b_0}}{F}.
\end{align*}
\end{defn}

\begin{exercises}
\item Show that the map
\begin{equation*}
(\sphere{1}\to X)\to \sm{x:X}\id{x}{x}
\end{equation*}
given by $\lam{f}\pairr{f(\base),\ap{f}{\lloop}}$ is an equivalence.
\item Show that the map
\begin{equation*}
(\sphere{1}\to X)\to \sm{x,y:X} (\id{x}{y})\times(\id{x}{y})
\end{equation*}
given by $\lam{f}\pairr{f(\base),f(\base),\refl{f(\base)},\ap{f}{\lloop}}$ is an equivalence.
\item Use the negation equivalence $\mathsf{neg}:\eqv{\bool}{\bool}$ of \autoref{ex:neg_equiv} to construct a fiber sequence
\begin{equation*}
\bool\hookrightarrow \sphere{1}\twoheadrightarrow \sphere{1}.
\end{equation*}
\item 
\begin{subexenum}
\item Show that for every $x:\sphere{1}$, we have an equivalence
\begin{equation*}
\eqv{\Big(\sm{e:\eqv{\sphere{1}}{\sphere{1}}}e(\base)= x \Big)}{\bool}
\end{equation*}
\item Show that for every $x:\sphere{1}$, we have an equivalence
\begin{equation*}
\eqv{\Big(\sm{f:\sphere{1}\to \sphere{1}}f(\base)= x \Big)}{\Z}
\end{equation*}
\end{subexenum}
\item Show that $\eqv{(\eqv{\sphere{1}}{\sphere{1}})}{\sphere{1}+\sphere{1}}$. Conclude that a univalent universe containing a circle is not a $1$-type.
\item 
\begin{subexenum}
\item Show that a type $X$ is a set if and only if the map
\begin{equation*}
\lam{x}{t} x : X \to (\sphere{1}\to X)
\end{equation*}
is an equivalence.
\item Show that a type $X$ is a set if and only if the map
\begin{equation*}
\lam{f}f(\base) : (\sphere{1}\to X)\to X
\end{equation*}
is an equivalence.
\end{subexenum}
\end{exercises}
