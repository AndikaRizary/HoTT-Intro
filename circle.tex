\chapter{The circle}

We have seen inductive types, in which we describe a type by its constructors and an induction principle that allows us to construct sections of dependent types. Inductive types are freely generated by their constructors, which describe how we can construct their terms. 

However, many familiar constructions in algebra involve the construction of algebras by generators and relations. 
For example, the free abelian group with two generators is described as the group with generators $x$ and $y$, and the relation $xy=yx$. 

In this chapter we introduce higher inductive types, where we follow a similar idea: to allow in the specification of inductive types not only \emph{point constructors}, but also \emph{path constructors} that give us relations between the point constructors. 
The ideas behind the definition of higher inductive types are introduced by studying the simplest non-trivial example: the \emph{circle}.
Moreover, we show that the loop space of the circle is equivalent to $\mathbb{Z}$ by constructing the universal cover of the circle as an application of the univalence axiom. 

\section{The induction principle of the circle}
The \define{circle} is defined as a higher inductive type $\sphere{1}$ that comes equipped with
\begin{align*}
\base & : \sphere{1} \\
\lloop & : \id{\base}{\base}.
\end{align*}
Just like for ordinary inductive types, the induction principle for higher inductive types provides us with a way of constructing sections of dependent types. However, we need to take the path constructor $\lloop$ into account in the induction principle. To see how we should do that, let us first see how \emph{dependent} functions act on paths.

If we are given a dependent function $f:\prd{x:A}B(x)$ and an identification $p:\id{x}{y}$ in $A$, we cannot directly compare $f(x)$ with $f(y)$, since the types $B(x)$ and $B(y)$ might be judgmentally different types. However, using the transport function we can first transport $f(x)$ along the identification $p:\id{x}{y}$. This way, we obtain a term $\trans{p}{f(x)}:B(y)$, which is now of the same type as $f(y)$, and therefore may be compared to $f(y)$.

\begin{defn}\label{defn:apd}
Given a dependent function $f:\prd{a:A}B(a)$ and a path $p:\id{x}{y}$ in $A$, we construct a path
\begin{equation*}
\apd{f}{p} : \id{\trans{p}{f(x)}}{f(y)}.
\end{equation*}
\end{defn}

\begin{constr}
The path $\apd{f}{p}$ is constructed by path induction on $p$. Thus, it suffices to construct a path
\begin{equation*}
\apd{f}{\refl{x}}:\id{\trans{(\refl{x})}{f(x)}}{f(x)}.
\end{equation*}
Since transporting along $\refl{x}$ is the identity function on $B(x)$, we simply take $\apd{f}{\refl{x}}\defeq\refl{f(x)}$. 
\end{constr}

By applying a section $f:\prd{t:\sphere{1}}P(t)$ to the base point of the circle, we obtain a term $f(\base):P(\base)$. Moverover, using the dependent action on paths of $f$ of \autoref{defn:apd} we also obtain for any dependent function $f:\prd{t:\sphere{1}}P(t)$ a path
\begin{align*}
\apd{f}{\lloop} & : \id{\trans{\lloop}{f(\base)}}{f(\base)}
\end{align*}
in the fiber $P(\base)$.
In other words, the operation assigning to each dependent function its \emph{action on generators} is a map
\begin{equation}\label{eq:kappa_circle}
\mathsf{dgen}_{\sphere{1}}:\Big(\prd{t:\sphere{1}}P(t)\Big)\to\Big(\sm{y:P(\base)}\id{\trans{\lloop}{y}}{y}\Big)
\end{equation}
given by $\mathsf{dgen}_{\sphere{1}}(f)\defeq\pairr{f(\base),\apd{f}{\lloop}}$.
The induction principle of the circle can now be formulated as a map in the opposite direction, and the computation rule is the homotopy witnessing that the induction principle is a section of $\mathsf{dgen}_{\sphere{1}}$.

\begin{defn}
The \define{induction principle of the circle} is a class of maps
\begin{equation*}
\ind{\sphere{1}}(P):\Big(\sm{y:P(\base)}\id{\trans{\lloop}{y}}{y}\Big)\to\Big(\prd{t:\sphere{1}} P(t)\Big)
\end{equation*}
ranging over all depenent types $P:\sphere{1}\to\type$. 
The \define{computation rule} asserts that $\mathsf{dgen}_{\sphere{1}}(P) \circ\ind{\sphere{1}}(P) \htpy \idfunc$ for each dependent type $P:\sphere{1}\to\type$. 
\end{defn} 

\section{The fundamental cover of the circle}

The \emph{fundamental cover} of the circle is a family of sets over the circle with contractible total space.
Classically, the fundamental cover is described as a map $\mathbb{R}\to\sphere{1}$ that winds the real line around the circle.
In homotopy type theory there is no analogue of such a construction.
However, when we see the fundamental cover as a \emph{continuously varying family of spaces}, we can attempt to construct the fundamental cover as a dependent type over the circle. 

The type of (small) dependent types over $\sphere{1}$ is just $\sphere{1}\to\UU$, so in fact we may use the recursion principle of the circle to construct dependent types over the circle. 
In other words, to give a type family over the circle we must provide a type $X$, and an equality $X=X$ in the universe.
This is where the univalence axiom comes in. By the map
\begin{equation*}
\mathsf{eq\usc{}equiv}_{X,X}:(\eqv{X}{X})\to (X=X)
\end{equation*}
it suffices to provide an equivalence $\eqv{X}{X}$. We make these ideas precise in the following lemma.

\begin{defn}\label{defn:circle_descent}
Consider a type $X$ and every equivalence $e:\eqv{X}{X}$.
We will construct a dependent type $\mathcal{D}(X,e):\sphere{1}\to\UU$ with $\eqv{\mathcal{D}(X,e,\base)}{X}$ for which the square
\begin{equation*}
\begin{tikzcd}
\mathcal{D}(X,e,\base)\arrow[r,"\eqvsym"] \arrow[d,swap,"\mathsf{tr}^{\mathcal{D}(X,e)}(\lloop)"] & X \arrow[d,"e"] \\
\mathcal{D}(X,e,\base)\arrow[r,swap,"\eqvsym"] & X
\end{tikzcd}
\end{equation*}
commutes. The type $\sm{X:\UU}\eqv{X}{X}$ is also called the type of \define{descent data} for the circle.
\end{defn}

\begin{constr}
By \autoref{ex:tr_ap} we have an identification
\begin{equation*}
\mathsf{equiv\usc{}eq}(\ap{P}{\lloop})=\mathsf{tr}^P(\lloop)
\end{equation*}
for each dependent type $P:\sphere{1}\to\UU$. Therefore we see that the triangle
\begin{equation*}
\begin{tikzcd}
& (\sphere{1}\to \UU) \arrow[dl,swap,"\mathsf{gen}_{\sphere{1}}"] \arrow[dr,"\mathsf{desc}_{\sphere{1}}"] \\
\sm{X:\UU}X=X \arrow[rr,swap,"\total{\lam{X}\mathsf{equiv\usc{}eq}_{X,X}}"] & & \sm{X:\UU}\eqv{X}{X}
\end{tikzcd}
\end{equation*}
commutes, where the map $\mathsf{desc}_{\sphere{1}}$ is given by $P\mapsto\pairr{P(\base),\mathsf{tr}^P(\lloop)}$ and the bottom map is an equivalence by the univalence axiom.
Now it follows that $\mathsf{desc}_{\sphere{1}}$ has a section, since by the recursion principle the map $\mathsf{gen}_{\sphere{1}}$ has a section. 
This means that for every type $X$ and every $e:\eqv{X}{X}$ there is a type family $\mathcal{D}(X,e):\sphere{1}\to\UU$ such that
\begin{equation*}
\pairr{\mathcal{D}(X,e,\base),\mathsf{tr}^{\mathcal{D}(X,e)}(\lloop)}=\pairr{X,e}.
\end{equation*}
Equivalently, we have $p:\id{\mathcal{D}(X,e,\base)}{X}$ and $\trans{p}{\mathsf{tr}^{\mathcal{D}(X,e)}(\lloop)}=e$. Thus, we obtain $\mathsf{equiv\usc{}eq}(p):\eqv{\mathcal{D}(X,e,\base)}{X}$, for which the square
\begin{equation*}
\begin{tikzcd}[column sep=huge]
\mathcal{D}(X,e,\base)\arrow[r,"\mathsf{equiv\usc{}eq}(p)"] \arrow[d,swap,"\mathsf{tr}^{\mathcal{D}(X,e)}(\lloop)"] & X \arrow[d,"e"] \\
\mathcal{D}(X,e,\base)\arrow[r,swap,"\mathsf{equiv\usc{}eq}(p)"] & X
\end{tikzcd}
\end{equation*}
commutes.
\end{constr}

\begin{comment}
\begin{defn}\label{defn:fiber_sequence}
A \define{fiber sequence} 
\begin{equation*}
F \hookrightarrow E \twoheadrightarrow B
\end{equation*}
consists of a \define{base type} $B$ with a base point $b_0$ and a dependent type $P:B\to\type$, a type $F$ called the \define{fiber} with an equivalence $\eqv{P(b_0)}{F}$, and a type $E$ called the \define{total space} with a map $p:E\to B$ and an equivalence $e:\eqv{(\sm{b:B}P(b))}{E}$ such that the triangle
\begin{equation*}
\begin{tikzcd}
\Big(\sm{b:B}P(b)\Big) \arrow[rr,"e"] \arrow[dr,swap,"\proj 1"] & & E \arrow[dl,"p"] \\
& B
\end{tikzcd}
\end{equation*}
commutes.
\end{defn}
\end{comment}

\begin{defn}
The \define{fundamental cover} of the circle is the dependent type $\mathcal{E}\defeq\mathcal{D}(\Z,\mathsf{succ}):\sphere{1}\to\UU$ that is obtained by applying the construction of \autoref{defn:circle_descent}.
\end{defn}

\begin{thm}
The total space $\sm{t:\sphere{1}}\mathcal{E}(t)$ of the fundamental cover of $\sphere{1}$ is contractible.
\end{thm}

\begin{proof}
We show that $\mathcal{E}(t)$ is a retract of $\base= t$, for each $t:\sphere{1}$, i.e.~we apply \autoref{ex:id_fundamental_retr}. Thus, we will define for each $t:\sphere{1}$ a map $r(t):\mathcal{E}(t)\to(\base=t)$, and for each $t:\sphere{1}$ and $x:\mathcal{E}(t)$ we will show that
\begin{equation*}
\ind{\base=}(0_\Z,r(t,x))=x.
\end{equation*}
We construct $r(t)$ by induction on $t:\sphere{1}$. For the base point we construct a map $\Z\to (\base=\base)$ with \autoref{ex:int_ptd_auto} using the base point $\refl{\base}$ and the equivalence $l\mapsto \ct{\lloop}{l}$. This defines $r(\base)$. 

Our next goal is to show that
\begin{equation*}
\trans{\lloop}{r(\base)}=r(\base).
\end{equation*}
Computing the transport, we see that we have to show that the square
\begin{equation*}
\begin{tikzcd}[column sep=large]
\mathcal{E}(\base) \arrow[r,"r(\base)"] \arrow[d,swap,"\mathsf{tr}^{\mathcal{E}}(\lloop)"] & (\base=\base) \arrow[d,"l\mapsto \ct{\lloop}{l}"] \\
\mathcal{E}(\base) \arrow[r,swap,"r(\base)"] & (\base=\base)
\end{tikzcd}
\end{equation*}
commutes.
\end{proof}

\begin{cor}
The loop space of the circle is equivalent to $\Z$.
\end{cor}

\begin{cor}
The circle is a $1$-type and it is not a $0$-type.
\end{cor}

\begin{exercises}
\item \label{ex:circle_up} Show that for each type $X$, the \emph{action on generators}
\begin{equation*}
\mathsf{gen}_{\sphere{1}}:(\sphere{1}\to X)\to \sm{x:X}x=x
\end{equation*}
given by $f\mapsto (f(\base),\ap{f}{\lloop})$ is an equivalence.
\item \label{ex:circle_dup} Show that for each dependent type $P:\sphere{1}\to\type$, the \emph{dependent action on generators}
\begin{equation*}
\Big(\prd{t:\sphere{1}}P(t)\Big)\to \sm{u:P(\base)}\trans{\lloop}{u}=u
\end{equation*}
is an equivalence.
\item Show that the map
\begin{equation*}
(\sphere{1}\to X)\to \sm{x,y:X} (\id{x}{y})\times(\id{x}{y})
\end{equation*}
given by $\lam{f}\pairr{f(\base),f(\base),\refl{f(\base)},\ap{f}{\lloop}}$ is an equivalence.
\item Let $P:\sphere{1}\to\prop$ be a family of propositions over the circle. Show that
\begin{equation*}
P(\base)\to\prd{t:\sphere{1}}P(t).
\end{equation*}
In this sense the circle is \emph{connected}.
\item Use the fundamental cover of the circle to show that
\begin{equation*}
\neg\Big(\prd{t:\sphere{1}}\base=t\Big).
\end{equation*}
\item \label{ex:circle_constant}
Show that for any type $X$ and any $x:X$, the map
\begin{equation*}
\rec{\sphere{1}}(x,\refl{x}):\sphere{1}\to X
\end{equation*}
is homotopic to the constant map $\lam{t}x$.
\item \label{ex:circle_degk}
\begin{subexenum}
\item Show that for every $x:\sphere{1}$, we have an equivalence
\begin{equation*}
\eqv{\Big(\sm{f:\sphere{1}\to \sphere{1}}f(\base)= x \Big)}{\Z}
\end{equation*}
The base point preserving map $f:\sphere{1}\to\sphere{1}$ corresponding to $k:\Z$ is called the \define{degree $k$ map} on the circle, and is denoted by $\mathsf{deg}(k)$.
\item Show that for every $x:\sphere{1}$, we have an equivalence
\begin{equation*}
\eqv{\Big(\sm{e:\eqv{\sphere{1}}{\sphere{1}}}e(\base)= x \Big)}{\bool}
\end{equation*}
\end{subexenum}
\item \label{ex:circle_double_cover} The \define{(twisted) double cover} of the circle is defined as the type family $\mathcal{T}\defeq\mathcal{D}(\bool,\mathsf{neg}):\sphere{1}\to\UU$, where $\mathsf{neg}:\eqv{\bool}{\bool}$ is the negation equivalence of \autoref{ex:neg_equiv}.
\begin{subexenum}
\item Show that $\neg(\prd{t:\sphere{1}}\mathcal{T}(t))$.
\item Construct an equivalence $e:\eqv{\sphere{1}}{\sm{t:\sphere{1}}\mathcal{T}(t)}$ for which the triangle
\begin{equation*}
\begin{tikzcd}
\sphere{1} \arrow[rr,"e"] \arrow[dr,swap,"\mathsf{deg}(2)"] & & \sm{t:\sphere{1}}\mathcal{T}(t) \arrow[dl,"\proj 1"] \\
& \sphere{1}
\end{tikzcd}
\end{equation*}
commutes.
\end{subexenum}
\item 
\begin{subexenum}
\item Show that a type $X$ is a set if and only if the map
\begin{equation*}
\lam{x}{t} x : X \to (\sphere{1}\to X)
\end{equation*}
is an equivalence.
\item Show that a type $X$ is a set if and only if the map
\begin{equation*}
\lam{f}f(\base) : (\sphere{1}\to X)\to X
\end{equation*}
is an equivalence.
\end{subexenum}
\item Show that $\eqv{(\eqv{\sphere{1}}{\sphere{1}})}{\sphere{1}+\sphere{1}}$. Conclude that a univalent universe containing a circle is not a $1$-type.
\end{exercises}
