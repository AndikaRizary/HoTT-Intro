\chapter{Homotopy pushouts}

\section{General pushouts}

\begin{defn}
A \define{span} from $A$ to $B$ consists of a type $S$ and maps $f:S\to A$ and $g:S\to B$, as indicated in the diagram
\begin{equation*}
\begin{tikzcd}
& S \arrow[dl,swap,"f"] \arrow[dr,"g"] \\
A & & B.
\end{tikzcd}
\end{equation*}
\end{defn}

\begin{defn}
Let $X$ be a type, and let $\mathcal{S}\jdeq (S,f,g)$ be a span from $A$ to $B$.
A \define{cocone with vertex $X$} on $\mathcal{S}$ consists of maps $i:A\to X$ and $j:B\to X$, and a homotopy $H:i\circ f\htpy j\circ g$.
We write $\mathsf{cocone}_{\mathcal{S}}(X)$ for the type of cocones with vertex $X$.
\end{defn}

Thus, a cocone $(i,j,H)$ on a span $\mathcal{S}$ can be represented as a commutative square
\begin{equation*}
\begin{tikzcd}
S \arrow[r,"g"] \arrow[d,swap,"f"] & B \arrow[d,"j"] \\
A \arrow[r,swap,"i"] & X 
\end{tikzcd}
\end{equation*}

\begin{defn}
Let $\mathcal{X}\defeq(i,j,H)$ be a cocone with vertex $X$ on $\mathcal{S}$. For any type $Y$ we define a map
\begin{equation*}
\mathsf{cocone\usc{}map}(\mathcal{X}):(X\to Y)\to \mathsf{cocone}(Y)
\end{equation*}
by $f\mapsto (f\circ i,f\circ j,f\cdot H)$. We say that $\mathcal{X}$ is \define{colimiting} if $\mathsf{cocone\usc{}map}(\mathcal{X})$ is an equivalence for every type $Y$. 
\end{defn}

\begin{thm}\label{thm:pushout_up}
Let $\mathcal{X}\defeq(i,j,H)$ be a cocone with vertex $X$ on $\mathcal{S}$. The following are equivalent:
\begin{enumerate}
\item The cocone $\mathcal{X}$ is colimiting.
\item The type $X$ is $\mathcal{S}$-inductive, in the sense that for any type fampy $P:X\to \UU$ the map
\begin{equation*}
\Big(\prd{x:X}P(x)\Big)\to \Big(\sm{i':\prd{a:A}P(i(a))}{j':\prd{b:B}P(j(b))} i'\htpy_H j'\Big)
\end{equation*}
given by $h\mapsto (h\circ i,h\circ j,\mathsf{apd}_h(H(s)))$ has a section. Here we define
\begin{equation*}
i'\htpy_H j'\defeq \prd{s:S}\dpath{P}{H(s)}{i'(f(s))}{j'(g(s))}.
\end{equation*}
\end{enumerate}
\end{thm}

From now on we will assume that for any span $\mathcal{S}\jdeq (S,f,g)$ from $A$ to $B$ there is a type $A\sqcup^{\mathcal{S}}B$ equipped with
\begin{align*}
\inl & : A \to A \sqcup^S B \\
\inr & : B \to A \sqcup^S B \\
\glue & : \prd{x:S} \id{\inl(f(x))}{\inr(g(s))},
\end{align*}
which is $\mathcal{S}$-inductive in the sense of \autoref{thm:pushout_up}. Moreover, if $A$ and $B$ are types in $\UU$, then we assume that also $A\sqcup^{\mathcal{S}} B$ is in $\UU$. We call $A\sqcup^{\mathcal{S}} B$ the \define{canonical pushout} of $\mathcal{S}$. 

\section{Suspension and join}
Many interesting types can be defined as homotopy pushouts. 

\begin{defn}
Let $X$ be a type. We define the \define{suspension} $\susp X$ of $X$ to be the pushout of the span
\begin{equation*}
\begin{tikzcd}
& X \arrow[dl,swap,"\mathsf{const}_\ttt"] \arrow[dr,"\mathsf{const}_\ttt"] \\
\unit & & \unit.
\end{tikzcd}
\end{equation*}
\end{defn}

\begin{defn}
We define the \define{sphere} $\sphere{n}$ for any $n:\N$ by induction on $n$, by taking
\begin{align*}
\sphere{0} & \defeq \bool \\
\sphere{n+1} & \defeq \susp{\sphere{n}}.
\end{align*}
\end{defn}

\begin{defn}
We define the \define{join} $\join{X}{Y}$ of $X$ and $Y$ to be the pushout 
\begin{equation*}
\begin{tikzcd}
X\times Y \arrow[r,"\proj 2"] \arrow[d,swap,"\proj 1"] & Y \arrow[d,"\inr"] \\
X \arrow[r,swap,"\inl"] & X \ast Y 
\end{tikzcd}
\end{equation*}
\end{defn}

\begin{comment}
\begin{defn}
Let $X$ and $Y$ be types with base points $x_0$ and $y_0$, respectively.
We define the \define{wedge} $X\lor Y$ of $X$ and $Y$ to be the pushout
\begin{equation*}
\begin{tikzcd}[column sep=8em]
\bool \arrow[r,"{\ind{\bool}(\inl(x_0),\inr(y_0))}"] \arrow[d,swap,"\mathsf{const}_\ttt"] & X+Y \arrow[d,"\inr"] \\
\unit \arrow[r,swap,"\inl"] & X\lor Y
\end{tikzcd}
\end{equation*}
\end{defn}

\begin{defn}
Let $X$ and $Y$ be types with base points $x_0$ and $y_0$, respectively.
We define a map
\begin{equation*}
\mathsf{wedge\usc{}incl} : X \lor Y \to X\times Y.
\end{equation*}
as the unique map obtained from the commutative square
\begin{equation*}
\begin{tikzcd}[column sep=8em]
\bool \arrow[r,"{\ind{\bool}(\inl(x_0),\inr(y_0))}"] \arrow[d,swap,"\mathsf{const}_\ttt"] & X+Y \arrow[d,"{\ind{X+Y}(\lam{x}\pairr{x,y_0},\lam{y}\pairr{x_0,y})}"] \\
\unit \arrow[r,swap,"\lam{t}\pairr{x_0,y_0}"] & X\times Y.
\end{tikzcd}
\end{equation*}
\end{defn}

\begin{defn}
We define the \define{smash product} $X\wedge Y$ of $X$ and $Y$ to be the pushout
\begin{equation*}
\begin{tikzcd}[column sep=huge]
X\lor Y \arrow[r,"\mathsf{wedge\usc{}incl}"] \arrow[d,swap,"\mathsf{const}_\ttt"] & X\times Y \arrow[d,"\inr"] \\
\unit \arrow[r,swap,"\inl"] & X\wedge Y.
\end{tikzcd}
\end{equation*}
\end{defn}
\end{comment}

\begin{exercises}
\item Show that $\join{P}{Q}= P\lor Q$, for any two propositions $P$ and $Q$.
\item Show that $\sphere{1}$ is equivalent to $\susp\bool$. 
\item Show that $\eqv{A\sqcup^{\mathcal{S}} B}{B\sqcup^{\mathcal{S}'} A}$, where $\mathcal{S'}\defeq (S,g,f)$ is the \define{opposite span} of $\mathcal{S}$. 
\item Let $Q$ be a proposition, and let $A$ be a type. Show that $\inr:A\to \join{Q}{A}$ is an equivalence if and only if $Q\to\iscontr(A)$.
\item Show that if $\mathcal{S}\jdeq(S,f,g)$ is a span from $A$ to $B$ and $f:S\to A$ is an equivalence, then so is $\inr:B\to A\sqcup^\mathcal{S} B$. Use this observation to conclude the following:
\begin{subexenum}
\item If $X$ is contractible, then $\susp X$ is contractible.
\item There is an equivalence $\eqv{X}{\join{\emptyt}{X}}$.
\item If $X$ is contractible, then $\join{X}{Y}$ is contractible. 
\end{subexenum}
\item Show that if
\begin{equation*}
\begin{tikzcd}
S \arrow[r] \arrow[d] & Y \arrow[d] \\
X \arrow[r] & Z
\end{tikzcd}
\end{equation*}
is a pushout square, then so is
\begin{equation*}
\begin{tikzcd}
A\times S \arrow[r] \arrow[d] & A\times Y \arrow[d] \\
A\times X \arrow[r] & A\times Z
\end{tikzcd}
\end{equation*}
\item Show that if
\begin{equation*}
\begin{tikzcd}
S_1 \arrow[r] \arrow[d] & Y_1 \arrow[d] & S_2 \arrow[r] \arrow[d] & Y_2 \arrow[d] \\
X_1 \arrow[r] & Z_1 & X_2 \arrow[r] & Z_2
\end{tikzcd}
\end{equation*}
are pushout squares, then so is
\begin{equation*}
\begin{tikzcd}
S_1+S_2 \arrow[r] \arrow[d] & Y_1+ Y_2 \arrow[d] \\
X_1 +X_2 \arrow[r] & Z_1+Z_2. 
\end{tikzcd}
\end{equation*}
\item Consider a span $(S,f,g)$ from $A$ to $B$. Show that the pushout of the span
\begin{equation*}
\begin{tikzcd}
& S+S \arrow[dl] \arrow[dr] \\
A + B & & S
\end{tikzcd}
\end{equation*}
is equivalent to the pushout of $(S,f,g)$.
\end{exercises}
