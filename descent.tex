\chapter{Descent}

\section{Homotopy pullbacks}
[Tell something about why pullbacks are important.]
\begin{defn}
Given maps $f:A\to X$ and $g:B\to X$, we define \[A\times_X B\defeq \sm{x:A}{y:B}f(x)=f(y).\] The \define{canonical pullback square} of $f$ and $g$ is the commuting square
\begin{equation*}
\begin{tikzcd}
A\times_X B \arrow[r,"\pi_2"] \arrow[d,swap,"\pi_1"] & B \arrow[d,"g"] \\
A \arrow[r,swap,"f"] & X,
\end{tikzcd}
\end{equation*}
where $\pi_1(x,y,p)\defeq x$, and $\pi_2(x,y,p)\defeq y$, and the commutativity is witnessed by the homotopy $\lam{\pairr{x,y,p}}p$. 
\end{defn}

\begin{thm}
For any type $C$, the map
\begin{equation*}
(C\to A\times_X B)\to \sm{h:C\to A}{k:C\to B}f\circ h\htpy g\circ k
\end{equation*}
is an equivalence.
\end{thm}

\begin{lem}[Pullback pasting lemma]\label{lem:pb_pasting}
Consider 
\begin{equation*}
\begin{tikzcd}
A \arrow[d,swap,"f"] \arrow[r,"j"] & B \arrow[d,swap,"g"] \arrow[r,"l"] & C \arrow[d,"h"] \\
X \arrow[r,swap,"i"] & Y \arrow[r,swap,"k"] & Z
\end{tikzcd}
\end{equation*}
with homotopies $H:i\circ f\htpy g\circ j$ and $K:k\circ g\htpy h\circ l$ witnessing that the two squares commute, and suppose that the square on the right is a pullback square. Then the square on the left is a pullback square if and only if the outer rectangle is a pullback square.
\end{lem}

\section{The flattening lemma}

\begin{exercises}
\item Show that the square
\begin{equation*}
\begin{tikzcd}
(x=y) \arrow[r] \arrow[d] & \unit \arrow[d,"\mathsf{const}_y"] \\
\unit \arrow[r,swap,"\mathsf{const}_x"] & A
\end{tikzcd}
\end{equation*}
is a pullback square.
\item Let $f:A\to B$ be a map, and consider $P:A\to\type$, and $Q:B\to\type$. Furthermore, consider a fiberwise transformation
\begin{equation*}
g:\prd{x:A}P(x)\to Q(f(x)).
\end{equation*}
Show that the following are equivalent:
\begin{subexenum}
\item The fiberwise map $g$ is a fiberwise equivalence.
\item The commuting square
\begin{equation*}
\begin{tikzcd}[column sep=large]
\sm{x:A}P(x) \arrow[d,swap,"\proj 1"] \arrow[r,"\total{g}"] & \sm{y:B}Q(y) \arrow[d,"\proj 1"] \\
A \arrow[r,swap,"f"] & B
\end{tikzcd}
\end{equation*}
is a pullback square.
\end{subexenum}
\end{exercises}
