\chapter{Equivalences}

\section{Homotopies}
In homotopy type theory, a homotopy is just a pointwise equality between two functions $f$ and $g$.

\begin{defn}
Let $f,g:\prd{x:A}P(x)$ be two dependent functions. The type of \define{homotopies} from $f$ to $g$ is defined as
\begin{equation*}
f\htpy g \defeq \prd{x:A} f(x)=g(x).
\end{equation*}
\end{defn}

Since we formulated homotopies using dependent functions, we may also consider homotopies \emph{between} homotopies, and further homotopies between those higher homotopies. 
Explicitly, if $H,K:f\htpy g$, then the type $H\htpy K$ of homotopies is just the type
\begin{equation*}
\prd{x:A} H(x)=K(x).
\end{equation*}

In the following definition we define the groupoid-like structure of homotopies. Note that we implement the groupoid-laws as \emph{homotopies} rather than as identifications.

\begin{defn}
For any dependent type $B:A\to\type$ there are operations
\begin{align*}
& \mathsf{htpy.refl} & & : \prd{f:\prd{x:A}B(x)}f\htpy f \\
& \mathsf{htpy.inv} & & : \prd*{f,g:\prd{x:A}B(x)} (f\htpy g)\to(g\htpy f) \\
& \mathsf{htpy.concat} & & : \prd*{f,g,h:\prd{x:A}B(x)} (f\htpy g)\to (g\htpy h)\to (f\htpy h).
\end{align*}
We will write $H^{-1}$ for $\mathsf{hinv}(H)$, and $\ct{H}{K}$ for $\mathsf{hconcat}(H,K)$. 

Furthermore, we define
\begin{align*}
& \mathsf{htpy.assoc}(H,K,L) & & : \ct{(\ct{H}{K})}{L}\htpy\ct{H}{(\ct{K}{L})} \\
& \mathsf{htpy.left\usc{}unit}(H) & & : \ct{\mathsf{htpy.refl}_f}{H}\htpy H \\
& \mathsf{htpy.right\usc{}unit}(H) & & : \ct{H}{\mathsf{htpy.refl}_g}\htpy H \\
& \mathsf{htpy.left\usc{}inv}(H) & & : \ct{H^{-1}}{H} \htpy \mathsf{htpy.refl}_g \\
& \mathsf{htpy.right\usc{}inv}(H) & & : \ct{H}{H^{-1}} \htpy \mathsf{htpy.refl}_f
\end{align*}
for any $H:f\htpy g$, $K:g\htpy h$ and $L:h\htpy i$, where $f,g,h,i:\prd{x:A}B(x)$.
\end{defn}


Apart from the groupoid operations and their laws, we will occasionally need \emph{whiskering} operations.

\begin{defn}
We define the following \define{whiskering} operations on homotopies:
\begin{enumerate}
\item Suppose $H:f\htpy g$ for two functions $f,g:A\to B$, and let $h:B\to C$. We define
\begin{equation*}
hH\defeq \lam{x}\ap{h}{H(x)}:h\circ f\htpy h\circ g.
\end{equation*}
\item Suppose $f:A\to B$ and $H:g\htpy h$ for two functions $g,h:B\to C$. We define
\begin{equation*}
Hf\defeq\lam{x}H(f(x)):h\circ f\htpy g\circ f.
\end{equation*}
\end{enumerate}
\end{defn}

\section{Bi-invertible maps}
\begin{defn}
Let $f:A\to B$ be a function. We say that $f$ has a \define{section} if there is a term of type
\begin{equation*}
\mathsf{sec}(f) \defeq \sm{g:B\to A} f\circ g\htpy \idfunc[B].
\end{equation*}
Dually, we say that $f$ has a \define{retraction} if there is a term of type
\begin{equation*}
\mathsf{retr}(f) \defeq \sm{h:B\to A} h\circ f\htpy \idfunc[A].
\end{equation*}
If $f$ has a retraction, we also say that $A$ is a \define{retract} of $B$.
We say that a function $f:A\to B$ is an \define{equivalence} if it has both a section and a retraction, i.e.~if it comes equipped with a term of type
\begin{equation*}
\isequiv(f)\defeq\mathsf{sec}(f)\times\mathsf{retr}(f).
\end{equation*}
We will write $\eqv{A}{B}$ for the type $\sm{f:A\to B}\isequiv(f)$.
\end{defn}

\begin{rmk}
An equivalence, as we defined it here, can be thought of as a \define{bi-invertible} map, since it comes equipped with a separate left and right inverse. Explicitly, if $f$ is an equivalence, then there are
\begin{align*}
g & : B\to A & h & : B\to A \\
G & : f\circ g \htpy \idfunc[B] & H & : h\circ f \htpy \idfunc[A].
\end{align*}
Clearly, if $f$ is \define{invertible} in the sense that it comes equipped with a function $g:B\to A$ such that $f\circ g\htpy\idfunc[B]$ and $g\circ f\htpy\idfunc[A]$, then $f$ is an equivalence.
\end{rmk}

\begin{defn}\label{defn:inv_equiv}
Any equivalence can be given the structure of an invertible map.
\end{defn}

\begin{constr}
For any $y:B$, we have 
\begin{equation*}
\begin{tikzcd}[column sep=huge]
g(y) \arrow[r,equals,"H(g(y))^{-1}"] & hfg(y) \arrow[r,equals,"\ap{h}{G(y)}"] & h(y)
\end{tikzcd}
\end{equation*} 
from which we obtain a homotopy $K:g\htpy h$.
This allows us to show that $g$ is a retraction of $f$: for $x:A$ we take
\begin{equation*}
\begin{tikzcd}[column sep=large]
gf(x) \arrow[r,equals,"K(f(x))"] & hf(x) \arrow[r,equals,"H(x)"] & x.
\end{tikzcd}\qedhere
\end{equation*}
\end{constr}

\begin{thm}\label{thm:id_equiv}
For any type $A$, the identity function $\idfunc[A]$ is an equivalence.
\end{thm}

\begin{proof}
The identity function is trivially its own section and its own retraction.
\end{proof}

\section{The identity type of a \texorpdfstring{$\Sigma$-}{dependent pair }type}
\begin{thm}\label{thm:eq_sigma}
Let $B:A\to\type$ be a type family, and let $\pairr{x,y},\pairr{x',y'}:\sm{x:A}B(x)$. Then the map
\begin{equation*}
\mathsf{eq\usc{}pair} : \Big(\sm{p:x=x'}\id{\trans{p}{y}}{y'}\Big)\to(\id{\pairr{x,y}}{\pairr{x',y'}})
\end{equation*}
defined by double path induction by sending $\pairr{\refl{x},\refl{y}}$ to $\refl{\pairr{x,y}}$ is an equivalence.
\end{thm}

\begin{proof}
The map $\mathsf{pair\usc{}eq}$ in the converse direction is constructed by path induction, taking $\refl{\pairr{x,y}}$ to $\pairr{\refl{x},\refl{y}}$.
It remains to show that $\mathsf{pair\usc{}eq}$ and $\mathsf{eq\usc{}pair}$ are mutual inverses. 

We first show that $\mathsf{eq\usc{}pair}(\mathsf{pair\usc{}eq}(p))=p$ for each $p:\pairr{x,y}=\pairr{x',y'}$. We proceed by path induction on $p$. 
Our goal is now to construct an identification $\mathsf{eq\usc{}pair}\pairr{\refl{x},\refl{y}}=\refl{\pairr{x,y}}$. 
Thus, we may take $\refl{\refl{\pairr{x,y}}}$.

Finally, we construct an identification $\mathsf{pair\usc{}eq}(\mathsf{eq\usc{}pair}(p,q))=\pairr{p,q}$ for each $\pairr{p,q}:\sm{p:x=x'}\id{\trans{p}{y}}{y'}$. We proceed by path induction on $p$, followed by path induction on $q$. Our goal is now to construct a term of type
\begin{equation*}
\mathsf{pair\usc{}eq}(\mathsf{eq\usc{}pair}\pairr{\refl{x},\refl{y}})=\pairr{\refl{x},\refl{y}}
\end{equation*}
By the definition of $\mathsf{eq\usc{}pair}$ we have $\mathsf{eq\usc{}pair}\pairr{\refl{x},\refl{y}}\jdeq \refl{\pairr{x,y}}$, and by the definition of $\mathsf{pair\usc{}eq}$ we have $\mathsf{pair\usc{}eq}(\refl{\pairr{x,y}})\jdeq\pairr{\refl{x},\refl{y}}$. Thus we may take $\refl{\pairr{\refl{x},\refl{y}}}$ to complete the proof.
\end{proof}

\begin{exercises}
\item Suppose that $f$ has a section $g$ and a retraction $h$. Construct a homotopy $g\htpy h$. Conclude that $f$ is invertible if it is bi-invertible.
\item Show that $\mathsf{inv}:(\id{x}{y})\to(\id{y}{x})$, $\mathsf{concat}(p):(\id{y}{z})\to(\id{x}{z})$, and $\transfibf{P}(p):P(x)\to P(y)$ are equivalences. What are their inverses?
\item \label{ex:htpy_equiv} Consider two functions $f,g:A\to B$ and a homotopy $H:f\htpy g$. Then
\begin{equation*}
\isequiv(f)\leftrightarrow\isequiv(g).
\end{equation*}
\item \label{ex:3_for_2} (The 3-for-2 property) Let $f:A\to B$ and $g:B\to C$ be functions. Show that if any two of the functions
\begin{equation*}
f,\qquad g,\qquad g\circ f
\end{equation*}
are equivalences, then so is the third.
\item \label{ex:neg_equiv} Show that the negation function on the booleans is an equivalence. Also show that for any function $f:\bool\to\bool$, if $f(\bfalse)=f(\btrue)$ then $f$ is \emph{not} an equivalence.
\item \label{ex:succ_equiv} Show that the successor function on the integers is an equivalence.
\item Construct a equivalences $\eqv{A+B}{B+A}$ and $\eqv{A\times B}{B\times A}$.
\item Construct for any map $f:A\to B$ an equivalence $e:\eqv{A}{\sm{y:B}\fib{f}{y}}$ and a homotopy $H:f\htpy \proj 1\circ e$ witnessing that the triangle
\begin{equation*}
\begin{tikzcd}[column sep=small]
A \arrow[rr,"e"] \arrow[dr,swap,"f"] & & \sm{y:B}\fib{f}{y} \arrow[dl,"\proj 1"] \\
& B
\end{tikzcd}
\end{equation*}
commutes. The projection $\proj 1 : (\sm{y:B}\fib{f}{y})\to B$ is sometimes also called the \define{fibrant replacement} of $f$.
\item \label{ex:retr_id} Consider a section-retraction pair
\begin{equation*}
\begin{tikzcd}
A \arrow[r,"i"] & B \arrow[r,"r"] & A,
\end{tikzcd}
\end{equation*}
with $H:r\circ i\htpy \idfunc$. Show that $\id{x}{y}$ is a retract of $\id{i(x)}{i(y)}$.
\item Let $B:A\to \type$, and let $C:\prd{x:A}B(a)\to\type$. Construct an equivalence
\begin{equation*}
\mathsf{\Sigma.assoc}:\eqv{\Big(\sm{p:\sm{x:A}B(x)}C(\proj 1 p,\proj 2 p)\Big)}{\Big(\sm{x:A}\sm{y:B(x)}C(x,y)\Big)}
\end{equation*}
\end{exercises}
