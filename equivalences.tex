\chapter{Equivalences}

\section{Bi-invertible maps}

\begin{defn}
Let $f:A\to B$ be a function. We say that $f$ has a \define{section} if there is a term of type
\begin{equation*}
\mathsf{sec}(f) \defeq \sm{g:B\to A} f\circ g\htpy \idfunc[B].
\end{equation*}
Dually, we say that $f$ has a \define{retraction} if there is a term of type
\begin{equation*}
\mathsf{retr}(f) \defeq \sm{h:B\to A} h\circ f\htpy \idfunc[A].
\end{equation*}
If $f$ has a retraction, we also say that $A$ is a \define{retract} of $B$.
We say that a function $f:A\to B$ is an \define{equivalence} if it has both a section and a retraction, i.e.~if it comes equipped with a term of type
\begin{equation*}
\isequiv(f)\defeq\mathsf{sec}(f)\times\mathsf{retr}(f).
\end{equation*}
We will write $\eqv{A}{B}$ for the type $\sm{f:A\to B}\isequiv(f)$.
\end{defn}

\begin{rmk}
An equivalence, as we defined it here, can be thought of as a \define{bi-invertible} map, since it comes equipped with a separate left and right inverse. Explicitly, if $f$ is an equivalence, then there are
\begin{align*}
g & : B\to A & h & : B\to A \\
G & : f\circ g \htpy \idfunc[B] & H & : h\circ f \htpy \idfunc[A].
\end{align*}
Clearly, if $f$ is \define{invertible} in the sense that it comes equipped with a function $g:B\to A$ such that $f\circ g\htpy\idfunc[B]$ and $g\circ f\htpy\idfunc[A]$, then $f$ is an equivalence.
\end{rmk}

\begin{defn}\label{defn:inv_equiv}
Any equivalence can be given the structure of an invertible map.
\end{defn}

\begin{constr}
We have $\ct{(Hg)^{-1}}{hG}:g\htpy h$. This allows us to show that $g$ is a retraction of $f$: we simply take the homotopy $\ct{H}{(\ct{(Hg)^{-1}}{hG})f}:g\circ f \htpy \idfunc[A]$. 
\end{constr}

\begin{thm}\label{thm:id_equiv}
For any type $A$, the identity function $\idfunc[A]$ is an equivalence.
\end{thm}

\begin{proof}
The identity function is its own section and its own retraction.
\end{proof}

\begin{thm}\label{thm:eq_sigma}
Let $B:A\to\type$ be a type family, and let $\pairr{x,y},\pairr{x',y'}:\sm{x:A}B(x)$. Then the map
\begin{equation*}
\mathsf{eq\usc{}pair} : \Big(\sm{p:x=x'}\id{\trans{p}{y}}{y'}\Big)\to(\id{\pairr{x,y}}{\pairr{x',y'}})
\end{equation*}
defined by double path induction by sending $\pairr{\refl{x},\refl{y}}$ to $\refl{\pairr{x,y}}$ is an equivalence.
\end{thm}

\begin{proof}
The map $\mathsf{pair\usc{}eq}$ in the converse direction is constructed by path induction, taking $\refl{\pairr{x,y}}$ to $\pairr{\refl{x},\refl{y}}$.
It remains to show that $\mathsf{pair\usc{}eq}$ and $\mathsf{eq\usc{}pair}$ are mutual inverses. 

We first show that $\mathsf{eq\usc{}pair}(\mathsf{pair\usc{}eq}(p))=p$ for each $p:\pairr{x,y}=\pairr{x',y'}$. We proceed by path induction on $p$. 
Our goal is now to construct an identification $\mathsf{eq\usc{}pair}\pairr{\refl{x},\refl{y}}=\refl{\pairr{x,y}}$. 
Thus, we may take $\refl{\refl{\pairr{x,y}}}$.

Finally, we construct an identification $\mathsf{pair\usc{}eq}(\mathsf{eq\usc{}pair}(p,q))=\pairr{p,q}$ for each $\pairr{p,q}:\sm{p:x=x'}\id{\trans{p}{y}}{y'}$. We proceed by path induction on $p$, followed by path induction on $q$. Our goal is now to construct a term of type
\begin{equation*}
\mathsf{pair\usc{}eq}(\mathsf{eq\usc{}pair}\pairr{\refl{x},\refl{y}})=\pairr{\refl{x},\refl{y}}
\end{equation*}
By the definition of $\mathsf{eq\usc{}pair}$ we have $\mathsf{eq\usc{}pair}\pairr{\refl{x},\refl{y}}\jdeq \refl{\pairr{x,y}}$, and by the definition of $\mathsf{pair\usc{}eq}$ we have $\mathsf{pair\usc{}eq}(\refl{\pairr{x,y}})\jdeq\pairr{\refl{x},\refl{y}}$. Thus we may take $\refl{\pairr{\refl{x},\refl{y}}}$ to complete the proof.
\end{proof}


\section{Contractible types and contractible maps}

\begin{defn}
A type $A$ is said to be \define{contractible} if there is a term of type
\begin{equation*}
\iscontr(A) \defeq \sm{a:A}\prd{x:A}a=x.
\end{equation*}
Given a term $(c,C):\iscontr(A)$, we call $c:A$ the \define{center of contraction} of $A$, and we call $C:\prd{x:A}a=x$ the \define{contraction} of $A$.
\end{defn}

\begin{rmk}
Suppose $A$ is a contractible type with center of contraction $c$ and contraction $C$. Then the type of $C$ is (judgmentally) equal to the type
\begin{equation*}
\mathsf{const}_a\htpy\idfunc[A].
\end{equation*}
In other words, the contraction $C$ is a \emph{homotopy} from the constant function to the identity function.
\end{rmk}

\begin{thm}
The unit type $\unit$ is contractible.
\end{thm}

\begin{proof}
The unit type comes equipped with a point $\ttt:\unit$. Now we have to construct the contraction, which is of type $\prd{x:\unit}\ttt=x$. We do this by induction on $x$, so we only have to provide a term of type $\ttt=\ttt$. Here we can just take $\refl{\ttt}$. 
\end{proof}

\begin{defn}
Let $f:A\to B$ be a function, and let $b:B$. The \define{fiber} of $f$ at $b$ is defined to be the type
\begin{equation*}
\fib{f}{b}\defeq\sm{a:A}f(a)=b.
\end{equation*}
\end{defn}

\begin{defn}
We say that a function $f:A\to B$ is \define{contractible} if there is a term of type
\begin{equation*}
\iscontr(f)\defeq\prd{b:B}\iscontr(\fib{f}{b}).
\end{equation*}
\end{defn}

\begin{thm}\label{thm:equiv_contr}
Any contractible map is an equivalence.
\end{thm}

\begin{proof}
Let $f:A\to B$ be a contractible map. Using the center of contraction of each $\fib{f}{y}$, we obtain a term of type
\begin{align*}
\lam{y}\pairr{g(y),G(y)}:\prd{y:B}\fib{f}{y}.
\end{align*}
Thus, we get map $g:B\to A$, and a homotopy $G:\prd{y:B} f(g(y))=y$. In other words, we get a section of $f$.

It remains to construct a retraction of $f$. Taking $g$ as our retraction, we have to show that $\prd{x:A} g(f(x))=x$. Note that we get an identification $p:f(g(f(x)))=f(x)$ since $g$ is a section of $f$. Moreover, since $\fib{f}{f(x)}$ is contractible we get an identification $q:\pairr{g(f(x)),p}=\pairr{x,\refl{f(x)}}$. The base path of this identification is an identification of type $g(f(x))=x$, as desired.
\end{proof}

\begin{thm}\label{thm:contr_equiv}
Any equivalence is a contractible map.
\end{thm}

\begin{proof}
Since every equivalence has the structure of an invertible map by \autoref{defn:inv_equiv}, it suffices to show that any invertible map is contractible.

Let $f:A\to B$ be a map, with $g:B\to A$, $G:f\circ g\htpy\idfunc[B]$, and $H:h\circ f\htpy \idfunc[A]$.
We have for any $y:B$ the term $\pairr{g(y),G(y)}:\fib{f}{y}$. However, as our center of contraction we take
$\pairr{g(y),\epsilon(y)}$, where
\begin{equation*}
\varepsilon(y) \defeq \ct{\ap{fg}{G(y)}^{-1}}{\ap{f}{H(g(y))}}{G(y)}.
\end{equation*}
Now it remains to construct the contraction. Let $x:A$, and let $p:f(x)=y$.
Since $p:f(x)=y$ has a free endpoint, we can apply path induction on it. 
Our goal is now to construct an identification
\begin{equation*}
\pairr{g(f(x)),\varepsilon(f(x))}=\pairr{x,\refl{f(x)}}.
\end{equation*}
We will construct an identification of the form $\mathsf{eq\usc{}pair}(H(x),\nameless)$,
so it remains to construct an identification of type
\begin{equation*}
\trans{H(x)}{\varepsilon(f(x))}=\refl{f(x)}.
\end{equation*}
Using \autoref{ex:trans_ap} we see that it suffices to show that the square
\begin{equation*}
\begin{tikzcd}[column sep=8em]
fgfgf(x) \arrow[r,equals,"\ap{fg}{G(f(x))}"] \arrow[d,equals,swap,"\ap{f}{H(gf(x))}"] & fgf(x) \arrow[d,equals,"\ap{f}{H(x)}"] \\
fgf(x) \arrow[r,equals,swap,"G(f(x))"] & f(x)
\end{tikzcd}
\end{equation*}
commutes. Recall from \autoref{ex:htpy_nat} that we have $H(gf(x))=\ap{gf}{H(x)}$ and $\ap{fg}{G(y)}=G(fg(y))$. Using these two identifications and \autoref{ex:ap_ap}, we see that it suffices to show that the square
\begin{equation*}
\begin{tikzcd}[column sep=8em]
fgfgf(x) \arrow[r,equals,"G(fgf(x))"] \arrow[d,equals,swap,"\ap{fgf}{H(x)}"] & fgf(x) \arrow[d,equals,"\ap{f}{H(x)}"] \\
fgf(x) \arrow[r,equals,swap,"G(f(x))"] & f(x)
\end{tikzcd}
\end{equation*}
commutes. However, this is just a naturality square of homotopies, which commutes by \autoref{ex:htpy_nat}.
\end{proof}

\begin{cor}
Let $A$ be a type, and let $a:A$. Then the type
\begin{equation*}
\sm{x:A}x=a
\end{equation*}
is contractible.
\end{cor}

\begin{proof}
By \autoref{thm:id_equiv}, the identity function is an equivalence. Therefore, the fibers of the identity function are contractible by \autoref{thm:contr_equiv}. Note that $\sm{x:A}x=a$ is exactly the fiber of $\idfunc[A]$ at $a:A$.
\end{proof}

\begin{comment}
\begin{proof}
We have the term $(a,\refl{a}):\sm{x:A}a=x$, which we take for the center of contraction. To construct the contraction, we have to show that
\begin{equation*}
\prd{p:\sm{x:A}a=x} (a,\refl{a})=p.
\end{equation*}
By the induction principle for dependent pair types it suffices to construct a term of type
\begin{equation*}
\prd{x:A}{p:a=x} (a,\refl{a})=(x,p)
\end{equation*}
Note that we may proceed here by path induction on $p$. That is, it suffices to consider the case $p\jdeq\refl{a}$, and show that $(a,\refl{a})=(a,\refl{a})$. Here we choose $\refl{(a,\refl{a})}$.
\end{proof}
\end{comment}

\begin{exercises}
\item Suppose that $f$ has a section $g$ and a retraction $h$. Construct a homotopy $g\htpy h$. Conclude that $f$ is invertible if it is bi-invertible.
\item Show that $\mathsf{inv}:(\id{x}{y})\to(\id{y}{x})$, $\mathsf{concat}(p):(\id{y}{z})\to(\id{x}{z})$, and $\transfibf{P}(p):P(x)\to P(y)$ are equivalences. What are their inverses?
\item Construct an equivalence 
\begin{equation*}
\eqv{\big(\sm{x:A}f(x)=y\big)}{\big(\sm{x:A}y=f(x)\big)}.
\end{equation*}
Conclude that $\sm{x:A}a=x$ is contractible for any $a:A$.
\item \label{ex:htpy_equiv} Consider two functions $f,g:A\to B$ and a homotopy $H:f\htpy g$. Then
\begin{equation*}
\isequiv(f)\leftrightarrow\isequiv(g).
\end{equation*}
\item \label{ex:3_for_2} (The 3-for-2 property) Let $f:A\to B$ and $g:B\to C$ be functions. Show that if any two of the functions
\begin{equation*}
f,\qquad g,\qquad g\circ f
\end{equation*}
are equivalences, then so is the third.
\item \label{ex:contr_retr}Suppose that $A$ is a retract of $B$. Show that
\begin{equation*}
\iscontr(B)\to\iscontr(A).
\end{equation*}
\item \label{ex:contr_equiv}Show that for any type $A$, the map $\mathsf{const}_\ttt : A\to \unit$ is contractible if and only if $A$ is contractible. Conclude that for any map $f:A\to B$, if any two of the three assertions
\begin{enumerate}
\item $A$ is contractible
\item $B$ is contractible
\item $f$ is an equivalence
\end{enumerate}
hold, then so does the third.
\item \label{ex:neg_equiv} Show that the negation function on the booleans is an equivalence. Also show that for any function $f:\bool\to\bool$, if $f(\bfalse)=f(\btrue)$ then $f$ is \emph{not} an equivalence.
\item \label{ex:succ_equiv} Show that the successor function on the integers is an equivalence.
\item Construct an equivalence $\eqv{A\times B}{B\times A}$.
\item \label{ex:contr_ind} Let $C$ be a contractible type with center of contraction $c:C$. Furthermore, let $B:C\to\type$ be a type family. 
\begin{subexenum}
\item Show that the map $b\mapsto\pairr{c,b}:B(c)\to\sm{x:C}B(x)$ is an equivalence.
\item Construct a section
\begin{equation*}
\mathsf{contr\usc{}ind}:B(c)\to \prd{x:C}B(x),
\end{equation*}
of the map $\mathsf{ev}_c\defeq\lam{f}f(c)$.
\item Construct a homotopy $\mathsf{contr\usc{}ind}(\mathsf{ev}_c(f))\htpy f$ for each $f:\prd{x:C}B(x)$.
\end{subexenum}
\item Construct for any map $f:A\to B$ an equivalence $e:\eqv{A}{\sm{y:B}\fib{f}{y}}$ and a homotopy $H:f\htpy \proj 1\circ e$ witnessing that the triangle
\begin{equation*}
\begin{tikzcd}[column sep=small]
A \arrow[rr,"e"] \arrow[dr,swap,"f"] & & \sm{y:B}\fib{f}{y} \arrow[dl,"\proj 1"] \\
& B
\end{tikzcd}
\end{equation*}
commutes. The projection $\proj 1 : (\sm{y:B}\fib{f}{y})\to B$ is sometimes also called the \define{fibrant replacement} of $f$.
\item \label{ex:retr_id} Consider a section-retraction pair
\begin{equation*}
\begin{tikzcd}
A \arrow[r,"i"] & B \arrow[r,"r"] & A,
\end{tikzcd}
\end{equation*}
with $H:r\circ i\htpy \idfunc$. Show that $\id{x}{y}$ is a retract of $\id{i(x)}{i(y)}$.
\end{exercises}
