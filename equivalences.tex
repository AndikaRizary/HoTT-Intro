\chapter{Equivalences}

\section{Homotopies}
In homotopy type theory, a homotopy is just a pointwise equality between two functions $f$ and $g$.

\begin{defn}
Let $f,g:\prd{x:A}P(x)$ be two dependent functions. The type of \define{homotopies}\index{homotopy|textbf} from $f$ to $g$ is defined as
\begin{equation*}
f\htpy g \defeq \prd{x:A} f(x)=g(x).
\end{equation*}
\end{defn}

Since we formulated homotopies using dependent functions, we may also consider homotopies \emph{between}\index{homotopy!iterated} homotopies, and further homotopies between those higher homotopies. 
Explicitly, if $H,K:f\htpy g$, then the type $H\htpy K$ of homotopies is just the type
\begin{equation*}
\prd{x:A} H(x)=K(x).
\end{equation*}

In the following definition we define the groupoid-like structure of homotopies. Note that we implement the groupoid-laws as \emph{homotopies} rather than as identifications.

\begin{defn}\label{defn:htpy_groupoid}\index{groupoid laws!of homotopies|textbf}
For any dependent type $B:A\to\type$ there are operations
\begin{align*}
& \mathsf{htpy.refl} & & : \prd{f:\prd{x:A}B(x)}f\htpy f \\
& \mathsf{htpy.inv} & & : \prd*{f,g:\prd{x:A}B(x)} (f\htpy g)\to(g\htpy f) \\
& \mathsf{htpy.concat} & & : \prd*{f,g,h:\prd{x:A}B(x)} (f\htpy g)\to (g\htpy h)\to (f\htpy h).
\end{align*}
We will write $H^{-1}$ for $\mathsf{htpy.inv}(H)$, and $\ct{H}{K}$ for $\mathsf{htpy.concat}(H,K)$. 

Furthermore, we define
\begin{align*}
& \mathsf{htpy.assoc}(H,K,L) & & : \ct{(\ct{H}{K})}{L}\htpy\ct{H}{(\ct{K}{L})} \\
& \mathsf{htpy.left\usc{}unit}(H) & & : \ct{\mathsf{htpy.refl}_f}{H}\htpy H \\
& \mathsf{htpy.right\usc{}unit}(H) & & : \ct{H}{\mathsf{htpy.refl}_g}\htpy H \\
& \mathsf{htpy.left\usc{}inv}(H) & & : \ct{H^{-1}}{H} \htpy \mathsf{htpy.refl}_g \\
& \mathsf{htpy.right\usc{}inv}(H) & & : \ct{H}{H^{-1}} \htpy \mathsf{htpy.refl}_f
\end{align*}
for any $H:f\htpy g$, $K:g\htpy h$ and $L:h\htpy i$, where $f,g,h,i:\prd{x:A}B(x)$.
\end{defn}

\begin{constr}
We define
\begin{align*}
\mathsf{htpy.refl}(f) & \defeq \lam{x} \refl{f(x)} \\
\mathsf{htpy.inv}(H) & \defeq \lam{x} H(x)^{-1} \\
\mathsf{htpy.concat}(H,K) & \defeq \lam{x}\ct{H(x)}{K(x)},
\end{align*}
where $H:f\htpy g$ and $K:g\htpy h$ are homotopies. Furthermore, we define
\begin{align*}
\mathsf{htpy.assoc}(H,K,L) & \defeq \lam{x}\mathsf{assoc}(H(x),K(x),L(x)) \\
\mathsf{htpy.left\usc{}unit}(H) & \defeq \lam{x}\mathsf{left\usc{}unit}(H(x)) \\
\mathsf{htpy.right\usc{}unit}(H) & \defeq \lam{x}\mathsf{right\usc{}unit}(H(x)) \\
\mathsf{htpy.left\usc{}inv}(H) & \defeq \lam{x}\mathsf{left\usc{}inv}(H(x)) \\
\mathsf{htpy.right\usc{}inv}(H) & \defeq \lam{x}\mathsf{right\usc{}inv}(H(x)).\qedhere
\end{align*}
\end{constr}


Apart from the groupoid operations and their laws, we will occasionally need \emph{whiskering} operations.

\begin{defn}
We define the following \define{whiskering}\index{homotopy!whiskering operations|textbf}\index{whiskering operations!of homotopies|textbf} operations on homotopies:
\begin{enumerate}
\item Suppose $H:f\htpy g$ for two functions $f,g:A\to B$, and let $h:B\to C$. We define
\begin{equation*}
hH\defeq \lam{x}\ap{h}{H(x)}:h\circ f\htpy h\circ g.
\end{equation*}
\item Suppose $f:A\to B$ and $H:g\htpy h$ for two functions $g,h:B\to C$. We define
\begin{equation*}
Hf\defeq\lam{x}H(f(x)):h\circ f\htpy g\circ f.
\end{equation*}
\end{enumerate}
\end{defn}

\section{Bi-invertible maps}
\begin{defn}
Let $f:A\to B$ be a function. We say that $f$ has a \define{section}\index{section!of a map|textbf} if there is a term of type\index{sec(f)@{$\mathsf{sec}(f)$}|textbf}
\begin{equation*}
\mathsf{sec}(f) \defeq \sm{g:B\to A} f\circ g\htpy \idfunc[B].
\end{equation*}
Dually, we say that $f$ has a \define{retraction}\index{retraction} if there is a term of type\index{retr(f)@{$\mathsf{retr}(f)$}|textbf}
\begin{equation*}
\mathsf{retr}(f) \defeq \sm{h:B\to A} h\circ f\htpy \idfunc[A].
\end{equation*}
If $f$ has a retraction, we also say that $A$ is a \define{retract}\index{retract!of a type} of $B$.
We say that a function $f:A\to B$ is an \define{equivalence}\index{equivalence|textbf}\index{bi-invertible map|see {equivalence}} if it has both a section and a retraction, i.e.~if it comes equipped with a term of type\index{is_equiv@{$\isequiv$}|textbf}
\begin{equation*}
\isequiv(f)\defeq\mathsf{sec}(f)\times\mathsf{retr}(f).
\end{equation*}
We will write $\eqv{A}{B}$\index{equiv@{$\eqv{A}{B}$}|textbf} for the type $\sm{f:A\to B}\isequiv(f)$.
\end{defn}

\begin{rmk}
An equivalence, as we defined it here, can be thought of as a \emph{bi-invertible} map, since it comes equipped with a separate left and right inverse. Explicitly, if $f$ is an equivalence, then there are
\begin{align*}
g & : B\to A & h & : B\to A \\
G & : f\circ g \htpy \idfunc[B] & H & : h\circ f \htpy \idfunc[A].
\end{align*}
Clearly, if $f$ is \define{invertible}\index{invertible map} in the sense that it comes equipped with a function $g:B\to A$ such that $f\circ g\htpy\idfunc[B]$ and $g\circ f\htpy\idfunc[A]$, then $f$ is an equivalence. We write\index{is_invertible@{$\mathsf{is\usc{}invertible}$}|textbf}
\begin{equation*}
\mathsf{is\usc{}invertible}(f)\defeq\sm{g:B\to A} (f\circ g\htpy \idfunc[B])\times (g\circ f\htpy\idfunc[A]).
\end{equation*}
\end{rmk}

\begin{defn}\label{defn:inv_equiv}
Any equivalence can be given the structure of an invertible map.\index{equivalence!invertibility of}
\end{defn}

\begin{constr}
First we construct for any equivalence $f$ with right inverse $g$ and left inverse $h$ a homotopy $K:g\htpy h$. For any $y:B$, we have 
\begin{equation*}
\begin{tikzcd}[column sep=huge]
g(y) \arrow[r,equals,"H(g(y))^{-1}"] & hfg(y) \arrow[r,equals,"\ap{h}{G(y)}"] & h(y).
\end{tikzcd}
\end{equation*} 
Therefore we define $K\defeq \ct{(Hg)^{-1}}{hG}$.
from which we obtain a homotopy $K:g\htpy h$.
This allows us to show that $g$ is also a left inverse of $f$. For $x:A$ we have the identification
\begin{equation*}
\begin{tikzcd}[column sep=large]
gf(x) \arrow[r,equals,"K(f(x))"] & hf(x) \arrow[r,equals,"H(x)"] & x.
\end{tikzcd}\qedhere
\end{equation*}
\end{constr}

\begin{cor}
The inverse of an equivalence is again an equivalence.
\end{cor}

\begin{proof}
Let $f:A\to B$ be an equivalence. By \cref{defn:inv_equiv} it follows that the section of $f$ is also a retraction. Therefore it follows that the section is itself an invertible map, with inverse $f$. Hence it is an equivalence.
\end{proof}

\begin{thm}\label{thm:id_equiv}
For any type $A$, the identity function $\idfunc[A]$ is an equivalence.\index{identity function!is an equivalence|textit}
\end{thm}

\begin{proof}
The identity function is trivially its own section and its own retraction.
\end{proof}

\begin{eg}
Let $A$ and $B$ be types in context $\Gamma$. 
For any $\Gamma,x:A,y:B\vdash C(x,y)~\mathrm{type}$, the map
\begin{equation*}
\Big(\prd{x:A}{y:B}C(x,y)\Big)\to\Big(\prd{y:B}{x:A}C(x,y)\Big)
\end{equation*}
is an equivalence by \cref{ex:swap}.\index{swap function!is an equivalence|textit}
\end{eg}

\section{The identity type of a \texorpdfstring{$\Sigma$-}{dependent pair }type}

In the following theorem we characterize the identity type of a $\Sigma$-type as a $\Sigma$-type of identity types.


\begin{thm}\label{thm:eq_sigma}
Let $B$ be a type family over $A$, let $s:\sm{x:A}B(x)$, and consider the dependent function\index{pair_eq@{$\mathsf{pair\usc{}eq}$}|textbf}
\begin{equation*}
\mathsf{pair\usc{}eq}_{s}:\prd{t:\sm{x:A}B(x)} (s=t)\to \sm{\alpha:\proj 1(s)=\proj 1(t)} \mathsf{tr}_B(\alpha,\proj 2(s))=\proj 2(t)
\end{equation*}
defined as $\mathsf{pair\usc{}eq}_s \defeq \ind{s=}(\refl{\proj 1(s)},\refl{\proj 2(s)})$. Then $\mathsf{pair\usc{}eq}_{s,t}$ is an equivalence for every $t:\sm{x:A}B(x)$.\index{Sigma type@{$\Sigma$-type}!identity types of|textit}\index{identity type!of a Sigma-type@{of a $\Sigma$-type}|textit}
\end{thm}

\begin{proof}
The maps in the converse direction\index{eq_pair@{$\mathsf{eq\usc{}pair}$}}
\begin{equation*}
\mathsf{eq\usc{}pair}_{s,t} : \Big(\sm{p:\proj 1(s)=\proj 1(t)}\id{\mathsf{tr}_B(p,\proj 2(s))}{\proj 2(t)}\Big)\to(\id{s}{t})
\end{equation*}
are defined by repeated $\Sigma$-induction. By $\Sigma$-induction on $s$ and $t$  we see that it suffices to define a map
\begin{equation*}
\mathsf{eq\usc{}pair}_{(x,y),(x',y')} : \Big(\sm{p:x=x'}\id{\mathsf{tr}_B(p,y)}{y'}\Big)\to(\id{(x,y)}{(x',y')}).
\end{equation*}
A map of this type is again defined by $\Sigma$-induction. Thus it suffices to define a dependent function of type
\begin{equation*}
\prd{p:x=x'} (\id{\mathsf{tr}_B(p,y)}{y'}) \to (\id{(x,y)}{(x',y')}).
\end{equation*}
Such a dependent function is defined by double path induction by sending $\pairr{\refl{x},\refl{y}}$ to $\refl{\pairr{x,y}}$.

Next, we must show that $\mathsf{eq\usc{}pair}_{s,t}$ is a section of $\mathsf{pair\usc{}eq}_{s,t}$. In other words, we must construct an identification
\begin{equation*}
\mathsf{pair\usc{}eq}(\mathsf{eq\usc{}pair}(p,q))=\pairr{p,q}
\end{equation*}
for each $\pairr{p,q}:\sm{p:x=x'}\id{\mathsf{tr}_B(p,y)}{y'}$. We proceed by path induction on $p$, followed by path induction on $q$. Our goal is now to construct a term of type
\begin{equation*}
\mathsf{pair\usc{}eq}(\mathsf{eq\usc{}pair}\pairr{\refl{x},\refl{y}})=\pairr{\refl{x},\refl{y}}
\end{equation*}
By the definition of $\mathsf{eq\usc{}pair}$ we have $\mathsf{eq\usc{}pair}\pairr{\refl{x},\refl{y}}\jdeq \refl{\pairr{x,y}}$, and by the definition of $\mathsf{pair\usc{}eq}$ we have $\mathsf{pair\usc{}eq}(\refl{\pairr{x,y}})\jdeq\pairr{\refl{x},\refl{y}}$. Thus we may take $\refl{\pairr{\refl{x},\refl{y}}}$ to complete the construction of the homotopy $\mathsf{pair\usc{}eq}\circ\mathsf{eq\usc{}pair}\htpy\idfunc$.

To complete the proof, we must show that $\mathsf{eq\usc{}pair}_{s,t}$ is a retraction of $\mathsf{pair\usc{}eq}_{s,t}$. In other words, we must construct an identification
\begin{equation*}
\mathsf{eq\usc{}pair}(\mathsf{pair\usc{}eq}(p))=p
\end{equation*}
for each $p:s=t$. We proceed by identity elimination on $p:s=t$, so it suffices to construct an identification 
\begin{equation*}
\mathsf{eq\usc{}pair}\pairr{\refl{\proj 1(s)},\refl{\proj 2(s)}}=\refl{s}.
\end{equation*}
Now we proceed by $\Sigma$-induction on $s:\sm{x:A}B(x)$, so it suffices to construct an identification
\begin{equation*}
\mathsf{eq\usc{}pair}\pairr{\refl{x},\refl{y}}=\refl{(x,y)}.
\end{equation*}
Since $\mathsf{eq\usc{}pair}\pairr{\refl{x},\refl{y}}$ computes to $\refl{(x,y)}$, we may simply take $\refl{\refl{(x,y)}}$.
\end{proof}

\begin{cor}
Let $B$ be a type family over $A$, and let $\pairr{x,y},\pairr{x',y'}:\sm{x:A}B(x)$. Then the map
\begin{equation*}
\mathsf{pair\usc{}eq}_{(x,y),(x',y')} : (\id{\pairr{x,y}}{\pairr{x',y'}})\to \Big(\sm{p:x=x'}\id{\mathsf{tr}_B(p,y)}{y'}\Big)
\end{equation*}
is an equivalence.
\end{cor}

\begin{exercises}
\item Show that for any term $a:A$ the functions
\begin{align*}
\ind{\unit}(a) & : \unit \to A \\
\mathsf{const}_a & : \unit \to A
\end{align*}
are homotopic.
\item Let $A$ and $B$ be types, and consider the constant map $\mathsf{const}_b:A\to B$ for some $b:B$. Construct a homotopy
\begin{equation*}
\mathsf{ap}_{\mathsf{const}_b}(x,y)\htpy \mathsf{const}_{\refl{b}}
\end{equation*}
for any $x,y:A$.
\item \label{ex:equiv_grpd_ops}Show that $\mathsf{inv}:(\id{x}{y})\to(\id{y}{x})$, $\mathsf{concat}(p):(\id{y}{z})\to(\id{x}{z})$, and $\mathsf{tr}_B(p):B(x)\to B(y)$ are equivalences. What are their inverses?
\item \label{ex:htpy_equiv}\index{equivalence!homotopic maps} Consider two functions $f,g:A\to B$ and a homotopy $H:f\htpy g$. Then
\begin{equation*}
\isequiv(f)\leftrightarrow\isequiv(g).
\end{equation*}
\item \label{ex:3_for_2}\index{equivalence!three@{3-for-2 property}} 
Consider a commuting triangle
\begin{equation*}
\begin{tikzcd}[column sep=tiny]
A \arrow[rr,"h"] \arrow[dr,swap,"f"] & & B \arrow[dl,"g"] \\
& X.
\end{tikzcd}
\end{equation*}
with $H:f\htpy g\circ h$.
\begin{subexenum}
\item Suppose that the map $h$ has a section. Show that $f$ has a section if and only if $g$ has a section.
\item Suppose that the map $g$ has a retraction. Show that $f$ has a retraction if and only if $h$ has a retraction.
\item (The \define{3-for-2 property} for equivalences.) Show that if any two of the functions
\begin{equation*}
f,\qquad g,\qquad g\circ f
\end{equation*}
are equivalences, then so is the third.
\end{subexenum}
\item \label{ex:neg_equiv} Show that the negation function on the booleans is an equivalence. Also show that for any function $f:\bool\to\bool$, if $f(\bfalse)=f(\btrue)$ then $f$ is \emph{not} an equivalence.\index{negation function!is an equivalence}
\item \label{ex:succ_equiv} Show that the successor function on the integers is an equivalence.\index{successor function!on Z@{on $\Z$}!is an equivalence}
\item \label{ex:comm_prod}Construct a equivalences $\eqv{A+B}{B+A}$ and $\eqv{A\times B}{B\times A}$.\index{coproduct!is symmetric}
\item \label{ex:retr_id} Consider a section-retraction pair
\begin{equation*}
\begin{tikzcd}
A \arrow[r,"i"] & B \arrow[r,"r"] & A,
\end{tikzcd}
\end{equation*}
with $H:r\circ i\htpy \idfunc$. Show that $\id{x}{y}$ is a retract of $\id{i(x)}{i(y)}$.\index{retract!identity types of}
\item \label{ex:sigma_assoc}\index{Sigma type@{$\Sigma$-type}!associativity of}Let $B:A\to \type$, and let $C$ be a family over $x:A,y:B(x)$. Construct an equivalence
\begin{equation*}
\mathsf{\Sigma.assoc}:\eqv{\Big(\sm{p:\sm{x:A}B(x)}C(\proj 1(p),\proj 2(p))\Big)}{\Big(\sm{x:A}\sm{y:B(x)}C(x,y)\Big)}.
\end{equation*}
\item \label{ex:sigma_swap}Let $A$ and $B$ be types, and let $C$ be a family over $x:A,y:B$. Construct an equivalence
\begin{equation*}
\mathsf{\Sigma.swap}:\eqv{\Big(\sm{x:A}{y:B}C(x,y)\Big)}{\Big(\sm{y:B}{x:A}C(x,y)\Big)}.
\end{equation*}
\item \label{ex:int_group_laws}\index{Z@{$\Z$}!group laws} To define all the proof terms involved in showing that the integers form an abelian group is fairly involved. We suggest to show first that $\Z$ is a retract of $\N\times\N$\index{Z@{$\Z$}!as retract of $\N\times\N$} in a way that is compatible with addition.
\begin{subexenum}
\item Define the map $\Z\to\N\times\N$ by
\begin{align*}
\mathsf{neg}(k) & \mapsto (0,k) \\
0 & \mapsto (0,0) \\
\mathsf{pos}(k) & \mapsto (k,0)
\end{align*}
for $k:\N$. Construct a retraction $r:\N\times \N\to \Z$ of this map, in such a way that 
\begin{equation*}
r((m+m',n+n'))=r(m,n)+r(m',n')
\end{equation*}
for any $m,n:\N$.
\item Use this retraction to show that the operations $k,l\mapsto k+l$ and $k\mapsto -k$ on the integers defined in \autoref{ex:int_group_ops} satisfy the group laws:
\begin{align*}
k+(l+m) & = (k+l)+m \\
k+0 & = k \\
0+k & = k \\
k+(-k) & = 0 \\
(-k) + k & = 0,
\end{align*}
and that $k+l=l+k$, making $\pairr{\Z,0,+,-}$ into an abelian group.
\end{subexenum}
\end{exercises}
