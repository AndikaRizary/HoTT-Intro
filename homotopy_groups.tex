\chapter{Homotopy groups of types}

\section{Pointed types}

\begin{defn}
\begin{enumerate}
\item A pointed type consists of a type $X$ equipped with a base point $x:X$. We will write $\UU_\ast$ for the type $\sm{X:\UU}X$ of all pointed types.
\item Let $(X,\ast_X)$ be a pointed type. A \define{pointed family} over $(X,\ast_X)$ consists of a type family $P:X\to \UU$ equipped with a base point $\ast_P:P(\ast_X)$. 
\item Let $(P,\ast_P)$ be a pointed family over $(X,\ast_X)$. A \define{pointed section} of $(P,\ast_P)$ consists of a dependent function $f:\prd{x:X}P(x)$ and an identification $p:f(\ast_X)=\ast_P$. We define the \define{pointed $\Pi$-type} to be the type of pointed sections:
\begin{equation*}
\Pi^\ast_{(x:X)}P(x) \defeq \sm{f:\prd{x:X}P(x)}f(\ast_X)=\ast_P
\end{equation*}
In the case of two pointed types $X$ and $Y$, we may also view $Y$ as a pointed family over $X$. In this case we write $X\to_\ast Y$ for the type of pointed functions.
\item Given any two pointed sections $f$ and $g$ of a pointed family $P$ over $X$, we define the type of pointed homotopies
\begin{equation*}
f\htpy_\ast g \defeq \Pi^\ast_{(x:X)} f(x)=g(x),
\end{equation*}
where the family $x\mapsto f(x)=g(x)$ is equipped with the base point $\ct{p}{q^{-1}}$. 
\end{enumerate}
\end{defn}

\begin{eg}
The circle $\sphere{1}$ is a pointed type with base point $\base:\sphere{1}$.
\end{eg}

\begin{eg}
If $X$ is a pointed type, then in the suspension of $X$ we have the canonical identification $\merid(\ast_X):\north=\south$. Therefore we do not have to worry about whether to choose $\north$ or $\south$ as the base point of $\susp{X}$. 
\end{eg} 

\begin{rmk}
Since pointed homotopies are defined as certain pointed sections, we can use the same definition of pointed homotopies again to consider pointed homotopies between pointed homotopies, and so on.
\end{rmk}

\begin{defn}
\begin{enumerate}
\item For any pointed type $X$, we define the \define{pointed identity function} $\mathsf{id}^\ast_X\defeq (\idfunc[X],\refl{\ast})$. 
\item For any two pointed maps $f:X\to_\ast Y$ and $g:Y\to_\ast Z$, we define the \define{pointed composite}
\begin{equation*}
g\mathbin{\circ_\ast} f \defeq (g\circ f,\ct{\ap{g}{p_f}}{p_g}).
\end{equation*}
\end{enumerate}
\end{defn}

\section{Loop spaces}
\begin{defn}
Let $X$ be a pointed type with base point $x$. We define the \define{loop space} $\loopspace{X,x}$ of $X$ at $x$ to be the pointed type $x=x$ with base point $\refl{x}$. 
\end{defn}

\begin{defn}
The loop space operation $\loopspacesym$ is \emph{functorial} in the sense that
\begin{enumerate}
\item For every pointed map $f:X\to_\ast Y$ there is a pointed map
\begin{equation*}
\loopspace{f}:\loopspace{X}\to_\ast \loopspace{Y},
\end{equation*}
defined by $\loopspace{f}(\omega)\defeq \ct{p_f}{\ap{f}{\omega}}{p_f^{-1}}$, which is base point preserving by $\mathsf{right\usc{}inv}(p_f)$. 
\item For every pointed type $X$ there is a pointed homotopy
\begin{equation*}
\loopspace{\mathsf{id}_X^\ast}\htpy_\ast \mathsf{id}^\ast_{\loopspace{X}}.
\end{equation*}
\item For any two pointed maps $f:X\to_\ast Y$ and $g:Y\to_\ast X$, there is a pointed homotopy witnessing that the triangle
\begin{equation*}
\begin{tikzcd}
& \loopspace{Y} \arrow[dr,"\loopspace{g}"] \\
\loopspace{X} \arrow[rr,swap,"\loopspace{g\circ_\ast f}"] \arrow[ur,"\loopspace{f}"] & & \loopspace{Z}
\end{tikzcd}
\end{equation*}
of pointed types commutes.
\end{enumerate}
\end{defn}

\begin{thm}
Consider two pointed types $(X,x_0)$ and $(Y,y_0)$. Then there is an equivalence
\begin{equation*}
\eqv{(\susp X \to_\ast Y)}{(X \to_\ast \loopspace Y)}
\end{equation*}
\end{thm}

\begin{proof}
Computing with the universal property of the suspension
\begin{align*}
\susp X \to_\ast Y & \eqvsym \sm{y,y':Y} (X \to (y=y'))\times (y'=y_0) \\
& \eqvsym \sm{y:Y} X\to (y=y_0) \\
& \eqvsym \sm{f:X\to (y=y_0)}f(x_0)=\refl{y_0}.
\end{align*}
In the last equivalence we used \cref{ex:coh_intro}.
\end{proof}

\section{Homotopy groups}
In homotopy type theory we use $0$-types to define groups.
\begin{defn}
A \define{group} $\mathcal{G}$ consists of a set $G$ with a unit $e:G$, a multiplication $x,y\mapsto x\cdot y$, and an inverse operation $x\mapsto x^{-1}$ satisfying the \define{group laws}:
\begin{align*}
(x\cdot y)\cdot z & =x\cdot(y\cdot z) & x^{-1}\cdot x & = e \\
e\cdot x & = x & x\cdot x^{-1} & = e. \\
x\cdot e & =x
\end{align*}
\end{defn}

\begin{defn}
For $n\geq 1$, the \define{$n$-th homotopy group} of a type $X$ at a base point $x:X$ consists of the type
\begin{equation*}
|\pi_n(X,x)| \defeq \trunc{0}{\loopspace[n]{X,x}}
\end{equation*}
equipped with the group operations inherited from the path operations on $\loopspace[n]{X,x}$. 
Often we will simply write $\pi_n(X)$ when it is clear from the context what the base point of $X$ is.

For $n\jdeq 0$ we define $\pi_0(X,x)\defeq \trunc{0}{X}$. 
\end{defn}

\begin{eg}
In \cref{cor:circle_loopspace} we established that $\eqv{\loopspace{\sphere{1}}}{\Z}$. It follows that
\begin{equation*}
\pi_1(\sphere{1})=\Z \qquad\text{ and }\qquad\pi_n(\sphere{1})=0\qquad\text{for $n\geq 2$.}
\end{equation*}
Furthermore, we have seen in \autoref{circle_conn} that $\trunc{0}{\sphere{1}}$ is contractible. 
Therefore we also have $\pi_0(\sphere{1})=0$.
\end{eg}

\section{The Eckmann-Hilton argument}

Given a diagram of identifications
\begin{equation*}
\begin{tikzcd}[column sep=7em]
x \arrow[r,equals,bend left=60,"p",""{name=A,below}] \arrow[r,equals,""{name=B},""{name=E,below},"{p'}"{near end}] \arrow[r,equals,bend right=60,"{p''}"{below},""{name=F,above}] \arrow[from=A,to=B,phantom,"r\Downarrow"] \arrow[from=E,to=F,phantom,"{r'\Downarrow}"] 
& y
\end{tikzcd}
\end{equation*}
in a type $A$, where $r:p=p'$ and $r':p'=p''$,
we obtain by concatenation an identification $\ct{r}{r'}:p=p''$. This operation on identifications of identifications is sometimes called the \define{vertical concatenation}, because there is also a \emph{horizontal} concatenation operation.

\begin{defn}
Consider identifications of identifications $r:p=p'$ and $s:q=q'$, where $p,p':x=y$, and $q,q':y=z$ are identifications in a type $A$, as indicated in the diagram
\begin{equation*}
\begin{tikzcd}[column sep=huge]
x \arrow[r,equals,bend left=30,"p",""{name=A,below}] \arrow[r,equals,bend right=30,""{name=B,above},"{p'}"{below}] \arrow[from=A,to=B,phantom,"r\Downarrow"] & y \arrow[r,equals,bend left=30,"q",""{name=C,below}] \arrow[r,equals,bend right=30,""{name=D,above},"{q'}"{below}] \arrow[from=C,to=D,phantom,"s\Downarrow"] & z.
\end{tikzcd}
\end{equation*}
We define the \define{horizontal concatenation} $\ct[h]{r}{s}:\ct{p}{q}=\ct{p'}{q'}$ of $r$ and $s$.
\end{defn}

\begin{proof}
First we induct on $r$, so it suffices to define $\ct[h]{\refl{p}}{s}:\ct{p}{q}=\ct{p}{q'}$. 
Next, we induct on $p$, so it suffices to define $\ct[h]{\refl{\refl{y}}}{s}:\ct{\refl{y}}{q}=\ct{\refl{y}}{q'}$. 
Since $\ct{\refl{y}}{q}\jdeq q$ and $\ct{\refl{y}}{q'}\jdeq q'$, we take $\ct[h]{\refl{\refl{y}}}{s}\defeq s$. 
\end{proof}

\begin{lem}
Horizontal concatenation satisfies the left and right unit laws.
\end{lem}

In the following lemma we establish the \define{interchange law} for horizontal and vertical concatenation.

\begin{lem}
Consider a diagram of the form
\begin{equation*}
\begin{tikzcd}[column sep=7em]
x \arrow[r,equals,bend left=60,"p",""{name=A,below}] \arrow[r,equals,""{name=B},""{name=E,below}] \arrow[r,equals,bend right=60,"{p''}"{below},""{name=F,above}] \arrow[from=A,to=B,phantom,"r\Downarrow"] \arrow[from=E,to=F,phantom,"{r'\Downarrow}"] 
& y \arrow[r,equals,bend left=60,"q",""{name=C,below}] \arrow[r,equals,""{name=G,above},""{name=H,below}] \arrow[r,equals,bend right=60,""{name=D,above},"{q''}"{below}] \arrow[from=C,to=G,phantom,"s\Downarrow"] \arrow[from=H,to=D,phantom,"{s'\Downarrow}"] & z.
\end{tikzcd}
\end{equation*}
Then there is an identification
\begin{equation*}
\ct[h]{(\ct{r}{r'})}{(\ct{s}{s'})}=\ct{(\ct[h]{r}{s})}{(\ct[h]{r'}{s'})}.
\end{equation*}
\end{lem}

\begin{proof}
We use path induction on both $r$ and $r'$, followed by path induction on $p$. Then it suffices to show that
\begin{equation*}
\ct[h]{(\ct{\refl{\refl{y}}}{\refl{\refl{y}}})}{(\ct{s}{s'})}=\ct{(\ct[h]{\refl{\refl{y}}}{s})}{(\ct[h]{\refl{\refl{y}}}{s'})}.
\end{equation*}
Using the computation rules, we see that this reduces to
\begin{equation*}
\ct{s}{s'}=\ct{s}{s'},
\end{equation*}
which we have by reflexivity.
\end{proof}

\begin{thm}
For $n\geq 2$, the $n$-th homotopy group is abelian.
\end{thm}

\begin{proof}
Our goal is to show that 
\begin{equation*}
\prd{r,s:\pi_2(X)} r\cdot s=s\cdot r.
\end{equation*}
Since we are constructing an identification in a set, we can use the universal property of $0$-truncation on both $r$ and $s$. Therefore it suffices to show that
\begin{equation*}
\prd{r,s:\refl{x_0}=\refl{x_0}} \tproj{0}r\cdot \tproj{0}s=\tproj{0}s\cdot \tproj{0}r.
\end{equation*}
Now we use that $\tproj{0}{r}\cdot\tproj{0}{s}\jdeq \tproj{0}{\ct{r}{s}}$ and $\tproj{0}{s}\cdot\tproj{0}{r}\jdeq \tproj{0}{\ct{s}{r}}$, to see that it suffices to show that $\ct{r}{s}=\ct{s}{r}$, for every $r,s:\refl{x}=\refl{x}$. Using the unit laws and the interchange law, this is a simple computation:
\begin{align*}
\ct{r}{s} & = \ct{(\ct[h]{r}{\refl{x}})}{(\ct[h]{\refl{x}}{s})} \\
& = \ct[h]{(\ct{r}{\refl{x}})}{(\ct{\refl{x}}{s})} \\
& = \ct[h]{(\ct{\refl{x}}{r})}{(\ct{s}{\refl{x}})} \\
& = \ct{(\ct[h]{\refl{x}}{s})}{(\ct[h]{r}{\refl{x}})} \\
& = \ct{s}{r}.\qedhere
\end{align*}
\end{proof}

\section{Simply connectedness of the $2$-sphere}

\begin{defn}
A pointed type $X$ is said to be \define{$n$-connected} if its homotopy groups $\pi_i(X)$ are trivial for $i\leq n$. A $0$-connected type is also just called \define{connected}, and a $1$-connected type is also called \define{simply connected}. 
\end{defn}

We write $\ast$ for the base point of the sphere $\sphere{n}$.

\begin{thm}
For any $n:\N$ and any family $P$ of $n$-types over the $(n+2)$-sphere $\sphere{n+2}$, the function
\begin{equation*}
\Big(\prd{x:\sphere{n+2}}P(x)\Big)\to P(\ast)
\end{equation*}
given by $f\mapsto f(\ast)$, is an equivalence.
\end{thm}

\begin{cor}
The $2$-sphere is simply connected.
\end{cor}

\begin{proof}
Our goal is to show that $\pi_1(\sphere{2})$ is contractible. In other words, we have to show that $\trunc{0}{\loopspace{\sphere{2}}}$ is contractible. We do this by constructing a term of type
\begin{equation*}
\prd{t:\sphere{2}}\iscontr(\trunc{0}{\ast=t}).
\end{equation*}
First we note that
\begin{equation*}
\prd{t:\sphere{2}}\trunc{0}{\ast=t}
\end{equation*}
is equivalent to the type $\trunc{0}{\ast=\ast}$, of which we have the term $\tproj{0}{\refl{\ast}}$. Thus we obtain a dependent function $\alpha:\prd{t:\sphere{2}}\trunc{0}{\ast=t}$ equipped with $\alpha(\ast)=\tproj{0}{\refl{\ast}}$. Now we proceed to show that
\begin{equation*}
\prd{t:\sphere{2}}{p:\trunc{0}{\ast=t}} \alpha(t)=p
\end{equation*}
by the dependent universal property of $0$-truncation. Therefore it suffices to construct a term of type
\begin{equation*}
\prd{t:\sphere{2}}{p:\ast=t}\alpha(t)=\tproj{0}{p}.
\end{equation*}
This is immediate by path induction and the fact that $\alpha(\ast)=\tproj{0}{\refl{\ast}}$.
\end{proof}

\begin{exercises}
\item Show that the type of pointed families over a pointed type $(X,x)$ is equivalent to the type
\begin{equation*}
\sm{Y:\UU_\ast} Y\to_\ast X.
\end{equation*}
\item Given two pointed types $A$ and $X$, we say that $A$ is a (pointed) retract of $X$ if we have $i:A\to_\ast X$, a retraction $r:X\to_\ast A$, and a pointed homotopy $H:r\circ_\ast i\htpy_\ast \idfunc^\ast$. 
\begin{subexenum}
\item Show that if $A$ is a pointed retract of $X$, then $\loopspace{A}$ is a pointed retract of $\loopspace{X}$. 
\item Show that if $A$ is a pointed retract of $X$ and $\pi_n(X)$ is a trivial group, then $\pi_n(A)$ is a trivial group.
\end{subexenum}
\item Construct by path induction a family of maps
\begin{equation*}
\prd{A,B:\UU}{a:A}{b:B} (\id{\pairr{A,a}}{\pairr{B,b}})\to \sm{e:\eqv{A}{B}}e(a)=b,
\end{equation*}
and show that this map is an equivalence. In other words, an \emph{identification of pointed types} is a base point preserving equivalence.
\item Let $\pairr{A,a}$ and $\pairr{B,b}$ be two pointed types. Construct by path induction a family of maps
\begin{equation*}
\prd{f,g:A\to B}{p:f(a)=b}{q:g(a)=b} (\id{\pairr{f,p}}{\pairr{g,q}})\to \sm{H:f\htpy g} p = \ct{H(a)}{q},
\end{equation*}
and show that this map is an equivalence. In other words, an \emph{identification of pointed maps} is a base point preserving homotopy.
\item Show that if $A\leftarrow S\rightarrow B$ is a span of pointed types, then for any pointed type $X$ the square
\begin{equation*}
\begin{tikzcd}
(A\sqcup^S B \to_\ast X) \arrow[r] \arrow[d] & (B \to_\ast X) \arrow[d] \\
(A\to_\ast X) \arrow[r] & (S\to_\ast X)
\end{tikzcd}
\end{equation*}
is a pullback square.
\item \label{ex:yoneda_ptd_types}Let $f:A\to_\ast B$ be a pointed map. Show that the following are equivalent:
\begin{enumerate}
\item $f$ is an equivalence.
\item For any pointed type $X$, the precomposition map
\begin{equation*}
\blank\mathbin{\circ_\ast}f:(B\to_\ast X)\to_\ast (A\to_\ast X)
\end{equation*}
is an equivalence. 
\end{enumerate}
\item In this exercise we prove the suspension-loopspace adjunction.
\begin{subexenum}
\item Construct a pointed equivalence
\begin{equation*}
\tau_{X,Y}:(\susp(X)\to_\ast Y) \eqvsym_\ast (X\to \loopspace{Y})
\end{equation*}
for any two pointed spaces $X$ and $Y$.
\item Show that for any $f:X\to_\ast X'$ and $g:Y'\to_\ast Y$, there is a pointed homotopy witnessing that the square
\begin{equation*}
\begin{tikzcd}[column sep=large]
(\susp(X')\to_\ast Y') \arrow[r,"\tau_{X',Y'}"] \arrow[d,swap,"h\mapsto g\circ h\circ \susp(f)"] & (X'\to_\ast \loopspace{Y'}) \arrow[d,"h\mapsto\loopspace{g}\circ h\circ f"] \\
(\susp(X)\to_\ast Y) \arrow[r,swap,"\tau_{X,Y}"] & (X\to_\ast \loopspace{Y})
\end{tikzcd}
\end{equation*}
\end{subexenum}
\item Show that if
\begin{equation*}
\begin{tikzcd}
C \arrow[r] \arrow[d] & B \arrow[d] \\
A \arrow[r] & X
\end{tikzcd}
\end{equation*}
is a pullback square of pointed types, then so is
\begin{equation*}
\begin{tikzcd}
\loopspace{C} \arrow[r] \arrow[d] & \loopspace{B} \arrow[d] \\
\loopspace{A} \arrow[r] & \loopspace{X}.
\end{tikzcd}
\end{equation*}
\item 
\begin{subexenum}
\item Show that if $X$ is $k$-truncated, then its $n$-th homotopy group $\pi_n(X)$ is trivial for each choice of base point, and each $n> k$.
\item Show that if $X$ is $(k+l)$-truncated, and for each $0< i\leq l$ the $(k+i)$-th homotopy groups $\pi_{k+i}(X)$ are trivial for each choice of base point, then $X$ is $k$-trunctated.
\end{subexenum}
It is consistent to assume that there are types for which all homotopy groups are trivial, but which aren't contractible nontheless. Such types are called \define{$\infty$-connected}.
\end{exercises}
